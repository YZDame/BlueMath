\chapter{复数与向量}

\section{向量补充知识}
\subsection{外积的定义及基本性质}
两个向量 $\vec{a} , \vec{b}$ 的外积是一个新的向量 $\vec{a} \times \vec{b}$ , 其模长为 $|\vec{a}| \cdot|\vec{b}| \cdot$ $\sin \langle\vec{a}, \vec{b}\rangle$ , 方向垂直于 $\vec{a}$ 和 $\vec{b}$ 且 $\vec{a} , \vec{b} , \vec{a} \times \vec{b}$ 构成右手系.
从定义易知, $\vec{a} \times \vec{a}=\overrightarrow{0}$. 设 $\vec{a}=\overrightarrow{O A}, \vec{b}=\overrightarrow{O B}$, 则 $|\vec{a} \times \vec{b}|=2 S_{\triangle O A B}$.
外积满足下列运算法则:

(1) $\vec{a} \times \vec{b}=-\vec{b} \times \vec{a}$;

(2) $(\vec{a}+\vec{b}) \times \vec{c}=\vec{a} \times \vec{c}+\vec{b} \times \vec{c}$;

(3) $(\lambda \vec{a}) \times \vec{b}=\vec{a} \times(\lambda \vec{b})=\lambda(\vec{a} \times \vec{b}) \quad(\lambda \in \mathbb{R})$.

设 $\vec{a}=\left(a_1, a_2, a_3\right), \vec{b}=\left(b_1, b_2, b_3\right)$, 则
\begin{align*}
	\begin{aligned}
		\vec{a} \times \vec{b} & =\begin{vmatrix}
			                          a_2 & a_3 \\
			                          b_2 & b_3
		                          \end{vmatrix} \vec{i}-\begin{vmatrix}
			                                                a_1 & a_3 \\
			                                                b_1 & b_3
		                                                \end{vmatrix} \vec{j}+\begin{vmatrix}
			                                                                      a_1 & a_2 \\
			                                                                      b_1 & b_2
		                                                                      \end{vmatrix} \vec{k} \\
		                       & =\begin{vmatrix}
			                          \vec{i} & \vec{j} & \vec{k} \\
			                          a_1     & a_2     & a_3     \\
			                          b_1     & b_2     & b_3
		                          \end{vmatrix} .
	\end{aligned}
\end{align*}

\subsection{向量的混合积}
三个向量 $\vec{a} , \vec{b} , \vec{c}$ 作运算 $(\vec{a} \times \vec{b}) \cdot \vec{c}$ 称为 $\vec{a} , \vec{b} , \vec{c}$ 的混合积, 记为 $(\vec{a}, \vec{b}, \vec{c})$.
设 $\vec{a}=\left(a_1, a_2, a_3\right), \vec{b}=\left(b_1, b_2, b_3\right), \vec{c}=\left(c_1, c_2, c_3\right)$, 则
\begin{align*}
	(\vec{a}, \vec{b}, \vec{c})=\begin{vmatrix}
		                            a_1 & a_2 & a_3 \\
		                            b_1 & b_2 & b_3 \\
		                            c_1 & c_2 & c_3
	                            \end{vmatrix} .
\end{align*}

由行列式的运算性质可知:
\begin{align*}
	\begin{aligned}
		{(\vec{a}, \vec{b}, \vec{c}) } & =(\vec{b}, \vec{c}, \vec{a})=(\vec{c}, \vec{a}, \vec{b})                                  \\
		                               & =-(\vec{b}, \vec{a}, \vec{c})=-(\vec{a}, \vec{c}, \vec{b})=-(\vec{c}, \vec{b}, \vec{a}) .
	\end{aligned}
\end{align*}

混合积可用来判断三个向量共面: $\vec{a}, \vec{b}, \vec{c}$ 共面当且仅当 $(\vec{a}, \vec{b}, \vec{c})=0$.

\subsection{空间平面方程}
过点 $P\left(x_0, y_0, z_0\right)$ 且与非零向量 $\vec{n}=(A, B, C)$ 垂直的平面 $\Pi$ 的方程为
\begin{align*}
	A\left(x-x_0\right)+B\left(y-y_0\right)+C\left(z-z_0\right)=0,
\end{align*}

这称为 $\Pi$ 的点法式方程, $\vec{n}$ 称为 $\Pi$ 的法向量.
由点法式方程可将 $\Pi$ 的方程化为 $A x+B y+C z+D=0$, 称之为 $\Pi$ 的一般式方程. 给定 $\Pi_1: A_1 x+B_1 y+C_1 z+D_1=0, \Pi_2: A_2 x+B_2 y+C_2 z+D_2=$ 0 , 则两平面的夹角 $\theta$ 即两平面法向量 $\vec{n}_1, \vec{n}_2$ 的夹角 (取非钝角), 故
\begin{align*}
	\cos \theta=\frac{\left|\overrightarrow{n_1} \cdot \overrightarrow{n_2}\right|}{\left|\overrightarrow{n_1}\right| \cdot\left|\overrightarrow{n_2}\right|}=\frac{\left|A_1 A_2+B_1 B_2+C_1 C_2\right|}{\sqrt{A_1^2+B_1^2+C_1^2} \cdot \sqrt{A_2^2+B_2^2+C_2^2}} .
\end{align*}

\subsection{空间直线方程}
过点 $P\left(x_0, y_0, z_0\right)$ 且平行于非零向量 $\vec{s}=(m, n, p)$ 的直线方程为
\begin{align*}
	\frac{x-x_0}{m}=\frac{y-y_0}{n}=\frac{z-z_0}{p},
\end{align*}

称之为直线的点向式方程, $\vec{s}$ 称为其方向向量, 其参数式方程可表示为
\begin{align*}
	\left\{\begin{array}{l}
		       x=x_0+m t,                   \\
		       y=y_0+n t,(t \in \mathbb{R}) \\
		       z=z_0+p t .
	       \end{array}\right.
\end{align*}

需要说明的是, 点向式方程中, 允许 $m ,  n ,  p$ 中的一个或两个为 0 , 此时就意味着其所对应的分子为 0 .

直线方程的另一种表示方法为 $\left\{\begin{array}{l}A_1 x+B_1 y+C_1 z+D_1=0, \\ A_2 x+B_2 y+C_2 z+D_2=0,\end{array}\right.$ 其意义为将直线表示为两个平面的交线, 我们称之为直线的交面式方程.

两条直线的夹角 $\theta$ 即为两直线方向向量 $\vec{s}_1 ,  \vec{s}_2$ 的夹角 (取非针角), 故
\begin{align*}
	\cos \theta=\frac{\left|\overrightarrow{s_1} \cdot \overrightarrow{s_2}\right|}{\left|\vec{s}_1\right| \cdot\left|\overrightarrow{s_2}\right|} .
\end{align*}

给定平面 $\Pi: A x+B y+C z+D=0$ 和直线 $l: \frac{x-x_0}{m}=\frac{y-y_0}{n}=\frac{z-z_0}{p}$,当 $\Pi$ 与 $l$ 垂直时, 定义它们的夹角为 $\frac{\pi}{2}$, 当 $\Pi$ 与 $l$ 不垂直时, 定义它们的夹角 $\theta$为 $l$ 与其在平面 $\Pi$ 上的投影直线 $l^{\prime}$ 的夹角 (取锐角), 故
\begin{align*}
	\sin \theta=\frac{|\vec{s} \cdot \vec{n}|}{|\vec{s}| \cdot|\vec{n}|}=\frac{|A m+B n+C p|}{\sqrt{A^2+B^2+C^2} \cdot \sqrt{m^2+n^2+p^2}},
\end{align*}

其中 $\vec{s}$ 为 $l$ 的方向向量, $\vec{n}$ 为 $\Pi$ 的法向量.

过直线
\begin{align*}
	\left\{\begin{array}{l}
		       A_1 x+B_1 y+C_1 z+D_1=0, \\
		       A_2 x+B_2 y+C_2 z+D_2=0
	       \end{array}\right.
\end{align*}
的平面束方程为 $\lambda_1\left(A_1 x+B_1 y+C_1 z+D_1\right)+\lambda_2\left(A_2 x+B_2 y+C_2 z+D_2\right)=$ 0 , 其中 $\lambda_1 ,  \lambda_2$ 不全为零. 通常我们固定 $\lambda_1=1$, 记 $\lambda_2=\lambda$, 得到简化写法
\begin{align*}
	A_1 x+B_1 y+C_1 z+D_1+\lambda\left(A_2 x+B_2 y+C_2 z+D_2\right)=0,
\end{align*}

\begin{example}
	设点 $A(1,2,3), B(3,4,5), C(2,4,7)$. 求三角形 $A B C$ 的面积.
\end{example}
\begin{solution}
	\begin{align*}
		\begin{aligned}
			S_{\triangle A B C} & = \frac{1}{2}|\overrightarrow{A B}| \cdot|\overrightarrow{A C}| \cdot \sin \angle B A C \\
			                    & =\frac{1}{2}|\overrightarrow{A B} \times \overrightarrow{A C}|                          \\
			                    & =\frac{1}{2}\left\lvert\left|\begin{array}{lll}
				                                                   \vec{i} & \vec{j} & \vec{k} \\
				                                                   2       & 2       & 2       \\
				                                                   1       & 2       & 4
			                                                   \end{array}\right| \right\rvert\,                          \\
			                    & =\frac{1}{2}|(4,-6,2)|                                                                  \\
			                    & =\sqrt{14} .
		\end{aligned}
	\end{align*}
\end{solution}

\begin{example}
	求过三点 $M_1(2,-1,4), M_2(-1,3,-2), M_3(0,2,3)$ 的平面 $\Pi$ 的方程.
\end{example}
\begin{solution}
	解法 1: 设 $\Pi$ 的法向量为 $\vec{n}$. 由于 $\vec{n} \perp \overrightarrow{M_1 M_2}, \vec{n} \perp \overrightarrow{M_1 M_3}$, 故可取
	\begin{align*}
		\vec{n}=\overrightarrow{M_1 M_2} \times \overrightarrow{M_1 M_3}=\left|\begin{array}{ccc}
			                                                                       \vec{i} & \vec{j} & \vec{k} \\
			                                                                       -3      & 4       & -6      \\
			                                                                       -2      & 3       & -1
		                                                                       \end{array}\right|=(14,9,-1),
	\end{align*}

	又 $M_1 \in \Pi$, 故 $\Pi$ 的方程为 $14(x-2)+9(y+1)-(z-4)=0$, 即 $14 x+9 y$ $-z-15=0$.

	解法 2: 设 $M(x, y, z)$ 为 $\Pi$ 上任意一点, 由 $\overrightarrow{M_1 M} ,  \overrightarrow{M_1 M_2}, \overrightarrow{M_1 M_3}$ 共面知, $\left(\overrightarrow{M_1 M}, \overrightarrow{M_1 M_2}, \overrightarrow{M_1 M_3}\right)=0$, 即 $\left|\begin{array}{ccc}x-2 & y+1 & z-4 \\ -3 & 4 & -6 \\ -2 & 3 & -1\end{array}\right|=0$, 将行列式展开, 即得 $14 x+9 y-z-15=0$.
\end{solution}
\begin{note}
	一般地, 过三点 $M_k\left(x_k, y_k, z_k\right)(k=1,2,3)$ 的平面方程为
	\begin{align*}
		\left|\begin{array}{ccc}
			      x-x_1   & y-y_1   & z-z_1   \\
			      x_2-x_1 & y_2-y_1 & z_2-z_1 \\
			      x_3-x_1 & y_3-y_1 & z_3-z_1
		      \end{array}\right|=0 .
	\end{align*}
\end{note}

\begin{example}
	设 $P_0\left(x_0, y_0, z_0\right)$ 是平面 $A x+B y+C z+D=0$ 外一点, 求 $P_0$ 到平面的距离 $d$.
\end{example}
\begin{solution}
	平面的法向量为 $\vec{n}=(A, B, C)$, 在平面上取一点 $P_1\left(x_1, y_1\right.$, $z_1$ ), 则 $P_0$ 到平面的距离为
		\begin{align*}
			\begin{aligned}
				d & =\left|\overrightarrow{P_1 P_0}\right| \cdot\left|\cos \left\langle\vec{n}, \overrightarrow{P_1 P_0}\right\rangle\right|=\frac{\left|\overrightarrow{P_1 P_0} \cdot \vec{n}\right|}{|\vec{n}|} \\
				  & =\frac{\left|A\left(x_0-x_1\right)+B\left(y_0-y_1\right)+C\left(z_0-z_1\right)\right|}{\sqrt{A^2+B^2+C^2}},
			\end{aligned}
		\end{align*}
		由 $A x_1+B y_1+C z_1+D=0$ 可知,
	$$
		d=\frac{\left|A x_0+B y_0+C z_0+D\right|}{\sqrt{A^2+B^2+C^2}}.
	$$
\end{solution}

\begin{example}
	用点向式表示直线 $\left\{\begin{array}{l}x+y+z+1=0, \\ 2 x-y+3 z+4=0 .\end{array}\right.$
\end{example}
\begin{solution}
	先在直线上找一点.
	令 $x=1$, 解方程组 $\left\{\begin{array}{l}y+z=-2, \\ y-3 z=6\end{array}\right.$, 得 $y=0, z=-2$, 故 $(1,0,-2)$ 为直线上一点.
	再求直线的方向向量 $\vec{s}$.
	由 $\vec{s}$ 与平面 $x+y+z+1=0$ 的法向量 $\vec{n}_1=(1,1,1)$ 垂直, 且与平面 $2 x-y+3 z+4=0$ 的法向量 $\vec{n}_2=(2,-1,3)$ 垂直,故可取
	\begin{align*}
		\vec{s}=\overrightarrow{n_1} \times \overrightarrow{n_2}=\left|\begin{array}{ccc}
			                                                               \vec{i} & \vec{j} & \vec{k} \\
			                                                               1       & 1       & 1       \\
			                                                               2       & -1      & 3
		                                                               \end{array}\right|=(4,-1,-3) .
	\end{align*}
	因此所给直线的点向式方程为 $\frac{x-1}{4}=\frac{y}{-1}=\frac{z+2}{-3}$.
\end{solution}

\begin{example}
	求过点 $P(2,1,3)$ 且与直线 $\frac{x+1}{3}=\frac{y-1}{2}=\frac{z}{-1}$ 垂直相交的直线方程.
\end{example}
\begin{solution}
	设已知直线为 $l$, 则 $l$ 的方向向量为 $\vec{s}=(3,2,-1)$.
	设过点 $P$ 且垂直于 $l$ 的平面为 $\Pi$, 则 $\Pi$ 的法向量 $\vec{n}$ 即为 $\vec{s}$, 故 $\Pi$ 的方程为 $3(x-2)+2(y-1)-(z-3)=0$.
	易知 $l$ 与 $\Pi$ 的交点在所求直线上, 我们先求 $l$ 与 $\Pi$ 的交点 $Q$.

	将 $l$ 化为参数方程 $\left\{\begin{array}{l}x=3 t-1, \\ y=2 t+1, \\ z=-t,\end{array}\right.$ 代入 $\Pi$ 的方程, 即
	\begin{align*}
		3(3 t-1-2)+2(2 t+1-1)-(-t-3)=0,
	\end{align*}

	解得 $t=\frac{3}{7}$, 故 $Q\left(\frac{2}{7}, \frac{13}{7},-\frac{3}{7}\right)$.
	取 $\overrightarrow{Q P}=\left(\frac{12}{7},-\frac{6}{7}, \frac{24}{7}\right)$ 为所求直线的方向向量, 即得所求直线方程为
	\begin{align*}
		\frac{x-2}{2}=\frac{y-1}{-1}=\frac{z-3}{4} .
	\end{align*}
\end{solution}

\begin{example}
	求直线 $\left\{\begin{array}{l}x+y-z-1=0, \\ x-y+z+1=0\end{array}\right.$ 在平面 $x+y+z=0$ 上的投影直线方程.
\end{example}
\begin{solution}
	过已知直线的平面束方程为 $x+y-z-1+\lambda(x-y+z+1)=0$, 即
	\begin{align*}
		(1+\lambda) x+(1-\lambda) y+(-1+\lambda) z+(-1+\lambda)=0,
	\end{align*}

	其法向量为 $\vec{n}_1=(1+\lambda, 1-\lambda,-1+\lambda)$.

	已知平面的法向量为 $\overrightarrow{n_2}=(1,1,1)$, 令 $\vec{n}_1 \cdot \overrightarrow{n_2}=0$, 解得 $\lambda=-1$, 即得与已知平面垂直且过已知直线的平面方程: $y-z-1=0$, 由此得到所求投影直线的方程为 $\left\{\begin{array}{l}y-z-1=0, \\ x+y+z=0 .\end{array}\right.$
\end{solution}
\begin{note}
	作为一个练习, 可尝试再将所求直线方程化为点向式.
\end{note}

\begin{example}
	求过点 $M_0(1,1,1)$ 且与两直线 $L_1:\left\{\begin{array}{l}y=2 x, \\ z=x-1,\end{array} L_2:\left\{\begin{array}{l}y=3 x-4, \\ z=2 x-1\end{array}\right.\right.$ 均相交的直线 $L$ 的方程.
\end{example}
\begin{solution}
	$L_1 ,  L_2$ 的参数方程分别为 $L_1:\left\{\begin{array}{l}x=t, \\ y=2 t, \\ z=t-1,\end{array} L_2:\left\{\begin{array}{l}x=t, \\ y=3 t-4, \\ z=2 t-1,\end{array}\right.\right.$

	设 $L$ 与它们的交点分别为 $M_1\left(t_1, 2 t_1, t_1-1\right), M_2\left(t_2, 3 t_2-4,2 t_2-1\right)$.由 $M_0 ,  M_1 ,  M_2$ 三点共线知: $\overrightarrow{M_0 M_1} / / \overrightarrow{M_0 M_2}$, 故
	\begin{align*}
		\overrightarrow{M_0 M_1} \times \overrightarrow{M_0 M_2} = \vec{0} .
	\end{align*}
	即
	\begin{align*}
		\left|\begin{array}{ccc}
			      \vec{i} & \vec{j} & \vec{k} \\
			      t_1-1   & 2t_1-1  & t_1-2   \\
			      t_2-1   & 3t_2-5  & 2t_2-2
		      \end{array}\right|=\vec{0},
	\end{align*}
	解得$t_1=0, t_2=2$.

	故 $M_1(0,0,-1), M_2(2,2,3)$, 由此得 $L$ 的方向向量 $\vec{s}=(2,2,4)$,故 $L$ 的方程为 $\frac{x-1}{1}=\frac{y-1}{1}=\frac{z-1}{2}$.
\end{solution}

最后是习题中可能会用到的几个简单的结论, 请自行证明.
\begin{enumerate}
	\item 过点 $P\left(x_0, y_0, z_0\right)$ 且与非零向量 $\vec{n}=(A, B, C)$ 垂直的平面方程为
	      \begin{align*}
		      A\left(x-x_0\right)+B\left(y-y_0\right)+C\left(z-z_0\right)=0 .
	      \end{align*}
	\item 过点 $P\left(x_0, y_0, z_0\right)$ 且平行于非零向量 $\vec{s}=(m, n, p)$ 的直线方程为
	      \begin{align*}
		      \frac{x-x_0}{m}=\frac{y-y_0}{n}=\frac{z-z_0}{p} .
	      \end{align*}
	\item  点 $P$ 到过 $A ,  B$ 的直线的距离
	      \begin{align*}
		      d=\frac{|\overrightarrow{P A} \times \overrightarrow{P B}|}{|\overrightarrow{A B}|} .
	      \end{align*}
	\item 过点 $A ,  B$ 与过点 $C ,  D$ 的两异面直线之间的距离
	      \begin{align*}
		      d=\frac{|(\overrightarrow{A C}, \overrightarrow{A B}, \overrightarrow{C D})|}{|\overrightarrow{A B} \times \overrightarrow{C D}|}.
	      \end{align*}
\end{enumerate}

\subsection{习题}
\begin{exercise}
	求垂直于平面 $z=0$ 且通过点 $M_0(1,-1,1)$ 到直线 $L:\left\{\begin{array}{l}y-z+1=0, \\ x=0\end{array}\right.$ 垂线的平面方程.
\end{exercise}
\begin{solution}
	直线 $L$ 的方向向量 $\vec{l}=\{0,-1,-1\}$, 故过点 $M_0$ 且与直线 $L$ 垂直的平面 $N$ 的方程为
	\begin{align*}
		-(y+1)-(z-1)=0,
	\end{align*}

	即 $y+z=0$.
	解方程组 $\left\{\begin{array}{l}y-z+1=0, \\ x=0, \\ y+z=0,\end{array}\right.$ 得直线 $L$ 与平面 $N$ 的交点 $M_1\left(0,-\frac{1}{2}, \frac{1}{2}\right)$.
	由题意, 设所求平面方程为 $A x+B y+D=0$, 将 $M_0 ,  M_1$ 坐标代人, 得 $\left\{\begin{array}{l}A-B+D=0, \\ -\frac{B}{2}+D=0,\end{array}\right.$ 解得 $A=D, B=2 D$.
	故所求的平面方程为: $x+2 y+1=0$.
\end{solution}

\begin{exercise}
	证明两直线
	\begin{align*}
		\frac{x-2}{1}=\frac{y+2}{-1}=\frac{z-3}{2} \text { 与 } \frac{x-1}{-1}=\frac{y+1}{2}=\frac{z-1}{1}
	\end{align*}
	共面,并求该平面方程.
\end{exercise}
\begin{solution}
	记 $M_1(2,-2,3), \vec{l}_1=(1,-1,2), M_2(1,-1,1), \vec{l}_2=(-1,2,1)$.则 $\overrightarrow{M_1 M_2}=(-1,1,-2)$ .

	因为
	\begin{align*}
		\left(\vec{l}_1 \times \vec{l}_2\right) \cdot \overrightarrow{M_1 M_2}=\left|\begin{array}{ccc}
			                                                                             1  & -1 & 2  \\
			                                                                             -1 & 2  & 1  \\
			                                                                             -1 & 1  & -2
		                                                                             \end{array}\right|=0,
	\end{align*}

	所以两直线共面.
	取 $\vec{n}=\vec{l}_1 \times \vec{l}_2=(-5,-3,1)$, 则所求平面方程为
	\begin{align*}
		-5(x-2)-3(y+2)+(z-3)=0,
	\end{align*}

	即 $5 x+3 y-z-1=0$ .
\end{solution}

\begin{exercise}
	在平面 $N: x+y+z=1$ 上求一直线 $L$, 使其与直线 $L_1:\left\{\begin{array}{l}y=1, \\ z=-1\end{array}\right.$ 垂直且相交.
\end{exercise}
\begin{solution}
	$\left\{\begin{array}{l}x+y+z=1, \\ y=1, \\ z=-1,\end{array}\right.$ 解得交点 $D(1,1,-1)$.
	过点 $D$ 与直线 $L_1$ 垂直的平面方程为 $x-1=0$, 所求的直线 $L$ 的方程为 $\left\{\begin{array}{l}x+y+z=1, \\ x-1=0 .\end{array}\right.$
\end{solution}

\begin{exercise}
	求两异面直线
	\begin{align*}
		L_1: x+1=y=\frac{z-1}{2} \text { 与 } L_2: x=\frac{y+1}{3}=\frac{z-2}{4}
	\end{align*}
	之间的最短距离.
\end{exercise}
\begin{solution}
	在 $L_1$ 上找点 $A(-1,0,1), B(0,1,3)$, 在 $L_2$ 上找点 $C(0,-1,2)$, $D(1,2,6)$.

	则 $\overrightarrow{A C}=(1,-1,1), \overrightarrow{A B}=(1,1,2), \overrightarrow{C D}=(1,3,4)$. 由混合积的几何意义可知:
	\begin{align*}
		d=\frac{\left| \begin{vmatrix}
				               1 & -1 & 1 \\
				               1 & 1  & 2 \\
				               1 & 3  & 4
			               \end{vmatrix} \right|}{\left| \begin{vmatrix}
				                                             \vec{i} & \vec{j} & \vec{k} \\
				                                             1       & 1       & 2       \\
				                                             1       & 3       & 4
			                                             \end{vmatrix} \right|}=\frac{\sqrt{3}}{3} .
	\end{align*}
\end{solution}


\section{单位根}
对于方程
\begin{align*}
	x^n-1=0,\left(n \in \mathbb{N}^*, n \geqslant 2\right)
\end{align*}

由复数开方法则得到它的 $n$ 个根
\begin{align*}
	\varepsilon_k=\cos \frac{2 k \pi}{n}+\operatorname{isin} \frac{2 k \pi}{n} .(k=0,1,2, \cdots, n-1)
\end{align*}

它们显然是 1 的 $n$ 次方根, 称为 $n$ 次单位根.

利用复数乘方公式, 有
\begin{align*}
	\varepsilon_k=\left(\cos \frac{2 \pi}{n}+\mathrm{i} \sin \frac{2 \pi}{n}\right)^k=\varepsilon_1^k .
\end{align*}

这说明, $n$ 个 $n$ 次单位根可以表示为
\begin{align*}
	1, \varepsilon_1, \varepsilon_1^2, \cdots, \varepsilon_1^{n-1} .
\end{align*}

关于 $n$ 次单位根, 有如下一些性质:

(1) $\left|\varepsilon_k\right|=1 .(k \in \mathbb{N})$

(2) $\varepsilon_j \varepsilon_k=\varepsilon_{j+k} . \quad(j, k \in \mathbb{N})$

(3) $1+\varepsilon_1+\varepsilon_1^2+\cdots+\varepsilon_1^{n-1}=0 . \quad(n \geqslant 2)$

(4) 设 $m$ 是整数,则
\begin{align*}
	1+\varepsilon_1^m+\varepsilon_2^m+\cdots+\varepsilon_{n-1}^m=\left\{\begin{array}{l}
		                                                                    n, \text { 当 } m \text { 是 } n \text { 的倍数时; } \\
		                                                                    0, \text { 当 } m \text { 不是 } n \text { 的倍数时. }
	                                                                    \end{array}\right.
\end{align*}

\begin{example}\label{ex:已知单位圆的内接正 $n$ 边形}
	已知单位圆的内接正 $n$ 边形 $A_1 A_2 \cdots A_n$ 及圆周上一点 $P$, 求证:
	\begin{align*}
		\sum_{k=1}^n\left|P A_k\right|^2=2 n .
	\end{align*}
\end{example}

\begin{proof}
	设 $\zeta=\mathrm{e}^{\frac{2 \pi \mathrm{in}}{n}}, A_1, \cdots, A_n$ 对应的复数是 $1, \zeta, \zeta^2, \cdots, \zeta^{-1}$. 又设 $P$ 点 (对应的复数) 为 $z=\mathrm{e}^{i \theta}$. 则我们有
	\begin{align*}
		\begin{aligned}
			\sum_{k=1}^n\left|P A_k\right|^2 & =\sum_{k=0}^{n-1}\left|z-\zeta^k\right|^2=\sum_{k=0}^{n-1}\left(z-\zeta^k\right)\left(\bar{z}-\zeta^{-k}\right) \\
			                                 & =\sum_{k=0}^{n-1}\left(|z|^2-\zeta^k \bar{z}-\zeta^{-k} z+1\right)                                              \\
			                                 & =2 n-\bar{z} \sum_{k=0}^{n-1} \zeta^k-z \sum_{k=0}^{n-1} \zeta^{-k}=2 n,
		\end{aligned}
	\end{align*}
	(最后一步应用了 $1+\zeta+\zeta^2+\cdots+\zeta^{-1}=0$ ).
\end{proof}
\begin{note}
	更多有关单位圆内接正 $n$ 边形的类似问题请参见练习.
\end{note}

\begin{example}
	设 $P(x), Q(x), R(x)$ 及 $S(x)$ 都是多项式, 且
	\begin{align*}
		P\left(x^5\right)+x Q\left(x^5\right)+x^2 R\left(x^5\right)=\left(x^4+x^3+x^2+x+1\right) S(x),
	\end{align*}
	求证: $x-1$ 是 $P(x), Q(x), R(x)$ 及 $S(x)$ 的公因式.
\end{example}
\begin{proof}
	设 $\zeta$ 是一个 5 次单位根 $(\zeta \neq 1)$, 在 (1) 中取 $x=\zeta, \zeta^2, \zeta^3, \zeta^4$ , 得出
	\begin{align*}
		\left(\zeta^k\right)^2 R(1)+\zeta^k Q(1)+P(1)=0(k=1,2,3,4),
	\end{align*}

	这意味着多项式 $x^2 R(1)+x Q(1)+P(1)$ 有四个不同的零点.

	从而必须 $R(1)=Q(1)=P(1)=0$ .

	再将 $x=1$ 代入 (1), 得 $S(1)=0$ .

	于是 $P(x), Q(x), R(x)$ 及 $S(x)$ 都有因式 $x-1$.
\end{proof}

\begin{example}
	求最小的正整数 $n$,使 $n \times n$ 格纸可以划分为若干 $40 \times 40$ 和 $49 \times 49$格纸(这两种格纸都要有).
\end{example}
\begin{solution}
	$n=2000$ 时, 将 $2000 \times 2000$ 分出一个 $1960 \times 1960$ (用 $49 \times 49$ 铺满), 别的部分用 $40 \times 40$ 铺满, 故 $n=2000$ 满足.

	设用 $a$ 块 $40 \times 40, b$ 块 $49 \times 49$ 铺满 $n \times n$ 格纸, 则 $40^2 a+49^2 b=n^2$.

	将从上往下第 $k$ 行, 从左往右第 $j$ 列的交叉格内填上 $z^k w^j$ , 其中 $z=$ $\cos \frac{2 \pi}{40}+i \sin \frac{2 \pi}{40}, w=\cos \frac{2 \pi}{49}+i \sin \frac{2 \pi}{49}$, 则每个 $40 \times 40 ,  49 \times 49$ 盖住的格子内数之和均为 0 , 故方格表内所有数之和均为 0 , 即
	\begin{align*}
		0=\left(z+z^2+\cdots+z^n\right)\left(w+w^2+\cdots+w^n\right)=z w \frac{z^n-1}{z-1} \cdot \frac{w^n-1}{w-1},
	\end{align*}
	故 $z^n=1$ 或 $w^n=1$, 因此 $40 \mid n$ 或 $49 \mid n$.

	若 $40 \mid n$, 则 $40^2 \mid b$, 即 $b \geqslant 40^2$, 故 $n^2>49^2 b \geqslant(40 \times 49)^2$, 因此 $n>1960$,故 $n \geqslant 2000$.

	若 $49 \mid n$, 与上述方法类似可得 $n \geqslant 2009$.

	因此 $n$ 最小为 2000 .
\end{solution}

\begin{example}\label{ex:设varepsilon}
	设 $\varepsilon=\cos \frac{2 \pi}{n}+\mathrm{i} \sin \frac{2 \pi}{n}$, 求证:

	(1) $(1-\varepsilon)\left(1-\varepsilon^2\right) \cdots\left(1-\varepsilon^{n-1}\right)=n$;

	(2) $\sin \frac{\pi}{n} \sin \frac{2 \pi}{n} \cdots \sin \frac{(n-1) \pi}{n}=\frac{n}{2^{n-1}}$.
\end{example}
\begin{proof}
	方程 $x^n-1=0$ 的 $n$ 个单位根是
	\begin{align*}
		\varepsilon_k=\cos \frac{2 k \pi}{n}+i \sin \frac{2 k \pi}{n},(k=0,1, \cdots, n-1)
	\end{align*}

	注意到 $\varepsilon=\cos \frac{2 \pi}{n}+\mathrm{i} \sin \frac{2 \pi}{n}$, 从而有 $\varepsilon_k=\varepsilon^k$.

	于是, 由
	\begin{align*}
		x^n-1=(x-1)(x-\varepsilon)\left(x-\varepsilon^2\right) \cdots\left(x-\varepsilon^{n-1}\right)
	\end{align*}

	得
	\begin{align*}
		(x-\varepsilon)\left(x-\varepsilon^2\right) \cdots\left(x-\varepsilon^{n-1}\right)=\frac{x^n-1}{x-1}=x^{n-1}+x^{n-2}+\cdots+x+1 .
	\end{align*}

	即有
	\begin{equation}\label{eq:11-1}
		(x-\varepsilon)\left(x-\varepsilon^2\right) \cdots\left(x-\varepsilon^{n-1}\right)=x^{n-1}+x^{n-2}+\cdots+x+1 .
	\end{equation}
	(1) 在\autoref{eq:11-1}中, 令 $x=1$, 立得
	\begin{equation}\label{eq:11-2}
		(1-\varepsilon)\left(1-\varepsilon^2\right) \cdots\left(1-\varepsilon^{n-1}\right)=n .
	\end{equation}
	(2) 对\autoref{eq:11-2}的两边取模, 并注意到
	\begin{align*}
		\left|1-\varepsilon^k\right|=2 \sin \frac{k \pi}{n},
	\end{align*}

	立得
	\begin{align*}
		2^{n-1} \sin \frac{\pi}{n} \sin \frac{2 \pi}{n} \cdots \sin \frac{(n-1) \pi}{n}=n,
	\end{align*}

	即有
	\begin{align*}
		\sin \frac{\pi}{n} \sin \frac{2 \pi}{n} \cdots \sin \frac{(n-1) \pi}{n}=\frac{n}{2^{n-1}}.
	\end{align*}
\end{proof}

% 这题有一定难度,暂时删去
\begin{comment}
\begin{example}
	试求一切有序正整数对 $(n, k)$, 使得 $x^n+x+1$ 被 $x^k+x+1$ 整除.
\end{example}
\begin{solution}
	显然,  $n \geqslant k$ .
	当 $n>k$ 时, 设 $\omega$ 是 $x^k+x+1=0$ 的一个根, 则 $\omega \neq 0, \omega^n+\omega+1=0$,于是
	\begin{align*}
		\omega^n-\omega^k=\omega^k\left(\omega^{n-k}-1\right)=0 .
	\end{align*}

	从而有
	\begin{align*}
		\omega^{n-k}=1 .
	\end{align*}

	由 $|\omega|^{n-k}=\left|\omega^{n-k}\right|=1$, 知 $|\omega|=1$ .
	由 $1=|\omega|^k=\left|\omega^k\right|=|\omega+1|$, 可知 $\omega$ 的实部为 $-\frac{1}{2}$, 则 $k \geqslant 2$.
	$\omega_1=\frac{-1+\sqrt{3} \mathrm{i}}{2}$ 或 $\omega_2=\frac{-1-\sqrt{3} \mathrm{i}}{2}$ 是 $x^k+x+1=0$ 的所有根, 从而有
	\begin{align*}
		x^k+x+1=\left(x-\omega_1\right)^{k_1}\left(x-\omega_2\right)^{k-k_1}, k_1 \in \mathbb{Z}, 0 \leqslant k_1 \leqslant k .
	\end{align*}

	若 $k>2$, 考虑上面等式两边含 $x^{k-1}$ 的项的系数, 便有 $k_1 \omega_1+\left(k-k_1\right) \omega_2=$ 0 , 考虑实部即有 $k=0$, 产生矛盾.

	若 $k=2$, 令 $n \equiv l(\bmod 3), 0 \leqslant l<3$. 由 $\omega^n+\omega+1=\omega^l+\omega+1=0$,得 $l=2, n \equiv 2(\bmod 3)$ .
	故知 $(n, k)=(k, k)$ 或 $(3 m+2,2), m$ 是正整数.
\end{solution}
\end{comment}

\begin{example}
	求证:不存在四个整系数多项式 $f_k(x)(k=1,2,3,4)$ , 使得恒等式
	\begin{align}\label{eq:不存在四个整系数多项式-1}
		9 x+4=f_1^3(x)+f_2^3(x)+f_3^3(x)+f_4^3(x)
	\end{align}
	成立.
\end{example}
\begin{proof}
	记 $\omega$ 是三次单位根 $(\omega \neq 1)$ , 则对任意整系数多项式 $f(x)$ , 利用 $\omega^3=1$ 及 $\omega^2=-1-\omega$ 可将 $f(\omega)$ 化为 $a+b \omega$ ( $a ,  b$ 是整数), 于是 (注意 $1+$ $\left.\omega+\omega^2=0\right)$
	\begin{align*}
		f^3(\omega)=(a+b \omega)^3=a^3+b^3-3 a b^2+3 a b(a-b) \omega .
	\end{align*}

	由于 $a b(a-b)$ 总是偶数, 故若存在形如\autoref{eq:不存在四个整系数多项式-1}的恒等式, 以 $x=\omega$ 代入, 即得
	\begin{align}\label{eq:不存在四个整系数多项式-2}
		9 \omega+4=A+B \omega .
	\end{align}

	这里 $A ,  B$ 都是整数, 且 $B$ 是偶数. 但由\autoref{eq:不存在四个整系数多项式-2}易知 $B=9$, 这显然不可能.
\end{proof}

\begin{example}
	设 $z_k(k=0,1, \cdots, n-1)$ 是 $z^n-1=0$ 的 $n$ 个根, 定义
	\begin{align*}
		f(x)=a_m x^m+a_{m-1} x^{m-1}+\cdots+a_1 x+a_0,
	\end{align*}
	其中 $m$ 为小于 $n$ 的正整数, 求证: $\frac{1}{n} \sum_{k=0}^{n-1} f\left(z_k\right)=a_0$.
\end{example}
\begin{proof}
	令 $z_k=\cos \frac{2 k \pi}{n}+\operatorname{isin} \frac{2 k \pi}{n}=z_1^k(k=0,1, \cdots, n-1)$, 则由 $m<n$时, $z_1^m \neq 1, z_1^n=1$, 知
	\begin{align}\label{eq:单位根m次幂求和}
		\sum_{k=0}^{n-1} z_1^{km}=\frac{1-\left(z_1^m\right)^n}{1-z_1^m}=0
	\end{align}
	所以 $\frac{1}{n} \sum_{k=0}^{n-1} f\left(z_k\right)=a_0$(\autoref{eq:单位根m次幂求和}用到了性质4).
\end{proof}

\begin{comment}
\begin{note}
	本题可以看作一个很重要的引理, 使用很方便, 可参考下例.
\end{note}

% 这题有一定难度,暂时删去

\begin{example}
	单位圆周上任意 $n$ 个点 $z_1, \cdots, z_n$. 求证:
	\begin{align}\label{eq:单位圆周上任意n个点}
		\max _{|z|=1}\left|z-z_1\right| \cdots\left|z-z_n\right| \geqslant 2,
	\end{align}
	并证明等号成立的充要条件是 $z_1, \cdots, z_n$ 构成正 $n$ 边形.
\end{example}
\begin{solution}
	因为通过适当的旋转, 可设 $z_1 z_2 \cdots z_n=(-1)^n$. 记
	\begin{align*}
		P(z)=\left(z-z_1\right) \cdots\left(z-z_n\right)=z^n+a_1 z^{n-1}+\cdots+a_{n-1} z+1=z^n+f(z)+1,
	\end{align*}

	其中 $f(z)$ 或为零, 或次数不超过 $n-1$. 设 $\zeta_1, \zeta_2, \cdots, \zeta_n$ 是全部 $n$ 次单位根,则由上例知
	\begin{align*}
		f\left(\zeta_1\right)+\cdots+f\left(\zeta_n\right)=0 .
	\end{align*}

	如果 $f(z)$ 不恒为 0 , 则存在 $j$ 使 $f\left(\zeta_j\right) \neq 0$, 且 $\operatorname{Re} f\left(\zeta_j\right) \geqslant 0$, 故 $\left|P\left(\zeta_j\right)\right|=$ $\left|2+f\left(\zeta_j\right)\right|>2$; 如 $f(z)$ 恒为 0 , 则当然有 $\left|P\left(\zeta_j\right)\right|=2$ .这就证明了\autoref{eq:单位圆周上任意n个点}.

	上面的论证还表明, 如果\autoref{eq:单位圆周上任意n个点}等号成立, 必须 $f\left(\zeta_j\right)=0(j=1,2, \cdots, n)$,这意味着 $f(z)=0$, 即 $P(z)=z^n+1$, 所以 $z_1, \cdots, z_n$ 构成正 $n$ 边形, 证毕.
\end{solution}
\end{comment}

\subsection{习题}
\begin{exercise}
	求证: $\sin 1+\sin 2+\cdots+\sin n \leqslant \frac{1}{\sin \frac{1}{2}}$.
\end{exercise}
\begin{proof}
	\begin{align*}
		\begin{aligned}
			\left|\sum_{k=1}^n \sin k\right| & \leqslant\left|\sum_{k=1}^n \mathrm{e}^{\mathrm{i} k}\right|=\left|\frac{\mathrm{e}^{\mathrm{i}}\left(1-\mathrm{e}^{\mathrm{i} n}\right)}{1-\mathrm{e}^{\mathrm{i}}}\right|=\left|\frac{1-\mathrm{e}^{\mathrm{i} n}}{1-\mathrm{e}^{\mathrm{i}}}\right| \\
			                                 & =\frac{\left|1-\mathrm{e}^{\mathrm{i} n}\right|}{2 \sin \frac{1}{2}} \leqslant \frac{1+\left|\mathrm{e}^{\mathrm{i} n}\right|}{2 \sin \frac{1}{2}}=\frac{1}{\sin \frac{1}{2}} .
		\end{aligned}
	\end{align*}
\end{proof}

% 这题有一定难度,暂时删去
\begin{comment}
\begin{exercise}
	方程 $x^{10}+(13 x-1)^{10}=0$ 的 10 个复数根分别为 $r_1, \overline{r_1}, r_2, \overline{r_2}, r_3, \overline{r_3}$, $r_4, \overline{r_4}, r_5, \overline{r_5}$. 求代数式 $\frac{1}{r_1 \overline{r_1}}+\frac{1}{r_2 \overline{r_2}}+\cdots+\frac{1}{r_5 \overline{r_5}}$ 的值.
\end{exercise}
\begin{solution}
	设 $\varepsilon=\cos \frac{\pi}{10}+i \sin \frac{\pi}{10}$, 则 $\varepsilon^{10}=-1$.

	由方程 $(13 x-1)^{10}=-x^{10}$ , 我们可设 $13 r_k-1=r_k \cdot \varepsilon^{2 k-1}, k=1,2, \cdots, 5$.于是 $\frac{1}{r_k}=13-\varepsilon^{2 k-1}$. 所以, 有
	\begin{align*}
		\begin{aligned}
			\sum_{k=1}^5 \frac{1}{r_k \overline{r_k}} & =\sum_{k=1}^5\left(13-\varepsilon^{2 k-1}\right)\left(13-(\bar{\varepsilon})^{2 k-1}\right)                                     \\
			                                          & =\sum_{k=1}^5\left[170-13\left(\varepsilon^{2 k-1}+(\bar{\varepsilon})^{2 k-1}\right)\right]                                    \\
			                                          & =850-13 \sum_{k=1}^5\left(\varepsilon^{2 k-1}+(\bar{\varepsilon})^{2 k-1}\right)                                                \\
			                                          & =850-26\left(\cos \frac{\pi}{10}+\cos \frac{3 \pi}{10}+\cos \frac{5 \pi}{10}+\cos \frac{7 \pi}{10}+\cos \frac{9 \pi}{10}\right) \\
			                                          & =850 .
		\end{aligned}
	\end{align*}
	于是, 所求代数式的值为850.
\end{solution}

\begin{exercise}
	设 $f(x)=x^4+x^3+x^2+x+1$, 求 $f\left(x^5\right)$ 被 $f(x)$ 除得的余数.
\end{exercise}
\begin{solution}
	设 $f\left(x^5\right)=f(x) q(x)+r(x)$, 这里 $r(x)=0$ 或 $\operatorname{deg} r \leqslant 3$.

	设 $\zeta \neq 1$ 是一个 5 次单位根, 则 $r(\zeta)=r\left(\zeta^2\right)=r\left(\zeta^3\right)=r\left(\zeta^1\right)=5$ , 而 $\operatorname{deg} r \leqslant 3$, 故必须 $r(x)=5$, 即余式是常数 5.
\end{solution}
\end{comment}

\begin{exercise}
	设 $f(x)$ 是复系数多项式, $n$ 是正整数, 求证: 如果 $(x-1) \mid f\left(x^n\right)$, 则 $\left(x^n-1\right) \mid f\left(x^n\right)$.
\end{exercise}
\begin{proof}
	$f\left(x^n\right)=(x-1) g(x)$.
	取 $\zeta=\mathrm{e}^{\frac{2 \pi i}{n}}$ 是一个 $n$ 次单位根, 由 $f(1)=0$ 知,  $f\left(\zeta^{k n}\right)=0(k=1, \cdots, n)$.故 $f\left(x^n\right)$ 被 $(x-\zeta)\left(x-\zeta^2\right) \cdots\left(x-\zeta^n\right)=x^n-1$ 整除.
\end{proof}


\begin{exercise}
	设 $g(\theta)=\lambda_1 \cos \theta+\lambda_2 \cos 2 \theta+\cdots+\lambda_n \cos n \theta$, 其中 $\lambda_1, \lambda_2, \cdots, \lambda_n, \theta$ 均为实数. 若对一切实数 $\theta$, 恒有 $g(\theta) \geqslant-1$. 求证: $\lambda_1+\lambda_2+\cdots+\lambda_n \leqslant n$.
\end{exercise}
\begin{proof}
	令 $\theta_k=\frac{2 k \pi}{n+1}, k=0,1,2, \cdots, n$, 则有
	\begin{align}\label{eq:等分点余弦求和为0}
		\sum_{k=0}^n \cos m \theta_k=\sum_{k=0}^n \sin m \theta_k=0, m=1,2, \cdots, n .
	\end{align}
	(事实上,  $\sum_{k=0}^n \mathrm{e}^{\mathrm{i} m \theta_k}=\frac{1-\mathrm{e}^{\mathrm{i} m \cdot 2 \pi}}{1-\mathrm{e}^{\mathrm{i} m \frac{2 \pi}{n+1}}}=0$ , 于是\autoref{eq:等分点余弦求和为0}成立), 因此
	\begin{align*}
		\begin{aligned}
			  & g(0)+g\left(\theta_1\right)+g\left(\theta_2\right)+\cdots+g\left(\theta_n\right)                                                    \\
			= & \lambda_1\left(\cos 0+\cos \theta_1+\cdots+\cos \theta_n\right)+\lambda_2\left(\cos 0+\cos 2 \theta_1+\cdots+\cos 2 \theta_n\right) \\
			  & +\cdots+\lambda_n\left(\cos 0+\cos n \theta_1+\cdots+\cos n \theta_n\right)=0 .
		\end{aligned}
	\end{align*}

	故由 $g\left(\theta_1\right) \geqslant-1, g\left(\theta_2\right) \geqslant-1, \cdots, g\left(\theta_n\right) \geqslant-1$ 得
	\begin{align*}
		\lambda_1+\lambda_2+\cdots+\lambda_n=g(0)=-\left[g\left(\theta_1\right)+g\left(\theta_2\right)+\cdots+g\left(\theta_n\right)\right] \leqslant n,
	\end{align*}
\end{proof}

\begin{exercise}
	有 $m$ 个男孩与 $n$ 个女孩围坐在一个圆周上 $(m>0, n>0, m+n \geqslant 3)$,将顺序相邻的 3 人中恰有 1 个男孩的组数记作 $a$, 顺序相邻的 3 人中恰有 1 个女孩的组数记作 $b$, 求证: $a-b$ 是 3 的倍数.
\end{exercise}
\begin{proof}
	用 $a_k$ 表示小孩, 且将 $a_k$ 赋值为 $a_k=\left\{\begin{array}{l}\omega, a_k \text { 表示男孩时, } \\ \bar{\omega}, a_k \text { 表示女孩时. }\end{array}\right.$其中 $\omega=-\frac{1}{2}+\frac{\sqrt{3}}{2} \mathrm{i}$, 有 $\omega^{3 m}=1$, 并且
	\begin{align*}
		a_k a_{k+1} a_{k+2}= \begin{cases}\omega^{-1}, & \left(a_k, a_{k+1}, a_{k+2} \text { 中恰有一个男孩 }\right) \\ \omega, & \left(a_k, a_{k+1}, a_{k+2} \text { 中恰有一个女孩 }\right) \\ 1, & \left(a_k, a_{k+1}, a_{k+2} \text { 全是男(女) 孩 }\right)\end{cases}
	\end{align*}
	从而得
	\begin{align*}
		\begin{aligned}
			1 & =\left(a_1 a_2 \cdots a_{m+n}\right)^3                                               \\
			  & =\left(a_1 a_2 a_3\right)\left(a_2 a_3 a_4\right) \cdots\left(a_{m+n} a_1 a_2\right) \\
			  & =\omega^{b-a},
		\end{aligned}
	\end{align*}
	故 $a-b$ 是 3 的倍数.
\end{proof}
\begin{note}
	本题相当于一个复数赋值问题.
\end{note}

\begin{exercise}
	已知单位圆的内接正 $n$ 边形 $A_1 A_2 \cdots A_n$ 及圆周上一点 $P$, 求证:
	\begin{multicols}{2}
		(a) $\dsum_{k=1}^n\left|P A_k\right|^2=2 n$.

		(b) $\dsum_{k=1}^n\left|P A_k\right|^4=6 n$.

		(c) $\dsum_{j, k=1}^n\left|A_j A_k\right|^2=2 n^2$.

		(d) $\dprod_{k=2}^n\left|A_1 A_k\right|=n$.

		(e) $\max \dprod_{k=1}^n\left|P A_k\right|=2$.
	\end{multicols}
\end{exercise}

\begin{comment}
(f) $\max \dsum_{k=1}^n\left|P A_k\right|=\frac{2}{\sin \frac{\pi}{2 n}}$.

(g) $\min \dsum_{k=1}^n\left|P A_k\right|=2 \cot \frac{\pi}{2 n}$.
\end{comment}

\begin{proof}
	以圆心 $O$ 为原点, 设 $A_1, A_2, \cdots, A_n$ 分别为 $1, \varepsilon, \varepsilon^2, \cdots, \varepsilon^{n-1}$, 这里 $\varepsilon$ 为 $n$ 次单位根 $\mathrm{e}^{\frac{2 \pi \mathrm{i}}{n}}$.又设 $P$ 为 $z = \mathrm{e}^{\mathrm{i} \theta}$.

	(a) 见\autoref{ex:已知单位圆的内接正 $n$ 边形}. 关键在于应用了以下两个等式:
	\begin{align}
		\sum_{k=0}^{n-1} \varepsilon^k=0, \\
		z \bar{z} = |z|^2.
	\end{align}

	(b)
	\begin{align*}
		\sum_{k=1}^n\left|P A_k\right|^4 & =\sum_{k=0}^{n-1}\left|z-\varepsilon^k\right|^4                                                                                           \\
		                                 & =\sum_{k=0}^{n-1}\left(z^2-2 z \varepsilon^k+\varepsilon^{2 k}\right)\left(\bar{z}^2-2 \bar{z} \varepsilon^{-k}+\varepsilon^{-2 k}\right) \\
		                                 & =\sum_{k=0}^{n-1}\left(6-4 z \varepsilon^{-k}-4 \bar{z} \varepsilon^k+z^2 \varepsilon^{-2 k}+\bar{z}^2 \varepsilon^{2 k}\right)           \\
		                                 & =6 n
	\end{align*}

	(c)
	\begin{align*}
		\sum_{j, k=1}^n\left|A_j A_k\right|^2 & =\sum_{j, k=1}^n\left(\varepsilon^j-\varepsilon^k\right)\left(\varepsilon^{-j}-\varepsilon^{-k}\right) \\
		                                      & =\sum_{j, k=1}^n\left(2-\varepsilon^{j-k}-\varepsilon^{k-j}\right)                                     \\
		                                      & =2 n^2 .
	\end{align*}

	(d)其实就是\autoref{ex:设varepsilon}中的(1), 首先
	\begin{align*}
		\prod_{k=2}^n\left|A_1 A_k\right|=\left|(1-\varepsilon)\left(1-\varepsilon^2\right) \cdots\left(1-\varepsilon^{n-1}\right)\right|.
	\end{align*}
	因为
	\begin{align}\label{eq:z^n-1}
		z^n-1=(z-1)(z-\varepsilon)\left(z-\varepsilon^2\right) \cdots\left(z-\varepsilon^{n-1}\right),
	\end{align}
	所以
	\begin{align}
		z^{n-1}+z^{n-2}+\cdots+z+1=(z-\varepsilon)\left(z-\varepsilon^2\right) \cdots\left(z-\varepsilon^{n-1}\right).
	\end{align}
	令 $z=1$ 得
	\begin{align*}
		n=(1-\varepsilon)\left(1-\varepsilon^2\right) \cdots\left(1-\varepsilon^{n-1}\right).
	\end{align*}
	从而
	\begin{align*}
		\prod_{k=2}^n\left|A_1 A_k\right|=n.
	\end{align*}

	(e) 在\autoref{eq:z^n-1}两边取模得
	\begin{align*}
		\left|(z-1)(z-\varepsilon) \cdots\left(z-\varepsilon^{n-1}\right)\right|=\left|z^n-1\right| \leqslant|z|^n+1=2,
	\end{align*}
	即 $\dprod_{k=1}^n\left|P A_k\right| \leqslant 2,$

	等号在 $z$ 为 $-1$ 的 $n$ 次根, 即 $P$ 为 $\arc{{A_k}A_{k+1}}(k=1,2, \cdots, n$; $\left.A_{n+1}=A_1\right)$ 中点时成立.
\end{proof}

\begin{comment}
(f) 不妨设 $P$ 在 $\arc{A_n A_1}$ 上, 这时
\begin{align*}
	\sum_{k=1}^n\left|P A_k\right| & =\sum_{k=0}^{n-1}\left|z-\varepsilon^k\right| \cdot\left|\mathrm{e}^{\frac{-(k-1) \pi i}{n}}\right|                                                                                                                                                                   \\
	                               & =\sum_{k=0}^{n-1}\left|z \mathrm{e}^{\frac{-(k-1) \pi i}{n}}-\mathrm{e}^{\frac{(k+1) \times i}{n}}\right|=\left|\sum_{k=0}^{n-1}\left(z \mathrm{e}^{\frac{(k-1) \pi i}{n}}-\mathrm{e}^{\frac{(k+1) \pi i}{n}}\right)\right|                                           \\
	                               & =\left|\frac{2 z \mathrm{e}^{\frac{\pi}{n}}}{1-\mathrm{e}^{-\frac{\pi}{n}}}-\frac{2 \mathrm{e}^{\frac{\pi i}{n}}}{1-\mathrm{e}^{\frac{\pi i}{n}}}\right|=\frac{2\left|z \mathrm{e}^{\frac{\pi j}{n}}+1\right|}{\left|1-\mathrm{e}^{\frac{\pi}{\frac{\pi}{n}}}\right|} \\
	                               & =\frac{2 \sqrt{\left(1+\cos \left(\theta+\frac{\pi}{n}\right)\right)^2+\sin ^2\left(\theta+\frac{\pi}{n}\right)}}{\sqrt{\left(1-\cos \frac{\pi}{n}\right)^2+\sin ^2 \frac{\pi}{n}}}=\frac{2\left|\cos \frac{\theta+\frac{\pi}{n}}{2}\right|}{\sin \frac{\pi}{2 n}}    \\
	                               & \leqslant \frac{2}{\sin \frac{\pi}{2 n}} .
\end{align*}
当且仅当 $\theta=-\frac{\pi}{n}$, 即 $P$ 为 $\arc{A_n A_1}$ 中点时等号成立.

在证明中将 $P A_k$ 旋转 $-\frac{(k-1) \pi}{n}$ (即将 $z-\varepsilon^k$ 乘以 $\mathrm{e}^{-\frac{(k-1) \pi}{n}}$ ),是为了使 $\overrightarrow{P A_k}$ 都与 $\overrightarrow{P A_1}$ 同向 (即相应的复数辐角相同), 从而各复数模的和与和的模相等, 并通过求和得出结果.

(g) 在 $(f)$ 中, $\theta=0$ 或 $-\frac{\pi}{2 n}$ (即 $P$ 与 $A_1$ 或 $A_n$ 重合) 时, 取得最小值.
\end{comment}


\section{复数的模与辐角}
\begin{example}
	对于给定的角 $\alpha_1, \alpha_2, \cdots, \alpha_n$, 试讨论方程
	\begin{align*}
		x^n+x^{n-1} \sin \alpha_1+x^{n-2} \sin \alpha_2+\cdots+x \sin \alpha_{n-1}+\sin \alpha_n=0
	\end{align*}
	是否有模大于 2 的复数根?
\end{example}
\begin{solution}
	答案是否定的. 可以考虑从反面入手去解决.
	假定存在 $x_0$ 是原方程的复数解, 并且 $\left|x_0\right|>2$, 则有
	\begin{align*}
		x_0^n=-x_0^{n-1} \sin \alpha_1-\cdots-x_0 \sin \alpha_{n-1}-\sin \alpha_n,
	\end{align*}
	从而对上式两边取模, 并应用模的不等式, 得
	\begin{align*}
		\begin{aligned}
			\left|x_0\right|^n & \leqslant\left|x_0\right|^{n-1}\left|\sin \alpha_1\right|+\cdots+\left|x_0\right|\left|\sin \alpha_{n-1}\right|+\left|\sin \alpha_n\right|         \\
			                   & \leqslant\left|x_0\right|^{n-1}+\left|x_0\right|^{n-2}+\cdots+\left|x_0\right|+1                                                                   \\
			                   & =\frac{\left|x_0\right|^n-1}{\left|x_0\right|-1}<\frac{\left|x_0\right|^n}{\left|x_0\right|-1}<\frac{\left|x_0\right|^n}{2-1}=\left|x_0\right|^n .
		\end{aligned}
	\end{align*}
	这显然产生矛盾, 由此说明原方程没有模大于 2 的复数根.
\end{solution}
\begin{note}
	将一个等于 0 的式子中起主要作用的项移到 0 的那边, 再两边取模,用不等式放缩, 是一个重要的技巧.
\end{note}

% 这题没什么意思
\begin{comment}
\begin{example}
	设 $n(\geqslant 3)$ 个复数 $z_1, z_2, \cdots, z_n$ 满足

	(1) $z_1+z_2+\cdots+z_n=0$;

	(2) $\left|z_i\right|<1, i=1,2, \cdots, n$.

	证明: 存在 $i ,  j$, 使得 $1 \leqslant i<j \leqslant n$, 且 $\left|z_i+z_j\right|<1$.
\end{example}
\begin{solution}
	只需证明: 复数 $z_i(1 \leqslant i \leqslant n)$ 中, 必有两个复数 $z_k$ 和 $z_l(k \neq l)$, 它们之间的夹角不小于 $120^{\circ}$ .

	对此用反证法予以证明, 若不存在满足条件的 $z_k$ 和 $z_l$, 经过对复平面作适当的旋转, 不妨设 $z_1$ 对应的向量 $\overrightarrow{O Z_1}$ 落在实轴的正半轴上, 作射线 $O A$ ,  $O B$, 使得
	\begin{align*}
		\angle x O A=120^{\circ}, \angle A O B=120^{\circ},
	\end{align*}

	则 $z_2, \cdots, z_n$ 对应的向量 $\overrightarrow{O Z_2}, \cdots, \overrightarrow{O Z}_n$ 都落在 $\angle x O A$ 与 $\angle x O B$ 内, 如图 8-1 所示.

	由 $z_1+z_2+\cdots+z_n=0$ 可知 $z_2, \cdots, z_n$ 中必有一个复数的实部小于 0 . 从而 $\overrightarrow{O Z_2}, \cdots, \overrightarrow{O Z}_n$ 中必有一个向量落在 $\angle y O A$ 或 $\angle y^{\prime} O B$ 内, 不妨设 $\overrightarrow{O Z_2}$ 落在 $\angle y O A$ 内. 作射线 $O C$, 使得 $\angle Z_2 O C=120^{\circ}$, 则 $z_3$, $\cdots, z_n$ 对应的向量不能落在 $\angle B O C$ 内.

	图8-1

	综上所述, 可知 $\overrightarrow{O Z_1}, \overrightarrow{O Z_2}, \cdots, \overrightarrow{O Z_n}$, 都落在 $\angle A O C$ 内, 于是, 将该复平面适当旋转后, 可使向量 $\overrightarrow{O Z_1}, \overrightarrow{O Z_2}, \cdots, \overrightarrow{O Z_n}$ 都落在 $y$ 轴的右方, 它们的实部都不小于零, 这与 (1) 矛盾.

	所以, 在 $z_1, z_2, \cdots, z_n$ 中, 存在 $i ,  j, 1 \leqslant i<j \leqslant n$, 使得 $\left|z_i+z_j\right|<1$,证毕.
\end{solution}
\begin{note}
	证明中没有考虑存在某个 $z_i=0$ 的情形, 因为此时结论是平凡的. 另外, 当证完存在向量 $\overrightarrow{O Z}_i ,  \overrightarrow{O Z}_j$ 所成的角不小于 $120^{\circ}$ 后, 只需利用图82 , 令 $z=z_i+z_j$, 则可知 $\angle Z_i O Z$ 和 $\angle Z O Z_j$ 中必有一个 $\geqslant 60^{\circ}$. 而 $\overrightarrow{Z_i Z}=\overrightarrow{O Z_j}, \overrightarrow{Z_j Z}=\overrightarrow{O Z}_i$ 及 $\angle O Z_i Z=$

	图8-2

	$\angle O Z_j Z=180^{\circ}-\angle Z_i O Z_j \leqslant 60^{\circ}$, 就可知
	\begin{align*}
		|z| \leqslant \max \left\{\left|z_i\right|,\left|z_j\right|\right\}<1 .
	\end{align*}
\end{note}
\end{comment}

% 这题有点难度,暂时删去
\begin{comment}
\begin{example}
	设 $p=\overline{a_n a_{n-1} \cdots a_0}=a_n \times 10^n+a_{n-1} \times 10^{n-1}+\cdots+a_1 \times 10+a_0$ 是十进制表示下的一个质数, 这里 $a_n>0$. 证明: $f(x)=a_n x^n+\cdots+a_0$ 在整系数范围内不可约.
\end{example}
\begin{solution}
	解析 从 $f(x)$ 的根 $x_0$ 出发, 先证明: $\operatorname{Re}\left(x_0\right) \leqslant 0$ 或者 $\left|x_0\right|<4$, 这里 $\operatorname{Re}\left(x_0\right)$ 表示 $x_0$ 的实部.

	事实上, 若 $\operatorname{Re}\left(x_0\right) \leqslant 0$ 或 $\left|x_0\right| \leqslant 1$, 则上述论断已成立. 当 $\operatorname{Re}\left(x_0\right)>0$,且 $\left|x_0\right|>1$ 时, 有 $\operatorname{Re}\left(\frac{1}{x_0}\right)=\frac{\operatorname{Re}\left(x_0\right)}{\left|x_0\right|^2}>0$. 于是, 有
	\begin{align*}
		0=\left|\frac{f\left(x_0\right)}{x_0^n}\right| \geqslant\left|a_n+\frac{a_{n-1}}{x_0}\right|-\frac{a_{n-2}}{\left|x_0\right|^2}-\cdots-\frac{a_0}{\left|x_0\right|^n}
	\end{align*}

	\begin{align*}
		\begin{aligned}
			 & \geqslant \operatorname{Re}\left(a_n+\frac{a_{n-1}}{x_0}\right)-\left(\frac{9}{\left|x_0\right|^2}+\cdots+\frac{9}{\left|x_0\right|^n}\right) \\
			 & \geqslant a_n-\frac{9}{\left|x_0\right|^2-\left|x_0\right|} \geqslant 1-\frac{9}{\left|x_0\right|^2-\left|x_0\right|},
		\end{aligned}
	\end{align*}

	于是 $\left|x_0\right|^2-\left|x_0\right|-9 \leqslant 0$, 故 $\left|x_0\right| \leqslant \frac{1+\sqrt{37}}{2}<4$.
	下面, 利用上述论断证明 $f(x)$ 在整系数范围内中不可约.
	若存在非常数的整系数多项式 $g(x)$ 和 $h(x)$ , 使得 $f(x)=g(x) h(x)$ , 设 $g(x)=b_m\left(x-r_1\right) \cdots\left(x-r_m\right)$ . 对于 $g(10)$ 而言, 一方面 $g(10) \in \mathbb{Z}$ , 另一方面, 对 $1 \leqslant i \leqslant m$ , 由于 $r_i$ 也是 $f(x)$ 的根, 如果 $r_i \in \mathbb{R}$ , 则 $r_i \leqslant 0$ (否则, 由 $f(x)$ 的系数均非负, 将导数 $f\left(r_i\right)>0$, )故 $10-r_i \geqslant 10$ ;如果 $r_i \notin \mathbb{R}$ , 则 $\bar{r}_i$ 也是 $f(x)$的根, 这时
	\begin{align*}
		\left(10-r_i\right)\left(10-\bar{r}_i\right)=100-20 \operatorname{Re}\left(r_i\right)+\left|r_i\right|^2>20,
	\end{align*}

	所以, 总有 $|g(10)|>\left|b_m\right| \geqslant 1$ , 同理 $|h(10)|>1$.
	但是, $f(10)=g(10) h(10)$ 为质数,矛盾. 证毕.
	注 从本题的证明过程中我们知道:多项式根的分布情况对多项式的分解起着举足轻重的作用.
\end{solution}
\end{comment}

% 这道题目关键在于Vieta定理
\begin{example}
	是否存在 2002 个不同的正实数 $a_1, a_2, \cdots, a_{2002}$ , 使得对任意正整数 $k, 1 \leqslant k \leqslant 2002$, 多项式 $a_{k+2001} x^{2001}+a_{k+2000} x^{2000}+\cdots+a_{k+1} x+a_k$ 的每个复根 $z$ 都满足 $|\operatorname{Im} z| \leqslant|\operatorname{Re} z|$ ?(约定 $a_{2002+i}=a_i, i=1,2, \cdots, 2001$ )
\end{example}
\begin{solution}
	不存在.

	用反证法. 若存在正实数 $a_1, a_2, \cdots, a_{2002}$ 满足题设要求, 对一固定的 $k$ , 设 $a_{k+2001} x^{2001}+a_{k+2000} x^{2000}+\cdots+a_{k+1} x+a_k=0$ 的复根为 $z_1, z_2, \cdots, z_{2001}$ , 那么由于 $\left|\operatorname{Im} z_j\right| \leqslant\left|\operatorname{Re} z_j\right|(1 \leqslant j \leqslant 2001)$ , 而
	\begin{align*}
		\begin{aligned}
			z_j^2 & =\left(\operatorname{Re} z_j+\mathrm{i} \operatorname{Im} z_j\right)^2                                                                                       \\
			      & =\left(\operatorname{Re} z_j\right)^2-\left(\operatorname{Im} z_j\right)^2+2\left(\operatorname{Re} z_j\right)\left(\operatorname{Im} z_j\right) \mathrm{i},
		\end{aligned}
	\end{align*}
	即 $z_j^2$ 的实部 $\operatorname{Re}\left(z_j^2\right)=\left(\operatorname{Re} z_j\right)^2-\left(\operatorname{Im} z_j\right)^2 \geqslant 0(1 \leqslant j \leqslant 2001)$ , 所以
	\begin{align*}
		\operatorname{Re}\left(z_1^2+z_2^2+\cdots+z_{2001}^2\right)=\operatorname{Re}\left(z_1^2\right)+\operatorname{Re}\left(z_2^2\right)+\cdots+\operatorname{Re}\left(z_{2001}^2\right) \geqslant 0 .
	\end{align*}
	而由韦达定理
	\begin{align*}
		z_1+z_2+\cdots+z_{2001}=\frac{-a_{k+2000}}{a_{k+2001}},
	\end{align*}
	\begin{align*}
		\sum_{1 \leqslant j \leqslant l \leqslant 2001} z_j z_l=\frac{a_{k+1999}}{a_{k+2001}}
	\end{align*}
	所以
	\begin{align*}
		\begin{aligned}
			z_1^2+z_2^2+\cdots+z_{2001}^2 & =\left(z_1+z_2+\cdots+z_{2001}\right)^2-2 \sum_{1 \leqslant j \leqslant 1 \leqslant 2001} z_j z_l \\
			                              & =\frac{a_{k+2000}^2-2 a_{k+1999} a_{k+2001}}{a_{k+2001}^2},
		\end{aligned}
	\end{align*}
	即 $z_1^2+z_2^2+\cdots+z_{2001}^2$ 是一个实数.
	又由(1)知, 其实部 $\geqslant 0$, 所以它是一个非负实数, 即
	\begin{align*}
		\frac{a_{k+2000}^2-2 a_{k+1999} a_{k+2001}}{a_{k+2001}^2} \geqslant 0 \Rightarrow a_{k+2000}^2-2 a_{k+1999} a_{k+2001} \geqslant 0 .
	\end{align*}
	上式对每个 $1 \leqslant k \leqslant 2002$ 均成立, 即当 $1 \leqslant j \leqslant 2002$, 均有 $a_j^2-$ $2 a_{j-1} a_{j+1} \geqslant 0$ . 但这是不可能的, 事实上:
	设 $a_{j_0}$ 是 $a_1, a_2, \cdots, a_{2002}$ 中最小的一个, 那么
	\begin{align*}
		a_{j_0}^2-2 a_{j_0-1} a_{j_0+1} \leqslant a_{j_0}^2-2 a_{j_0} a_{j_0}=-a_{j_0}^2<0,
	\end{align*}
	矛盾.
\end{solution}

% 这些题目有点难度,暂时删去
\begin{comment}
\begin{example}
	$n$ 是正整数, $a_j(j=1,2, \cdots, n)$ 为复数, 且对集合 $\{1,2, \cdots, n\}$的任一非空子集 $I$, 均有
	\begin{align*}
		\left|\prod_{j \in I}\left(1+a_j\right)-1\right| \leqslant \frac{1}{2} .
	\end{align*}
	证明: $\sum_{j=1}^n\left|a_j\right| \leqslant 3$.
\end{example}
\begin{solution}
	设 $1+a_j=r_j \mathrm{e}^{i_j},\left|\theta_j\right| \leqslant \pi, j=1,2, \cdots, n$, 则题设条件变为
	\begin{align*}
		\left|\prod_{j \in I} r_j \cdot \mathrm{e}^{i \sum_{j \in I_j}^{\theta_j}}-1\right| \leqslant \frac{1}{2} .
	\end{align*}

	先证如下引理: 设 $r, \theta$ 为实数, $r>0,|\theta| \leqslant \pi$, $\left|r \mathrm{e}^{i \theta}-1\right| \leqslant \frac{1}{2}$, 则 $\frac{1}{2} \leqslant r \leqslant \frac{3}{2},|\theta| \leqslant \frac{\pi}{6}$, $\left|r \mathrm{e}^{i \theta}-1\right| \leqslant|r-1|+|\theta|$.

	引理的证明: 如图 8-3, 由复数的几何意义, 有 $\frac{1}{2} \leqslant r \leqslant \frac{3}{2},|\theta| \leqslant \frac{\pi}{6}$.
	又由

	图8-3
	\begin{align*}
		\begin{aligned}
			\left|r \mathrm{e}^{i \theta}-1\right| & =|r(\cos \theta+\mathrm{i} \sin \theta)-1|                                            \\
			                                       & =|(r-1)(\cos \theta+\mathrm{i} \sin \theta)+[(\cos \theta-1)+\mathrm{i} \sin \theta]| \\
			                                       & \leqslant|r-1|+\sqrt{(\cos \theta-1)^2+\sin ^2 \theta}                                \\
			                                       & =|r-1|+\sqrt{2(1-\cos \theta)}                                                        \\
			                                       & =|r-1|+2\left|\sin \frac{\theta}{2}\right|                                            \\
			                                       & \leqslant|r-1|+|\theta|,
		\end{aligned}
	\end{align*}

	得引理的另一部分.
	由(1)及引理, 对 $|I|$ 用数学归纳法知:
	\begin{align*}
		\frac{1}{2} \leqslant \prod_{j \in I} r_j \leqslant \frac{3}{2},\left|\sum_{j \in I} \theta_j\right| \leqslant \frac{\pi}{6},
	\end{align*}

	由(1)及引理知
	\begin{align*}
		\left|a_j\right|=\left|r_j e^{i_j}-1\right| \leqslant\left|r_j-1\right|+\left|\theta_j\right|,
	\end{align*}

	因此
	\begin{align*}
		\begin{aligned}
			\sum_{j=1}^n\left|a_j\right| & \leqslant \sum_{j=1}^n\left|r_j-1\right|+\sum_{j=1}^n\left|\theta_j\right|                                                                                         \\
			                             & =\sum_{r_j \geqslant 1}\left|r_j-1\right|+\sum_{r_j<1}\left|r_j-1\right|+\sum_{\theta_j \geqslant 0}\left|\theta_j\right|+\sum_{\theta_j<0}\left|\theta_j\right| .
		\end{aligned}
	\end{align*}

	由(2)知
	\begin{align*}
		\begin{aligned}
			\sum_{r_j \geqslant 1}\left|r_j-1\right| & =\sum_{r_j \geqslant 1}\left(r_j-1\right) \leqslant \prod_{r_j \geqslant 1}\left(1+r_j-1\right)-1                                             \\
			                                         & \leqslant \frac{3}{2}-1=\frac{1}{2},                                                                                                          \\
			\sum_{r_j<1}\left|r_j-1\right|           & =\sum_{r_j<1}\left(1-r_j\right) \leqslant \prod_{r_j<1}\left(1-\left(1-r_j\right)\right)^{-1}-1                                               \\
			                                         & \leqslant 2-1=1,                                                                                                                              \\
			\sum_{j=1}^n\left|\theta_j\right|=       & \sum_{\theta_j \geqslant 0} \theta_j-\sum_{\theta_j<0} \theta_j \leqslant \frac{\pi}{6}-\left(-\frac{\pi}{6}\right) \leqslant \frac{\pi}{3} .
		\end{aligned}
	\end{align*}

	综上, 有
	\begin{align*}
		\sum_{j=1}^n\left|a_j\right| \leqslant \frac{1}{2}+1+\frac{\pi}{3}<3 .
	\end{align*}

	证毕.
\end{solution}

\begin{example}
	设复数 $x ,  y ,  z$ 满足 $|x|^2+|y|^2+|z|^2=1$. 证明:
	\begin{align*}
		\left|x^3+y^3+z^3-3 x y z\right| \leqslant 1 .
	\end{align*}
\end{example}
\begin{solution}
	设 $w=\mathrm{e}^{\frac{2 \pi j}{3}}$ , 我们有因式分解
	\begin{align*}
		x^3+y^3+z^3-3 x y z=(x+y+z)\left(x+w y+w^2 z\right)\left(x+w^2 y+w z\right) .
	\end{align*}
	故
	\begin{align*}
		\begin{aligned}
			          & \left|x^3+y^3+z^3-3 x y z\right|^2                                                                                           \\
			=         & |x+y+z|^2 \cdot\left|x+w y+w^2 z\right|^2 \cdot\left|x+w^2 y+w z\right|^2                                                    \\
			\leqslant & \frac{1}{27}\left(|x+y+z|^2+\left|x+w y+w^2 z\right|^2+\left|x+w^2 y+w z\right|^2\right)^3                                   \\
			=         & \frac{1}{27}\left((x+y+z)(\bar{x}+\bar{y}+\bar{z})+\left(x+w y+w^2 z\right)\left(\bar{x}+w^2 \bar{y}+w \bar{z}\right)\right. \\
			          & \left.+\left(x+w^2 y+w z\right)\left(\bar{x}+w \bar{y}+w^2 \bar{z}\right)\right)^3                                           \\
			=         & \left(|x|^2+|y|^2+|z|^2\right)^3=1 .
		\end{aligned}
	\end{align*}

	因此, $\left|x^3+y^3+z^3-3 x y z\right| \leqslant 1$.
\end{solution}

\begin{example}
	设 $x_1, x_2, \cdots, x_n(n \geqslant 2)$ 是 $n$ 个实数,满足
	\begin{align*}
		\begin{gathered}
			A=\left|\sum_{i=1}^n x_i\right| \neq 0, \\
			B=\max _{1 \leq i<j \leqslant n}\left|x_j-x_i\right| \neq 0 .
		\end{gathered}
	\end{align*}
	求证:对平面上的任意 $n$ 个向量 $\alpha_1, \alpha_2, \cdots, \alpha_n$ , 存在 $1,2, \cdots, n$ 的一个排列 $k_1, k_2, \cdots, k_n$ 使得
	\begin{align*}
		\left|\sum_{i=1}^n x_{k_i} \alpha_i\right| \geqslant \frac{A B}{2 A+B} \max _{1 \leqslant i \leqslant n}\left|\alpha_i\right| .
	\end{align*}
\end{example}
\begin{solution}
	设 $\left|\alpha_k\right|=\max _{1 \leqslant i \leqslant n}\left|\alpha_i\right|, k \in\{1,2, \cdots, n\}$. 我们只须证明
	\begin{align*}
		\max _{\left(k_1, k_2, \cdots, k_n\right) \in S_n}\left|\sum_{i=1}^n x_{k_i} \alpha_i\right| \geqslant \frac{A B}{2 A+B}\left|\alpha_k\right|,
	\end{align*}

	其中 $S_n$ 为 $1,2, \cdots, n$ 的排列的集合.
	不妨设
	\begin{align*}
		\begin{gathered}
			\left|x_n-x_1\right|=\max _{1 \leqslant i<j \leqslant n}\left|x_j-x_i\right|=B, \\
			\left|\alpha_n-\alpha_1\right|=\max _{1 \leqslant i<j \leqslant n}\left|\alpha_j-\alpha_i\right| .
		\end{gathered}
	\end{align*}
	考虑两个向量
	\begin{align*}
		\begin{aligned}
			 & \beta_1=x_1 \alpha_1+x_2 \alpha_2+\cdots+x_{n-1} \alpha_{n-1}+x_n \alpha_n, \\
			 & \beta_2=x_n \alpha_1+x_2 \alpha_2+\cdots+x_{n-1} \alpha_{n-1}+x_1 \alpha_n,
		\end{aligned}
	\end{align*}

	则
	\begin{align*}
		\begin{aligned}
			          & \max _{\left(k_1, k_2, \cdots, k_n\right) \in S_n}\left|\sum_{i=1}^n x_{k_i} a_i\right|                             \\
			\geqslant & \max \left\{\left|\beta_1\right|,\left|\beta_2\right|\right\}                                                       \\
			\geqslant & \frac{1}{2}\left(\left|\beta_1\right|+\left|\beta_2\right|\right) \geqslant \frac{1}{2}\left|\beta_2-\beta_1\right| \\
			=         & \frac{1}{2}\left|x_1 \alpha_n+x_n \alpha_1-x_1 \alpha_1-x_n \alpha_n\right|                                         \\
			=         & \frac{1}{2}\left|x_n-x_1\right|\left|\alpha_n-\alpha_1\right|                                                       \\
			=         & \frac{1}{2} B\left|\alpha_n-\alpha_1\right| .
		\end{aligned}
	\end{align*}

	设 $\left|\alpha_n-\alpha_1\right|=x\left|a_k\right|$, 由三角形不等式易知 $0 \leqslant x \leqslant 2$. 因此(1)中的不等式可写为
	\begin{align*}
		\max _{\left(k_1, k_2, \cdots, k_n\right) \in S_n}\left|\sum_{i=1}^n x_{k_i} \alpha_i\right| \geqslant \frac{1}{2} B x\left|\alpha_k\right| .
	\end{align*}

	另一方面, 考虑 $n$ 个向量
	\begin{align*}
		\begin{gathered}
			\gamma_1=x_1 \alpha_1+x_2 \alpha_2+\cdots+x_{n-1} \alpha_{n-1}+x_n \alpha_n, \\
			\gamma_2=x_2 \alpha_1+x_3 \alpha_2+\cdots+x_n \alpha_{n-1}+x_1 \alpha_n, \\
			\gamma_3=x_3 \alpha_1+x_4 \alpha_2+\cdots+x_1 \alpha_{n-1}+x_2 \alpha_n, \\
			\cdots \\
			\gamma_n=x_n \alpha_1+x_1 \alpha_2+\cdots+x_{n-2} \alpha_{n-1}+x_{n-1} \alpha_n .
		\end{gathered}
	\end{align*}

	则
	\begin{align*}
		\begin{aligned}
			          & \max _{\left(k_1, k_2, \cdots, k_n\right) \in S_n}\left|\sum_{i=1}^n x_{k_i} a_i\right|                                                                                                 \\
			\geqslant & \max _{1 \leqslant i \leqslant n}\left|\gamma_i\right| \geqslant \frac{1}{n}\left(\left|\gamma_1\right|+\left|\gamma_2\right|+\cdots+\left|\gamma_n\right|\right)                       \\
			\geqslant & \frac{1}{n}\left|\gamma_1+\gamma_2+\cdots+\gamma_n\right|=\frac{A}{n}\left|\alpha_1+\alpha_2+\cdots+\alpha_n\right|                                                                     \\
			=         & \frac{A}{n}\left|n \alpha_k-\sum_{j \neq k}\left(\alpha_k-\alpha_j\right)\right| \geqslant \frac{A}{n}\left(n\left|\alpha_k\right|-\sum_{j \neq k}\left|\alpha_k-\alpha_j\right|\right)
		\end{aligned}
	\end{align*}
	\begin{align*}
		\begin{aligned}
			 & \geqslant \frac{A}{n}\left(n\left|\alpha_k\right|-(n-1)\left|\alpha_n-\alpha_1\right|\right) \\
			 & =\frac{A}{n}\left(n\left|\alpha_k\right|-(n-1) x\left|\alpha_k\right|\right)                 \\
			 & =A\left(1-\frac{n-1}{n} x\right)\left|\alpha_k\right| .
		\end{aligned}
	\end{align*}

	结合(2), (3),可得
	\begin{align*}
		\begin{aligned}
			\max _{\left(k_1, k_2, \cdots, k_n\right) \in S_n}\left|\sum_{i=1}^n x_{k_i} \alpha_i\right| & \geqslant \max \left\{\frac{B x}{2}, A\left(1-\frac{n-1}{n} x\right)\right\}\left|\alpha_k\right|                                                        \\
			                                                                                             & \geqslant \frac{\frac{B x}{2} \cdot A \cdot \frac{n-1}{n}+A\left(1-\frac{n-1}{n} x\right) \frac{B}{2}}{A \frac{n-1}{n}+\frac{B}{2}}\left|\alpha_k\right| \\
			                                                                                             & =\frac{A B}{2 A+B-\frac{2 A}{n}}\left|\alpha_k\right| \geqslant \frac{A B}{2 A+B}\left|\alpha_k\right| .
		\end{aligned}
	\end{align*}
	证毕.
\end{solution}

\begin{example}
	复系数多项式 $f(z)=z^n+a_1 z^{n-1}+\cdots+a_{n-1} z+a_n$ 的 $n$ 个根为 $z_1$, $z_2, \cdots, z_n$, 且 $\sum_{k=1}^n\left|a_k\right|^2 \leqslant 1$, 求证: $\sum_{k=1}^n\left|z_k\right|^2 \leqslant n$.
\end{example}
\begin{solution}
	引理 1:若正实数 $x_1, x_2, \cdots, x_m$ 都不大于 1(或都不小于 1), 则
	\begin{align*}
		x_1+x_2+\cdots+x_m \leqslant(m-1)+x_1 x_2 \cdots x_m .
	\end{align*}

	引理1的证明:因为
	\begin{align*}
		\begin{aligned}
			  & (m-1)+x_1 x_2 \cdots x_m-\left(x_1+x_2+\cdots+x_m\right)                                                                           \\
			= & \left(1-x_1\right)\left(1-x_2\right)+\left(1-x_1 x_2\right)\left(1-x_3\right)+\left(1-x_1 x_2 x_3\right)\left(1-x_4\right)+\cdots+ \\
			  & \left(1-x_1 x_2 \cdots x_{m-1}\right)\left(1-x_m\right) \geqslant 0,
		\end{aligned}
	\end{align*}

	故引理1成立.


	引理 2: 若 $f(x)=g(x) h(x)(f, g, h$ 均为复系数多项式) 满足
	\begin{align*}
		\begin{gathered}
			f(x)=a_0 x^n+a_1 x^{n-1}+\cdots+a_{n-1} x+a_n, \\
			g(x)=b_0 x^k+b_1 x^{k-1}+\cdots+b_{k-1} x+b_k, \quad(n=k+l) \\
			h(x)=c_0 x^l+c_1 x^{l-1}+\cdots+c_{l-1} x+c_l,
		\end{gathered}
	\end{align*}

	则
	\begin{align*}
		\left|b_0 c_l\right|^2+\left|c_0 b_k\right|^2 \leqslant\left|a_0\right|^2+\left|a_1\right|^2+\cdots+\left|a_n\right|^2 .
	\end{align*}
	引理2的证明: 由已知可知 $a_m=\sum_i b_i c_{m-i}(m=0,1,2, \cdots, n)$ (规定 $p \geqslant$ $k+1$ 或 $p \leqslant-1$ 时, $b_p=0 ; q \geqslant l+1$ 或 $q \leqslant-1$ 时, $\left.c_q=0\right)$.
	考虑
	\begin{align*}
		\begin{aligned}
			  & \sum_{m^{\prime} \in \mathbb{Z}}\left|\sum_i b_i \overline{c_{m^{\prime}+i}}\right|^2                                                                                                                     \\
			= & \sum_{m^{\prime} \in \mathbb{Z}}\left(\sum_i b_i \overline{c_{m^{\prime}+i}}\right)\left(\sum_i \overline{b_j c_{m^{\prime}+j}}\right)                                                                    \\
			= & \sum_{m^{\prime} \in \mathbb{Z}}\left(\sum_{i, j} b_i c_{m^{\prime}+j}\right) \overline{b_j c_{m^{\prime}+i}}=\sum_{m^{\prime} \in \mathbb{Z}, i, j} b_i c_{m^{\prime}+j} \overline{b_j c_{m^{\prime}+i}} \\
			= & \sum_{(m i j), i, j} b_i c_{m-i} \overline{b_j c_{m-j}}=\sum_m \sum_{i, j} b_i c_{m-i} \overline{b_j c_{m-j}}                                                                                             \\
			= & \sum_m\left(\sum_i b_i c_{m-i}\right)\left(\sum_j \overline{b_j c_{m-j}}\right)                                                                                                                           \\
			= & \sum_m\left|a_m\right|^2\left(m \leqslant-1 \text { 或 } m \geqslant n+1 \text { 时 } a_m=0\right),
		\end{aligned}
	\end{align*}

	所以
	\begin{align*}
		\begin{aligned}
			\sum_{m=0}^n\left|a_m\right|^2 & =\sum_{m^{\prime} \in \mathbb{Z}}\left|\sum_i b_i \overline{c_{m^{\prime}+i}}\right|^2             \\
			                               & \geqslant\left|\sum_i b_i \overline{c_{i-k}}\right|^2+\left|\sum_i b_i \overline{c_{i+l}}\right|^2 \\
			                               & =\left|b_k \overline{c_0}\right|^2+\left|b_0 \overline{c_l}\right|^2                               \\
			                               & =\left|b_k c_0\right|^2+\left|b_0 c_l\right|^2,
		\end{aligned}
	\end{align*}

	引理2 得证.

	下面看原命题: 设 $z_1, z_2, \cdots, z_n$ 中, $z_1, z_2, \cdots, z_k$ 的模小于 $1, z_{k+1}$, $z_{k+2}, \cdots, z_n$ 的模不小于 1 . 则由引理 1 可得
	\begin{align*}
		\begin{aligned}
			\sum_{i=1}^n\left|z_i\right|^2 & =\sum_{i=1}^k\left|z_i\right|^2+\sum_{j=k+1}^n\left|z_j\right|^2                                    \\
			                               & \leqslant k-1+\left|z_1 z_2 \cdots z_k\right|^2+(n-k-1)+\left|z_{k+1} z_{k+2} \cdots z_n\right|^2 .
		\end{aligned}
	\end{align*}

	在引理 2 中令
	\begin{align*}
		\begin{gathered}
			g(x)=\left(x-z_1\right)\left(x-z_2\right) \cdots\left(x-z_k\right), \\
			h(x)=\left(x-z_{k+1}\right)\left(x-z_{k+2}\right) \cdots\left(x-z_n\right),
		\end{gathered}
	\end{align*}

	则有
	\begin{align*}
		\left|z_1 z_2 \cdots z_k\right|^2+\left|z_{k+1} z_{k+2} \cdots z_n\right|^2 \leqslant 1^2+\left|a_1\right|^2+\cdots+\left|a_n\right|^2 \leqslant 2,
	\end{align*}

	所以 $\sum_{i=1}^n\left|z_i\right|^2 \leqslant k-1+(n-k-1)+2=n$, 证毕.
\end{solution}
\end{comment}

\section{赛题选讲}
\begin{example}
	给定一个凸六边形,其任意两条对边具有如下性质: 它们的中点之间的距离等于它们的长度和的 $\frac{\sqrt{3}}{2}$ 倍. 证明:该六边形的所有内角相等(一个凸六边形 $A B C D E F$ 有 3 组对边: $A B$ 和 $D E, B C$ 和 $E F, C D$ 和 $F A$ ).
\end{example}
\begin{solution}
	引理: $\triangle P Q R$ 中, $\angle Q P R \geqslant 60^{\circ}, L$ 为 $Q R$ 中点. 则 $P L \leqslant \frac{\sqrt{3}}{2} Q R$,等号当且仅当 $\triangle P Q R$ 为正三角形时取到.

	引理的证明: 设 $S$ 为平面上一点, 使得 $P$ 与 $S$ 在 $Q R$ 的同侧, 而 $\triangle Q R S$ 为正三角形. 则由于 $\angle Q P R \geqslant 60^{\circ}$, 故 $P$ 在 $\triangle Q R S$ 的外接圆的内部 (包括边界).而 $\triangle Q R S$ 的外接圆落在以 $L$ 为圆心, $\frac{\sqrt{3}}{2} Q R$ 为半径的圆内. 所以引理获证.

	设 $A B C D E F$ 为给定的凸六边形, 记 $\vec{a}=\overrightarrow{A B}, \vec{b}=\overrightarrow{B C}, \cdots, \vec{f}=\overrightarrow{F A}$. 并设 $M 、 N$ 分别为 $A B$ 和 $D E$ 的中点. 则
	\begin{align*}
		\begin{aligned}
			\overrightarrow{M N} & =\frac{1}{2} \vec{a}+\vec{b}+\vec{c}+\frac{1}{2} \vec{d},   \\
			\overrightarrow{M N} & =-\frac{1}{2} \vec{a}-\vec{f}-\vec{e}-\frac{1}{2} \vec{d} .
		\end{aligned}
	\end{align*}
	于是
	\begin{align}\label{eq:3.4.1-1}
		\overrightarrow{M N}=\frac{1}{2}(\vec{b}+\vec{c}-\vec{e}-\vec{f}).
	\end{align}
	由条件, 我们有
	\begin{align}\label{eq:3.4.1-2}
		\overrightarrow{M N}=\frac{\sqrt{3}}{2}(|\vec{a}|+|\vec{d}|) \geqslant \frac{\sqrt{3}}{2}|\vec{a}-\vec{d}| .
	\end{align}
	记 $\vec{x}=\vec{a}-\vec{d}, \vec{y}=\vec{c}-\vec{f}, \vec{z}=\vec{e}-\vec{b}$, 由\autoref{eq:3.4.1-1}\autoref{eq:3.4.1-2}可得
	\begin{align}\label{eq:3.4.1-3}
		 & |\vec{y}-\vec{z}| \geqslant \sqrt{3}|\vec{x}| . \\
		 & |\vec{z}-\vec{x}| \geqslant \sqrt{3}|\vec{y}|,  \\
		 & |\vec{x}-\vec{y}| \geqslant \sqrt{3}|\vec{z}| .
	\end{align}
	上式可写成
	\begin{align*}
		|\vec{y}|^2-2 \vec{y} \cdot \vec{z}+|\vec{z}|^2 \geqslant 3|\vec{x}|^2; \\
		|\vec{z}|^2-2 \vec{z} \cdot \vec{x}+|\vec{x}|^2 \geqslant 3|\vec{y}|^2; \\
		|\vec{x}|^2-2 \vec{x} \cdot \vec{y}+|\vec{y}|^2 \geqslant 3|\vec{z}|^2.
	\end{align*}
	上述 3 式相加,得
	\begin{align*}
		-|\vec{x}|^2-|\vec{y}|^2-|\vec{z}|^2-2 \vec{y} \cdot \vec{z}-2 \vec{z} \cdot \vec{x}-2 \vec{x} \cdot \vec{y} \geqslant 0 .
	\end{align*}
	即 $-|\vec{x}+\vec{y}+\vec{z}| \geqslant 0$. 因此 $\vec{x}+\vec{y}+\vec{z}=0$, 并且上述所有不等式全部取等号. 于是
	\begin{align*}
		\vec{x}+\vec{y}+\vec{z}=0,
	\end{align*}
	\begin{align*}
		\begin{aligned}
			 & |\vec{y}-\vec{z}|=\sqrt{3}|\vec{x}|, \vec{a} / / \vec{d} / / \vec{x},  \\
			 & |\vec{z}-\vec{x}|=\sqrt{3}|\vec{y}|, \vec{c} / / \vec{f} / / \vec{y},  \\
			 & |\vec{x}-\vec{y}|=\sqrt{3}|\vec{z}|, \vec{e} / / \vec{b} / / \vec{z} .
		\end{aligned}
	\end{align*}
	现在设 $\triangle P Q R$ 中, $\overrightarrow{P Q}=\vec{x}, \overrightarrow{Q R}=\vec{y}, \overrightarrow{R P}=\vec{z}$, 并不妨设 $\angle Q P R \geqslant 60^{\circ}$. $L$ 为 $Q R$ 中点, 则 $P L=\frac{1}{2}|\vec{z}-\vec{x}|=\frac{\sqrt{3}}{2}|\vec{y}|=\frac{\sqrt{3}}{2} Q R$. 利用引理可知, $\triangle P Q R$ 为正三角形. 于是, $\angle A B C=\angle B C D=\cdots=\angle F A B=120^{\circ}$, 证毕.
\end{solution}

\begin{example}
	设实数 $a 、 b 、 c 、 d$ 满足 $b-d \geqslant 5$ ,实数 $x_1 、 x_2 、 x_3 、 x_4$ 为多项式 $P(x)=x^4+a x^3+b x^2+c x+d$ 的零点, 求 $\left(x_1^2+1\right)\left(x_2^2+1\right)\left(x_3^2+1\right)\left(x_4^2+1\right)$的最小值.
\end{example}
\begin{solution}
	\begin{align*}
		\begin{aligned}
			\prod_{k=1}^4\left(x_k^2+1\right) & =\prod_{k=1}^4\left(x_k-\mathrm{i}\right)\left(x_k+\mathrm{i}\right)                                                                            \\
			                                  & =\left[\prod_{k=1}^4\left(\mathrm{i}-x_k\right)\right]\left[\prod_{k=1}^4\left(-\mathrm{i}-x_k\right)\right]=P(\mathrm{i}) \cdot P(-\mathrm{i}) \\
			                                  & =\left(\mathrm{i}^4+a \mathrm{i}^3+b \mathrm{i}^2+c \mathrm{i}+d\right)\left(\mathrm{i}^4-a \mathrm{i}^3+b \mathrm{i}^2-c \mathrm{i}+d\right)   \\
			                                  & =[(1+d-b)+\mathrm{i}(c-a)][(1+d-b)-\mathrm{i}(c-a)]                                                                                             \\
			                                  & =(b-d-1)^2+(c-a)^2,
		\end{aligned}
	\end{align*}
	由 $b-d \geqslant 5$ 知, $(b-d-1)^2+(c-a)^2 \geqslant 4^2+0^2=16$, 当
	\begin{align*}
		P(x)=x^4+4 x^3+6 x^2+4 x+1=(x+1)^4
	\end{align*}
	时, $b-d=5, x_1=x_2=x_3=x_4=-1,\left(x_1^2+1\right)\left(x_2^2+1\right)\left(x_3^2+1\right)\left(x_4^2+1\right)$取到最小值 16.
\end{solution}

\begin{example}
	设 $x_1, x_2, \cdots, x_n ; y_1, y_2, \cdots, y_n$ 均为模等于 1 的复数,设
	\begin{align*}
		z_i=x y_i+y x_i-x_i y_i(i=1,2, \cdots, n),
	\end{align*}
	其中 $x=\frac{1}{n} \sum_{i=1}^n x_i, y=\frac{1}{n} \sum_{i=1}^n y_i$.

	证明: $\sum_{i=1}^n\left|z_i\right| \leqslant n$.
\end{example}
\begin{solution}
	因为
	\begin{equation}\label{eq:3.4.3-1}
		\begin{aligned}
			2 \sum_{i=1}^n\left|z_i\right| & =\sum_{i=1}^n\left|\left(2 x y_i-x_i y_i\right)+\left(2 y x_i-x_i y_i\right)\right|         \\
			                               & \leqslant \sum_{i=1}^n\left|2 x y_i-x_i y_i\right|+\sum_{i=1}^n\left|2 y x_i-x_i y_i\right| \\
			                               & =\sum_{i=1}^n\left|2 x-x_i\right|+\sum_{i=1}^n\left|2 y-y_i\right| .
		\end{aligned}
	\end{equation}
	由柯西不等式知
	\begin{align*}
		\begin{aligned}
			\left(\sum_{i=1}^n\left|2 x-x_i\right|\right)^2 & \leqslant n \sum_{i=1}^n\left|2 x-x_i\right|^2                                             \\
			                                                & =n \sum_{i=1}^n\left(2 x-x_i\right)\left(2 \bar{x}-\overline{x_i}\right)                   \\
			                                                & =n \cdot\left(4 n|x|^2+n-2 x \sum_{i=1}^n \overline{x_i}-2 \bar{x} \sum_{i=1}^n x_i\right) \\
			                                                & =n^2 .
		\end{aligned}
	\end{align*}
	即
	\begin{align}\label{eq:3.4.3-2}
		\sum_{i=1}^n\left|2 x-x_i\right| \leqslant n,
	\end{align}
	同理
	\begin{align}\label{eq:3.4.3-3}
		\sum_{i=1}^n\left|2 y-y_i\right| \leqslant n .
	\end{align}
	由\autoref{eq:3.4.3-1}\autoref{eq:3.4.3-2}\autoref{eq:3.4.3-3}即得 $\sum_{i=1}^n\left|z_i\right| \leqslant n$, 得证.
\end{solution}