\chapter{三角函数}
\section{三角函数图象与性质}
\begin{example}
	(1) 将函数 $f(x)=2 \sin 2 x$ 的图象向左平移 $\frac{\pi}{12}$ 个单位长度得到 $g(x)$ 的图象. 若函数 $g(x)$ 在区间 $\left[0, \frac{a}{3}\right],\left[2 a, \frac{7 \pi}{6}\right]$ 上单调增, 求实数 $a$ 的取值范围. (2017 年清华大学 THUSSAT 测试题)

	(2) 已知函数 $f(x)=\sin (\pi x), g(x)=\left\{\begin{array}{ll}\frac{1}{2-2 x}, & x \neq 1, \\ 0, & x=1,\end{array}\right.$ 则函数 $h(x)=f(x)-g(x), x \in(-2,4]$ 的所有零点的和是 \$ \textbackslash qquad \$ .
\end{example}

\begin{solution}
	(1) 根据题意,
	$$
		g(x)=2 \sin \left(2 x+\frac{\pi}{6}\right)
	$$

	它的单调递增区间为
	$$
		\cdots,\left[-\frac{\pi}{3}, \frac{\pi}{6}\right],\left[\frac{2 \pi}{3}, \frac{7 \pi}{6}\right], \cdots,\left[-\frac{\pi}{3}+k \pi, \frac{\pi}{6}+k \pi\right], k \in \mathbf{N}_{+}
	$$

	于是 $\left\{\begin{array}{l}0<\frac{a}{3} \leqslant \frac{\pi}{6}, \\ -\frac{\pi}{3}+k \pi \leqslant 2 a \leqslant \frac{\pi}{6}+k \pi, \quad k \in \mathbf{N}_{+}\end{array}\right. $
    
    $\Rightarrow\left\{\begin{array}{l}0<a \leqslant \frac{\pi}{2}, \\ -\frac{\pi}{6}+\frac{k \pi}{2} \leqslant a \leqslant \frac{\pi}{12}+\frac{k \pi}{2}, \quad k \in \mathbf{N}_{+},\end{array}\right.$

	所以
	$$
		\frac{\pi}{3} \leqslant a \leqslant \frac{\pi}{2}
	$$

	所以实数 $a$ 的取值范围是 $\left[\frac{\pi}{3}, \frac{\pi}{2}\right]$.

	(2) 如图 $1-1$,

	图 1 - 1

	函数 $f(x)$ 和 $g(x)$ 的图象均关于点 $(1,0)$ 对称, 且
	$$
		f\left(\frac{1}{2}\right)=g\left(\frac{1}{2}\right)=1
	$$

	于是函数 $h(x)$ 的零点共有 9 个, 因此所有零点的和为 9 .
\end{solution}

\begin{note}
	第(1)小题也可以直接由 $0<\frac{a}{3} \leqslant \frac{\pi}{6}$, 及 $\frac{2 \pi}{3} \leqslant 2 a \leqslant \frac{7 \pi}{6}$ 得出结论.第 (2) 小题对于零点的个数常结合图象求之. 本题关键是注意到 $f(x)$ 与 $g(x)$均关于点 $(1,0)$ 对称.
\end{note}



\section{三角函数恒等变换}
\section{三角形中的三角函数}
\section{反三角函数与简单的三角方程}
\section{三角不等式}
\section{三角函数的综合应用}