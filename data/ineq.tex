% This LaTeX document needs to be compiled with XeLaTeX.
%\usepackage[utf8]{inputenc}
%\usepackage{ucharclasses}
%\usepackage{amsmath}
%\usepackage{amsfonts}
%\usepackage{amssymb}
%\usepackage[version=4]{mhchem}
%\usepackage{stmaryrd}
%\usepackage{graphicx}
%\usepackage[export]{adjustbox}
%\graphicspath{ {./images/} }
%\usepackage[fallback]{xeCJK}
%\usepackage{polyglossia}
%\usepackage{fontspec}
%\setCJKmainfont{Noto Serif CJK JP}
%
%\setmainlanguage{english}
%\setotherlanguages{bengali}
%\newfontfamily\bengalifont{Noto Serif Bengali}
%\newfontfamily\lgcfont{CMU Serif}
%\setDefaultTransitions{\lgcfont}{}
%\setTransitionsFor{Bengali}{\bengalifont}{\lgcfont}
%
%\def\Perp{\perp\!\!\!\perp}
%
\chapter{均值不等式与柯西不等式}
\section{平均值不等式及其证明}
平均值不等式是最基本的重要不等式之一, 在不等式理论研究和证明中占有重要的位置. 平均值不等式的证明有许多种方法. 这里, 我们选了部分具有代表意义的证明方法, 其中用来证明平均值不等式的许多结论, 其本身又具有重要的意义. 特别是, 在许多竞赛的书籍中, 都有专门的章节介绍和讨论, 如数学归纳法、变量替换、恒等变形和分析综合方法等, 这些也是证明不等式的常用方法和技巧. 希望大家能认真思考和好好掌握, 熟悉不等式的证明.

\subsection{平均值不等式}
对任意非负实数 $a 、 b$, 有
$$
(\sqrt{a}-\sqrt{b})^{2} \geqslant 0
$$
于是, 得
$$
\frac{a+b}{2} \geqslant \sqrt{a b}
$$

一般地, 假设 $a_{1}, a_{2}, \cdots, a_{n}$ 为 $n$ 个非负实数, 它们的算术平均值记为
$$
A_{n}=\frac{a_{1}+a_{2}+\cdots+a_{n}}{n}
$$
几何平均值记为
$$
G_{n}=\left(a_{1} a_{2} \cdots a_{n}\right)^{\frac{1}{n}}=\sqrt[n]{a_{1} a_{2} \cdots a_{n}}
$$

算术平均值与几何平均值之间有如下的关系
$$
\frac{a_{1}+a_{2}+\cdots+a_{n}}{n} \geqslant \sqrt[n]{a_{1} a_{2} \cdots a_{n}}
$$
即
$$
A_{n} \geqslant G_{n}
$$
当且仅当 $a_{1}=a_{2}=\cdots=a_{n}$ 时, 等号成立.

上述不等式称为平均值不等式,或简称为均值不等式.

平均值不等式的表达形式简单,容易记住,但它的证明和应用非常灵活、广泛, 其证明有多种不同的方法. 为使大家理解和掌握, 这里我们选择了其中的几种典型的证明方法. 当然, 有些方法是几个知识点的结合, 很难将它们归类,有些大体相同或相似,但选择的变量不同,或处理的方式不同,导致证明的难易不同,所以,我们将它们看作是不同的方法.

\subsection{平均值不等式的证明}
\subsubsection*{证法一(归纳法)}

(1) 当 $n=2$ 时,已知结论成立.

(2)假设对 $n=k$ (正整数 $k \geqslant 2$ )时命题成立, 即对于 $a_{i}>0, i=1$, $2, \cdots, k$, 有
$$
\left(a_{1} a_{2} \cdots a_{k}\right)^{\frac{1}{k}} \leqslant \frac{a_{1}+a_{2}+\cdots+a_{k}}{k}
$$
那么, 当 $n=k+1$ 时, 由于
$$
A_{k+1}=\frac{a_{1}+a_{2}+\cdots+a_{k+1}}{k+1}, G_{k+1}=\sqrt[k+1]{a_{1} a_{2} \cdots a_{k} a_{k+1}}
$$
关于 $a_{1}, a_{2}, \cdots, a_{k+1}$ 是对称的, 任意对调 $a_{i}$ 与 $a_{j}(i \neq j)$, 即将 $a_{i}$ 写成 $a_{j}, a_{j}$写成 $a_{i}, A_{k+1}$ 和 $G_{k+1}$ 的值不改变,因此不妨设 $a_{1}=\min \left\{a_{1}, a_{2}, \cdots, a_{k+1}\right\}$, $a_{k+1}=\max \left\{a_{1}, a_{2}, \cdots, a_{k+1}\right\}$, 显然 $a_{1} \leqslant A_{k+1} \leqslant a_{k+1}$, 以及
$$
A_{k+1}\left(a_{1}+a_{k+1}-A_{k+1}\right)-a_{1} a_{k+1}=\left(a_{1}-A_{k+1}\right)\left(A_{k+1}-a_{k+1}\right) \geqslant 0
$$
即
$$
A_{k+1}\left(a_{1}+a_{k+1}-A_{k+1}\right) \geqslant a_{1} a_{k+1}
$$

对 $k$ 个正数 $a_{2}, a_{3}, \cdots, a_{k}, a_{1}+a_{k+1}-A_{k+1}$, 由归纳假设, 得
$$
\frac{a_{2}+a_{3}+\cdots+a_{k}+\left(a_{1}+a_{k+1}-A_{k+1}\right)}{k} \geqslant \sqrt[k]{a_{2} a_{3} \cdots a_{k}\left(a_{1}+a_{k+1}-A_{k+1}\right)}
$$
而
$$
\frac{a_{2}+a_{3}+\cdots+a_{k}+\left(a_{1}+a_{k+1}-A_{k+1}\right)}{k}=\frac{(k+1) A_{k+1}-A_{k+1}}{k}=A_{k+1},
$$
于是
$$
A_{k+1}^{k} \geqslant a_{2} a_{3} \cdots a_{k}\left(a_{1}+a_{k+1}-A_{k+1}\right)
$$

两边乘以 $A_{k+1}$, 得
$$
\begin{aligned}
A_{k+1}^{k+1} & \geqslant a_{2} a_{3} \cdots a_{k} A_{k+1}\left(a_{1}+a_{k+1}-A_{k+1}\right) \\
& \geqslant a_{2} a_{3} \cdots a_{k}\left(a_{1} a_{k+1}\right)=G_{k+1}^{k+1}
\end{aligned}
$$

从而, 有 $A_{k+1} \geqslant G_{k+1}$.

直接验证可知, 当且仅当所有的 $a_{i}$ 相等时等号成立, 故命题成立.

\begin{note}
说明 利用了证明与正整数有关的命题的常用方法,即数学归纳法.数学归纳法证题技巧的应用, 可以说是五彩缤纷, 千姿百态. 应用数学归纳法,除了需要验证当 $n=1$ 或 $n=n_{0}$ (这里 $n_{0}$ 为某个固定的正整数) 外, 其关键是要在 $n=k$ 时成立的假设之下, 导出当 $n=k+1$ 时命题也成立,要完成这一步,需要一定的技巧和处理问题的能力, 只有通过多做练习来实现理解和掌握.
\end{note}

\subsubsection*{证法二(归纳法,与证法一的不同处理)}

(1) 当 $n=2$ 时,已知结论成立.

(2)假设对 $n=k$ (正整数 $k \geqslant 2$ ) 时命题成立, 即对于 $a_{i}>0, i=1$, $2, \cdots, k$, 有
$$
\left(a_{1} a_{2} \cdots a_{k}\right)^{\frac{1}{k}} \leqslant \frac{a_{1}+a_{2}+\cdots+a_{k}}{k}
$$

那么,当 $n=k+1$ 时, 由归纳假设得
\begin{align*}
& a_{1}+a_{2}+\cdots+a_{k}+a_{k+1}  \tag{1}\\
& =a_{1}+a_{2}+\cdots+a_{k}+(a_{k+1}+\overbrace{G_{k+1}+\cdots+G_{k+1}}^{(k-1) \text{个} G_{k+1}})-(k-1) G_{k+1}  \tag{2}\\
& \geqslant k \sqrt[k]{a_{1} a_{2} \cdots a_{k}}+k \sqrt[k]{a_{k+1} G_{k+1}^{k-1}}-(k-1) G_{k+1}  \tag{3}\\
& \geqslant 2 k \sqrt{\sqrt[k]{a_{1} a_{2} \cdots a_{k}} \sqrt[k]{a_{k+1} G_{k+1}^{k-1}}}-(k-1) G_{k+1}  \tag{4}\\
& =2 k \sqrt[2 k]{a_{1} a_{2} \cdots a_{k+1} G_{k+1}^{k-1}}-(k-1) G_{k+1}  \tag{5}\\
& =2 k \sqrt[2 k]{G_{k+1}^{k+1} G_{k+1}^{k-1}}-(k-1) G_{k+1}  \tag{6}\\
& =(k+1) G_{k+1} \tag{7}
\end{align*}

于是 $A_{k+1} \geqslant G_{k+1}$.

不难看出, 当且仅当所有的 $a_{i}$ 相等时等号成立, 故命题成立.
\begin{note}
在这个证明中, 为了利用归纳假设, 将(1)写成(2)的形式. 由归纳假设, 从(2)得到(3),由于当 $n=2$ 时,不等式成立,则由 (3)得到了 (4).
\end{note}

\subsubsection*{证法三(归纳法,另一种处理方式)}

(1) 当 $n=2$ 时,已知结论成立.

(2)假设对 $n=k$ (正整数 $k \geqslant 2$ ) 时命题成立, 即对于 $a_{i}>0, i=1$, $2, \cdots, k$, 有
$$
\left(a_{1} a_{2} \cdots a_{k}\right)^{\frac{1}{k}} \leqslant \frac{a_{1}+a_{2}+\cdots+a_{k}}{k}
$$

那么, 当 $n=k+1$ 时, 由归纳假设得
$$
\begin{aligned}
A_{k+1} & =\frac{1}{2 k}\left[(k+1) A_{k+1}+(k-1) A_{k+1}\right] \\
& =\frac{1}{2 k}(a_{1}+a_{2}+\cdots+a_{k+1}+\underbrace{A_{k+1}+A_{k+1}+\cdots+A_{k+1}}_{\text {共k-1 }}) \\
& \geqslant \frac{1}{2 k}\left(k \sqrt[k]{a_{1} a_{2} \cdots a_{k}}+k \sqrt[k]{a_{k+1} A_{k+1}^{k-1}}\right) \\
& \geqslant \sqrt[2 k]{a_{1} a_{2} \cdots a_{k} a_{k+1} A_{k+1}^{k-1}}
\end{aligned}
$$

所以 $A_{k+1}^{2 k} \geqslant a_{1} a_{2} \cdots a_{k+1} A_{k+1}^{k-1}$, 故得 $A_{k+1} \geqslant G_{k+1}$.

\begin{note}
(1) 在上面的证明中, 将 $A_{k+1}$ 表示为 $A_{k+1}=\frac{1}{2 k}\left[(k+1) A_{k+1}+\right.$ $\left.(k-1) A_{k+1}\right]$ 是一步较为关键和重要的变形技巧.

(2)我们也可以从 $G_{n+1}$ 出发进行处理,由归纳假设,得到
$$
\begin{aligned}
G_{n+1}= & {\left[\left(G_{n+1}\right)^{\frac{n+1}{n}}\left(G_{n+1}\right)^{\frac{n-1}{n}}\right]^{\frac{1}{2}} } \\
= & {\left[\left(a_{1} \cdots a_{n} a_{n+1}\right)^{\frac{1}{n}}\left(G_{n+1}\right)^{\frac{n-1}{n}}\right]^{\frac{1}{2}} } \\
= & G_{n}^{\frac{1}{2}}\left(a_{n+1}^{\frac{1}{n}+1} G_{n+1}^{\frac{n-1}{n}}\right)^{\frac{1}{2}} \leqslant \frac{1}{2}\left(G_{n}+a_{n^{\frac{1}{n}+1}}^{{ }^{\frac{n}{2}} G_{n+1}^{n}}\right) \\
\leqslant & \frac{1}{2}\left(G_{n}+\frac{a_{n+1}+(n-1) G_{n}}{n}\right) \\
\leqslant & \frac{1}{2}\left(A_{n}+\frac{a_{n+1}+(n-1) G_{n}}{n}\right) \\
= & \frac{n A_{n}+a_{n+1}}{2 n}+\frac{(n-1) G_{n+1}}{2 n} \\
= & \frac{(n+1) A_{n+1}}{2 n}+\frac{(n-1) G_{n+1}}{2 n} 
\end{aligned}
$$
即
\begin{align}
  \frac{(n+1) A_{n+1}}{2 n} \geqslant \frac{(n+1) G_{n+1}}{2 n}
\end{align}
故 $A_{n+1} \geqslant G_{n+1}$.
\end{note}

\subsubsection*{证法四(归纳法和变换)}

在证明原命题之前,首先令
$$
y_{1}=\frac{a_{1}}{G_{n}}, y_{2}=\frac{a_{2}}{G_{n}}, \cdots, y_{n}=\frac{a_{n}}{G_{n}}
$$

其中 $G_{n}=\sqrt[n]{a_{1} a_{2} \cdots a_{n}}$, 则 $y_{1} y_{2} \cdots y_{n}=1\left(y_{i}>0\right)$, 且平均值不等式等价于
$$
y_{1}+y_{2}+\cdots+y_{n} \geqslant n
$$

即在条件 $y_{1} y_{2} \cdots y_{n}=1\left(y_{i}>0\right)$ 之下, 证明 $y_{1}+y_{2}+\cdots+y_{n} \geqslant n$.

我们用归纳法证明上述不等式.

(1) 当 $n=1$ 时, $y_{1}=1 \geqslant 1$, 显然成立.

(2)假设当 $n=k$ 时不等式成立,则对于 $n=k+1$, 由于 $y_{1} y_{2} \cdots y_{n}=1$ $\left(y_{i}>0\right)$, 那么 $y_{i}$ 中必有大于或等于 1 者, 也有小于或等于 1 者, 不妨设 $y_{k} \geqslant$ $1, y_{k+1} \leqslant 1$, 并令 $y=y_{k} y_{k+1}$, 则 $y_{1} y_{2} \cdots y_{k-1} y=1$, 从而由归纳假设,得
$$
y_{1}+y_{2}+\cdots+y_{k-1}+y \geqslant k
$$

于是
$$
\begin{aligned}
& y_{1}+y_{2}+\cdots+y_{k-1}+y_{k}+y_{k+1} \\
\geqslant & k+y_{k}+y_{k+1}-y_{k} y_{k+1} \\
= & k+1+\left(y_{k}-1\right)\left(1-y_{k+1}\right) \\
\geqslant & k+1 .
\end{aligned}
$$

不难看出, 当且仅当 $y_{1}=y_{2}=\cdots=y_{n}=1$, 从而 $a_{1}=a_{2}=\cdots=a_{n}$ 时,等号成立.

故当 $n=k+1$ 时, 命题也成立.

说明 通过变量替换,将原问题化为一个与正整数有关的形式简单的不等式, 在证明中运用了我们比较熟悉的手段和技巧.

\subsubsection*{证法五 (归纳法和二项展开式)}

(1) 当 $n=2$ 时,已知结论成立.

(2)假设对 $n=k$ (正整数 $k \geqslant 2$ ) 时命题成立, 即对于 $a_{i}>0, i=1$, $2, \cdots, k$, 有
$$
\left(a_{1} a_{2} \cdots a_{k}\right)^{\frac{1}{k}} \leqslant \frac{a_{1}+a_{2}+\cdots+a_{k}}{k}
$$

那么, 当 $n=k+1$ 时, 不妨假设 $a_{k+1}=\max \left\{a_{1}, a_{2}, \cdots, a_{k+1}\right\}$, 于是由归纳假设, 得
$$
a_{k+1} \geqslant \frac{a_{1}+a_{2}+\cdots+a_{k}}{k}=A_{k} \geqslant G_{k}=\sqrt[k]{a_{1} a_{2} \cdots a_{k}}
$$

从而, 得


\begin{align*}
A_{k+1}^{k+1} & =\left(\frac{a_{1}+a_{2}+\cdots+a_{k}+a_{k+1}}{k+1}\right)^{k+1} \\
& =\left(\frac{k A_{k}+a_{k+1}}{k+1}\right)^{k+1}=\left(A_{k}+\frac{a_{k+1}-A_{k}}{k+1}\right)^{k+1}  \tag{8}\\
& =A_{k}^{k+1}+(k+1) A_{k}^{k}\left(\frac{a_{k+1}-A_{k}}{k+1}\right)+\cdots+\left(\frac{a_{k+1}-A_{k}}{k+1}\right)^{k+1}  \tag{9}\\
& \geqslant A_{k}^{k+1}+(k+1) A_{k}^{k}\left(\frac{a_{k+1}-A_{k}}{k+1}\right)=A_{k}^{k+1}+A_{k}^{k}\left(a_{k+1}-A_{k}\right)  \tag{1}\\
& =A_{k}^{k} a_{k+1} \geqslant G_{k}^{k} a_{k+1}=a_{1} a_{2} \cdots a_{k} a_{k+1}  \tag{1}\\
& =G_{k+1}^{k+1} \tag{12}
\end{align*}


所以 $A_{k+1} \geqslant G_{k+1}$.

不难看出, 当且仅当所有的 $a_{i}$ 相等时等号成立, 故命题成立.

说明 在证明过程中, 考虑 $A_{k+1}^{k+1}$, 并通过一定的处理和运算, 导出所需要的结果. 有时候可能利用到其他的有用结论.

\subsubsection*{证法六 (归纳法和函数)}

(1)当 $n=2$ 时,易知结论成立.

(2)假设 $n=k$ (正整数 $k \geqslant 2$ ) 时命题成立,即 $A_{k} \geqslant G_{k}$. 那么, 当 $n=k+$ 1 时, 作函数 $f(x)=\left(\frac{a_{1}+\cdots+a_{n}+x}{n+1}\right)^{n+1}-a_{1} \cdots a_{n} x, x \in \mathbf{R}$, 并令
$$
f^{\prime}(x)=\left(\frac{a_{1}+\cdots+a_{n}+x}{n+1}\right)^{n}-a_{1} \cdots a_{n}=0
$$

解之得 $x_{0}=-\left(a_{1}+\cdots+a_{n}\right)+(n+1) \sqrt[n]{a_{1} \cdots a_{n}}$
$$
=-n A_{n}+(n+1) G_{n}
$$

不难验证, $x_{0}$ 为 $f(x)$ 的唯一极小值点, 且为最小值点, 以及
$$
f\left(x_{0}\right)=n\left(G_{n}\right)^{n}\left(A_{n}-G_{n}\right)
$$

由归纳假设, 得 $f(x) \geqslant f\left(x_{0}\right) \geqslant 0$, 即
$$
\left(\frac{a_{1}+\cdots+a_{n}+x}{n+1}\right)^{n+1} \geqslant a_{1} \cdots a_{n} x
$$

令 $x=a_{n+1} \geqslant 0$, 则 $A_{n+1}^{n+1} \geqslant G_{n+1}^{n+1}$.\\
故 $A_{n+1} \geqslant G_{n+1}$.

\subsubsection*{证法七(归纳法与 Jacobsthai 不等式)}

为了证明平均值不等式, 需要证明一个引理.

\begin{lemma}\label{lem:证法七引理}
假设 $x 、 y$ 为正实数, $n$ 为正整数,则
$$
x^{n+1}+n y^{n+1} \geqslant(n+1) y^{n} x
$$
\end{lemma}
\begin{proof}
由于 $x 、 y$ 与 $x^{k} 、 y^{k}(1 \leqslant k \leqslant n)$ 同序, 所以
$$
(x-y)\left(x^{k}-y^{k}\right) \geqslant 0
$$
于是
$$
\begin{aligned}
& x^{n+1}+n y^{n+1}-(n+1) x y^{n} \\
= & x\left(x^{n}-y^{n}\right)-n y^{n}(x-y) \\
= & (x-y)\left[x\left(x^{n-1}+x^{n-2} y+\cdots+y^{n-1}\right)-n y^{n}\right] \\
= & (x-y)\left[\left(x^{n}-y^{n}\right)+\left(x^{n-1}-y^{n-1}\right) y+\cdots+(x-y) y^{n-1}\right] \\
\geqslant & 0
\end{aligned}
$$
故引理\ref{lem:证法七引理}成立. 
\end{proof}
现在, 我们利用引理\ref{lem:证法七引理}和数学归纳法证明平均值不等式.

(1)当 $n=2$ 时,已知结论成立.

(2)假设对 $n=k$ (正整数 $k \geqslant 2$ ) 时命题成立. 那么,当 $n=k+1$ 时,令 $a_{1} a_{2} \cdots a_{k}=y^{k(k+1)}, a_{k+1}=x^{k+1}, x, y \geqslant 0$, 则由归纳假设和引理\ref{lem:证法七引理} , 得
$$
\begin{aligned}
& a_{1}+a_{2}+\cdots+a_{k}+a_{k+1}-(k+1) G_{k+1} \\
\geqslant & k \sqrt[k]{a_{1} a_{2} \cdots a_{k}}+a_{k+1}-(k+1) G_{k+1} \\
= & k y^{k+1}+x^{k+1}-(k+1) y^{k} x \geqslant 0 .
\end{aligned}
$$

不难看出,当且仅当所有的 $a_{i}$ 相等时等式成立,故命题成立.

\begin{note}
(1) 引理\ref{lem:证法七引理} 中的不等式称为 Jacobsthai 不等式.

(2) 在 Jacobsthai 不等式中, 取 $y=1$, 得到

\subsubsection*{伯努利不等式} \quad $x^{n} \geqslant 1+n(x-1), x>0, n \geqslant 1$.
\end{note}

关于伯努利(Bernoulli)不等式和平均值不等式, 我们有如下的结论.

\begin{theorem}
  伯努利不等式与平均值不等式等价.
\end{theorem}
事实上, 如果假设伯努利不等式成立, 则对 $\frac{A_{n}}{A_{n-1}}>0$, 有
$$
\left(\frac{A_{n}}{A_{n-1}}\right)^{n} \geqslant 1+n\left(\frac{A_{n}}{A_{n-1}}-1\right)=\frac{n A_{n}-(n-1) A_{n-1}}{A_{n-1}}=\frac{a_{n}}{A_{n-1}}, n \geqslant 2
$$
于是
$$
A_{n}^{n} \geqslant a_{n} A_{n-1}^{n-1}, \quad n \geqslant 2
$$
从而, $A_{n}^{n} \geqslant a_{n} A_{n-1}^{n-1} \geqslant a_{n} a_{n-1} A_{n-2}^{n-2} \geqslant \cdots \geqslant a_{n} a_{n-1} \cdots a_{2} a_{1}=G_{n}^{n}$.故 $A_{n} \geqslant G_{n}$.\\

反之, 如果平均值不等式成立, 则当 $n=1$ 时, 伯努利不等式成立.

当 $n \geqslant 2$ 时, 若 $0<x \leqslant 1-\frac{1}{n}$, 则伯努利不等式成立.

若 $x>1-\frac{1}{n}$, 则 $1+n(x-1)>0$, 由平均值不等式, 得
$$
\begin{aligned}
x^{n} & =(\frac{(1+n(x-1))+\overbrace{1+\cdots+1}^{n-1 \uparrow 1}}{n})^{n} \\
& \geqslant(1+n(x-1)) \cdot 1 \cdot \cdots \cdot 1=1+n(x-1)
\end{aligned}
$$

从而, 伯努利不等式成立.

\begin{note}
伯努利不等式的一般形式:

设 $x>-1$, 则实数 $r \leqslant 0$ 或 $r \geqslant 1$ 时, $(1+x)^{r} \geqslant 1+r x$;

若 $0 \leqslant r \leqslant 1$, 则 $(1+x)^{r} \leqslant 1+r x$;

设 $x_{i} \geqslant-1,1 \leqslant i \leqslant n$, 且 $x_{i}$ 与 $x_{j}$ 同号 $, 1 \leqslant i, j \leqslant n$, 则
$$
\left(1+x_{1}\right)\left(1+x_{2}\right) \cdots\left(1+x_{n}\right) \geqslant 1+x_{1}+\cdots+x_{n}
$$
\end{note}

\subsubsection*{证法八(数列与 Jacobsthai 不等式)}

令 $f(n)=n\left(\frac{a_{1}+\cdots+a_{n}}{n}-\sqrt[n]{a_{1} a_{2} \cdots a_{n}}\right)$, 如果能证明 $f(n)$ 关于 $n$ 单调不减, 即 $f(n) \leqslant f(n+1), n \geqslant 2$. 那么, 由 $f(2) \geqslant 0$, 得到 $f(n) \geqslant f(2) \geqslant$ 0 , 则平均值不等式成立.

下面利用 Jacobsthai 不等式证明 $f(n)$ 的单调性.

令 $a_{1} a_{2} \cdots a_{n}=y^{n(n+1)}, a_{n+1}=x^{n+1}, x, y \geqslant 0$, 则由引理\ref{lem:证法七引理} , 得
$$
\begin{aligned}
& f(n+1)-f(n) \\
= & (n+1)\left(\frac{a_{1}+\cdots+a_{n+1}}{n+1}-\sqrt[n+1]{a_{1} \cdots a_{n+1}}\right) \\
& -n\left(\frac{a_{1}+\cdots+a_{n}}{n}-\sqrt[n]{a_{1} \cdots a_{n}}\right) \\
= & a_{n+1}-(n+1) \sqrt[n+1]{a_{1} \cdots a_{n+1}}+n \sqrt[n]{a_{1} \cdots a_{n}} \\
= & x^{n+1}-(n+1) y^{n} x+n y^{n+1} \geqslant 0 .
\end{aligned}
$$

这表明 $f(n+1) \geqslant f(n)$.

另外, 由于 $f(2) \geqslant 0$, 则对任意 $n \geqslant 2$, 得
$$
f(n) \geqslant f(n-1) \geqslant \cdots \geqslant f(2) \geqslant 0
$$

不难看出, 当且仅当所有的 $a_{i}$ 相等时等号成立, 故平均值不等式成立.

\subsubsection*{证法九(倒向归纳法)}

倒向归纳法, 也称 “留空回填” 法. 基本思想是先对自然数的一个子列 $\left\{n_{m}\right\}$ 证明命题成立, 然后再回过来证明 $\{n\} \backslash\left\{n_{m}\right\}$ 相应的命题成立.

首先证明当 $n=2^{m}$ ( $m$ 为正整数) 时, 平均值不等式成立. 为此, 对 $m$ 用数学归纳法.

当 $m=1$ 时, 显然有 $\sqrt{a_{1} a_{2}} \leqslant \frac{a_{1}+a_{2}}{2}$.

假设 $m=k$ 时命题成立, 则当 $m=k+1$ 时,
$$
\begin{aligned}
& \sqrt[2^{k+1}]{a_{1} a_{2} \cdots a_{2^{k}} a_{2^{k}+1} \cdots a_{2^{k+1}}}
\end{aligned}
$$

\begin{center}
此处有图片 % \includegraphics[max width=\textwidth]{2024_05_22_4ff05a14ba9ad07b725fg-010}
\end{center}
$$
\begin{aligned}
& \leqslant \frac{1}{2}\left(\sqrt[2^{k}]{a_{1} a_{2} \cdots a_{2^{k}}}+\sqrt[2^{k}]{a_{2^{k}+1} \cdots a_{2^{k+1}}}\right) \\
& \leqslant \frac{1}{2}\left(\frac{a_{1}+a_{2}+\cdots+a_{2^{k}}}{2^{k}}+\frac{a_{2^{k}+1}+\cdots+a_{2^{k+1}}}{2^{k}}\right) \\
& =\frac{a_{1}+a_{2}+\cdots+a_{2^{k}}+a_{2^{k}+1}+\cdots+a_{2^{k+1}}}{2^{k+1}}
\end{aligned}
$$

所以对于具有 $n=2^{m}$ 形式的正整数 $n$, 平均值不等式成立, 即对无穷多个正整数 $2,4,8, \cdots, 2^{m}, \cdots$, 平均值不等式成立.

现假设 $n=k+1$ 时, 平均值不等式成立.

当 $n=k$ 时, $A_{k}=\frac{a_{1}+a_{2}+\cdots+a_{k}}{k}$, 则由假设, 得
$$
\sqrt[k+1]{a_{1} a_{2} \cdots a_{k} A_{k}} \leqslant \frac{a_{1}+a_{2}+\cdots+a_{k}+A_{k}}{k+1}=\frac{k A_{k}+A_{k}}{k+1}=A_{k},
$$

所以 $G_{k} \leqslant A_{k}$, 也就是说当 $n=k$ 时命题也成立.

综上可知, 对一切正整数 $n$, 平均值不等式成立. 不难看出, 当且仅当所有的 $a_{i}$ 相等时等号成立,故命题成立.

注 由上述证明知, 对任意整数 $n \geqslant 1$, 有
$$
\frac{a_{1}+a_{2}+\cdots+a_{2^{n}}}{2^{n}} \geqslant \sqrt[2^{n}]{a_{1} a_{2} \cdots a_{2^{n}}}
$$

如果取 $a_{1}, a_{2}, \cdots, a_{n}, a_{n+1}=a_{n+2}=\cdots=a_{2^{n}}=A_{n}$, 则
$$
\begin{aligned}
A_{n} & =\frac{n A_{n}+\left(2^{n}-n\right) A_{n}}{2^{n}} \geqslant \sqrt[2^{n}]{a_{1} a_{2} \cdots a_{n} A_{n}^{2^{n}-n}} \\
& =\left(a_{1} a_{2} \cdots a_{n}\right)^{\frac{1}{2^{n}}} \cdot A_{n}^{1-\frac{n}{2^{n}}}
\end{aligned}
$$

从而 $A_{n} \geqslant G_{n}$. 故平均值不等式成立.

\subsubsection*{证法十(利用排序不等式)}

为了利用与上面不同的方法证明平均值不等式, 我们首先介绍和证明另一个重要的结论, 即排序不等式.
\begin{lemma}[排序不等式]\label{lem:排序不等式}
$\quad$ 设两个实数组 $a_{1}, a_{2}, \cdots, a_{n}$ 和 $b_{1}, b_{2}, \cdots, b_{n}$, 满足
$$
a_{1} \leqslant a_{2} \leqslant \cdots \leqslant a_{n} ; b_{1} \leqslant b_{2} \leqslant \cdots \leqslant b_{n}
$$
则
$$
a_{1} b_{1}+a_{2} b_{2}+\cdots+a_{n} b_{n}(\text { 同序乘积之和 })
$$
$$
\begin{aligned}
& \geqslant a_{1} b_{j_{1}}+a_{2} b_{j_{2}}+\cdots+a_{n} b_{j_{n}} \text { (乱序乘积之和) } \\
& \geqslant a_{1} b_{n}+a_{2} b_{n-1}+\cdots+a_{n} b_{1} \text { (反序乘积之和) }
\end{aligned}
$$

其中 $j_{1}, j_{2}, \cdots, j_{n}$ 是 $1,2, \cdots, n$ 的一个排列, 并且等号同时成立的充分必要条件是 $a_{1}=a_{2}=\cdots=a_{n}$ 或 $b_{1}=b_{2}=\cdots=b_{n}$ 成立.
\end{lemma}
\begin{proof}
令 $A=a_{1} b_{j_{1}}+a_{2} b_{j_{2}}+\cdots+a_{n} b_{j_{n}}$. 如果 $j_{n} \neq n$, 且假设此时 $b_{n}$ 所在的项是 $a_{j_{m}} b_{n}$, 则由 $\left(b_{n}-b_{j_{n}}\right)\left(a_{n}-a_{j_{m}}\right) \geqslant 0$, 得
$$
a_{n} b_{n}+a_{j_{m}} b_{j_{n}} \geqslant a_{j_{m}} b_{n}+a_{n} b_{j_{n}}
$$
也就是说, $j_{n} \neq n$ 时, 调换 $A$ 中 $b_{n}$ 与 $b_{j_{n}}$ 的位置, 其余都不动, 则得到 $a_{n} b_{n}$ 项,并使 $A$ 变为 $A_{1}$, 且 $A_{1} \geqslant A$. 用同样的方法, 可以再得到 $a_{n-1} b_{n-1}$ 项, 并使 $A_{1}$变为 $A_{2}$, 且 $A_{2} \geqslant A_{1}$.

继续这个过程, 至多经过 $n-1$ 次调换, 得 $a_{1} b_{1}+a_{2} b_{2}+\cdots+a_{n} b_{n}$, 故
$$
a_{1} b_{1}+a_{2} b_{2}+\cdots+a_{n} b_{n} \geqslant A
$$

同样可以证明 $A \geqslant a_{1} b_{n}+a_{2} b_{n-1}+\cdots+a_{n} b_{1}$.

显然当 $a_{1}=a_{2}=\cdots=a_{n}$ 或 $b_{1}=b_{2}=\cdots=b_{n}$ 时,两个等号同时成立.反之, 如果 $\left\{a_{1}, a_{2}, \cdots, a_{n}\right\}$ 及 $\left\{b_{1}, b_{2}, \cdots, b_{n}\right\}$ 中的数都不全相同时, 则必有 $a_{1} \neq a_{n}, b_{1} \neq b_{n}$. 于是 $a_{1} b_{1}+a_{n} b_{n}>a_{1} b_{n}+a_{n} b_{1}$, 且
$$
a_{2} b_{2}+\cdots+a_{n-1} b_{n-1} \geqslant a_{2} b_{n-1}+\cdots+a_{n-1} b_{2}
$$
从而有
$$
a_{1} b_{n}+a_{2} b_{2}+\cdots+a_{n} b_{n}>a_{1} b_{n}+a_{2} b_{n-1}+\cdots+a_{n} b_{1}
$$
故这两个等式中至少有一个不成立.
\end{proof}
现在,利用引理\ref{lem:排序不等式} 证明平均值不等式.

令 $y_{k}=\frac{a_{1} a_{2} \cdots a_{k}}{G_{n}^{k}}, k=1,2, \cdots, n$. 由排序不等式, 得
$$
\begin{aligned}
& y_{1} \times \frac{1}{y_{1}}+y_{2} \times \frac{1}{y_{2}}+\cdots+y_{n} \times \frac{1}{y_{n}} \\
\leqslant & y_{1} \times \frac{1}{y_{n}}+y_{2} \times \frac{1}{y_{1}}+\cdots+y_{n} \times \frac{1}{y_{n-1}} \\
= & \frac{a_{1}+a_{2}+\cdots+a_{n}}{G_{n}},
\end{aligned}
$$

所以 $A_{n} \geqslant G_{n}$.

显然当 $a_{1}=a_{2}=\cdots=a_{n}$ 时, $A_{n}=G_{n}$. 如果 $a_{1}, a_{2}, \cdots, a_{n}$ 不全相等,不妨设 $a_{1} \neq a_{2}$, 令 $b=\frac{a_{1}+a_{2}}{2}$, 则 $a_{1} a_{2}<b^{2}$, 且 $b+b=a_{1}+a_{2}$,
$$
G_{n}<\sqrt[n]{b \cdot b \cdot a_{3} \cdots a_{n}} \leqslant \frac{b+b+a_{3}+\cdots+a_{n}}{n}=A_{n}
$$

故当 $A_{n}=G_{n}$ 时必有 $a_{1}=a_{2}=\cdots=a_{n}$. 反之亦然.

注 (1) 我们可以类似于证法四, 由 $G_{n}=\sqrt[n]{a_{1} a_{2} \cdots a_{n}}$, 令
$$
y_{1}=\frac{a_{1}}{G_{n}}, y_{2}=\frac{a_{2}}{G_{n}}, \cdots, y_{n}=\frac{a_{n}}{G_{n}}
$$

则 $y_{1} y_{2} \cdots y_{n}=1\left(y_{i}>0\right)$, 且平均值不等式等价于
$$
y_{1}+y_{2}+\cdots+y_{n} \geqslant n
$$

下面利用排序不等式证明这个不等式.

任取 $x_{1}>0$, 再取 $x_{2}>0$, 使得 $y_{1}=\frac{x_{1}}{x_{2}}$, 再取 $x_{3}>0$, 使得 $y_{2}=\frac{x_{2}}{x_{3}}, \cdots$,最后取 $x_{n}>0$, 使得 $y_{n-1}=\frac{x_{n-1}}{x_{n}}$. 所以
$$
y_{n}=\frac{1}{y_{1} y_{2} \cdots y_{n-1}}=\frac{1}{\frac{x_{1}}{x_{2}} \frac{x_{2}}{x_{3}} \cdots \frac{x_{n-1}}{x_{n}}}=\frac{x_{n}}{x_{1}}
$$

由引理\ref{lem:排序不等式} , 得
$$
y_{1}+y_{2}+\cdots+y_{n}=\frac{x_{1}}{x_{2}}+\frac{x_{2}}{x_{3}}+\cdots+\frac{x_{n-1}}{x_{n}}+\frac{x_{n}}{x_{1}} \geqslant n
$$

当且仅当 $x_{1}=x_{2}=\cdots=x_{n}$ 时等号成立, 从而当且仅当 $y_{1}=y_{2}=\cdots=y_{n}$时等号成立,所以当且仅当 $a_{1}=a_{2}=\cdots=a_{n}$ 时等号成立.

(2)排序不等式是一个重要的基本的不等式, 可以利用排序不等式直接证明许多其他有关的不等式. 例如:

\textbf{切比雪夫(Chebyshev)不等式} \quad 设 $a_{1}, a_{2}, \cdots, a_{n}, b_{1}, b_{2}, \cdots, b_{n}$ 满足 $a_{1}$ $\leqslant a_{2} \leqslant \cdots \leqslant a_{n}, b_{1} \leqslant b_{2} \leqslant \cdots \leqslant b_{n}$, 则
$$
n \sum_{k=1}^{n} a_{k} b_{n-k+1} \leqslant \sum_{k=1}^{n} a_{k} \sum_{k=1}^{n} b_{k} \leqslant n \sum_{k=1}^{n} a_{k} b_{k}
$$

当且仅当 $a_{1}=a_{2}=\cdots=a_{n}$ 或 $b_{1}=b_{2}=\cdots=b_{n}$ 时等号成立.

\begin{proof}
显然
$$
\begin{aligned}
& n \sum_{k=1}^{n} a_{k} b_{k}-\sum_{k=1}^{n} a_{k} \sum_{k=1}^{n} b_{k} \\
= & \sum_{k=1}^{n} \sum_{j=1}^{n}\left(a_{k} b_{k}-a_{k} b_{j}\right)=\sum_{j=1}^{n} \sum_{k=1}^{n}\left(a_{j} b_{j}-a_{j} b_{k}\right) \\
= & \frac{1}{2} \sum_{k=1}^{n} \sum_{j=1}^{n}\left(a_{k} b_{k}+a_{j} b_{j}-a_{k} b_{j}-a_{j} b_{k}\right) \\
= & \frac{1}{2} \sum_{k=1}^{n} \sum_{j=1}^{n}\left(a_{k}-a_{j}\right)\left(b_{k}-b_{j}\right) \geqslant 0
\end{aligned}
$$
故命题成立.
\end{proof}

\subsubsection*{证法十一(调整法)}

(1)首先,如果 $a_{1}=a_{2}=\cdots=a_{n}$, 那么必有 $A_{n}=G_{n}$. 下设这些数不全等,不妨设 $a_{1}=\min \left\{a_{1}, a_{2}, \cdots, a_{n}\right\}, a_{2}=\max \left\{a_{1}, a_{2}, \cdots, a_{n}\right\}$, 则 $a_{1}<$ $A_{n}<a_{2}, a_{1}<G_{n}<a_{2}$. 令 $b_{1}=A_{n}, b_{2}=a_{1}+a_{2}-A_{n}, b_{i}=a_{i}, i \geqslant 3$. 并记 $A_{n}^{1}=\frac{b_{1}+b_{2}+\cdots+b_{n}}{n}=\frac{a_{1}+a_{2}+\cdots+a_{n}}{n}$, 则 $A_{n}^{1}=A_{n}$, 且由于
$$
\begin{aligned}
b_{1} b_{2}-a_{1} a_{2} & =A_{n}\left(a_{1}+a_{2}-A_{n}\right)-a_{1} a_{2} \\
& =\left(A_{n}-a_{1}\right)\left(a_{2}-A_{n}\right)>0
\end{aligned}
$$

则 $G_{n} \leqslant G_{n}^{1}=\sqrt[n]{b_{1} b_{2} \cdots b_{n}}$.

(2) 如果 $b_{1}=b_{2}=\cdots=b_{n}$, 则命题成立. 若不全等,则必有最大和最小者, 而且它们都不等于 $A_{n}$, 仿照上面作法, 可以得到 $c_{1}, c_{2}, \cdots, c_{n}$, 这组数中,有两个数为 $A_{n}$, 且 $A_{n}^{2}=\frac{c_{1}+c_{2}+\cdots+c_{n}}{n}=\frac{b_{1}+b_{2}+\cdots+b_{n}}{n}=A_{n}^{1}=A_{n}$, $G_{n}^{2}=\sqrt[n]{c_{1} c_{2} \cdots c_{n}} \geqslant G_{n}^{1} \geqslant G_{n}$. 如果 $c_{1}=c_{2}=\cdots=c_{n}$, 那么 $A_{n}^{2}=G_{n}^{2}$, 从而 $A_{n}=A_{n}^{2} \geqslant G_{n}$. 如果 $c_{1}, c_{2}, \cdots, c_{n}$ 仍然不全相等, 再按上述方法, 进行第三次变换, 所得到的新的数组中必有 3 个数都为 $A_{n}$. 这样下去, 一定存在某个数 $m(2 \leqslant m \leqslant n)$ 使得
$$
A_{n}=A_{n}^{1}=\cdots=A_{n}^{m}, G_{n} \leqslant G_{n}^{1} \leqslant G_{n}^{2} \leqslant \cdots \leqslant G_{n}^{m}, A_{n}^{m}=G_{n}^{m},
$$

从而得 $A_{n} \geqslant G_{n}$, 且只要 $a_{1}, a_{2}, \cdots, a_{n}$ 不全相等, 必有 $A_{n}>G_{n}$. 故命题成立.

注 调整法是证明不等式或求最值的一种有效方法, 特别是对那些当变量相等时取等号或取到最值的有关问题.

\subsubsection*{证法十二 (利用辅助命题)}

为了证明平均值不等式, 首先证明另一个不等式, 即

\begin{lemma}\label{lem:证法十二引理}
如果 $x_{k} \geqslant 0$, 且 $x_{k} \geqslant x_{k-1}(k=2,3, \cdots, n)$, 则
$$
x_{n}^{n} \geqslant x_{1}\left(2 x_{2}-x_{1}\right)\left(3 x_{3}-2 x_{2}\right) \cdots\left[n x_{n}-(n-1) x_{n-1}\right]
$$
当且仅当 $x_{1}=x_{2}=\cdots=x_{n}$ 时等号成立.
\end{lemma}
\begin{proof}
因为 $x_{k} \geqslant x_{k-1}$, 则
$$
x_{k}^{k-1}+x_{k}^{k-2} x_{k-1}+\cdots+x_{k-1}^{k-1} \geqslant k x_{k-1}^{k-1},
$$
所以
$$
\begin{aligned}
x_{k}^{k}-x_{k-1}^{k} & =\left(x_{k}-x_{k-1}\right)\left(x_{k}^{k-1}+x_{k}^{k-2} x_{k-1}+\cdots+x_{k-1}^{k-1}\right) \\
& \geqslant k x_{k-1}^{k-1}\left(x_{k}-x_{k-1}\right)
\end{aligned}
$$
即
$$
x_{k}^{k} \geqslant x_{k-1}^{k-1}\left[k x_{k}-(k-1) x_{k-1}\right](k=1,2, \cdots, n)
$$
当且仅当 $x_{k}=x_{k-1}$ 时等号成立.

所以
$$
x_{n}^{n}=x_{1} \frac{x_{2}^{2}}{x_{1}} \frac{x_{3}^{3}}{x_{2}^{2}} \cdots \frac{x_{n}^{n}}{x_{n-1}^{n-1}} \geqslant x_{1}\left(2 x_{2}-x_{1}\right)\left(3 x_{3}-2 x_{2}\right) \cdots\left[n x_{n}-(n-1) x_{n-1}\right]
$$
\end{proof}

现在利用引理\ref{lem:证法十二引理}证明平均值不等式.

不妨假设 $a_{n} \geqslant a_{n-1} \geqslant \cdots \geqslant a_{2} \geqslant a_{1}>0$. 由 $A_{k}=\frac{a_{1}+a_{2}+\cdots+a_{k}}{k}$, 则 $A_{k} \geqslant A_{k-1}>0(k=2,3, \cdots, n)$, 且 $k A_{k}-(k-1) A_{k-1}=a_{k}$. 由引理 3 , 得
$$
A_{n}^{n} \geqslant a_{1} a_{2} \cdots a_{n}
$$

即 $A_{n} \geqslant G_{n}$. 当且仅当 $A_{1}=A_{2}=\cdots=A_{n}$, 即 $a_{1}=a_{2}=\cdots=a_{n}$ 时等号成立.

\subsubsection*{证法十三 (函数方法)}
\begin{lemma}\label{lem:证法十三引理}
如果函数 $f(x):(a, b) \rightarrow \mathbf{R}$ 满足
\begin{equation*}
f\left(\frac{x+y}{2}\right) \geqslant \frac{f(x)+f(y)}{2}, x, y \in(a, b) \tag{13}
\end{equation*}
那么
\begin{equation*}
f\left(\frac{x_{1}+x_{2}+\cdots+x_{n}}{n}\right) \geqslant \frac{f\left(x_{1}\right)+f\left(x_{2}\right)+\cdots+f\left(x_{n}\right)}{n} \tag{14}
\end{equation*}
其中 $x_{i} \in(a, b)$.
\end{lemma}
\begin{proof}
对 $n$ 用归纳法.

当 $n=1,2$ 时,结论显然成立.

设当 $n=k$ 时结论成立. 对于 $n=k+1$, 有

并记
$$
\begin{gathered}
A_{k+1}=\frac{a_{1}+a_{2}+\cdots+a_{k}}{2 k}+\frac{a_{k+1}+(k-1) A_{k+1}}{2 k} \\
B=\frac{a_{k+1}+(k-1) A_{k+1}}{k}
\end{gathered}
$$
则
$$
\begin{aligned}
f\left(A_{k+1}\right)= & f\left(\frac{A_{k}+B}{2}\right) \\
\geqslant & \frac{1}{2}\left[f\left(A_{k}\right)+f(B)\right] \\
\geqslant & \frac{1}{2}\left\{\frac{1}{k}\left[f\left(a_{1}\right)+f\left(a_{2}\right)+\cdots+f\left(a_{k}\right)\right]\right. \\
& \left.+\frac{1}{k}\left[f\left(a_{k+1}\right)+(k-1) f\left(A_{k+1}\right)\right]\right\}
\end{aligned}
$$
所以
$$
f\left(\frac{a_{1}+a_{2}+\cdots+a_{k+1}}{k+1}\right) \geqslant \frac{f\left(a_{1}\right)+f\left(a_{2}\right)+\cdots+f\left(a_{k+1}\right)}{k+1}
$$

我们称满足(13)式的函数为凹函数 (可以证明, 如果函数 $f$ 二阶可导, 则当 $f^{\prime \prime}(x) \leqslant 0$ 时, $f$ 为凹函数). 特别的, 不难验证函数 $f(x)=\ln x$ 在 $(0$, $+\infty)$ 上是凹函数, 于是, 对 $a_{i} \in(0,+\infty), i=1,2, \cdots, n$, 我们有
$$
f\left(\frac{a_{1}+a_{2}+\cdots+a_{n}}{n}\right) \geqslant \frac{f\left(a_{1}\right)+f\left(a_{2}\right)+\cdots+f\left(a_{n}\right)}{n}
$$

从而
$$
\ln \frac{a_{1}+a_{2}+\cdots+a_{n}}{n} \geqslant \ln \left(a_{1} a_{2} \cdots a_{n}\right)^{\frac{1}{n}}
$$

由对数函数的单调性, 得
$$
\frac{a_{1}+a_{2}+\cdots+a_{n}}{n} \geqslant\left(a_{1} a_{2} \cdots a_{n}\right)^{\frac{1}{n}}
$$

故命题成立.
\end{proof}

下面验证 $\ln x$ 为凹函数. 对任意 $x, y$, 要使得:
$$
f\left(\frac{x+y}{2}\right) \geqslant \frac{f(x)+f(y)}{2}
$$
即
$$
\ln \frac{x+y}{2} \geqslant \frac{\ln (x)+\ln (y)}{2}
$$
等价于
$$
\ln \frac{x+y}{2} \geqslant \ln (x y)^{\frac{1}{2}}
$$
由函数的单调性, 等价于
$$
\frac{x+y}{2} \geqslant(x y)^{\frac{1}{2}}
$$
这个可以由 $(\sqrt{x}-\sqrt{y})^{2}>0$ 直接导出.

另外, 设 $p>0, q>0$, 且 $\frac{1}{p}+\frac{1}{q}=1$, 由于函数 $f(x)=\ln x, x \in \mathbf{R}_{+}$为凹函数, 则对 $x, y>0$, 有

即
$$
\begin{gathered}
\frac{1}{p} \ln x+\frac{1}{q} \ln y \leqslant \ln \left(\frac{1}{p} x+\frac{1}{q} y\right) \\
x^{\frac{1}{p}} y^{\frac{1}{q}} \leqslant \frac{1}{p} x+\frac{1}{q} y
\end{gathered}
$$
等号成立的充分必要条件是 $x=y$.

这个不等式称为 Young 不等式.

\begin{note}
引理\ref{lem:证法十三引理}中的不等式(44, 称为琴生 (Jensen)不等式, 它的一般形式为设 $y=f(x), x \in(a, b)$ 为凹函数, 则对任意 $x_{i} \in(a, b)(i=1,2, \cdots$, $n)$, 我们有加权的琴生不等式
$$
\frac{1}{p_{1}} f\left(x_{1}\right)+\frac{1}{p_{2}} f\left(x_{2}\right)+\cdots+\frac{1}{p_{n}} f\left(x_{n}\right) \leqslant f\left(\frac{x_{1}}{p_{1}}+\frac{x_{2}}{p_{2}}+\cdots+\frac{x_{n}}{p_{n}}\right),
$$
其中 $p_{i}>0(i=1,2, \cdots, n)$ 且 $\sum_{i=1}^{n} \frac{1}{p_{i}}=1$.
\end{note}


\subsubsection*{证法十四(平均值不等式与函数不等式)
}
利用函数 $f(x)=\mathrm{e}^{x}-1-x$ 的性质,不难得到

\begin{lemma}\label{lem:证法十四引理}
$5 \mathrm{e}^{x} \geqslant 1+x, x \in \mathbf{R}$, 当且仅当 $x=0$ 时, 等号成立.

设 $a_{1}, a_{2}, \cdots, a_{n} \geqslant 0$. 令 $a_{k}=\left(1+x_{k}\right) A_{n}$, 其中 $\sum_{i=1}^{n} x_{i}=0$.
\end{lemma}

由引理\ref{lem:证法十四引理}$\mathrm{e}^{x_{k}} \geqslant 1+x_{k}$, 于是
$$
\begin{aligned}
G_{n} & =\sqrt[n]{a_{1} a_{2} \cdots a_{n}}=\left(\prod_{k=1}^{n}\left(1+x_{k}\right) A_{n}\right)^{\frac{1}{n}} \\
& =A_{n}\left(\prod_{k=1}^{n}\left(1+x_{k}\right)\right)^{\frac{1}{n}} \leqslant A_{n}\left(\prod_{k=1}^{n} \mathrm{e}^{x_{k}}\right)^{\frac{1}{n}} \\
& =A_{n} \mathrm{e}^{\frac{1}{n} \sum_{k=1}^{n} x_{k}}=A_{n}
\end{aligned}
$$

从而 $A_{n} \geqslant G_{n}$, 故平均值不等式成立.

\subsubsection*{证法十五(几何方法)}

作函数 $y=\mathrm{e}^{\frac{x}{G_{n}}}$ 的图象, 并过点 $\left(G_{n}, \mathrm{e}\right)$ 作该曲线的切线 $y=\frac{\mathrm{e}}{G_{n}} x$. 易知 $\mathrm{e}^{\frac{x}{G_{n}}} \geqslant \frac{\mathrm{e}}{G_{n}} x, x \geqslant 0$

当且仅当 $x=G_{n}$ 时, 等号成立.

对 $a_{i} \geqslant 0,1 \leqslant i \leqslant n$, 有 $\mathrm{e}^{\frac{a_{i}}{G_{n}}} \geqslant \frac{\mathrm{e}}{G_{n}} a_{i}, 1 \leqslant i \leqslant n$, 将这 $n$ 个不等式相乘,得 $\mathrm{e}^{\frac{a_{1}+\cdots+a_{n}}{G_{n}}} \geqslant \frac{\mathrm{e}^{n}}{G_{n}^{n}} a_{1} a_{2} \cdots a_{n}$, 即 $\mathrm{e}^{\frac{a_{1}+\cdots+a_{n}}{G_{n}}} \geqslant \mathrm{e}^{n}$.

由函数 $y=\mathrm{e}^{x}$ 的单调性, 得 $\frac{a_{1}+\cdots+a_{n}}{G_{n}} \geqslant n$.

从而 $A_{n} \geqslant G_{n}$, 且当且仅当 $a_{1}=\cdots=a_{n}$ 时等号成立. 故平均值不等式成立.

在这部分, 我们利用不同的方法证明了平均值不等式成立. 在证明过程中,利用了各种技巧和方法.
\begin{comment}

  \section*{习 题 1}
$\square$ 设 $a, b, c>0, a b c=1$. 求证:
$$
\frac{1}{a}+\frac{1}{b}+\frac{1}{c} \geqslant \sqrt{a}+\sqrt{b}+\sqrt{c}
$$

22 设 $a, b, c \geqslant 0, a+b+c>0$, 求证: $\frac{(a+b)^{3}(b+c)^{2}(c+a)}{(a+b+c)^{6}} \leqslant \frac{4}{27}$, 并指\\
出等号成立的条件.

3 已知 $0<a, b, c<1$, 并且 $a b+b c+c a=1$. 证明:
$$
\frac{a}{1-a^{2}}+\frac{b}{1-b^{2}}+\frac{c}{1-c^{2}} \geqslant \frac{3 \sqrt{3}}{2}
$$

4设 $a, b, c, d \in \mathbf{R}_{+}$满足 $a b c d=1, a+b+c+d>\frac{a}{b}+\frac{b}{c}+\frac{c}{d}+\frac{d}{a}$.

求证: $a+b+c+d<\frac{b}{a}+\frac{c}{b}+\frac{d}{c}+\frac{a}{d}$.

5 设 $a_{i}>0,1 \leqslant i \leqslant n$. 求证: $\sum_{i=1}^{n} \frac{a_{i}^{3}+a_{i+1}^{3}}{a_{i}^{2}+a_{i} a_{i+1}+a_{i+1}^{2}} \geqslant \frac{2}{3} \sum_{i=1}^{n} a_{i}$, 其中, $a_{n+1}=$ $a_{1}$.

6 设 $a_{1}, a_{2}, \cdots, a_{n}>0$ 且 $a_{1}+a_{2}+\cdots+a_{n}=1$. 求证:
$$
\left(\frac{1}{a_{1}^{2}}-1\right)\left(\frac{1}{a_{2}^{2}}-1\right) \cdots\left(\frac{1}{a_{n}^{2}}-1\right) \geqslant\left(n^{2}-1\right)^{n}
$$

7 设 $a, b, c, d>0$, 满足 $a^{2}+b^{2}+c^{2}+d^{2}=1$. 求证:
$$
a+b+c+d+\frac{1}{a b c d} \geqslant 18
$$

8 设 $x_{i} \geqslant 0,1 \leqslant i \leqslant n, n \geqslant 3$ 满足 $x_{1}+\cdots+x_{n}=1$, 求证:
$$
\frac{3}{4} \leqslant \frac{x_{1}^{2}+x_{2}+\cdots+x_{n-1}+x_{n}}{x_{1}+x_{2}+\cdots+x_{n-1}+x_{n}^{2}} \leqslant \frac{4}{3}
$$

9 设 $x_{1}, x_{2}, \cdots, x_{n}>0$, 且 $x_{1} x_{2} \cdots x_{n}=1$. 求证:
$$
\frac{1}{x_{1}\left(1+x_{1}\right)}+\frac{1}{x_{2}\left(1+x_{2}\right)}+\cdots+\frac{1}{x_{n}\left(1+x_{n}\right)} \geqslant \frac{n}{2}
$$

10 设 $a, b, c>0$ 且 $a^{2}+b^{2}+c^{2}+(a+b+c)^{2} \leqslant 4$. 求证:
$$
\frac{a b+1}{(a+b)^{2}}+\frac{b c+1}{(b+c)^{2}}+\frac{c a+1}{(c+a)^{2}} \geqslant 3
$$

11 已知 $a 、 b 、 c$ 为正实数, 且 $a b c=8$, 求证:
$$
\frac{a^{2}}{\sqrt{\left(1+a^{3}\right)\left(1+b^{3}\right)}}+\frac{b^{2}}{\sqrt{\left(1+b^{3}\right)\left(1+c^{3}\right)}}+\frac{c^{2}}{\sqrt{\left(1+c^{3}\right)\left(1+a^{3}\right)}} \geqslant \frac{4}{3}
$$

12 设 $a, b \in \mathbf{R}, \frac{1}{a}+\frac{1}{b}=1$. 求证: 对一切正整数 $n$, 有
$$
(a+b)^{n}-a^{n}-b^{n} \geqslant 2^{2 n}-2^{n+1}
$$

13 设 $a, b \in \mathbf{R}_{+}$, 求证: $\sqrt{a}+1>\sqrt{b}$ 成立的充要条件是对任意 $x>1$, 有 $a x+$ $\frac{x}{x-1}>b$.

14 设 $x_{1}, x_{2} \in \mathbf{R}$, 且 $x_{1}^{2}+x_{2}^{2} \leqslant 1$. 求证:对任意 $y_{1}, y_{2} \in \mathbf{R}$, 有
$$
\left(x_{1} y_{1}+x_{2} y_{2}-1\right)^{2} \geqslant\left(x_{1}^{2}+x_{2}^{2}-1\right)\left(y_{1}^{2}+y_{2}^{2}-1\right)
$$

15 设 $a 、 b 、 c$ 为正实数, 求证:
$$
\left(1+\frac{a}{b}\right)\left(1+\frac{b}{c}\right)\left(1+\frac{c}{a}\right) \geqslant 2\left(1+\frac{a+b+c}{\sqrt[3]{a b c}}\right)
$$

16 设 $x_{1}, x_{2}, x_{3} \in \mathbf{R}_{+}$, 证明:
$$
\frac{x_{2}}{x_{1}}+\frac{x_{3}}{x_{2}}+\frac{x_{1}}{x_{3}} \leqslant\left(\frac{x_{1}}{x_{2}}\right)^{2}+\left(\frac{x_{2}}{x_{3}}\right)^{2}+\left(\frac{x_{3}}{x_{1}}\right)^{2}
$$

17 设 $a 、 b 、 c$ 为正实数,且 $a+b+c=1$. 求证:
$$
(1+a)(1+b)(1+c) \geqslant 8(1-a)(1-b)(1-c)
$$

18 设 $x 、 y 、 z$ 为正实数,且 $x \geqslant y \geqslant z$. 求证:
$$
\frac{x^{2} y}{z}+\frac{y^{2} z}{x}+\frac{z^{2} x}{y} \geqslant x^{2}+y^{2}+z^{2}
$$

19 设 $a 、 b 、 c$ 为正实数,满足 $a^{2}+b^{2}+c^{2}=1$. 求证:
$$
\frac{a b}{c}+\frac{b c}{a}+\frac{c a}{b} \geqslant \sqrt{3}
$$

20 设 $a 、 b 、 c 、 d$ 是非负实数,满足 $a b+b c+c d+d a=1$. 求证:
$$
\frac{a^{3}}{b+c+d}+\frac{b^{3}}{a+c+d}+\frac{c^{3}}{a+d+b}+\frac{d^{3}}{a+b+c} \geqslant \frac{1}{3}
$$

21 设 $n$ 为给定的自然数, $n \geqslant 3$, 对于 $n$ 个给定的实数 $a_{1}, a_{2}, \cdots, a_{n}$, 记 $\left|a_{i}-a_{j}\right|(1 \leqslant i<j \leqslant n)$ 的最小值为 $m$, 求在 $a_{1}^{2}+\cdots+a_{n}^{2}=1$ 时, $m$ 的最大值.

22 设 $x, y \in \mathbf{R}_{+}, x+y^{2016}>1$, 求证:
$$
x^{2016}+y>1-\frac{1}{100}
$$

23 设 $x_{i}>0,1 \leqslant i \leqslant n, x_{1} x_{2} \cdots x_{n}=1$. 求证:
$$
\prod_{i=1}^{n}\left(\sqrt{2}+x_{i}\right) \geqslant(1+\sqrt{2})^{n}
$$

24 设 $a, b, c, d>0$. 求证:
$$
a^{4}+b^{4}+c^{4}+d^{4} \geqslant 4 a b c d+4(a-b)^{2} \sqrt{a b c d}
$$

25 设 $a, b, c, d>0$ 满足 $a+b+c+d=3$. 求证:
$$
\frac{1}{a^{3}}+\frac{1}{b^{3}}+\frac{1}{c^{3}}+\frac{1}{d^{3}} \leqslant \frac{1}{(a b c d)^{3}}
$$

26 设 $x_{i} \in \mathbf{R}$, 满足 $\sum_{i=1}^{n} x_{i}^{2}=1, h \geqslant 2$. 求证:
$$
\sum_{k=1}^{n}\left(1-\frac{k}{\sum_{i=1}^{n} i x_{i}^{2}}\right)^{2} \frac{x_{k}^{2}}{k} \leqslant\left(\frac{n-1}{n+1}\right)^{2} \sum_{k=1}^{n} \frac{x_{k}^{2}}{k}
$$

并确定等式成立的条件.

\section*{2}
\section*{平均值不等式的应用}
\section*{2.1 平均值不等式在不等式证明中的应用}
下面举例说明平均值不等式在证明各种竞赛问题中的应用. 在证明过程中,应用灵活,具有较高的技巧性.

例 1 设 $f(x)=\frac{a}{a^{2}-1}\left(a^{x}-a^{-x}\right)(a>0, a \neq 1)$, 证明: 对正整数 $n \geqslant 2$,有
$$
f(n)>n
$$

证明 当 $n \geqslant 2$ 时, 由平均值不等式, 得
$$
\begin{aligned}
f(n) & =\frac{a}{a^{2}-1}\left(a^{n}-a^{n}\right)=\frac{a}{a^{2}-1}\left(a^{n}-\frac{1}{a^{n}}\right) \\
& =\frac{a}{a^{2}-1}\left(a-\frac{1}{a}\right)\left(a^{n-1}+a^{n-2} \frac{1}{a}+a^{n-3} \frac{1}{a^{2}}+\cdots+a \frac{1}{a^{n-2}}+\frac{1}{a^{n-1}}\right) \\
& \geqslant \frac{a}{a^{2}-1}\left(a-\frac{1}{a}\right) n \sqrt[n]{a^{n-1} a^{n-2} \cdots a^{2} a \frac{1}{a} \frac{1}{a^{2}} \cdots \frac{1}{a^{n-1}}}=n
\end{aligned}
$$

当且仅当 $a=1$ 时等号成立,故命题成立.

例 2 设 $x>0$, 证明: $2^{12 \sqrt{x}}+2^{\sqrt[4]{x}} \geqslant 2 \cdot 2^{6^{x}}$.

证明 由该不等式的外形, 很容易想到平均值不等式. 由平均值不等式, 得
$$
2^{12 \sqrt{x}}+2^{4 \sqrt[4]{x}} \geqslant 2 \cdot \sqrt{2^{12 \sqrt{x}} 2^{\sqrt[4]{x}}}=2 \cdot 2^{\frac{12 \sqrt{x}+\sqrt{x}}{2}}
$$

又
$$
\frac{\sqrt[12]{x}+\sqrt[4]{x}}{2} \geqslant\left(x^{\frac{1}{12}} x^{\frac{1}{4}}\right)^{\frac{1}{2}}=x^{\frac{1}{6}}
$$

所以
$$
2^{12 \sqrt{x}}+2^{\sqrt[4]{x}} \geqslant 2 \cdot 2^{\sqrt[6]{x}}
$$

例 3 设 $a_{i}>0, i=1,2, \cdots, n$ 满足 $a_{1} a_{2} \cdots a_{n}=1$. 证明:
$$
\left(2+a_{1}\right)\left(2+a_{2}\right) \cdots\left(2+a_{n}\right) \geqslant 3^{n}
$$

证明 由于对任意的 $i$,
$$
\begin{gathered}
2+a_{i}=1+1+a_{i} \geqslant 3 \sqrt[3]{a_{i}} \\
\left(2+a_{1}\right)\left(2+a_{2}\right) \cdots\left(2+a_{n}\right) \geqslant 3^{n} \sqrt[3]{a_{1} a_{2} \cdots a_{n}}=3^{n}
\end{gathered}
$$

故

注 此题也可以用归纳法证明.

当 $n=1$ 时, 则 $a_{1}=1$, 显然成立. 假定当 $n=k$ 时成立, 那么, 对于 $n=$ $k+1$, 由于 $a_{1} a_{2} \cdots a_{k} a_{k+1}=1$, 如果有某个 $a_{i}=1$, 则由归纳假设, 命题成立. 如果 $a_{i}$ 都不为 1 , 则必有大于 1 的, 且必有小于 1 的, 不妨设 $a_{k}>1, a_{k+1}<1$. 则由归纳假设, 得
$$
\left(2+a_{1}\right)\left(2+a_{2}\right) \cdots\left(2+a_{k-1}\right)\left(2+a_{k} a_{k+1}\right) \geqslant 3^{k}
$$

于是, 为了证明命题, 只要证明
$$
\left(2+a_{k}\right)\left(2+a_{k+1}\right) \geqslant 3\left(2+a_{k} a_{k+1}\right)
$$

便可.

因为
$$
\left(2+a_{k}\right)\left(2+a_{k+1}\right) \geqslant 3\left(2+a_{k} a_{k+1}\right)
$$

等价于
$$
4+2 a_{k}+2 a_{k+1}+a_{k} a_{k+1} \geqslant 6+3 a_{k} a_{k+1}
$$

等价于
$$
a_{k}+a_{k+1}-a_{k} a_{k+1}-1 \geqslant 0
$$

等价于
$$
\left(a_{k}-1\right)\left(1-a_{k+1}\right) \geqslant 0
$$

由假设最后不等式成立,故命题成立.

注 这里, 选取 $a_{k}>1, a_{k+1}<1$, 在平均值不等式的证明方法四中有过类似的考虑.

例 4 设 $a>b>0$, 求证: $\sqrt{2} a^{3}+\frac{3}{a b-b^{2}} \geqslant 10$.

证明 因为 $a b-b^{2}=b(a-b) \leqslant \frac{[b+(a-b)]^{2}}{4}=\frac{a^{2}}{4}$, 所以
$$
\begin{aligned}
& \sqrt{2} a^{3}+\frac{3}{a b-b^{2}} \geqslant \sqrt{2} a^{3}+\frac{12}{a^{2}} \\
= & \frac{\sqrt{2}}{2} a^{3}+\frac{\sqrt{2}}{2} a^{3}+\frac{4}{a^{2}}+\frac{4}{a^{2}}+\frac{4}{a^{2}} \\
\geqslant & 5 \sqrt[5]{\frac{\sqrt{2}}{2} a^{3} \cdot \frac{\sqrt{2}}{2} a^{3} \cdot \frac{4}{a^{2}} \cdot \frac{4}{a^{2}} \cdot \frac{4}{a^{2}}}=10,
\end{aligned}
$$

即命题成立.

注 为了消去 $a$, 将 $\sqrt{2} a^{3}$ 写成两项, $\frac{12}{a^{2}}$ 写成三项. 这样, 利用平均值不等式,它们的乘积为一个常数.

例 $\mathbf{5}$ 设 $a, b, c \in \mathbf{R}_{+}$. 证明:
$$
\frac{1+a^{2}}{1+b}+\frac{1+b^{2}}{1+c}+\frac{1+c^{2}}{1+a} \geqslant 6(\sqrt{2}-1)
$$

解 由平均值不等式得


\begin{equation*}
\frac{1+a^{2}}{1+b}+\frac{1+b^{2}}{1+c}+\frac{1+c^{2}}{1+a} \geqslant 3 \cdot \sqrt[3]{\frac{1+a^{2}}{1+a} \cdot \frac{1+b^{2}}{1+b} \cdot \frac{1+c^{2}}{1+c}} \tag{1}
\end{equation*}


首先证明: 对任意 $x \in \mathbf{R}_{+}$, 有


\begin{equation*}
\frac{1+x^{2}}{1+x} \geqslant 2(\sqrt{2}-1) \tag{2}
\end{equation*}


事实上,
$$
\text { (2) } \begin{aligned}
& \Leftrightarrow x^{2}-2(\sqrt{2}-1) x+1-(2 \sqrt{2}-2) \geqslant 0 \\
& \Leftrightarrow x^{2}-2(\sqrt{2}-1) x+(\sqrt{2}-1)^{2} \geqslant 0 \\
& \Leftrightarrow(x-(\sqrt{2}-1))^{2} \geqslant 0
\end{aligned}
$$

即(2)成立.

由(1)、(2), 得到
$$
\frac{1+a^{2}}{1+b}+\frac{1+b^{2}}{1+c}+\frac{1+c^{2}}{1+a} \geqslant 3 \cdot \sqrt[3]{8(\sqrt{2}-1)^{3}}=6(\sqrt{2}-1)
$$

从而命题成立.

注 (1)不难验证, 当且仅当 $a=b=c=\sqrt{2}-1$ 时, 等式成立.

(2) 由于 $\left(1+a^{2}, 1+b^{2}, 1+c^{2}\right)$ 与 $\left(\frac{1}{1+a}, \frac{1}{1+b}, \frac{1}{1+c}\right)$ 为反序三元组.所以,由排序不等式和不等式 (2), 得到
$$
\frac{1+a^{2}}{1+b}+\frac{1+b^{2}}{1+c}+\frac{1+c^{2}}{1+a} \geqslant \frac{1+a^{2}}{1+a}+\frac{1+b^{2}}{1+b}+\frac{1+c^{2}}{1+c} \geqslant 6(\sqrt{2}-1)
$$

即命题成立.

例 6 设 $n$ 为正整数, $x_{i} \in \mathbf{R}_{+}, 1 \leqslant i \leqslant n$, 满足 $x_{1} \cdots x_{n}=1$. 求证:
$$
\sum_{i=1}^{n} x_{i} \sqrt{x_{1}^{2}+\cdots+x_{i}^{2}} \geqslant \frac{n+1}{2} \sqrt{n}
$$

证明 由平均值不等式, 得到
$$
\begin{aligned}
& \sum_{i=1}^{n} x_{i} \sqrt{x_{1}^{2}+\cdots+x_{i}^{2}} \geqslant \sum_{i=1}^{n} x_{i} \frac{x_{1}+\cdots+x_{i}}{\sqrt{i}} \\
= & \sum_{i=1}^{n} \sum_{j=1}^{i} \frac{x_{i} x_{j}}{\sqrt{i}} \geqslant \frac{1}{\sqrt{n}} \sum_{i=1}^{n} \sum_{j=1}^{i} x_{i} x_{j} \\
= & \frac{1}{2 \sqrt{n}} \sum_{1 \leqslant j \leqslant i \leqslant n} 2 x_{i} x_{j} \\
= & \frac{1}{2 \sqrt{n}}\left(\sum_{i=1}^{n} x_{i}^{2}+\left(\sum_{i=1}^{n} x_{i}\right)^{2}\right) \\
\geqslant & \frac{1}{2 \sqrt{n}}\left[n \cdot \sqrt[n]{x_{1}^{2} \cdots x_{n}^{2}}+\left(n \cdot \sqrt[n]{x_{1} \cdots x_{n}}\right)^{2}\right] \\
= & \frac{n+n^{2}}{2 \sqrt{n}}=\frac{n+1}{2} \sqrt{n} .
\end{aligned}
$$

即命题成立.

注 这里用到了双求和符号及有关性质.

例 7 设 $a_{1}, a_{2}, \cdots, a_{n} \in \mathbf{R}_{+}, S=a_{1}+a_{2}+\cdots+a_{n}$. 求证:
$$
\left(1+a_{1}\right)\left(1+a_{2}\right) \cdots\left(1+a_{n}\right) \leqslant 1+S+\frac{S^{2}}{2!}+\cdots+\frac{S^{n}}{n!}
$$

证明 由于 $G_{n} \leqslant A_{n}$, 得
$$
\begin{aligned}
& \left(1+a_{1}\right)\left(1+a_{2}\right) \cdots\left(1+a_{n}\right) \\
\leqslant & \left(\frac{n+a_{1}+a_{2}+\cdots+a_{n}}{n}\right)^{n}=\left(1+\frac{S}{n}\right)^{n} \\
= & 1+\mathrm{C}_{n}^{1}\left(\frac{S}{n}\right)+\mathrm{C}_{n}^{2}\left(\frac{S}{n}\right)^{2}+\cdots+\mathrm{C}_{n}^{m}\left(\frac{S}{n}\right)^{m}+\cdots+\mathrm{C}_{n}^{n}\left(\frac{S}{n}\right)^{n}
\end{aligned}
$$

因为 $n!=(n-m)!(n-m+1) \cdots n \leqslant(n-m)!n^{m}$,

所以
$$
\mathrm{C}_{n}^{m}\left(\frac{S}{n}\right)^{m}=\frac{n!}{m!(n-m)!} \cdot \frac{1}{n^{m}} S^{m} \leqslant \frac{S^{m}}{m!}
$$

从而命题成立.

例 8 设 $k 、 n$ 为正整数, 且 $1 \leqslant k \leqslant n, a_{i} \in \mathbf{R}_{+}$, 满足 $a_{1}+a_{2}+\cdots+$ $a_{k}=a_{1} a_{2} \cdots a_{k}$. 求证:
$$
a_{1}^{n-1}+a_{2}^{n-1}+\cdots+a_{k}^{n-1} \geqslant k n,
$$

并确定等号成立的充要条件.

证明 令 $a=a_{1}+a_{2}+\cdots+a_{k}=a_{1} a_{2} \cdots a_{k}$. 由平均值不等式, 得
$$
a \geqslant k a^{\frac{1}{k}} \text {, 即 } a \geqslant k^{\frac{k}{k-1}} .
$$

又因为
$$
a_{1}^{n-1}+a_{2}^{n-1}+\cdots+a_{k}^{n-1} \geqslant k\left(a_{1} a_{2} \cdots a_{k}\right)^{\frac{n-1}{k}}=k a^{\frac{n-1}{k}} \geqslant k \cdot k^{\frac{n-1}{k-1}},
$$

于是只需证明
$$
k^{\frac{n-1}{k-1}} \geqslant n
$$

再由平均值不等式,得
$$
k=\frac{(k-1) n+(n-k) \times 1}{n-1} \geqslant n^{\frac{k-1}{n-1}},
$$

从而不等式成立.

不难看出, 当 $k=n$ 且 $a_{1}=a_{2}=\cdots=a_{k}$ 时等号成立.

例 9 设 $a_{i}>0(i=1,2, \cdots, n)$, 求证:
$$
\sum_{k=1}^{n} k a_{k} \leqslant \frac{n(n-1)}{2}+\sum_{k=1}^{n} a_{k}^{k} .
$$

证明 因为 $\frac{n(n-1)}{2}=\sum_{k=1}^{n}(k-1)$, 所以由平均值不等式, 得
$$
\begin{aligned}
\frac{n(n-1)}{2}+\sum_{k=1}^{n} a_{k}^{k} & =\sum_{k=1}^{n}\left[(k-1)+a_{k}^{k}\right] \\
& =\sum_{k=1}^{n}\left(1+1+\cdots+1+a_{k}^{k}\right) \\
& \geqslant \sum_{k=1}^{n} k \sqrt[k]{1^{k-1} \cdot a_{k}^{k}}=\sum_{k=1}^{n} k a_{k}
\end{aligned}
$$

故命题成立.

注 应用平均值不等式时, 通常要将乘幂看作连乘积, 有时还要巧妙地添上数 1.

例 10 设 $a_{i}>0, b_{i}>0$ 且满足 $a_{1}+a_{2}+\cdots+a_{n} \leqslant 1, b_{1}+b_{2}+\cdots+b_{n} \leqslant n$.求证:
$$
\left(\frac{1}{a_{1}}+\frac{1}{b_{1}}\right)\left(\frac{1}{a_{2}}+\frac{1}{b_{2}}\right) \cdots\left(\frac{1}{a_{n}}+\frac{1}{b_{n}}\right) \geqslant(n+1)^{n}
$$

证明 由已知条件和平均值不等式, 得
$$
\begin{aligned}
& a_{1} a_{2} \cdots a_{n} \leqslant\left(\frac{a_{1}+a_{2}+\cdots+a_{n}}{n}\right)^{n} \leqslant \frac{1}{n^{n}} \\
& b_{1} b_{2} \cdots b_{n} \leqslant\left(\frac{b_{1}+b_{2}+\cdots+b_{n}}{n}\right)^{n} \leqslant 1 . \\
& \quad \frac{1}{a_{i}}+\frac{1}{b_{i}}=\frac{1}{n a_{i}}+\cdots+\frac{1}{n a_{i}}+\frac{1}{b_{i}} \geqslant(n+1) \sqrt[n+1]{\left(\frac{1}{n a_{i}}\right)^{n}\left(\frac{1}{b_{i}}\right)}
\end{aligned}
$$

从而
$$
\begin{aligned}
& \left(\frac{1}{a_{1}}+\frac{1}{b_{1}}\right)\left(\frac{1}{a_{2}}+\frac{1}{b_{2}}\right) \cdots\left(\frac{1}{a_{n}}+\frac{1}{b_{n}}\right) \\
\geqslant & (n+1)^{n^{n+1}} \sqrt{\frac{1}{\left(n^{n}\right)^{n}} \frac{1}{\left(a_{1} a_{2} \cdots a_{n}\right)^{n}} \frac{1}{b_{1} b_{2} \cdots b_{n}}} \\
\geqslant & (n+1)^{n} .
\end{aligned}
$$

故命题成立.

注 此题证明的关键是将 $\frac{1}{a_{i}}$ 写成 $\frac{1}{n a_{i}}+\cdots+\frac{1}{n a_{i}}$.

例 11 假设 $a 、 b 、 c$ 都是正数,证明:
$$
a b c \geqslant(a+b-c)(b+c-a)(c+a-b)
$$

证明 如果 $a+b-c, b+c-a, c+a-b$ 中有负数, 不妨设 $a+b-c<0$, 则 $c>a+b$. 故 $b+c-a$ 与 $c+a-b$ 均为正数, 则结论显然成立.

若 $a+b-c, b+c-a, c+a-b$ 均非负, 则由平均值不等式, 得
$$
\sqrt{(a+b-c)(b+c-a)} \leqslant \frac{(a+b-c)+(b+c-a)}{2}=b
$$

同理可得
$$
\begin{aligned}
& \sqrt{(b+c-a)(c+a-b)} \leqslant \frac{(b+c-a)+(c+a-b)}{2}=c \\
& \sqrt{(c+a-b)(a+b-c)} \leqslant \frac{(c+a-b)+(a+b-c)}{2}=a
\end{aligned}
$$

将三式相乘,即得到我们要证明的问题,故命题成立.

注 通过对部分变量应用平均值不等式, 而且轮换使用, 从而得到结论的证明.

例 12 设整数 $n \geqslant 2, x_{i} \in \mathbf{R}_{+}, 1 \leqslant i \leqslant n$, 满足 $\sum_{i=1}^{n} x_{i}=1$.\\
求证: $\left(\sum_{i=1}^{n} \frac{1}{1-x_{i}}\right)\left(\sum_{1 \leqslant i<j \leqslant n} x_{i} x_{j}\right) \leqslant \frac{n}{2}$.

证明 首先证局部不等式, 即对每个 $1 \leqslant k \leqslant n$, 有


\begin{equation*}
\left(2 \sum_{1 \leqslant i<j \leqslant n} x_{i} x_{j}\right) \frac{1}{1-x_{k}} \leqslant 2 x_{k}+\frac{n-2}{n-1} \sum_{i \neq k} x_{i} \tag{1}
\end{equation*}


事实上,由平均值不等式
$$
\sum_{i \neq k} x_{i}^{2} \geqslant \frac{2}{n-2} \sum_{i, j \neq k} x_{i} x_{j}
$$

从而,
$$
2 \sum_{\substack{1 \leqslant i<j \leqslant n \\ i, j \neq k}} x_{i} x_{j} \leqslant \frac{n-2}{n-1}\left(\sum_{i \neq k} x_{i}\right)^{2}
$$

于是
$$
\begin{aligned}
& \left(2 \sum_{1 \leqslant i<j \leqslant n} x_{i} x_{j}\right) \frac{1}{1-x_{k}} \\
= & \left(2 x_{k}\left(1-x_{k}\right)+2 \sum_{\substack{1 \leqslant i<j \leqslant n \\
i, j \neq k}} x_{i} x_{j}\right) \frac{1}{1-x_{k}} \\
= & 2 x_{k}+\frac{2 \sum_{i \neq j \neq k} x_{i} x_{j}}{\sum_{i \neq k} x_{i}} \\
\leqslant & 2 x_{k}+\frac{n-2}{n-1} \sum_{i \neq k} x_{i} .
\end{aligned}
$$

所以(1)成立.

由不等式(1)关于 $1 \leqslant k \leqslant n$ 求和, 得到
$$
2 \sum_{1 \leqslant i<j \leqslant n} x_{i} x_{j} \sum_{k=1}^{n} \frac{1}{1-x_{k}} \leqslant 2 \sum_{k=1}^{n} x_{k}+\frac{n-2}{n-1} \sum_{k=1}^{n} \sum_{i \neq k} x_{i}=n
$$

即
$$
\sum_{1 \leqslant i<j \leqslant n} x_{i} x_{j} \cdot \sum_{i=1}^{n} \frac{1}{1-x_{i}} \leqslant \frac{n}{2}
$$

故命题成立.

另外,我们也可以用切比雪夫不等式证明.

由于
$$
2 \sum_{1 \leqslant i<j \leqslant n} x_{i} x_{j}=\sum_{i=1}^{n} x_{i} \sum_{1 \leqslant i<j \leqslant n} x_{j}=\sum_{i=1}^{n} x_{i}\left(1-x_{i}\right)
$$

所以,原不等式等价于


\begin{equation*}
\left(\sum_{i=1}^{n} \frac{1}{1-x_{i}}\right)\left(\sum_{i=1}^{n} x_{i}\left(1-x_{i}\right)\right) \leqslant n \tag{2}
\end{equation*}


不妨设 $0<x_{1} \leqslant x_{2} \leqslant \cdots \leqslant x_{n} \leqslant 1$.

由于对任意 $1 \leqslant i<j \leqslant n$, 有 $x_{i}+x_{j} \leqslant 1,0<x_{i}<x_{j} \leqslant 1$.

从而, $\left(x_{i}-x_{j}\right)\left(1-x_{i}-x_{j}\right) \leqslant 0$, 即
$$
x_{i}\left(1-x_{i}\right) \leqslant x_{j}\left(1-x_{j}\right)
$$

于是 $x_{1}\left(1-x_{1}\right) \leqslant x_{2}\left(1-x_{2}\right) \leqslant \cdots \leqslant x_{n}\left(1-x_{n}\right)$, 以及
$$
\frac{1}{1-x_{1}} \leqslant \frac{1}{1-x_{2}} \leqslant \cdots \leqslant \frac{1}{1-x_{n}}
$$

由切比雪夫不等式,得
$$
\frac{1}{n}\left(\sum_{i=1}^{n} \frac{1}{1-x_{i}}\right)\left(\sum_{i=1}^{n} x_{i}\left(1-x_{i}\right)\right) \leqslant\left(\sum_{i=1}^{n} \frac{1}{1-x_{i}} \cdot x_{i}\left(1-x_{i}\right)\right)=1
$$

所以(2)成立. 故命题成立.

例 13 设 $n$ 为正整数,证明:
$$
n\left[(n+1)^{\frac{1}{n}}-1\right] \leqslant 1+\frac{1}{2}+\frac{1}{3}+\cdots+\frac{1}{n} \leqslant n-(n-1)\left(\frac{1}{n}\right)^{\frac{1}{n-1}}
$$

证明 只证明不等式的左边,不等式的右边可同样处理.

令 $A=\frac{1+\frac{1}{2}+\frac{1}{3}+\cdots+\frac{1}{n}+n}{n}$, 则左边的不等式等价于
$$
A \geqslant(n+1)^{\frac{1}{n}}
$$

由平均值不等式, 得
$$
\begin{aligned}
A & =\frac{(1+1)+\left(1+\frac{1}{2}\right)+\cdots+\left(1+\frac{1}{n}\right)}{n} \\
& =\frac{2+\frac{3}{2}+\frac{4}{3}+\cdots+\frac{n+1}{n}}{n} \\
& \geqslant \sqrt[n]{2 \cdot \frac{3}{2} \cdot \frac{4}{3} \cdot \cdots \cdot \frac{n+1}{n}}=(n+1)^{\frac{1}{n}}
\end{aligned}
$$

从而得
$$
1+\frac{1}{2}+\frac{1}{3}+\cdots+\frac{1}{n} \geqslant n\left[(n+1)^{\frac{1}{n}}-1\right] .
$$

不难看出, 当 $n=1$ 时等号成立.\\
例 14 设 $a_{i}>0, i=1,2, \cdots, n, m>0$ 且满足 $\sum_{i=1}^{n} \frac{1}{1+a_{i}^{m}}=1$. 求证:
$$
a_{1} a_{2} \cdots a_{n} \geqslant(n-1)^{\frac{n}{m}}
$$

证法一 令 $x_{i}=\frac{1}{1+a_{i}^{m}}$, 则 $a_{i}^{m}=\frac{1-x_{i}}{x_{i}}$, 且 $x_{i}>0, i=1,2, \cdots, n$, $\sum_{i=1}^{n} x_{i}=1$. 于是
$$
\begin{aligned}
a_{1}^{m} a_{2}^{m} \cdots a_{n}^{m} & =\frac{\left(x_{2}+\cdots+x_{n}\right) \cdots\left(x_{1}+x_{2}+\cdots+x_{n-1}\right)}{x_{1} x_{2} \cdots x_{n}} \\
& \geqslant \frac{(n-1) \sqrt[n-1]{x_{2} x_{3} \cdots x_{n}} \cdots(n-1) \sqrt[n-1]{x_{1} x_{2} \cdots x_{n-1}}}{x_{1} x_{2} \cdots x_{n}} \\
& =(n-1)^{n} .
\end{aligned}
$$

故命题成立.

证法二 令 $a_{i}^{m}=\tan ^{2} \alpha_{i}$, 则 $\sum_{i=1}^{n} \frac{1}{1+a_{i}^{m}}=1$ 等价于
$$
\sum_{i=1}^{n} \cos ^{2} \alpha_{i}=1
$$

其结论等价于
$$
\tan ^{2} \alpha_{1} \tan ^{2} \alpha_{2} \cdots \tan ^{2} \alpha_{n} \geqslant(n-1)^{n}
$$

即
$$
\sin ^{2} \alpha_{1} \sin ^{2} \alpha_{2} \cdots \sin ^{2} \alpha_{n} \geqslant(n-1)^{n} \cos ^{2} \alpha_{1} \cos ^{2} \alpha_{2} \cdots \cos ^{2} \alpha_{n}
$$

由平均值不等式, 得
$$
\begin{aligned}
\sin ^{2} \alpha_{1} & =1-\cos ^{2} \alpha_{1}=\cos ^{2} \alpha_{2}+\cdots+\cos ^{2} \alpha_{n} \\
& \geqslant(n-1) \sqrt[n-1]{\cos ^{2} \alpha_{2} \cos ^{2} \alpha_{3} \cdots \cos ^{2} \alpha_{n}}
\end{aligned}
$$

一般地,
$$
\begin{aligned}
\sin ^{2} \alpha_{i} & =1-\cos ^{2} \alpha_{i} \\
& =\cos ^{2} \alpha_{1}+\cdots+\cos ^{2} \alpha_{i-1}+\cos ^{2} \alpha_{i+1}+\cdots+\cos ^{2} \alpha_{n} \\
& \geqslant(n-1) \sqrt[n-1]{\cos ^{2} \alpha_{1} \cdots \cos ^{2} \alpha_{i-1} \cos ^{2} \alpha_{i+1} \cdots \cos ^{2} \alpha_{n}}
\end{aligned}
$$

将它们相乘, 得
$$
\sin ^{2} \alpha_{1} \sin ^{2} \alpha_{2} \cdots \sin ^{2} \alpha_{n} \geqslant(n-1)^{n} \cos ^{2} \alpha_{1} \cos ^{2} \alpha_{2} \cdots \cos ^{2} \alpha_{n}
$$

故命题成立.\\
例 15 设 $a, b, c \in \mathbf{R}_{+}$, 且 $a^{2}+b^{2}+c^{2}=1$. 求证:
$$
\frac{a}{1-a^{2}}+\frac{b}{1-b^{2}}+\frac{c}{1-c^{2}} \geqslant \frac{3 \sqrt{3}}{2}
$$

证明 原不等式
$$
\frac{a}{1-a^{2}}+\frac{b}{1-b^{2}}+\frac{c}{1-c^{2}} \geqslant \frac{3 \sqrt{3}}{2}
$$

等价于
$$
\frac{a^{2}}{a\left(1-a^{2}\right)}+\frac{b^{2}}{b\left(1-b^{2}\right)}+\frac{c^{2}}{c\left(1-c^{2}\right)} \geqslant \frac{3 \sqrt{3}}{2}
$$

由于 $a^{2}+b^{2}+c^{2}=1$, 如果能证明 $x\left(1-x^{2}\right) \leqslant \frac{2}{3 \sqrt{3}}$, 则上述不等式成立.由平均值不等式, 得
$$
\begin{aligned}
x\left(1-x^{2}\right) & =\sqrt{\frac{2 x^{2}\left(1-x^{2}\right)\left(1-x^{2}\right)}{2}} \\
& \leqslant \sqrt{\frac{1}{2}\left[\frac{2 x^{2}+\left(1-x^{2}\right)+\left(1-x^{2}\right)}{3}\right]^{3}} \\
& =\sqrt{\frac{1}{2} \cdot\left(\frac{2}{3}\right)^{3}}=\frac{2}{3 \sqrt{3}}
\end{aligned}
$$

故不等式成立.

注 由于分子之和 $a^{2}+b^{2}+c^{2}=1$, 所以当各分母被控制在某个常数之内时,便可以推出命题成立. 这个方法在分式不等式证明中常常使用.

例 16 设 $a_{1}, a_{2}, \cdots, a_{n}$ 是 $1,2, \cdots, n$ 的一个排列. 求证:
$$
\frac{1}{2}+\frac{2}{3}+\cdots+\frac{n-1}{n} \leqslant \frac{a_{1}}{a_{2}}+\frac{a_{2}}{a_{3}}+\cdots+\frac{a_{n-1}}{a_{n}}
$$

证明 因为 $a_{1}, a_{2}, \cdots, a_{n}$ 是 $1,2, \cdots, n$ 的一个排列, 所以
$$
\begin{aligned}
& \left(1+a_{1}\right)\left(1+a_{2}\right) \cdots\left(1+a_{n-1}\right) \\
\geqslant & (1+1)(1+2) \cdots[1+(n-1)] \\
= & a_{1} a_{2} \cdots a_{n}
\end{aligned}
$$

于是
$$
\begin{aligned}
& \frac{a_{1}}{a_{2}}+\frac{a_{2}}{a_{3}}+\cdots+\frac{a_{n-1}}{a_{n}}+\frac{1}{1}+\frac{1}{2}+\cdots+\frac{1}{n} \\
= & \frac{a_{1}}{a_{2}}+\frac{a_{2}}{a_{3}}+\cdots+\frac{a_{n-1}}{a_{n}}+\frac{1}{a_{1}}+\frac{1}{a_{2}}+\cdots+\frac{1}{a_{n}}
\end{aligned}
$$
$$
\begin{aligned}
& =\frac{1}{a_{1}}+\frac{1+a_{1}}{a_{2}}+\frac{1+a_{2}}{a_{3}}+\cdots+\frac{1+a_{n-1}}{a_{n}} \\
& \geqslant n \sqrt[n]{\frac{\left(1+a_{1}\right)\left(1+a_{2}\right) \cdots\left(1+a_{n-1}\right)}{a_{1} a_{2} \cdots a_{n}}} \geqslant n
\end{aligned}
$$

又因为 $n=\left(1+\frac{1}{2}+\frac{1}{3}+\cdots+\frac{1}{n}\right)+\left(\frac{1}{2}+\frac{2}{3}+\cdots+\frac{n-1}{n}\right)$, 所以
$$
\frac{a_{1}}{a_{2}}+\frac{a_{2}}{a_{3}}+\cdots+\frac{a_{n-1}}{a_{n}} \geqslant \frac{1}{2}+\frac{2}{3}+\cdots+\frac{n-1}{n}
$$

注 对于该不等式的证明,首先要充分理解 $a_{1}, a_{2}, \cdots, a_{n}$ 是 $1,2, \cdots, n$的一个排列, 此外, 两边同时相加 $1+\frac{1}{2}+\frac{1}{3}+\cdots+\frac{1}{n}\left(\right.$ 即 $\left.\frac{1}{a_{1}}+\frac{1}{a_{2}}+\cdots+\frac{1}{a_{n}}\right)$ 也是很重要的一步.

例 17 设 $a 、 b 、 c$ 为正实数, 求证:
$$
\frac{a}{\sqrt{a^{2}+8 b c}}+\frac{b}{\sqrt{b^{2}+8 a c}}+\frac{c}{\sqrt{c^{2}+8 a b}} \geqslant 1
$$

证明 容易看出, 如果我们能证明 $\frac{a}{\sqrt{a^{2}+8 b c}} \geqslant \frac{a^{\frac{4}{3}}}{a^{\frac{4}{3}}+b^{\frac{4}{3}}+c^{\frac{4}{3}}}$, 那么,将它们相加便得到所要证明的不等式. 因为
$$
\frac{a}{\sqrt{a^{2}+8 b c}} \geqslant \frac{a^{\frac{4}{3}}}{a^{\frac{4}{3}}+b^{\frac{4}{3}}+c^{\frac{4}{3}}}
$$

等价于
$$
\left(a^{\frac{4}{3}}+b^{\frac{4}{3}}+c^{\frac{4}{3}}\right)^{2} \geqslant a^{\frac{2}{3}}\left(a^{2}+8 b c\right)
$$

再由平均值不等式, 得
$$
\begin{aligned}
\left(a^{\frac{4}{3}}+b^{\frac{4}{3}}+c^{\frac{4}{3}}\right)^{2}-\left(a^{\frac{4}{3}}\right)^{2} & =\left(b^{\frac{4}{3}}+c^{\frac{4}{3}}\right)\left(a^{\frac{4}{3}}+a^{\frac{4}{3}}+b^{\frac{4}{3}}+c^{\frac{4}{3}}\right) \\
& \geqslant 2 b^{\frac{2}{3}} c^{\frac{2}{3}} \cdot 4 a^{\frac{2}{3}} b^{\frac{1}{3}} c^{\frac{1}{3}}=8 a^{\frac{2}{3}} b c
\end{aligned}
$$

于是
$$
\left(a^{\frac{4}{3}}+b^{\frac{4}{3}}+c^{\frac{4}{3}}\right)^{2} \geqslant\left(a^{\frac{4}{3}}\right)^{2}+8 a^{\frac{2}{3}} b c=a^{\frac{2}{3}}\left(a^{2}+8 b c\right)
$$

从而
$$
\frac{a}{\sqrt{a^{2}+8 b c}} \geqslant \frac{a^{\frac{4}{3}}}{a^{\frac{4}{3}}+b^{\frac{4}{3}}+c^{\frac{4}{3}}}
$$

同理可得
$$
\frac{b}{\sqrt{b^{2}+8 a c}} \geqslant \frac{b^{\frac{4}{3}}}{a^{\frac{4}{3}}+b^{\frac{4}{3}}+c^{\frac{4}{3}}}, \frac{c}{\sqrt{c^{2}+8 a b}} \geqslant \frac{c^{\frac{4}{3}}}{a^{\frac{4}{3}}+b^{\frac{4}{3}}+c^{\frac{4}{3}}}
$$

于是
$$
\frac{a}{\sqrt{a^{2}+8 b c}}+\frac{b}{\sqrt{b^{2}+8 a c}}+\frac{c}{\sqrt{c^{2}+8 a b}} \geqslant 1
$$

注 这是一道 IMO 试题, 有多种不同的证明方法, 后面将再次遇到. 这里的指数 $\frac{4}{3}$ 是这样得到的: 取 $x$ 为待定常数.

设
$$
\frac{a}{\sqrt{a^{2}+8 b c}} \geqslant \frac{a^{x}}{a^{x}+b^{x}+c^{x}}
$$

上式等价于
$$
\begin{aligned}
& a^{2}\left(a^{x}+b^{x}+c^{x}\right)^{2} \geqslant a^{2 x}\left(a^{2}+8 b c\right) \\
\Leftrightarrow & \left(a^{x}+b^{x}+c^{x}\right)^{2} \geqslant a^{2 x-2}\left(a^{2}+8 b c\right) \\
\Leftrightarrow & a^{2 x}+2 a^{x}\left(b^{x}+c^{x}\right)+\left(b^{x}+c^{x}\right)^{2} \geqslant a^{2 x}+8 a^{2 x-2} b c \\
\Leftrightarrow & 2 a^{x}\left(b^{x}+c^{x}\right)+\left(b^{x}+c^{x}\right)^{2} \geqslant 8 a^{2 x-2} b c .
\end{aligned}
$$

由于
$$
b^{x}+c^{x} \geqslant 2 b^{\frac{x}{2}} c^{\frac{x}{2}}
$$

只需
$$
2 a^{x} \cdot 2 b^{\frac{x}{2}} c^{\frac{x}{2}}+\left(2 b^{\frac{x}{2}} c^{\frac{x}{2}}\right)^{2} \geqslant 8 a^{2 x-2} b c
$$

即
$$
a^{x} b^{\frac{x}{2}} c^{\frac{x}{2}}+b^{x} c^{x} \geqslant 2 a^{2 x-2} b c
$$

由于
$$
a^{x} b^{\frac{x}{2}} c^{\frac{x}{2}}+b^{x} c^{x} \geqslant 2 \sqrt{a^{x} b^{\frac{3}{2} x} c^{\frac{3}{2} x}}=2 a^{\frac{x}{2}} b^{\frac{3}{4} x} c^{\frac{3}{4} x}
$$

所以只需
$$
a^{\frac{x}{2}} b^{\frac{3}{4} x} c^{\frac{3}{4} x} \geqslant a^{2 x-2} b c
$$

显然取 $x=\frac{4}{3}$ 满足要求.

例 18 已知正整数 $n \geqslant 2$, 实数 $a_{i} 、 b_{i}$, 满足
$$
a_{1} \geqslant a_{2} \geqslant \cdots \geqslant a_{n}>0, b_{1} \geqslant b_{2} \geqslant \cdots \geqslant b_{n}>0
$$

并且
$$
a_{1} a_{2} \cdots a_{n}=b_{1} b_{2} \cdots b_{n},
$$
$$
\sum_{1 \leqslant i<j \leqslant n}\left(a_{i}-a_{j}\right) \leqslant \sum_{1 \leqslant i<j \leqslant n}\left(b_{i}-b_{j}\right)
$$

求证: $\sum_{i=1}^{n} a_{i} \leqslant(n-1) \sum_{i=1}^{n} b_{i}$.

证明 当 $n=2$ 时,
$$
\left(a_{1}+a_{2}\right)^{2}-\left(a_{1}-a_{2}\right)^{2}=4 a_{1} a_{2}=4 b_{1} b_{2}=\left(b_{1}+b_{2}\right)^{2}-\left(b_{1}-b_{2}\right)^{2},
$$

由假设得 $a_{1}-a_{2} \leqslant b_{1}-b_{2}$, 所以 $a_{1}+a_{2} \leqslant b_{1}+b_{2}$.

当 $n \geqslant 3$ 时, 不妨设 $b_{1} b_{2} \cdots b_{n}=1$ (否则用 $a^{\prime}{ }_{i}=\frac{a_{i}}{\sqrt[n]{a_{1} a_{2} \cdots a_{n}}}, b_{i}^{\prime}=$ $\frac{b_{i}}{\sqrt[n]{b_{1} b_{2} \cdots b_{n}}}$ 代替 $\left.a_{i}, b_{i}(1 \leqslant i \leqslant n)\right)$.

如果 $a_{1} \leqslant n-1$, 则由平均值不等式, 得
$$
\sum_{i=1}^{n} a_{i} \leqslant n(n-1)=(n-1) n \sqrt[n]{b_{1} b_{2} \cdots b_{n}} \leqslant(n-1) \sum_{i=1}^{n} b_{i}
$$

下设 $a_{1}>n-1$, 因为
$$
\begin{aligned}
\sum_{1 \leqslant i<j \leqslant n}\left(a_{i}-a_{j}\right) \geqslant & {\left[\left(a_{1}-a_{n}\right)+\left(a_{2}-a_{n}\right)+\cdots+\left(a_{n-1}-a_{n}\right)\right] } \\
& +\left[\left(a_{1}-a_{2}\right)+\left(a_{2}-a_{3}\right)+\cdots+\left(a_{n-2}-a_{n-1}\right)\right] \\
= & \sum_{i=1}^{n} a_{i}+\left(a_{1}-a_{n-1}\right)-n a_{n} \\
\sum_{1 \leqslant i<j \leqslant n}\left(b_{i}-b_{j}\right)= & \sum_{i=1}^{n}(n-2 i+1) b_{i} \\
= & \sum_{i=1}^{n}\left[(n-1) b_{i}+(2-2 i) b_{i}\right] \\
= & (n-1) \sum_{i=1}^{n} b_{i}+\sum_{i=1}^{n}(2-2 i) b_{i} \\
\leqslant & (n-1) \sum_{i=1}^{n} b_{i}-2 b_{2}-2(n-1) b_{n}
\end{aligned}
$$

所以, 当 $a_{1}-a_{n-1}-n a_{n}+2 b_{2}+2(n-1) b_{n} \geqslant 0$ 时, 结论成立.

当 $a_{1}-a_{n-1}-n a_{n}+2 b_{2}+2(n-1) b_{n}<0$ 时,得
$$
n a_{n}>2(n-1) b_{n}+2 b_{2}+a_{1}-a_{n-1} \geqslant 2(n-1) b_{n}+2 b_{2} \geqslant 2 n b_{n}
$$

即 $a_{n}>2 b_{n}$.

又由 $a_{1} a_{2} \cdots a_{n}=1$, 得 $a_{n} \leqslant 1$, 所以
$$
a_{1}-(n-1) a_{n}>n-1-(n-1)=0
$$

于是 $\quad 2 b_{2}<a_{n-1}+n a_{n}-a_{1}-2(n-1) b_{n}<a_{n-1}+a_{n} \leqslant 2 a_{n-1}$,

即 $b_{2}<a_{n-1}$. 从而可得
$$
b_{1} b_{2} \cdots b_{n}=a_{1} a_{2} \cdots a_{n}>2 b_{n} b_{2} a_{1} a_{2} \cdots a_{n-2}
$$

即
$$
b_{1} b_{3} \cdots b_{n-1}>2 a_{1} a_{2} \cdots a_{n-2}
$$

而
$$
\begin{aligned}
& b_{3} \leqslant b_{2}<a_{n-1} \leqslant a_{n-2} \\
& b_{4} \leqslant b_{3}<a_{n-2} \leqslant a_{n-3} \\
& \cdots \cdots \\
& b_{n-1} \leqslant b_{n-2}<a_{3} \leqslant a_{2}
\end{aligned}
$$

所以
$$
b_{1}>2 a_{1},(n-1) \sum_{i=1}^{n} b_{i}>2(n-1) a_{1}>n a_{1} \geqslant \sum_{i=1}^{n} a_{i}
$$

例 19 给定 $n \geqslant 2, n \in \mathbf{Z}_{+}$, 求所有 $m \in \mathbf{Z}_{+}$, 使得对 $a_{i} \in \mathbf{R}_{+}, i=1,2$, $\cdots, n$, 满足 $a_{1} a_{2} \cdots a_{n}=1$, 则
$$
a_{1}^{m}+a_{2}^{m}+\cdots+a_{n}^{m} \geqslant \frac{1}{a_{1}}+\frac{1}{a_{2}}+\cdots+\frac{1}{a_{n}}
$$

解 取 $x=a_{1}=a_{2}=\cdots=a_{n-1}>0, a_{n}=\frac{1}{x^{n-1}}$, 则
$$
(n-1) x^{m}+\frac{1}{x^{(n-1) m}} \geqslant \frac{n-1}{x}+x^{n-1}
$$

由此得到 $m \geqslant n-1$. 现在, 假设 $m \geqslant n-1$, 则
$$
\begin{aligned}
& (n-1)\left(a_{1}^{m}+a_{2}^{m}+\cdots+a_{n}^{m}\right)+n(m-n+1) \\
= & \left(a_{1}^{m}+a_{2}^{m}+\cdots+a_{n-1}^{m}+1+1+\cdots+1\right)(\text { 共 } m-n+1 \text { 个 } 1) \\
& +\left(a_{1}^{m}+a_{2}^{m}+\cdots+a_{n-2}^{m}+a_{n}^{m}+1+1+\cdots+1\right)+\cdots \\
& +\left(a_{2}^{m}+a_{3}^{m}+\cdots+a_{n}^{m}+1+1+\cdots+1\right) \\
\geqslant & m \sqrt[m]{\left(a_{1} a_{2} \cdots a_{n-1}\right)^{m}}+\cdots+m \sqrt[m]{\left(a_{2} \cdots a_{n}\right)^{m}} \\
= & m\left(a_{1} a_{2} \cdots a_{n-1}+\cdots+a_{2} a_{3} \cdots a_{n}\right) \\
= & m\left(\frac{1}{a_{1}}+\frac{1}{a_{2}}+\cdots+\frac{1}{a_{n}}\right) .
\end{aligned}
$$

所以
$$
a_{1}^{m}+a_{2}^{m}+\cdots+a_{n}^{m} \geqslant \frac{m}{n-1}\left(\frac{1}{a_{1}}+\frac{1}{a_{2}}+\cdots+\frac{1}{a_{n}}\right)-\frac{n}{n-1}(m-n+1)
$$

于是, 只要证明
$$
\frac{m}{n-1}\left(\frac{1}{a_{1}}+\frac{1}{a_{2}}+\cdots+\frac{1}{a_{n}}\right)-\frac{n}{n-1}(m-n+1) \geqslant \frac{1}{a_{1}}+\frac{1}{a_{2}}+\cdots+\frac{1}{a_{n}}
$$

即
$$
(m-n+1)\left(\frac{1}{a_{1}}+\frac{1}{a_{2}}+\cdots+\frac{1}{a_{n}}-n\right) \geqslant 0
$$

由假设以及平均值不等式, 得
$$
\left(\frac{1}{a_{1}}+\frac{1}{a_{2}}+\cdots+\frac{1}{a_{n}}-n\right) \geqslant n \sqrt[n]{\frac{1}{a_{1}} \cdot \frac{1}{a_{2}} \cdots \cdots \cdot \frac{1}{a_{n}}}-n=0
$$

所以原不等式成立.

故对所有满足 $m \geqslant n-1$ 的 $m \in \mathbf{Z}_{+}$均可.

例 20 设 $n(n \geqslant 2)$ 是整数, $a_{1}, a_{2}, \cdots, a_{n} \in \mathbf{R}_{+}$, 求证:
$$
\left(a_{1}^{3}+1\right)\left(a_{2}^{3}+1\right) \cdots\left(a_{n}^{3}+1\right) \geqslant\left(a_{1}^{2} a_{2}+1\right)\left(a_{2}^{2} a_{3}+1\right) \cdots\left(a_{n}^{2} a_{1}+1\right) .
$$

证明 先证对于正实数 $x_{i}, y_{i}(i=1,2,3)$, 有
$$
\Pi\left(x_{i}^{3}+y_{i}^{3}\right) \geqslant\left(\prod x_{i}+\prod y_{i}\right)^{3}
$$

实际上, 由平均值不等式, 得
$$
\begin{aligned}
& \sqrt[3]{\frac{x_{1}^{3} x_{2}^{3} x_{3}^{3}}{\prod\left(x_{i}^{3}+y_{i}^{3}\right)}} \leqslant \frac{1}{3}\left(\sum_{i=1}^{3} \frac{x_{i}^{3}}{x_{i}^{3}+y_{i}^{3}}\right) \\
& \sqrt[3]{\frac{y_{1}^{3} y_{2}^{3} y_{3}^{3}}{\prod\left(x_{i}^{3}+y_{i}^{3}\right)}} \leqslant \frac{1}{3}\left(\sum_{i=1}^{3} \frac{y_{i}^{3}}{x_{i}^{3}+y_{i}^{3}}\right)
\end{aligned}
$$

所以

即
$$
\begin{aligned}
& \sqrt[3]{\frac{x_{1}^{3} x_{3}^{3} x_{3}^{3}}{\prod^{3}\left(x_{i}^{3}+y_{i}^{3}\right)}}+\sqrt[3]{\frac{y_{1}^{3} y_{2}^{3} y_{3}^{3}}{\Pi\left(x_{i}^{3}+y_{i}^{3}\right)}} \leqslant 1, \\
& \prod\left(x_{i}^{3}+y_{i}^{3}\right) \geqslant\left(\Pi x_{i}+\Pi y_{i}\right)^{3} .
\end{aligned}
$$

令 $x_{1}=x_{2}=a_{k}, x_{3}=a_{k+1}, a_{n+1}=a_{1}, y_{1}=y_{2}=y_{3}=1, k=1,2, \cdots$, $n$, 则
$$
\left(a_{k}^{3}+1\right)^{2}\left(a_{k+1}^{3}+1\right) \geqslant\left(a_{k}^{2} a_{k+1}+1\right)^{3}, k=1,2, \cdots, n
$$

将它们相乘,则
$$
\prod\left(a_{i}^{3}+1\right)^{3} \geqslant \prod\left(a_{i}^{2} a_{i+1}+1\right)^{3}
$$

故

$\left(a_{1}^{3}+1\right)\left(a_{2}^{3}+1\right) \cdots\left(a_{n}^{3}+1\right) \geqslant\left(a_{1}^{2} a_{2}+1\right)\left(a_{2}^{2} a_{3}+1\right) \cdots\left(a_{n}^{2} a_{1}+1\right)$.

例 21 给定整数 $n \geqslant 2$. 设 $0<a_{1} \leqslant a_{2} \leqslant \cdots \leqslant a_{n}$, 以及 $a_{1} \geqslant \frac{a_{2}}{2} \geqslant \cdots \geqslant$ $\frac{a_{n}}{n}$

求证: $\frac{A_{n}}{G_{n}} \leqslant \frac{n+1}{2 \cdot \sqrt[n]{n!}}$. 其中 $A_{1}=\frac{a_{1}+\cdots+a_{n}}{n}, G_{n}=\sqrt[n]{a_{1} \cdots a_{n}}$.

证明 由假设 $0<a_{1} \leqslant a_{2} \leqslant \cdots \leqslant a_{n}, \frac{1}{a_{1}} \leqslant \frac{2}{a_{2}} \leqslant \cdots \leqslant \frac{n}{a_{n}}$, 以及切比雪夫不等式, 得到


\begin{equation*}
\left(\frac{1}{n} \sum_{i=1}^{n} a_{i}\right)\left(\frac{1}{n} \sum_{i=1}^{n} \frac{i}{a_{i}}\right) \leqslant \frac{1}{n} \sum_{i=1}^{n} a_{i} \cdot \frac{i}{a_{i}}=\frac{n+1}{2} \tag{1}
\end{equation*}


又由平均值不等式得


\begin{equation*}
\frac{1}{n} \sum_{i=1}^{n} \frac{i}{a_{i}} \geqslant \sqrt[n]{\frac{n!}{a_{1} \cdots a_{n}}}=\frac{\sqrt[n]{n!}}{G_{n}} \tag{2}
\end{equation*}


由(1)、(2)得
$$
\frac{A_{n}}{G_{n}} \leqslant \frac{n+1}{2 \cdot \sqrt[n]{n!}}
$$

从而, 命题成立.

例 22 已知 $x, y, z \in \mathbf{R}_{+} \cup\{0\}$, 且 $x+y+z=2$.

证明: $x^{2} y^{2}+y^{2} z^{2}+z^{2} x^{2}+x y z \leqslant 1$, 并求上式取等号时, $x 、 y 、 z$ 的值.

证明 注意到


\begin{align*}
& x^{2} y^{2}+y^{2} z^{2}+z^{2} x^{2}+x y z \\
= & \frac{1}{2}\left(2 x^{2} y^{2}+2 y^{2} z^{2}+2 z^{2} x^{2}+2 x y z\right) \\
= & \frac{1}{2}(x y \cdot 2 x y+y z \cdot 2 y z+z x \cdot 2 z x+2 x y z) \\
\leqslant & \frac{1}{2}\left[x y\left(x^{2}+y^{2}\right)+y z\left(y^{2}+z^{2}\right)+z x\left(z^{2}+x^{2}\right)+2 x y z\right]  \tag{1}\\
= & \frac{1}{2}\left[(x y+y z+z x)\left(x^{2}+y^{2}+z^{2}\right)-x y z^{2}-y z x^{2}-z x y^{2}+2 x y z\right] \\
= & \frac{1}{2}\left[(x y+y z+z x)\left(x^{2}+y^{2}+z^{2}\right)-x y z(x+y+z-2)\right]
\end{align*}
$$
=\frac{1}{2}(x y+y z+z x)\left(x^{2}+y^{2}+z^{2}\right)
$$

由此得到


\begin{equation*}
x^{2} y^{2}+y^{2} z^{2}+z^{2} x^{2}+x y z \leqslant \frac{1}{2}\left[(x y+y z+z x)\left(x^{2}+y^{2}+z^{2}\right)\right] \tag{2}
\end{equation*}


由式(1)知, 当 $x=y=z$ 或 $x=y, z=0$ 或 $y=z, x=0$ 或 $z=x, y=0$时,式(2)取等号.

又 $x+y+z=2$, 因此, 当
$$
(x, y, z)=\left(\frac{2}{3}, \frac{2}{3}, \frac{2}{3}\right) \text { 或 }(1,1,0) \text { 或 }(1,0,1) \text { 或 }(0,1,1)
$$

时,式(2)取等号.

运用常见不等式
$$
\alpha \beta \leqslant\left(\frac{\alpha+\beta}{2}\right)^{2}(\alpha, \beta \in \mathbf{R})
$$

令 $\alpha=2 x y+2 y z+2 z x, \beta=x^{2}+y^{2}+z^{2}$. 则
$$
\frac{1}{2}(x y+y z+z x)\left(x^{2}+y^{2}+z^{2}\right)
$$


\begin{align*}
& =\frac{1}{4}(2 x y+2 y z+2 z x)\left(x^{2}+y^{2}+z^{2}\right) \\
& \leqslant \frac{1}{4}\left(\frac{2 x y+2 y z+2 z x+x^{2}+y^{2}+z^{2}}{2}\right)^{2} \\
& =\frac{1}{16}(x+y+z)^{4}=1 \tag{3}
\end{align*}


结合式(2)得


\begin{equation*}
x^{2} y^{2}+y^{2} z^{2}+z^{2} x^{2}+x y z \leqslant 1 \tag{4}
\end{equation*}


由式(3)取等号的条件知, 当
$$
\alpha=\beta \Leftrightarrow 2 x y+2 y z+2 z x=x^{2}+y^{2}+z^{2}
$$

时,式(4)等号成立.

故 $(x, y, z)=(1,1,0)$ 或 $(1,0,1)$ 或 $(0,1,1)$.

例 23 已知 $a 、 b 、 c$ 为正实数. 证明:
$$
\frac{a^{2} b(b-c)}{a+b}+\frac{b^{2} c(c-a)}{b+c}+\frac{c^{2} a(a-b)}{c+a} \geqslant 0
$$

\section*{证明}
$$
\begin{aligned}
\text { 原式 } & \Leftrightarrow \frac{a^{2} b^{2}}{a+b}+\frac{b^{2} c^{2}}{b+c}+\frac{c^{2} a^{2}}{c+a} \geqslant a b c\left(\frac{a}{a+b}+\frac{b}{b+c}+\frac{c}{c+a}\right) \\
& \Leftrightarrow \frac{a b}{c(a+b)}+\frac{b c}{a(b+c)}+\frac{a c}{b(c+a)} \geqslant \frac{a}{a+b}+\frac{b}{b+c}+\frac{c}{c+a} \\
& \Leftrightarrow(a b+b c+a c)\left(\frac{1}{a c+b c}+\frac{1}{a b+a c}+\frac{1}{b c+a b}\right) \\
& \geqslant \frac{a c}{a c+b c}+\frac{a b}{a b+a c}+\frac{b c}{b c+a b}+3 .
\end{aligned}
$$

下面进行换元. 令
$$
\left\{\begin{array}{l}
x=a b+a c \\
y=b c+b a \\
z=c a+c b
\end{array}, \Rightarrow\left\{\begin{array}{l}
a c=\frac{x+z-y}{2} \\
a b=\frac{x+y-z}{2}, \Rightarrow a b+b c+c a=\frac{x+y+z}{2} \\
b c=\frac{y+z-x}{2}
\end{array}\right.\right.
$$

故
$$
\begin{aligned}
\text { 原式 } & \Leftrightarrow \frac{1}{2}(x+y+z)\left(\frac{1}{x}+\frac{1}{y}+\frac{1}{z}\right) \geqslant \frac{x+z-y}{2 z}+\frac{x+y-z}{2 x}+\frac{y+z-}{2 y} \\
& \Leftrightarrow(x+y+z)\left(\frac{1}{x}+\frac{1}{y}+\frac{1}{z}\right) \geqslant \frac{x-y}{z}+\frac{y-z}{x}+\frac{z-x}{y}+9 \\
& \Leftrightarrow 3+\frac{y}{x}+\frac{z}{x}+\frac{x}{y}+\frac{z}{y}+\frac{x}{z}+\frac{y}{z} \geqslant \frac{x-y}{z}+\frac{y-z}{x}+\frac{z-x}{y}+9 \\
& \Leftrightarrow \frac{2 y}{z}+\frac{2 z}{x}+\frac{2 x}{y} \geqslant 6 \\
& \Leftrightarrow \frac{y}{z}+\frac{z}{x}+\frac{x}{y} \geqslant 3 .
\end{aligned}
$$

由均值不等式即知结论成立.

例 24 设正实数 $x_{1}, x_{2}, \cdots, x_{n}$ 满足 $x_{1} x_{2} \cdots x_{n}=1$. 证明:
$$
\sum_{i=1}^{n} \frac{1}{n-1+x_{i}} \leqslant 1
$$

证明 用反证法.

假设 $\quad \sum_{i=1}^{n} \frac{1}{n-1+x_{i}}>1$.\\
则对任意的 $k(k \in\{1,2, \cdots, n\})$, 由式(1)有
$$
\begin{aligned}
\frac{1}{n-1+x_{k}} & >1-\sum_{\substack{1 \leqslant i \leqslant n \\
i \neq k}} \frac{1}{n-1+x_{i}} \\
& =\sum_{\substack{1 \leqslant i \leqslant n \\
i \neq k}}\left(\frac{1}{n-1}-\frac{1}{n-1+x_{i}}\right) \\
& =\sum_{\substack{1 \leqslant i \leqslant n \\
i \neq k}} \frac{x_{i}}{(n-1)\left(n-1+x_{i}\right)} \\
& \geqslant(n-1)\left[\prod_{\substack{\leqslant i \leqslant n \\
i \neq k}} \frac{x_{i}}{(n-1)\left(n-1+x_{i}\right)}\right]^{\frac{1}{n-1}} \\
& =\left(\prod_{\substack{1 \leq i \leq n \\
i \neq k}} \frac{x_{i}}{n-1+x_{i}}\right)^{\frac{1}{n-1}}
\end{aligned}
$$

即对 $1 \leqslant k \leqslant n ,$ 均有
$$
\frac{1}{n-1+x_{k}}>\left(\prod_{\substack{1 \leq i \leq n \\ i \neq k}} \frac{x_{i}}{n-1+x_{i}}\right)^{\frac{1}{n-1}}
$$

取积得
$$
\prod_{k=1}^{n} \frac{1}{n-1+x_{k}}>\prod_{k=1}^{n}\left(\prod_{\substack{1 \leqslant i \leqslant n \\ i \neq k}} \frac{x_{i}}{n-1+x_{i}}\right)^{\frac{1}{n-1}}=\prod_{k=1}^{n} \frac{x_{k}}{n-1+x_{k}}
$$

则 $\prod_{k=1}^{n} x_{k}<1$, 这与 $\prod_{k=1}^{n} x_{k}=1$ 矛盾.

所以, 假设不成立,必有
$$
\sum_{i=1}^{n} \frac{1}{n-1+x_{i}} \leqslant 1
$$

例 25 设实数 $a 、 b 、 c$ 满足 $a+b+c=1, a b c>0$. 求证:
$$
a b+b c+c a<\frac{\sqrt{a b c}}{2}+\frac{1}{4}
$$

证明 由于 $a b c>0$, 则 $a 、 b 、 c$ 中或者一个正数, 两个负数; 或者三个都是正数.

如果一个为正数,两个为负数, 不妨设 $a>0, b, c<0$. 则
$$
a b+b c+c a=b(a+c)+c a=b(1-b)+c a<0<\frac{\sqrt{a b c}}{2}+\frac{1}{4}
$$

结论成立.

下面设 $a, b, c>0$, 且不妨设 $a \geqslant b \geqslant c$. 则 $a \geqslant \frac{1}{3}, 0<c \leqslant \frac{1}{3}$.

于是 $\quad a b+b c+c a-\frac{\sqrt{a b c}}{2}=c(a+b)+\sqrt{a b}\left(\sqrt{a b}-\frac{\sqrt{c}}{2}\right)$
$$
=c(1-c)+\sqrt{a b}\left(\sqrt{a b}-\frac{\sqrt{c}}{2}\right)
$$

由于 $\sqrt{a b} \geqslant \sqrt{\frac{b}{3}} \geqslant \sqrt{\frac{c}{3}}>\frac{\sqrt{c}}{2}$, 且 $\sqrt{a b} \leqslant \frac{a+b}{2}=\frac{1-c}{2}$, 所以,
$$
\begin{aligned}
& c(1-c)+\sqrt{a b}\left(\sqrt{a b}-\frac{\sqrt{c}}{2}\right) \\
\leqslant & c(1-c)+\frac{1-c}{2}\left(\frac{1-c}{2}-\frac{\sqrt{c}}{2}\right) \\
= & \frac{1}{4}-\frac{3 c^{2}}{4}+\frac{c \sqrt{c}}{4}+\frac{c}{2}-\frac{\sqrt{c}}{4}
\end{aligned}
$$

于是, 只要证明 $\frac{3 c^{2}}{4}-\frac{c \sqrt{c}}{4}-\frac{c}{2}+\frac{\sqrt{c}}{4}>0$, 即


\begin{equation*}
3 c \sqrt{c}-c-2 \sqrt{c}+1>0 \tag{1}
\end{equation*}


由于 $0<c \leqslant \frac{1}{3}$, 所以 $\frac{1}{3}-c \geqslant 0$.

又由平均值不等式


\begin{equation*}
3 c \sqrt{c}+\frac{1}{3}+\frac{1}{3} \geqslant 3\left(3 c \sqrt{c} \cdot \frac{1}{3} \cdot \frac{1}{3}\right)^{\frac{1}{3}}=\sqrt[3]{9} \sqrt{c}>2 \sqrt{c} \tag{3}
\end{equation*}


将(2)、(3)两式相加即得(1)式成立. 因此, 原不等式成立.

注 这个不等式可以看作舒尔 (Schur)不等式的一种弱形式.

舒尔(Schur)不等式 设 $x, y, z \geqslant 0, r \in \mathbf{R}_{+}$. 则
$$
x^{r}(x-y)(x-z)+y^{r}(y-z)(y-x)+z^{r}(z-x)(z-y) \geqslant 0
$$

特别, 当 $r=1$ 时, 有


\begin{equation*}
x^{3}+y^{3}+z^{3}+3 x y z \geqslant x y(x+y)+y z(y+z)+z x(z+x) \tag{3}
\end{equation*}


由 (3), 有 $x^{3}+y^{3}+z^{3}+3 x y z \geqslant 2\left(x^{3 / 2} y^{3 / 2}+y^{3 / 2} z^{3 / 2}+z^{3 / 2} x^{3 / 2}\right)$.\\
当 $a, b, c>0$ 时, 令 $x=a^{2 / 3}, y=b^{2 / 3}, z=c^{2 / 3}$, 由 (4) 式, 得到
$$
a^{2}+b^{2}+c^{2}+3(a b c)^{2 / 3} \geqslant 2(a b+b c+c a)
$$

即
$$
(a+b+c)^{2}-2(a b+c b+c a)+3(a b c)^{2 / 3} \geqslant 2(a b+b c+c a)
$$

于是


\begin{equation*}
a b+b c+c a \leqslant \frac{3}{4}(a b c)^{2 / 3}+\frac{1}{4} \tag{5}
\end{equation*}


由(5)知, 为证明原不等式, 只需证明


\begin{gather*}
3(a b c)^{2 / 3}<2 \sqrt{a b c} \\
a b c<\left(\frac{2}{3}\right)^{6} \tag{6}
\end{gather*}


由平均值不等式,$a b c \leqslant\left(\frac{a+b+c}{3}\right)^{3}=\left(\frac{1}{3}\right)^{3}<\left(\frac{2}{3}\right)^{6}$.

故不等式(6)成立.

注 关于舒尔不等式的证明和应用, 可以参考有关文献.

例 26 设 $x, y, z \in \mathbf{R}_{+}$, 求证:
$$
\frac{x y}{z}+\frac{y z}{x}+\frac{z x}{y}>2 \sqrt[3]{x^{3}+y^{3}+z^{3}}
$$

证明 欲证的不等式等价于
$$
\begin{aligned}
& \left(\frac{x y}{z}+\frac{y z}{x}+\frac{z x}{y}\right)^{3}>8\left(x^{3}+y^{3}+z^{3}\right) \\
\Leftrightarrow & \left(\frac{x y}{z}\right)^{3}+\left(\frac{y z}{x}\right)^{3}+\left(\frac{z x}{y}\right)^{3}+6 x y z+3 x^{3}\left(\frac{y}{z}+\frac{z}{y}\right) \\
& +3 y^{3}\left(\frac{x}{z}+\frac{z}{x}\right)+3 z^{3}\left(\frac{y}{x}+\frac{x}{y}\right) \\
> & 8\left(x^{3}+y^{3}+z^{3}\right)
\end{aligned}
$$

因为 $\frac{y}{z}+\frac{z}{y} \geqslant 2, \frac{x}{z}+\frac{z}{x} \geqslant 2, \frac{y}{x}+\frac{x}{y} \geqslant 2$, 所以只需证


\begin{equation*}
\left(\frac{x y}{z}\right)^{3}+\left(\frac{y z}{x}\right)^{3}+\left(\frac{z x}{y}\right)^{3}+6 x y z>2\left(x^{3}+y^{3}+z^{3}\right) \tag{1}
\end{equation*}


不妨设 $x \geqslant y \geqslant z$, 记
$$
f(x, y, z)=\left(\frac{x y}{z}\right)^{3}+\left(\frac{y z}{x}\right)^{3}+\left(\frac{z x}{y}\right)^{3}+6 x y z-2\left(x^{3}+y^{3}+z^{3}\right)
$$

下证 $f(x, y, z)-f(y, y, z) \geqslant 0, f(y, y, z) \geqslant 0$.

事实上,
$$
\begin{aligned}
& f(x, y, z)-f(y, y, z) \\
= & \left(\frac{x y}{z}\right)^{3}+\left(\frac{y z}{x}\right)^{3}+\left(\frac{z x}{y}\right)^{3}+6 x y z-2\left(x^{3}+y^{3}+z^{3}\right) \\
& -\left[\left(\frac{y^{2}}{z}\right)^{3}+z^{3}+z^{3}+6 y^{2} z-2\left(y^{3}+y^{3}+z^{3}\right)\right] \\
= & \left(\frac{x y}{z}\right)^{3}-\frac{y^{6}}{z^{3}}+\left(\frac{y z}{x}\right)^{3}+\left(\frac{z x}{y}\right)^{3}-2 z^{3}+6 y z(x-y)-2\left(x^{3}-y^{3}\right) \\
= & \left(x^{3}-y^{3}\right)\left(\frac{y^{3}}{z^{3}}+\frac{z^{3}}{y^{3}}-2+\frac{6 y z}{x^{2}+x y+y^{2}}-\frac{z^{3}}{x^{3}}\right),
\end{aligned}
$$

而 $x^{3}-y^{3} \geqslant 0, \frac{y^{3}}{z^{3}}+\frac{z^{3}}{y^{3}} \geqslant 2$,
$$
\frac{6 y z}{x^{2}+x y+y^{2}}-\frac{z^{3}}{x^{3}} \geqslant \frac{2 y z}{x^{2}}-\frac{z^{3}}{x^{3}}=\frac{z\left(2 x y-z^{2}\right)}{x^{3}}>0
$$

所以 $f(x, y, z)-f(y, y, z) \geqslant 0$.

又
$$
\begin{aligned}
f(y, y, z) & =\left(\frac{y^{2}}{z}\right)^{3}+z^{3}+z^{3}+6 y^{2} z-2\left(y^{3}+y^{3}+z^{3}\right) \\
& =\frac{y^{6}}{z^{3}}+2 y^{2} z+2 y^{2} z+2 y^{2} z-4 y^{3} \\
& \geqslant 4 \sqrt[4]{2^{3} y^{12}}-4 y^{3}=4(\sqrt[4]{8}-1) y^{3}>0
\end{aligned}
$$

从而(1)式得证,原命题得证.

注 本题所用的方法叫调整法,在各级竞赛中偶尔出现时, 因难度大而得分率极低. 此题的解答由第 50 届 IMO 金牌获得者郑志伟给出.

例 27 设 $x 、 y 、 z$ 为非负实数, 且 $x+y+z=1$, 求证:
$$
x y+y z+z x-2 x y z \leqslant \frac{7}{27}
$$

证明 不妨设 $x \geqslant y \geqslant z$.

当 $x \geqslant \frac{1}{2}$ 时, 则 $y z-2 x y z \leqslant 0$, 所以
$$
x y+y z+z x-2 x y z \leqslant x y+z x=x(1-x) \leqslant \frac{1}{4}<\frac{7}{27}
$$

当 $x<\frac{1}{2}$ 时, 则 $y \leqslant \frac{1}{2}, z \leqslant \frac{1}{2}$.
$$
(1-2 x)(1-2 y)(1-2 z)=1-2+4(x y+y z+z x)-8 x y z
$$

又由平均值不等式, 得
$$
(1-2 x)(1-2 y)(1-2 z) \leqslant\left[\frac{3-2(x+y+z)}{3}\right]^{3}=\frac{1}{27}
$$

从而
$$
x y+y z+z x-2 x y z \leqslant \frac{1}{4}\left(\frac{1}{27}+1\right)=\frac{7}{27}
$$

例 28 设 $n$ 为正整数, $\left(x_{1}, x_{2}, \cdots, x_{n}\right),\left(y_{1}, y_{2}, \cdots, y_{n}\right)$ 为两个正数数列. 假设正实数列 $\left(z_{1}, z_{2}, \cdots, z_{2 n}\right)$, 满足
$$
z_{i+j}^{2} \geqslant x_{i} y_{j}, 1 \leqslant i, j \leqslant n
$$

令 $M=\max \left\{z_{1}, z_{2}, \cdots, z_{2 n}\right\}$, 证明:
$$
\left(\frac{M+z_{2}+z_{3}+\cdots+z_{2 n}}{2 n}\right)^{2} \geqslant\left(\frac{x_{1}+x_{2}+\cdots+x_{n}}{n}\right)\left(\frac{y_{1}+y_{2}+\cdots+y_{n}}{n}\right)
$$

证明 $\quad$ 令 $X=\max \left\{x_{1}, x_{2}, \cdots, x_{n}\right\}, Y=\max \left\{y_{1}, y_{2}, \cdots, y_{n}\right\}$, 不妨假设 $X=Y=1$ (否则用 $a_{i}=\frac{x_{i}}{X}, b_{i}=\frac{y_{i}}{Y}, c_{i}=\frac{z_{i}}{\sqrt{X Y}}$ 代替 $)$.

我们将证明
$$
M+z_{2}+z_{3}+\cdots+z_{2 n} \geqslant x_{1}+x_{2}+\cdots+x_{n}+y_{1}+y_{2}+\cdots+y_{n}
$$

于是
$$
\frac{M+z_{2}+z_{3}+\cdots+z_{2 n}}{2 n} \geqslant \frac{1}{2}\left(\frac{x_{1}+x_{2}+\cdots+x_{n}}{n}+\frac{y_{1}+y_{2}+\cdots+y_{n}}{n}\right)
$$

由平均值不等式得到原不等式成立.

为了证明上述不等式, 我们将证明: 对任意 $r>0$, 左边大于 $r$ 的项数不小于右边相应的项数, 那么, 对每个 $k$, 左边第 $k$ 个最大的项大于或等于右边第 $k$个最大的项, 这样就证明了上述不等式成立. 证明如下:

如果 $r \geqslant 1$, 则右边没有项大于 $r$, 所以只考虑 $r<1$.

令 $A=\left\{x_{i} \mid x_{i}>r, 1 \leqslant i \leqslant n\right\}, a=|A|, B=\left\{y_{i} \mid y_{i}>r, 1 \leqslant i \leqslant\right.$ $n\}, b=|B|$. 由于 $X=Y=1$, 所以 $a, b$ 大于 0 .\\
由于 $x_{i}>r, y_{j}>r$ 推出 $z_{i+j} \geqslant \sqrt{x_{i} y_{j}}>r$. 于是
$$
A+B=\{\alpha+\beta \mid \alpha \in A, \beta \in B\} \subseteq C=\left\{z_{i} \mid z_{i}>r, 2 \leqslant i \leqslant 2 n\right\}
$$

但是, 由于如果 $A=\left\{i_{1}, i_{2}, \cdots, i_{a}\right\}, i_{1}<i_{2}<\cdots<i_{a}, B=\left\{j_{1}, j_{2}, \cdots\right.$, $\left.j_{b}\right\}, j_{1}<j_{2}<\cdots<j_{b}$, 则 $a+b-1$ 个数 $i_{1}+j_{1}, i_{1}+j_{2}, \cdots, i_{1}+j_{b}, i_{2}+j_{b}, \cdots$, $i_{a}+j_{b}$ 互不相同, 且属于 $A+B$. 所以 $|A+B| \geqslant|A|+|B|-1$. 因此 $|C| \geqslant$ $a+b-1$. 特别, $|C| \geqslant 1$, 于是对某个 $k, z_{k}>r$, 那么 $M>r$. 所以上式的左边至少有 $a+b$ 项大于 $r$, 由于 $a+b$ 为右边大于 $r$ 的项数, 于是, 上述不等式成立.

注 在证明的过程中, 其实平均值不等式的作用是较小的. 本题的证明有一定的难度, 是因为像这样证明不等式的方法和处理技巧并不多见. 此解答由第 51 届 IMO 金牌获得者李嘉伦给出.

例 29 已知 $a \geqslant b \geqslant c>0$, 求证: $\frac{b}{a}+\frac{c}{b}+\frac{a}{c}+a b c \geqslant a+b+c+1$.

\section*{证明 原不等式等价于}
$$
b^{2} c+c^{2} a+a^{2} b+a^{2} b^{2} c^{2} \geqslant a^{2} b c+a b^{2} c+a b c^{2}+a b c
$$

而
$$
\begin{aligned}
& \frac{a^{2} b^{2} c^{2}+a^{2} b+a^{2} c}{3} \geqslant a^{2} b c \\
& \frac{a^{2} b^{2} c^{2}+b^{2} a+b^{2} c}{3} \geqslant a b^{2} c \\
& \frac{a^{2} b^{2} c^{2}+c^{2} a+c^{2} b}{3} \geqslant a b c^{2}
\end{aligned}
$$

三式相加得
$$
a^{2} b^{2} c^{2}+\frac{1}{3}\left(a^{2} b+a^{2} c+b^{2} a+b^{2} c+c^{2} a+c^{2} b\right) \geqslant a^{2} b c+a b^{2} c+a b c^{2}
$$

只需证明
$$
\frac{2}{3}\left(a^{2} b+b^{2} c+c^{2} a\right) \geqslant \frac{1}{3}\left(a b^{2}+b c^{2}+c a^{2}\right)+a b c
$$

而
$$
\frac{1}{3}\left(a^{2} b+b^{2} c+c^{2} a\right) \geqslant a b c
$$

只需证明
$$
a^{2} b+b^{2} c+c^{2} a \geqslant a b^{2}+b c^{2}+c a^{2}
$$

而 $a^{2} b+b^{2} c+c^{2} a-\left(a b^{2}+b c^{2}+c a^{2}\right)=(a-b)(b-c)(a-c) \geqslant 0$,得证.

\section*{2. 2 平均值不等式在求极值中的应用}
不等式在求极值中起着重要的作用,在利用平均值不等式求极值的过程中, 要注意“缩”或“放”的结果是否为常数 (通常是和与积), 同时必须指出等号成立的条件.

例 1 设 $a 、 b 、 c$ 为正实数, 求
$$
\frac{a+3 c}{a+2 b+c}+\frac{4 b}{a+b+2 c}-\frac{8 c}{a+b+3 c}
$$

的最小值.

解法一 令 $x=a+2 b+c, y=a+b+2 c, z=a+b+3 c$, 则有 $x-$ $y=b-c, z-y=c$, 由此可得 $a+3 c=2 y-x, b=z+x-2 y, c=z-$ $y$, 从而
$$
\begin{aligned}
& \frac{a+3 c}{a+2 b+c}+\frac{4 b}{a+b+2 c}-\frac{8 c}{a+b+3 c} \\
= & \frac{2 y-x}{x}+\frac{4(z+x-2 y)}{y}-\frac{8(z-y)}{z} \\
= & -17+2 \frac{y}{x}+4 \frac{x}{y}+4 \frac{z}{y}+8 \frac{y}{z} \\
\geqslant & -17+2 \sqrt{8}+2 \sqrt{32}=-17+12 \sqrt{2} .
\end{aligned}
$$

取 $a=3-2 \sqrt{2}, b=\sqrt{2}-1, c=\sqrt{2}$ 时, 等号成立.

故最小值为 $-17+12 \sqrt{2}$.

解法二 不妨设 $a+b+c=1$, 则
$$
\begin{aligned}
& \frac{a+3 c}{a+2 b+c}+\frac{4 b}{a+b+2 c}-\frac{8 c}{a+b+3 c} \\
= & \frac{1+2 c-b}{1+b}+\frac{4 b}{1+c}-\frac{8 c}{1+2 c} \\
= & -1+\frac{2+2 c}{1+b}+\frac{4 b+4}{1+c}-\frac{4}{1+c}+\frac{4}{1+2 c}-4 \\
= & -5+2 \frac{1+c}{1+b}+4 \frac{1+b}{1+c}-\frac{4 c}{(1+c)(1+2 c)}
\end{aligned}
$$
$$
\begin{aligned}
& \geqslant-5+2 \sqrt{8}-\frac{4}{\frac{1}{c}+3+2 c} \\
& \geqslant-5+4 \sqrt{2}-\frac{4}{3+2 \sqrt{2}} \\
& =12 \sqrt{2}-17 .
\end{aligned}
$$

取 $a=3-2 \sqrt{2}, b=\sqrt{2}-1, c=\sqrt{2}$ 时, 等号成立.

故最小值为 $-17+12 \sqrt{2}$.

例 2 设非负实数 $a$ 和 $d$, 正数 $b$ 和 $c$, 满足条件 $b+c \geqslant a+d$, 求 $\frac{b}{c+d}+$ $\frac{c}{a+b}$ 的最小值.

解 不妨设 $a+b \geqslant c+d$. 因为
$$
\frac{b}{c+d}+\frac{c}{a+b}=\frac{b+c}{c+d}-c\left(\frac{1}{c+d}-\frac{1}{a+b}\right),
$$

注意到 $c \leqslant c+d$ 及 $b+c \geqslant a+d \Leftrightarrow b+c \geqslant \frac{1}{2}(a+b+c+d)$. 因此, 得
$$
\begin{aligned}
\frac{b}{c+d}+\frac{c}{a+b} & \geqslant \frac{1}{2} \frac{a+b+c+d}{c+d}-(c+d)\left(\frac{1}{c+d}-\frac{1}{a+b}\right) \\
& =\frac{1}{2} \frac{a+b}{c+d}+\frac{c+d}{a+b}-\frac{1}{2} \geqslant 2 \sqrt{\frac{a+b}{2(c+d)} \frac{c+d}{a+b}}-\frac{1}{2} \\
& =\sqrt{2}-\frac{1}{2}
\end{aligned}
$$

等号成立当且仅当 $a=\sqrt{2}+1, b=\sqrt{2}-1, c=2, d=0$, 所以 $\frac{b}{c+d}+\frac{c}{a+b}$的最小值为 $\sqrt{2}-\frac{1}{2}$.

例 3 设 $2 x>3 y>0$, 求 $\sqrt{2} x^{3}+\frac{3}{2 x y-3 y^{2}}$ 的最小值.

解 因为 $2 x>3 y>0$, 所以 $2 x-3 y>0$. 由平均值不等式, 得
$$
\begin{aligned}
2 x y-3 y^{2} & =y(2 x-3 y)=\frac{1}{3} \cdot 3 y(2 x-3 y) \\
& \leqslant \frac{1}{3} \cdot\left[\frac{3 y+(2 x-3 y)}{2}\right]^{2}=\frac{1}{3} x^{2}
\end{aligned}
$$

所以
$$
\begin{aligned}
& \sqrt{2} x^{3}+\frac{3}{2 x y-3 y^{2}} \geqslant \sqrt{2} x^{3}+\frac{9}{x^{2}} \\
= & \frac{\sqrt{2}}{2} x^{3}+\frac{\sqrt{2}}{2} x^{3}+\frac{3}{x^{2}}+\frac{3}{x^{2}}+\frac{3}{x^{2}} \\
\geqslant & 5 \sqrt[5]{\left(\frac{\sqrt{2}}{2} x^{3}\right)^{2} \cdot\left(\frac{3}{x^{2}}\right)^{3}}=5 \sqrt[5]{\frac{27}{2}}
\end{aligned}
$$

等号成立当且仅当 $3 y=2 x-3 y, \frac{\sqrt{2}}{2} x^{3}=\frac{3}{x^{2}}$, 即 $x=18^{\frac{1}{10}}, y=\frac{1}{3} \cdot 18^{\frac{1}{10}}$ 时取到.

因此, $\sqrt{2} x^{3}+\frac{3}{2 x y-3 y^{2}}$ 的最小值为 $5 \sqrt[5]{\frac{27}{2}}$.

例 4 若 $x 、 y 、 z$ 是正实数, 求 $\frac{x y z}{(1+5 x)(4 x+3 y)(5 y+6 z)(z+18)}$ 的最大值, 并证明你的结论.

解 在取定 $y$ 的情况下,
$$
\begin{aligned}
& \frac{x}{(1+5 x)(4 x+3 y)} \\
= & \frac{x}{20 x^{2}+(15 y+4) x+3 y} \\
= & \frac{1}{20 x+\frac{3 y}{x}+15 y+4} \\
\leqslant & \frac{1}{2 \sqrt{20 \times 3 y}+15 y+4} \\
= & \frac{1}{(\sqrt{15 y}+2)^{2}}
\end{aligned}
$$

当且仅当 $x=\sqrt{\frac{3 y}{20}}$ 时, 等号成立.

同理可得,
$$
\frac{z}{(5 y+6 z)(z+18)} \leqslant \frac{1}{2 \sqrt{6 \times 90 y}+5 y+108}=\frac{1}{(\sqrt{5 y}+6 \sqrt{3})^{2}}
$$

当且仅当 $z=\sqrt{15 y}$ 时, 等号成立. 所以,
$$
\frac{x y z}{(1+5 x)(4 x+3 y)(5 y+6 z)(z+18)}
$$
$$
\begin{aligned}
& \leqslant \frac{y}{(\sqrt{15 y}+2)^{2}(\sqrt{5 y}+6 \sqrt{3})^{2}} \\
& =\left[\frac{\sqrt{y}}{(\sqrt{15 y}+2)(\sqrt{5 y}+6 \sqrt{3})}\right]^{2} \\
& =\left[\frac{1}{5 \sqrt{3 y}+\frac{12 \sqrt{3}}{\sqrt{y}}+20 \sqrt{5}}\right]^{2} \\
& \leqslant\left[\frac{1}{2 \sqrt{5 \sqrt{3} \times 12 \sqrt{3}}+20 \sqrt{5}}\right]^{2} \\
& =\left(\frac{1}{32 \sqrt{5}}\right)^{2}=\frac{1}{5120}
\end{aligned}
$$

当且仅当 $x=\frac{3}{5}, y=\frac{12}{5}, z=6$ 时,上式取得最大值 $\frac{1}{5120}$.

例 5 若对于任何正实数, $\frac{a^{2}}{\sqrt{a^{4}+3 b^{4}+3 c^{4}}}+\frac{k}{a^{3}} \cdot\left(\frac{c^{4}}{b}+\frac{b^{4}}{c}\right) \geqslant \frac{2 \sqrt{2}}{3}$ 均成立, 求实数 $k$ 的最小值.

解 $\frac{a^{2}}{\sqrt{a^{4}+3 b^{4}+3 c^{4}}}=\frac{\sqrt{2} a^{4}}{\sqrt{2 a^{4}\left(a^{4}+3 b^{4}+3 c^{4}\right)}}$
$$
\begin{aligned}
& \geqslant \frac{\sqrt{2} a^{4}}{\frac{1}{2}\left[2 a^{4}+\left(a^{4}+3 b^{4}+3 c^{4}\right)\right]} \\
& =\frac{2 \sqrt{2}}{3} \cdot \frac{a^{4}}{a^{4}+b^{4}+c^{4}}
\end{aligned}
$$

从形式上猜测, 须证明 $\frac{k}{a^{3}} \cdot\left(\frac{c^{4}}{b}+\frac{b^{4}}{c}\right) \geqslant \frac{2 \sqrt{2}}{3} \cdot \frac{b^{4}+c^{4}}{a^{4}+b^{4}+c^{4}}$, 又从等号成立的条件 $2 a^{4}=a^{4}+3 b^{4}+3 c^{4}$ 以及 $b, c$ 的对称性, 猜测 $k$ 可能在 $a^{4}=6 b^{4}=$ $6 c^{4}$ 时取到尽可能大的值, 而该值即为使不等式对任意 $a 、 b 、 c$ 成立的最小值.

令 $a=\sqrt[4]{6}, b=c=1$, 知 $\frac{\sqrt{2}}{2}+\frac{k}{(\sqrt[4]{6})^{3}} \cdot 2 \geqslant \frac{2 \sqrt{2}}{3} \Rightarrow k \geqslant \frac{1}{\sqrt[4]{24}}$.

下面证明当 $k=\frac{1}{\sqrt[4]{24}}$ 时, $\frac{k}{a^{3}} \cdot\left(\frac{c^{4}}{b}+\frac{b^{4}}{c}\right) \geqslant \frac{2 \sqrt{2}}{3} \cdot \frac{b^{4}+c^{4}}{a^{4}+b^{4}+c^{4}}$.

等价于证明 $\left(a^{4}+b^{4}+c^{4}\right)\left(b^{5}+c^{5}\right) \geqslant \frac{4 \sqrt[4]{6}}{3} a^{3} b c\left(b^{4}+c^{4}\right)$.

因为 $\left(b^{9}+c^{9}\right)-\left(b^{5} c^{4}+b^{4} c^{5}\right)=\left(b^{5}-c^{5}\right)\left(b^{4}-c^{4}\right) \geqslant 0$, 所以
$$
b^{9}+c^{9} \geqslant b^{5} c^{4}+b^{4} c^{5}
$$

由加权平均值不等式, 得:
$$
\begin{aligned}
a^{4} b^{5}+2 b^{5} c^{4} & =6 \cdot \frac{a^{4} b^{5}}{6}+2 \cdot b^{5} c^{4} \\
& \geqslant 8 \sqrt[8]{\left(\frac{a^{4} b^{5}}{6}\right)^{6} \cdot\left(b^{5} c^{4}\right)^{2}} \\
& =\frac{4 \sqrt[4]{6}}{3} a^{3} b^{5} c \\
a^{4} c^{5}+2 c^{5} b^{4} & =6 \cdot \frac{a^{4} c^{5}}{6}+2 \cdot c^{5} b^{4} \\
& \geqslant 8 \sqrt[8]{\left(\frac{a^{4} c^{5}}{6}\right)^{6} \cdot\left(c^{5} b^{4}\right)^{2}} \\
& =\frac{4 \sqrt[4]{6}}{3} a^{3} b c^{5} .
\end{aligned}
$$

三式相加, 整理后即得
$$
\left(a^{4}+b^{4}+c^{4}\right)\left(b^{5}+c^{5}\right) \geqslant \frac{4 \sqrt[4]{6}}{3} a^{3} b c\left(b^{4}+c^{4}\right)
$$

故原左式 $\geqslant \frac{2 \sqrt{2}}{3} \cdot \frac{a^{4}}{a^{4}+b^{4}+c^{4}}+\frac{2 \sqrt{2}}{3} \cdot \frac{b^{4}+c^{4}}{a^{4}+b^{4}+c^{4}}=\frac{2 \sqrt{2}}{3}=$ 原右式, 即所求 $k$ 的最小值为 $\frac{1}{\sqrt[4]{24}}$.

例 6 已知两两不同的正整数 $a 、 b 、 c 、 d 、 e 、 f 、 g 、 h 、 n$ 满足
$$
n=a b+c d=e f+g h
$$

求 $n$ 的最小值.

解 若 $a 、 b 、 c 、 d 、 e 、 f 、 g 、 h$ 中没有一个等于 1 , 则
$$
\begin{aligned}
2 n & =a b+c d+e f+g h \\
& \geqslant 4 \sqrt[4]{a b c d e f g h} \\
& \geqslant 4 \sqrt[4]{2 \times 3 \times 4 \times 5 \times 6 \times 7 \times 8 \times 9} \\
& =4 \sqrt[4]{2^{7} \times 3^{4} \times 5 \times 7} \\
& =4 \times 4 \times 3 \times \sqrt[4]{\frac{35}{2}} \\
& >4 \times 4 \times 3 \times 2=96
\end{aligned}
$$

所以 $n \geqslant 48$. 设 $a 、 b 、 c 、 d 、 e 、 f 、 g 、 h$ 中有一个等于 1 , 不妨设 $h=1$,则 $2 n=a b+c d+e f+g$, 且存在最小值. 此时 $g$ 一定是这些数中最大的一个.于是有
$$
\begin{aligned}
2 n & =a b+c d+e f+g \\
& \geqslant g+3 \sqrt[3]{a b c d e f} \\
& \geqslant g+3 \sqrt[3]{2 \times 3 \times 4 \times 5 \times 6 \times 7} \\
& =g+3 \sqrt[3]{5040} \\
& >g+3 \sqrt[3]{4913}=g+51
\end{aligned}
$$

因为 $g \geqslant 8$, 所以, $2 n \geqslant 60, n \geqslant 30$.

如果 $g \geqslant 9$, 则 $2 n \geqslant 61, n \geqslant 31$.

假设 $n=30$, 则 $g=8$. 于是, $a 、 b 、 c 、 d 、 e 、 f 、 g 、 h$ 是集合 $\{1,2,3,4$, $5,6,7,8\}$ 的一个排列, 特别地, 有一个数是 5 , 不妨设 $a=5$. 所以 $30=a b+$ $c d$, 即 $c d$ 可以被 5 整除. 矛盾.

因此 $n \geqslant 31$. 又因为 $31=1 \times 7+4 \times 6=2 \times 8+3 \times 5$, 因此 $n$ 的最小值为 31 .

例 7 (1) 如果 $a 、 b 、 c 、 d$ 是实数, 求证:
$$
a^{6}+b^{6}+c^{6}+d^{6}-6 a b c d \geqslant-2
$$

并指出等号何时成立;

(2) 对于哪些正整数 $k$, 不等式
$$
a^{k}+b^{k}+c^{k}+d^{k}-k a b c d \geqslant M_{k}
$$

对所有实数 $a 、 b 、 c 、 d$ 成立? 求 $M_{k}$ 的最大可能值, 并指出等号何时成立.

证明 (1)给定不等式变形为
$$
a^{6}+b^{6}+c^{6}+d^{6}+1+1 \geqslant 6 a b c d
$$

根据算术一几何平均值不等式, 得
$$
\begin{aligned}
& \frac{a^{6}+b^{6}+c^{6}+d^{6}+1^{6}+1^{6}}{6} \\
\geqslant & \sqrt[6]{|a|^{6} \cdot|b|^{6} \cdot|c|^{6} \cdot|d|^{6} \cdot 1^{6} \cdot 1^{6}} \\
= & |a b c d| \geqslant a b c d
\end{aligned}
$$

因为算术一几何平均值不等式当
$$
|a|=|b|=|c|=|d|=1
$$

时等号成立. 而最后的不等式, 当偶数个变量为负时等号成立. 因此, 等号成立的情形是
$$
\begin{aligned}
(a, b, c, d)= & (1,1,1,1),(1,1,-1,-1),(1,-1,1,-1) \\
& (-1,1,1,-1),(1,-1,-1,1),(-1,1,-1,1) \\
& (-1,-1,1,1),(-1,-1,-1,-1)
\end{aligned}
$$

之一;

(2)注意到,当 $k$ 是奇数时,因为绝对值足够大的负值 $a 、 b 、 c 、 d$ 的选取得出了绝对值足够大的负值 $a^{k}+b^{k}+c^{k}+d^{k}-k a b c d$. 因此 $M_{k}$ 这样的数不存在.

当 $k=2$ 时,取 $a=b=c=d=r$, 得到
$$
a^{2}+b^{2}+c^{2}+d^{2}-2 a b c d=4 r^{2}-2 r^{4}
$$

对足够大的正数 $r$ 的选取也得出绝对值任意大的负值. 因此, $M_{k}$ 这样的数不存在.

当 $k$ 是偶数, 且 $k \geqslant 4$ 时, 取 $a=b=c=d=1$, 得
$$
a^{k}+b^{k}+c^{k}+d^{k}-k a b c d=4-k
$$

同(1)得
$$
a^{k}+b^{k}+c^{k}+d^{k}-k a b c d \geqslant 4-k
$$

即 $\frac{a^{k}+b^{k}+c^{k}+d^{k}+(k-4) \cdot 1^{k}}{k} \geqslant a b c d$.

等号成立的条件与 $(1)$ 相同.

故此时 $M_{k}$ 的最大值为 $4-k$.

例 8 设 $a, b, c \in \mathbf{R}_{+}$, 满足 $a+b+c=a b c$. 求 $a^{7}(b c-1)+$ $b^{7}(a c-1)+c^{7}(a b-1)$ 的最小值.

解 因为 $a, b, c>0$, 且 $a+b+c=a b c$, 所以 $c(a b-1)=a+b$.

同理可得 $b(a c-1)=a+c, a(b c-1)=b+c$.

由平均值不等式, 得
$$
a b c=a+b+c \geqslant 3 \sqrt[3]{a b c}
$$

推出 $a b c \geqslant 3 \sqrt{3}$, 等号成立当且仅当 $a=b=c=\sqrt{3}$, 所以
$$
\begin{aligned}
& a^{7}(b c-1)+b^{7}(a c-1)+c^{7}(a b-1) \\
= & a^{6}(b+c)+b^{6}(a+c)+c^{6}(a+b)
\end{aligned}
$$
$$
\begin{aligned}
& \geqslant 6 \sqrt[6]{a^{6} b a^{6} c b^{6} a b^{6} c c^{6} a c^{6} b}=6 \sqrt[6]{a^{14} b^{14} c^{14}}=6(a b c)^{\frac{7}{3}} \\
& \geqslant 6(\sqrt{3})^{7}=6 \times 27 \sqrt{3}=162 \sqrt{3}
\end{aligned}
$$

等号成立的充要条件是 $a=b=c=\sqrt{3}$, 故所求的最小值为 $162 \sqrt{3}$.

例 9 对满足 $a b c=1$ 的正实数 $a 、 b 、 c$, 求
$$
\left(a-1+\frac{1}{b}\right)\left(b-1+\frac{1}{c}\right)\left(c-1+\frac{1}{a}\right)
$$

的最大值.

解 由于表达式关于 $a 、 b 、 c$ 是对称的, 当 $a=b=c=1$ 时, 得
$$
\left(a-1+\frac{1}{b}\right)\left(b-1+\frac{1}{c}\right)\left(c-1+\frac{1}{a}\right)=1
$$

下面我们证明最大值为 1 , 即证明对任意满足 $a b c=1$ 的实数 $a 、 b 、 c$, 有
$$
\left(a-1+\frac{1}{b}\right)\left(b-1+\frac{1}{c}\right)\left(c-1+\frac{1}{a}\right) \leqslant 1
$$

首先,我们将非齐次的式子转换为齐次式, 即对正实数 $x 、 y 、 z$, 令 $a=$ $\frac{x}{y}, b=\frac{y}{z}, c=\frac{z}{x}$ (例如 $x=1, y=\frac{1}{a}, z=\frac{1}{a b}$ ), 则上式等价于证明
$$
(x-y+z)(y-z+x)(z-x+y) \leqslant x y z .
$$

令 $u=x-y+z, v=y-z+x, w=z-x+y$, 由于 $u 、 v 、 w$ 的任意两个之和为正, 所以它们中最多有一个为负, 所以不妨假设 $u \geqslant 0, v \geqslant 0$, 由平均值不等式, 得
$$
\begin{aligned}
\sqrt{u v} & =\sqrt{(x-y+z)(y-z+x)} \\
& \leqslant \frac{1}{2}[(x-y+z)+(y-z+x)]=x
\end{aligned}
$$

同理, $\sqrt{v w} \leqslant y, \sqrt{w u} \leqslant z$, 故 $u v w \leqslant x y z$.

例 10 设 $a 、 b 、 c$ 为正实数, 满足
$$
a+b+c+3 \sqrt[3]{a b c} \geqslant k(\sqrt{a b}+\sqrt{b c}+\sqrt{c a})
$$

求 $k$ 的最大值.

解 由于当 $a=b=c$ 时, 由 $6 \geqslant 3 k$, 得 $k \leqslant 2$. 下面证明
$$
a+b+c+3 \sqrt[3]{a b c} \geqslant 2(\sqrt{a b}+\sqrt{b c}+\sqrt{c a})
$$

令 $f(a, b, c)=a+b+c+3 \sqrt[3]{a b c}-2(\sqrt{a b}+\sqrt{b c}+\sqrt{c a})$.不妨假设 $a \leqslant b \leqslant c$, 作如下调整,
$$
a=a^{\prime}, b^{\prime}=c^{\prime}=\sqrt{b^{\prime} c^{\prime}}=A
$$

则
$$
\begin{aligned}
f\left(a^{\prime}, b^{\prime}, c^{\prime}\right) & =\left(a+2 A+3 \cdot a^{\frac{1}{3}} \cdot A^{\frac{2}{3}}\right)-2\left[A+2(a A)^{\frac{1}{2}}\right] \\
& =a+3 \cdot a^{\frac{1}{3}} \cdot A^{\frac{2}{3}}-4(a A)^{\frac{1}{2}} \geqslant 0
\end{aligned}
$$

等号成立当且仅当 $a=0$ 或 $a=b=c$.

再证明 $f(a, b, c) \geqslant f\left(a^{\prime}, b^{\prime}, c^{\prime}\right)$. 因为
$$
f(a, b, c)-f\left(a^{\prime}, b^{\prime}, c^{\prime}\right)=b+c-2 a^{\frac{1}{2}}\left(b^{\frac{1}{2}}+c^{\frac{1}{2}}-2 A^{\frac{1}{2}}\right)-2 A
$$

由于 $a \leqslant A, b^{\frac{1}{2}}+c^{\frac{1}{2}} \geqslant 2 A^{\frac{1}{2}}$, 所以
$$
\begin{aligned}
& f(a, b, c)-f\left(a^{\prime}, b^{\prime}, c^{\prime}\right) \\
\geqslant & b+c-2 \sqrt{A}(\sqrt{b}+\sqrt{c}-2 \sqrt{A})-2 A \\
= & b+c-2(\sqrt{b}+\sqrt{c}) \sqrt{A}+2 A \\
= & (\sqrt{b}-\sqrt{A})^{2}+(\sqrt{c}-\sqrt{A})^{2} \\
\geqslant & 0 .
\end{aligned}
$$

从而 $f(a, b, c) \geqslant 0$, 故 $k$ 的最大值为 2 .

注 此解答由第 48 届 IMO 金牌选手付雷给出.

例 11 对 $a, b, c \in \mathbf{R}_{+}$, 求
$$
\frac{(a+b)^{2}+(a+b+4 c)^{2}}{a b c}(a+b+c)
$$

的最小值.

解 由平均值不等式, 得
$$
\begin{aligned}
(a+b)^{2}+(a+b+4 c)^{2} & =(a+b)^{2}+[(a+2 c)+(b+2 c)]^{2} \\
& \geqslant(2 \sqrt{a b})^{2}+(2 \sqrt{2 a c}+2 \sqrt{2 b c})^{2} \\
& =4 a b+8 a c+8 b c+16 c \sqrt{a b}
\end{aligned}
$$

于是
$$
\begin{aligned}
& \frac{(a+b)^{2}+(a+b+4 c)^{2}}{a b c} \cdot(a+b+c) \\
\geqslant & \frac{4 a b+8 a c+8 b c+16 c \sqrt{a b}}{a b c} \cdot(a+b+c) \\
= & \left(\frac{4}{c}+\frac{8}{b}+\frac{8}{a}+\frac{16}{\sqrt{a b}}\right)(a+b+c)
\end{aligned}
$$
$$
\begin{aligned}
& =8\left(\frac{1}{2 c}+\frac{1}{b}+\frac{1}{a}+\frac{1}{\sqrt{a b}}+\frac{1}{\sqrt{a b}}\right)\left(\frac{a}{2}+\frac{a}{2}+\frac{b}{2}+\frac{b}{2}+c\right) \\
& \geqslant 8\left(5 \sqrt[5]{\frac{1}{2 a^{2} b^{2} c}}\right)\left(5 \sqrt[5]{\frac{a^{2} b^{2} c}{2^{4}}}\right)=100
\end{aligned}
$$

当 $a=b=2 c>0$ 时取等号. 故所求最小值为 100 .

例 12 设 $a_{i} \in \mathbf{R}, 1 \leqslant i \leqslant 2016$. 满足 $9 a_{i}>11 a_{i+1}^{2}, 1 \leqslant i \leqslant 2015$.

求 $\left(a_{1}-a_{2}^{2}\right)\left(a_{2}-a_{3}^{2}\right) \cdots\left(a_{2015}-a_{2016}^{2}\right)\left(a_{2016}-a_{1}^{2}\right)$ 的最大值.

解 令 $S=\left(a_{1}-a_{2}^{2}\right)\left(a_{2}-a_{3}^{2}\right) \cdots\left(a_{2015}-a_{2016}^{2}\right)\left(a_{2016}-a_{1}^{2}\right)$. 由假设, $a_{i}-$ $a_{i+1}^{2}>\frac{11}{9} a_{i+1}^{2}-a_{i+1}^{2} \geqslant 0,1 \leqslant i \leqslant 2015$.

若 $a_{2016}-a_{1}^{2} \leqslant 0$, 则 $S \leqslant 0$.

以下考虑 $a_{2016}-a_{1}^{2}>0$ 的情况. 约定 $a_{2017}=a_{1}$. 由平均值不等式得
$$
\begin{aligned}
S^{\frac{1}{2016}} & \leqslant \frac{1}{2016} \sum_{i=1}^{2016}\left(a_{i}-a_{i+1}^{2}\right) \\
& =\frac{1}{2016}\left(\sum_{i=1}^{2016} a_{i}-\sum_{i=1}^{2016} a_{i+1}^{2}\right) \\
& =\frac{1}{2016}\left(\sum_{i=1}^{2016} a_{i}-\sum_{i=1}^{2016} a_{i}^{2}\right)=\frac{1}{2016} \sum_{i=1}^{2016} a_{i}\left(1-a_{i}\right) \\
& \leqslant \frac{1}{2016} \sum_{i=1}^{2016}\left(\frac{a_{i}+\left(1-a_{i}\right)}{2}\right)^{2} \\
& =\frac{1}{2016} \cdot 2016 \cdot \frac{1}{4}=\frac{1}{4} .
\end{aligned}
$$

于是, $S \leqslant \frac{1}{4^{2016}}$.

当 $a_{1}=a_{2}=\cdots=a_{2016}=\frac{1}{2}$ 时, 上述不等式等号成立, 且有 $9 a_{i}>11 a_{i+1}^{2}$, $1 \leqslant i \leqslant 2015$, 此时,$S=\frac{1}{4^{2016}}$.

故, 所求最大值为 $\frac{1}{4^{2016}}$.

例 13 设 $0<x_{n}<x_{n-1}<\cdots<x_{1}, y_{i} \leqslant 1,1 \leqslant i \leqslant n$

以及 $\quad x_{1} \cdots x_{i} \leqslant y_{1} \cdots y_{i}, 1 \leqslant i \leqslant n$.

求 $\frac{\left(1-y_{1}\right) \cdots\left(1-y_{n}\right)}{\left(1-x_{1}\right) \cdots\left(1-x_{n}\right)}$ 的最小值.\\
解 显然, 当 $x_{i}=y_{i}, 1 \leqslant i \leqslant n$ 时, 满足假设条件, 此时,
$$
\frac{\left(1-y_{1}\right) \cdots\left(1-y_{n}\right)}{\left(1-x_{1}\right) \cdots\left(1-x_{n}\right)}=1
$$

下面证明:对满足条件的 $\left\{x_{i}\right\},\left\{y_{i}\right\}$, 有


\begin{equation*}
\frac{\left(1-y_{1}\right) \cdots\left(1-y_{n}\right)}{\left(1-x_{1}\right) \cdots\left(1-x_{n}\right)} \leqslant 1 \tag{1}
\end{equation*}


由平均值不等式,
$$
\frac{\left(1-y_{1}\right) \cdots\left(1-y_{n}\right)}{\left(1-x_{1}\right) \cdots\left(1-x_{n}\right)} \leqslant\left(\frac{1}{n} \sum_{i=1}^{n} \frac{1-y_{i}}{1-x_{i}}\right)^{n}
$$

为证明 (1), 只要证明


\begin{equation*}
\sum_{i=1}^{n} \frac{1-y_{i}}{1-x_{i}} \leqslant n \tag{2}
\end{equation*}


由于 $f(x)=\frac{x}{1-x}$ 在 $(-\infty, 1)$ 上单调增加, 则
$$
\frac{x_{i+1}}{1-x_{i+1}}<\frac{x_{i}}{1-x_{i}}, \quad 1 \leqslant i \leqslant n-1
$$

令 $z_{i}=\frac{y_{i}-x_{i}}{x_{i}}$, 则由假设及平均值不等式得,
$$
\sum_{j=1}^{i} z_{j}=\sum_{j=1}^{i} \frac{y_{j}}{x_{j}}-i \geqslant i \sqrt[i]{\frac{y_{1} \cdots y_{i}}{x_{1} \cdots x_{i}}}-i \geqslant 0
$$

由 Abel 求和公式得
$$
\begin{aligned}
\sum_{i=1}^{n} \frac{y_{i}-x_{i}}{1-x_{i}} & =\sum_{i=1}^{n} z_{i} \frac{x_{i}}{1-x_{i}} \\
& =\sum_{i=1}^{n+1}\left(\sum_{j=1}^{i} z_{j}\right)\left(\frac{x_{i}}{1-x_{i}}-\frac{x_{i+1}}{1-x_{i+1}}\right)+\frac{x_{n}}{1-x_{n}} \sum_{i=1}^{n} z_{i} \geqslant 0
\end{aligned}
$$

所以, $\sum_{i=1}^{n} \frac{1-y_{i}}{1-x_{i}}=\sum_{i=1}^{n}\left(1-\frac{y_{i}-x_{i}}{1-x_{i}}\right)=n-\sum_{i=1}^{n} \frac{y_{i}-x_{i}}{1-x_{i}} \leqslant n$.

即不等式(2)成立.

从而, $\frac{\left(1-y_{1}\right) \cdots\left(1-y_{n}\right)}{\left(1-x_{1}\right) \cdots\left(1-x_{n}\right)}$ 的最大值为 1 .

注 阿贝尔 (Abel)求和公式:设 $a_{1}, \cdots, a_{n}, b_{1}, \cdots, b_{n}, A_{i}=a_{1}+\cdots+$ $a_{i}, 1 \leqslant i \leqslant n, A_{0}=0$, 则
$$
\sum_{i=1}^{n} a_{i} b_{i}=\sum_{i=1}^{n-1} A_{i}\left(b_{i}-b_{i+1}\right)+b_{n} A_{n}
$$

例 14 设整数 $n>3$, 非负实数 $a_{1}, a_{2}, \cdots, a_{n}$ 满足 $a_{1}+a_{2}+\cdots+$ $a_{n}=2$.

求 $\frac{a_{1}}{a_{2}^{2}+1}+\frac{a_{2}}{a_{3}^{2}+1}+\cdots+\frac{a_{n}}{a_{1}^{2}+1}$ 的最小值.

解 由 $a_{1}+a_{2}+\cdots+a_{n}=2$ 知, 问题等价于求下式的最大值:
$$
\begin{aligned}
& a_{1}-\frac{a_{1}}{a_{2}^{2}+1}+a_{2}-\frac{a_{2}}{a_{3}^{2}+1}+\cdots+a_{n}-\frac{a_{n}}{a_{1}^{2}+1} \\
= & \frac{a_{1} a_{2}^{2}}{a_{2}^{2}+1}+\frac{a_{2} a_{3}^{2}}{a_{3}^{2}+1}+\cdots+\frac{a_{n} a_{1}^{2}}{a_{1}^{2}+1} \\
\leqslant & \frac{1}{2}\left(a_{1} a_{2}+a_{2} a_{3}+\cdots+a_{n} a_{1}\right) .
\end{aligned}
$$

上式最后一步的不等式成立是因为 $x^{2}+1 \geqslant 2 x$. 当 $x>0, y \geqslant 0$ 时, $\frac{1}{2 x} \geqslant$ $\frac{1}{x^{2}+1}, \frac{y x^{2}}{2 x} \geqslant \frac{y x^{2}}{x^{2}+1}$, 即 $\frac{y x^{2}}{x^{2}+1} \leqslant \frac{1}{2} x y$; 当 $x=0$ 时,上式也成立.

于是, 问题转化为求 $a_{1} a_{2}+a_{2} a_{3}+\cdots+a_{n} a_{1}$ 的最大值.

引理 若 $a_{1}, a_{2}, \cdots, a_{n} \geqslant 0, n \geqslant 4$,则
$$
4\left(a_{1} a_{2}+a_{2} a_{3}+\cdots+a_{n-1} a_{n}+a_{n} a_{1}\right) \leqslant\left(a_{1}+a_{2}+\cdots+a_{n}\right)^{2}
$$

设 $f\left(a_{1}, a_{2}, \cdots, a_{n}\right)=4\left(a_{1} a_{2}+a_{2} a_{3}+\cdots+a_{n-1} a_{n}+a_{n} a_{1}\right)-\left(a_{1}+\right.$ $\left.a_{2}+\cdots+a_{n}\right)^{2}$. 下面用数学归纳法证明


\begin{equation*}
f\left(a_{1}, a_{2}, \cdots, a_{n}\right) \leqslant 0 \tag{1}
\end{equation*}


当 $n=4$ 时, 不等式 (1) 等价于 $4\left(a_{1}+a_{3}\right)\left(a_{2}+a_{4}\right) \leqslant\left(a_{1}+a_{2}+a_{3}+a_{4}\right)^{2}$,由平均值不等式知, 命题成立. 假设不等式(1)对 $n=k(k \geqslant 4)$ 时成立. 对于 $n=k+1$, 不妨设 $a_{k}=\min \left\{a_{1}, a_{2}, \cdots, a_{k}, a_{k+1}\right\}$, 则
$$
\begin{aligned}
& f\left(a_{1}, a_{2}, \cdots, a_{k+1}\right)-f\left(a_{1}, a_{2}, \cdots, a_{k-1}, a_{k}+a_{k+1}\right) \\
= & 4\left(a_{k-1} a_{k}+a_{k} a_{k+1}+a_{1} a_{k+1}-a_{k-1}\left(a_{k}+a_{k+1}\right)-\left(a_{k}+a_{k+1}\right) a_{1}\right) \\
= & -4\left(\left(a_{k-1}-a_{k}\right) a_{k+1}+a_{1} a_{k}\right) \leqslant 0
\end{aligned}
$$

即
$$
f\left(a_{1}, a_{2}, \cdots, a_{k+1}\right) \leqslant f\left(a_{1}, a_{2}, \cdots, a_{k-1}, a_{k}+a_{k+1}\right)
$$

由归纳假设知, 上式右边 $\leqslant 0$. 即当 $n=k+1$ 时 (1) 成立. 引理证毕. 由引理知:
$$
\frac{1}{2}\left(a_{1} a_{2}+a_{2} a_{3}+\cdots+a_{n} a_{1}\right) \leqslant \frac{1}{8}\left(a_{1}+a_{2}+\cdots+a_{n}\right)^{2}=\frac{1}{8} \times 2^{2}=\frac{1}{2}
$$

所以
$$
\frac{a_{1} a_{2}^{2}}{a_{2}^{2}+1}+\frac{a_{2} a_{3}^{2}}{a_{3}^{2}+1}+\cdots+\frac{a_{n} a_{1}^{2}}{a_{1}^{2}+1} \leqslant \frac{1}{2}
$$

即
$$
\frac{a_{1}}{a_{2}^{2}+1}+\frac{a_{2}}{a_{3}^{2}+1}+\cdots+\frac{a_{n}}{a_{1}^{2}+1} \geqslant \frac{3}{2}
$$

当 $a_{1}=a_{2}=1, a_{3}=\cdots=a_{n}=0$ 时, 上式可取等号. 故所求最小值为 $\frac{3}{2}$.

例 15 给定正整数 $n \geqslant 2$. 若对任意 $a_{i} \in \mathbf{R}_{+}, 1 \leqslant i \leqslant n$, 恒有
$$
\frac{a_{1}}{a_{1}+a_{2}}+\frac{a_{2}}{a_{2}+a_{3}}+\cdots+\frac{a_{n-1}}{a_{n-1}+a_{n}} \geqslant \frac{a_{1}}{1+a_{1}}-\frac{a_{n}}{t+a_{n}}
$$

求实数 $t$ 的最大值.

解 首先, 取 $a_{i}=n^{i}, 1 \leqslant i \leqslant n$, 则
$$
\frac{a_{i}}{a_{i}+a_{i+1}}=\frac{n^{i}}{n^{i}+n^{i+1}}=\frac{1}{1+n}, \quad 1 \leqslant i \leqslant n-1
$$

原不等式化为 $\frac{n-1}{1+n} \geqslant \frac{n}{1+n}-\frac{n^{n}}{t+n^{n}}$, 即 $\frac{n^{n}}{t+n^{n}} \geqslant \frac{1}{1+n}$.

令 $t>0$, 则 $t \leqslant n^{n+1}$.

下面, 证明对任意 $a_{i} \in \mathbf{R}_{+}, 1 \leqslant i \leqslant n$, 有
$$
\begin{gathered}
\sum_{i=1}^{n-1} \frac{a_{i}}{a_{i}+a_{i+1}} \geqslant \frac{a_{1}}{1+a_{1}}-\frac{a_{n}}{n^{n+1}+a_{n}} . \\
\text { 令 } x_{1}=a_{1}, x_{2}=\frac{a_{2}}{a_{1}}, \cdots, x_{n}=\frac{a_{n}}{a_{n-1}} \text {, 则 } x_{i}>0,1 \leqslant i \leqslant n, \\
x_{1} x_{2} \cdots x_{n}=a_{n} .
\end{gathered}
$$

不等式(1)即为
$$
\frac{1}{1+x_{2}}+\frac{1}{1+x_{3}}+\cdots+\frac{1}{1+x_{n}} \geqslant \frac{x_{1}}{1+x_{1}}-\frac{x_{1} x_{2} \cdots x_{n}}{n^{n+1}+x_{1} x_{2} \cdots x_{n}}
$$

亦即


\begin{equation*}
\frac{1}{1+x_{1}}+\frac{1}{1+x_{2}}+\frac{1}{1+x_{3}}+\cdots+\frac{1}{1+x_{n}} \geqslant \frac{n^{n+1}}{n^{n+1}+x_{1} x_{2} \cdots x_{n}} \tag{2}
\end{equation*}


若 $\sum_{i=1}^{n} \frac{1}{1+x_{i}} \geqslant 1$, 则 (2) 显然成立.

若 $\sum_{i=1}^{n} \frac{1}{1+x_{i}}<1$, 取 $x_{n+1}=\frac{1}{1-\sum_{i=1}^{n} \frac{1}{1+x_{i}}}-1$, 则

$\sum_{i=1}^{n+1} \frac{1}{1+x_{i}}=1$, 且 $x_{n+1}>0$. 令 $y_{i}=\frac{1}{1+x_{i}}, 1 \leqslant i \leqslant n+1$,

则 $\sum_{i=1}^{n+1} y_{i}=1,0<y_{i}<1,1 \leqslant i \leqslant n+1$.

于是 $\quad \prod_{i=1}^{n+1} x_{i}=\prod_{i=1}^{n+1}\left(\frac{1}{y_{i}}-1\right)=\prod_{i=1}^{n+1} \frac{\sum_{j \neq i} y_{i}}{y_{i}} \geqslant \prod_{i=1}^{n+1} \frac{\sqrt[n]{\prod_{j \neq i} y_{j}}}{y_{i}}$
$$
=n^{n+1} \frac{y_{1} y_{2} \cdots y_{n+1}}{y_{1} y_{2} \cdots y_{n+1}}=n^{n+1}
$$

所以
$$
x_{n+1} \geqslant \frac{n^{n+1}}{x_{1} x_{2} \cdots x_{n}}
$$

$\sum_{i=1}^{n} \frac{1}{1+x_{i}}=1-\frac{1}{1+x_{n+1}} \geqslant 1-\frac{1}{1+\frac{n^{n+1}}{x_{1} x_{2} \cdots x_{n}}}=\frac{n^{n+1}}{n^{n+1}+x_{1} x_{2} \cdots x_{n}}$.

故 $t$ 的最大值为 $n^{n+1}$.

例 16 设 $a_{i} \in \mathbf{R}_{+}, 1 \leqslant i \leqslant n$. 求最大实数 $\lambda$, 使
$$
1+\sum_{i=1}^{n} \frac{1}{a_{i}^{2}} \geqslant \lambda \sum_{k=1}^{n} \frac{1}{\left(1+\sum_{i=1}^{k} a_{i}\right)^{2}}
$$

解 令 $S_{k}=1+\sum_{i=1}^{k} a_{i}, S_{0}=1$, 由平均值不等式得
$$
\begin{aligned}
S_{k}^{2}\left(\frac{1}{a_{k}^{2}}+\frac{1}{S_{k-1}^{2}}\right) & =\frac{a_{k}^{2}+2 a_{k} S_{k-1}+S_{k-1}^{2}}{a_{k}^{2}}+\frac{a_{k}^{2}+2 a_{k} S_{k-1}+S_{k-1}^{2}}{S_{k-1}^{2}} \\
& \geqslant 2+\frac{a_{k}^{2}}{S_{k-1}^{2}}+\frac{S_{k-1}^{2}}{a_{k}^{2}}+2\left(\frac{a_{k}}{S_{k-1}}+\frac{S_{k-1}}{a_{k}}\right) \geqslant 8
\end{aligned}
$$

即
$$
\frac{1}{a_{k}^{2}}+\frac{1}{S_{k-1}^{2}} \geqslant \frac{8}{S_{k}^{2}}
$$

于是 $\sum_{k=1}^{n} \frac{1}{a_{k}^{2}} \geqslant \sum_{k=1}^{n} \frac{7}{S_{k}^{2}}+\frac{1}{S_{n}^{2}}-1$, 即
$$
\sum_{k=1}^{n} \frac{1}{a_{k}^{2}}+1 \geqslant 7 \sum_{k=1}^{n} \frac{1}{S_{k}^{2}}
$$

从而 $\lambda \geqslant 7$.

取 $a_{i}=2^{i-1}, i \in \mathbf{Z}_{+}$, 则
$$
1+\sum_{k=1}^{n} \frac{1}{4^{k-1}} \geqslant \lambda \sum_{k=1}^{n} \frac{1}{\left(1+\sum_{i=1}^{k} 2^{i-1}\right)^{2}}
$$

即
$$
\begin{gathered}
1+\sum_{k=1}^{n} \frac{1}{4^{k-1}} \geqslant \lambda \sum_{k=1}^{n} \frac{1}{\left(2^{k}\right)^{2}} \\
\frac{7}{3}-\frac{4}{3 \cdot 4^{n}} \geqslant \lambda\left(\frac{1}{3}-\frac{1}{3 \cdot 4^{n}}\right)
\end{gathered}
$$

令 $n \rightarrow+\infty$, 则 $\lambda \leqslant 7$.

故实数 $\lambda$ 的最大值为 7 .

\section*{2.3 平均值不等式在几何不等式中的应用}
对于几何中出现的不等式证明, 常用的方法有: 几何方法、代数方法和三角方法, 当然, 我们不能将它们截然地分开, 常常是要综合地运用各种知识.如果采用代数方法证明几何命题, 那么, 灵活运用平均值不等式和柯西不等式, 对解决问题将有极大的帮助.

例 1 对于任意一个 $\triangle A B C$, 记其面积为 $S$, 周长为 $l, P 、 Q 、 R$ 依次为 $\triangle A B C$ 内切圆在边 $B C 、 C A 、 A B$ 上的切点. 证明:
$$
\left(\frac{A B}{P Q}\right)^{3}+\left(\frac{B C}{Q R}\right)^{3}+\left(\frac{C A}{R P}\right)^{3} \geqslant \frac{2}{\sqrt{3}} \cdot \frac{l^{2}}{S}
$$

证明 记 $B C=a, C A=b, A B=c, Q R=p, R P=q, P Q=r$. 设 $A R=x, B P=y, C Q=z$. 由 $x+y=c, y+z=a, z+x=b$, 得
$$
x=t-a, y=t-b, z=t-c\left(t=\frac{a+b+c}{2}\right)
$$

在 $\triangle A B C 、 \triangle A R Q$ 中, 由余弦定理分别得
$$
\begin{gathered}
a^{2}=b^{2}+c^{2}-2 b c \cos A=(b-c)^{2}+2 b c(1-\cos A) \\
p^{2}=2 x^{2}(1-\cos A)=2(t-a)^{2}(1-\cos A)
\end{gathered}
$$

两式消去 $1-\cos A$ 有


\begin{align*}
p^{2} & =(t-a)^{2} \frac{a^{2}-(b-c)^{2}}{b c} \\
& =\frac{4(t-a)(t-b)(t-c)}{a b c} a(t-a) \tag{1}
\end{align*}


注意到
$$
4(t-a)(t-b)=(b+c-a)(a-b+c)=c^{2}-(b-a)^{2} \leqslant c^{2}
$$

同理, $4(t-b)(t-c) \leqslant a^{2}, 4(t-c)(t-a) \leqslant b^{2}$. 则
$$
8(t-a)(t-b)(t-c) \leqslant a b c
$$

代人式(1)得
$$
p^{2} \leqslant \frac{a(t-a)}{2} \text { 或 }\left(\frac{a}{p}\right)^{3} \geqslant 2 \sqrt{2}\left(\frac{a}{t-a}\right)^{\frac{3}{2}} \text {. }
$$

同理, $\left(\frac{b}{q}\right)^{3} \geqslant 2 \sqrt{2}\left(\frac{b}{t-b}\right)^{\frac{3}{2}},\left(\frac{c}{r}\right)^{3} \geqslant 2 \sqrt{2}\left(\frac{c}{t-c}\right)^{\frac{3}{2}}$.

记 $M$ 为所证不等式左边. 则


\begin{align*}
M & \geqslant 2 \sqrt{2}\left[\left(\frac{a}{t-a}\right)^{\frac{3}{2}}+\left(\frac{b}{t-b}\right)^{\frac{3}{2}}+\left(\frac{c}{t-c}\right)^{\frac{3}{2}}\right] \\
& \geqslant \frac{2 \sqrt{2}}{\sqrt{3}}\left(\frac{a}{t-a}+\frac{b}{t-b}+\frac{c}{t-c}\right)^{\frac{3}{2}} \tag{2}
\end{align*}


又 $a \geqslant b \geqslant c \Leftrightarrow \frac{1}{t-a} \geqslant \frac{1}{t-b} \geqslant \frac{1}{t-c}$.

由切比雪夫不等式及平均值不等式得


\begin{align*}
& \frac{a}{t-a}+\frac{b}{t-b}+\frac{c}{t-c} \\
\geqslant & \frac{1}{3}(a+b+c)\left(\frac{1}{t-a}+\frac{1}{t-b}+\frac{1}{t-c}\right) \\
\geqslant & \frac{a+b+c}{[(t-a)(t-b)(t-c)]^{\frac{1}{3}}} \\
= & \frac{(a+b+c) t^{\frac{1}{3}}}{[t(t-a)(t-b)(t-c)]^{\frac{1}{3}}} \\
= & \frac{1}{2^{\frac{1}{3}}}\left(\frac{l^{2}}{S}\right)^{\frac{2}{3}} \tag{3}
\end{align*}


由式(2)、(3)得
$$
M \geqslant \frac{2 \sqrt{2}}{\sqrt{3}} \cdot \frac{1}{\sqrt{2}} \cdot \frac{l^{2}}{S}=\frac{2}{\sqrt{3}} \cdot \frac{l^{2}}{S} .
$$

例 2 设 $a, b, c, d \in \mathbf{R}_{+}, a^{2}+b^{2}+c^{2}+d^{2}=1$, 求证:
$$
a+b+c+d+\frac{1}{a b c d} \geqslant 18
$$

证明 $a+b+c+d+\frac{1}{a b c d}$
$$
\begin{aligned}
&= \sqrt{\frac{a^{2}}{a^{2}+b^{2}+c^{2}+d^{2}}}+\sqrt{\frac{b^{2}}{a^{2}+b^{2}+c^{2}+d^{2}}} \\
&+\sqrt{\frac{c^{2}}{a^{2}+b^{2}+c^{2}+d^{2}}}+\sqrt{\frac{d^{2}}{a^{2}+b^{2}+c^{2}+d^{2}}} \\
&+\underbrace{\frac{\left(a^{2}+b^{2}+c^{2}+d^{2}\right)^{2}}{32 a b c d}+\cdots+\frac{\left(a^{2}+b^{2}+c^{2}+d^{2}\right)^{2}}{32 a b c d}} \\
& \geqslant 36 \cdot \sqrt[36]{\sqrt{\frac{a^{2} b^{2} c^{2} d^{2}}{\left(a^{2}+b^{2}+c^{2}+d^{2}\right)^{4}}} \cdot\left[\frac{\left(a^{2}+b^{2}+c^{2}+d^{2}\right)^{2}}{32 a b c d}\right]^{32}} \\
&=36 \cdot \sqrt[36]{\frac{\left(a^{2}+b^{2}+c^{2}+d^{2}\right)^{62}}{32^{32}(a b c d)^{31}}} \\
& \geqslant 36 \sqrt[36]{\frac{\left(4 \sqrt[4]{a^{2} b^{2} c^{2} d^{2}}\right)^{62}}{32^{32}(a b c d)^{31}}} \\
&=18
\end{aligned}
$$

例 $3 \triangle A B C$ 的三边长 $a 、 b 、 c$ 满足 $a+b+c=1$. 求证:
$$
5\left(a^{2}+b^{2}+c^{2}\right)+18 a b c \geqslant \frac{7}{3}
$$

证明 因为 $a^{2}+b^{2}+c^{2}$
$$
\begin{aligned}
& =(a+b+c)^{2}-2(a b+b c+c a) \\
& =1-2(a b+b c+c a)
\end{aligned}
$$

所以, 欲证的不等式等价于
$$
\frac{5}{9}(a b+b c+c a)-a b c \leqslant \frac{4}{27}
$$

构造函数
$$
f(x)=(x-a)(x-b)(x-c),
$$

一方面,
$$
f(x)=x^{3}-(a+b+c) x^{2}+(a b+b c+c a) x-a b c,
$$

所以,
$$
f\left(\frac{5}{9}\right)=\left(\frac{5}{9}\right)^{3}-\left(\frac{5}{9}\right)^{2}+\frac{5}{9}(a b+b c+c a)-a b c
$$

另一方面, 因为 $a 、 b 、 c$ 是三角形三边长, 所以 $0<a, b, c<\frac{1}{2}$, 且 $\frac{5}{9}-$ $a, \frac{5}{9}-b, \frac{5}{9}-c$ 均为正数, 利用平均值不等式, 有
$$
\begin{aligned}
f\left(\frac{5}{9}\right) & =\left(\frac{5}{9}-a\right)\left(\frac{5}{9}-b\right)\left(\frac{5}{9}-c\right) \\
& \leqslant \frac{1}{27}\left[\left(\frac{5}{9}-a\right)+\left(\frac{5}{9}-b\right)+\left(\frac{5}{9}-c\right)\right]^{3} \\
& =\frac{8}{729}
\end{aligned}
$$

所以,
$$
\frac{5}{9}(a b+b c+c a)-a b c \leqslant \frac{8}{729}-\left(\frac{5}{9}\right)^{3}+\left(\frac{5}{9}\right)^{2}=\frac{4}{27}
$$

从而, 欲证不等式成立.

例 4 设 $P$ 是锐角 $\triangle A B C$ 内的任意一点, 直线 $A P 、 B P 、 C P$ 分别交 $\triangle P B C 、 \triangle P C A 、 \triangle P A B$ 的外接圆于另一点 $A_{1} 、 B_{1} 、 C_{1}$ (不同于 $P$ ). 求证:
$$
\left(1+2 \cdot \frac{P A}{P A_{1}}\right)\left(1+2 \cdot \frac{P B}{P B_{1}}\right)\left(1+2 \cdot \frac{P C}{P C_{1}}\right) \geqslant 8
$$

证明 如图 $2-1$, 连结 $A_{1} B 、 A_{1} C 、 B_{1} C 、 B_{1} A$ 、 $C_{1} A 、 C_{1} B$, 并记 $\angle B A_{1} C=\angle C A B_{1}=\angle B A C_{1}=\alpha$, $\angle C B_{1} A=\angle A B C_{1}=\angle C B A_{1}=\beta, \angle A C_{1} B=$ $\angle B C A_{1}=\angle A C B_{1}=\gamma$

在四边形 $P B A_{1} C$ 中, 由托勒密 (Ptolemy) 定理得
$$
\begin{gathered}
P A_{1} \cdot B C=P B \cdot A_{1} C+P C \cdot A_{1} B \\
P A_{1}=\frac{A_{1} C}{B C} \cdot P B+\frac{A_{1} B}{B C} \cdot P C
\end{gathered}
$$

\begin{center}
此处有图片 % \includegraphics[max width=\textwidth]{2024_05_22_4ff05a14ba9ad07b725fg-062}
\end{center}

图 $2-1$\\
再在 $\triangle A_{1} B C$ 中, 由正弦定理得


\begin{equation*}
P A_{1}=\frac{\sin \beta}{\sin \alpha} \cdot P B+\frac{\sin \gamma}{\sin \alpha} \cdot P C \tag{1}
\end{equation*}


同理可得


\begin{align*}
& P B_{1}=\frac{\sin \gamma}{\sin \beta} \cdot P C+\frac{\sin \alpha}{\sin \beta} \cdot P A  \tag{2}\\
& P C_{1}=\frac{\sin \alpha}{\sin \gamma} \cdot P A+\frac{\sin \beta}{\sin \gamma} \cdot P B \tag{3}
\end{align*}


由(1)、(2)、(3)联立方程组解得
$$
\begin{aligned}
& 2 \cdot P A=\frac{\sin \beta}{\sin \alpha} \cdot P B_{1}+\frac{\sin \gamma}{\sin \alpha} \cdot P C_{1}-P A_{1} \\
& 2 \cdot P B=\frac{\sin \gamma}{\sin \beta} \cdot P C_{1}+\frac{\sin \alpha}{\sin \beta} \cdot P A_{1}-P B_{1} \\
& 2 \cdot P C=\frac{\sin \alpha}{\sin \gamma} \cdot P A_{1}+\frac{\sin \beta}{\sin \gamma} \cdot P B_{1}-P C_{1}
\end{aligned}
$$

于是
$$
\begin{aligned}
2 \cdot P A+P A_{1} & =\frac{\sin \beta}{\sin \alpha} \cdot P B_{1}+\frac{\sin \gamma}{\sin \alpha} \cdot P C_{1} \\
& \geqslant 2 \cdot \sqrt{\frac{\sin \beta}{\sin \alpha} \cdot P B_{1} \cdot \frac{\sin \gamma}{\sin \alpha} \cdot P C_{1}}
\end{aligned}
$$

同理
$$
\begin{aligned}
& 2 \cdot P B+P B_{1} \geqslant 2 \cdot \sqrt{\frac{\sin \gamma}{\sin \beta} \cdot P C_{1} \cdot \frac{\sin \alpha}{\sin \beta} \cdot P A_{1}} \\
& 2 \cdot P C+P C_{1} \geqslant 2 \cdot \sqrt{\frac{\sin \alpha}{\sin \gamma} \cdot P A_{1} \cdot \frac{\sin \beta}{\sin \gamma} \cdot P B_{1}}
\end{aligned}
$$

将以上三个不等式相乘, 得
$$
\left(2 \cdot P A+P A_{1}\right)\left(2 \cdot P B+P B_{1}\right)\left(2 \cdot P C+P C_{1}\right) \geqslant 8 \cdot P A_{1} \cdot P B_{1} \cdot P C_{1}
$$

故
$$
\left(1+2 \cdot \frac{P A}{P A_{1}}\right)\left(1+2 \cdot \frac{P B}{P B_{1}}\right)\left(1+2 \cdot \frac{P C}{P C_{1}}\right) \geqslant 8
$$

例 5 设 $P$ 为 $\triangle A B C$ 内一点, $D 、 E 、 F$ 分别为 $P$ 到 $B C 、 C A 、 A B$ 各边的\\
垂足. 试确定点 $P$, 使 $P D \times P E \times P F$ 最大.

解 记 $\triangle A B C$ 的三个内角为 $A 、 B 、 C$, 其对边为 $a 、 b 、 c$. 记 $\triangle A B C$ 的面积为 $S$ ( 以下各题记号均同,不再注明).

设 $P D=x, P E=y, P F=z$. 如图 2-2, 连结 $A P 、 B P 、 C P$. 易知

\begin{center}
此处有图片 % \includegraphics[max width=\textwidth]{2024_05_22_4ff05a14ba9ad07b725fg-064}
\end{center}

图 2 - 2
$$
\begin{aligned}
& S_{1}(\triangle P B C \text { 面积 })=\frac{1}{2} a x, \\
& S_{2}(\triangle P C A \text { 面积 })=\frac{1}{2} b y, \\
& S_{3}(\triangle P A B \text { 面积 })=\frac{1}{2} c z .
\end{aligned}
$$

从而有 $a x+b y+c z=2\left(S_{1}+S_{2}+S_{3}\right)=2 S=$ 定值. 由平均值不等式,得
$$
a x \cdot b y \cdot c z \leqslant\left(\frac{a x+b y+c z}{3}\right)^{3}=\left(\frac{2 S}{3}\right)^{3}
$$

即
$$
x y z \leqslant \frac{8 S^{3}}{27 a b c}
$$

\section*{063}
上式等号当且仅当 $a x=b y=c z$ 时成立. 这就是说, $S_{1}=S_{2}=S_{3}=\frac{1}{3} S$使得 $x y z$ 取最大. 这时 $P$ 为 $\triangle A B C$ 的重心.

例 6 设 $a 、 b 、 c$ 为三角形的三条边的长度, $\delta$ 为面积. 求证:
$$
\delta \leqslant \frac{\sqrt{3}}{4}\left(\frac{a+b+c}{3}\right)^{2}
$$

当且仅当 $a=b=c$ 时等号成立.

证明 由海伦公式, 原不等式等价于
$$
\sqrt{p(p-a)(p-b)(p-c)} \leqslant \frac{\sqrt{3}}{4}\left(\frac{2 p}{3}\right)^{2}=\sqrt{3}\left(\frac{p}{3}\right)^{2}
$$

等价于
$$
(p-a)(p-b)(p-c) \leqslant \frac{p^{3}}{27}
$$

由平均值不等式, 得
$$
(p-a)(p-b)(p-c) \leqslant\left(\frac{3 p-a-b-c}{3}\right)^{3}=\frac{p^{3}}{27}
$$

此式当且仅当 $p-a=p-b=p-c$, 即 $a=b=c$ 时等号成立.

例 7 设 $T_{a} 、 T_{b} 、 T_{c}$ 为 $\triangle A B C$ 的角平分线的延长线与外接圆相交所得的线段长. 求证:
$$
a b c \leqslant \frac{3 \sqrt{3}}{8} T_{a} T_{b} T_{c}
$$

证明 设 $|A E|=T_{a}, D$ 为 $A E$ 与 $B C$ 的交点, 则
$$
\begin{aligned}
& B E^{2}=c^{2}+T_{a}^{2}-2 c T_{a} \cos \frac{A}{2} \\
& C E^{2}=b^{2}+T_{a}^{2}-2 b T_{a} \cos \frac{A}{2}
\end{aligned}
$$

因为 $B E=C E$, 所以
$$
T_{a}=\frac{b+c}{2 \cos \frac{A}{2}}
$$

再由平均值不等式, 得 $T_{a} \geqslant \frac{\sqrt{b c}}{\cos \frac{A}{2}}$.

同理可得 $T_{b} \geqslant \frac{\sqrt{a c}}{\cos \frac{B}{2}}, T_{c} \geqslant \frac{\sqrt{a b}}{\cos \frac{C}{2}}$. 于是
$$
T_{a} T_{b} T_{c} \geqslant \frac{a b c}{\cos \frac{A}{2} \cos \frac{B}{2} \cos \frac{C}{2}}
$$

再由 $\cos \frac{A}{2} \cos \frac{B}{2} \cos \frac{C}{2} \leqslant \frac{3 \sqrt{3}}{8}$ 得到命题成立.

例 8 设 $P$ 为 $\triangle A B C$ 内部或边界上一点, 点 $P$ 到三边的距离分别为 $P D 、 P E 、 P F$. 求证:
$$
P A+P B+P C \geqslant 2(P D+P E+P F)
$$

证明 设 $P A=x, P B=y, P C=z, P D=p, P E=q, P F=r$, 其中 $D 、 E 、 F$ 为点 $P$ 在三边上的射影. 则 $C 、 D 、 P 、 E$ 四点共圆, 得
$$
\begin{aligned}
D E & =\sqrt{p^{2}+q^{2}+2 p q \cos C} \\
& =\sqrt{(p \sin B+q \sin A)^{2}+(p \cos B-q \cos A)^{2}} \\
& \geqslant p \sin B+q \sin A
\end{aligned}
$$

从而
$$
z=\frac{D E}{\sin C} \geqslant \frac{p \sin B+q \sin A}{\sin C}
$$

同理可得
$$
x \geqslant \frac{r \sin B+q \sin C}{\sin A}, y \geqslant \frac{r \sin A+p \sin C}{\sin B}
$$

于是
$$
\begin{aligned}
x+y+z & \geqslant \frac{r \sin B+q \sin C}{\sin A}+\frac{r \sin A+p \sin C}{\sin B}+\frac{p \sin B+q \sin A}{\sin C} \\
& \geqslant 2(p+q+r)
\end{aligned}
$$

等号成立当且仅当 $\triangle A B C$ 为正三角形,且 $P$ 为 $\triangle A B C$ 的中心.

例 9 设 $A B C D E F$ 是凸六边形, 且 $A B / / E D, B C / / F E, C D / / A F$. 又设 $R_{A} 、 R_{C} 、 R_{E}$ 分别表示 $\triangle F A B 、 \triangle B C D 、 \triangle D E F$ 的外接圆半径, $p$ 表示六边形的周长,证明:
$$
R_{A}+R_{B}+R_{C} \geqslant \frac{p}{2}
$$

证明 过点 $A$ 作 $B C$ 的垂线, 记该垂线夹在平行线 $E F 、 B C$ 之间线段的长度为 $h$, 设 $A B 、 B C 、 C D 、 D E 、 E F 、 F A$ 的长度分别为 $a 、 b 、 c 、 d 、 e 、 f$. 则
$$
B F \geqslant h=a \sin B+f \sin F
$$

由于
$$
A B / / D E, B C / / E F, A F / / C D
$$

所以
$$
\begin{gathered}
\angle A=\angle D, \angle B=\angle E, \angle C=\angle F \\
2 R_{A}=\frac{B F}{\sin A} \geqslant a \frac{\sin B}{\sin A}+f \frac{\sin F}{\sin A}
\end{gathered}
$$

同理可得
$$
\begin{aligned}
& 2 R_{C}=\frac{B D}{\sin C} \geqslant b \frac{\sin B}{\sin C}+c \frac{\sin D}{\sin C} \\
& 2 R_{E}=\frac{D F}{\sin E} \geqslant d \frac{\sin D}{\sin E}+e \frac{\sin F}{\sin E}
\end{aligned}
$$

所以
$$
\begin{aligned}
& 2\left(R_{A}+R_{C}+R_{E}\right) \\
\geqslant & a \frac{\sin B}{\sin A}+b \frac{\sin B}{\sin C}+c \frac{\sin D}{\sin C}+d \frac{\sin D}{\sin E}+e \frac{\sin F}{\sin E}+f \frac{\sin F}{\sin A}
\end{aligned}
$$

分两种情况讨论:

(1) 当 $a=d$ 时, 由题设
$$
A B / / D E, B C / / E F, C D / / F A,
$$

可得 $\triangle A F E \cong \triangle B C D$,

则 $\quad 2\left(R_{A}+R_{C}+R_{E}\right)$
$$
\begin{aligned}
& 0 \geqslant a\left(\frac{\sin B}{\sin A}+\frac{\sin D}{\sin E}\right)+b\left(\frac{\sin B}{\sin C}+\frac{\sin F}{\sin E}\right)+ \\
& \quad c\left(\frac{\sin D}{\sin C}+\frac{\sin F}{\sin A}\right) \\
& \geqslant 2(a+b+c)
\end{aligned}
$$

\begin{center}
此处有图片 % \includegraphics[max width=\textwidth]{2024_05_22_4ff05a14ba9ad07b725fg-067}
\end{center}

图 $2-3$

所以 $R_{A}+R_{C}+R_{E} \geqslant a+b+c=\frac{p}{2}$.

(2) 当 $a \neq d$ 时, 不妨设 $a>d$. 如图 2-3, 作 $F A_{1} \Perp A B, F E_{1} \Perp D E$, 连 $D E_{1}$, 连 $B A_{1}$ 延长交 $D E_{1}$ 于点 $C_{1}$, 则

且
$$
\begin{gathered}
\angle E_{1} A_{1} C_{1}=\pi-A, \angle A_{1} C_{1} E_{1}=\pi-C, \angle C_{1} E_{1} A_{1}=\pi-E=\pi-B \\
\quad \frac{\sin A}{\sin B}=\frac{C_{1} E_{1}}{A_{1} C_{1}}, \frac{\sin B}{\sin C}=\frac{A_{1} C_{1}}{A_{1} E_{1}}, \frac{\sin C}{\sin A}=\frac{A_{1} E_{1}}{C_{1} E_{1}} \\
\\
a=d+A_{1} E_{1}, e=b+E_{1} C_{1}, c=f+C_{1} A_{1}
\end{gathered}
$$

利用平均值不等式, 得
$$
\begin{aligned}
& 2\left(R_{A}+R_{B}+R_{C}\right) \\
\geqslant & 2(b+d+f)+\frac{A_{1} E_{1} \cdot A_{1} C_{1}}{E_{1} C_{1}}+\frac{E_{1} C_{1} \cdot A_{1} E_{1}}{A_{1} C_{1}}+\frac{A_{1} C_{1} \cdot E_{1} C_{1}}{A_{1} E_{1}}
\end{aligned}
$$

再由排序不等式, 得
$$
\frac{A_{1} E_{1} \cdot A_{1} C_{1}}{E_{1} C_{1}}+\frac{E_{1} C_{1} \cdot A_{1} E_{1}}{A_{1} C_{1}}+\frac{A_{1} C_{1} \cdot E_{1} C_{1}}{A_{1} E_{1}} \geqslant A_{1} E_{1}+C_{1} A_{1}+E_{1} C_{1}
$$

从而
$$
2\left(R_{A}+R_{B}+R_{C}\right) \geqslant 2(b+d+f)+A_{1} E_{1}+C_{1} A_{1}+E_{1} C_{1}=p
$$

综合 (1)、(2)知命题成立.

\section*{2.4 平均值不等式的变形及应用}
对于平均值不等式, 有各种不同的变形和推广, 由于这些问题可以包括在命题的证明和讨论中, 这里就不展开讨论了.\\
对任意正数 $a_{1}, a_{2}, \cdots, a_{n}$, 由平均值不等式,得
$$
\sum_{i=1}^{n} a_{i} \cdot \sum_{i=1}^{n} \frac{1}{a_{i}} \geqslant n \sqrt[n]{a_{1} a_{2} \cdots a_{n}} \cdot n \sqrt[n]{\frac{1}{a_{1}} \cdot \frac{1}{a_{2}} \cdot \cdots \cdot \frac{1}{a_{n}}}=n^{2}
$$

从而
$$
\frac{\sum_{i=1}^{n} a_{i}}{n} \geqslant \frac{n}{\sum_{i=1}^{n} \frac{1}{a_{i}}}
$$

令 $H_{n}=\frac{n}{\frac{1}{a_{1}}+\frac{1}{a_{2}}+\cdots+\frac{1}{a_{n}}}$, 则称 $H_{n}$ 为 $n$ 个正实数 $a_{1}, a_{2}, \cdots, a_{n}$ 的调和平均值.

由于 $\frac{x_{1}+x_{2}+\cdots+x_{n}}{n} \geqslant \sqrt[n]{x_{1} x_{2} \cdots x_{n}}$, 令 $x_{i}=\frac{1}{a_{i}}$, 则
$$
\frac{n}{\frac{1}{a_{1}}+\frac{1}{a_{2}}+\cdots+\frac{1}{a_{n}}} \leqslant \sqrt[n]{a_{1} a_{2} \cdots a_{n}} .
$$

即 $H_{n} \leqslant G_{n}$, 调和平均值不大于几何平均值.

对任意实数 $a_{1}, a_{2}, \cdots, a_{n}$, 有
$$
n \sum_{i=1}^{n} a_{i}^{2}-\left(\sum_{i=1}^{n} a_{i}\right)^{2}=\sum_{1 \leqslant i<j \leqslant n}\left(a_{i}-a_{j}\right)^{2} \geqslant 0
$$

故得到
$$
\frac{a_{1}+a_{2}+\cdots+a_{n}}{n} \leqslant \sqrt{\frac{a_{1}^{2}+a_{2}^{2}+\cdots+a_{n}^{2}}{n}}
$$

令 $Q_{n}=\sqrt{\frac{a_{1}^{2}+a_{2}^{2}+\cdots+a_{n}^{2}}{n}}$, 称 $Q_{n}$ 为 $n$ 个实数 $a_{1}, a_{2}, \cdots, a_{n}$ 的平方平均值.

所以, 对任意实数 $a_{1}, a_{2}, \cdots, a_{n}, A_{n} \leqslant Q_{n}$, 即算术平均值不大于平方平均值.

于是, 对任意正实数 $a_{1}, a_{2}, \cdots, a_{n}$, 得到四个平均值有如下的关系
$$
H_{n} \leqslant G_{n} \leqslant A_{n} \leqslant Q_{n},
$$

且等式成立的充分必要条件是 $a_{1}=a_{2}=\cdots=a_{n}$.

例 1 设 $a 、 b 、 c$ 是正实数, 且满足 $a^{2}+b^{2}+c^{2}=3$. 证明:
$$
\frac{1}{1+2 a b}+\frac{1}{1+2 b c}+\frac{1}{1+2 c a} \geqslant 1
$$

证明 由算术平均值大于或等于几何平均值及算术平均值大于或等于调和平均值可得
$$
\begin{aligned}
& \frac{1}{1+2 a b}+\frac{1}{1+2 b c}+\frac{1}{1+2 c a} \\
\geqslant & \frac{1}{1+a^{2}+b^{2}}+\frac{1}{1+b^{2}+c^{2}}+\frac{1}{1+c^{2}+a^{2}} \\
\geqslant & 3 \cdot \frac{3}{\left(1+a^{2}+b^{2}\right)+\left(1+b^{2}+c^{2}\right)+\left(1+c^{2}+a^{2}\right)} \\
= & \frac{9}{3+2\left(a^{2}+b^{2}+c^{2}\right)}=1
\end{aligned}
$$

例 2 已知正实数 $a 、 b 、 c$ 满足
$$
a b+b c+c a \leqslant 3 a b c
$$

证明:
$$
\sqrt{\frac{a^{2}+b^{2}}{a+b}}+\sqrt{\frac{b^{2}+c^{2}}{b+c}}+\sqrt{\frac{c^{2}+a^{2}}{c+a}}+3
$$
$$
\leqslant \sqrt{2}(\sqrt{a+b}+\sqrt{b+c}+\sqrt{c+a})
$$

证明 由 $Q_{2} \geqslant A_{2}$ 得
$$
\begin{aligned}
& \sqrt{2} \cdot \sqrt{a+b}=2 \sqrt{\frac{a b}{a+b}} \cdot \sqrt{\frac{1}{2}\left(2+\frac{a^{2}+b^{2}}{a b}\right)} \\
\geqslant & 2 \sqrt{\frac{a b}{a+b}} \cdot \frac{1}{2}\left(\sqrt{2}+\sqrt{\frac{a^{2}+b^{2}}{a b}}\right) \\
= & \sqrt{\frac{2 a b}{a+b}}+\sqrt{\frac{a^{2}+b^{2}}{a+b}}
\end{aligned}
$$

同理,
$$
\begin{aligned}
& \sqrt{2} \cdot \sqrt{b+c} \geqslant \sqrt{\frac{2 b c}{b+c}}+\sqrt{\frac{b^{2}+c^{2}}{b+c}} \\
& \sqrt{2} \cdot \sqrt{c+a} \geqslant \sqrt{\frac{2 c a}{c+a}}+\sqrt{\frac{c^{2}+a^{2}}{c+a}}
\end{aligned}
$$

再由 $Q_{3} \geqslant H_{3}$ 得
$$
\sqrt{\frac{\left(\sqrt{\frac{a+b}{2 a b}}\right)^{2}+\left(\sqrt{\frac{b+c}{2 b c}}\right)^{2}+\left(\sqrt{\frac{c+a}{2 c a}}\right)^{2}}{3}}
$$
$$
\geqslant \frac{1}{\frac{1}{\sqrt{\frac{a+b}{2 a b}}}+\frac{1}{\sqrt{\frac{b+c}{2 b c}}}+\frac{1}{\sqrt{\frac{c+a}{2 c a}}}}
$$

所以
$$
\begin{aligned}
& \sqrt{\frac{2 a b}{a+b}}+\sqrt{\frac{2 b c}{b+c}}+\sqrt{\frac{2 c a}{c+a}} \\
& \geqslant 3 \sqrt{\left(\sqrt{\frac{a+b}{2 a b}}\right)^{2}+\left(\sqrt{\frac{b+c}{2 b c}}\right)^{2}+\left(\sqrt{\frac{c+a}{2 c a}}\right)^{2}} \\
& =3 \sqrt{\frac{3 a b c}{a b+b c+c a}} \geqslant 3
\end{aligned}
$$

于是, 原不等式成立.

例 3 设正实数 $x 、 y 、 z$ 满足 $x^{2}+y^{2}+z^{2}=1$. 求证:
$$
x^{2} y z+y^{2} x z+z^{2} x y \leqslant \frac{1}{3}
$$

证明 因为 $\sqrt[3]{x y z} \leqslant \frac{x+y+z}{3} \leqslant \sqrt{\frac{x^{2}+y^{2}+z^{2}}{3}}=\frac{1}{\sqrt{3}}$, 所以
$$
x y z \leqslant \frac{1}{3 \sqrt{3}}, x+y+z \leqslant \sqrt{3}
$$

故
$$
x^{2} y z+y^{2} x z+z^{2} x y \leqslant \frac{1}{3}
$$

例 4 设 $a, b, c, d \in \mathbf{R}_{+}$, 求证:
$$
\sqrt[3]{\frac{a b c+b c d+c d a+a d b}{4}} \leqslant \sqrt{\frac{a^{2}+b^{2}+c^{2}+d^{2}}{4}}
$$

证明 首先两次应用 $G_{2} \leqslant A_{2}$, 得
$$
\begin{aligned}
& \frac{a b c+b c d+c d a+a d b}{4} \\
= & \frac{1}{2}\left(a b \cdot \frac{c+d}{2}+c d \cdot \frac{a+b}{2}\right) \\
\leqslant & \frac{1}{2}\left[\left(\frac{a+b}{2}\right)^{2} \cdot \frac{c+d}{2}+\left(\frac{c+d}{2}\right)^{2} \cdot \frac{a+b}{2}\right] \\
= & \frac{a+b}{2} \cdot \frac{c+d}{2} \cdot \frac{a+b+c+d}{4}
\end{aligned}
$$
$$
\begin{aligned}
& \leqslant\left(\frac{\frac{a+b}{2}+\frac{c+d}{2}}{2}\right)^{2} \frac{a+b+c+d}{4} \\
& =\left(\frac{a+b+c+d}{4}\right)^{3}
\end{aligned}
$$

即再由 $A_{4} \leqslant Q_{4}$, 得
$$
\frac{a+b+c+d}{4} \leqslant \sqrt{\frac{a^{2}+b^{2}+c^{2}+d^{2}}{4}}
$$

故原不等式成立.

例 $\mathbf{5}$ 设 $x_{i} \geqslant 1(i=1,2, \cdots, n)$, 证明:
$$
\frac{\prod\left(x_{i}-1\right)}{\left(\sum\left(x_{i}-1\right)\right)^{n}} \leqslant \frac{\prod x_{i}}{\left(\sum x_{i}\right)^{n}}
$$

证明 原不等式等价于
$$
\left(\frac{\prod\left(x_{i}-1\right)}{\prod x_{i}}\right)^{\frac{1}{n}} \leqslant \frac{\sum\left(x_{i}-1\right)}{\sum x_{i}}
$$

这个不等式可以由下面的事实推出.

由平均值不等式, 得
$$
\left(\frac{\prod\left(x_{i}-1\right)}{\prod x_{i}}\right)^{\frac{1}{n}}=\left(\prod \frac{x_{i}-1}{x_{i}}\right)^{\frac{1}{n}} \leqslant \frac{1}{n} \sum \frac{x_{i}-1}{x_{i}}=1-\frac{1}{n} \sum \frac{1}{x_{i}}
$$

以及
$$
\begin{gathered}
\frac{\sum\left(x_{i}-1\right)}{\sum x_{i}}=\frac{\sum x_{i}-n}{\sum x_{i}}=1-\frac{n}{\sum x_{i}} \\
\frac{1}{n} \sum \frac{1}{x_{i}} \geqslant \frac{n}{\sum x_{i}}
\end{gathered}
$$

从而可知命题成立.

例 6 设 $x_{i} \in\left[0, \frac{\pi}{2}\right], i=1,2, \cdots, 10$, 满足
$$
\sin ^{2} x_{1}+\sin ^{2} x_{2}+\cdots+\sin ^{2} x_{10}=1
$$

证明:
$$
3\left(\sin x_{1}+\cdots+\sin x_{10}\right) \leqslant \cos x_{1}+\cdots+\cos x_{10}
$$

证明 由于 $\sin ^{2} x_{1}+\sin ^{2} x_{2}+\cdots+\sin ^{2} x_{10}=1$,
$$
\cos x_{i}=\sqrt{\sum_{j \neq i} \sin ^{2} x_{j}}
$$

则对 $1 \leqslant i \leqslant 10$, 得
$$
\cos x_{i}=\sqrt{\sum_{j \neq i} \sin ^{2} x_{j}} \geqslant \frac{\sum_{j \neq i} \sin x_{j}}{3}
$$

从而
$$
\sum_{i=1}^{10} \cos x_{i} \geqslant \sum_{i=1}^{10} \sum_{j \neq i} \frac{\sin x_{j}}{3}=\sum_{i=1}^{10} 9 \cdot \frac{\sin x_{i}}{3}=3 \sum_{i=1}^{10} \sin x_{i}
$$

故命题成立.

例 7 设 $a_{i} \in \mathbf{R}_{+}, i=1,2, \cdots, n$, 且 $\sum_{i=1}^{n} a_{i}=1$, 求
$$
M=\sum_{i=1}^{n} \frac{a_{i}}{1+\sum_{j \neq i, j=1}^{n} a_{j}}
$$

的最小值.

解 由 $M+n=\left(\frac{a_{1}}{2-a_{1}}+1\right)+\left(\frac{a_{2}}{2-a_{2}}+1\right)+\cdots+\left(\frac{a_{n}}{2-a_{n}}+1\right)$
$$
\begin{aligned}
& =\frac{2}{2-a_{1}}+\frac{2}{2-a_{2}}+\cdots+\frac{2}{2-a_{n}}\left(\text { 由 } H_{n} \leqslant A_{n}\right) \\
& \geqslant \frac{n^{2}}{\frac{1}{2}\left(2-a_{1}\right)+\frac{1}{2}\left(2-a_{2}\right)+\cdots+\frac{1}{2}\left(2-a_{n}\right)} \\
& =\frac{n^{2}}{\frac{1}{2}(2 n-1)}=\frac{2 n^{2}}{2 n-1} .
\end{aligned}
$$

所以 $M \geqslant \frac{n}{2 n-1}$, 当且仅当 $a_{1}=a_{2}=\cdots=a_{n}=\frac{1}{n}$ 时, $M=\frac{n}{2 n-1}$.

于是 $M$ 的最小值为 $\frac{n}{2 n-1}$.\\
例 8 已知 $a, b, c \in \mathbf{R}_{+}$, 且满足 $\frac{a^{2}}{1+a^{2}}+\frac{b^{2}}{1+b^{2}}+\frac{c^{2}}{1+c^{2}}=1$, 求证:
$$
a b c \leqslant \frac{\sqrt{2}}{4}
$$

证明 令 $x=\frac{a^{2}}{1+a^{2}}, y=\frac{b^{2}}{1+b^{2}}, z=\frac{c^{2}}{1+c^{2}}$, 则
$$
0<x, y, z<1, x+y+z=1
$$
$$
a^{2}=\frac{x}{1-x}, b^{2}=\frac{y}{1-y}, c^{2}=\frac{z}{1-z}, a^{2} b^{2} c^{2}=\frac{x y z}{(1-x)(1-y)(1-z)}
$$

于是,原问题化为证明
$$
\frac{x y z}{(1-x)(1-y)(1-z)} \leqslant \frac{1}{8}
$$

由 $H_{3} \leqslant A_{3}$, 并注意到 $x+y+z=1$, 有 $\frac{1}{x}+\frac{1}{y}+\frac{1}{z} \geqslant 9$, 则
$$
9\left(\frac{1}{x}+\frac{1}{y}+\frac{1}{z}-1\right) \geqslant 8\left(\frac{1}{x}+\frac{1}{y}+\frac{1}{z}\right)
$$

由于 $G_{3} \geqslant H_{3}$, 得
$$
(x y z)^{\frac{1}{3}} \geqslant \frac{3}{\frac{1}{x}+\frac{1}{y}+\frac{1}{z}} \geqslant \frac{3}{\frac{9}{8}\left(\frac{1}{x}+\frac{1}{y}+\frac{1}{z}-1\right)}
$$

所以
$$
(x y z)^{-\frac{1}{3}} \leqslant \frac{3}{8}\left(\frac{1}{x}+\frac{1}{y}+\frac{1}{z}-1\right)
$$

又由
$$
\begin{aligned}
& \frac{1}{8}(1-x)(1-y)(1-z)-\frac{1}{3}(x y z)^{\frac{2}{3}} \\
= & \frac{1}{8}(x y+y z+z x-x y z)-\frac{1}{3}(x y z)^{\frac{2}{3}} \\
= & \frac{1}{8} x y z\left[\frac{1}{x}+\frac{1}{y}+\frac{1}{z}-1-\frac{8}{3}(x y z)^{-\frac{1}{3}}\right] \geqslant 0,
\end{aligned}
$$

所以
$$
(1-x)(1-y)(1-z) \geqslant \frac{8}{3}(x y z)^{\frac{2}{3}}
$$

由 $A_{3} \geqslant G_{3}$, 得
$$
(x y z)^{\frac{1}{3}} \leqslant \frac{1}{3}(x+y+z)=\frac{1}{3} .
$$

于是
$$
\frac{x y z}{(1-x)(1-y)(1-z)} \leqslant \frac{x y z}{\frac{8}{3}(x y z)^{\frac{2}{3}}}=\frac{3}{8}(x y z)^{\frac{1}{3}} \leqslant \frac{1}{8}
$$

进一步, 设 $a_{1}, a_{2}, \cdots, a_{n}$ 为正实数, 实数 $r>0$, 则称
$$
M_{r}=\left(\frac{\sum_{i=1}^{n} a_{i}^{r}}{n}\right)^{\frac{1}{r}}
$$

为 $a_{1}, a_{2}, \cdots, a_{n}$ 的 $r$ 次幂平均值.

对于 $M_{r}$, 我们有幂平均不等式, 即:

对 $\alpha>\beta$, 则 $M_{\alpha} \geqslant M_{\beta}$, 即
$$
\left(\frac{\sum_{i=1}^{n} a_{i}^{\alpha}}{n}\right)^{\frac{1}{\alpha}} \geqslant\left(\frac{\sum_{i=1}^{n} a_{i}^{\beta}}{n}\right)^{\frac{1}{\beta}}
$$

等号成立的充要条件是 $a_{1}=a_{2}=\cdots=a_{n}$.

特别当 $\alpha>1, \beta=1$ 时,
$$
\frac{\sum_{i=1}^{n} a_{i}^{\alpha}}{n} \geqslant\left(\frac{\sum_{i=1}^{n} a_{i}}{n}\right)^{\alpha}
$$

\section*{加权平均值不等式}
设 $a_{i}>0, \alpha_{i}>0,1 \leqslant i \leqslant n, \sum_{i=1}^{n} \alpha_{i}=1$. 由加权琴生不等式和函数 $f(x)=\ln x, x \in \mathbf{R}_{+}$的凹凸性, 得到加权平均值不等式:
$$
a_{1}^{\alpha_{1}} a_{2}^{\alpha_{2}} \cdots a_{n}^{\alpha_{n}} \leqslant \alpha_{1} a_{1}+\alpha_{2} a_{2}+\cdots+\alpha_{n} a_{n} .
$$

由加权平均值不等式得

卡尔松(Carlson)不等式 设 $a_{i j} \geqslant 0,1 \leqslant i \leqslant n, 1 \leqslant j \leqslant m, \alpha_{i} \geqslant 0,1$ $\leqslant i \leqslant m$ 满足 $\alpha_{1}+\alpha_{2}+\cdots+\alpha_{m}=1$, 则
$$
\prod_{j=1}^{m}\left(\sum_{i=1}^{n} a_{i j}\right)^{\alpha_{j}} \geqslant \sum_{i=1}^{n}\left(\prod_{j=1}^{m} a_{i j}^{\alpha_{j}}\right)
$$

事实上, 由齐次性, 不妨设 $a_{1 j}+a_{2 j}+\cdots+a_{n j}=1,1 \leqslant j \leqslant m$, 则需要证明
$$
\sum_{i=1}^{n} \prod_{j=1}^{m} a_{i j}^{a_{j}} \leqslant 1
$$

由加权平均值不等式得
$$
\prod_{j=1}^{m} a_{i j}^{a_{j}} \leqslant \sum_{j=1}^{m} \alpha_{j} a_{i j}, \quad 1 \leqslant i \leqslant n
$$

于是
$$
\sum_{i=1}^{n} \prod_{j=1}^{m} a_{i j}^{a_{j}} \leqslant \sum_{i=1}^{n} \sum_{j=1}^{m} \alpha_{j} a_{i j}=\sum_{j=1}^{m} \alpha_{j} \sum_{i=1}^{n} a_{i j}=\sum_{j=1}^{m} \alpha_{j}=1
$$

我们将在第三章中,进一步讨论卡尔松不等式.

例 9 给定正整数 $k$, 当 $x^{k}+y^{k}+z^{k}=1$ 时, 求 $x^{k+1}+y^{k+1}+z^{k+1}$ 的最小值.

解 由假设和幂平均不等式, 得
$$
\left(\frac{x^{k+1}+y^{k+1}+z^{k+1}}{3}\right)^{\frac{1}{k+1}} \geqslant\left(\frac{x^{k}+y^{k}+z^{k}}{3}\right)^{\frac{1}{k}}=\left(\frac{1}{3}\right)^{\frac{1}{k}}
$$

所以
$$
x^{k+1}+y^{k+1}+z^{k+1} \geqslant 3\left(\frac{1}{3}\right)^{\frac{k+1}{k}}=3^{-\frac{1}{k}}
$$

当 $x=y=z=3^{-\frac{1}{k}}$ 时等号成立, 所以最小值为 $3^{-\frac{1}{k}}$.

例 10 设三角形三边长分别为 $a 、 b 、 c$, 面积为 $S$, 则
$$
a^{n}+b^{n}+c^{n} \geqslant 2^{n} \cdot 3^{\frac{4 n}{4}} S^{\frac{n}{2}}, n \in \mathbf{N}
$$

证明 当 $n=1$ 时, $a+b+c \geqslant 2 \sqrt{3 \sqrt{3} S}$ 是常见的几何不等式, 即 $n=1$时成立.

假设当 $n=k$ 时命题成立, 即有不等式
$$
a^{k}+b^{k}+c^{k} \geqslant 2^{k} \cdot 3^{\frac{4-k}{4}} S^{\frac{k}{2}}
$$

则当 $n=k+1$ 时, 由幂平均不等式,
$$
\left(\frac{a_{1}^{a}+a_{2}^{a}+\cdots+a_{n}^{a}}{n}\right)^{\frac{1}{a}} \geqslant\left(\frac{a_{1}^{\beta}+a_{2}^{\beta}+\cdots+a_{n}^{\beta}}{n}\right)^{\frac{1}{\beta}}
$$

其中 $a_{i} \in \mathbf{R}_{+}, i=1,2, \cdots, n, \alpha \geqslant \beta$.

得
$$
\left(\frac{a^{k+1}+b^{k+1}+c^{k+1}}{3}\right)^{\frac{1}{k+1}} \geqslant\left(\frac{a^{k}+b^{k}+c^{k}}{3}\right)^{\frac{1}{k}} \geqslant\left(\frac{2^{k}}{3} \cdot 3^{\frac{4-k}{4}} \cdot S^{\frac{k}{2}}\right)^{\frac{1}{k}}
$$

从而
$$
a^{k+1}+b^{k+1}+c^{k+1} \geqslant 3\left(\frac{2^{k}}{3} \cdot 3^{\frac{4-k}{4}} \cdot S^{\frac{k}{2}}\right)^{\frac{k+1}{k}}=2^{k+1} \cdot 3^{\frac{4-k+1)}{4}} S^{\frac{k+1}{2}}
$$

即当 $n=k+1$ 时, 原不等式成立.

例 11 设整数 $n \geqslant 2, a_{i} \in \mathbf{R}_{+}, 1 \leqslant i \leqslant n$. 证明:
$$
\left(\frac{\sum_{i=1}^{n} G_{i}}{n A_{n}}\right)^{\frac{1}{n}}+\frac{G_{n}}{\sum_{i=1}^{n} G_{i}} \leqslant \frac{n+1}{n}
$$

其中 $A_{n}=\frac{1}{n} \sum_{i=1}^{n} a_{i}, G_{i}=\sqrt[i]{a_{1} \cdots a_{i}}, 1 \leqslant i \leqslant n$.

证明 令 $G_{0}=1$. 则 $a_{i}=\frac{G_{i}^{i}}{G_{i-1}^{i-1}}, 1 \leqslant i \leqslant n$.

由齐次性, 不妨设 $\sum_{i=1}^{n} G_{i}=\sum_{i=1}^{n} \sqrt[i]{a_{1} \cdots a_{i}}=1$, 则原不等式等价于
$$
\left(\sum_{j=1}^{n} \frac{G_{j}^{j}}{G_{j-1}^{j-1}}\right)^{-\frac{1}{n}}+G_{n} \leqslant \frac{n+1}{n}
$$

易知函数 $f(x)=x^{-\frac{1}{n}}$ 在 $(0,+\infty)$ 上是凸函数, 我们有
$$
\begin{aligned}
\left(\sum_{j=1}^{n} \frac{G_{j}^{j}}{G_{j-1}^{j-1}}\right)^{-\frac{1}{n}} & =\left(\sum_{j=1}^{n} G_{j} \frac{G_{j}^{j-1}}{G_{j-1}^{j-1}}\right)^{-\frac{1}{n}} \\
& \leqslant \sum_{j=1}^{n} G_{j}\left(\frac{G_{j}^{j-1}}{G_{j-1}^{j-1}}\right)^{-\frac{1}{n}}(\text { 加权琴生不等式 }) \\
& =\sum_{j=1}^{n}\left(G_{j-1}\right)^{\frac{i-1}{n}}\left(G_{j}\right)^{1-\frac{i-1}{n}} \\
& \leqslant \sum_{j=1}^{n}\left(\frac{j-1}{n} G_{j-1}+\left(1-\frac{j-1}{n}\right) G_{j}\right)(\text { 加权平均值不等式 }) \\
& =\frac{n+1}{n} \sum_{j=1}^{n} G_{j}+\sum_{j=1}^{n}\left(\frac{j-1}{n} G_{j-1}-\frac{j}{n} G_{j}\right) \\
& =\frac{n+1}{n}-G_{n}
\end{aligned}
$$

从而
$$
\left(\sum_{j=1}^{n} \frac{G_{j}^{j}}{G_{j-1}^{i-1}}\right)^{-\frac{1}{n}}+G_{n} \leqslant \frac{n+1}{n}
$$

由于一般的幂平均不等式在竞赛中很少出现, 这里就不展开讨论了.

等式或不等式的变形, 是证明数学问题和运算中常使用的方法和技巧.前面我们介绍了平均值不等式的证明和应用, 但对于某些问题, 通过变形处理可能比运用基本定理来证明更简单,下面我们以一个例子来说明, 希望读\\
者能有所体会和了解.

例 12 设 $a 、 b 、 c$ 是正实数, 证明:
$$
\frac{(2 a+b+c)^{2}}{2 a^{2}+(b+c)^{2}}+\frac{(2 b+a+c)^{2}}{2 b^{2}+(c+a)^{2}}+\frac{(2 c+a+b)^{2}}{2 c^{2}+(a+b)^{2}} \leqslant 8
$$

证法一 通过变形直接证明. 不妨假设 $a+b+c=1$. 则原不等式等价于
$$
\begin{gathered}
\frac{(1+a)^{2}}{2 a^{2}+(1-a)^{2}}+\frac{(1+b)^{2}}{2 b^{2}+(1-b)^{2}}+\frac{(1+c)^{2}}{2 c^{2}+(1-c)^{2}} \leqslant 8 \\
\frac{a^{2}+2 a+1}{3 a^{2}-2 a+1}+\frac{b^{2}+2 b+1}{3 b^{2}-2 b+1}+\frac{c^{2}+2 c+1}{3 c^{2}-2 c+1} \leqslant 8
\end{gathered}
$$

两边同乘以 3 , 得
$$
\frac{3 a^{2}+6 a+3}{3 a^{2}-2 a+1}+\frac{3 b^{2}+6 b+3}{3 b^{2}-2 b+1}+\frac{3 c^{2}+6 c+3}{3 c^{2}-2 c+1} \leqslant 24
$$

消除分子的二次项,得
$$
\frac{8 a+2}{3 a^{2}-2 a+1}+\frac{8 b+2}{3 b^{2}-2 b+1}+\frac{8 c+2}{3 c^{2}-2 c+1} \leqslant 21
$$

因为 $3 x^{2}-2 x+1=3\left(x-\frac{1}{3}\right)^{2}+\frac{2}{3} \geqslant \frac{2}{3}$, 所以
$$
\begin{aligned}
& \frac{8 a+2}{3 a^{2}-2 a+1}+\frac{8 b+2}{3 b^{2}-2 b+1}+\frac{8 c+2}{3 c^{2}-2 c+1} \\
\leqslant & \frac{8 a+2}{\frac{2}{3}}+\frac{8 b+2}{\frac{2}{3}}+\frac{8 c+2}{\frac{2}{3}}=\frac{3}{2}(8+6) \leqslant 21
\end{aligned}
$$

故命题成立.

证法二 对一个 $n$ 个变量的函数 $f$, 定义它的对称和
$$
\sum_{\mathrm{sym}} f\left(x_{1}, x_{2}, \cdots, x_{n}\right)=\sum_{\sigma} f\left(x_{\sigma(1)}, x_{\sigma(2)}, \cdots, x_{\sigma(n)}\right)
$$

这里 $\sigma$ 是 $1,2, \cdots, n$ 的所有的排列, sym 表示对称求和. 例如, 将 $x_{1}$ 、 $x_{2} 、 x_{3}$ 记为 $x 、 y 、 z$, 当 $n=3$ 时, 有
$$
\begin{gathered}
\sum_{\mathrm{sym}} x^{3}=2 x^{3}+2 y^{3}+2 z^{3} \\
\sum_{\mathrm{sym}} x^{2} y=x^{2} y+y^{2} z+z^{2} x+x^{2} z+y^{2} x+z^{2} y
\end{gathered}
$$
$$
\sum_{\text {sym }} x y z=6 x y z
$$

则
$$
8-\frac{(2 a+b+c)^{2}}{2 a^{2}+(b+c)^{2}}+\frac{(2 b+a+c)^{2}}{2 b^{2}+(c+a)^{2}}+\frac{(2 c+a+b)^{2}}{2 c^{2}+(a+b)^{2}}=\frac{A}{B}
$$

其中 $B>0$,
$$
A=\sum_{\text {sym }}\left(4 a^{6}+4 a^{5} b+a^{4} b^{2}+5 a^{4} b c+5 a^{3} b^{3}-26 a^{3} b^{2} c+7 a^{2} b^{2} c^{2}\right)
$$

下面证明 $A>0$.

由加权平均值不等式, 得
$$
4 a^{6}+b^{6}+c^{6} \geqslant 6 a^{4} b c, 3 a^{5} b+3 a^{5} c+b^{5} a+c^{5} a \geqslant 8 a^{4} b c
$$

得
$$
\sum_{\text {sym }} 6 a^{6} \geqslant \sum_{\text {sym }} 6 a^{4} b c, \sum_{\text {sym }} 8 a^{5} b \geqslant \sum_{\text {sym }} 8 a^{4} b c
$$

于是
$$
\sum_{\text {sym }}\left(4 a^{6}+4 a^{5} b+5 a^{4} b c\right) \geqslant \sum_{\text {sym }} 13 a^{4} b c
$$

再由平均值不等式, 得
$$
a^{4} b^{2}+b^{4} c^{2}+c^{4} a^{2} \geqslant 3 a^{2} b^{2} c^{2}, a^{3} b^{3}+b^{3} c^{3}+c^{3} a^{3} \geqslant 3 a^{2} b^{2} c^{2}
$$

从而
$$
\sum_{\text {sym }}\left(a^{4} b^{2}+5 a^{3} b^{3}\right) \geqslant \sum_{\text {sym }} 6 a^{2} b^{2} c^{2}
$$

或者
$$
\sum_{\mathrm{sym}}\left(a^{4} b^{2}+5 a^{3} b^{3}+7 a^{2} b^{2} c^{2}\right) \geqslant \sum_{\mathrm{sym}} 13 a^{2} b^{2} c^{2}
$$

回顾舒尔不等式,
$$
\begin{aligned}
& a^{3}+b^{3}+c^{3}+3 a b c-\left(a^{2} b+b^{2} c+c^{2} a+a b^{2}+b c^{2}+c a^{2}\right) \\
= & a(a-b)(a-c)+b(b-a)(b-c)+c(c-a)(c-b) \geqslant 0
\end{aligned}
$$

或者
$$
\sum_{\text {sym }}\left(a^{3}-2 a^{2} b+a b c\right) \geqslant 0
$$

于是
$$
\sum_{\mathrm{sym}}\left(13 a^{4} b c-26 a^{3} b^{2} c+13 a^{2} b^{2} c^{2}\right) \geqslant 13 a b c \sum_{\mathrm{sym}}\left(a^{3}-2 a^{2} b+a b c\right) \geqslant 0
$$

综上可得 $A>0$. 证毕.

注 此题还有多种不同的证明方法, 感兴趣的读者不妨自己试试.

\section*{2. 5 带参数的平均值不等式}
引进适当的参数, 是解决不等式问题的重要技巧.

一般地, 当 $a_{i}>0, \lambda_{i}>0(i=1,2, \cdots, n)$, 且 $\prod_{i=1}^{n} \lambda_{i}=1$ 时, 我们有
$$
\frac{1}{n} \sum_{i=1}^{n} \lambda_{i} a_{i} \geqslant \sqrt[n]{a_{1} a_{2} \cdots a_{n}}
$$

例 1 设 $a 、 b 、 c 、 d$ 是不全为 0 的实数,求
$$
f=\frac{a b+2 b c+c d}{a^{2}+b^{2}+c^{2}+d^{2}}
$$

的最大值.

解 假设 $f$ 的最大值为 $M$, 则
$$
a b+2 b c+c d \leqslant M\left(a^{2}+b^{2}+c^{2}+d^{2}\right)
$$

因此, 要建立一个上面形式的不等式, 并找一组 $a 、 b 、 c 、 d$ 的值, 使不等式等号成立.

设 $\alpha, \beta, \gamma>0$, 则
$$
\frac{\alpha}{2} a^{2}+\frac{b^{2}}{2 \alpha} \geqslant a b, \beta b^{2}+\frac{c^{2}}{\beta} \geqslant 2 b c, \frac{\gamma}{2} c^{2}+\frac{d^{2}}{2 \gamma} \geqslant c d
$$

将上面三式相加, 得
$$
\begin{gathered}
\frac{\alpha}{2} a^{2}+\left(\frac{1}{2 \alpha}+\beta\right) b^{2}+\left(\frac{1}{\beta}+\frac{\gamma}{2}\right) c^{2}+\frac{d^{2}}{2 \gamma} \geqslant a b+2 b c+c d . \\
\text { 令 } \frac{\alpha}{2}=\frac{1}{2 \alpha}+\beta=\frac{1}{\beta}+\frac{\gamma}{2}=\frac{1}{2 \gamma}, \text { 得 } \gamma=\frac{1}{\alpha}, \beta=\frac{1}{2}\left(\alpha-\frac{1}{\alpha}\right), \text { 及 } \alpha^{4}-
\end{gathered}
$$

$6 \alpha^{2}+1=0$, 解方程得 $\alpha=\sqrt{2}+1$ (另外三个解不合要求), 于是

即
$$
\begin{gathered}
\frac{\sqrt{2}+1}{2}\left(a^{2}+b^{2}+c^{2}+d^{2}\right) \geqslant a b+2 b c+c d \\
f \leqslant \frac{\sqrt{2}+1}{2} .
\end{gathered}
$$

又当 $a=d=1, b=c=\sqrt{2}+1$ 时, 不等式等号成立, 故 $f$ 的最大值为 $\frac{\sqrt{2}+1}{2}$.\\
例 2 求出最大的正数 $\lambda$, 使得对于满足 $x^{2}+y^{2}+z^{2}=1$ 的任何实数 $x$ 、 $y 、 z$ 成立不等式:
$$
|\lambda x y+y z| \leqslant \frac{\sqrt{5}}{2}
$$

解 由于
$$
\begin{aligned}
1 & =x^{2}+y^{2}+z^{2}=x^{2}+\frac{\lambda^{2}}{1+\lambda^{2}} y^{2}+\frac{1}{1+\lambda^{2}} y^{2}+z^{2} \\
& \geqslant \frac{2}{\sqrt{1+\lambda^{2}}}(\lambda|x y|+|y z|) \\
& \geqslant \frac{2}{\sqrt{1+\lambda^{2}}}(|\lambda x y+y z|)
\end{aligned}
$$

且当 $y=\frac{\sqrt{2}}{2}, x=\frac{\sqrt{2} \lambda}{2 \sqrt{\lambda^{2}+1}}, z=\frac{\sqrt{2}}{2 \sqrt{\lambda^{2}+1}}$ 时, 上述两个等号可同时取到. 因此 $\frac{\sqrt{1+\lambda^{2}}}{2}$ 是 $|\lambda x y+y z|$ 的最大值. 令 $\frac{\sqrt{1+\lambda^{2}}}{2} \leqslant \frac{\sqrt{5}}{2}$, 解得 $|\lambda| \leqslant 2$.

故 $\lambda$ 的最大值为 2 .

例 3 已知 $\alpha, \beta, \gamma>0$, 且
$$
\frac{1}{\alpha^{2}+1}+\frac{1}{\beta^{2}+1}+\frac{1}{\gamma^{2}+1}=1
$$

求函数 $u=\frac{\alpha x y+\beta y z+\gamma z x}{x^{2}+y^{2}+z^{2}}$ 的最大值.

解 对于任意正实数 $a 、 b 、 c$, 有
$$
\begin{aligned}
& x^{2}+y^{2}+z^{2} \\
= & \frac{b}{b+c} x^{2}+\frac{c}{b+c} x^{2}+\frac{a}{a+c} y^{2}+\frac{c}{a+c} y^{2}+\frac{a}{a+b} z^{2}+\frac{b}{a+b} z^{2} \\
= & \left(\frac{b}{b+c} x^{2}+\frac{a}{a+c} y^{2}\right)+\left(\frac{c}{c+a} y^{2}+\frac{b}{b+a} z^{2}\right)+\left(\frac{a}{a+b} z^{2}+\frac{c}{c+b} x^{2}\right) \\
\geqslant & 2\left(\sqrt{\frac{a b}{(a+c)(b+c)}} x y+\sqrt{\frac{b c}{(b+a)(c+a)}} y z+\sqrt{\frac{c a}{(c+b)(a+b)}} z x\right) \\
= & 2 \sqrt{\frac{a b c}{(a+b)(b+c)(c+a)}}\left(\sqrt{\frac{a+b}{c}} x y+\sqrt{\frac{b+c}{a}} y z+\sqrt{\frac{c+a}{b}} z x\right)
\end{aligned}
$$

当且仅当
$$
\left\{\begin{array}{l}
\sqrt{\frac{b}{b+c}} x=\sqrt{\frac{a}{a+c}} y \\
\sqrt{\frac{c}{c+a}} y=\sqrt{\frac{b}{b+a}} z \\
\sqrt{\frac{a}{a+b}} z=\sqrt{\frac{c}{c+b}} x
\end{array}\right.
$$

亦即 $\frac{x}{\sqrt{a b+a c}}=\frac{y}{\sqrt{b a+b c}}=\frac{z}{\sqrt{c a+c b}}$ 时, 上式取等号.
$$
\begin{array}{r}
\text { 令 } \alpha=\sqrt{\frac{a+b}{c}}, \beta=\sqrt{\frac{b+c}{a}}, \gamma=\sqrt{\frac{c+a}{b}} \text {, 则 } \\
\frac{1}{\alpha^{2}+1}+\frac{1}{\beta^{2}+1}+\frac{1}{\gamma^{2}+1}=1,
\end{array}
$$

且
$$
\begin{gathered}
x^{2}+y^{2}+z^{2} \geqslant \frac{2}{\alpha \beta \gamma}(\alpha x y+\beta y z+\gamma z x) \\
\frac{\alpha x y+\beta y z+\gamma z x}{x^{2}+y^{2}+z^{2}} \leqslant \frac{\alpha \beta \gamma}{2}
\end{gathered}
$$

所以, $u$ 的最大值为 $\frac{\alpha \beta \gamma}{2}$.

注 (1) 如果 $\frac{1}{\alpha^{2}+k}+\frac{1}{\beta^{2}+k}+\frac{1}{\gamma^{2}+k}=\frac{1}{k}(\alpha, \beta, \gamma, k>0)$, 那么 $u=$ $\frac{\alpha x y+\beta y z+\gamma z x}{x^{2}+y^{2}+z^{2}}$ 有最大值 $\frac{\alpha \beta \gamma}{2 k}$.

这是因为
$$
\begin{gathered}
\frac{1}{\left(\frac{\alpha}{\sqrt{k}}\right)^{2}+1}+\frac{1}{\left(\frac{\beta}{\sqrt{k}}\right)^{2}+1}+\frac{1}{\left(\frac{\gamma}{\sqrt{k}}\right)^{2}+1}=1 \\
\frac{\frac{\alpha}{\sqrt{k}} x y+\frac{\beta}{\sqrt{k}} y z+\frac{\gamma}{\sqrt{k}} z x}{x^{2}+y^{2}+z^{2}} \leqslant \frac{\frac{\alpha}{\sqrt{k}} \cdot \frac{\beta}{\sqrt{k}} \cdot \frac{\gamma}{\sqrt{k}}}{2} \\
\frac{\alpha x y+\beta y z+\gamma z x}{x^{2}+y^{2}+z^{2}} \leqslant \frac{\alpha \beta \gamma}{2 k}
\end{gathered}
$$

化简整理后

(2) 若\\
$\frac{k_{1} k_{2}}{\alpha^{2}+k_{1} k_{2} k}+\frac{k_{2} k_{3}}{\beta^{2}+k_{2} k_{3} k}+\frac{k_{1} k_{3}}{\gamma^{2}+k_{1} k_{3} k}=\frac{1}{k}\left(\alpha, \beta, \gamma, k_{1}, k_{2}, k_{3}, k>0\right)$,

则函数 $\frac{\alpha x y+\beta y z+\gamma z x}{x^{2}+y^{2}+z^{2}}$ 有最大值 $\frac{\alpha \beta \gamma}{2 k_{1} k_{2} k_{3} k}$.

事实上, 只须令 $x^{\prime}=\sqrt{k_{1}} x, y^{\prime}=\sqrt{k_{2}} y, z^{\prime}=\sqrt{k_{3}} z, \alpha^{\prime}=\frac{\alpha}{\sqrt{k_{1} k_{2}}}$, $\beta^{\prime}=\frac{\beta}{\sqrt{k_{2} k_{3}}}, \gamma^{\prime}=\frac{\gamma}{\sqrt{k_{1} k_{3}}}$ 即可化归为 $(1)$ 的情形.

例 4 设 $a$ 为实数, 求函数 $f(x)=|\sin x(a+\cos x)|(x \in \mathbf{R})$ 的最大值.

解 设 $\alpha$ 为参数, 使得
$$
\begin{aligned}
f^{2}(x) & =\frac{1}{\alpha^{2}} \sin ^{2} x(a \alpha+\alpha \cos x)^{2} \leqslant \frac{1}{\alpha^{2}} \sin ^{2} x\left(\alpha^{2}+\cos ^{2} x\right)\left(a^{2}+\alpha^{2}\right) \\
& \leqslant \frac{1}{\alpha^{2}}\left(\frac{\sin ^{2} x+\alpha^{2}+\cos ^{2} x}{2}\right)^{2}\left(a^{2}+\alpha^{2}\right)=\frac{1}{\alpha^{2}}\left(\frac{\alpha^{2}+1}{2}\right)^{2}\left(a^{2}+\alpha^{2}\right)
\end{aligned}
$$

当且仅当 $\alpha^{2}=a \cos x, \sin ^{2} x=\alpha^{2}+\cos ^{2} x$ 时等号成立.

消除 $x$, 得 $2 \alpha^{4}+a^{2} \alpha^{2}-a^{2}=0$.

解方程, 得 $\alpha^{2}=\frac{1}{4}\left(\sqrt{a^{4}+8 a^{2}}-a^{2}\right)$, 从而 $\cos x=\frac{1}{4}\left(\sqrt{a^{2}+8}-a\right)$.

所以当 $x=2 k \pi \pm \arccos \left[\frac{1}{4}\left(\sqrt{a^{2}+8}-a\right)\right](k \in \mathbf{Z})$ 时,
$$
f(x)_{\max }=\frac{\sqrt{a^{4}+8 a^{2}}-a^{2}+4}{8} \cdot \sqrt{\frac{\sqrt{a^{4}+8 a^{2}}+a^{2}+2}{2}}
$$

例 5 设 $x, y, z \in \mathbf{R}_{+}$, 且 $x^{4}+y^{4}+z^{4}=1$, 求
$$
f(x, y, z)=\frac{x^{3}}{1-x^{8}}+\frac{y^{3}}{1-y^{8}}+\frac{z^{3}}{1-z^{8}}
$$

的最小值.

解 将原式变形为
$$
f(x, y, z)=\frac{x^{4}}{x\left(1-x^{8}\right)}+\frac{y^{4}}{y\left(1-y^{8}\right)}+\frac{z^{4}}{z\left(1-z^{8}\right)}
$$

对于 $w \in(0,1)$, 令 $\phi(w)=w\left(1-w^{8}\right)$, 先求 $\phi(w)$ 的最大值.

选一参数 $a$, 并利用 $G_{9} \leqslant A_{9}$, 得
$$
\begin{aligned}
a(\phi(w))^{8} & =a w^{8}\left(1-w^{8}\right)^{8} \\
& \leqslant\left[\frac{1}{9}\left(a w^{8}+8\left(1-w^{8}\right)\right)\right]^{9} \\
& =\left[\frac{1}{9}\left(8+(a-8) w^{8}\right)\right]^{9}
\end{aligned}
$$

取 $a=8$, 得
$$
8(\phi(w))^{8} \leqslant\left(\frac{8}{9}\right)^{9}
$$

由于 $\phi(w)>0$, 从而
$$
\phi(w) \leqslant \frac{8}{\sqrt[4]{3^{9}}}
$$

于是
$$
f(x, y, z) \geqslant \frac{x^{4}+y^{4}+z^{4}}{8} \cdot \sqrt[4]{3^{9}}=\frac{9}{8} \sqrt[4]{3}
$$

当 $x=y=z=\frac{1}{\sqrt[4]{3}}$ 时, 等号成立.

故 $f(x, y, z)$ 的最小值为 $\frac{9}{8} \sqrt[4]{3}$.

注 这里选择 $a=8$, 是为了消除变量 $w^{8}$, 使得右边为常数.

例 6 求最小的正整数 $k$, 使得对满足 $0 \leqslant a \leqslant 1$ 的所有 $a$ 和所有正整数 $n$, 都有不等式
$$
a^{k}(1-a)^{n} \leqslant \frac{1}{(n+1)^{3}}
$$

解 先设法消除参数 $a$, 然后求 $k$ 的最小值.

由平均值不等式, 得

所以
$$
\begin{gathered}
\sqrt[n+k]{a^{k}\left[\frac{k}{n}(1-a)\right]^{n}} \leqslant \frac{k a+n\left[\frac{k}{n}(1-a)\right]}{k+n}=\frac{k}{k+n} \\
a^{k}(1-a)^{n} \leqslant \frac{k^{k} n^{n}}{(n+k)^{n+k}}
\end{gathered}
$$

当且仅当 $a=\frac{k(1-a)}{n}$, 即 $a=\frac{k}{n+k}$ 时, 等号成立.

于是我们要求出最小的正整数 $k$, 使得对任何正整数 $n$ 都有
$$
\frac{k^{k} n^{n}}{(n+k)^{n+k}}<\frac{1}{(1+n)^{3}}
$$

当 $k=1$ 时, 取 $n=1$, 上式为 $\frac{1}{2^{2}}<\frac{1}{2^{3}}$, 矛盾.

当 $k=2$ 时, 取 $n=1$, 则 $\frac{4}{27}<\frac{1}{8}$, 亦矛盾.

当 $k=3$ 时, 取 $n=3$, 则 $\frac{1}{64}<\frac{1}{64}$, 矛盾.

因此 $k \geqslant 4$. 下面证明 $k=4$ 时命题成立, 即
$$
4^{4} n^{n}(n+1)^{3}<(n+4)^{n+4}
$$

当 $n=1,2,3$ 时, 容易证明成立.

当 $n \geqslant 4$ 时, 再由平均值不等式, 得
$$
\begin{aligned}
\sqrt[n+4]{4^{4} n^{n}(n+1)^{3}} & =\sqrt[n+4]{16(2 n)(2 n)(2 n)(2 n) n^{n-4}(n+1)^{3}} \\
& \leqslant \frac{16+8 n+n(n-4)+3(n+1)}{n+4} \\
& =\frac{n^{2}+7 n+19}{n+4}<\frac{n^{2}+8 n+16}{n+4}=n+4
\end{aligned}
$$

故 $k$ 的最小值为 4.

注 在第一次利用平均值不等式时, 引进参数 $\alpha=\frac{k}{n}$, 消除 $a$, 具有很强的技巧性.

例 7 设整数 $n \geqslant 2, a_{i}, b_{i} \geqslant 0,1 \leqslant i \leqslant n$. 证明:
$$
\left(\frac{n}{n-1}\right)^{n-1}\left(\frac{1}{n} \sum_{i=1}^{n} a_{i}^{2}\right)+\left(\frac{1}{n} \sum_{i=1}^{n} b_{i}\right)^{2} \geqslant\left[\left(a_{1}^{2}+b_{1}^{2}\right)\left(a_{2}^{2}+b_{2}^{2}\right) \cdots\left(a_{n}^{2}+b_{n}^{2}\right)\right]^{\frac{1}{n}}
$$

证明 令 $\lambda=\left(\frac{n}{n-1}\right)^{n-1}$ 对 $i \in\{1,2, \cdots, n\}$, 将 $a_{j}, b_{j}, j \neq i$ 都固定,同时固定 $p=a_{i}^{2}+b_{i}^{2}$. 于是, 不等式左边为 $b_{i}$ 的二次函数 $b_{i} \in[0, \sqrt{p}]$, 其首项系数 $-\frac{\lambda}{n}+\frac{1}{n^{2}}<0$. 从而, 左边的最小值在端点取得, 即 $b_{i}=0$ 或 $a_{i}=0$.

对每个 $i$, 进行上述调整, 假设 $a_{i} b_{i}=0,1 \leqslant i \leqslant n$.

(1) 若 $a_{i}=0,1 \leqslant i \leqslant n$, 则由平均值不等式得
$$
\left(\frac{1}{n} \sum_{i=1}^{n} b_{i}\right)^{2} \geqslant\left(b_{1} b_{2} \cdots b_{n}\right)^{\frac{2}{n}}
$$

(2) 若 $b_{i}=0,1 \leqslant i \leqslant n$, 则由平均值不等式得
$$
\lambda\left(\frac{1}{n} \sum_{i=1}^{n} a_{i}^{2}\right) \geqslant \frac{1}{n} \sum_{i=1}^{n} a_{i}^{2} \geqslant\left(a_{1} a_{2} \cdots a_{n}\right)^{\frac{2}{n}}
$$

(3)若 $b_{1}=\cdots=b_{k}=0, a_{k+1}=\cdots=a_{n}=0,1 \leqslant k \leqslant n$.记 $a^{k}=a_{1} a_{2} \cdots a_{k}, b^{n-k}=b_{k+1} b_{k+2} \cdots b_{n}, a, b \geqslant 0$.

由平均值不等式得 $\sum_{i=1}^{k} a_{i}^{2} \geqslant k a^{2}, \sum_{i=k+1}^{n} b_{i} \geqslant(n-k) b$.

只需证明 $\quad \frac{\lambda k}{n} a^{2}+\frac{(n-k)^{2}}{n^{2}} b^{2} \geqslant a^{\frac{2 k}{n}} b \frac{2(n-k)}{n}$.

由于
$$
\frac{\lambda k}{n} a^{2}+\frac{(n-k)^{2}}{n^{2}} b^{2} \geqslant \lambda^{\frac{k}{n}} a^{\frac{2 k}{n}} \cdot\left(\frac{n-k}{n}\right)^{\frac{n-k}{n}} b^{\frac{2(n-k)}{n}}
$$

所以, 只需证明 $\lambda^{\frac{k}{n}}\left(\frac{n-k}{n}\right)^{\frac{n-k}{n}} \geqslant 1$, 即 $\left(\frac{n}{n-k}\right)^{n-k} \leqslant \lambda^{k}$,

由平均值不等式 $\left(\frac{n}{n-k}\right)^{n-k} \leqslant\left(\frac{n+(n k-n)}{n k-k}\right)^{n k-k}=\left(\frac{n}{n-1}\right)^{(n-1) k}=\lambda^{k}$.

故命题成立.

例 8 设 $a 、 b 、 c$ 为某三角形三边之长. 令
$$
A=\sum_{\mathrm{cyc}} \frac{a^{2}+b c}{b+c}, B=\sum_{\mathrm{cyc}} \frac{1}{\sqrt{(a+b-c)(b+c-a)}}
$$

其中 “ $\sum_{\text {cyc }} ”$ 表示循环求和. 证明: $A B \geqslant 9$.

证明 设 $a=y+z, b=x+z, c=x+y, x, y, z \in \mathbf{R}_{+}$, 则
$$
\begin{aligned}
B= & \sum_{\text {cyc }} \frac{1}{2 \sqrt{x y}}, \\
A= & \sum_{\text {cyc }} \frac{x^{2}+y^{2}+z^{2}+x y+2 x+3 y z}{2 x+y+z} \\
A B= & \left(\sum_{\text {cyc }} \frac{1}{2 \sqrt{y z}}\right)\left(\sum_{\text {cyc }} \frac{\sum_{\text {cyc }} x^{2}+\sum_{\text {cyc }} x y+2 y z}{2 x+y+z}\right) \\
\geqslant & {\left[\sqrt{\sum_{\text {cyc }} \frac{1}{2 \sqrt{y z}} \frac{\sum_{\text {cyc }} x^{2}+\sum_{\text {cyc }} x y+2 y z}{2 x+y+z}}\right]^{2} } \\
& \frac{1}{2 \sqrt{y z}} \cdot \frac{x^{2}+y^{2}+z^{2}+x y+z x+3 y z}{2 x+y+z} \geqslant 1 \\
& \Leftrightarrow\left(x^{2}+y^{2}+z^{2}+x y+z x+3 y z\right)^{2} \geqslant 4 y z(2 x+y+z)^{2}
\end{aligned}
$$

由于
$$
\Leftrightarrow \sum_{\mathrm{cyc}} x^{4}+3 \sum_{\mathrm{cyc}} x^{2} y^{2}+2 \sum_{\mathrm{sym}} x^{3} y \geqslant 8 x y^{2} z+8 x^{2} z+8 x y z^{2}
$$

由幂平均值不等式得
$$
\begin{aligned}
x^{4}+y^{4}+z^{4} & \geqslant \frac{1}{27}(x+y+z)^{4} \\
& \geqslant x y z(x+y+z) \\
& =x y^{2} z+x^{2} y z+x y z^{2}
\end{aligned}
$$

再由平均值不等式得
$$
\begin{aligned}
& 3 x^{2} y^{2}+3 x^{2} z^{2}+3 y^{2} z^{2} \geqslant 3\left(x y^{2} z+x^{2} y z+x y z^{2}\right) \\
& \quad x^{3} y+x y^{3}+x^{3} z+x z^{3}+y^{3} z+y z^{3}-2 x y z(x+y+z) \\
& =\left(x^{3} y+y z^{3}-x y z^{2}-x^{2} y z\right)+\left(y^{3} z+x^{3} z-x^{2} y z-x y^{2} z\right) \\
& \quad+\left(z^{3} x+x y^{3}-x y^{2} z-x y z^{2}\right) \\
& =y(x+z)(x-z)^{2}+z(x+y)(x-y)^{2}+x(y+z)(y-z)^{2} \geqslant 0
\end{aligned}
$$

从而
$$
A B \geqslant\left(\sqrt{\sum_{\text {cyc }} 1}\right)^{2}=3^{2}=9
$$

故命题成立.

\section*{085}
\section*{习 题 2}
1 已知 $0<a<1,0<b<1$, 且 $a b=\frac{1}{36}$. 求 $u=\frac{1}{1-a}+\frac{1}{1-b}$ 的最小值.

2 设 $a, b, c \in \mathbf{R}_{+}$, 且 $a+b+c=3$. 证明:
$$
\sum_{\mathrm{cyc}} \frac{1}{a \sqrt{2\left(a^{2}+b c\right)}} \geqslant \sum_{\mathrm{cyc}} \frac{1}{a+b c}
$$

其中, “ $\sum_{\text {cyc }} ”$ 表示轮换对称和.

3 设 $x_{i}>0,1 \leqslant i \leqslant n$. 证明:
$$
\left(1+x_{1}\right)\left(1+x_{1}+x_{2}\right) \cdots\left(1+x_{1}+\cdots+x_{n}\right) \geqslant \sqrt{(n+1)^{n+1} x_{1} x_{2} \cdots x_{n}}
$$

4 设 $a_{1}, a_{2}, \cdots, a_{n}(n \geqslant 2)$ 是正实数, 且满足 $a_{1}+a_{2}+\cdots+a_{n}<1$. 证明:
$$
\frac{a_{1} a_{2} \cdots a_{n}\left[1-\left(a_{1}+\cdots+a_{n}\right)\right]}{\left(a_{1}+a_{2}+\cdots+a_{n}\right)\left(1-a_{1}\right)\left(1-a_{2}\right) \cdots\left(1-a_{n}\right)} \leqslant \frac{1}{n^{n+1}}
$$

5 设 $a_{i}, b_{i} \in \mathbf{R}_{+}(i=1,2, \cdots, n)$, 求证:
$$
\left[\prod_{i=1}^{n}\left(a_{i}+b_{i}\right)\right]^{\frac{1}{n}} \geqslant\left(\prod_{i=1}^{n} a_{i}\right)^{\frac{1}{n}}+\left(\prod_{i=1}^{n} b_{i}\right)^{\frac{1}{n}}
$$

其中 $\prod_{i=1}^{n} a_{i}=a_{1} a_{2} \cdots a_{n}$.

6 6设 $x, y \in \mathbf{R}_{+}, x \neq y$. 令
$$
Q=\sqrt{\frac{x^{2}+y^{2}}{2}}, A=\frac{x+y}{2}, G=\sqrt{x y}, H=\frac{2 x y}{x+y} .
$$

证明: $G-H<Q-A<A-G$.

7 设 $a_{1}, a_{2}, \cdots, a_{n+1}$ 为正等差数列 (公差 $d \geqslant 0$ ). 求证:
$$
n\left(\sqrt[n]{\frac{a_{n+1}}{a_{1}}}-1\right) \leqslant \sum_{i=1}^{n} \frac{d}{a_{i}} \leqslant \frac{d}{a_{1}}+(n-1)\left(1-\sqrt[n-1]{\frac{a_{1}}{a_{n}}}\right)
$$

8 设 $x_{1}, x_{2}, \cdots, x_{n}$ 为正有理数,且各不相同. 求证:
$$
\left(\frac{x_{1}^{2}+x_{2}^{2}+\cdots+x_{n}^{2}}{x_{1}+x_{2}+\cdots+x_{n}}\right)^{x_{1}+x_{2}+\cdots+x_{n}}>x_{1}{ }^{x_{1}} x_{2}{ }^{x_{2}} \cdots x_{n}{ }^{x_{n}}
$$

9 已知 $5 n$ 个实数 $r_{i}, s_{i}, t_{i}, u_{i}, v_{i}>1(1 \leqslant i \leqslant n)$, 记
$$
R=\frac{1}{n} \sum_{i=1}^{n} r_{i}, S=\frac{1}{n} \sum_{i=1}^{n} s_{i}, T=\frac{1}{n} \sum_{i=1}^{n} t_{i}, U=\frac{1}{n} \sum_{i=1}^{n} u_{i}, V=\frac{1}{n} \sum_{i=1}^{n} v_{i}
$$

求证: $\prod_{i=1}^{n} \frac{r_{i} s_{i} t_{i} u_{i} v_{i}+1}{r_{i} s_{i} t_{i} u_{i} v_{i}-1} \geqslant\left(\frac{R S T U V+1}{R S T U V-1}\right)^{n}$.

10 设 $a, b, c, x, y, z \in \mathbf{R}_{+}$满足 $\frac{1}{x}+\frac{1}{y}+\frac{1}{z}=1$.

求证: $a^{x}+b^{y}+c^{z} \geqslant \frac{4 a b c x y z}{(x+y+z-3)^{2}}$.

11 设 $x, y, z>0$ 满足(1) $\frac{1}{\sqrt{2}} \leqslant z \leqslant \frac{1}{2} \min \{\sqrt{2} x, \sqrt{3} y\}$, (2) $x+\sqrt{3} z \leqslant \sqrt{6}$,

(3) $\sqrt{3} y+\sqrt{10} z \geqslant 2 \sqrt{5}$.

求函数 $f(x, y, z)=\frac{1}{x^{2}}+\frac{2}{y^{2}}+\frac{3}{z^{2}}$ 的最大值.

12 对任意正数 $a_{1}, a_{2}, \cdots, a_{n}, n \geqslant 2$, 求 $\sum_{i=1}^{n} \frac{a_{i}}{S-a_{i}}$ 的最小值, 其中 $S=$ E্seat

13 设 $x_{i}>0,1 \leqslant i \leqslant n, \sum_{i=1}^{n} x_{i}=1$. 求
$$
f=\min \left\{\frac{x_{1}}{1+x_{1}}, \frac{x_{2}}{1+x_{1}+x_{2}}, \cdots, \frac{x_{n}}{1+x_{1}+\cdots+x_{n}}\right\}
$$

的最大值.

14 设 $x, y, z \geqslant 0$, 且最多一个为零. 求证:
$$
\sum_{\mathrm{cyc}} \sqrt{\frac{x^{2}+256 y z}{y^{2}+z^{2}}} \geqslant 12
$$

15 设正实数 $a_{1}, a_{2}, \cdots, a_{n}(n \geqslant 2)$ 满足 $a_{1}+a_{2}+\cdots+a_{n}=1$, 求
$$
\sum_{i=1}^{n} \frac{a_{i}}{2-a_{i}}
$$

的最小值.

16 设 $a>0, x_{1}, x_{2}, \cdots, x_{n} \in[0, a](n \geqslant 2)$ 且满足
$$
x_{1} x_{2} \cdots x_{n}=\left(a-x_{1}\right)^{2}\left(a-x_{2}\right)^{2} \cdots\left(a-x_{n}\right)^{2} .
$$

求 $x_{1} x_{2} \cdots x_{n}$ 的最大值.

17 设 $n \geqslant 2, x_{1}, x_{2}, \cdots, x_{n}$ 为实数, 且 $\sum_{i=1}^{n} x_{i}^{2}+\sum_{i=1}^{n-1} x_{i} x_{i+1}=1$, 对每个给定的正整数 $k, 1 \leqslant k \leqslant n$, 求 $\left|x_{k}\right|$ 的最大值.

18 证明: 对任意边长为 $a 、 b 、 c$, 且面积为 $S$ 的三角形, 有
$$
\frac{a b+b c+c a}{4 S} \geqslant \sqrt{3}
$$

19 证明: 如果 $A D 、 B E$ 与 $C F$ 是 $\triangle A B C$ 的角平分线,则 $\triangle D E F$ 的面积不超过 $\triangle A B C$ 的面积的四分之一.

20 设 $\triangle A B C$ 的外接圆 $K$ 的半径为 $R$, 内角平分线分别交圆 $K$ 于 $A^{\prime} 、 B^{\prime}$ 、 $C^{\prime}$, 证明: $16 Q^{3} \geqslant 27 R^{4} P$. 其中, $Q 、 P$ 分别为 $\triangle A^{\prime} B^{\prime} C^{\prime}$ 与 $\triangle A B C$ 的面积.

21 设 $\triangle A B C$ 的三边长为 $a 、 b 、 c$, 现将 $A B 、 A C$ 分别延长 $a$ 单位长度, 将 $B C 、 B A$ 分别延长 $b$ 单位长度. $C A 、 C B$ 分别延长 $c$ 单位长度. 设这样得到六个端点所构成的凸多边形面积为 $G, \triangle A B C$ 的面积为 $F$. 证明: $\frac{G}{F} \geqslant 13$

22 设等腰梯形的最大边长为 13 , 周长为 28 .

(1) 设梯形的面积为 27 , 求它的边长;

(2)这种梯形的面积能否等于 $27.001 ?$

23 设圆 $O$ 的半径为 $\frac{1}{2}$, 两弦 $C D$ 和 $E F$ 均与直径 $A B$ 交 $45^{\circ}$. 设 $P 、 Q$ 分别为 $A B$ 与 $C D 、 E F$ 的交点. 求证:
$$
2 P C \cdot Q E+2 P D \cdot Q F \leqslant 1
$$

24 在 $\triangle A B C$ 中, 三条边长分别为 $a 、 b 、 c$, 且 $a 、 b 、 c$ 为有理数, 求证:
$$
\left(1+\frac{b-c}{a}\right)^{a}\left(1+\frac{c-a}{b}\right)^{b}\left(1+\frac{a-b}{c}\right)^{c} \leqslant 1
$$

25 设 $n \geqslant 2$, 求乘积 $x_{1} x_{2} \cdots x_{n}$ 在条件 $x_{i} \geqslant \frac{1}{n}(i=1,2, \cdots, n)$ 与 $x_{1}^{2}+x_{2}^{2}+\cdots+x_{n}^{2}=1$ 下的最大值和最小值.

26 求最小正数 $\lambda$, 使得对于任一三角形的三边长 $a 、 b 、 c$, 只要 $a \geqslant \frac{b+c}{3}$, 就有 $a c+b c-c^{2} \leqslant \lambda\left(a^{2}+b^{2}+3 c^{2}+2 a b-4 b c\right)$.

27 对每个正整数 $n$, 求证:
$$
\sum_{j=1}^{n} \frac{2 j+1}{j^{2}}>n\left[(n+1)^{\frac{2}{n}}-1\right]
$$

28 设 $A 、 B 、 C$ 为三角形的三个内角, 求证:
$$
\sin 3 A+\sin 3 B+\sin 3 C \leqslant \frac{3}{2} \sqrt{3}
$$

29 设 $\alpha 、 \beta 、 \gamma$ 为一个给定三角形的三个内角, 求证:
$$
\csc ^{2} \frac{\alpha}{2}+\csc ^{2} \frac{\beta}{2}+\csc ^{2} \frac{\gamma}{2} \geqslant 12
$$

并求等号成立的条件.

30 设 $x, y, z \geqslant 0$, 且满足 $y z+z x+x y=1$, 求证:
$$
x\left(1-y^{2}\right)\left(1-z^{2}\right)+y\left(1-z^{2}\right)\left(1-x^{2}\right)+z\left(1-x^{2}\right)\left(1-y^{2}\right) \leqslant \frac{4}{9} \sqrt{3}
$$

31 对 $a_{i} \in \mathbf{R}_{+}(i=1,2, \cdots, n)$, 求证: $\sum_{k=1}^{n} \sqrt[k]{a_{1} \cdots a_{k}} \leqslant \mathrm{e} \sum_{k=1}^{n} a_{k}$, 其中 $\mathrm{e}=$ $\lim _{n \rightarrow \infty}\left(1+\frac{1}{n}\right)^{n}$

32 设 $x_{i} \in \mathbf{R}(i=1,2, \cdots, n, n \geqslant 3)$. 令 $p=\sum_{i=1}^{n} x_{i}, q=\sum_{1 \leqslant i<j \leqslant n} x_{i} x_{j}$,\\
求证:

(1) $\frac{n-1}{n} p^{2}-2 q \geqslant 0$;

(2) $\left|x_{i}-\frac{p}{n}\right| \leqslant \frac{n-1}{n} \sqrt{p^{2}-\frac{2 n}{n-1} q}, i=1,2, \cdots, n$.

33 求最大的实数 $\lambda$, 使得当实系数多项式 $f(x)=x^{3}+a x^{2}+b x+c$ 的所有根都是非负实数时, 只要 $x \geqslant 0$, 就有 $f(x) \geqslant \lambda(x-a)^{3}$, 并求等号成立的条件.

34 设 $x 、 y 、 z$ 为互不相同的实数, 求
$$
\left|\frac{1+y z}{y-z}+\frac{1+z x}{z-x}+\frac{1+x y}{x-y}\right| \text { 的最小值. }
$$

35 设 $a, b, c \geqslant 0$, 满足 $a+b+c=1$. 求 $\sqrt{a+b^{2}}+\sqrt{b+c^{2}}+\sqrt{c+a^{2}}$ 的最小值.

36 设 $a, b, c, d>0$. 求证: $\sum_{\text {cyc }} \sqrt{\frac{a^{3}}{a^{3}+15 b c d}} \geqslant 1$.

37 设 $a, b, c>0, a b c=1 . n \geqslant 2, n \in \mathbf{Z}$. 求证:
$$
\frac{a}{\sqrt[n]{b+c}}+\frac{b}{\sqrt[n]{c+a}}+\frac{c}{\sqrt[n]{a+b}} \geqslant \frac{3}{\sqrt[n]{2}}
$$

前面我们介绍了平均值不等式及其在不等式证明中的一些应用, 同时,也介绍了证明不等式的一些方法和技巧. 但是, 任何一个结论的使用, 都有它的局限性, 平均值不等式也是如此. 在不等式的证明过程中, 要求我们了解不等式的性质和证明不等式的常用方法, 需要掌握一些基本的结论和重要的定理, 并能灵活地应用有关知识. 在这里, 我们将再介绍另一个重要的基本不等式, 即柯西不等式, 与平均值不等式类似, 它的表达形式简单, 它的证明方法多样,在应用中具有较强的灵活性和技巧性.

\section*{3. 1 柯西不等式及其证明}
柯西(Cauchy)不等式 设 $a_{1}, a_{2}, \cdots, a_{n}$ 及 $b_{1}, b_{2}, \cdots, b_{n}$ 为任意实数, 则
$$
\left(a_{1} b_{1}+a_{2} b_{2}+\cdots+a_{n} b_{n}\right)^{2} \leqslant\left(a_{1}^{2}+a_{2}^{2}+\cdots+a_{n}^{2}\right)\left(b_{1}^{2}+b_{2}^{2}+\cdots+b_{n}^{2}\right)
$$

当且仅当 $\frac{a_{1}}{b_{1}}=\frac{a_{2}}{b_{2}}=\cdots=\frac{a_{n}}{b_{n}}$ (规定 $a_{i}=0$ 时, $b_{i}=0$ ) 时等号成立.

柯西不等式的证明方法很多, 这里我们选择其中一些简单和具有一定技巧的证明.

证法一

不妨假设 $A_{n}=\sum_{i=1}^{n} a_{i}^{2} \neq 0, C_{n}=\sum_{i=1}^{n} b_{i}^{2} \neq 0$, 令 $x_{i}=\frac{a_{i}}{\sqrt{A_{n}}}, y_{i}=\frac{b_{i}}{\sqrt{C_{n}}}$,则
$$
\sum_{i=1}^{n} x_{i}^{2}=\sum_{i=1}^{n} y_{i}^{2}=1
$$

则原不等式等价于
$$
x_{1} y_{1}+x_{2} y_{2}+\cdots+x_{n} y_{n} \leqslant 1
$$

即
$$
2\left(x_{1} y_{1}+x_{2} y_{2}+\cdots+x_{n} y_{n}\right) \leqslant x_{1}^{2}+x_{2}^{2}+\cdots+x_{n}^{2}+y_{1}^{2}+y_{2}^{2}+\cdots+y_{n}^{2}
$$

又等价于
$$
\left(x_{1}-y_{1}\right)^{2}+\left(x_{2}-y_{2}\right)^{2}+\cdots+\left(x_{n}-y_{n}\right)^{2} \geqslant 0
$$

这个不等式显然成立, 且等号成立的充要条件为 $x_{i}=y_{i}(i=1,2, \cdots$, $n)$, 从而原不等式成立, 且等号成立的充要条件是
$$
b_{i}=k a_{i}\left(k=\frac{\sqrt{C_{n}}}{\sqrt{A_{n}}}\right)
$$

\section*{证法二 (比值法)}
按上述证明方法和记号, 不妨假设 $A_{n} \neq 0, C_{n} \neq 0$, 令 $x_{i}=\frac{\left|a_{i}\right|}{\sqrt{A_{n}}}, y_{i}=$ $\frac{\left|b_{i}\right|}{\sqrt{C_{n}}}$, 则
$$
\sum_{i=1}^{n} x_{i}^{2}=\sum_{i=1}^{n} y_{i}^{2}=1
$$

由于 $\frac{\left|\sum_{i=1}^{n} a_{i} b_{i}\right|}{\sqrt{A_{n}} \cdot \sqrt{C_{n}}} \leqslant \sum_{i=1}^{n} x_{i} y_{i} \leqslant \sum_{i=1}^{n} \frac{1}{2}\left(x_{i}^{2}+y_{i}^{2}\right)$
$$
=\frac{1}{2}\left(\sum_{i=1}^{n} x_{i}^{2}+\sum_{i=1}^{n} y_{i}^{2}\right)=1
$$

且等号成立当且仅当
$$
\begin{aligned}
\left|\sum_{i=1}^{n} a_{i} b_{i}\right| & =\sum_{i=1}^{n}\left|a_{i} b_{i}\right| \\
\frac{a_{i}^{2}}{\sum_{i=1}^{n} a_{i}^{2}} & =\frac{b_{i}^{2}}{\sum_{i=1}^{n} b_{i}^{2}}
\end{aligned}
$$

由第一个条件表明 $a_{1} b_{1}, a_{2} b_{2}, \cdots, a_{n} b_{n}$ 同号. 第二个条件成立的充分必要条件是 $\frac{a_{i}^{2}}{b_{i}^{2}}=\frac{A_{n}}{C_{n}}$, 即 $\frac{\left|a_{i}\right|}{\left|b_{i}\right|}$ 为常数.

由于 $a_{1} b_{1}, a_{2} b_{2}, \cdots, a_{n} b_{n}$ 同号, 从而命题成立.

证法三 (比值法, 类似证法二)

令 $A_{n}=a_{1}^{2}+a_{2}^{2}+\cdots+a_{n}^{2}, B_{n}=a_{1} b_{1}+a_{2} b_{2}+\cdots+a_{n} b_{n}, C_{n}=b_{1}^{2}+$ $b_{2}^{2}+\cdots+b_{n}^{2}$, 则
$$
\begin{aligned}
\frac{A_{n} C_{n}}{B_{n}^{2}}+1 & =\sum_{i=1}^{n} \frac{a_{i}^{2} C_{n}}{B_{n}^{2}}+\sum_{i=1}^{n} \frac{b_{i}^{2}}{C_{n}} \\
& =\sum_{i=1}^{n}\left(\frac{a_{i}^{2} C_{n}}{B_{n}^{2}}+\frac{b_{i}^{2}}{C_{n}}\right) \\
& \geqslant \sum_{i=1}^{n} 2 \cdot \frac{a_{i} b_{i}}{B_{n}}=2
\end{aligned}
$$

所以
$$
\frac{A_{n} C_{n}}{B_{n}^{2}}+1 \geqslant 2
$$

即
$$
B_{n}^{2} \leqslant A_{n} C_{n}
$$

等号成立当且仅当 $\frac{a_{i}}{b_{i}}(i=1,2, \cdots, n)$ 为一个常数.

注 (1) 这两个证明方法比较简单, 但是对于不等式的证明来讲, 怎样人手是十分重要的. 比值法是证明不等式的一种常用、基本的方法.

(2)上述两种方法也称为标准化方法, 这个方法可以简化许多不等式的证明. 在前面我们也使用过. 如为了证明 $G_{n} \leqslant A_{n}$, 令 $y_{i}=\frac{a_{i}}{G_{n}}$, 则问题化为在条件 $y_{1} y_{2} \cdots y_{n}=1\left(y_{i}>0\right)$ 下, 证明 $\sum_{i=1}^{n} y_{i} \geqslant n$.

\section*{证法四(归纳法)}
众所周知, 归纳法是证明不等式的一种强有力和常用的方法, 这里, 利用归纳法证明一个更强的结论, 即
$$
\sum_{i=1}^{n}\left|a_{i} b_{i}\right| \leqslant \sqrt{\sum_{i=1}^{n} a_{i}^{2}} \sqrt{\sum_{i=1}^{n} b_{i}^{2}}
$$

(1) 当 $n=2$ 时,
$$
\begin{aligned}
\left(a_{1} b_{1}+a_{2} b_{2}\right)^{2} & =a_{1}^{2} b_{1}^{2}+2 a_{1} b_{1} a_{2} b_{2}+a_{2}^{2} b_{2}^{2} \\
& \leqslant a_{1}^{2} b_{1}^{2}+a_{1}^{2} b_{2}^{2}+a_{2}^{2} b_{1}^{2}+a_{2}^{2} b_{2}^{2} \\
& =\left(a_{1}^{2}+a_{2}^{2}\right)\left(b_{1}^{2}+b_{2}^{2}\right)
\end{aligned}
$$

且等号成立当且仅当 $\frac{a_{1}}{b_{1}}=\frac{a_{2}}{b_{2}}$, 命题成立.

(2)假设当 $n=k$ 时命题成立, 那么对于 $n=k+1$, 由归纳假设,
$$
\begin{aligned}
& \sqrt{\sum_{i=1}^{k+1} a_{i}^{2}} \cdot \sqrt{\sum_{i=1}^{k+1} b_{i}^{2}} \\
= & \sqrt{\sum_{i=1}^{k} a_{i}^{2}+a_{k+1}^{2}} \cdot \sqrt{\sum_{i=1}^{k} b_{i}^{2}+b_{k+1}^{2}}
\end{aligned}
$$
$$
\begin{aligned}
& \geqslant \sqrt{\sum_{i=1}^{k} a_{i}^{2}} \cdot \sqrt{\sum_{i=1}^{k} b_{i}^{2}}+\left|a_{k+1} b_{k+1}\right| \\
& \geqslant \sum_{i=1}^{k}\left|a_{i} b_{i}\right|+\left|a_{k+1} b_{k+1}\right|=\sum_{i=1}^{k+1}\left|a_{i} b_{i}\right|
\end{aligned}
$$

所以对一切的 $n$ 命题成立.

不难得到等号成立的充分必要条件.

证法五(归纳与综合法)

(1) 当 $n=2$ 时,有
$$
\begin{aligned}
\left(a_{1} b_{1}+a_{2} b_{2}\right)^{2} & =a_{1}^{2} b_{1}^{2}+2 a_{1} b_{1} a_{2} b_{2}+a_{2}^{2} b_{2}^{2} \\
& \leqslant a_{1}^{2} b_{1}^{2}+a_{1}^{2} b_{2}^{2}+a_{2}^{2} b_{1}^{2}+a_{2}^{2} b_{2}^{2} \\
& =\left(a_{1}^{2}+a_{2}^{2}\right)\left(b_{1}^{2}+b_{2}^{2}\right)
\end{aligned}
$$

且等号成立当且仅当 $\frac{a_{1}}{b_{1}}=\frac{a_{2}}{b_{2}}$, 命题成立.

(2)假设当 $n=k$ 时命题成立. 对于 $n=k+1$, 令 $A_{k}=a_{1}^{2}+a_{2}^{2}+\cdots+a_{k}^{2}$, $B_{k}=a_{1} b_{1}+a_{2} b_{2}+\cdots+a_{k} b_{k}, C_{k}=b_{1}^{2}+b_{2}^{2}+\cdots+b_{k}^{2}$, 则由归纳假设
$$
B_{k}^{2} \leqslant A_{k} C_{k}
$$

由于我们要证明
$$
\begin{aligned}
& \left(a_{1} b_{1}+a_{2} b_{2}+\cdots+a_{k} b_{k}+a_{k+1} b_{k+1}\right)^{2} \\
\leqslant & \left(a_{1}^{2}+a_{2}^{2}+\cdots+a_{k}^{2}+a_{k+1}^{2}\right)\left(b_{1}^{2}+b_{2}^{2}+\cdots+b_{k}^{2}+b_{k+1}^{2}\right)
\end{aligned}
$$

等价于证明
$$
\begin{aligned}
& \left(B_{k}+a_{k+1} b_{k+1}\right)^{2} \leqslant\left(A_{k}+a_{k+1}^{2}\right)\left(C_{k}+b_{k+1}^{2}\right) \\
\Leftrightarrow & B_{k}^{2}+2 B_{k} a_{k+1} b_{k+1} \leqslant A_{k} C_{k}+A_{k} b_{k+1}^{2}+C_{k} a_{k+1}^{2} \\
\Leftrightarrow & A_{k} C_{k}-B_{k}^{2}+A_{k} b_{k+1}^{2}+C_{k} a_{k+1}^{2}-2 B_{k} a_{k+1} b_{k+1} \geqslant 0 \\
\Leftrightarrow & A_{k} C_{k}-B_{k}^{2}+\left(\sqrt{A_{k}} b_{k+1}-\sqrt{C_{k}} a_{k+1}\right)^{2}+2\left(\sqrt{A_{k}} \sqrt{C_{k}}-B_{k}\right) a_{k+1} b_{k+1} \geqslant 0
\end{aligned}
$$

由归纳假设, 上述不等式成立, 且等式成立当且仅当 $\frac{a_{1}}{b_{1}}=\frac{a_{2}}{b_{2}}=\cdots=$ $\frac{a_{k+1}}{b_{k+1}}$, 故对任意 $n \geqslant 1$, 命题成立.

证法六(归纳法和平均值不等式)

(1) 当 $n=2$ 时, 有
$$
\begin{aligned}
\left(a_{1} b_{1}+a_{2} b_{2}\right)^{2} & =a_{1}^{2} b_{1}^{2}+2 a_{1} b_{1} a_{2} b_{2}+a_{2}^{2} b_{2}^{2} \\
& \leqslant a_{1}^{2} b_{1}^{2}+a_{1}^{2} b_{2}^{2}+a_{2}^{2} b_{1}^{2}+a_{2}^{2} b_{2}^{2} \\
& =\left(a_{1}^{2}+a_{2}^{2}\right)\left(b_{1}^{2}+b_{2}^{2}\right)
\end{aligned}
$$

即命题成立.

(2)假设当 $n=k$ 时命题成立. 对于 $n=k+1$, 由于
$$
\begin{aligned}
& \left(a_{1} b_{1}+a_{2} b_{2}+\cdots+a_{k} b_{k}+a_{k+1} b_{k+1}\right)^{2} \\
= & \left(a_{1} b_{1}+a_{2} b_{2}+\cdots+a_{k} b_{k}\right)^{2} \\
& +2\left(a_{1} b_{1}+a_{2} b_{2}+\cdots+a_{k} b_{k}\right) a_{k+1} b_{k+1}+a_{k+1}^{2} b_{k+1}^{2}
\end{aligned}
$$

由平均值不等式, 得
$$
\begin{aligned}
& 2\left(a_{1} b_{1}+a_{2} b_{2}+\cdots+a_{k} b_{k}\right) a_{k+1} b_{k+1} \\
\leqslant & a_{k+1}^{2}\left(b_{1}^{2}+b_{2}^{2}+\cdots+b_{k}^{2}\right)+b_{k+1}^{2}\left(a_{1}^{2}+a_{2}^{2}+\cdots+a_{k}^{2}\right)
\end{aligned}
$$

由归纳假设, 得
$$
\begin{aligned}
& \left(a_{1} b_{1}+a_{2} b_{2}+\cdots+a_{k} b_{k}+a_{k+1} b_{k+1}\right)^{2} \\
= & \left(a_{1} b_{1}+a_{2} b_{2}+\cdots+a_{k} b_{k}\right)^{2}+2\left(a_{1} b_{1}+a_{2} b_{2}+\cdots+a_{k} b_{k}\right) a_{k+1} b_{k+1}+a_{k+1}^{2} b_{k+1}^{2} \\
\leqslant & \left(a_{1} b_{1}+a_{2} b_{2}+\cdots+a_{k} b_{k}\right)^{2}+a_{k+1}^{2}\left(b_{1}^{2}+b_{2}^{2}+\cdots+b_{k}^{2}\right) \\
& +b_{k+1}^{2}\left(a_{1}^{2}+a_{2}^{2}+\cdots+a_{k}^{2}\right)+a_{k+1}^{2} b_{k+1}^{2} \\
= & \left(a_{1}^{2}+a_{2}^{2}+\cdots+a_{k+1}^{2}\right)\left(b_{1}^{2}+b_{2}^{2}+\cdots+b_{k+1}^{2}\right)
\end{aligned}
$$

结合平均值不等式等号成立的条件, 不难得到柯西不等式等号成立的充要条件, 故命题成立.

注 (1)在上述的证明中, 我们反复利用了平均值不等式.

(2)上述几种证明均用归纳法, 由于证明过程中,对表达式的处理的不同, 所以难易程度也就不同.

证法七(利用排序不等式) 由于
$$
\sum_{i=1}^{n} a_{i}^{2} \sum_{i=1}^{n} b_{i}^{2}=a_{1}^{2} \sum_{i=1}^{n} b_{i}^{2}+a_{2}^{2} \sum_{i=1}^{n} b_{i}^{2}+\cdots+a_{n}^{2} \sum_{i=1}^{n} b_{i}^{2}
$$

则有
$$
\begin{gathered}
a_{1} b_{1}, \cdots, a_{1} b_{n}, a_{2} b_{1}, \cdots, a_{2} b_{n}, \cdots, a_{n} b_{1}, \cdots, a_{n} b_{n}, \\
a_{1} b_{1}, \cdots, a_{1} b_{n}, a_{2} b_{1}, \cdots, a_{2} b_{n}, \cdots, a_{n} b_{1}, \cdots, a_{n} b_{n}
\end{gathered}
$$

两行相同, 共 $n^{2}$ 列, 且是同序的.

另一方面, 有乱序
$$
\begin{aligned}
& a_{1} b_{1}, \cdots, a_{1} b_{n}, a_{2} b_{1}, \cdots, a_{2} b_{n}, \cdots, a_{n} b_{1}, \cdots, a_{n} b_{n} \\
& a_{1} b_{1}, \cdots, a_{n} b_{1}, a_{1} b_{2}, \cdots, a_{n} b_{2}, \cdots, a_{1} b_{n}, \cdots, a_{n} b_{n}
\end{aligned}
$$

两行, 共 $n^{2}$ 列, 且两行为乱序, 其乘积为
$$
\sum_{i=1}^{n} \sum_{j=1}^{n}\left(a_{i} b_{j}\right)\left(a_{j} b_{i}\right)=\left(\sum_{i=1}^{n} a_{i} b_{i}\right)^{2}
$$

由引理\ref{lem:证法七引理} , 得
$$
\left(\sum_{i=1}^{n} a_{i} b_{i}\right)^{2} \leqslant \sum_{i=1}^{n} a_{i}^{2} \sum_{i=1}^{n} b_{i}^{2}
$$

当且仅当 $\frac{a_{1}}{b_{1}}=\frac{a_{2}}{b_{2}}=\cdots=\frac{a_{n}}{b_{n}}$ 时等号成立.

\section*{证法八(利用参数平均值不等式)}
由于对 $m \in \mathbf{R}_{+}$, 得
$$
a_{i} b_{i} \leqslant \frac{1}{2}\left(m^{2} a_{i}^{2}+\frac{b_{i}^{2}}{m^{2}}\right)
$$

令 $m^{2}=\sqrt{\frac{\sum_{i=1}^{n} b_{i}^{2}}{\sum_{i=1}^{n} a_{i}^{2}}}$, 则

\begin{center}
此处有图片 % \includegraphics[max width=\textwidth]{2024_05_22_4ff05a14ba9ad07b725fg-096}
\end{center}

从而
$$
\sum_{i=1}^{n}\left|a_{i} b_{i}\right| \leqslant \frac{1}{2}\left(\sqrt{\frac{\sum_{i=1}^{n} b_{i}^{2}}{\sum_{i=1}^{n} a_{i}^{2}}} \sum_{i=1}^{n} a_{i}^{2}+\sqrt{\frac{\sum_{i=1}^{n} a_{i}^{2}}{\sum_{i=1}^{n} b_{i}^{2}}} \sum_{i=1}^{n} b_{i}^{2}\right)
$$

故
$$
\begin{aligned}
\sum_{i=1}^{n} a_{i} b_{i} & \leqslant \sum_{i=1}^{n}\left|a_{i} b_{i}\right| \leqslant \frac{1}{2}\left(\sqrt{\sum_{i=1}^{n} b_{i}^{2} \sum_{i=1}^{n} a_{i}^{2}}+\sqrt{\sum_{i=1}^{n} a_{i}^{2} \sum_{i=1}^{n} b_{i}^{2}}\right) \\
& =\left(\sum_{i=1}^{n} a_{i}^{2}\right)^{\frac{1}{2}}\left(\sum_{i=1}^{n} b_{i}^{2}\right)^{\frac{1}{2}}
\end{aligned}
$$

注 利用含参数的基本不等式来证明不等式, 具有较高的灵活性和技巧, 为了让大家熟悉这种证明方法, 后面, 我们将专门介绍.

\section*{证法九(利用行列式性质)}
$$
\begin{aligned}
S & =\sum_{i=1}^{n} a_{i}^{2} \cdot \sum_{i=1}^{n} b_{i}^{2}-\left(\sum_{i=1}^{n} a_{i} b_{i}\right)^{2} \\
& =\left|\begin{array}{cc}
a_{1}^{2}+a_{2}^{2}+\cdots+a_{n}^{2} & a_{1} b_{1}+a_{2} b_{2}+\cdots+a_{n} b_{n} \\
a_{1} b_{1}+a_{2} b_{2}+\cdots+a_{n} b_{n} & b_{1}^{2}+b_{2}^{2}+\cdots+b_{n}^{2}
\end{array}\right| \\
& =\sum_{i=1}^{n}\left|\begin{array}{cc}
a_{1}^{2}+a_{2}^{2}+\cdots+a_{n}^{2} & a_{i} b_{i} \\
a_{1} b_{1}+a_{2} b_{2}+\cdots+a_{n} b_{n} & b_{i}^{2}
\end{array}\right| \\
& =\sum_{i=1}^{n} \sum_{j=1}^{n}\left|\begin{array}{cc}
a_{j}^{2} & a_{i} b_{i} \\
a_{j} b_{j} & b_{i}^{2}
\end{array}\right| \\
& =\sum_{i=1}^{n} \sum_{j=1}^{n} a_{j} b_{i}\left|\begin{array}{cc}
a_{j} & a_{i} \\
b_{j} & b_{i}
\end{array}\right|
\end{aligned}
$$

又
$$
\begin{aligned}
S & =\sum_{j=1}^{n} \sum_{i=1}^{n} a_{i} b_{j}\left|\begin{array}{cc}
a_{i} & a_{j} \\
b_{i} & b_{j}
\end{array}\right| \\
& =\sum_{j=1}^{n} \sum_{i=1}^{n} a_{i} b_{j}(-1)\left|\begin{array}{cc}
a_{j} & a_{i} \\
b_{j} & b_{i}
\end{array}\right| \\
& =\sum_{i=1}^{n} \sum_{j=1}^{n} a_{i} b_{j}(-1)\left|\begin{array}{cc}
a_{j} & a_{i} \\
b_{j} & b_{i}
\end{array}\right|
\end{aligned}
$$

所以
$$
\begin{aligned}
2 S & =\sum_{i=1}^{n} \sum_{j=1}^{n}\left(a_{j} b_{i}-a_{i} b_{j}\right)\left|\begin{array}{ll}
a_{j} & a_{i} \\
b_{j} & b_{i}
\end{array}\right| \\
& =\sum_{i=1}^{n} \sum_{j=1}^{n}\left(a_{j} b_{i}-a_{i} b_{j}\right)^{2} \geqslant 0
\end{aligned}
$$

即 $S \geqslant 0$, 故不等式成立.

证法十(利用拉格朗日 (Lagrange)恒等式)

对 $a_{1}, a_{2}, \cdots, a_{n}$ 与 $b_{1}, b_{2}, \cdots, b_{n}$, 我们有如下的拉格朗日恒等式
$$
\sum_{i=1}^{n} a_{i}^{2} \cdot \sum_{i=1}^{n} b_{i}^{2}-\left(\sum_{i=1}^{n} a_{i} b_{i}\right)^{2}=\sum_{1 \leqslant i<j \leqslant n}\left(a_{i} b_{j}-a_{j} b_{i}\right)^{2} \geqslant 0
$$

不难看出命题成立.

注 实际上,证法十是证法九的一种特殊情况,但在证明不等式中,拉格朗日恒等式往往作为已知的结果使用, 此外, 拉格朗日恒等式也可以用其他方法来证明.

证法十一 (内积法)

令 $\boldsymbol{\alpha}=\left(a_{1}, a_{2}, \cdots, a_{n}\right), \boldsymbol{\beta}=\left(b_{1}, b_{2}, \cdots, b_{n}\right)$, 对任意实数 $t$, 我们有
$$
0 \leqslant(\boldsymbol{\alpha}+t \boldsymbol{\beta}, \boldsymbol{\alpha}+t \boldsymbol{\beta})=(\boldsymbol{\alpha}, \boldsymbol{\alpha})+2(\boldsymbol{\alpha}, \boldsymbol{\beta}) t+(\boldsymbol{\beta}, \boldsymbol{\beta}) t^{2}
$$

于是
$$
\sum_{i=1}^{n} a_{i}^{2}+2 t \sum_{i=1}^{n} a_{i} b_{i}+\left(\sum_{i=1}^{n} b_{i}^{2}\right) t^{2} \geqslant 0
$$

由 $t$ 的任意性, 得
$$
4\left[\left(\sum_{i=1}^{n} a_{i} b_{i}\right)^{2}-\sum_{i=1}^{n} a_{i}^{2} \sum_{i=1}^{n} b_{i}^{2}\right] \leqslant 0
$$

故命题成立.

证法十二 (向量法)

令 $\boldsymbol{\alpha}=\left(a_{1}, a_{2}, \cdots, a_{n}\right), \boldsymbol{\beta}=\left(b_{1}, b_{2}, \cdots, b_{n}\right)$, 则对向量 $\boldsymbol{\alpha} 、 \boldsymbol{\beta}$, 我们有
$$
\cos \langle\boldsymbol{\alpha}, \boldsymbol{\beta}\rangle=\frac{\boldsymbol{\alpha} \cdot \boldsymbol{\beta}}{|\boldsymbol{\alpha}| \cdot|\boldsymbol{\beta}|}
$$

从而
$$
\frac{\boldsymbol{\alpha} \cdot \boldsymbol{\beta}}{|\boldsymbol{\alpha}| \cdot|\boldsymbol{\beta}|}=\cos (\boldsymbol{\alpha}, \boldsymbol{\beta}) \leqslant 1
$$

由 $\boldsymbol{\alpha} \cdot \boldsymbol{\beta}=a_{1} b_{1}+a_{2} b_{2}+\cdots+a_{n} b_{n},|\boldsymbol{\alpha}|^{2}=\sum_{i=1}^{n} a_{i}^{2},|\boldsymbol{\beta}|^{2}=\sum_{i=1}^{n} b_{i}^{2}$, 且等号成立当且仅当 $\cos \langle\boldsymbol{\alpha}, \boldsymbol{\beta}\rangle=1$, 即 $\boldsymbol{\alpha}$ 与 $\boldsymbol{\beta}$ 平行. 故命题成立.

注 内积法和向量法有着密切的联系, 内积亦称为点积, 其定义为: 对任意两个向量 $\boldsymbol{\alpha} 、 \boldsymbol{\beta}$, 它们的内积为
$$
(\boldsymbol{\alpha}, \boldsymbol{\beta})=\boldsymbol{\alpha} \cdot \boldsymbol{\beta}=\sum_{i=1}^{n} a_{i} b_{i}
$$

容易验证,对任意向量 $\boldsymbol{\alpha} \neq \overrightarrow{0} ,$
$$
(\boldsymbol{\alpha}, \boldsymbol{\alpha})=\sum_{i=1}^{n} a_{i}^{2}>0
$$

在证法十一中, 就是利用了这个性质.

证法十三(构造单调数列)

构造数列 $\left\{S_{n}\right\}$, 其中
$$
S_{n}=\left(a_{1} b_{1}+a_{2} b_{2}+\cdots+a_{n} b_{n}\right)^{2}-\left(a_{1}^{2}+a_{2}^{2}+\cdots+a_{n}^{2}\right)\left(b_{1}^{2}+b_{2}^{2}+\cdots+b_{n}^{2}\right),
$$

则
$$
S_{1}=\left(a_{1} b_{1}\right)^{2}-a_{1}^{2} b_{1}^{2}=0
$$
$$
\begin{aligned}
S_{n+1}-S_{n}= & {\left[\left(a_{1} b_{1}+a_{2} b_{2}+\cdots+a_{n+1} b_{n+1}\right)^{2}\right.} \\
& \left.-\left(a_{1}^{2}+a_{2}^{2}+\cdots+a_{n+1}^{2}\right)\left(b_{1}^{2}+b_{2}^{2}+\cdots+b_{n+1}^{2}\right)\right] \\
& -\left[\left(a_{1} b_{1}+a_{2} b_{2}+\cdots+a_{n} b_{n}\right)^{2}-\left(a_{1}^{2}+a_{2}^{2}+\cdots+a_{n}^{2}\right) \cdot\right. \\
& \left.\left(b_{1}^{2}+b_{2}^{2}+\cdots+b_{n}^{2}\right)\right]
\end{aligned}
$$
$$
\begin{aligned}
= & 2\left(a_{1} b_{1}+a_{2} b_{2}+\cdots+a_{n} b_{n}\right) a_{n+1} b_{n+1}+a_{n+1}^{2} b_{n+1}^{2} \\
& -\left(a_{1}^{2}+a_{2}^{2}+\cdots+a_{n}^{2}\right) b_{n+1}^{2} \\
& -a_{n+1}^{2}\left(b_{1}^{2}+b_{2}^{2}+\cdots+b_{n}^{2}\right)-a_{n+1}^{2} b_{n+1}^{2} \\
= & -\left[\left(a_{1} b_{n+1}-b_{1} a_{n+1}\right)^{2}+\left(a_{2} b_{n+1}-b_{2} a_{n+1}\right)^{2}\right. \\
& \left.+\cdots+\left(a_{n} b_{n+1}-b_{n} a_{n+1}\right)^{2}\right] \leqslant 0
\end{aligned}
$$

即 $S_{n+1} \leqslant S_{n}$, 所以数列 $\left\{S_{n}\right\}$ 单调减少, 从而对一切 $n \geqslant 1$, 有 $S_{n} \leqslant S_{1}=0$, 故命题成立.

\section*{证法十四 (二次函数的判别式)}
令 $A_{n}=a_{1}^{2}+a_{2}^{2}+\cdots+a_{n}^{2}, B_{n}=a_{1} b_{1}+a_{2} b_{2}+\cdots+a_{n} b_{n}, C_{n}=b_{1}^{2}+$ $b_{2}^{2}+\cdots+b_{n}^{2}$, 作二次函数 $f(x)=A_{n} x^{2}+2 B_{n} x+C_{n}=\sum_{i=1}^{n}\left(a_{i} x+b_{i}\right)^{2} \geqslant 0$, 且 $f(x)=0$ 的充要条件是 $\frac{a_{i}}{b_{i}}=\lambda$ 为常数.

由于 $A_{n}>0, f(x) \geqslant 0$, 则它的判别式 $\Delta=4\left(B_{n}^{2}-A_{n} C_{n}\right) \leqslant 0$, 即
$$
B_{n}^{2} \leqslant A_{n} C_{n} .
$$

等号成立当且仅当 $\frac{a_{1}}{b_{1}}=\frac{a_{2}}{b_{2}}=\cdots=\frac{a_{n}}{b_{n}}$ 为常数.

用类似的方法,可以证明下列不等式:

Aczel 不等式 设 $a_{i}, b_{i} \in \mathbf{R}, 1 \leqslant i \leqslant n$, 满足 $a_{1}^{2}-a_{2}^{2}-\cdots-a_{n}^{2}>0$ 或 $b_{1}^{2}-b_{2}^{2}-\cdots-b_{n}^{2}>0$, 求证:
$$
\left(a_{1} b_{1}-a_{2} b_{2}-\cdots-a_{n} b_{n}\right)^{2} \geqslant\left(a_{1}^{2}-a_{2}^{2}-\cdots-a_{n}^{2}\right)\left(b_{1}^{2}-b_{2}^{2}-\cdots-b_{n}^{2}\right)
$$

证明 按上述记号, 不妨设 $A_{n}>0$, 考虑函数
$$
g(x)=A_{n} x^{2}+2 B_{n} x+C_{n}=\left(a_{1} x+b_{1}\right)^{2}-\sum_{i=2}^{n}\left(a_{i} x+b_{i}\right)^{2}
$$

则存在 $x_{0}=-\frac{b_{1}}{a_{1}}, a_{1} \neq 0$, 使得 $g\left(x_{0}\right) \leqslant 0$, 由于二次函数开口向上, 从而存在 $x_{1}$ 充分大, 使得 $g\left(x_{1}\right)>0$. 则它的判别式 $\Delta=4\left(B_{n}^{2}-A_{n} C_{n}\right) \geqslant 0$, 即
$$
B_{n}^{2} \geqslant A_{n} C_{n}
$$

等号成立当且仅当 $\frac{a_{1}}{b_{1}}=\frac{a_{2}}{b_{2}}=\cdots=\frac{a_{n}}{b_{n}}$ 为常数.

\section*{证法十五(凹函数方法)}
令 $A_{n}=a_{1}^{2}+a_{2}^{2}+\cdots+a_{n}^{2}, B_{n}=a_{1} b_{1}+a_{2} b_{2}+\cdots+a_{n} b_{n}, C_{n}=b_{1}^{2}+$\\
$b_{2}^{2}+\cdots+b_{n}^{2}$, 且不妨假设 $a_{i}>0, b_{i}>0$, 由前面的引理 4, 对凹函数 $f(x)=$ $\ln x$, 有
$$
\begin{aligned}
& \frac{1}{2} \ln \frac{a_{i}^{2}}{A_{n}}+\frac{1}{2} \ln \frac{b_{i}^{2}}{C_{n}} \leqslant \ln \frac{\frac{a_{i}^{2}}{A_{n}}+\frac{b_{i}^{2}}{C_{n}}}{2} \\
\Leftrightarrow & \ln \left(\frac{a_{i}^{2}}{A_{n}} \frac{b_{i}^{2}}{C_{n}}\right)^{\frac{1}{2}} \leqslant \ln \frac{\frac{a_{i}^{2}}{A_{n}}+\frac{b_{i}^{2}}{C_{n}}}{2} \\
\Leftrightarrow & \left(\frac{a_{i}^{2}}{A_{n}} \frac{b_{i}^{2}}{C_{n}}\right)^{\frac{1}{2}} \leqslant \frac{\frac{a_{i}^{2}}{A_{n}}+\frac{b_{i}^{2}}{C_{n}}}{2}
\end{aligned}
$$

于是
$$
\begin{aligned}
& \sum_{i=1}^{n} \frac{a_{i}}{A_{n}^{\frac{1}{2}}} \frac{b_{i}}{C_{n}^{\frac{1}{2}}} \leqslant \frac{1}{2}\left(\frac{1}{A_{n}} \sum_{i=1}^{n} a_{i}^{2}+\frac{1}{C_{n}} \sum_{i=1}^{n} b_{i}^{2}\right)=1 \\
\Leftrightarrow & \sum_{i=1}^{n} a_{i} b_{i} \leqslant A_{n}^{\frac{1}{2}} C_{n}^{\frac{1}{2}}
\end{aligned}
$$

不难得到, 等式成立的充要条件是 $\frac{a_{1}}{b_{1}}=\frac{a_{2}}{b_{2}}=\cdots=\frac{a_{n}}{b_{n}}$.

另外, 如果令 $x=\frac{a_{i}^{2}}{A_{n}}, y=\frac{b_{i}^{2}}{C_{n}}, p=q=2$, 则由 Young 不等式, 容易得到柯西不等式.

\section*{3.2 柯西不等式的变形和推广}
变形 $\mathbf{1}$ 设 $a_{i} \in \mathbf{R}, b_{i}>0(i=1,2, \cdots, n)$, 则
$$
\sum_{i=1}^{n} \frac{a_{i}^{2}}{b_{i}} \geqslant \frac{\left(\sum_{i=1}^{n} a_{i}\right)^{2}}{\sum_{i=1}^{n} b_{i}}
$$

等号成立的充分必要条件是 $a_{i}=\lambda b_{i}(i=1,2, \cdots, n)$.

变形 2 设 $a_{i}, b_{i}(i=1,2, \cdots, n)$ 同号且不为零, 则
$$
\sum_{i=1}^{n} \frac{a_{i}}{b_{i}} \geqslant \frac{\left(\sum_{i=1}^{n} a_{i}\right)^{2}}{\sum_{i=1}^{n} a_{i} b_{i}}
$$

等号成立的充分必要条件是 $b_{1}=b_{2}=\cdots=b_{n}$.\\
柯西不等式的推广为赫尔德 (Hölder)不等式,即

赫尔德不等式 设 $a_{i}>0, b_{i}>0(i=1,2, \cdots, n), p>0, q>0$, 满足 $\frac{1}{p}+\frac{1}{q}=1$, 则
$$
\sum_{i=1}^{n} a_{i} b_{i} \leqslant\left(\sum_{i=1}^{n} a_{i}^{p}\right)^{\frac{1}{p}}\left(\sum_{i=1}^{n} b_{i}^{q}\right)^{\frac{1}{q}}
$$

等号成立的充分必要条件是 $a_{i}^{p}=\lambda b_{i}^{q}(i=1,2, \cdots, n, \lambda>0)$.

证明 由 Young 不等式, 得
$$
\begin{aligned}
& \sum_{i=1}^{n}\left[\frac{a_{i}^{p}}{\sum_{i=1}^{n} a_{i}^{p}}\right]^{\frac{1}{p}} \cdot\left[\frac{b_{i}^{q}}{\sum_{i=1}^{n} b_{i}^{q}}\right]^{\frac{1}{q}} \\
\leqslant & \sum_{i=1}^{n}\left[\frac{1}{p} \frac{a_{i}^{p}}{\sum_{i=1}^{n} a_{i}^{p}}\right]+\sum_{i=1}^{n}\left[\frac{1}{q} \frac{b_{i}^{q}}{\sum_{i=1}^{n} b_{i}^{q}}\right] \\
= & \frac{1}{p}+\frac{1}{q}=1
\end{aligned}
$$

等号成立的充分必要条件是
$$
\frac{a_{i}^{p}}{\sum_{i=1}^{n} a_{i}^{p}}=\frac{b_{i}^{q}}{\sum_{i=1}^{n} b_{i}^{q}}
$$

即 $a_{i}^{p}=\lambda b_{i}^{q}(i=1,2, \cdots, n, \lambda>0)$.

赫尔德不等式也可以变形为
$$
\sum_{i=1}^{n} \frac{a_{i}^{m+1}}{b_{i}^{m}} \geqslant \frac{\left(\sum_{i=1}^{n} a_{i}\right)^{m+1}}{\left(\sum_{i=1}^{n} b_{i}\right)^{m}}
$$

等号成立的充分必要条件是 $a_{i}=\lambda b_{i}(i=1,2, \cdots, n)$. 其中 $a_{i}>0, b_{i}>0$ $(i=1,2, \cdots, n), m>0$ 或 $m<-1$.

证明 当 $m>0$ 时, 由赫尔德不等式, 得
$$
\begin{aligned}
\sum_{i=1}^{n} a_{i} & =\sum_{i=1}^{n}\left(\frac{a_{i}}{b_{i}^{\frac{m}{m+1}}}\right) \cdot b_{i}^{\frac{m}{m+1}} \\
& \leqslant\left[\sum_{i=1}^{n}\left(\frac{a_{i}}{b_{i}^{\frac{m}{m+1}}}\right)^{m+1}\right]^{\frac{1}{m+1}} \cdot\left[\sum_{i=1}^{n}\left(b_{i}^{\frac{m}{m+1}}\right)^{\frac{m+1}{m}}\right]^{\frac{m}{m+1}}
\end{aligned}
$$
$$
=\left(\sum_{i=1}^{n} \frac{a_{i}^{m+1}}{b_{i}^{m}}\right)^{\frac{1}{m+1}} \cdot\left(\sum_{i=1}^{n} b_{i}\right)^{\frac{m}{m+1}}
$$

故
$$
\sum_{i=1}^{n} \frac{a_{i}^{m+1}}{b_{i}^{m}} \geqslant \frac{\left(\sum_{i=1}^{n} a_{i}\right)^{m+1}}{\left(\sum_{i=1}^{n} b_{i}\right)^{m}}
$$

当 $m<-1$ 时, $-(m+1)>0$, 对于数组 $\left(b_{1}, b_{2}, \cdots, b_{n}\right)$ 和 $\left(a_{1}, a_{2}, \cdots\right.$, $\left.a_{n}\right)$ 有

即
$$
\begin{gathered}
\sum_{i=1}^{n} \frac{b_{i}^{-(m+1)+1}}{a_{i}^{-(m+1)}} \geqslant \frac{\left(\sum_{i=1}^{n} b_{i}\right)^{-(m+1)+1}}{\left(\sum_{i=1}^{n} a_{i}\right)^{-(m+1)}} \\
\sum_{i=1}^{n} \frac{a_{i}^{m+1}}{b_{i}^{m}} \geqslant \frac{\left(\sum_{i=1}^{n} a_{i}\right)^{m+1}}{\left(\sum_{i=1}^{n} b_{i}\right)^{m}}
\end{gathered}
$$

等号成立当且仅当 $\left(\frac{a_{i}}{b_{i}^{m+1}}\right)^{m+1}=\mu\left(b_{i}^{\frac{m}{m+1}}\right)^{\frac{m+1}{m}}$, 即 $a_{i}=\lambda b_{i}(i=1$, $2, \cdots, n)$.

由赫尔德不等式可以推出另一个重要的不等式, 即

闵可夫斯基(Minkowski)不等式 对 $a_{i}, b_{i} \in \mathbf{R}_{+}, 1 \leqslant i \leqslant n, k>1$, 则
$$
\left[\sum_{i=1}^{n}\left(a_{i}+b_{i}\right)^{k}\right]^{\frac{1}{k}} \leqslant\left(\sum_{i=1}^{n} a_{i}^{k}\right)^{\frac{1}{k}}+\left(\sum_{i=1}^{n} b_{i}^{k}\right)^{\frac{1}{k}}
$$

当且仅当 $\frac{a_{1}}{b_{1}}=\frac{a_{2}}{b_{2}}=\cdots=\frac{a_{n}}{b_{n}}$ 时, 等号成立.

证明 $\quad$ 由赫尔德不等式, 得
$$
\begin{aligned}
\sum_{i=1}^{n}\left(a_{i}+b_{i}\right)^{k} & =\sum_{i=1}^{n} a_{i}\left(a_{i}+b_{i}\right)^{k-1}+\sum_{i=1}^{n} b_{i}\left(a_{i}+b_{i}\right)^{k-1} \\
& \leqslant\left(\sum_{i=1}^{n} a_{i}^{k}\right)^{\frac{1}{k}}\left[\sum_{i=1}^{n}\left(a_{i}+b_{i}\right)^{k}\right]^{\frac{k-1}{k}}+\left(\sum_{i=1}^{n} b_{i}^{k}\right)^{\frac{1}{k}}\left[\sum_{i=1}^{n}\left(a_{i}+b_{i}\right)^{k}\right]^{\frac{k-1}{k}}
\end{aligned}
$$

所以
$$
\left[\sum_{i=1}^{n}\left(a_{i}+b_{i}\right)^{)^{1}}\right]^{\frac{1}{k}} \leqslant\left(\sum_{i=1}^{n} a_{i}^{k}\right)^{\frac{1}{k}}+\left(\sum_{i=1}^{n} b_{i}^{k}\right)^{\frac{1}{k}}
$$

不难知, 当且仅当 $\frac{a_{1}}{b_{1}}=\frac{a_{2}}{b_{2}}=\cdots=\frac{a_{n}}{b_{n}}$ 时, 等号成立.\\
例 1 对一切正数 $x_{1}, x_{2}, \cdots, x_{n}$, 求证:
$$
\begin{aligned}
& x_{1}^{3}+\left(\frac{x_{1}+x_{2}}{2}\right)^{3}+\left(\frac{x_{1}+x_{2}+x_{3}}{3}\right)^{3}+\cdots+\left(\frac{x_{1}+x_{2}+\cdots+x_{n}}{n}\right)^{3} \\
< & \frac{27}{8}\left(x_{1}^{3}+x_{2}^{3}+\cdots+x_{n}^{3}\right) .
\end{aligned}
$$

证明 记 $A_{k}=\frac{x_{1}+x_{2}+\cdots+x_{k}}{k}$, 并约定 $A_{0}=0$, 则 $x_{k}=k A_{k}-(k-1) A_{k-1}$ $(k=1,2, \cdots, n)$

先证: $\quad \sum_{k=1}^{n} A_{k}^{3} \leqslant \frac{3}{2} \sum_{k=1}^{n} x_{k} A_{k}^{2}$
$$
\Leftrightarrow \sum_{k=1}^{n} A_{k}^{3} \leqslant \frac{3}{2} \sum_{k=1}^{n}\left[k A_{k}-(k-1) A_{k-1}\right] A_{k}^{2}
$$

而右式 $=\frac{3}{2} \sum_{k=1}^{n} k A_{k}^{3}-\frac{3}{2} \sum_{k=1}^{n} A_{k-1} A_{k}^{2} \cdot(k-1)$
$$
\begin{aligned}
& \geqslant \frac{3}{2} \sum_{k=1}^{n} k A_{k}^{3}-\frac{3}{2} \sum_{k=1}^{n} \frac{A_{k-1}^{3}+2 A_{k}^{3}}{3} \cdot(k-1) \\
& =\sum_{k=1}^{n} A_{k}^{3}+\frac{n}{2} A_{n}^{3} \geqslant \sum_{k=1}^{n} A_{k}^{3}
\end{aligned}
$$

$\circledast$ 得证.

原不等式 $\Leftrightarrow \sum_{k=1}^{n} A_{k}^{3} \leqslant \frac{27}{8} \sum_{k=1}^{n} x_{k}^{3}$.

由赫尔德不等式:


\begin{gather*}
\left(\sum_{k=1}^{n} x_{k}^{3}\right) \cdot\left(\sum_{k=1}^{n} A_{k}^{3}\right)^{2} \geqslant\left(\sum_{k=1}^{n} x_{k} A_{k}^{2}\right)^{3} \geqslant \frac{8}{27} \cdot\left(\sum_{k=1}^{n} A_{k}^{3}\right)^{3} \\
\Rightarrow \sum_{k=1}^{n} A_{k}^{3} \leqslant \frac{27}{8} \sum_{k=1}^{n} x_{k}^{3} .
\end{gather*}


综上, 命题得证.

关于赫尔德不等式的推广, 有如下定理.

定理 对于 $n \times m$ 矩阵
$$
\left(\begin{array}{cccc}
a_{11} & a_{12} & \cdots & a_{1 m} \\
a_{21} & a_{22} & \cdots & a_{2 m} \\
\cdots & \cdots & \cdots & \cdots \\
a_{n 1} & a_{n 2} & \cdots & a_{n m}
\end{array}\right)
$$

其中, $a_{i j} \geqslant 0(i=1,2, \cdots, n, j=1,2, \cdots, m)$, 则
$$
\left[\prod_{j=1}^{m}\left(\sum_{i=1}^{n} a_{i j}\right)\right]^{\frac{1}{m}} \geqslant \sum_{i=1}^{n}\left(\prod_{j=1}^{m} a_{i j}\right)^{\frac{1}{m}}
$$

其中, 等号成立的充要条件是至少有一列数都是 0 或所有行中的数对应成比例.

这个不等式称为卡尔松不等式.

证明 记
$$
\begin{aligned}
A_{j} & =\sum_{i=1}^{n} a_{i j}(j=1,2, \cdots, m) \\
G_{i} & =\prod_{j=1}^{m} a_{i j}(i=1,2, \cdots, n)
\end{aligned}
$$

若某个 $A_{j}=0$, 则由 $a_{i j} \geqslant 0(i=1,2, \cdots, n)$, 得 $a_{1 j}=a_{2 j}=\cdots=$ $a_{n j}=0$.

此时, $G_{1}=G_{2}=\cdots=G_{n}=0$,
$$
\left[\prod_{j=1}^{m}\left(\sum_{i=1}^{n} a_{i j}\right)\right]^{\frac{1}{m}}=\sum_{i=1}^{n}\left(\prod_{j=1}^{m} a_{i j}\right)^{\frac{1}{m}}=0
$$

从而, 不等式成立.

若所有的 $A_{j}>0$, 由均值不等式得
$$
\frac{a_{i 1}}{A_{1}}+\frac{a_{i 2}}{A_{2}}+\cdots+\frac{a_{i n}}{A_{m}} \geqslant m\left(\frac{\prod_{j=1}^{m} a_{i j}}{\prod_{j=1}^{m} A_{j}}\right)^{\frac{1}{m}}(i=1,2, \cdots, n)
$$

将以上 $n$ 个不等式相加得
$$
m \geqslant m \sum_{i=1}^{n}\left(\frac{\prod_{j=1}^{m} a_{i j}}{\prod_{j=1}^{m} A_{j}}\right)^{\frac{1}{m}}=m \frac{\sum_{i=1}^{n} G_{i}^{\frac{1}{m}}}{\left(\prod_{j=1}^{m} A_{j}\right)^{\frac{1}{m}}}
$$

故 $\left[\prod_{j=1}^{m}\left(\sum_{i=1}^{n} a_{i j}\right)\right]^{\frac{1}{m}} \geqslant \sum_{i=1}^{n}\left(\prod_{j=1}^{m} a_{i j}\right)^{\frac{1}{m}}$.

等号成立的充要条件是至少有一列数都是 0 或 $\frac{a_{i 1}}{A_{1}}=\frac{a_{i 2}}{A_{2}}=\cdots=\frac{a_{\text {in }}}{A_{m}}$, 即所有行中的数对应成比例.

利用卡尔松不等式可以推证柯西不等式、均值不等式及幂平均不等式.

(1)构造 $n \times 2$ 矩阵 $\left(\begin{array}{cc}a_{1}^{2} & b_{1}^{2} \\ a_{2}^{2} & b_{2}^{2} \\ \cdots & \cdots \\ a_{n}^{2} & b_{n}^{2}\end{array}\right)$.

利用卡尔松不等式得柯西不等式
$$
\left[\left(\sum_{i=1}^{n} a_{i}^{2}\right) \sum_{i=1}^{n} b_{i}^{2}\right]^{\frac{1}{2}} \geqslant \sum_{i=1}^{n} a_{i} b_{i}
$$

(2) 构造 $n \times n$ 矩阵
$$
\left(\begin{array}{cccc}
x_{1} & x_{2} & \cdots & x_{n} \\
x_{2} & x_{3} & \cdots & x_{1} \\
\cdots & \cdots & \cdots & \cdots \\
x_{n} & x_{1} & \cdots & x_{n-1}
\end{array}\right)
$$

利用卡尔松不等式得
$$
\left[\left(\sum_{i=1}^{n} x_{i}\right)^{n}\right]^{\frac{1}{n}} \geqslant n\left(\prod_{i=1}^{n} x_{i}\right)^{\frac{1}{n}}
$$

即均值不等式
$$
\frac{\sum_{i=1}^{n} x_{i}}{n} \geqslant\left(\prod_{i=1}^{n} x_{i}\right)^{\frac{1}{n}}
$$

(3)构造 $n \times \alpha$ 矩阵
$$
\left(\begin{array}{ccccccc}
x_{1}^{\alpha} & x_{1}^{\alpha} & \cdots & x_{1}^{\alpha} & 1 & \cdots & 1 \\
x_{2}^{\alpha} & x_{2}^{\alpha} & \cdots & x_{2}^{\alpha} & 1 & \cdots & 1 \\
\cdots & \cdots & \cdots & \cdots & \cdots & \cdots & \cdots \\
x_{n}^{\alpha} & x_{n}^{\alpha} & \cdots & x_{n}^{\alpha} & 1 & \cdots & 1
\end{array}\right),
$$

其中, $x_{i}^{x}$ 共有 $\beta$ 列, 1 共有 $\alpha-\beta$ 列.

利用卡尔松不等式得
$$
\begin{aligned}
& {\left[\left(x_{1}^{\alpha}+x_{2}^{\alpha}+\cdots+x_{n}^{\alpha}\right)^{\beta} n^{\alpha-\beta}\right]^{\frac{1}{\alpha}} } \\
\geqslant & {\left[\left(x_{1}^{\alpha}\right)^{\beta} \cdot 1^{\alpha-\beta}\right]^{\frac{1}{\alpha}}+\left[\left(x_{2}^{\alpha}\right)^{\beta} \cdot 1^{\alpha-\beta}\right]^{\frac{1}{\alpha}}+\cdots+\left[\left(x_{n}^{\alpha}\right)^{\beta} \cdot 1^{\alpha \beta}\right]^{\frac{1}{\alpha}} } \\
= & x_{1}^{\beta}+x_{2}^{\beta}+\cdots+x_{n}^{\beta},
\end{aligned}
$$

即幂平均不等式
$$
\left(\frac{x_{1}^{\alpha}+x_{2}^{\alpha}+\cdots+x_{n}^{\alpha}}{n}\right)^{\frac{1}{\alpha}} \geqslant\left(\frac{x_{1}^{\beta}+x_{2}^{\beta}+\cdots+x_{n}^{\beta}}{n}\right)^{\frac{1}{\beta}}\left(\alpha, \beta \in \mathbf{N}_{+}, \alpha>\beta\right)
$$

【说明】(1) 卡尔松不等式和均值不等式是等价的, 柯西不等式是卡尔松不等式的一种特殊形式, 即 $n \times m$ 矩阵中 $m=2$ 的情形.

(2)利用卡尔松不等式证明不等式的关键是构造矩阵, 充分利用条件和结论提供的信息, 注意取等号的条件是构造矩阵的关键.

\section*{习题 3}
$\square$ 设 $a, b, c \in \mathbf{R}_{+}$, 求证: $a^{2 a} b^{2 b} c^{2 c} \geqslant a^{b+c} b^{c+a} c^{a+b}$.

\begin{enumerate}
  \setcounter{enumi}{1}
  \item 设 $a, b, c, d>0$ 且 $a+b+c+d=1$, 求证:
\end{enumerate}
$$
\frac{1}{4 a+3 b+c}+\frac{1}{3 a+b+4 d}+\frac{1}{a+4 c+3 d}+\frac{1}{4 b+3 c+d} \geqslant 2
$$

3 已知 $a, b \in \mathbf{R}_{+}, n \geqslant 2, n \in \mathbf{N}_{+}$. 求证:
$$
\sum_{i=1}^{n} \frac{1}{a+i b}<\frac{n}{\sqrt{a(a+n b)}}
$$

4 已知 $a, b, c>0$, 且 $a b+b c+c d=3$, 求证:
$$
\frac{a+b^{2} c^{2}}{b+c}+\frac{b+c^{2} a^{2}}{c+a}+\frac{c+a^{2} b^{2}}{a+b} \geqslant 3
$$

5 设正实数 $a 、 b 、 c$ 满足 $a b+b c+c a=\frac{1}{3}$. 证明:
$$
\frac{a}{a^{2}-b c+1}+\frac{b}{b^{2}-c a+1}+\frac{c}{c^{2}-a b+1} \geqslant \frac{1}{a+b+c}
$$

6 已知 $x 、 y 、 z$ 为正实数. 证明:
$$
\frac{1+x y+x z}{(1+y+z)^{2}}+\frac{1+y z+y x}{(1+z+x)^{2}}+\frac{1+z x+z y}{(1+x+y)^{2}} \geqslant 1
$$

$7 x, y, z \in \mathbf{R}_{+}$, 且 $x y z \geqslant 1$. 求证:
$$
\frac{x^{5}-x^{2}}{x^{5}+y^{2}+z^{2}}+\frac{y^{5}-y^{2}}{y^{5}+z^{2}+x^{2}}+\frac{z^{5}-z^{2}}{z^{5}+x^{2}+y^{2}} \geqslant 0
$$

8 设 $a, b, c \in \mathbf{R}_{+}$, 且 $a+b+c=3$. 证明:
$$
\sum_{\text {cyc }} \frac{a^{4}}{b^{2}+c} \geqslant \frac{3}{2}
$$

其中, “ $\sum_{\text {cyc }} ”$ 表示轮换对称和.

9 设实数 $a, b, c>0$, 且满足 $a+b+c=3$. 证明:
$$
\frac{a^{2}+3 b^{2}}{a b^{2}(4-a b)}+\frac{b^{2}+3 c^{2}}{b c^{2}(4-b c)}+\frac{c^{2}+3 a^{2}}{c a^{2}(4-c a)} \geqslant 4
$$

10 设 $a, b, c \in \mathbf{R}_{+}$, 且 $a b c=1$, 求证:
$$
\frac{1}{1+2 a}+\frac{1}{1+2 b}+\frac{1}{1+2 c} \geqslant 1
$$

11 设 $a_{1}, a_{2}, \cdots, a_{n}$ 为实数,证明:
$$
\sqrt[3]{a_{1}^{3}+a_{2}^{3}+\cdots+a_{n}^{3}} \leqslant \sqrt{a_{1}^{2}+a_{2}^{2}+\cdots+a_{n}^{2}}
$$

12 已知 $a 、 b 、 c$ 为正实数, 证明:
$$
\frac{9}{a+b+c} \leqslant 2\left(\frac{1}{a+b}+\frac{1}{b+c}+\frac{1}{c+a}\right)
$$

13 设 $a_{i} \in \mathbf{R}_{+}(i=1,2, \cdots, n)$, 求证:
$$
\frac{1}{a_{1}}+\frac{2}{a_{1}+a_{2}}+\cdots+\frac{n}{a_{1}+\cdots+a_{n}}<2 \sum_{i=1}^{n} \frac{1}{a_{i}}
$$

14 设 $a_{i} 、 b_{i} 、 c_{i} 、 d_{i}$ 为正实数 $(i=1,2, \cdots, n)$, 求证:
$$
\left(\sum_{i=1}^{n} a_{i} b_{i} c_{i} d_{i}\right)^{4} \leqslant \sum_{i=1}^{n} a_{i}^{4} \sum_{i=1}^{n} b_{i}^{4} \sum_{i=1}^{n} c_{i}^{4} \sum_{i=1}^{n} d_{i}^{4}
$$

15 设 $n(n \geqslant 2)$ 为正整数, 求证:
$$
\sum_{i=1}^{n} \frac{x_{i}}{x_{i+1}-x_{i+1}^{2}} \geqslant \frac{n^{3}}{n^{2}-1}, \text { 其中 } x_{n+1}=x_{1} \text {. }
$$

16 设 $a_{1}, a_{2}, \cdots, a_{n}$ 为正实数,证明:
$$
\frac{\left(\sum_{i=1}^{n} a_{i}\right)^{2}}{2 \sum_{i=1}^{n} a_{i}^{2}} \leqslant \frac{a_{1}}{a_{2}+a_{3}}+\frac{a_{2}}{a_{3}+a_{4}}+\cdots+\frac{a_{n}}{a_{1}+a_{2}}
$$

17 设 $a 、 b 、 c 、 d$ 为正数,证明:
$$
\sqrt{\frac{a^{2}+b^{2}+c^{2}+d^{2}}{4}} \geqslant \sqrt[3]{\frac{a b c+b c d+c d a+d a b}{4}}
$$

18 设 $n$ 是大于 1 的自然数, 求证:
$$
\sqrt{\mathrm{C}_{n}^{1}}+2 \cdot \sqrt{\mathrm{C}_{n}^{2}}+\cdots+n \cdot \sqrt{\mathrm{C}_{n}^{n}}<\sqrt{2^{n-1} \cdot n^{3}}
$$

19 给定 $a_{i} \in \mathbf{R}_{+}, 1 \leqslant i \leqslant n$. 证明: 存在 $x_{i} \in \mathbf{R}_{+}, 1 \leqslant i \leqslant n$, 满足 $\sum_{i=1}^{n} x_{i}=$ 1 , 且对任何满足 $\sum_{i=1}^{n} y_{i}=1$ 的正实数 $y_{1}, \cdots, y_{n}$, 均有
$$
\sum_{i=1}^{n} \frac{a_{i} x_{i}}{x_{i}+y_{i}} \geqslant \frac{1}{2} \sum_{i=1}^{n} a_{i}
$$

20 设 $0<x_{i}<1,1 \leqslant i \leqslant n, n \geqslant 2$. 求证:
$$
\frac{\sqrt{1-x_{1}}}{x_{1}}+\frac{\sqrt{1-x_{2}}}{x_{2}}+\cdots+\frac{\sqrt{1-x_{n}}}{x_{n}}<\frac{\sqrt{n-1}}{x_{1} x_{2} \cdots x_{n}}
$$

21 设实数 $a 、 b 、 c$ 满足 $|a|,|b|,|c| \leqslant 1,1+2 a b c \geqslant a^{2}+b^{2}+c^{2}$.求证: $1+2(a b c)^{n} \geqslant a^{2 n}+b^{2 n}+c^{2 n}, n \in \mathbf{Z}_{+}$.

22 设 $a_{i} \in \mathbf{R}_{+}, 1 \leqslant i \leqslant n$, 求证:
$$
\sum_{i=1}^{n} \frac{1}{a_{i}^{2}}+\frac{1}{\left(\sum_{i=1}^{n} a_{i}\right)^{2}} \geqslant \frac{n^{3}+1}{\left(n^{2}+1\right)^{2}}\left(\sum_{i=1}^{n} \frac{1}{a_{i}}+\frac{1}{\sum_{i=1}^{n} a_{i}}\right)^{2}
$$

23 设 $a_{i} \in \mathbf{R}, 1 \leqslant i \leqslant n$. 求证:
$$
\sum_{i=1}^{n} a_{i}^{2} \geqslant \sum_{i=1}^{n-1} a_{i} a_{i+1}+\frac{3}{n(n+1)(2 n+1)}\left(\sum_{i=1}^{n} a_{i}\right)^{2}
$$

24 设 $x_{i}, y_{i} \in \mathbf{R}_{+}, 1 \leqslant i \leqslant n$. 求证:
$$
\sum_{i=1}^{n} \frac{x_{i}^{\frac{3}{2}}}{y_{i}^{\frac{1}{2}}} \geqslant \frac{\left(\sum_{i=1}^{n} x_{i}\right)^{\frac{3}{2}}}{\left(\sum_{i=1}^{n} y_{i}\right)^{\frac{1}{2}}}
$$

\section*{柯西不等式的应用}
\section*{4. 1 柯西不等式在证明不等式中的应用}
运用柯西不等式, 证明其他不等式的关键是构造两组数, 并按照柯西不等式形式进行探索, 巧妙选取两组数.

例 1 已知 $a, b, c \in \mathbf{R}_{+}$, 且 $a+b+c=1$, 求证:
$$
36 \leqslant \frac{1}{a}+\frac{4}{b}+\frac{9}{c}
$$

证明 由柯西不等式,得
$$
\begin{aligned}
\frac{1}{a}+\frac{4}{b}+\frac{9}{c}= & \left(\frac{1}{a}+\frac{4}{b}+\frac{9}{c}\right) \cdot(a+b+c) \\
\geqslant & \left(\sqrt{a} \cdot \frac{1}{\sqrt{a}}+\sqrt{b} \cdot \frac{2}{\sqrt{b}}+\sqrt{c} \cdot \frac{3}{\sqrt{c}}\right)^{2}=36 \\
& \frac{1}{a}+\frac{4}{b}+\frac{9}{c} \geqslant 36
\end{aligned}
$$

所以

例 2 设 $a, b, c \in \mathbf{R}_{+}$, 满足 $a \cos ^{2} \alpha+b \sin ^{2} \alpha<c$, 求证:
$$
\sqrt{a} \cos ^{2} \alpha+\sqrt{b} \sin ^{2} \alpha<\sqrt{c}
$$

证明 由柯西不等式, 得
$$
\begin{aligned}
\sqrt{a} \cos ^{2} \alpha+\sqrt{b} \sin ^{2} \alpha & =\sqrt{a} \cos \alpha \cdot \cos \alpha+\sqrt{b} \sin \alpha \cdot \sin \alpha \\
& \leqslant\left[(\sqrt{a} \cos \alpha)^{2}+(\sqrt{b} \sin \alpha)^{2}\right]^{\frac{1}{2}} \cdot\left(\cos ^{2} \alpha+\sin ^{2} \alpha\right)^{\frac{1}{2}} \\
& =\left(a \cos ^{2} \alpha+b \sin ^{2} \alpha\right)^{\frac{1}{2}}<\sqrt{c},
\end{aligned}
$$

故命题成立.\\
例 3 设 $a_{i}>0(i=1,2, \cdots, n)$ 满足 $\sum_{i=1}^{n} a_{i}=1$, 求证:
$$
\frac{a_{1}^{2}}{a_{1}+a_{2}}+\frac{a_{2}^{2}}{a_{2}+a_{3}}+\cdots+\frac{a_{n}^{2}}{a_{n}+a_{1}} \geqslant \frac{1}{2}
$$

证明 令 $a_{n+1}=a_{1}$, 由柯西不等式, 得
$$
\begin{aligned}
\left(\sum_{i=1}^{n} a_{i}\right)^{2} & =\left(\sum_{i=1}^{n} \frac{a_{i}}{\sqrt{a_{i}+a_{i+1}}} \cdot \sqrt{a_{i}+a_{i+1}}\right)^{2} \\
& \leqslant \sum_{i=1}^{n} \frac{a_{i}^{2}}{a_{i}+a_{i+1}} \cdot \sum_{i=1}^{n}\left(a_{i}+a_{i+1}\right) \\
& =2 \sum_{i=1}^{n} \frac{a_{i}^{2}}{a_{i}+a_{i+1}} \cdot \sum_{i=1}^{n} a_{i}
\end{aligned}
$$

于是
$$
\sum_{i=1}^{n} \frac{a_{i}^{2}}{a_{i}+a_{i+1}} \geqslant \frac{1}{2} \sum_{i=1}^{n} a_{i}=\frac{1}{2}
$$

注 在证明过程中, 注意条件的利用和不等式的变形.

例 4 设 $a 、 b 、 c$ 是正实数, 且满足 $a+b+c=1$. 证明:
$$
\frac{a-b c}{a+b c}+\frac{b-c a}{b+c a}+\frac{c-a b}{c+a b} \leqslant \frac{3}{2}
$$

证明 注意到
$$
1-\frac{a-b c}{a+b c}=\frac{2 b c}{a+b c}=\frac{2 b c}{1-b-c+b c}=\frac{2 b c}{(1-b)(1-c)}
$$

同理, $1-\frac{b-c a}{b+c a}=\frac{2 c a}{(1-c)(1-a)}, 1-\frac{c-a b}{c+a b}=\frac{2 a b}{(1-a)(1-b)}$.故原不等式等价于
$$
\frac{2 b c}{(1-b)(1-c)}+\frac{2 c a}{(1-c)(1-a)}+\frac{2 a b}{(1-a)(1-b)} \geqslant \frac{3}{2}
$$

化简后得
$$
4(b c+c a+a b-3 a b c) \geqslant 3(b c+c a+a b+1-a-b-c-a b c)
$$

即
$$
a b+b c+c a \geqslant 9 a b c
$$

从而要证 $\frac{1}{a}+\frac{1}{b}+\frac{1}{c} \geqslant 9$.

而 $\frac{1}{a}+\frac{1}{b}+\frac{1}{c}=(a+b+c)\left(\frac{1}{a}+\frac{1}{b}+\frac{1}{c}\right) \geqslant 9$, 因此, 原不等式成立.\\
例 5 设正实数 $a 、 b 、 c$ 满足 $a+b+c=3$. 证明:
$$
\frac{1}{2+a^{2}+b^{2}}+\frac{1}{2+b^{2}+c^{2}}+\frac{1}{2+c^{2}+a^{2}} \leqslant \frac{3}{4}
$$

证明 用符号 $\sum_{\mathrm{cyc}}$ 表示循环和, 即证明:


\begin{equation*}
\sum_{\mathrm{cyc}} \frac{1}{2+a^{2}+b^{2}} \leqslant \frac{3}{4} \tag{1}
\end{equation*}


由柯西不等式得
$$
\left(\sum_{\mathrm{cyc}} \frac{a^{2}+b^{2}}{2+a^{2}+b^{2}}\right) \sum_{\mathrm{cyc}}\left(2+a^{2}+b^{2}\right) \geqslant\left(\sum_{\mathrm{cyc}} \sqrt{a^{2}+b^{2}}\right)^{2}
$$

又
$$
\left(\sum_{\text {cyc }} \sqrt{a^{2}+b^{2}}\right)^{2}=2 \sum_{\text {cyc }} a^{2}+2 \sum_{\text {cyc }} \sqrt{\left(a^{2}+b^{2}\right)\left(a^{2}+c^{2}\right)}
$$

及
$$
\sqrt{\left(a^{2}+b^{2}\right)\left(a^{2}+c^{2}\right)} \geqslant a^{2}+b c
$$

则
$$
\begin{aligned}
& \left(\sum_{\text {cyc }} \sqrt{a^{2}+b^{2}}\right)^{2} \\
\geqslant & 2 \sum_{\text {cyc }} a^{2}+2 \sum_{\text {cyc }} a^{2}+2 \sum_{\text {cyc }} a b \\
= & 3 \sum_{\text {cyc }} a^{2}+(a+b+c)^{2}=9+3 \sum_{\text {cyc }} a^{2} \\
= & \frac{3}{2}\left(6+2 \sum_{\text {cyc }} a^{2}\right)=\frac{3}{2} \sum_{\text {cyc }}\left(2+a^{2}+b^{2}\right)
\end{aligned}
$$

故 $\left(\sum_{\text {cyc }} \frac{a^{2}+b^{2}}{2+a^{2}+b^{2}}\right) \sum_{\text {сус }}\left(2+a^{2}+b^{2}\right) \geqslant \frac{3}{2} \sum_{\text {сус }}\left(2+a^{2}+b^{2}\right)$.

所以,


\begin{equation*}
\sum_{\text {сус }} \frac{a^{2}+b^{2}}{2+a^{2}+b^{2}} \geqslant \frac{3}{2} \tag{2}
\end{equation*}


式(2)两边乘以 -1 ,再加 3 , 再除以 2 即得式(1).

例 6 已知 $x, y, z>0$, 且 $x y z=1$. 求证:

$\frac{(x+y-1)^{2}}{z}+\frac{(y+z-1)^{2}}{x}+\frac{(z+x-1)^{2}}{y} \geqslant 4(x+y+z)-12+\frac{9}{x+y+z}$.

证明 因为
$$
(a-b)^{2}=a^{2}-2 a b+b^{2},
$$

所以
$$
a^{2}=2 a b-b^{2}+(a-b)^{2},
$$

在 $b>0$ 时有
$$
\frac{a^{2}}{b}=2 a-b+\frac{(a-b)^{2}}{b}
$$

利用上式及柯西不等式, 可知
$$
\begin{aligned}
& \frac{(x+y-1)^{2}}{z}+\frac{(y+z-1)^{2}}{x}+\frac{(z+x-1)^{2}}{y} \\
= & 2(x+y-1)-z+\frac{(x+y-z-1)^{2}}{z} \\
& +2(y+z-1)-x+\frac{(y+z-x-1)^{2}}{x} \\
& +2(z+x-1)-y+\frac{(z+x-y-1)^{2}}{y} \\
= & 3(x+y+z)-6+\frac{(x+y-z-1)^{2}}{z}+\frac{(y+z-x-1)^{2}}{x}+\frac{(z+x-y-1)^{2}}{y} \\
\geqslant & 3(x+y+z)-6+\frac{(x+y+z-3)^{2}}{x+y+z} \\
= & 3(x+y+z)-6+\frac{(x+y+z)^{2}-6(x+y+z)+9}{x+y+z} \\
= & 4(x+y+z)-12+\frac{9}{x+y+z} .
\end{aligned}
$$

例 7 设非负实数 $a_{1}, a_{2}, \cdots, a_{n}$ 与 $b_{1}, b_{2}, \cdots, b_{n}$ 同时满足以下条件:

(1) $\sum_{i=1}^{n}\left(a_{i}+b_{i}\right)=1$;

(2) $\sum_{i=1}^{n} i\left(a_{i}-b_{i}\right)=0$;

(3) $\sum_{i=1}^{n} i^{2}\left(a_{i}+b_{i}\right)=10$.

求证:对任意 $1 \leqslant k \leqslant n$, 都有 $\max \left\{a_{k}, b_{k}\right\} \leqslant \frac{10}{10+k^{2}}$.

证明 对任意 $1 \leqslant k \leqslant n$, 有
$$
\begin{aligned}
\left(k a_{k}\right)^{2} & \leqslant\left(\sum_{i=1}^{n} i a_{i}\right)^{2}=\left(\sum_{i=1}^{n} i b_{i}\right)^{2} \\
& \leqslant\left(\sum_{i=1}^{n} i^{2} b_{i}\right)\left(\sum_{i=1}^{n} b_{i}\right) \quad \text { (柯西不等式) } \\
& =\left(10-\sum_{i=1}^{n} i^{2} a_{i}\right)\left(1-\sum_{i=1}^{n} a_{i}\right) \\
& \leqslant\left(10-k^{2} a_{k}\right)\left(1-a_{k}\right) \\
& =10-\left(10+k^{2}\right) a_{k}+k^{2} a_{k}^{2},
\end{aligned}
$$

从而 $a_{k} \leqslant \frac{10}{10+k^{2}}$.

同理有 $b_{k} \leqslant \frac{10}{10+k^{2}}$, 所以 $\max \left\{a_{k}, b_{k}\right\} \leqslant \frac{10}{10+k^{2}}$.

例 8 设 $a_{i} \in \mathbf{R}_{+}(i=1,2, \cdots, n)$, 如果对任意 $x_{i} \geqslant 0$,
$$
\sum_{i=1}^{n} r_{i}\left(x_{i}-a_{i}\right) \leqslant \sqrt{\sum_{i=1}^{n} x_{i}^{2}}-\sqrt{\sum_{i=1}^{n} a_{i}^{2}}
$$

求 $r_{i}(i=1,2, \cdots, n)$.

解 令 $x_{i}=0$, 则 $\sum_{i=1}^{n} r_{i} a_{i} \geqslant \sqrt{\sum_{i=1}^{n} a_{i}^{2}}$.

再令 $x_{i}=2 a_{i}$, 则 $\sum_{i=1}^{n} r_{i} a_{i} \leqslant \sqrt{\sum_{i=1}^{n} a_{i}^{2}}$.

于是
$$
\sum_{i=1}^{n} r_{i} a_{i}=\sqrt{\sum_{i=1}^{n} a_{i}^{2}}
$$

令 $x_{i}=r_{i}$, 则 $\sum_{i=1}^{n} r_{i}\left(r_{i}-a_{i}\right) \leqslant \sqrt{\sum_{i=1}^{n} r_{i}^{2}}-\sqrt{\sum_{i=1}^{n} a_{i}^{2}}$,

推出
$$
\sum_{i=1}^{n} r_{i}^{2} \leqslant \sqrt{\sum_{i=1}^{n} r_{i}^{2}}
$$

即 $\sum_{i=1}^{n} r_{i}^{2} \leqslant 1$. 由柯西不等式, 得
$$
\left(\sum_{i=1}^{n} r_{i} a_{i}\right)^{2} \leqslant\left(\sum_{i=1}^{n} r_{i}^{2}\right)\left(\sum_{i=1}^{n} a_{i}^{2}\right)
$$

等号成立充要条件是 $r_{i}=\lambda a_{i}$.

从而 $\sum_{i=1}^{n} r_{i}^{2} \geqslant 1$, 于是
$$
\sum_{i=1}^{n} r_{i}^{2}=1, \lambda=\frac{1}{\sqrt{\sum_{i=1}^{n} a_{i}^{2}}}, r_{i}=\frac{a_{i}}{\sqrt{\sum_{i=1}^{n} a_{i}^{2}}}
$$

经验证, $r_{i}=\frac{a_{i}}{\sqrt{\sum_{i=1}^{n} a_{i}^{2}}}(i=1,2, \cdots, n)$ 为所求.\\
上面的条件可以改为一般的形式:
$$
\sum_{i=1}^{n} r_{i}\left(x_{i}-a_{i}\right) \leqslant\left(\sum_{i=1}^{n} x_{i}^{m}\right)^{\frac{1}{m}}-\left(\sum_{i=1}^{n} a_{i}^{m}\right)^{\frac{1}{m}}
$$

其中 $m>1$ 为给定的常数.

利用赫尔德不等式, 得
$$
r_{i}=\left[\frac{a_{i}^{m}}{\sum_{i=1}^{n} a_{i}^{m}}\right]^{\frac{m-1}{m}}(i=1,2, \cdots, n)
$$

例 9 设 $x_{i}, y_{i}, \cdots, z_{i} \in \mathbf{R}(i=1,2, \cdots, n)$, 求证:
$$
\sum_{i=1}^{n} \sqrt{x_{i}^{2}+y_{i}^{2}+\cdots+z_{i}^{2}} \geqslant \sqrt{\left(\sum_{i=1}^{n} x_{i}\right)^{2}+\left(\sum_{i=1}^{n} y_{i}\right)^{2}+\cdots+\left(\sum_{i=1}^{n} z_{i}\right)^{2}}
$$

证明 令 $a=\sum_{i=1}^{n} x_{i}, b=\sum_{i=1}^{n} y_{i}, \cdots, c=\sum_{i=1}^{n} z_{i}$. 不妨设 $a^{2}+b^{2}+\cdots+$ $c^{2} \neq 0$, 则由柯西不等式, 得
$$
\left(a^{2}+b^{2}+\cdots+c^{2}\right)\left(x_{i}^{2}+y_{i}^{2}+\cdots+z_{i}^{2}\right) \geqslant\left(a x_{i}+b y_{i}+\cdots+c z_{i}\right)^{2}
$$

即

$a x_{i}+b y_{i}+\cdots+c z_{i} \leqslant \sqrt{a^{2}+b^{2}+\cdots+c^{2}} \cdot \sqrt{x_{i}^{2}+y_{i}^{2}+\cdots+z_{i}^{2}}$.

求和, 得

$a^{2}+b^{2}+\cdots+c^{2} \leqslant \sqrt{a^{2}+b^{2}+\cdots+c^{2}} \sum_{i=1}^{n} \sqrt{x_{i}^{2}+y_{i}^{2}+\cdots+z_{i}^{2}}$.

故
$$
\sum_{i=1}^{n} \sqrt{x_{i}^{2}+y_{i}^{2}+\cdots+z_{i}^{2}} \geqslant \sqrt{a^{2}+b^{2}+\cdots+c^{2}}
$$

本例如果用向量方法证明, 会更简洁.

例 10 设 $a_{i} \in \mathbf{R}_{+}, 1 \leqslant i \leqslant n$. 证明:
$$
\frac{1}{\frac{1}{1+a_{1}}+\frac{1}{1+a_{2}}+\cdots+\frac{1}{1+a_{n}}}-\frac{1}{\frac{1}{a_{1}}+\frac{1}{a_{2}}+\cdots+\frac{1}{a_{n}}} \geqslant \frac{1}{n}
$$

证明 令 $\sum_{i=1}^{n} \frac{1}{a_{i}}=a$, 则 $\sum_{i=1}^{n} \frac{1+a_{i}}{a_{i}}=n+a$. 由柯西不等式, 得
$$
\sum_{i=1}^{n} \frac{a_{i}}{1+a_{i}} \cdot \sum_{i=1}^{n} \frac{1+a_{i}}{a_{i}} \geqslant n^{2}
$$

所以 $\sum_{i=1}^{n} \frac{a_{i}}{a_{i}+1} \geqslant \frac{n^{2}}{n+a}$, 以及
$$
\begin{aligned}
\sum_{i=1}^{n} \frac{1}{a_{i}+1} & =\sum_{i=1}^{n}\left(1-\frac{a_{i}}{a_{i}+1}\right)=n-\sum_{i=1}^{n} \frac{a_{i}}{a_{i}+1} \\
& \leqslant n-\frac{n^{2}}{n+a}=\frac{n a}{n+a}
\end{aligned}
$$

于是
$$
\begin{aligned}
& \frac{1}{\frac{1}{1+a_{1}}+\frac{1}{1+a_{2}}+\cdots+\frac{1}{1+a_{n}}}-\frac{1}{\frac{1}{a_{1}}+\frac{1}{a_{2}}+\cdots+\frac{1}{a_{n}}} \\
\geqslant & \frac{1}{\frac{n a}{n+a}}-\frac{1}{a}=\frac{n+a}{n a}-\frac{1}{a}=\frac{a}{n a}=\frac{1}{n}
\end{aligned}
$$

故命题成立.

注 此题的证明方法较多,不妨自己试一试.

例 11 设 $n$ 为正整数, $x_{1} \leqslant x_{2} \leqslant \cdots \leqslant x_{n}$ 为实数, 证明:

(1) $\left(\sum_{i, j=1}^{n}\left|x_{i}-x_{j}\right|\right)^{2} \leqslant \frac{2\left(n^{2}-1\right)}{3} \sum_{i, j=1}^{n}\left(x_{i}-x_{j}\right)^{2}$;

(2)第 (1) 小题等号成立的充要条件是 $x_{1}, x_{2}, \cdots, x_{n}$ 为等差数列.

证明 (1) 不失一般性, 可设 $\sum_{i=1}^{n} x_{i}=0$, 得
$$
\sum_{i, j=1}^{n}\left|x_{i}-x_{j}\right|=2 \sum_{i<j}\left(x_{j}-x_{i}\right)=2 \sum_{i=1}^{n}(2 i-n-1) x_{i}
$$

由柯西不等式, 得
$$
\begin{aligned}
\left(\sum_{i, j=1}^{n}\left|x_{i}-x_{j}\right|\right)^{2} & \leqslant 4 \sum_{i=1}^{n}(2 i-n-1)^{2} \sum_{i=1}^{n} x_{i}^{2} \\
& =4 \times \frac{n(n-1)(n+1)}{3} \sum_{i=1}^{n} x_{i}^{2}
\end{aligned}
$$

另一方面,
$$
\sum_{i, j=1}^{n}\left(x_{i}-x_{j}\right)^{2}=n \sum_{i=1}^{n} x_{i}^{2}-2 \sum_{i=1}^{n} x_{i} \sum_{j=1}^{n} x_{j}+n \sum_{j=1}^{n} x_{j}^{2}=2 n \sum_{i=1}^{n} x_{i}^{2}
$$

从而
$$
\left(\sum_{i, j=1}^{n}\left|x_{i}-x_{j}\right|\right)^{2} \leqslant \frac{2\left(n^{2}-1\right)}{3} \sum_{i, j=1}\left(x_{i}-x_{j}\right)^{2}
$$

(2)如果等号成立, 则对某个 $k, x_{i}=k(2 i-n-1)$, 则 $x_{1}, x_{2}, \cdots, x_{n}$ 为等差数列. 另一方面, 如果 $x_{1}, x_{2}, \cdots, x_{n}$ 为等差数列, 公差为 $d$, 则
$$
x_{i}=\frac{d}{2}(2 i-n-1)+\frac{x_{1}+x_{n}}{2}
$$

将每个 $x_{i}$ 减去 $\frac{x_{1}+x_{n}}{2}$, 就有 $x_{i}=\frac{d}{2}(2 i-n-1)$, 且 $\sum_{i=1}^{n} x_{i}=0$, 这时等号成立.

例 12 证明: 满足条件

(1) $a_{1}+a_{2}+\cdots+a_{n} \geqslant n^{2}$

(2) $a_{1}^{2}+a_{2}^{2}+\cdots+a_{n}^{2} \leqslant n^{3}+1$

的整数只有 $\left(a_{1}, a_{2}, \cdots, a_{n}\right)=(n, n, \cdots, n)$.

证明 设 $\left(a_{1}, a_{2}, \cdots, a_{n}\right)$ 是满足条件的整数组, 则由柯西不等式, 得
$$
a_{1}^{2}+a_{2}^{2}+\cdots+a_{n}^{2} \geqslant \frac{1}{n}\left(a_{1}+a_{2}+\cdots+a_{n}\right)^{2} \geqslant n^{3}
$$

结合 $a_{1}^{2}+a_{2}^{2}+\cdots+a_{n}^{2} \leqslant n^{3}+1$, 可知只能 $\sum_{i=1}^{n} a_{i}^{2}=n^{3}$ 或者 $\sum_{i=1}^{n} a_{i}^{2}=$ $n^{3}+1$.

当 $\sum_{i=1}^{n} a_{i}^{2}=n^{3}$ 时, 由柯西不等式取等号得 $a_{1}=a_{2}=\cdots=a_{n}$, 即 $a_{i}^{2}=n^{2}$, $1 \leqslant i \leqslant n$. 再由 $\sum_{i=1}^{n} a_{i} \geqslant n^{2}$, 则只有 $a_{1}=a_{2}=\cdots=a_{n}=n$.

当 $\sum_{i=1}^{n} a_{i}=n^{3}+1$ 时, 则令 $b_{i}=a_{i}-n$, 得
$$
\sum_{i=1}^{n} b_{i}^{2}=\sum_{i=1}^{n} a_{i}^{2}-2 n \sum_{i=1}^{n} a_{i}+n^{3} \leqslant 2 n^{3}+1-2 n \sum_{i=1}^{n} a_{i} \leqslant 1
$$

于是 $b_{i}^{2}$ 只能是 0 或者 1 , 且 $b_{1}^{2}, b_{2}^{2}, \cdots, b_{n}^{2}$ 中至多有一个为 1 . 如果都为零, 则 $a_{i}=n, \sum_{i=1}^{n} a_{i}^{2}=n^{3} \neq n^{3}+1$, 矛盾. 如果 $b_{1}^{2}, b_{2}^{2}, \cdots, b_{n}^{2}$ 中有一个为 1 ,则 $\sum_{i=1}^{n} a_{i}^{2}=n^{3} \pm 2 n+1 \neq n^{3}+1$, 也矛盾. 故只有 $\left(a_{1}, a_{2}, \cdots, a_{n}\right)=(n$,\\
$n, \cdots, n)$ 为唯一一组整数解.

例 13 证明:关于两个三角形的匹窦不等式
$$
a^{2}\left(b_{1}^{2}+c_{1}^{2}-a_{1}^{2}\right)+b^{2}\left(c_{1}^{2}+a_{1}^{2}-b_{1}^{2}\right)+c^{2}\left(a_{1}^{2}+b_{1}^{2}-c_{1}^{2}\right) \geqslant 16 S S_{1},
$$

这里 $a 、 b 、 c 、 S ; a_{1} 、 b_{1} 、 c_{1} 、 S_{1}$ 分别为两个三角形的边长和面积.

证明 由柯西不等式, 得
$$
\begin{aligned}
& 16 S S_{1}+2 a^{2} a_{1}^{2}+2 b^{2} b_{1}^{2}+2 c^{2} c_{1}^{2} \\
\leqslant & \left(16 S_{1}^{2}+2 a_{1}^{4}+2 b_{1}^{4}+2 c^{4}\right)^{\frac{1}{2}}\left(16 S^{2}+2 a^{4}+2 b^{4}+2 c^{4}\right)^{\frac{1}{2}} \\
= & \left(a_{1}^{2}+b_{1}^{2}+c_{1}^{2}\right)\left(a^{2}+b^{2}+c^{2}\right),
\end{aligned}
$$

所以
$$
a^{2}\left(b_{1}^{2}+c_{1}^{2}-a_{1}^{2}\right)+b^{2}\left(c_{1}^{2}+a_{1}^{2}-b_{1}^{2}\right)+c^{2}\left(a_{1}^{2}+b_{1}^{2}-c_{1}^{2}\right) \geqslant 16 S S_{1} .
$$

注 在证明过程中, 常常进行一些恒等的变形.

例 14 设 $a 、 b 、 c$ 是三角形的三边长,求证:
$$
a^{2} b(a-b)+b^{2} c(b-c)+c^{2} a(c-a) \geqslant 0
$$

证明 显然存在正数 $x 、 y 、 z$ 使得 $a=y+z, b=z+x, c=x+y$. 由于
$$
\begin{aligned}
& a^{2} b(a-b)=(y+z)^{2}(z+x)(y-x) \\
= & (y+z)(z+x)\left(y^{2}-z^{2}\right)+(y+z)^{2}\left(z^{2}-x^{2}\right)
\end{aligned}
$$

同样处理 $b^{2} c(b-c), c^{2} a(c-a)$, 所以
$$
\begin{aligned}
& a^{2} b(a-b)+b^{2} c(b-c)+c^{2} a(c-a) \\
= & 2 x(y-z) y^{2}+2 y(z-x) z^{2}+2 z(x-y) x^{2}
\end{aligned}
$$

原不等式等价于
$$
x y z(x+y+z) \leqslant x y^{3}+y z^{3}+z x^{3} .
$$

由柯西不等式, 得

则
$$
\begin{gathered}
x+y+z \leqslant\left(\frac{x^{2}}{y}+\frac{y^{2}}{z}+\frac{z^{2}}{x}\right)^{\frac{1}{2}}(x+y+z)^{\frac{1}{2}} \\
x+y+z \leqslant \frac{x^{2}}{y}+\frac{y^{2}}{z}+\frac{z^{2}}{x}
\end{gathered}
$$

于是原不等式成立, 且当 $a=b=c$ 时等号成立.\\
例 15 设 $x_{i}>0, x_{i} y_{i}-z_{i}^{2}>0, i=1,2$, 求证:
$$
\frac{8}{\left(x_{1}+x_{2}\right)\left(y_{1}+y_{2}\right)-\left(z_{1}+z_{2}\right)^{2}} \leqslant \frac{1}{x_{1} y_{1}-z_{1}^{2}}+\frac{1}{x_{2} y_{2}-z_{2}^{2}}
$$

证明 注意到不等式的右边 $\geqslant \frac{2}{\left[\left(x_{1} y_{1}-z_{1}^{2}\right)\left(x_{2} y_{2}-z_{2}^{2}\right)\right]^{\frac{1}{2}}}$, 考虑证明一个更强的结论:
$$
\begin{aligned}
& \left(x_{1}+x_{2}\right)\left(y_{1}+y_{2}\right)-\left(z_{1}+z_{2}\right)^{2} \geqslant 4\left[\left(x_{1} y_{1}-z_{1}^{2}\right)\left(x_{2} y_{2}-z_{2}^{2}\right)\right]^{\frac{1}{2}} . \\
& \text { 令 } u_{i}=\sqrt{x_{i} y_{i}-z_{i}^{2}}, i=1,2 \text {, 由于 } 4 u_{1} u_{2} \leqslant\left(u_{1}+u_{2}\right)^{2} \text {, 则只要证明 } \\
& \quad\left(x_{1}+x_{2}\right)\left(y_{1}+y_{2}\right)-\left(z_{1}+z_{2}\right)^{2} \geqslant\left(u_{1}+u_{2}\right)^{2},
\end{aligned}
$$

等价于
$$
\left(x_{1}+x_{2}\right)\left(y_{1}+y_{2}\right) \geqslant\left(u_{1}+u_{2}\right)^{2}+\left(z_{1}+z_{2}\right)^{2}
$$

由柯西不等式, 得
$$
\begin{aligned}
\left(x_{1}+x_{2}\right)\left(y_{1}+y_{2}\right) & \geqslant\left(\sqrt{x_{1} y_{1}}+\sqrt{x_{2} y_{2}}\right)^{2} \\
& =\left(\sqrt{u_{1}^{2}+z_{1}^{2}}+\sqrt{u_{2}^{2}+z_{2}^{2}}\right)^{2} \\
& =\left(u_{1}^{2}+z_{1}^{2}\right)+2 \sqrt{u_{1}^{2}+z_{1}^{2}} \sqrt{u_{2}^{2}+z_{2}^{2}}+\left(u_{2}^{2}+z_{2}^{2}\right) \\
& \geqslant\left(u_{1}^{2}+z_{1}^{2}\right)+2\left(u_{1} u_{2}+z_{1} z_{2}\right)+\left(u_{2}^{2}+z_{2}^{2}\right) \\
& =\left(u_{1}+u_{2}\right)^{2}+\left(z_{1}+z_{2}\right)^{2} .
\end{aligned}
$$

从而原不等式成立, 且等号成立的充分必要条件为 $x_{1}=x_{2}, y_{1}=y_{2}$, $z_{1}=z_{2}$.

注 该例题也可以直接用柯西不等式证明, 关于它的推广, 可以参见引文或练习。

例 16 设 $a_{i}, b_{i}, c_{i}, d_{i} \in \mathbf{R}_{+}(i=1,2, \cdots, n)$, 求证:
$$
\left(\sum a_{i} b_{i} c_{i} d_{i}\right)^{4} \leqslant \sum a_{i}^{4} \cdot \sum b_{i}^{4} \cdot \sum c_{i}^{4} \cdot \sum d_{i}^{4}
$$

证明 两次利用柯西不等式, 得
$$
\begin{aligned}
\text { 左边 } & =\left[\sum\left(a_{i} b_{i}\right)\left(c_{i} d_{i}\right)\right]^{4} \\
& \leqslant\left[\sum\left(a_{i} b_{i}\right)^{2}\right]^{2}\left[\sum\left(c_{i} d_{i}\right)^{2}\right]^{2} \\
& =\left[\sum a_{i}^{2} b_{i}^{2}\right]^{2}\left[\sum c_{i}^{2} d_{i}^{2}\right]^{2} \\
& \leqslant \sum a_{i}^{4} \cdot \sum b_{i}^{4} \cdot \sum c_{i}^{4} \cdot \sum d_{i}^{4}
\end{aligned}
$$

故命题成立.

例 17 设 $t_{a} 、 t_{b} 、 t_{c}$ 分别是 $\triangle A B C$ 的 $\angle A 、 \angle B 、 \angle C$ 的角平分线的长,证明:
$$
\sum_{\text {cyc }} \frac{b c}{t_{a}^{2}} \geqslant 4
$$

证明 不难求得 $t_{a}^{2}=\frac{b c\left[(b+c)^{2}-a^{2}\right]}{(b+c)^{2}}$, 则 $\frac{b c}{t_{a}^{2}}=\frac{(b+c)^{2}}{(b+c)^{2}-a^{2}}$.

同理可得
$$
\frac{a c}{t_{b}^{2}}=\frac{(a+c)^{2}}{(a+c)^{2}-b^{2}}, \frac{a b}{t_{c}^{2}}=\frac{(a+b)^{2}}{(a+b)^{2}-c^{2}}
$$

则
$$
\sum_{\mathrm{cyc}} \frac{b c}{t_{a}^{2}}=\sum_{\mathrm{cyc}} \frac{(a+b)^{2}}{(a+b)^{2}-c^{2}} \geqslant \frac{4(a+b+c)^{2}}{\sum_{\mathrm{cyc}}\left[(a+b)^{2}-c^{2}\right]}=\frac{4(a+b+c)^{2}}{(a+b+c)^{2}}=4
$$

所以原不等式成立.

例 18 设 $x, y, z, w \in \mathbf{R}_{+}, \alpha, \beta, \gamma, \theta$ 满足 $\alpha+\beta+\gamma+\theta=(2 k+1) \pi$, $k \in \mathbf{Z}$.

求证 $\quad(x \sin \alpha+y \sin \beta+z \sin \gamma+w \sin \theta)^{2}$
$$
\leqslant \frac{(x y+z w)(x z+y w)(x w+y z)}{x y z w}
$$

证明 令 $u=x \sin \alpha+y \sin \beta, v=z \sin \gamma+w \sin \theta$, 则
$$
\begin{aligned}
u^{2} & =(x \sin \alpha+y \sin \beta)^{2} \\
& \leqslant(x \sin \alpha+y \sin \beta)^{2}+(x \cos \alpha-y \cos \beta)^{2} \\
& =x^{2}+y^{2}-2 x y \cos (\alpha+\beta)
\end{aligned}
$$

于是
$$
\cos (\alpha+\beta) \leqslant \frac{x^{2}+y^{2}-u^{2}}{2 x y}
$$

同理
$$
\cos (\gamma+\theta) \leqslant \frac{z^{2}+w^{2}-v^{2}}{2 z w}
$$

两式相加, 并由假设得
$$
0=\cos (\alpha+\beta)+\cos (\gamma+\theta) \leqslant \frac{x^{2}+y^{2}-u^{2}}{2 x y}+\frac{z^{2}+w^{2}-v^{2}}{2 z w}
$$

即
$$
\frac{u^{2}}{x y}+\frac{v^{2}}{z w} \leqslant \frac{x^{2}+y^{2}}{x y}+\frac{z^{2}+w^{2}}{z w}
$$

从而
$$
(x \sin \alpha+y \sin \beta+z \sin \gamma+w \sin \theta)^{2}=(u+v)^{2}
$$
$$
\begin{aligned}
& \leqslant(x y+z w)\left(\frac{u^{2}}{x y}+\frac{v^{2}}{z w}\right) \\
& \leqslant(x y+z w)\left(\frac{x^{2}+y^{2}}{x y}+\frac{z^{2}+w^{2}}{z w}\right) \\
& =\frac{(x y+z w)(x z+y w)(x w+y z)}{x y z w} .
\end{aligned}
$$

故命题成立.

例 19 给定正整数 $n \geqslant 2$, 设正整数 $a_{i}(i=1,2, \cdots, n)$ 满足 $a_{1}<$ $a_{2}<\cdots<a_{n}$ 以及 $\sum_{i=1}^{n} \frac{1}{a_{i}} \leqslant 1$. 求证: 对任意实数 $x$, 有
$$
\left(\sum_{i=1}^{n} \frac{1}{a_{i}^{2}+x^{2}}\right)^{2} \leqslant \frac{1}{2} \cdot \frac{1}{a_{1}\left(a_{1}-1\right)+x^{2}}
$$

证明 当 $x^{2} \geqslant a_{1}\left(a_{1}-1\right)$ 时, 由于 $\sum \frac{1}{a_{i}} \leqslant 1$, 得
$$
\begin{aligned}
\left(\sum_{i=1}^{n} \frac{1}{a_{i}^{2}+x^{2}}\right)^{2} & \leqslant\left(\sum_{i=1}^{n} \frac{1}{2 a_{i}|x|}\right)^{2}=\frac{1}{4 x^{2}}\left(\sum_{i=1}^{n} \frac{1}{a_{i}}\right)^{2} \\
& \leqslant \frac{1}{4 x^{2}} \leqslant \frac{1}{2} \cdot \frac{1}{a_{1}\left(a_{1}-1\right)+x^{2}}
\end{aligned}
$$

当 $x^{2}<a_{1}\left(a_{1}-1\right)$ 时, 由柯西不等式, 得
$$
\left(\sum_{i=1}^{n} \frac{1}{a_{i}^{2}+x^{2}}\right)^{2} \leqslant\left(\sum_{i=1}^{n} \frac{1}{a_{i}}\right) \sum_{i=1}^{n} \frac{a_{i}}{\left(a_{i}^{2}+x^{2}\right)^{2}} \leqslant \sum_{i=1}^{n} \frac{a_{i}}{\left(a_{i}^{2}+x^{2}\right)^{2}}
$$

对于正整数 $a_{1}<a_{2}<\cdots<a_{n}$, 有 $a_{i+1} \geqslant a_{i}+1, i=1,2, \cdots, n-1$,且.
$$
\begin{aligned}
\frac{2 a_{i}}{\left(a_{i}^{2}+x^{2}\right)^{2}} & \leqslant \frac{2 a_{i}}{\left(a_{i}^{2}+x^{2}+\frac{1}{4}\right)^{2}-a_{i}^{2}} \\
& =\frac{1}{\left(a_{i}-\frac{1}{2}\right)^{2}+x^{2}}-\frac{1}{\left(a_{i}+\frac{1}{2}\right)^{2}+x^{2}} \\
& \leqslant \frac{1}{\left(a_{i}-\frac{1}{2}\right)^{2}+x^{2}}-\frac{1}{\left(a_{i+1}-\frac{1}{2}\right)^{2}+x^{2}}, i=1,2, \cdots, n-1
\end{aligned}
$$

同理
$$
\begin{aligned}
\frac{2 a_{n}}{\left(a_{n}^{2}+x^{2}\right)^{2}} & \leqslant \frac{1}{\left(a_{n}-\frac{1}{2}\right)^{2}+x^{2}}-\frac{1}{\left(a_{n}+\frac{1}{2}\right)^{2}+x^{2}} \\
& \leqslant \frac{1}{\left(a_{n}-\frac{1}{2}\right)^{2}+x^{2}}
\end{aligned}
$$

所以
$$
\begin{aligned}
\sum_{i=1}^{n} \frac{a_{i}}{\left(a_{i}^{2}+x^{2}\right)^{2}} \leqslant & \frac{1}{2} \sum_{i=1}^{n-1}\left[\frac{1}{\left(a_{i}-\frac{1}{2}\right)^{2}+x^{2}}-\frac{1}{\left(a_{i+1}-\frac{1}{2}\right)^{2}+x^{2}}\right] \\
& +\frac{1}{\left(a_{n}-\frac{1}{2}\right)^{2}+x^{2}} \\
\leqslant & \frac{1}{2} \cdot \frac{1}{\left(a_{1}-\frac{1}{2}\right)^{2}+x^{2}} \leqslant \frac{1}{2} \cdot \frac{1}{a_{1}\left(a_{1}-1\right)+x^{2}}
\end{aligned}
$$

故命题成立.

例 20 设 $x \in\left(0, \frac{\pi}{2}\right), n \in \mathbf{N}$, 求证:
$$
\left(\frac{1-\sin ^{2 n} x}{\sin ^{2 n} x}\right)\left(\frac{1-\cos ^{2 n} x}{\cos ^{2 n} x}\right) \geqslant\left(2^{n}-1\right)^{2}
$$

\section*{证明 因为}
$$
\begin{aligned}
1-\sin ^{2 n} x & =\left(1-\sin ^{2} x\right)\left(1+\sin ^{2} x+\sin ^{4} x+\cdots+\sin ^{2(n-1)} x\right) \\
& =\cos ^{2} x\left(1+\sin ^{2} x+\sin ^{4} x+\cdots+\sin ^{2(n-1)} x\right) \\
1-\cos ^{2 n} x & =\left(1-\cos ^{2} x\right)\left(1+\cos ^{2} x+\cos ^{4} x+\cdots+\cos ^{2(n-1)} x\right) \\
& =\sin ^{2} x\left(1+\cos ^{2} x+\cos ^{4} x+\cdots+\cos ^{2(n-1)} x\right)
\end{aligned}
$$

所以由柯西不等式, 得
$$
\begin{aligned}
& \left(\frac{1-\sin ^{2 n} x}{\sin ^{2 n} x}\right)\left(\frac{1-\cos ^{2 n} x}{\cos ^{2 n} x}\right) \\
= & \frac{1}{\sin ^{2 n-2} x}\left(1+\sin ^{2} x+\sin ^{4} x+\cdots+\sin ^{2 n-2} x\right) \\
& \cdot \frac{1}{\cos ^{2 n-2} x}\left(1+\cos ^{2} x+\cos ^{4} x+\cdots+\cos ^{2 n-2} x\right) \\
= & \left(1+\frac{1}{\sin ^{2} x}+\frac{1}{\sin ^{4} x}+\cdots+\frac{1}{\sin ^{2 n-2} x}\right)
\end{aligned}
$$
$$
\begin{aligned}
& \cdot\left(1+\frac{1}{\cos ^{2} x}+\frac{1}{\cos ^{4} x}+\cdots+\frac{1}{\cos ^{2 n-2} x}\right) \\
\geqslant & {\left[1+\frac{1}{\sin x \cos x}+\frac{1}{(\sin x \cos x)^{2}}+\cdots+\frac{1}{(\sin x \cos x)^{n-1}}\right]^{2} } \\
= & {\left[1+\frac{2}{\sin 2 x}+\frac{4}{(\sin 2 x)^{2}}+\cdots+\frac{2^{n-1}}{(\sin 2 x)^{n-1}}\right]^{2} } \\
\geqslant & \left(1+2+2^{2}+\cdots+2^{n-1}\right)^{2}=\left(2^{n}-1\right)^{2}
\end{aligned}
$$

例 21 设 $a_{1}, a_{2}, \cdots, a_{n}$ 是一个有无穷项的实数列, 对于所有正整数 $i$,存在一个实数 $c$, 使得 $0 \leqslant a_{i} \leqslant c$, 且 $\left|a_{i}-a_{j}\right| \geqslant \frac{1}{i+j}$ 对所有正整数 $i 、 j$ $(i \neq j)$ 成立. 证明: $c \geqslant 1$.

证明 对于 $n \geqslant 2$, 设 $\sigma(1), \sigma(2), \cdots, \sigma(n)$ 是 $1,2, \cdots, n$ 的一个排列,且满足
$$
0 \leqslant a_{\sigma(1)}<a_{\sigma(2)}<\cdots<a_{\sigma(n)} \leqslant c
$$

则
$$
c \geqslant a_{\sigma(n)}-a_{\sigma(1)}
$$
$$
\begin{aligned}
& \left(a_{\sigma(n)}-a_{\sigma(n-1)}\right)+\left(a_{\sigma(n-1)}-a_{\sigma(n-2)}\right)+\cdots+\left(a_{\sigma(2)}-a_{\sigma(1)}\right) \\
\geqslant & \frac{1}{\sigma(n)+\sigma(n-1)}+\frac{1}{\sigma(n-1)+\sigma(n-2)}+\cdots+\frac{1}{\sigma(2)+\sigma(1)}
\end{aligned}
$$

由柯西不等式, 得
$$
\begin{gathered}
{\left[\frac{1}{\sigma(n)+\sigma(n-1)}+\frac{1}{\sigma(n-1)+\sigma(n-2)}+\cdots+\frac{1}{\sigma(2)+\sigma(1)}\right]} \\
{[(\sigma(n)+\sigma(n-1))+(\sigma(n-1)+\sigma(n-2))+\cdots+(\sigma(2)+\sigma(1))] \geqslant(n-1)^{2}}
\end{gathered}
$$

对所有正整数 $n \geqslant 2$, 我们有
$$
\begin{aligned}
c & \geqslant \frac{1}{\sigma(n)+\sigma(n-1)}+\frac{1}{\sigma(n-1)+\sigma(n-2)}+\cdots+\frac{1}{\sigma(2)+\sigma(1)} \\
& \geqslant \frac{(n-1)^{2}}{2[\sigma(1)+\sigma(2)+\cdots+\sigma(n)]-\sigma(1)-\sigma(n)} \\
& =\frac{(n-1)^{2}}{n(n+1)-\sigma(1)-\sigma(n)} \\
& \geqslant \frac{(n-1)^{2}}{n^{2}+n-3} \geqslant \frac{n-1}{n+3}=1-\frac{4}{n+3}
\end{aligned}
$$

故 $c \geqslant 1$.

例 22 设 $a 、 b 、 c$ 为正实数. 求证:
$$
\frac{(2 a+b+c)^{2}}{2 a^{2}+(b+c)^{2}}+\frac{(a+2 b+c)^{2}}{2 b^{2}+(a+c)^{2}}+\frac{(a+b+2 c)^{2}}{2 c^{2}+(b+a)^{2}} \leqslant 8
$$

证明 在第二章中, 我们用了两种不同的方法证明了这个不等式, 这里,用柯西不等式给出另一种新的证明.

由柯西不等式, 得
$$
\begin{aligned}
\sqrt{\frac{2 a^{2}+\frac{(b+c)^{2}}{2}+\frac{(b+c)^{2}}{2}}{3}} & \geqslant \frac{\sqrt{2} a+\frac{\sqrt{2}}{2}(b+c)+\frac{\sqrt{2}}{2}(b+c)}{3} \\
& =\frac{\sqrt{2}(a+b+c)}{3} .
\end{aligned}
$$

于是 $2 a^{2}+(b+c)^{2} \geqslant \frac{2(a+b+c)^{2}}{3}$. 同理可得
$$
2 b^{2}+(c+a)^{2} \geqslant \frac{2(a+b+c)^{2}}{3}, 2 c^{2}+(a+b)^{2} \geqslant \frac{2(a+b+c)^{2}}{3}
$$

如果 $4 a \geqslant b+c, 4 b \geqslant c+a, 4 c \geqslant a+b$, 则
$$
\begin{aligned}
\frac{(2 a+b+c)^{2}}{2 a^{2}+(b+c)^{2}} & =2+\frac{(4 a-b-c)(b+c)}{2 a^{2}+(b+c)^{2}} \\
& \leqslant 2+\frac{3\left(4 a b+4 a c-b^{2}-2 b c-c^{2}\right)}{2(a+b+c)^{2}}
\end{aligned}
$$

同理可得
$$
\begin{aligned}
& \frac{(a+2 b+c)^{2}}{2 b^{2}+(a+c)^{2}} \leqslant 2+\frac{3\left(4 b c+4 b a-a^{2}-2 a c-c^{2}\right)}{2(a+b+c)^{2}} \\
& \frac{(a+b+2 c)^{2}}{2 c^{2}+(a+b)^{2}} \leqslant 2+\frac{3\left(4 c b+4 c a-a^{2}-2 b a-b^{2}\right)}{2(a+b+c)^{2}}
\end{aligned}
$$

三式相加, 得
$$
\begin{aligned}
& \frac{(2 a+b+c)^{2}}{2 a^{2}+(b+c)^{2}}+\frac{(a+2 b+c)^{2}}{2 b^{2}+(a+c)^{2}}+\frac{(a+b+2 c)^{2}}{2 c^{2}+(a+b)^{2}} \\
\leqslant & 6+\frac{3\left(6 a b+6 b c+6 c a-2 a^{2}-2 b^{2}-2 c^{2}\right)}{2(a+b+c)^{2}} \\
= & \frac{21}{2}-\frac{15}{2} \cdot \frac{a^{2}+b^{2}+c^{2}}{(a+b+c)^{2}} \\
\leqslant & \frac{21}{2}-\frac{15}{2} \times \frac{1}{3}=8 .
\end{aligned}
$$

当上述假设不成立时, 不妨设 $4 a<b+c$, 则
$$
\frac{(2 a+b+c)^{2}}{2 a^{2}+(b+c)^{2}}<2
$$

由柯西不等式, 得
$$
[b+b+(c+a)]^{2} \leqslant\left(b^{2}+b^{2}+(c+a)^{2}\right)(1+1+1)
$$

于是
$$
\frac{(2 b+a+c)^{2}}{2 b^{2}+(a+c)^{2}} \leqslant 3
$$

同理可得
$$
\frac{(2 c+b+a)^{2}}{2 c^{2}+(a+b)^{2}} \leqslant 3
$$

所以
$$
\frac{(2 a+b+c)^{2}}{2 a^{2}+(b+c)^{2}}+\frac{(a+2 b+c)^{2}}{2 b^{2}+(a+c)^{2}}+\frac{(a+b+2 c)^{2}}{2 c^{2}+(a+b)^{2}} \leqslant 8
$$

综上可知原不等式成立. 当且仅当 $a=b=c$ 时等号成立.

对于三种不同的证明方法,希望大家能好好理解.

例 23 设正整数 $n \geqslant 2, a_{i} \in \mathbf{R}, 1 \leqslant i \leqslant n$. 证明: 可以选择 $\varepsilon_{i} \in\{-1$, $1\}, 1 \leqslant i \leqslant n$. 使得
$$
\left(\sum_{i=1}^{n} a_{i}\right)^{2}+\left(\sum_{i=1}^{n} \varepsilon_{i} a_{i}\right)^{2} \leqslant(n+1) \sum_{i=1}^{n} a_{i}^{2}
$$

证法一 取 $\varepsilon_{i}=1,1 \leqslant i \leqslant\left[\frac{n}{2}\right] ; \varepsilon_{i}=-1,\left[\frac{n}{2}\right]+1 \leqslant i \leqslant n$, 这里 $[x]$为实数 $x$ 的整数部分. 则
$$
\begin{aligned}
& \left(\sum_{i=1}^{n} a_{i}\right)^{2}+\left(\sum_{i=1}^{n} \varepsilon_{i} a_{i}\right)^{2} \\
= & \left(\sum_{i=1}^{\left[\frac{n}{2}\right]} a_{i}+\sum_{j=\left[\frac{n}{2}\right]+1}^{n} a_{j}\right)^{2}+\left(\sum_{i=1}^{\left[\frac{n}{2}\right]} a_{i}-\sum_{j=\left[\frac{n}{2}\right]+1}^{n} a_{j}\right)^{2} \\
= & 2\left(\sum_{i=1}^{\left[\frac{n}{2}\right]} a_{i}\right)^{2}+2\left(\sum_{j=\left[\frac{n}{2}\right]+1}^{n} a_{j}\right)^{2} \\
\leqslant & 2\left[\frac{n}{2}\right]\left(\sum_{i=1}^{\left[\frac{n}{2}\right]} a_{i}^{2}\right)+2\left(n-\left[\frac{n}{2}\right]\right)\left(\sum_{j=\left[\frac{n}{2}\right]+1}^{n} a_{j}^{2}\right)(\text { 柯西不等式 }) \\
= & 2\left[\frac{n}{2}\right]\left(\sum_{i=1}^{\left[\frac{n}{2}\right]} a_{i}^{2}\right)+2\left(\left[\frac{n+1}{2}\right]\right)\left(\sum_{j=\left[\frac{n}{2}\right]+1}^{n} a_{j}^{2}\right)\left(n-\left[\frac{n}{2}\right]=\left[\frac{n+1}{2}\right]\right)
\end{aligned}
$$

$\leqslant n \sum_{i=1}^{\left[\frac{n}{2}\right]} a_{i}^{2}+(n+1) \sum_{j=\left[\frac{n}{2}\right]+1}^{n} a_{j}^{2}$

$\leqslant(n+1) \sum_{i=1}^{n} a_{i}^{2}$.

从而, 命题成立.

证法二 由对称性, 不妨设 $a_{1} \geqslant a_{2} \geqslant \cdots \geqslant a_{n}$. 此外, 若将 $a_{1}, \cdots, a_{n}$ 中负数均改变符号, 由于 $\left(\sum_{i=1}^{n} a_{i}\right)^{2}$ 不减, $\sum_{i=1}^{n} a_{i}^{2}$ 不变, 且不影响 $\varepsilon_{i}= \pm 1$ 的选取.所以,可进一步假设 $a_{1} \geqslant a_{2} \geqslant \cdots \geqslant a_{n} \geqslant 0$.

引理 $\quad$ 设 $a_{1} \geqslant a_{2} \geqslant \cdots \geqslant a_{n} \geqslant 0$, 则 $0 \leqslant \sum_{i=1}^{n}(-1)^{i-1} a_{i} \leqslant a_{1}$.

事实上, 由于 $a_{i} \geqslant a_{i+1}, 1 \leqslant i \leqslant n-1$. 则当 $n$ 为偶数时,
$$
\begin{aligned}
& \sum_{i=1}^{n}(-1)^{i-1} a_{i}=\left(a_{1}-a_{2}\right)+\cdots+\left(a_{n-1}-a_{n}\right) \geqslant 0 \\
& \sum_{i=1}^{n}(-1)^{i-1} a_{i}=a_{1}-\left(a_{2}-a_{3}\right)-\cdots-\left(a_{n-2}-a_{n-1}\right)-a_{n} \leqslant a_{1}
\end{aligned}
$$

当 $n$ 为奇数时
$$
\begin{aligned}
& \sum_{i=1}^{n}(-1)^{i-1} a_{i}=\left(a_{1}-a_{2}\right)+\cdots+\left(a_{n-2}-a_{n-1}\right)+a_{n} \geqslant 0 \\
& \sum_{i=1}^{n}(-1)^{i-1} a_{i}=a_{1}-\left(a_{2}-a_{3}\right)-\cdots-\left(a_{n-1}-a_{n}\right) \leqslant a_{1}
\end{aligned}
$$

所以,引理成立.

由柯西不等式和引理得
$$
\left(\sum_{i=1}^{n} a_{i}\right)^{2}+\left(\sum_{i=1}^{n}(-1)^{i-1} a_{i}\right)^{2} \leqslant n \sum_{i=1}^{n} a_{i}^{2}+a_{1}^{2} \leqslant(n+1) \sum_{i=1}^{n} a_{i}^{2}
$$

故命题成立.

利用变形的赫尔德不等式可以证明下列不等式.

例 24 证明: 对正实数 $a 、 b 、 c$, 有
$$
\frac{a}{\sqrt{a^{2}+8 b c}}+\frac{b}{\sqrt{b^{2}+8 a c}}+\frac{c}{\sqrt{c^{2}+8 a b}} \geqslant 1
$$

证明 由变形的柯西不等式,
$$
\text { 左边 }=\sum_{\text {cyc }} \frac{a}{\sqrt{a^{2}+8 b c}}=\sum_{\text {cyc }} \frac{a^{\frac{3}{2}}}{\sqrt{a^{3}+8 a b c}} \geqslant \frac{\left(\sum_{\mathrm{cyc}} a\right)^{\frac{3}{2}}}{\left[\sum_{\mathrm{cyc}}\left(a^{3}+8 a b c\right)\right]^{\frac{1}{2}}} \text {. }
$$

所以要证明原不等式, 只需要证明
$$
\frac{\left(\sum_{\mathrm{cyc}} a\right)^{\frac{3}{2}}}{\left[\sum_{\text {cyc }}\left(a^{3}+8 a b c\right)\right]^{\frac{1}{2}}} \geqslant 1
$$

等价于
$$
\left(\sum_{\text {cyc }} a\right)^{3} \geqslant \sum_{\text {cyc }} a^{3}+24 a b c
$$

等价于
$$
\sum_{\text {cyc }} a^{3}+3 \sum_{\text {cyc }}\left(a^{2} b+a b^{2}\right)+6 a b c \geqslant \sum_{\text {cyc }} a^{3}+24 a b c
$$

等价于
$$
\sum_{c y c}\left(a^{2} b+a b^{2}\right) \geqslant 6 a b c
$$

易知该不等式成立, 故原不等式成立.

注 前面, 我们用平均值不等式证明了这个不等式, 读者还可以用其他方法证明.

例 25 已知 $x 、 y 、 z$ 是正实数, 且 $x y z=x+y+z+2$. 求证:
$$
2(\sqrt{x y}+\sqrt{y z}+\sqrt{z x}) \leqslant x+y+z+6 .
$$

证明 由于 $x y z=x+y+z+2, z=\frac{x+y+2}{x y-1}, x y>1$.

设 $\frac{x+1}{y+1}=\frac{b}{a}, a, b \in \mathbf{R}_{+}$, 则 $(x+1) a=(y+1) b$.

令 $(x+1) a=(y+1) b=S$, 则 $S>a, S>b$,
$$
x=\frac{S-a}{a}, y=\frac{S-b}{b}
$$

由 $x y>1$, 得 $(S-a)(S-b)>a b$,
$$
\begin{aligned}
& S^{2}>(a+b) S \\
& S>a+b
\end{aligned}
$$

设 $c=S-(a+b), c>0$, 于是
$$
\begin{aligned}
& x=\frac{b+c}{a}, y=\frac{c+a}{b} \\
& z=\frac{\frac{b+c}{a}+\frac{c+a}{b}+2}{\frac{(b+c)(c+a)}{a b}-1}=\frac{b^{2}+b c+a^{2}+a c+2 a b}{c^{2}+a c+b c}=\frac{a+b}{c}
\end{aligned}
$$

因为 $(\sqrt{x}+\sqrt{y}+\sqrt{z})^{2}=x+y+z+2(\sqrt{x y}+\sqrt{y z}+\sqrt{z x})$,所以 $2(\sqrt{x y}+\sqrt{y z}+\sqrt{z x})=(\sqrt{x}+\sqrt{y}+\sqrt{z})^{2}-(x+y+z)$.原不等式等价于
$$
\begin{aligned}
& (\sqrt{x}+\sqrt{y}+\sqrt{z})^{2} \leqslant 2(x+y+z+3) \\
\Leftrightarrow & \sqrt{x}+\sqrt{y}+\sqrt{z} \leqslant \sqrt{2(x+y+z+3)} \\
\Leftrightarrow & \sqrt{\frac{b+c}{a}}+\sqrt{\frac{c+a}{b}}+\sqrt{\frac{a+b}{c}} \leqslant \sqrt{2\left(\frac{b+c}{a}+\frac{c+a}{b}+\frac{a+b}{c}+3\right)}
\end{aligned}
$$
$$
\text { 而 } 2\left(\frac{b+c}{a}+\frac{c+a}{b}+\frac{a+b}{c}+3\right)=2(a+b+c)\left(\frac{1}{a}+\frac{1}{b}+\frac{1}{c}\right) \text { , }
$$

因此只需证
$$
\sqrt{\frac{b+c}{a}}+\sqrt{\frac{c+a}{b}}+\sqrt{\frac{a+b}{c}} \leqslant \sqrt{2(a+b+c)\left(\frac{1}{a}+\frac{1}{b}+\frac{1}{c}\right)}
$$

由于
$$
\begin{aligned}
& 2(a+b+c)\left(\frac{1}{a}+\frac{1}{b}+\frac{1}{c}\right) \\
= & {[(a+b)+(b+c)+(c+a)] \cdot\left(\frac{1}{c}+\frac{1}{a}+\frac{1}{b}\right) } \\
\geqslant & \left(\sqrt{\frac{a+b}{c}}+\sqrt{\frac{b+c}{a}}+\sqrt{\frac{c+a}{b}}\right)^{2}
\end{aligned}
$$

故原不等式成立.

注 本题的条件式和结论式都非齐次, 用 $x=\frac{b+c}{a}, y=\frac{c+a}{b}, z=$ $\frac{a+b}{c}$ 将条件与结论都变成了齐次式.

例 $26 a 、 b 、 c$ 是非负实数, 但不全为 0 ,求证:
$$
\frac{a}{a+b+7 c}+\frac{b}{b+c+7 a}+\frac{c}{c+a+7 b}+\frac{2}{3} \cdot \frac{a b+b c+c a}{a^{2}+b^{2}+c^{2}} \leqslant 1
$$

证明 由于 $\frac{a}{a+b+c}-\frac{a}{a+b+7 c}=\frac{6 c a}{(a+b+c)(a+b+7 c)}$, 故
$$
\frac{a}{a+b+7 c}=\frac{a}{a+b+c}-\frac{6 c a}{(a+b+c)(a+b+7 c)}
$$

同理 $\frac{b}{b+c+7 a}=\frac{b}{a+b c}-\frac{6 a b}{(b+c+a)(b+c+7 a)}$,
$$
\frac{c}{c+a+7 b}=\frac{c}{c+a+b}-\frac{6 b c}{(c+a+b)(c+a+7 b)}
$$

三式相加, 得
$$
\begin{aligned}
& \frac{a}{a+b+7 c}+\frac{b}{b+c+7 a}+\frac{c}{c+a+7 b} \\
= & 1-\frac{6}{a+b+c}\left(\frac{c a}{a+b+7 c}+\frac{a b}{b+c+7 a}+\frac{b c}{c+a+7 b}\right)
\end{aligned}
$$

因此, 要证原不等式, 只需证明
$$
\begin{aligned}
& \frac{2}{3} \cdot \frac{a b+b c+c a}{a^{2}+b^{2}+c^{2}} \leqslant \frac{6}{a+b+c}\left(\frac{c a}{a+b+7 c}+\frac{a b}{b+c+7 a}+\frac{b c}{c+a+7 b}\right) \\
& \Leftrightarrow \frac{c a}{a+b+7 c}+\frac{a b}{b+c+7 a}+\frac{b c}{c+a+7 b} \geqslant \frac{(a+b+c)(a b+b c+c a)}{9\left(a^{2}+b^{2}+c^{2}\right)}
\end{aligned}
$$

由柯西不等式
$$
\begin{aligned}
& {[c a(a+b+7 c)+a b(b+c+7 a)+b c(c+a+7 b)]} \\
& \quad\left[\frac{c a}{a+b+7 c}+\frac{a b}{b+c+7 a}+\frac{b c}{c+a+7 b}\right] \\
& \geqslant(c a+a b+b c)^{2}
\end{aligned}
$$

故
$$
\begin{aligned}
& \frac{c a}{a+b+7 c}+\frac{a b}{b+c+7 a}+\frac{b c}{c+a+7 b} \\
\geqslant & \frac{(c a+a b+b c)^{2}}{a b^{2}+b c^{2}+c a^{2}+7\left(c^{2} a+a^{2} b+b^{2} c\right)+3 a b c}
\end{aligned}
$$

因此, 只需证明
$$
\begin{aligned}
& \quad \frac{(a b+b c+c a)^{2}}{a b^{2}+b c^{2}+c a^{2}+7\left(a^{2} b+b^{2} c+c^{2} a\right)+3 a b c} \geqslant \frac{(a+b+c)(a b+b c+c a)}{9\left(a^{2}+b^{2}+c^{2}\right)} \\
& \Leftrightarrow 9\left(a^{2}+b^{2}+c^{2}\right)(a b+b c+c a) \geqslant(a+b+c)\left[a b^{2}+b c^{2}+c a^{2}+7\left(a^{2} b+b^{2} c\right.\right. \\
& \left.\left.\quad+c^{2} a\right)+3 a b c\right] \\
& \Leftrightarrow 9\left[a^{3} b+a^{3} c+b^{3} a+b^{3} c+c^{3} a+c^{3} b+a b c(a+b+c)\right] \\
& \quad \geqslant\left[a^{3} c+b^{3} a+c^{3} b+a^{2} b^{2}+b^{2} c^{2}+c^{2} a^{2}+a b c(a+b+c)\right]+7\left[a^{3} b+b^{3} c\right. \\
& \left.\quad+c^{3} a+a^{2} c^{2}+b^{2} c^{2}+a^{2} b^{2}+a b c(a+b+c)\right]+3 a b c(a+b+c) \\
& \Leftrightarrow 2\left(a^{3} b+b^{3} c+c^{3} a\right)+8\left(a^{3} c+b^{3} a+c^{3} b\right) \geqslant 8\left(a^{2} c^{2}+b^{2} c^{2}+a^{2} b^{2}\right)+2 a b c(a
\end{aligned}
$$

$+b+c)$

$\Leftrightarrow a^{3} b+b^{3} c+c^{3} a+4\left(a^{3} c+b^{3} a+c^{3} b\right) \geqslant 4\left(a^{2} c^{2}+b^{2} c^{2}+a^{2} b^{2}\right)+a b c(a+b$ $+c)$

$\Leftrightarrow\left(a^{3} b+4 b^{3} a-4 a^{2} b^{2}\right)+\left(b^{3} c+4 c^{3} b-4 b^{2} c^{2}\right)+\left(c^{3} a+4 a^{3} c-4 a^{2} c^{2}\right) \geqslant a b c(a$

$+b+c)$

$\Leftrightarrow a b(a-2 b)^{2}+b c(b-2 c)^{2}+c a(c-2 a)^{2} \geqslant a b c(a+b+c)$

由于 $\left[a b(a-2 b)^{2}+b c(b-2 c)^{2}+c a(c-2 a)^{2}\right](c+a+b)$

$\geqslant[\sqrt{a b c}(a-2 b)+\sqrt{a b c}(b-2 c)+\sqrt{a b c}(c-2 a)]^{2}=a b c(a+b+c)^{2}$故原不等式成立.

例 27 对一切正实数 $a 、 b 、 c$. 求证:
$$
\frac{a^{2}}{b}+\frac{b^{2}}{c}+\frac{c^{2}}{a}+a+b+c \geqslant \frac{6\left(a^{2}+b^{2}+c^{2}\right)}{a+b+c}
$$

证明 不妨设 $a \geqslant b \geqslant c$.

原不等式等价于
$$
\begin{aligned}
& \left(\frac{a^{2}}{b}+b-2 a\right)+\left(\frac{b^{2}}{c}+c-2 b\right)+\left(\frac{c^{2}}{a}+a-2 c\right) \geqslant \frac{6\left(a^{2}+b^{2}+c^{2}\right)}{a+b+c}-2(a+b+c) \\
\Leftrightarrow & \frac{(a-b)^{2}}{b}+\frac{(b-c)^{2}}{c}+\frac{(c-a)^{2}}{a} \geqslant 2 \cdot \frac{(a-b)^{2}+(b-c)^{2}+(c-a)^{2}}{a+b+c} \\
\Leftrightarrow & (a-b)^{2}\left(\frac{1}{b}-\frac{2}{a+b+c}\right)+(b-c)^{2}\left(\frac{1}{c}-\frac{2}{a+b+c}\right)+(c-a)^{2}\left(\frac{1}{a}-\frac{2}{a+b+c}\right) \geqslant 0
\end{aligned}
$$

设 $S_{a}=\frac{1}{a}-\frac{2}{a+b+c}, S_{b}=\frac{1}{b}-\frac{2}{a+b+c}, S_{c}=\frac{1}{c}-\frac{2}{a+b+c}$.

又设 $x=a-b, y=b-c$, 则原不等式等价于

$x^{2} S_{b}+y^{2} S_{c}+(x+y)^{2} S_{a}=\left(S_{a}+S_{b}\right) x^{2}+\left(S_{a}+S_{c}\right) y^{2}+2 x y S_{a}$.

显然 $S_{b}>0, S_{c}>0$, 由于 $\frac{1}{a}+\frac{1}{b}+\frac{1}{c} \geqslant \frac{9}{a+b+c}$, 故 $S_{a}+S_{b}+S_{c}>0$.
$$
\begin{aligned}
& S_{a}+S_{c}=\frac{1}{a}+\frac{1}{c}-\frac{4}{a+b+c} \geqslant \frac{4}{a+c}-\frac{4}{a+b+c}>0 \\
& S_{a}+S_{b}=\frac{1}{a}+\frac{1}{b}-\frac{4}{a+b+c} \geqslant \frac{4}{a+b}-\frac{4}{a+b+c}>0, \\
& \Delta=\left(2 S_{a}\right)^{2}-4\left(S_{a}+S_{b}\right)\left(S_{a}+S_{c}\right)=-4\left(S_{a} S_{b}+S_{b} S_{c}+S_{c} S_{a}\right)
\end{aligned}
$$

下证: $S_{a} S_{b}+S_{b} S_{c}+S_{a} S_{c} \geqslant 0$.

即证: $\left(\frac{1}{a}-\frac{2}{a+b+c}\right)\left(\frac{1}{b}-\frac{2}{a+b+c}\right)+\left(\frac{1}{b}-\frac{2}{a+b+c}\right)\left(\frac{1}{c}-\right.$
$$
\begin{aligned}
& \left.\frac{2}{a+b+c}\right)+\left(\frac{1}{c}-\frac{2}{a+b+c}\right)\left(\frac{1}{a}-\frac{2}{a+b+c}\right) \geqslant 0 \\
& \quad \text { 左边 }=\frac{1}{a b c(a+b+c)^{2}}\left[a^{3}+b^{3}+c^{3}-\left(a^{2} b+a^{2} c+b^{2} a+b^{2} c+c^{2} a+c^{2} b\right)\right. \\
& +6 a b c] \geqslant 0
\end{aligned}
$$

例 28 设非负实数 $a_{1}, a_{2}, \cdots, a_{n}$ 满足 $a_{1}^{2}+a_{2}^{2}+\cdots+a_{n}^{2}=n$, 求证
$$
\sum_{i=1}^{n} \frac{1}{a_{i}^{2}+1} \leqslant \frac{n^{3}}{2\left(\sum_{i=1}^{n} a_{i}\right)^{2}}
$$

证明 由柯西不等式
$$
\begin{aligned}
& \frac{a_{1}^{2}}{a_{1}^{2}+a_{1}^{2}}+\frac{a_{2}^{2}}{a_{1}^{2}+a_{2}^{2}}+\frac{a_{3}^{2}}{a_{1}^{2}+a_{3}^{2}}+\cdots+\frac{a_{n}^{2}}{a_{1}^{2}+a_{n}^{2}} \\
\geqslant & \frac{\left(a_{1}+a_{2}+\cdots+a_{n}\right)^{2}}{n a_{1}^{2}+\left(a_{1}^{2}+a_{2}^{2}+\cdots+a_{n}^{2}\right)} \\
= & \frac{\left(\sum_{i=1}^{n} a_{i}\right)^{2}}{n a_{1}^{2}+n}
\end{aligned}
$$

将上式中的 $a_{1}$ 换成 $a_{k}$, 这样的 $n$ 个式子相加, 得
$$
\begin{aligned}
& \frac{n}{2}+\mathrm{C}_{n}^{2} \geqslant \sum_{i=1}^{n} \frac{\left(\sum_{i=1}^{n} a_{i}\right)^{2}}{n a_{i}^{2}+n} \\
& \frac{n^{2}}{2} \geqslant \sum_{i=1}^{n} \frac{\left(\sum_{i=1}^{n} a_{i}\right)^{2}}{n a_{i}^{2}+n}
\end{aligned}
$$

即 $\frac{n^{3}}{2\left(\sum_{i=1}^{n} a_{i}\right)^{2}} \geqslant \sum_{i=1}^{n} \frac{1}{a_{i}^{2}+1}$.

例 29 设 $x, y, z>0$ 且 $x^{2}+y^{2}+z^{2}=1$. 求证:
$$
x y z+\sqrt{x^{2} y^{2}+y^{2} z^{2}+z^{2} x^{2}} \geqslant \frac{4}{3} \sqrt{x y z(x+y+z)}
$$

证明 将原不等式两边除以 $x y z$,则欲证的不等式为
$$
1+\sqrt{\frac{1}{z^{2}}+\frac{1}{x^{2}}+\frac{1}{y^{2}}} \geqslant \frac{4}{3} \sqrt{\frac{1}{y z}+\frac{1}{z x}+\frac{1}{x y}}
$$

将 $x 、 y 、 z$ 分别换成 $\frac{1}{a} 、 \frac{1}{b} 、 \frac{1}{c}$, 则 $\frac{1}{a^{2}}+\frac{1}{b^{2}}+\frac{1}{c^{2}}=1$.

原不等式即为 $1+\sqrt{a^{2}+b^{2}+c^{2}} \geqslant \frac{4}{3} \sqrt{a b+b c+c a}$
$$
\begin{aligned}
& \Leftrightarrow 1+\sqrt{\left(a^{2}+b^{2}+c^{2}\right)\left(a^{-2}+b^{-2}+c^{-2}\right)} \\
& \quad \geqslant \frac{4}{3} \sqrt{(a b+b c+c a)\left(a^{-2}+b^{-2}+c^{-2}\right)}
\end{aligned}
$$

这时候, 由于上式是齐次式, 不妨 $a+b+c=1$, 则
$$
\begin{aligned}
& a^{2}+b^{2}+c^{2}=a^{2}+b^{2}+c^{2}-\frac{4}{3}(a+b+c)+\frac{4}{9} \cdot 3 \\
&=\left(\frac{2}{3}-a\right)^{2}+\left(\frac{2}{3}-b\right)^{2}+\left(\frac{2}{3}-c\right)^{2} . \\
& \sqrt{\left(a^{2}+b^{2}+c^{2}\right)\left(a^{-2}+b^{-2}+c^{-2}\right)} \\
&= \sqrt{\left[\left(\frac{2}{3}-a\right)^{2}+\left(\frac{2}{3}-b\right)^{2}+\left(\frac{2}{3}-c\right)^{2}\right]\left(a^{-2}+b^{-2}+c^{-2}\right)} \\
& \geqslant \sqrt{\left[\left(\frac{2}{3}-a\right) \cdot \frac{1}{a}+\left(\frac{2}{3}-b\right) \cdot \frac{1}{b}+\left(\frac{2}{3}-c\right) \cdot \frac{1}{c}\right]^{2}} \\
&=\left|\left(\frac{2}{3}-a\right) \cdot \frac{1}{a}+\left(\frac{2}{3}-b\right) \cdot \frac{1}{b}+\left(\frac{2}{3}-c\right) \cdot \frac{1}{c}\right| \\
&=\left|\frac{2}{3}\left(\frac{1}{a}+\frac{1}{b}+\frac{1}{c}\right)-3\right| \\
& \geqslant \frac{2}{3}\left(\frac{1}{a}+\frac{1}{b}+\frac{1}{c}\right)-3 \\
&= \frac{2}{3}\left(\frac{1}{a}+\frac{1}{b}+\frac{1}{c}\right)(a+b+c)-3 \\
&= \frac{2}{3}\left(\frac{b+c}{a}+\frac{c+a}{b}+\frac{a+b}{c}\right)-1 .
\end{aligned}
$$

因此, 只需证明
$$
\begin{aligned}
& \frac{b+c}{a}+\frac{c+a}{b}+\frac{a+b}{c} \geqslant 2 \sqrt{(a b+b c+c a)\left(a^{-2}+b^{-2}+c^{-2}\right)} \\
& \Leftrightarrow\left(\frac{b+c}{a}\right)^{2}+\left(\frac{c+a}{b}\right)^{2}+\left(\frac{a+b}{c}\right)^{2} \\
& \quad+2\left(\frac{c^{2}+a b+b c+c a}{a b}+\frac{a^{2}+a b+a c+b c}{b c}+\frac{b^{2}+a b+a c+b c}{c a}\right)
\end{aligned}
$$
$$
\begin{aligned}
& \geqslant 4\left(\frac{b+c}{a}+\frac{c+a}{b}+\frac{a+b}{c}+\frac{b c}{a^{2}}+\frac{c a}{b^{2}}+\frac{a b}{c^{2}}\right) \\
& \Leftrightarrow \frac{c^{2}+a b+b c+c a}{a b}+\frac{a^{2}+a b+b c+c a}{b c}+\frac{b^{2}+b c+a b+c a}{c a} \\
& \geqslant\left(\frac{b+c}{a}+\frac{c+a}{b}\right)+\left(\frac{c+a}{b}+\frac{a+b}{c}\right)+\left(\frac{a+b}{c}+\frac{b+c}{a}\right) \\
& \Leftrightarrow \frac{c^{2}+a b}{a b}+\frac{a^{2}+b c}{b c}+\frac{b^{2}+c a}{c a} \geqslant \frac{a^{2}+b^{2}}{a b}+\frac{b^{2}+c^{2}}{b c}+\frac{a^{2}+c^{2}}{c a} \\
& \Leftrightarrow c^{3}+a^{3}+b^{3}+3 a b c \geqslant a^{2} c+a^{2} b+b^{2} c+b^{2} a+c^{2} a+c^{2} b \\
& \Leftrightarrow a(a-b)(a-c)+b(b-a)(b-c)+c(c-a)(c-b) \geqslant 0
\end{aligned}
$$

也可以设 $a \geqslant b \geqslant c$, 则
$$
\begin{aligned}
& a(a-b)(a-c)+b(b-a)(b-c) \\
= & (a-b)[a(a-c)-b(b-c)] \\
= & (a-b)^{2}(a+b-c) \geqslant 0
\end{aligned}
$$

而 $c(c-a)(c-b) \geqslant 0$, 故原不等式成立.

例 30 设 $x_{1}, x_{2}, \cdots, x_{2019}$ 为实数, 且 $x_{12}=1$. 求 $\sum_{i, j=1}^{2019} \min \{i, j\} x_{i} x_{j}$ 的最小值.

解 记 $S_{n}=\sum_{i, j=1}^{n} \min \left\{x_{i}, x_{j}\right\} x_{i} x_{j}$. 则
$$
\begin{aligned}
& S_{1}=x_{1}^{2} \text {, } \\
& \begin{aligned}
S_{2} & =\sum_{i, j=1}^{2} \min \left\{x_{i}, x_{j}\right\} x_{i} x_{j} \\
& =x_{1}^{2}+x_{1} x_{2}+x_{2} x_{1}+2 x_{2}^{2}=\left(x_{1}+x_{2}\right)^{2}+x_{2}^{2}, \\
S_{3} & =\sum_{i, j=1}^{3} \min \left\{x_{i}, x_{j}\right\} x_{i} x_{j} \\
& =\left(x_{1}^{2}+x_{1} x_{2}+x_{1} x_{3}\right)+\left(x_{2} x_{1}+2 x_{2}^{2}+2 x_{2} x_{3}\right)+\left(x_{3} x_{1}+2 x_{3} x_{2}+3 x_{3}^{2}\right) \\
& =\left(x_{1}+x_{2}+x_{3}\right)^{2}+\left(x_{2}+x_{3}\right)^{2}+x_{3}^{2}, \\
& \quad \text { 假设 } S_{n}=\sum_{i=1}^{n}\left(x_{i}+x_{i+1}+\cdots+x_{n}\right)^{2} \text {. 则 } \\
& \quad S_{n+1}=\sum_{i, j=1}^{n+1} \min \{i, j\} x_{i} x_{j} \\
& =\sum_{i, j=1}^{n} \min \{i, j\} x_{i} x_{j}+\sum_{j=1}^{n+1} j x_{n+1} x_{j}+\sum_{i=1}^{n} i x_{i} x_{n+1}
\end{aligned}
\end{aligned}
$$
$$
\begin{aligned}
= & \sum_{i=1}^{n}\left(x_{i}+x_{i+1}+\cdots+x_{n}\right)^{2}+2 x_{n+1}\left(x_{1}+2 x_{2}+\cdots+n x_{n}\right)+(n \\
& +1) x_{n+1}^{2} \\
= & \sum_{i=1}^{n}\left(x_{i}+x_{i+1}+\cdots+x_{n}\right)^{2}+2 \cdot \sum_{i=1}^{n}\left(x_{i}+x_{i+1}+\cdots+x_{n}\right) \cdot x_{n+1}+ \\
& n x_{n+1}^{2}+x_{n+1}^{2} \\
= & \sum_{i=1}^{n}\left[\left(x_{i}+x_{i+1}+\cdots+x_{n}\right)^{2}+2 \cdot \sum_{i=1}^{n}\left(x_{i}+x_{i+1}+\cdots+x_{n}\right) \cdot x_{n+1}\right. \\
& \left.+x_{n+1}^{2}\right]+x_{n+1}^{2} \\
= & \sum_{i=1}^{n}\left(x_{i}+x_{i+1}+\cdots+x_{n+1}\right)^{2}+x_{n+1}^{2} \\
= & \sum_{i=1}^{n}\left(x_{i}+x_{i+1}+\cdots+x_{n+1}\right)^{2} .
\end{aligned}
$$

由归纳原理知 $S_{n}=\sum_{i=1}^{n}\left(x_{i}+x_{i+1}+\cdots+x_{n}\right)^{2}$.

设 $T_{n, i}=x_{i}+x_{i+1}+\cdots+x_{n}$, 则 $S_{n}=\sum_{i=1}^{n} T_{n, i}^{2}$,
$$
\begin{aligned}
& T_{n, 12}=1+x_{13}+x_{14}+\cdots+x_{n}, \\
& T_{11,13}=x_{13}+x_{14}+\cdots+x_{n}, \\
S_{n} \geqslant & T_{n, 12}^{2}+T_{n, 13}^{2}=\left(1+T_{n, 13}\right)^{2}+T_{n, 13}^{2} \\
\geqslant & \frac{1}{2}\left[\left(1+T_{n, 13}\right)-T_{n, 13}\right]^{2}=\frac{1}{2} .
\end{aligned}
$$

例如取 $x_{11}=-\frac{1}{2}, x_{13}=-\frac{1}{2}, x_{i}=0(i \neq 11,12,13)$ 时, $S_{n}=\frac{1}{2}$.故 $S_{2019}$ 的最小值为 $\frac{1}{2}$.

\section*{4. 2 柯西不等式在解方程组和求极值中的应用}
应用柯西不等式中等号成立的条件, 通过不等式夹逼, 求出方程组中各个未知数的值, 从而进一步求出有关代数式的值.

极值问题往往是关于对称式的问题. 先根据条件, 在各个未知元相等时的值得出极值, 然后证明相应的不等式.\\
例 1 求方程组

\[
\left\{\begin{array}{l}
a^{2}=\frac{\sqrt{b c} \sqrt[3]{b c d}}{(b+c)(b+c+d)}  \tag{1}\\
b^{2}=\frac{\sqrt{c d} \sqrt[3]{c d a}}{(c+d)(c+d+a)} \\
c^{2}=\frac{\sqrt{d a} \sqrt[3]{d a b}}{(d+a)(d+a+b)} \\
d^{2}=\frac{\sqrt{a b} \sqrt[3]{a b c}}{(a+b)(a+b+c)}
\end{array}\right.
\]

的实数解.

解 首先, 注意到没有一个变量等于零. 不失一般性, 假设 $b=0$, 由 (1)得 $a=0$, 由(4)得 $d=0$, 由(3)得 $c=0$, 这就意味着所有值为零, 但这是不可能的, 因为分母会为零.

其次, 注意到 $b c 、 c d 、 d a 、 a b$ 的平方根一定都存在, 这就表明 $a 、 b 、 c 、 d$一定都是负数或都是正数. 如果都是负数, 这些方程的右边是负的, 与它们是实数的平方相矛盾, 由此可得 4 个值一定都是正的.

根据算术一几何平均值不等式, 有
$$
\sqrt{b c} \leqslant \frac{b+c}{2}, \text { 即 } \frac{\sqrt{b c}}{b+c} \leqslant \frac{1}{2}
$$

和 $\sqrt[3]{b c d} \leqslant \frac{b+c+d}{3}$, 即
$$
\frac{\sqrt[3]{b c d}}{b+c+d} \leqslant \frac{1}{3}
$$

因此 $\quad a^{2}=\frac{\sqrt{b c} \sqrt[3]{b c d}}{(b+c)(b+c+d)} \leqslant \frac{1}{2} \times \frac{1}{3}=\frac{1}{6}$.

从而 $a \leqslant \frac{1}{\sqrt{6}}$.

类似地, 有 $b \leqslant \frac{1}{\sqrt{6}}, c \leqslant \frac{1}{\sqrt{6}}, d \leqslant \frac{1}{\sqrt{6}}$.

由此得 $(b+c)(b+c+d) \leqslant \frac{2}{\sqrt{6}} \times \frac{3}{\sqrt{6}}=1$.

同样地, 有
$$
\begin{aligned}
& (c+d)(c+d+a) \leqslant 1 \\
& (d+a)(d+a+b) \leqslant 1 \\
& (a+b)(a+b+c) \leqslant 1
\end{aligned}
$$

由(1) $\times(2) \times(3) \times(4)$, 可得
$$
1=(b+c)(b+c+d)(c+d)(c+d+a)(d+a)(d+a+b)(a+b)(a+b+c)
$$

因为 4 个小于或等于 1 的表达式的积等于 1 , 那么, 这 4 个表达式一定都等于 1 .

从而唯一的可能是每个变量取它的最大的可能值.

因此 $a=b=c=d=\frac{\sqrt{6}}{6}$ 为给定方程组的唯一解.

例 2 已知实数 $x, y, z>3$, 求方程
$$
\frac{(x+2)^{2}}{y+z-2}+\frac{(y+4)^{2}}{z+x-4}+\frac{(z+6)^{2}}{x+y-6}=36
$$

的所有实数解 $(x, y, z)$.

解 由 $x, y, z>3$, 知
$$
y+z-2>0, z+x-4>0, x+y-6>0
$$

由柯西一施瓦兹不等式得
$$
\begin{aligned}
& {\left[\frac{(x+2)^{2}}{y+z-2}+\frac{(y+4)^{2}}{x+z-4}+\frac{(z+6)^{2}}{x+y-6}\right] } \\
& {[(y+z-2)+(x+z-4)+(x+y-6)] } \\
\geqslant & (x+y+z+12)^{2} \\
\Leftrightarrow & \frac{(x+2)^{2}}{y+z-2}+\frac{(y+4)^{2}}{x+z-4}+\frac{(z+6)^{2}}{x+y-6} \\
\geqslant & \frac{1}{2} \cdot \frac{(x+y+z+12)^{2}}{x+y+z-6}
\end{aligned}
$$

结合题设等式得


\begin{equation*}
\frac{(x+y+z+12)^{2}}{x+y+z-6} \leqslant 72 \tag{1}
\end{equation*}


当 $\frac{x+2}{y+z-2}=\frac{y+4}{x+z-4}=\frac{z+6}{x+y-6}=\lambda$, 即

\[
\left\{\begin{array}{l}
\lambda(y+z)-x=2(\lambda+1)  \tag{2}\\
\lambda(x+z)-y=4(\lambda+1) \\
\lambda(x+y)-z=6(\lambda+1)
\end{array}\right.
\]

时, 式(1)等号成立.

设 $w=x+y+z+12$. 则
$$
\frac{(x+y+z+12)^{2}}{x+y+z-6}=\frac{w^{2}}{w-18}
$$

又
$$
\begin{aligned}
& \frac{w^{2}}{w-18} \geqslant 4 \times 18=72 \\
\Leftrightarrow & w^{2}-4 \times 18 w+4 \times 18^{2} \geqslant 0 \\
\Leftrightarrow & (w-36)^{2} \geqslant 0
\end{aligned}
$$

则


\begin{equation*}
\frac{(x+y+z+12)^{2}}{x+y+z-6} \geqslant 72 \tag{3}
\end{equation*}


当


\begin{align*}
& w=x+y+z+12=36 \\
\Leftrightarrow & x+y+z=24 \tag{4}
\end{align*}


时, 式(3)等号成立.

由式(1), (3)得
$$
\frac{(x+y+z+12)^{2}}{x+y+z-6}=72
$$

由方程组(2)与式(4)得
$$
\left\{\begin{array}{l}
(2 \lambda-1)(x+y+z)=12(\lambda+1), \\
x+y+z=24
\end{array} \Rightarrow \lambda=1\right.
$$

将 $\lambda=1$ 代人方程组(2)得
$$
\left\{\begin{array}{l}
y+z-x=4 \\
x+z-y=8, \Rightarrow(x, y, z)=(10,8,6) \\
x+y-z=12
\end{array}\right.
$$

所以,所求唯一实数解为 $(x, y, z)=(10,8,6)$.

例 3 设 $n$ 是一个正整数, $a_{1}, a_{2}, \cdots, a_{n}, b_{1}, b_{2}, \cdots, b_{n}$ 是 $2 n$ 个正实数, 满足 $a_{1}+a_{2}+\cdots+a_{n}=1, b_{1}+b_{2}+\cdots+b_{n}=1$, 求 $\frac{a_{1}^{2}}{a_{1}+b_{1}}+\frac{a_{2}^{2}}{a_{2}+b_{2}}+\cdots$\\
$+\frac{a_{n}^{2}}{a_{n}+b_{n}}$ 的最小值.

解 由柯西不等式知
$$
\begin{aligned}
& \quad\left(a_{1}+a_{2}+\cdots+a_{n}+b_{1}+b_{2}+\cdots+b_{n}\right)\left(\frac{a_{1}^{2}}{a_{1}+b_{1}}+\frac{a_{2}^{2}}{a_{2}+b_{2}}+\cdots+\frac{a_{n}^{2}}{a_{n}+b_{n}}\right) \\
& \geqslant\left(a_{1}+a_{2}+\cdots+a_{n}\right)^{2}=1, \\
& \text { 且 } \quad a_{1}+a_{2}+\cdots+a_{n}+b_{1}+b_{2}+\cdots+b_{n}=2, \\
& \text { 所以 } \quad \frac{a_{1}^{2}}{a_{1}+b_{1}}+\frac{a_{2}^{2}}{a_{2}+b_{2}}+\cdots+\frac{a_{n}^{2}}{a_{n}+b_{n}} \geqslant \frac{1}{2},
\end{aligned}
$$

且当 $a_{1}=a_{2}=\cdots=a_{n}=b_{1}=b_{2}=\cdots=b_{n}=\frac{1}{n}$ 时取到.

所以 $\frac{a_{1}^{2}}{a_{1}+b_{1}}+\frac{a_{2}^{2}}{a_{2}+b_{2}}+\cdots+\frac{a_{n}^{2}}{a_{n}+b_{n}}$ 的最小值为 $\frac{1}{2}$.

例 4 已知 $x 、 y 、 z$ 为实数, 且满足
$$
x+y+z=x y+y z+z x
$$

求 $\frac{x}{x^{2}+1}+\frac{y}{y^{2}+1}+\frac{z}{z^{2}+1}$ 的最小值.

解 令 $x=1, y=z=-1$. 则
$$
\frac{x}{x^{2}+1}+\frac{y}{y^{2}+1}+\frac{z}{z^{2}+1}=-\frac{1}{2}
$$

猜想最小值为 $-\frac{1}{2}$.

只须证:


\begin{align*}
& \frac{x}{x^{2}+1}+\frac{y}{y^{2}+1}+\frac{z}{z^{2}+1} \geqslant-\frac{1}{2} \\
\Leftrightarrow & \frac{(x+1)^{2}}{x^{2}+1}+\frac{(y+1)^{2}}{y^{2}+1} \geqslant \frac{(z-1)^{2}}{z^{2}+1} \tag{1}
\end{align*}


注意到 $z(x+y-1)=x+y-x y$.

若 $x+y-1=0$, 则 $x+y=x y=1$. 矛盾.

故 $x+y-1 \neq 0$. 于是, $z=\frac{x+y-x y}{x+y-1}$.

代人不等式(1)得


\begin{equation*}
\frac{(x+1)^{2}}{x^{2}+1}+\frac{(y+1)^{2}}{y^{2}+1} \geqslant \frac{(x y-1)^{2}}{(x+y-1)^{2}+(x+y-x y)^{2}} \tag{2}
\end{equation*}


由柯西不等式得
$$
\text { 式(2) 左 边 } \begin{aligned}
& \frac{[(1+x)(1-y)+(1+y)(1-x)]^{2}}{\left(1+x^{2}\right)(1-y)^{2}+\left(1+y^{2}\right)(1-x)^{2}} \\
& =\frac{4(x y-1)^{2}}{\left(1+x^{2}\right)(1-y)^{2}+\left(1+y^{2}\right)(1-x)^{2}} .
\end{aligned}
$$

于是, 只须证
$$
\begin{aligned}
& 4(x+y-1)^{2}+4(x+y-x y)^{2} \\
\geqslant & \left(1+x^{2}\right)(1-y)^{2}+\left(1+y^{2}\right)(1-x)^{2} \\
\Leftrightarrow & f(x)=\left(y^{2}-3 y+3\right) x^{2}-\left(3 y^{2}-8 y+3\right) x+3 y^{2}-3 y+1 \geqslant 0
\end{aligned}
$$

由 $\Delta=\left(3 y^{2}-8 y+3\right)^{2}-4\left(y^{2}-3 y+3\right)\left(3 y^{2}-3 y+1\right)=$ $-3\left(y^{2}-1\right)^{2} \leqslant 0$, 故 $f(x) \geqslant 0$ 恒成立.

从而, 猜想成立, 即 $\frac{x}{x^{2}+1}+\frac{y}{y^{2}+1}+\frac{z}{z^{2}+1}$ 的最小值为 $-\frac{1}{2}$.

例 5 设 $a 、 b 、 c 、 x 、 y 、 z$ 为实数,且
$$
a^{2}+b^{2}+c^{2}=25, x^{2}+y^{2}+z^{2}=36, a x+b y+c z=30
$$

求 $\frac{a+b+c}{x+y+z}$ 的值.

解 由柯西不等式, 得

$25 \times 36=\left(a^{2}+b^{2}+c^{2}\right)\left(x^{2}+y^{2}+z^{2}\right) \geqslant(a x+b y+c z)^{2}=30^{2}$.上述不等式等号成立, 得
$$
\frac{a}{x}=\frac{b}{y}=\frac{c}{z}=k
$$

于是 $k^{2}\left(x^{2}+y^{2}+z^{2}\right)=25$, 所以 $k= \pm \frac{5}{6}$ (负的舍去). 从而
$$
\frac{a+b+c}{x+y+z}=k=\frac{5}{6}
$$

例 6 设实数 $a 、 b 、 c 、 d 、 e$ 满足
$$
a+b+c+d+e=8, a^{2}+b^{2}+c^{2}+d^{2}+e^{2}=16
$$

求 $e$ 的最大值.

解 将条件改写为
$$
8-e=a+b+c+d, 16-e^{2}=a^{2}+b^{2}+c^{2}+d^{2}
$$

由此得到一个包含 $e$ 的不等式. 由柯西不等式, 得
$$
a+b+c+d \leqslant(1+1+1+1)^{\frac{1}{2}}\left(a^{2}+b^{2}+c^{2}+d^{2}\right)^{\frac{1}{2}}
$$

将条件代人并两边平方, 得
$$
\begin{gathered}
(8-e)^{2} \leqslant 4\left(16-e^{2}\right) \\
64-16 e+e^{2} \leqslant 64-4 e^{2} \\
5 e^{2}-16 e \leqslant 0, e(5-16 e) \leqslant 0
\end{gathered}
$$

从此得到 $0 \leqslant e \leqslant \frac{16}{5}$, 当 $a=b=c=d=\frac{6}{5}$ 时达到最大值 $\frac{16}{5}$.

注 用类似的方法可以证明下面的命题:

设 $n(\geqslant 3)$ 为正整数, $a 、 b$ 为给定的实数,实数 $x_{0}, x_{1}, x_{2}, \cdots, x_{n}$ 满足
$$
\begin{aligned}
& x_{0}+x_{1}+x_{2}+\cdots+x_{n}=a, \\
& x_{0}^{2}+x_{1}^{2}+x_{2}^{2}+\cdots+x_{n}^{2}=b,
\end{aligned}
$$

则当 $b<\frac{a^{2}}{n+1}$ 时, $x_{0}$ 不存在;
$$
\begin{aligned}
& \text { 当 } b=\frac{a^{2}}{n+1} \text { 时, } x_{0}=\frac{a}{n+1} \text {; } \\
& \text { 当 } b>\frac{a^{2}}{n+1} \text { 时, } x_{0} \text { 满足 } \\
& \frac{a-\frac{1}{2} \sqrt{\delta}}{n+1} \leqslant x_{0} \leqslant \frac{a+\frac{1}{2} \sqrt{\delta}}{n+1}
\end{aligned}
$$

其中 $\delta$ 为二次方程 $(n+1) x_{0}^{2}-2 a x_{0}+a^{2}-n b=0$ 的判别式.

例 7 设 $x \geqslant 0, y \geqslant 0, z \geqslant 0, a 、 b 、 c 、 l 、 m 、 n$ 是给定的正数, 并且 $a x+b y+c z=\delta$ 为常数,求
$$
w=\frac{l}{x}+\frac{m}{y}+\frac{n}{z}
$$

的最小值.

解 由柯西不等式, 得
$$
\begin{aligned}
w \cdot \delta & =\left[\left(\sqrt{\frac{l}{x}}\right)^{2}+\left(\sqrt{\frac{m}{y}}\right)^{2}+\left(\sqrt{\frac{n}{z}}\right)^{2}\right] \\
& \cdot\left[(\sqrt{a x})^{2}+(\sqrt{b y})^{2}+(\sqrt{c z})^{2}\right] \\
& \geqslant(\sqrt{a l}+\sqrt{b m}+\sqrt{c n})^{2} \\
& w \geqslant \frac{(\sqrt{a l}+\sqrt{b m}+\sqrt{c n})^{2}}{\delta}
\end{aligned}
$$

利用柯西等式成立的条件, 得 $x=k \sqrt{\frac{l}{a}}, y=k \sqrt{\frac{m}{b}}, z=k \sqrt{\frac{n}{c}}$, 其中 $k=\frac{\delta}{\sqrt{a l}+\sqrt{b m}+\sqrt{c n}}$, 它们使得 $a x+b y+c z=\delta$, 且
$$
w=\frac{(\sqrt{a l}+\sqrt{b m}+\sqrt{c n})^{2}}{\delta}
$$

所以
$$
w_{\min }=\frac{(\sqrt{a l}+\sqrt{b m}+\sqrt{c n})^{2}}{\delta}
$$

例 8 对满足 $a+b=1$ 的正实数 $a 、 b$, 求
$$
\left(a+\frac{1}{a}\right)^{2}+\left(b+\frac{1}{b}\right)^{2}
$$

的最小值.

解 当 $a=b=\frac{1}{2}$ 时, 我们有
$$
\left(a+\frac{1}{a}\right)^{2}+\left(b+\frac{1}{b}\right)^{2}=\frac{25}{2}
$$

下面证明
$$
\left(a+\frac{1}{a}\right)^{2}+\left(b+\frac{1}{b}\right)^{2} \geqslant \frac{25}{2}
$$

从而最小值为 $\frac{25}{2}$.
$$
\begin{aligned}
& \text { 令 } x=a+\frac{1}{a}, y=b+\frac{1}{b} \text {, 由 } \\
& \frac{x^{2}+y^{2}}{2} \geqslant\left(\frac{x+y}{2}\right)^{2}
\end{aligned}
$$

于是
$$
\begin{aligned}
\frac{1}{2}\left[\left(a+\frac{1}{a}\right)^{2}+\left(b+\frac{1}{b}\right)^{2}\right] & \geqslant\left\{\frac{1}{2}\left[\left(a+\frac{1}{a}\right)+\left(b+\frac{1}{b}\right)\right]\right\}^{2} \\
& =\left[\frac{1}{2}\left(1+\frac{1}{a}+\frac{1}{b}\right)\right]^{2}
\end{aligned}
$$

由柯西不等式, 得 $\left(\frac{1}{a}+\frac{1}{b}\right)(a+b) \geqslant(1+1)^{2}=4$. 则
$$
\left[\frac{1}{2}\left(1+\frac{1}{a}+\frac{1}{b}\right)\right]^{2} \geqslant\left[\frac{1}{2}\left(1+\frac{4}{a+b}\right)\right]^{2}=\left(\frac{1+4}{2}\right)^{2}=\frac{25}{4}
$$

从而命题成立.

例 9 设 $n$ 和 $k$ 是给定的正整数 $(k<n)$, 已知正实数 $a_{1}, a_{2}, \cdots, a_{k}$, 试求正实数 $a_{k+1}, a_{k+2}, \cdots, a_{n}$ 使得和式
$$
M=\sum_{i \neq j} \frac{a_{i}}{a_{j}}
$$

取最小值.

解 通过对 $n=1,2,3$ 计算, 得
$$
M=\left(a_{1}+a_{2}+\cdots+a_{n}\right)\left(\frac{1}{a_{1}}+\frac{1}{a_{2}}+\cdots+\frac{1}{a_{n}}\right)-n
$$

令 $a=a_{1}+a_{2}+\cdots+a_{k}, b=\frac{1}{a_{1}}+\frac{1}{a_{2}}+\cdots+\frac{1}{a_{k}}$, 则由假设 $a 、 b$ 为给定的常数. 因此由柯西不等式, 得
$$
\begin{aligned}
M & =\left(a+a_{k+1}+\cdots+a_{n}\right)\left(b+\frac{1}{a_{k+1}}+\cdots+\frac{1}{a_{n}}\right)-n \\
\geqslant & (\sqrt{a b}+1+\cdots+1)^{2}-n \\
& =(\sqrt{a b}+n-k)^{2}-n \\
& \quad \frac{\sqrt{a_{k+1}}}{\frac{1}{\sqrt{a_{k+1}}}}=\cdots=\frac{\sqrt{a_{n}}}{\frac{1}{\sqrt{a_{n}}}}=\frac{\sqrt{a}}{\sqrt{b}}
\end{aligned}
$$

且当

即 $a_{k+1}=\cdots=a_{n}=\sqrt{\frac{a}{b}}$ 时, $M$ 取最小值.

例 10 设 $2 n$ 个实数 $a_{1}, a_{2}, \cdots, a_{2 n}$ 满足 $\sum_{i=1}^{2 n-1}\left(a_{i+1}-a_{i}\right)^{2}=1$, 求
$$
\left(a_{n+1}+a_{n+2}+\cdots+a_{2 n}\right)-\left(a_{1}+a_{2}+\cdots+a_{n}\right)
$$

的最大值.

解 当 $n=1$ 时, $\left(a_{2}-a_{1}\right)^{2}=1$, 则 $a_{2}-a_{1}= \pm 1$, 最大值为 1 .

当 $n \geqslant 2$ 时, 设 $x_{1}=a_{1}, x_{i+1}=a_{i+1}-a_{i}, i=1,2, \cdots, 2 n-1$. 则 $\sum_{i=2}^{2 n} x_{i}^{2}=$ 1 , 且 $a_{k}=x_{1}+x_{2}+\cdots+x_{k}, k=1,2, \cdots, 2 n$.

由柯西不等式, 得
$$
\begin{aligned}
& a_{n+1}+a_{n+2}+\cdots+a_{2 n}-\left(a_{1}+a_{2}+\cdots+a_{n}\right) \\
= & x_{2}+2 x_{3}+\cdots+(n-1) x_{n}+n x_{n+1}+(n-1) x_{n+2}+\cdots+x_{2 n} \\
\leqslant & {\left[1+2^{2}+\cdots+(n-1)^{2}+n^{2}+(n-1)^{2}+\cdots+1\right]^{\frac{1}{2}} } \\
& \cdot\left(x_{2}^{2}+x_{3}^{2}+\cdots+x_{2 n}^{2}\right)^{\frac{1}{2}} \\
= & {\left[n^{2}+2 \cdot \frac{1}{6}(n-1) n(2(n-1)+1)\right]^{\frac{1}{2}}=\sqrt{\frac{n\left(2 n^{2}+1\right)}{3}} . }
\end{aligned}
$$

当 $a_{k}=\frac{\sqrt{3} k(k-1)}{2 \sqrt{n\left(2 n^{2}+1\right)}}, k=1,2, \cdots, n+1$,
$$
a_{n+k}=\frac{\sqrt{3}\left[2 n^{2}-(n-k)(n-k+1)\right]}{2 \sqrt{n\left(2 n^{2}+1\right)}}, k=1,2, \cdots, n-1
$$

时,上述不等式等号成立, 所求最大值为 $\sqrt{\frac{n\left(2 n^{2}+1\right)}{3}}$.

例 11 设 $x_{i} \geqslant 0, i=1,2, \cdots, n$, 满足
$$
\sum_{i=1}^{n} x_{i}^{2}+2 \sum_{1 \leqslant j<k \leqslant n} \sqrt{\frac{j}{k}} x_{j} x_{k}=1
$$

求 $x_{1}+x_{2}+\cdots+x_{n}$ 的最小值和最大值.

解 由
$$
\left(x_{1}+x_{2}+\cdots+x_{n}\right)^{2}=\sum_{i=1}^{n} x_{i}^{2}+2 \sum_{1 \leqslant j<k \leqslant n} x_{j} x_{k} \geqslant 1
$$

取 $x_{1}=1, x_{2}=\cdots=x_{n}=0$, 则 $x_{1}+x_{2}+\cdots+x_{n}$ 的最小值为 1 . 再令 $y_{i}=$ $\frac{x_{i}}{\sqrt{i}}$, 则条件化为
$$
\sum_{i=1}^{n} i y_{i}^{2}+2 \sum_{1 \leqslant j<k \leqslant n} j y_{j} y_{k}=1
$$

等价于
$$
\sum_{i=1}^{n}\left(y_{i}+y_{i+1}+\cdots+y_{n}\right)^{2}=1
$$

令 $t_{i}=y_{i}+y_{i+1}+\cdots+y_{n}$, 则 $y_{i}=t_{i}-t_{i+1}, x_{i} \geqslant 0$, 推出 $y_{i} \geqslant 0, t_{i}$ 不增, 则 $x_{i}=\sqrt{i} y_{i}=\sqrt{i}\left(t_{i}-t_{i+1}\right)$, 令 $t_{i+1}=0$, 则
$$
x_{1}+x_{2}+\cdots+x_{n}=\sum_{i=1}^{n} \sqrt{i}\left(t_{i}-t_{i+1}\right)=\sum_{i=1}^{n} t_{i}(\sqrt{i}-\sqrt{i-1})
$$

所以由 $\sum_{i=1}^{n} t_{i}^{2}=1$ 以及柯西不等式, 得
$$
\begin{aligned}
\left(x_{1}+x_{2}+\cdots+x_{n}\right)^{2}= & {\left[\sum_{i=1}^{n} t_{i}(\sqrt{i}-\sqrt{i-1})\right]^{2} } \\
\leqslant & \sum_{i=1}^{n}(\sqrt{i}-\sqrt{i-1})^{2} \cdot \sum_{i=1}^{n} t_{i}^{2} \\
& =\sum_{i=1}^{n}(\sqrt{i}-\sqrt{i-1})^{2}
\end{aligned}
$$

等号成立当且仅当 $\frac{t_{1}}{1}=\frac{t_{2}}{\sqrt{2}-1}=\cdots=\frac{t_{n}}{\sqrt{n}-\sqrt{n-1}}$ 及 $x_{1}^{2}+x_{2}^{2}+\cdots+$ $x_{n}^{2}=1$ 时, $\left(t_{1}, t_{2}, \cdots, t_{n}\right)$ 唯一确定,推出 $\left(x_{1}, x_{2}, \cdots, x_{n}\right)$ 唯一确定. 故 $x_{1}+$ $x_{2}+\cdots+x_{n}$ 的最大值为 $\sqrt{\sum_{i=1}^{n}(\sqrt{i}-\sqrt{i-1})^{2}}$.

例 12 设 $x 、 y 、 z$ 是大于 -1 的实数. 求
$$
\frac{1+x^{2}}{1+y+z^{2}}+\frac{1+y^{2}}{1+z+x^{2}}+\frac{1+z^{2}}{1+x+y^{2}}
$$

的最小值.

解 由于 $x, y, z>-1$, 则 $\frac{1+x^{2}}{1+y+z^{2}}, \frac{1+y^{2}}{1+z+x^{2}}, \frac{1+z^{2}}{1+x+y^{2}}$ 的分子、分母均为正,所以
$$
\begin{aligned}
& \frac{1+x^{2}}{1+y+z^{2}}+\frac{1+y^{2}}{1+z+x^{2}}+\frac{1+z^{2}}{1+x+y^{2}} \\
\geqslant & \frac{1+x^{2}}{1+z^{2}+\frac{1+y^{2}}{2}}+\frac{1+y^{2}}{1+x^{2}+\frac{1+z^{2}}{2}}+\frac{1+z^{2}}{1+y^{2}+\frac{1+x^{2}}{2}} \\
= & \frac{2 a}{2 c+b}+\frac{2 b}{2 a+c}+\frac{2 c}{2 b+a}
\end{aligned}
$$

其中 $a=\frac{1+x^{2}}{2}, b=\frac{1+y^{2}}{2}, c=\frac{1+z^{2}}{2}$.

由柯西不等式, 得
$$
\begin{aligned}
& \frac{a}{2 c+b}+\frac{b}{2 a+c}+\frac{c}{2 b+a} \\
\geqslant & \frac{(a+b+c)^{2}}{a(b+2 c)+b(c+2 a)+c(a+2 b)} \\
= & \frac{3(a b+b c+a c)+\frac{1}{2}\left[(b-c)^{2}+(c-a)^{2}+(a-b)^{2}\right]}{3(a b+b c+a c)} \\
\geqslant & 1
\end{aligned}
$$

且当 $a=b=c=1$ 时, 上式取到最小值.

故所求的最小值为 2 .

例 13 设 $S=\left\{a_{1}, a_{2}, \cdots, a_{n}\right\}, a_{i} \in \mathbf{Z}_{+}$, 且对任意 $S_{1}, S_{2} \subseteq S, S_{1} \neq$ $S_{2}$, 有 $\sum_{i \in S_{1}} i \neq \sum_{j \in S_{2}} j$. 求
$$
\sqrt{a_{1}}+\sqrt{a_{2}}+\cdots+\sqrt{a_{n}}
$$

的最小值.

解法一 不妨设 $a_{1}<a_{2}<\cdots<a_{n}$. 记 $T_{i}=\left\{a_{1}, a_{2}, \cdots, a_{i}\right\}, 1 \leqslant i \leqslant$ n. 则 $T_{i}$ 所有子集元素之和不同. 故 $a_{1}+a_{2}+\cdots+a_{i} \geqslant 2^{i}-1,1 \leqslant i \leqslant n$. 由 Abel 恒等式
$$
\begin{aligned}
\sum_{k=1}^{n} \sqrt{a_{k}} & =\sum_{k=1}^{n} a_{k} \frac{1}{\sqrt{a_{k}}} \\
& =\sum_{i=1}^{n} a_{i} \frac{1}{\sqrt{a_{n}}}+\sum_{k=1}^{n-1}\left(\sum_{i=1}^{k} a_{i}\right)\left(\frac{1}{\sqrt{a_{k}}}-\frac{1}{\sqrt{a_{k+1}}}\right) \\
& \geqslant \frac{1}{\sqrt{a_{n}}}\left(2^{n}-1\right)+\sum_{k=1}^{n-1}\left(2^{k}-1\right)\left(\frac{1}{\sqrt{a_{k}}}-\frac{1}{\sqrt{a_{k+1}}}\right) \\
& =\sum_{k=1}^{n} \frac{2^{k-1}}{\sqrt{a_{k}}}
\end{aligned}
$$

由柯西不等式, 得
$$
\left(\sum_{k=1}^{n} \frac{2^{k-1}}{\sqrt{a_{k}}}\right)\left(\sum_{k=1}^{n} \sqrt{a_{k}}\right) \geqslant\left(\sum_{k=1}^{n} 2^{\frac{k-1}{2}}\right)^{2}
$$

于是 $\quad \sum_{k=1}^{n} \sqrt{a_{k}} \geqslant \sum_{k=1}^{n} 2^{k-1}=(\sqrt{2}+1)\left(\sqrt{2}{ }^{n}-1\right)$.

当 $\left\{a_{1}, a_{2}, \cdots, a_{n}\right\}=\left\{1,2,4, \cdots, 2^{n-1}\right\}$ 时,
$$
\sum_{k=1}^{n} \sqrt{a_{k}}=\sum_{k=1}^{n} 2^{\frac{k-1}{2}}=(\sqrt{2}+1)\left(\sqrt{2}^{n}-1\right)
$$

故 $\sqrt{a_{1}}+\sqrt{a_{2}}+\cdots+\sqrt{a_{n}}$ 的最小值为 $(\sqrt{2}+1)\left(\sqrt{2}^{n}-1\right)$.

解法二 记 $b_{1}=1, b_{2}=2, \cdots, b_{n}=2^{n-1}$, 则 $b_{1}<b_{2}<\cdots<b_{n}$, 且 $a_{1}+a_{2}+\cdots+a_{i} \geqslant b_{1}+b_{2}+\cdots+b_{i}, 1 \leqslant i \leqslant n$.

首先容易证明下面的结论.

引理 设 $x, y \in \mathbf{R}_{+}$, 则 $\frac{x-y}{2 \sqrt{x}} \leqslant \sqrt{x}-\sqrt{y}$, 当且仅当 $x=y$ 时等号成立.利用上述引理, 得
$$
\begin{aligned}
& \sum_{i=1}^{n} \sqrt{a_{i}}-\sum_{i=1}^{n} \sqrt{b_{i}}=\sum_{i=1}^{n}\left(\sqrt{a_{i}}-\sqrt{b_{i}}\right) \geqslant \sum_{i=1}^{n} \frac{a_{i}-b_{i}}{2 \sqrt{a_{i}}} \\
= & \left(\frac{1}{2 \sqrt{a_{1}}}-\frac{1}{2 \sqrt{a_{2}}}\right)\left(a_{1}-b_{1}\right)+\left(\frac{1}{2 \sqrt{a_{2}}}-\frac{1}{2 \sqrt{a_{3}}}\right)\left(a_{1}+a_{2}-b_{1}-b_{2}\right) \\
& +\cdots+\left(\frac{1}{2 \sqrt{a_{n-1}}}-\frac{1}{2 \sqrt{a_{n}}}\right)\left(a_{1}+a_{2}+\cdots+a_{n-1}-b_{1}-b_{2}-\cdots-b_{n-1}\right) \\
& +\frac{1}{2 \sqrt{a_{n}}}\left(a_{1}+a_{2}+\cdots+a_{n}-b_{1}-b_{2}-\cdots-b_{n}\right) \\
\geqslant & 0,
\end{aligned}
$$

且当 $a_{i}=b_{i}, 1 \leqslant i \leqslant n$ 时等号成立, 从而 $\sqrt{a_{1}}+\sqrt{a_{2}}+\cdots+\sqrt{a_{n}}$ 的最小值为 $\sum_{i=1}^{n} \sqrt{b_{i}}=(\sqrt{2}+1)\left(\sqrt{2}^{n}-1\right)$.

例 14 设 $n>3$ 为给定的正整数, 实数 $x_{1}, x_{2}, \cdots, x_{n+1}, x_{n+2}$ 满足 $0<$ $x_{1}<x_{2}<\cdots<x_{n+1}<x_{n+2}$, 求
$$
\frac{\left(\sum_{i=1}^{n} \frac{x_{i+1}}{x_{i}}\right)\left(\sum_{j=1}^{n} \frac{x_{j+2}}{x_{j+1}}\right)}{\sum_{k=1}^{n} \frac{x_{k+1} x_{k+2}}{x_{k+1}^{2}+x_{k} x_{k+2}} \sum_{i=1}^{n} \frac{x_{i+1}^{2}+x_{i} x_{i+2}}{x_{i} x_{i+1}}}
$$

的最小值,并讨论等号成立的条件.

解 令 $t_{i}=\frac{x_{i+1}}{x_{i}}(1 \leqslant i \leqslant n+1)$, 则原式等于
$$
\frac{\sum_{i=1}^{n} t_{i} \sum_{i=1}^{n} t_{i+1}}{\sum_{i=1}^{n} \frac{t_{i} t_{i+1}}{t_{i}+t_{i+1}} \sum_{i=1}^{n}\left(t_{i}+t_{i+1}\right)}
$$

由柯西不等式, 得
$$
\begin{aligned}
& \sum_{i=1}^{n} \frac{t_{i} t_{i+1}}{t_{i}+t_{i+1}} \sum_{i=1}^{n}\left(t_{i}+t_{i+1}\right) \\
= & \left(\sum_{i=1}^{n} t_{i}-\sum_{i=1}^{n} \frac{t_{i}^{2}}{t_{i}+t_{i+1}}\right) \sum_{i=1}^{n}\left(t_{i}+t_{i+1}\right) \\
= & \left(\sum_{i=1}^{n} t_{i}\right) \sum_{i=1}^{n}\left(t_{i}+t_{i+1}\right)-\left(\sum_{i=1}^{n} \frac{t_{i}^{2}}{t_{i}+t_{i+1}}\right) \sum_{i=1}^{n}\left(t_{i}+t_{i+1}\right) \\
\leqslant & \sum_{i=1}^{n} t_{i} \sum_{i=1}^{n}\left(t_{i}+t_{i+1}\right)-\left(\sum_{i=1}^{n} \frac{t_{i}}{\sqrt{t_{i}+t_{i+1}}} \cdot \sqrt{t_{i}+t_{i+1}}\right)^{2} \\
= & \left(\sum_{i=1}^{n} t_{i}\right)^{2}+\left(\sum_{i=1}^{n} t_{i}\right) \sum_{i=1}^{n} t_{i+1}-\left(\sum_{i=1}^{n} t_{i}\right)^{2}=\sum_{i=1}^{n} t_{i} \sum_{i=1}^{n} t_{i+1} .
\end{aligned}
$$

所以最小值大于或等于 1.

由柯西不等式成立的条件, 得
$$
\frac{\sqrt{t_{i}+t_{i+1}}}{\frac{t_{i}}{\sqrt{t_{i}+t_{i+1}}}}=d(1 \leqslant i \leqslant n)
$$

即
$$
\frac{t_{i+1}}{t_{i}}=d-1=c, 1 \leqslant i \leqslant n
$$

再令 $t_{1}=b, t_{j}=b c^{j-1}, 1 \leqslant j \leqslant n+1$, 相应地有
$$
\frac{x_{j+1}}{x_{j}}=t_{j}=b c^{j-1}, 1 \leqslant j \leqslant n+1
$$

记 $x_{1}=a>0$, 得
$$
x_{k}=t_{k-1} t_{k-2} \cdots t_{1} a=a b^{k-1} c^{\frac{(k-1)(k-2)}{2}}, 2 \leqslant k \leqslant n+2
$$

因为 $x_{2}>x_{1}$, 所以 $b=\frac{x_{2}}{x_{1}}>1$.

又因为 $t_{j}=b c^{j-1}>1,1 \leqslant j \leqslant n+1$, 所以
$$
c>\sqrt[n]{\frac{1}{b}}\left(\geqslant \sqrt[n-1]{\frac{1}{b}}\right)
$$

故最小值为 1 , 且当且仅当 $x_{1}=a, x_{k}=a b^{k-1} c^{\frac{(k-1)(k-2)}{2}}(2 \leqslant k \leqslant n+2$, 其中 $\left.a>0, b>1, c>\sqrt[n]{\frac{1}{b}}\right)$ 时等号成立.

注 这个题目的表达形式看起来很复杂,但通过变量代换后, 可以发现各项之间的关系,借助于柯西不等式, 估计出它的下界.

\section*{4. 3 柯西不等式在证明分式不等式中的应用}
在各种不等式中, 分式不等式的问题由于自身的特点, 证明它们需要有较灵活的技巧和方法. 对于分式型的不等式, 通常运用柯西不等式的一些变形.

例 1 设 $a_{1}, a_{2}, \cdots, a_{n}$ 为正整数,求证:
$$
\frac{a_{1}^{2}}{a_{2}}+\frac{a_{2}^{2}}{a_{3}}+\cdots+\frac{a_{n}^{2}}{a_{1}} \geqslant a_{1}+a_{2}+\cdots+a_{n}
$$

证明 由柯西不等式, 得
$$
\begin{aligned}
& \left(\frac{a_{1}^{2}}{a_{2}}+\frac{a_{2}^{2}}{a_{3}}+\cdots+\frac{a_{n}^{2}}{a_{1}}\right)\left(a_{2}+a_{3}+\cdots+a_{1}\right) \\
\geqslant & \left(\frac{a_{1}}{\sqrt{a_{2}}} \cdot \sqrt{a_{2}}+\frac{a_{2}}{\sqrt{a_{3}}} \cdot \sqrt{a_{3}}+\cdots+\frac{a_{n}}{\sqrt{a_{1}}} \cdot \sqrt{a_{1}}\right)^{2} \\
= & \left(a_{1}+a_{2}+\cdots+a_{n}\right)^{2}, \\
& \quad \frac{a_{1}^{2}}{a_{2}}+\frac{a_{2}^{2}}{a_{3}}+\cdots+\frac{a_{n}^{2}}{a_{1}} \geqslant a_{1}+a_{2}+\cdots+a_{n}
\end{aligned}
$$

例 2 已知正数 $a_{1}, a_{2}, \cdots, a_{n}(n \geqslant 2)$ 满足 $\sum_{i=1}^{n} a_{i}=1$, 求证:
$$
\sum_{i=1}^{n} \frac{a_{i}}{2-a_{i}} \geqslant \frac{n}{2 n-1}
$$

证明 因为
$$
\sum_{i=1}^{n} \frac{a_{i}}{2-a_{i}}=\sum_{i=1}^{n}\left(\frac{2}{2-a_{i}}-1\right)=\sum_{i=1}^{n} \frac{2}{2-a_{i}}-n
$$

由柯西不等式, 得
$$
\left(\sum_{i=1}^{n} \frac{1}{2-a_{i}}\right)\left[\sum_{i=1}^{n}\left(2-a_{i}\right)\right] \geqslant n^{2}
$$

所以
$$
\sum_{i=1}^{n} \frac{1}{2-a_{i}} \geqslant \frac{n^{2}}{\sum_{i=1}^{n}\left(2-a_{i}\right)}=\frac{n^{2}}{2 n-1}
$$

故
$$
\sum_{i=1}^{n} \frac{a_{i}}{2-a_{i}} \geqslant \frac{2 n^{2}}{2 n-1}-n=\frac{n}{2 n-1}
$$

例 3 设 $a_{i}, b_{i}, i \geqslant 1$ 为正数, 满足 $\sum_{i=1}^{n} a_{i}=\sum_{i=1}^{n} b_{i}$. 求证:
$$
\sum_{i=1}^{n} \frac{a_{i}^{2}}{a_{i}+b_{i}} \geqslant \frac{1}{2}\left(\sum_{i=1}^{n} a_{i}\right)
$$

证明 由柯西不等式, 得
$$
\left(\sum_{i=1}^{n} \frac{a_{i}^{2}}{a_{i}+b_{i}}\right) \sum_{i=1}^{n}\left(a_{i}+b_{i}\right) \geqslant\left(\sum_{i=1}^{n} a_{i}\right)^{2}
$$

由于 $\sum_{i=1}^{n} a_{i}=\sum_{i=1}^{n} b_{i}$, 所以上式即为
$$
2\left(\sum_{i=1}^{n} a_{i}\right)\left(\sum_{i=1}^{n} \frac{a_{i}^{2}}{a_{i}+b_{i}}\right) \geqslant\left(\sum_{i=1}^{n} a_{i}\right)^{2}
$$

故命题成立.

例 4 设 $P_{1}, P_{2}, \cdots, P_{n}$ 是 $1,2, \cdots, n$ 的任一排列, 求证:
$$
\frac{1}{P_{1}+P_{2}}+\frac{1}{P_{2}+P_{3}}+\cdots+\frac{1}{P_{n-2}+P_{n-1}}+\frac{1}{P_{n-1}+P_{n}}>\frac{n-1}{n+2}
$$

证明 由柯西不等式, 得
$$
\begin{gathered}
{\left[\left(P_{1}+P_{2}\right)+\left(P_{2}+P_{3}\right)+\cdots+\left(P_{n-1}+P_{n}\right)\right] \cdot} \\
\left(\frac{1}{P_{1}+P_{2}}+\frac{1}{P_{2}+P_{3}}+\cdots+\frac{1}{P_{n-2}+P_{n-1}}+\frac{1}{P_{n-1}+P_{n}}\right) \geqslant(n-1)^{2}
\end{gathered}
$$

所以
$$
\begin{aligned}
& \frac{1}{P_{1}+P_{2}}+\frac{1}{P_{2}+P_{3}}+\cdots+\frac{1}{P_{n-2}+P_{n-1}}+\frac{1}{P_{n-1}+P_{n}} \\
\geqslant & \frac{(n-1)^{2}}{2\left(P_{1}+P_{2}+\cdots+P_{n}\right)-P_{1}-P_{n}}
\end{aligned}
$$
$$
\begin{aligned}
& =\frac{(n-1)^{2}}{n(n+1)-P_{1}-P_{n}} \geqslant \frac{(n-1)^{2}}{n(n+1)-1-2} \\
& =\frac{(n-1)^{2}}{(n-1)(n+2)-1}>\frac{(n-1)^{2}}{(n-1)(n+2)}=\frac{n-1}{n+2}
\end{aligned}
$$

例 5 设正数 $x_{i}$ 满足 $\sum_{i=1}^{n} x_{i}=1$, 求证:
$$
\sum_{i=1}^{n} \frac{x_{i}}{\sqrt{1-x_{i}}} \geqslant \frac{1}{\sqrt{n-1}} \sum_{i=1}^{n} \sqrt{x_{i}}
$$

证明 由柯西不等式, 得
$$
\sum_{i=1}^{n} \frac{1}{\sqrt{1-x_{i}}} \cdot \sum_{i=1}^{n} \sqrt{1-x_{i}} \geqslant n^{2}
$$

以及
$$
\sum_{i=1}^{n} \sqrt{1-x_{i}} \leqslant \sqrt{\sum_{i=1}^{n} 1 \cdot \sum_{i=1}^{n}\left(1-x_{i}\right)}=\sqrt{n(n-1)}
$$

所以
$$
\begin{aligned}
\sum_{i=1}^{n} \frac{x_{i}}{\sqrt{1-x_{i}}} & =\sum_{i=1}^{n} \frac{1}{\sqrt{1-x_{i}}}-\sum_{i=1}^{n} \sqrt{1-x_{i}} \\
& \geqslant \frac{n^{2}}{\sum_{i=1}^{n} \sqrt{1-x_{i}}}-\sum_{i=1}^{n} \sqrt{1-x_{i}} \\
& \geqslant \frac{n^{2}}{\sqrt{n(n-1)}}-\sqrt{n(n-1)} \\
& =\sqrt{\frac{n}{n-1}}
\end{aligned}
$$

又由柯西不等式, 得
$$
\sum_{i=1}^{n} \sqrt{x_{i}} \leqslant \sqrt{\sum_{i=1}^{n} 1 \cdot \sum_{i=1}^{n} x_{i}}=\sqrt{n}
$$

故命题成立

例 6 设 $a 、 b 、 c$ 是大于 -1 的实数,证明:
$$
\frac{1+a^{2}}{1+b+c^{2}}+\frac{1+b^{2}}{1+c+a^{2}}+\frac{1+c^{2}}{1+a+b^{2}} \geqslant 2
$$

证明 由假设我们有 $1+a^{2}, 1+b^{2}, 1+c^{2}, 1+b+c^{2}, 1+c+a^{2}, 1+a+$ $b^{2}$ 均大于零.

由柯西不等式, 得
$$
\begin{aligned}
& \left(\frac{1+a^{2}}{1+b+c^{2}}+\frac{1+b^{2}}{1+c+a^{2}}+\frac{1+c^{2}}{1+a+b^{2}}\right) \cdot\left[\left(1+a^{2}\right)\left(1+b+c^{2}\right)+\right. \\
& \left.\left(1+b^{2}\right)\left(1+c+a^{2}\right)+\left(1+c^{2}\right)\left(1+a+b^{2}\right)\right] \\
\geqslant & \left(1+a^{2}+1+b^{2}+1+c^{2}\right)^{2} .
\end{aligned}
$$

于是
$$
\begin{aligned}
& \frac{1+a^{2}}{1+b+c^{2}}+\frac{1+b^{2}}{1+c+a^{2}}+\frac{1+c^{2}}{1+a+b^{2}} \\
\geqslant & \frac{\left(a^{2}+b^{2}+c^{2}+3\right)^{2}}{\left(1+a^{2}\right)\left(1+b+c^{2}\right)+\left(1+b^{2}\right)\left(1+c+a^{2}\right)+\left(1+c^{2}\right)\left(1+a+b^{2}\right)} \\
= & \frac{a^{4}+b^{4}+c^{4}+9+2 a^{2} b^{2}+2 b^{2} c^{2}+2 c^{2} a^{2}+6 a^{2}+6 b^{2}+6 c^{2}}{a^{2} b^{2}+b^{2} c^{2}+c^{2} a^{2}+2\left(a^{2}+b^{2}+c^{2}\right)+a^{2} b+b^{2} c+c^{2} a+a+b+c+3} \\
= & 2+\frac{a^{4}+b^{4}+c^{4}+3+2 a^{2}+2 b^{2}+2 c^{2}-2\left(a^{2} b+b^{2} c+c^{2} a+a+b+c\right)}{a^{2} b^{2}+b^{2} c^{2}+c^{2} a^{2}+2\left(a^{2}+b^{2}+c^{2}\right)+a^{2} b+b^{2} c+c^{2} a+a+b+c+3} \\
= & 2+\frac{\left(a^{2}-b\right)^{2}+\left(b^{2}-c\right)^{2}+\left(c^{2}-a\right)^{2}+(a-1)^{2}+(b-1)^{2}+(c-1)^{2}}{a^{2} b^{2}+b^{2} c^{2}+c^{2} a^{2}+2\left(a^{2}+b^{2}+c^{2}\right)+a^{2} b+b^{2} c+c^{2} a+a+b+c+3} \\
\geqslant & 2
\end{aligned}
$$

当且仅当 $a=b=c=1$ 时等号成立.

例 7 正数 $a 、 b 、 c$ 满足 $a b c=1, n$ 为正整数,求证:

(1) $\frac{1}{1+2 a}+\frac{1}{1+2 b}+\frac{1}{1+2 c} \geqslant 1$;

(2) $\frac{c^{n}}{a+b}+\frac{b^{n}}{c+a}+\frac{a^{n}}{b+c} \geqslant \frac{3}{2}$.

证明 (1) 首先证明
$$
\begin{aligned}
& \frac{1}{1+2 a} \geqslant \frac{a^{-\frac{2}{3}}}{a^{-\frac{2}{3}}+b^{-\frac{2}{3}}+c^{-\frac{2}{3}}} \\
\Leftrightarrow & a^{-\frac{2}{3}}+b^{-\frac{2}{3}}+c^{-\frac{2}{3}} \geqslant a^{-\frac{2}{3}}+2 a^{\frac{1}{3}} \\
\Leftrightarrow & b^{-\frac{2}{3}}+c^{-\frac{2}{3}} \geqslant 2 b c^{-\frac{1}{3}},
\end{aligned}
$$

这是显然的.

同理有 $\frac{1}{1+2 b} \geqslant \frac{b^{-\frac{2}{3}}}{a^{-\frac{2}{3}}+b^{-\frac{2}{3}}+c^{-\frac{2}{3}}}, \frac{1}{1+2 c} \geqslant \frac{c^{-\frac{2}{3}}}{a^{-\frac{2}{3}}+b^{-\frac{2}{3}}+c^{-\frac{2}{3}}}$.\\
所以 $\frac{1}{1+2 a}+\frac{1}{1+2 b}+\frac{1}{1+2 c} \geqslant 1$.

(2) 不妨设 $a \geqslant b \geqslant c$, 那么 $a^{n-1} \geqslant b^{n-1} \geqslant c^{n-1}, \frac{a}{b+c} \geqslant \frac{b}{c+a} \geqslant \frac{c}{a+b}$.由排序不等式得到
$$
\begin{aligned}
& \frac{c^{n}}{a+b}+\frac{b^{n}}{c+a}+\frac{a^{n}}{b+c} \geqslant \frac{c a^{n-1}}{a+b}+\frac{b c^{n-1}}{c+a}+\frac{a b^{n-1}}{b+c} \\
& \frac{c^{n}}{a+b}+\frac{b^{n}}{c+a}+\frac{a^{n}}{b+c} \geqslant \frac{c b^{n-1}}{a+b}+\frac{b a^{n-1}}{c+a}+\frac{a c^{n-1}}{b+c}
\end{aligned}
$$

所以
$$
\frac{c^{n}}{a+b}+\frac{b^{n}}{c+a}+\frac{a^{n}}{b+c} \geqslant \frac{1}{3}\left(\frac{c}{a+b}+\frac{b}{c+a}+\frac{a}{b+c}\right)\left(a^{n-1}+b^{n-1}+c^{n-1}\right)
$$

而显然有 $a^{n-1}+b^{n-1}+c^{n-1} \geqslant 3$, 下面证明 $\frac{c}{a+b}+\frac{b}{c+a}+\frac{a}{b+c} \geqslant \frac{3}{2}$.

即证
$$
\begin{aligned}
& \frac{a+b+c}{a+b}+\frac{a+b+c}{c+a}+\frac{a+b+c}{b+c} \geqslant \frac{9}{2} \\
\Leftrightarrow & (a+b+c+a+b+c)\left(\frac{1}{a+b}+\frac{1}{c+a}+\frac{1}{b+c}\right) \geqslant 9
\end{aligned}
$$

这由柯西不等式可知是显然的. 所以
$$
\begin{aligned}
& \frac{c^{n}}{a+b}+\frac{b^{n}}{c+a}+\frac{a^{n}}{b+c} \\
\geqslant & \frac{1}{3}\left(\frac{c}{a+b}+\frac{b}{c+a}+\frac{a}{b+c}\right)\left(a^{n-1}+b^{n-1}+c^{n-1}\right) \geqslant \frac{3}{2}
\end{aligned}
$$

证毕.

例 8 证明:对任意满足 $x+y+z=0$ 的实数 $x 、 y 、 z$ 都有
$$
\frac{x(x+2)}{2 x^{2}+1}+\frac{y(y+2)}{2 y^{2}+1}+\frac{z(z+2)}{2 z^{2}+1} \geqslant 0
$$

证明 注意到 $\frac{x(x+2)}{2 x^{2}+1}=\frac{(2 x+1)^{2}}{2\left(2 x^{2}+1\right)}-\frac{1}{2}$, 原不等式等价于
$$
\frac{(2 x+1)^{2}}{2 x^{2}+1}+\frac{(2 y+1)^{2}}{2 y^{2}+1}+\frac{(2 z+1)^{2}}{2 z^{2}+1} \geqslant 3
$$

由柯西不等式, 我们有
$$
2 x^{2}=\frac{4}{3} x^{2}+\frac{2}{3}(y+z)^{2} \leqslant \frac{4}{3} x^{2}+\frac{4}{3}\left(y^{2}+z^{2}\right)
$$

所以
$$
\sum_{\mathrm{cyc}} \frac{(2 x+1)^{2}}{2 x^{2}+1} \geqslant 3 \sum_{\mathrm{cyc}} \frac{(2 x+1)^{2}}{4\left(x^{2}+y^{2}+z^{2}\right)+3}=3
$$

例 9 已知正数 $a_{1}, a_{2}, \cdots, a_{n}(n>2)$ 满足 $a_{1}+a_{2}+\cdots+a_{n}=1$. 证明:
$$
\frac{a_{2} a_{3} \cdots a_{n}}{a_{1}+n-2}+\frac{a_{1} a_{3} \cdots a_{n}}{a_{2}+n-2}+\cdots+\frac{a_{1} a_{2} \cdots a_{n-1}}{a_{n}+n-2} \leqslant \frac{1}{(n-1)^{2}}
$$

证明 由柯西不等式, 对于正数 $x_{1}, x_{2}, \cdots, x_{n}$, 有
$$
\frac{1}{\sum_{i=1}^{n} x_{i}} \leqslant \frac{1}{n^{2}} \sum_{i=1}^{n} \frac{1}{x_{i}}
$$

又 $a_{1}+a_{2}+\cdots+a_{n}=1(n>2)$, 则
$$
\begin{aligned}
& \sum_{i=1}^{n} \frac{1}{a_{i}\left(a_{i}+n-2\right)} \\
= & \sum_{i=1}^{n} \frac{1}{\dot{\vdots} a_{i} \sum_{\substack{j=1 \\
j \neq i}}^{n}\left(1-a_{j}\right)} \\
\leqslant & \sum_{i=1}^{n} \frac{1}{(n-1)^{2}} \sum_{\substack{j=1 \\
j \neq i}}^{n} \frac{1}{a_{i}\left(1-a_{j}\right)} \\
= & \frac{1}{(n-1)^{2}} \sum_{j=1}^{\substack{j}} \sum_{\substack{i=1 \\
i \neq j}}^{n} \frac{1}{a_{i}\left(1-a_{j}\right)}
\end{aligned}
$$

由已知得 $a_{i} \in(0,1)(i=1,2, \cdots, n)$.

于是, 对任意的 $j \in\{1,2, \cdots, n\}$, 有
$$
a_{i} \geqslant \frac{\prod_{k=1}^{n} a_{k}}{a_{i-1} a_{j}}, \quad a_{j+1} \geqslant \frac{\prod_{k=1}^{n} a_{k}}{a_{j-1} a_{j}}
$$

其中 $, i=1,2, \cdots, n, i \neq j, i \neq j+1, a_{0}=a_{n}$.

故
$$
\sum_{\substack{i=1 \\ i \neq j}}^{n} a_{i} \geqslant \sum_{\substack{i=1 \\ i \neq j}}^{n} \frac{\prod_{k=1}^{n} a_{k}}{a_{i} a_{j}} \Rightarrow\left(1-a_{j}\right) \frac{a_{j}}{\prod_{k=1}^{n} a_{k}} \geqslant \sum_{\substack{i=1 \\ i \neq j}}^{n} \frac{1}{a_{i}}
$$

即
$$
\sum_{\substack{i=1 \\ i \neq j}}^{n} \frac{1}{a_{i}\left(1-a_{j}\right)} \leqslant \frac{1}{\prod_{\substack{k=1 \\ k \neq j}}^{n} a_{k}}
$$

则
$$
\sum_{i=1}^{n} \frac{1}{a_{i}\left(a_{i}+n-2\right)} \leqslant \frac{1}{(n-1)^{2}} \sum_{j=1}^{n} \frac{1}{\prod_{\substack{k=1 \\ k \neq j}}^{n} a_{k}}
$$

故
$$
\sum_{i=1}^{n} \frac{\prod_{\substack{k=1 \\ k \neq i}}^{n} a_{k}}{a_{i}+n-2} \leqslant \frac{1}{(n-1)^{2}} \sum_{j=1}^{n} a_{j}=\frac{1}{(n-1)^{2}}
$$

例 10 设 $n \geqslant 2, a_{1}, a_{2}, \cdots, a_{n}$ 是 $n$ 个正实数, 满足:
$$
\left(a_{1}+\cdots+a_{n}\right)\left(\frac{1}{a_{1}}+\frac{1}{a_{2}}+\cdots+\frac{1}{a_{n}}\right) \leqslant\left(n+\frac{1}{2}\right)^{2}
$$

证明: $\max \left\{a_{1}, \cdots, a_{n}\right\} \leqslant 4 \min \left\{a_{1}, \cdots, a_{n}\right\}$.

证明 不妨设
$$
m=a_{1} \leqslant a_{2} \leqslant \cdots \leqslant a_{n}=M
$$

152 要证 $M \leqslant 4 m$.

当 $n=2$ 时, 条件为
$$
(m+M)\left(\frac{1}{m}+\frac{1}{M}\right) \leqslant \frac{25}{4}
$$

等价于
$$
4(m+M)^{2} \leqslant 25 m M
$$

即
$$
(4 M-m)(M-4 m) \leqslant 0
$$

而
$$
4 M-m \geqslant 3 M>0
$$

故 $M \leqslant 4 m$.

当 $n \geqslant 3$ 时, 利用柯西不等式可知
$$
\left(n+\frac{1}{2}\right)^{2} \geqslant\left(a_{1}+\cdots+a_{n}\right)\left(\frac{1}{a_{1}}+\cdots+\frac{1}{a_{n}}\right)
$$
$$
\begin{aligned}
& =\left(m+a_{2}+\cdots+a_{n-1}+M\right)\left(\frac{1}{M}+\frac{1}{a_{2}}+\cdots+\frac{1}{a_{n-1}}+\frac{1}{m}\right) \\
& \geqslant(\sqrt{\frac{m}{M}}+\underbrace{1+\cdots+1}_{n-2 \uparrow}+\sqrt{\frac{M}{m}})^{2}
\end{aligned}
$$

故
$$
n+\frac{1}{2} \geqslant \sqrt{\frac{m}{M}}+\sqrt{\frac{M}{m}}+n-2
$$

于是
$$
\sqrt{\frac{M}{m}}+\sqrt{\frac{m}{M}} \leqslant \frac{5}{2}
$$

从而
$$
2(m+M) \leqslant 5 \sqrt{m M},
$$

同 $n=2$ 的情形可得 $M \leqslant 4 m$. 命题获证.

例 11 设
$$
f(x, y, z)=\frac{x(2 y-z)}{1+x+3 y}+\frac{y(2 z-x)}{1+y+3 z}+\frac{z(2 x-y)}{1+z+3 x}
$$

其中 $x, y, z \geqslant 0$, 且 $x+y+z=1$. 求 $f(x, y, z)$ 的最大值和最小值.

解 先证 $f \leqslant \frac{1}{7}$, 当且仅当 $x=y=z=\frac{1}{3}$ 时等号成立. 因为


\begin{equation*}
f=\sum_{\mathrm{cyc}} \frac{x(x+3 y-1)}{1+x+3 y}=1-2 \sum_{\mathrm{cyc}} \frac{x}{1+x+3 y} \tag{1}
\end{equation*}


由柯西不等式
$$
\sum_{\mathrm{cyc}} \frac{x}{1+x+3 y} \geqslant \frac{\left(\sum_{\mathrm{cyc}} x\right)^{2}}{\sum_{\mathrm{cyc}} x(1+x+3 y)}=\frac{1}{\sum_{\mathrm{cyc}} x(1+x+3 y)}
$$

因为
$$
\sum_{\text {cyc }} x(1+x+3 y)=\sum_{\text {cyc }} x(2 x+4 y+z)=2+\sum_{\text {cyc }} x y \leqslant \frac{7}{3}
$$

从而
$$
\begin{aligned}
& \sum_{\text {cyc }} \frac{x}{1+x+3 y} \geqslant \frac{3}{7} \\
& f \leqslant 1-2 \times \frac{3}{7}=\frac{1}{7}
\end{aligned}
$$

$f_{\text {max }}=\frac{1}{7}$, 当且仅当 $x=y=z=\frac{1}{3}$ 时等号成立.

再证 $f \geqslant 0$, 当 $x=1, y=z=0$ 时等号成立.

事实上,
$$
\begin{aligned}
f(x, y, z)= & \frac{x(2 y-z)}{1+x+3 y}+\frac{y(2 z-x)}{1+y+3 z}+\frac{z(2 x-y)}{1+z+3 x} \\
= & x y\left(\frac{2}{1+x+3 y}-\frac{1}{1+y+3 z}\right) \\
& +x z\left(\frac{2}{1+z+3 x}-\frac{1}{1+x+3 y}\right) \\
& +y z\left(\frac{2}{1+y+3 z}-\frac{1}{1+z+3 x}\right) \\
= & \frac{7 x y z}{(1+x+3 y)(1+y+3 z)} \\
& +\frac{7 x y z}{(1+z+3 x)(1+x+3 y)} \\
& +\frac{7 x y z}{(1+y+3 z)(1+z+3 x)} \\
\geqslant & 0
\end{aligned}
$$

故 $f_{\min }=0$, 当 $x=1, y=z=0$ 时等号成立.

另证: 设 $z=\min \{x, y, z\}$, 若 $z=0$, 则
$$
f(x, y, 0)=\frac{2 x y}{1+x+3 y}-\frac{x y}{1+y}=\frac{2 x y}{2 x+4 y}-\frac{x y}{x+2 y}=0
$$

下设 $x, y \geqslant z>0$, 由(1)式, 要证 $f \geqslant 0$, 只要证


\begin{equation*}
\sum_{\text {cyc }} \frac{x}{1+x+3 y} \leqslant \frac{1}{2} \tag{2}
\end{equation*}


注意到
$$
\frac{1}{2}=\frac{x}{2 x+4 y}+\frac{y}{x+2 y}
$$

于是(2)等价于
$$
\begin{aligned}
\frac{z}{1+z+3 x} & \leqslant\left(\frac{x}{2 x+4 y}-\frac{x}{1+x+3 y}\right)+\left(\frac{y}{x+2 y}-\frac{y}{1+y+3 z}\right) \\
& =\frac{z}{2 x+4 y}\left(\frac{x}{1+x+3 y}+\frac{8 y}{1+y+3 z}\right)
\end{aligned}
$$

即


\begin{equation*}
\frac{2 x+4 y}{1+z+3 x} \leqslant \frac{x}{1+x+3 y}+\frac{8 y}{1+y+3 z} \tag{3}
\end{equation*}


而由柯西不等式, 可得
$$
\begin{aligned}
\frac{x}{1+x+3 y}+\frac{8 y}{1+y+3 z} & =\frac{x^{2}}{x(1+x+3 y)}+\frac{(2 y)^{2}}{y(1+y+3 z) / 2} \\
& \geqslant \frac{(x+2 y)^{2}}{\left(x+x^{2}+3 x y\right)+\left(y+y^{2}+3 y z\right) / 2} \\
& =\frac{2 x+4 y}{1+z+3 x}
\end{aligned}
$$

即(3)成立, 从而 $f \geqslant 0$, 故 $f_{\min }=0$, 当 $x=1, y=z=0$ 时等号成立.

例 12 设 $x, y, z \in \mathbf{R}_{+}$, 且 $x+y+z=1$, 证明:
$$
\sum_{\mathrm{cyc}} \frac{x^{4}}{y\left(1-y^{2}\right)} \geqslant \frac{1}{8}
$$

又 $\quad \sum_{\text {cyc }} x^{3}=\sum_{\text {cyc }} \frac{x^{4}}{x} \geqslant \frac{\left(\sum_{\text {cyc }} x^{2}\right)^{2}}{\sum_{\text {cyc }} x} \geqslant\left[\frac{\left(\sum_{\text {cyc }} x\right)^{2}}{3}\right]^{2}=\frac{1}{9}$,

证明 左边 $\geqslant \frac{\left(\sum_{\text {cyc }} x^{2}\right)^{2}}{\sum_{\text {cyc }} y\left(1-y^{2}\right)} \geqslant \frac{\left[\frac{\left(\sum_{\text {cyc }} x\right)^{2}}{3}\right]^{2}}{\sum_{\text {cyc }} x-\sum_{\text {cyc }} x^{3}}=\frac{\frac{1}{9}}{1-\sum_{\text {cyc }} x^{3}}$.

所以, 左边 $\geqslant \frac{\frac{1}{9}}{1-\frac{1}{9}}=\frac{1}{8}$, 故原不等式成立.

例 13 设 $x, y, z, w \in \mathbf{R}_{+}$, 证明:
$$
\frac{x}{y+2 z+3 w}+\frac{y}{z+2 w+3 x}+\frac{z}{w+2 x+3 y}+\frac{w}{x+2 y+3 z} \geqslant \frac{2}{3} .
$$

证明 左边 $=\sum_{\text {cyc }} \frac{x}{y+2 z+3 w}=\sum_{\text {cyc }} \frac{x^{2}}{x(y+2 z+3 w)}$
$$
\geqslant \frac{\left(\sum_{\text {cyc }} x\right)^{2}}{\sum_{\mathrm{cyc}} x(y+2 z+3 w)}=\frac{\left(\sum_{\text {cyc }} x\right)^{2}}{4 \sum_{\mathrm{cyc}} x y}
$$
$$
\begin{aligned}
& (x-y)^{2}+(x-z)^{2}+(x-w)^{2}+(y-z)^{2}+(y-w)^{2}+(z-w)^{2} \\
= & 3\left(x^{2}+y^{2}+z^{2}+w^{2}\right)-2(x y+x z+x w+y z+y w+z w) \\
= & 3(x+y+z+w)^{2}-8(x y+x z+x w+y z+y w+z w) \geqslant 0 . \\
& \text { 所以 }
\end{aligned}
$$
$$
\frac{\left(\sum_{\mathrm{cyc}} x\right)^{2}}{\sum_{\mathrm{cyc}} x y} \geqslant \frac{8}{3}
$$

故原不等式成立.

例 14 设 $x_{1}, x_{2}, \cdots, x_{n}$ 为任意实数,证明:
$$
\frac{x_{1}}{1+x_{1}^{2}}+\frac{x_{2}}{1+x_{1}^{2}+x_{2}^{2}}+\cdots+\frac{x_{n}}{1+x_{1}^{2}+x_{2}^{2}+\cdots+x_{n}^{2}}<\sqrt{n}
$$

证明 由柯西不等式, 得
$$
\begin{aligned}
& \left(\frac{x_{1}}{1+x_{1}^{2}}+\frac{x_{2}}{1+x_{1}^{2}+x_{2}^{2}}+\cdots+\frac{x_{n}}{1+x_{1}^{2}+x_{2}^{2}+\cdots+x_{n}^{2}}\right)^{2} \\
\leqslant & {\left[\left(\frac{x_{1}}{1+x_{1}^{2}}\right)^{2}+\left(\frac{x_{2}}{1+x_{1}^{2}+x_{2}^{2}}\right)^{2}+\cdots+\left(\frac{x_{n}}{1+x_{1}^{2}+x_{2}^{2}+\cdots+x_{n}^{2}}\right)^{2}\right] \cdot n }
\end{aligned}
$$

对 $k \geqslant 2$, 有
$$
\begin{aligned}
& \left(\frac{x_{k}}{1+x_{1}^{2}+\cdots+x_{k}^{2}}\right)^{2}=\frac{x_{k}^{2}}{\left(1+x_{1}^{2}+\cdots+x_{k}^{2}\right)^{2}} \\
\leqslant & \frac{x_{k}^{2}}{\left(1+x_{1}^{2}+\cdots+x_{k-1}^{2}\right)\left(1+x_{1}^{2}+\cdots+x_{k}^{2}\right)} \\
= & \frac{1}{1+x_{1}^{2}+\cdots+x_{k-1}^{2}}-\frac{1}{1+x_{1}^{2}+\cdots+x_{k}^{2}}
\end{aligned}
$$

对于 $k=1$, 有
$$
\left(\frac{x_{1}}{1+x_{1}^{2}}\right)^{2} \leqslant \frac{x_{1}^{2}}{1+x_{1}^{2}}=1-\frac{1}{1+x_{1}^{2}}
$$

所以
$$
\sum_{i=1}^{n}\left(\frac{x_{k}}{1+x_{1}^{2}+\cdots+x_{k}^{2}}\right)^{2} \leqslant 1-\frac{1}{1+x_{1}^{2}+\cdots+x_{n}^{2}}<1
$$

从而
$$
\left(\frac{x_{1}}{1+x_{1}^{2}}+\frac{x_{2}^{2}}{1+x_{1}^{2}+x_{2}^{2}}+\cdots+\frac{x_{n}^{2}}{1+x_{1}^{2}+\cdots+x_{n}^{2}}\right)^{2}<n
$$

故命题成立.

例 15 已知 $x_{i} \in \mathbf{R}_{+}(i \geqslant 1)$ 满足 $\sum_{i=1}^{n} x_{i}=\sum_{i=1}^{n} \frac{1}{x_{i}}$, 求证:
$$
\sum_{i=1}^{n} \frac{1}{n-1+x_{i}} \leqslant 1
$$

证明 令 $y_{i}=\frac{1}{n-1+x_{i}}$, 则 $x_{i}=\frac{1}{y_{i}}-(n-1), 0<y_{i}<\frac{1}{n-1}$.

如果 $\sum_{i=1}^{n} y_{i}>1$, 将证明 $\sum_{i=1}^{n} x_{i}<\sum_{i=1}^{n} \frac{1}{x_{i}}$, 即等价于
$$
\sum_{i=1}^{n}\left[\frac{1}{y_{i}}-(n-1)\right]<\sum_{i=1}^{n} \frac{y_{i}}{1-(n-1) y_{i}}
$$

对固定 $i$, 由柯西不等式, 得
$$
\begin{aligned}
& \sum_{j \neq i} \frac{1-(n-1) y_{i}}{1-(n-1) y_{j}} \\
\geqslant & \frac{\left[1-(n-1) y_{i}\right](n-1)^{2}}{\sum_{j \neq i}\left[1-(n-1) y_{j}\right]} \\
> & \frac{\left[1-(n-1) y_{i}\right](n-1)^{2}}{(n-1) y_{i}}=\frac{(n-1)\left[1-(n-1) y_{i}\right]}{y_{i}}
\end{aligned}
$$

对 $i$ 求和, 得
$$
\sum_{i=1}^{n} \sum_{j \neq i} \frac{1-(n-1) y_{i}}{1-(n-1) y_{j}} \geqslant(n-1) \sum_{i=1}\left[\frac{1}{y_{i}}-(n-1)\right]
$$

由于

故
$$
\begin{gathered}
\sum_{i=1}^{n} \sum_{j \neq i} \frac{1-(n-1) y_{i}}{1-(n-1) y_{j}} \leqslant \sum_{j=1}^{n} \frac{(n-1) y_{j}}{1-(n-1) y_{j}} \\
\sum_{i=1}^{n} \frac{y_{i}}{1-(n-1) y_{i}}>\sum_{j=1}^{n}\left[\frac{1}{y_{j}}-(n-1)\right]
\end{gathered}
$$

\section*{4. 4 柯西不等式在组合计数估计中的应用}
在研究组合, 特别是组合计数问题时, 常常需要由给定的条件, 对一些不\\
等式进行估计. 如果能灵活地应用, 柯西不等式在解决这些问题中能发挥很好的作用.

例 1 将 1650 个学生排成 22 行, 75 列的方阵,已知任意给定的两列处于同一行的两个人中, 性别相同的学生不超过 11 对, 证明: 男生的人数不超过 928 .

解 设第 $i$ 行的男生数为 $x_{i}$, 则女生数为 $75-x_{i}$, 依题意, 得
$$
\sum_{i=1}^{22}\left(\mathrm{C}_{x_{i}}^{2}+\mathrm{C}_{75-x_{i}}^{2}\right) \leqslant 11 \times \mathrm{C}_{75}^{2}
$$

于是
$$
\sum_{i=1}^{22}\left(x_{i}^{2}-75 x_{i}\right) \leqslant-30525
$$

即 $\sum_{i=1}^{22}\left(2 x_{i}-75\right)^{2} \leqslant 1650$. 由柯西不等式, 得
$$
\left[\sum_{i=1}^{22}\left(2 x_{i}-750\right)\right]^{2} \leqslant 22 \sum_{i=1}^{22}\left(2 x_{i}-75\right)^{2} \leqslant 36300
$$

因此 $\sum_{i=1}^{22}\left(2 x_{i}-75\right)<191$, 从而
$$
\sum_{i=1}^{22} x_{i}<\frac{191+1650}{2}<921
$$

故男生的人数不超过 928.

例 2 在一群数学家中, 每一个人都有一些朋友 (关系是互相的). 证明:存在一个数学家他所有的朋友的平均值不小于这群人的朋友的平均数.

证明 记 $M$ 为这群数学家的集合, $n=|M|, F(m)$ 表示数学家 $m$ 的朋友的集合, $f(m)$ 表示数学家 $m$ 的朋友数 $(f(m)=|F(m)|)$. 即命题等价于证明: 必有一个 $m_{0}$ 使
$$
\frac{1}{f\left(m_{0}\right)} \sum_{m \in F\left(m_{0}\right)} f(m) \geqslant \frac{1}{n} \sum_{m \in M} f(m)
$$

我们用反证法来证明这个命题, 如果不存在这样的数学家 $m_{0}$. 则对任意的 $m_{0}$, 有
$$
n \cdot \sum_{m \in F\left(m_{0}\right)} f(m)<f\left(m_{0}\right) \sum_{m \in M} f(m)
$$

对一切 $m_{0}$ 求和, 得
$$
n \cdot \sum_{m_{0}} \sum_{m \in F\left(m_{0}\right)} f(m)=n \sum_{m} \sum_{m \in F\left(m_{0}\right)} f(m)=n \sum_{m \in M} f^{2}(m)<\left(\sum_{m \in M} f(m)\right)^{2}
$$

这与柯西不等式矛盾, 故命题成立.

例 3 设空间中有 $2 n(n \geqslant 2)$ 个点, 其中任何 4 点都不共面. 在它们之间任意连接 $N$ 条线段, 这些线段都至少构成一个三角形. 求 $N$ 的最小值.

解 将 $2 n$ 个已知点均分为 $A 、 B$ 两组:
$$
A=\left\{A_{1}, A_{2}, \cdots, A_{n}\right\}, B=\left\{B_{1}, B_{2}, \cdots, B_{n}\right\}
$$

现将每对点 $A_{i}$ 和 $B_{i}$ 之间都连接一条线段 $A_{i} B_{i}$, 而同组的任意两点之间不连线, 则共有 $n^{2}$ 条线段. 这时, $2 n$ 个已知点中的任何 3 点中至少有两点属于同一组, 两者之间没有连线. 因而这 $n^{2}$ 条线段不能构成任何三角形. 这表明 $N$ 的最小值必大于 $n^{2}$. 由于 $2 n$ 个点之间连有 $n^{2}+1$ 条线段, 平均每点引出 $n$条线段还多, 故可以猜想有一条线段的两个端点引出的线段之和不小于 $2 n+$ 1. 下面证明 $N$ 的最小值为 $2 n+1$.

设从 $A_{1}, A_{2}, \cdots, A_{2 n}$ 引出的线段条数分别为 $a_{1}, a_{2}, \cdots, a_{2 n}$, 且对于任一线段 $A_{i} A_{j}$ 都有 $a_{i}+a_{j} \leqslant 2 n$. 于是, 所有线段的两端点所引出的线段条数之和不超过 $2 n\left(n^{2}+1\right)$. 但在此计数中, $A_{i}$ 点恰被计算了 $a_{i}$ 次, 故有
$$
\sum_{i=1}^{2 n} a_{i}^{2} \leqslant 2 n\left(n^{2}+1\right)
$$

另一方面, 显然有
$$
\sum_{i=1}^{2 n} a_{i}=2\left(n^{2}+1\right)
$$

故由柯西不等式, 得
$$
\left(\sum_{i=1}^{2 n} a_{i}\right)^{2} \leqslant 2 n\left(\sum_{i=1}^{2 n} a_{i}^{2}\right)
$$

即
$$
\sum_{i=1}^{2 n} a_{i}^{2} \geqslant \frac{1}{2 n} \cdot 4\left(n^{2}+1\right)^{2}>2 n\left(n^{2}+1\right)
$$

于是矛盾, 从而证明了必有一条线段, 从它的两端点引出的线段数之和不小于 $2 n+1$. 不妨设 $A_{1} A_{2}$ 是一条这样的线段, 从而又有 $A_{k}(k \geqslant 3)$, 使线段 $A_{1} A_{k}, A_{2} A_{k}$ 都存在, 于是 $\triangle A_{1} A_{2} A_{k}$ 即为所求.

例 4 在 $m \times m$ 方格纸中, 至少要挑出多少个小方格, 才能使得这些小方格中存在四个小方格, 它们的中心组成一个矩形的 4 个顶点, 而矩形的边平行于原正方形的边.

解 所求的最小值为 $\left[\frac{m}{2}(1+\sqrt{4 m-3})-1\right]+1$. 设最多能挑出 $k$ 个小\\
方格, 使得这些小方格中不存在任何四个小方格, 它们的中点组成一个矩形的 4 个顶点 (矩形的边平行于原正方形的边). 并假设位于第 $i$ 行的有 $k_{i}(i=$ $1,2, \cdots, m)$ 个, 则
$$
\sum_{i=1}^{n} k_{i}=k
$$

设第 $i$ 行的 $k_{i}$ 个小方格位于这行的第 $j_{1}, j_{2}, \cdots, j_{k_{i}}$ 列, $1 \leqslant j_{1}<$ $j_{2}<\cdots<j_{k_{i}} \leqslant m$. 如果第 $r$ 行的第 $j_{p}, j_{q}$ 列的两个方格已经挑出, 则任意的第 $s(s \neq r)$ 行的 $j_{p}, j_{q}$ 列的两个方格不能同时挑出, 否则将组成一个矩形的 4 个顶点. 所以对于每个 $i$, 考虑 $j_{1}, j_{2}, \cdots, j_{k_{i}}$ 中每两个的组合, 可得到 $\mathrm{C}_{k_{i}}^{2}$ 个组合. 对 $i=1,2, \cdots, m$, 可得 $\sum \mathrm{C}_{k_{i}}^{2}$ 个组合, 且其中任意两个不相同 (即无重复), 这些组合都是 $1,2, \cdots, m$ 中取两个的组合, 总数为 $\mathrm{C}_{m}^{2}$. 所以
$$
\sum_{i=1}^{m} \mathrm{C}_{k_{i}}^{2} \leqslant \mathrm{C}_{m}^{2}
$$

即
$$
\frac{1}{2} \sum_{i=1}^{m} k_{i}\left(k_{i}-1\right) \leqslant \frac{1}{2} m(m-1)
$$

由 $\sum_{i=1}^{m} k_{i}=k$, 得到 $\sum_{i=1}^{m} k_{i}^{2} \leqslant m(m-1)+k$. 由柯西不等式, 得
$$
\sum_{i=1}^{m} k_{i}^{2} \geqslant \frac{\left(\sum_{i=1}^{m} k_{i}\right)^{2}}{m}=\frac{k^{2}}{m}
$$

所以 $\frac{k^{2}}{m} \leqslant m(m-1)+k$, 故 $k \leqslant \frac{m}{2}(1+\sqrt{4 m-3})$.

因此, 至少要挑出 $\left[\frac{m}{2}(1+\sqrt{4 m-3})-1\right]+1$ 个小方格.

例 5 设 $A_{1}, A_{2}, \cdots, A_{30}$ 是集 $\{1,2, \cdots, 2003\}$ 的子集,且 $\left|A_{i}\right| \geqslant 660$ $(i=1,2, \cdots, 30)$. 证明: 存在 $i, j \in\{1,2, \cdots, 30\}, i \neq j$, 使得
$$
\left|A_{i} \cap A_{j}\right| \geqslant 203
$$

证明 不妨设每个 $A_{i}$ 的元素都为 660 个 (否则去除一些元素), 我们作一个集合、元素的关系表:表中每一行(除最上面的一行)表示 30 个集合,表的 $n$列(最左面一列除外)表示 2003 个元素 $1,2, \cdots, 2003$. 如果 $i \in A_{j}(i=1$, $2, \cdots, 2003,1 \leqslant j \leqslant 30)$, 则在 $i$ 所在的列与 $A_{j}$ 所在的交叉处填上 1 , 如果 $i \notin A_{j}$, 则写上 0 . 表中每一行有 660 个 1 , 因此共有 $30 \times 660$ 个 1 . 第 $j$ 列有\\
$m_{j}$ 个 $1(j=1,2, \cdots, 2003)$, 则
$$
\sum_{j=1}^{2003} m_{j}=30 \times 660
$$

由于每个元素 $j$ 属于 $\mathrm{C}_{m_{j}}^{2}$ 个交集 $A_{s} \cap A_{t}$, 因此
$$
\sum_{j=1}^{2003} \mathrm{C}_{m_{j}}^{2}=\sum_{1 \leqslant s<t \leqslant 30}\left|A_{s} \cap A_{t}\right|
$$

由柯西不等式, 得
$$
\sum_{j=1}^{2003} \mathrm{C}_{m_{j}}^{2}=\frac{1}{2}\left(\sum_{j=1}^{2003} m_{j}^{2}-\sum_{j=1}^{2003} m_{j}\right) \geqslant \frac{1}{2}\left[\frac{1}{2003}\left(\sum_{j=1}^{2003} m_{i}\right)^{2}-\sum_{j=1}^{2003} m_{j}\right]
$$

所以,必有 $i \neq j$, 满足
$$
\begin{aligned}
\left|A_{i} \cap A_{j}\right| & \geqslant \frac{1}{\mathrm{C}_{30}^{2}} \times \frac{1}{2}\left[\frac{1}{2003}\left(\sum_{j=1}^{2003} m_{j}\right)^{2}-\sum_{j=1}^{2003} m_{j}\right] \\
& =\frac{660(30 \times 660-2003)}{29 \times 2003}>202
\end{aligned}
$$

故 $\quad\left|A_{i} \cap A_{j}\right| \geqslant 203$.

例 6 给定平面上的 $n$ 个相异点. 证明: 其中距离为单位长的点对少于 $2 \sqrt{n^{3}}$ 对.

证明 对于平面上的点集 $\left\{P_{1}, P_{2}, \cdots, P_{n}\right\}$, 令 $a_{i}$ 为与 $P_{i}$ 相距为单位长的点 $P_{i}$ 的个数. 不妨设 $a_{i} \geqslant 1$, 则相距为单位长的点对的对数是
$$
A=\frac{a_{1}+a_{2}+\cdots+a_{n}}{2}
$$

设 $C_{i}$ 是以点 $P_{i}$ 为圆心, 以 1 为半径的圆.

因为每对圆至多有 2 个交点, 故所有的 $C_{i}$ 至多有 $2 \mathrm{C}_{n}^{2}=n(n-1)$ 个交点.

点 $P_{i}$ 作为 $C_{j}$ 的交点出现 $\mathrm{C}_{a_{j}}^{2}$ 次, 因此
$$
n(n-1) \geqslant \sum_{j=1}^{n} \mathrm{C}_{a_{j}}^{2}=\sum_{j=1}^{n} \frac{a_{j}\left(a_{j}-1\right)}{2} \geqslant \frac{1}{2} \sum_{j=1}^{n}\left(a_{j}-1\right)^{2}
$$

由柯西不等式, 得
$$
\left[\sum_{j=1}^{n}\left(a_{j}-1\right)\right]^{2} \leqslant n \cdot \sum_{j=1}^{n}\left(a_{j}-1\right)^{2} \leqslant n \cdot 2 n(n-1)<2 n^{3}
$$

于是
$$
\sum_{j=1}^{n}\left(a_{j}-1\right)<\sqrt{2} \cdot \sqrt{n^{3}}
$$

从而
$$
A=\frac{\sum_{j=1}^{n} a_{j}}{2}<\frac{n+\sqrt{2 n^{3}}}{2}<2 \sqrt{n^{3}}
$$

故命题成立.

例 7 在三维空间中给定一点 $O$ 以及由总长度为 1988 的若干条线段组成的有限集 $A$, 证明: 存在一个平面与集 $A$ 不相交且到点 $O$ 的距离不超过 574 .

证明 以点 $O$ 为原点建立直角坐标系, 并将所给的线段分别向 3 条坐标轴投影. 设 $A$ 中共有 $n$ 条线段且它们在 3 条轴上的投影长分别为
$$
\begin{aligned}
& x_{i}, y_{i}, z_{i}, i=1,2, \cdots, n \\
& \text { 记 } x=\sum x_{i}, y=\sum y_{i}, z=\sum z_{i} . \text { 于是, 由柯西不等式, 得 } \\
& x^{2}+y^{2}+z^{2}=\left(\sum x_{i}\right)^{2}+\left(\sum y_{i}\right)^{2}+\left(\sum z_{i}\right)^{2} \\
& =\sum_{i=1}^{n} \sum_{j=1}^{n}\left(x_{i} x_{j}+y_{i} y_{j}+z_{i} z_{j}\right) \\
& \leqslant \sum_{i=1}^{n} \sum_{j=1}^{n} \sqrt{\left(x_{i}^{2}+y_{i}^{2}+z_{i}^{2}\right)\left(x_{j}^{2}+y_{j}^{2}+z_{j}^{2}\right)} \\
& =\left(\sum_{i=1}^{n} \sqrt{x_{i}^{2}+y_{i}^{2}+z_{i}^{2}}\right)^{2}=1988^{2}
\end{aligned}
$$

不妨设 $x=\min \{x, y, z\}$, 于是
$$
x \leqslant \frac{1988}{\sqrt{3}}<2 \times 574
$$

从而在 $x$ 轴上的区间 $[-574,574]$ 内必有一点不在 $n$ 条给定线段的投影上, 过这点作与 $x$ 轴垂直的平面便满足题中的要求.

例 8 设 $O x y z$ 是空间直角坐标系, $S$ 是空间中一个有限点集, $S_{x} 、 S_{y} 、 S_{z}$分别是 $S$ 中所有点在 $O y z$ 平面, $O z x$ 平面和 $O x y$ 平面上的正投影所成的集合. 求证:
$$
|S|^{2} \leqslant\left|S_{x}\right| \cdot\left|S_{y}\right| \cdot\left|S_{z}\right|
$$

说明:所谓一个点在一个平面上的正投影是指由点向平面所作垂线的\\
垂足.

证明 设共有 $n$ 个平行于 $O x y$ 平面的平面上有 $S$ 中的点, 这些平面分别记为 $M_{1}, M_{2}, \cdots, M_{n}$. 对于平面 $M_{i}, 1 \leqslant i \leqslant n$, 设它与 $O z x 、 O z y$ 平面分别交于直线 $l_{y}$ 和 $l_{x}$, 并设 $M_{i}$ 上有 $m_{i}$ 个 $S$ 中的点. 显然, $m_{i} \leqslant\left|S_{z}\right|$.

设 $M_{i}$ 上的点在 $l_{x} 、 l_{y}$ 上的正投影的集合分别为 $A_{i}$ 和 $B_{i}$, 记 $a_{i}=\left|A_{i}\right|$, $b_{i}=\left|B_{i}\right|$, 则有 $m_{i} \leqslant a_{i} b_{i}$. 又因为
$$
\sum_{i=1}^{n} a_{i}=\left|S_{y}\right|, \sum_{i=1}^{n} b_{i}=\left|S_{x}\right|, \sum_{i=1}^{n} m_{i}=|S|
$$

从而由柯西不等式, 得
$$
\begin{aligned}
\left|S_{x}\right| \cdot\left|S_{y}\right| \cdot\left|S_{z}\right| & =\left(\sum_{i=1}^{n} b_{i}\right)\left(\sum_{i=1}^{n} a_{i}\right) \cdot\left|S_{z}\right| \\
& \geqslant\left(\sum_{i=1}^{n} \sqrt{a_{i} b_{i}}\right)^{2} \cdot\left|S_{z}\right| \\
& =\left(\sum_{i=1}^{n} \sqrt{a_{i} b_{i}\left|S_{z}\right|}\right)^{2} \\
& \geqslant\left(\sum_{i=1}^{n} m_{i}\right)^{2}=|S|^{2}
\end{aligned}
$$

得证.

例 9 某次考试共 $m$ 道试题, $n$ 个学生参加, 其中 $m, n \geqslant 2$ 为给定整数,每道题得分规则为: 若该题恰有 $x$ 个学生没有答对, 则每个答对该题的学生得 $x$ 分, 未答对的学生得零分. 每个学生的总分为其 $m$ 道题的得分总和. 将所有学生总分从高到低排列为 $p_{1} \geqslant p_{2} \geqslant \cdots \geqslant p_{n}$. 求 $p_{1}+p_{n}$ 的最大值.

解 设第 $k$ 题没有答对者有 $x_{k}$ 人, $1 \leqslant k \leqslant m$, 则第 $k$ 题答对者有 $n-x_{k}$人, 由得分规则知, 这 $n-x_{k}$ 个人在第 $k$ 题均得 $x_{k}$ 分. 设 $n$ 个学生的得分之和为 $S$. 则有
$$
\sum_{i=1}^{n} p_{i}=S=\sum_{i=1}^{m} x_{i}\left(n-x_{i}\right)=n \sum_{i=1}^{m} x_{i}-\sum_{i=1}^{m} x_{i}^{2}
$$

因为每一个人在第 $k$ 道题上至多得 $x_{k}$ 分, 则
$$
p_{1} \leqslant \sum_{k=1}^{m} x_{k}
$$

由于 $p_{2} \geqslant \cdots \geqslant p_{n}$, 故有 $p_{n} \leqslant \frac{p_{2}+p_{3}+\cdots+p_{n}}{n-1}=\frac{S-p_{1}}{n-1}$.\\
所以 $p_{1}+p_{n} \leqslant p_{1}+\frac{S-p_{1}}{n-1}=\frac{n-2}{n-1} p_{1}+\frac{S}{n-1}$
$$
\begin{aligned}
& \leqslant \frac{n-2}{n-1} \sum_{k=1}^{m} x_{k}+\frac{1}{n-1}\left(n \sum_{k=1}^{m} x_{k}-\sum_{k=1}^{m} x_{k}^{2}\right) \\
& =2 \sum_{k=1}^{m} x_{k}-\frac{1}{n-1} \sum_{k=1}^{m} x_{k}^{2}
\end{aligned}
$$

由柯西不等式得 $\quad \sum_{k=1}^{m} x_{k}^{2} \geqslant \frac{1}{m}\left(\sum_{k=1}^{m} x_{k}\right)^{2}$.

于是 $p_{1}+p_{n} \leqslant 2 \sum_{k=1}^{m} x_{k}-\frac{1}{m(n-1)}\left(\sum_{k=1}^{m} x_{k}\right)^{2}$
$$
=-\frac{1}{m(n-1)}\left(\sum_{k=1}^{m} x_{k}-m(n-1)\right)^{2}+m(n-1)
$$
$$
\leqslant m(n-1)
$$

另一方面, 若有一个学生全部答对, 其他 $n-1$ 个学生全部答错, 则
$$
p_{1}+p_{n}=p_{1}=\sum_{k=1}^{m}(n-1)=m(n-1)
$$

故 $p_{1}+p_{n}$ 的最大值为 $m(n-1)$.

\section*{4. 5 带参数的柯西不等式}
如果 $a_{i}, b_{i} \in \mathbf{R}, \lambda_{i}>0, i=1,2, \cdots, n$, 则
$$
\left(\sum_{i=1}^{n} a_{i} b_{i}\right)^{2} \leqslant \sum_{i=1}^{n} \lambda_{i} a_{i}^{2} \cdot \sum_{i=1}^{n} \frac{1}{\lambda_{i}} b_{i}^{2}
$$

例 1 已知正实数 $a 、 b 、 c 、 d$ 满足
$$
a\left(c^{2}-1\right)=b\left(b^{2}+c^{2}\right)
$$

且 $d \leqslant 1$. 证明:
$$
d\left(a \sqrt{1-d^{2}}+b^{2} \sqrt{1+d^{2}}\right) \leqslant \frac{(a+b) c}{2}
$$

证明 设参数 $\lambda>1$, 由柯西不等式得
$$
\begin{aligned}
& d\left(a \sqrt{1-d^{2}}+b^{2} \sqrt{1+d^{2}}\right) \\
\leqslant & d \sqrt{\left(\frac{a^{2}}{\lambda}+b^{4}\right)\left[\left(1-d^{2}\right) \lambda+\left(1+d^{2}\right)\right]}
\end{aligned}
$$
$$
\begin{aligned}
& =\sqrt{\left(\frac{a^{2}}{\lambda}+b^{4}\right)\left[(1-\lambda) d^{4}+(\lambda+1) d^{2}\right]} \\
& \leqslant \sqrt{\frac{1}{\lambda-1}\left(\frac{a^{2}}{\lambda}+b^{4}\right)} \cdot \frac{\lambda+1}{2}
\end{aligned}
$$

由已知条件知 $c^{2}=\frac{a+b^{3}}{a-b}$. 故 $a>b$, 取 $\lambda=\frac{a}{b}$. 则
$$
\frac{\lambda+1}{2} \sqrt{\frac{1}{\lambda-1}\left(\frac{a^{2}}{\lambda}+b^{4}\right)}=\frac{a+b}{2} \sqrt{\frac{a+b^{3}}{a-b}}=\frac{(a+b) c}{2}
$$

所以, 命题得证.

例 2 设 $p, q \in \mathbf{R}_{+}, x \in\left(0, \frac{\pi}{2}\right)$, 试求
$$
\frac{p}{\sqrt{\sin x}}+\frac{q}{\sqrt{\cos x}}
$$

的最小值.

解 由柯西不等式, 得
$$
(\sqrt{p m}+\sqrt{q n})^{2} \leqslant\left(\frac{p}{\sqrt{\sin x}}+\frac{q}{\sqrt{\cos x}}\right)(m \sqrt{\sin x}+n \sqrt{\cos x})
$$

当且仅当 $\frac{\frac{p}{\sqrt{\sin x}}}{m \sqrt{\sin x}}=\frac{\frac{q}{\sqrt{\cos x}}}{n \sqrt{\cos x}}$ 时, 等号成立. 故 $\tan x=\frac{n p}{m q}$.
$$
\text { 又 } \begin{aligned}
(m \sqrt{\sin x}+n \sqrt{\cos x})^{2} & =\left(\frac{m}{a} \cdot a \sqrt{\sin x}+\frac{n}{b} \cdot b \sqrt{\cos x}\right)^{2} \\
& \leqslant\left(\frac{m^{2}}{a^{2}}+\frac{n^{2}}{b^{2}}\right)\left(a^{2} \sin x+b^{2} \cos x\right) \\
& \leqslant\left(\frac{m^{2}}{a^{2}}+\frac{n^{2}}{b^{2}}\right) \sqrt{a^{4}+b^{4}}
\end{aligned}
$$

当且仅当 $\tan x=\frac{a^{2}}{b^{2}}, \frac{a^{2} \sin x}{\frac{m^{2}}{a^{2}}}=\frac{b^{2} \cos x}{\frac{n^{2}}{b^{2}}}$ 时, 即 $\tan x=\frac{b^{4} m^{2}}{a^{4} n^{2}}=\frac{a^{2}}{b^{2}}$ 时, 等号

成立. 故

且
$$
\begin{gathered}
\frac{m}{n}=\frac{a^{3}}{b^{3}}, \tan x=\left(\frac{m}{n}\right)^{\frac{2}{3}} \\
m \sqrt{\sin x}+n \sqrt{\cos x} \leqslant\left(m^{\frac{4}{3}}+n^{\frac{4}{3}}\right)^{\frac{3}{4}}
\end{gathered}
$$

从而
$$
\begin{aligned}
& \left(\frac{m}{n}\right)^{\frac{2}{3}}=\frac{n p}{m q} \\
& \frac{m}{n}=\left(\frac{p}{q}\right)^{\frac{3}{5}}
\end{aligned}
$$

即

令 $m=p^{\frac{3}{5}}, n=q^{\frac{3}{5}}$, 得
$$
\frac{p}{\sqrt{\sin x}}+\frac{q}{\sqrt{\cos x}} \geqslant \frac{(\sqrt{p m}+\sqrt{n q})^{2}}{\left(m^{\frac{4}{3}}+n^{\frac{4}{3}}\right)^{\frac{3}{4}}}=\left(p^{\frac{4}{5}}+q^{\frac{4}{5}}\right)^{\frac{5}{4}}
$$

当且仅当 $\tan x=\left(\frac{m}{n}\right)^{\frac{2}{3}}=\left[\left(\frac{p}{q}\right)^{\frac{3}{5}}\right]^{\frac{2}{3}}=\left(\frac{p}{q}\right)^{\frac{2}{5}}$ 时, 等号成立.

注 这里, 在两次利用柯西不等式时, 引进了参数 $n 、 m 、 a 、 b$.

例 3 (1) 设 3 个正实数 $a 、 b 、 c$ 满足
$$
\left(a^{2}+b^{2}+c^{2}\right)^{2}>2\left(a^{4}+b^{4}+c^{4}\right)
$$

求证: $a 、 b 、 c$ 一定是某个三角形的 3 条边长;

(2) 设 $n$ 个正实数 $a_{1}, a_{2}, \cdots, a_{n}(n \geqslant 4)$ 满足
$$
\left(a_{1}^{2}+a_{2}^{2}+\cdots+a_{n}^{2}\right)^{2}>(n-1)\left(a_{1}^{4}+a_{2}^{4}+\cdots+a_{n}^{4}\right)
$$

求证: 这些数中任意 3 个一定是某个三角形的 3 条边长.

证明 (1) 不妨设 $a \geqslant b \geqslant c>0$, 由题设, 得
$$
\left(a^{2}+b^{2}+c^{2}\right)^{2}-2\left(a^{4}+b^{4}+c^{4}\right)>0
$$

分解因式, 得
$$
(a+b+c)(a+b-c)(a+c-b)(b+c-a)>0
$$

所以 $b+c-a>0$, 即 $b+c>a$, 从而 $a 、 b 、 c$ 是某个三角形的 3 条边长;

(2) 在 $a_{1}, a_{2}, \cdots, a_{n}$ 中任取 3 个, 不妨设为 $a_{1} 、 a_{2} 、 a_{3}$. 由带参数的柯西不等式, 得
$$
\begin{aligned}
(n-1)\left(\sum a_{i}^{4}\right) & <\left(\sum_{i=1}^{n} a_{i}^{2}\right)^{2} \\
& =\left[\lambda\left(a_{1}^{2}+a_{2}^{2}+a_{3}^{2}\right) \cdot \frac{1}{\lambda}+\sum_{i=4}^{n} a_{i}^{2}\right]^{2} \\
& \leqslant\left[\lambda^{2}\left(a_{1}^{2}+a_{2}^{2}+a_{3}^{2}\right)^{2}+\sum_{i=4}^{n} a_{i}^{4}\right]\left(\frac{1}{\lambda^{2}}+n-3\right)
\end{aligned}
$$

令 $\frac{1}{\lambda^{2}}+n-3=n-1$, 即 $\lambda=\sqrt{\frac{1}{2}}$, 所以
$$
\left(a_{1}^{2}+a_{2}^{2}+a_{3}^{2}\right)^{2}>2\left(a_{1}^{4}+a_{2}^{4}+a_{3}^{4}\right) .
$$

由 (1) 知, $a_{1} 、 a_{2} 、 a_{3}$ 为某个三角形的三边长.

例 4 设 $a=\left(a_{1}, a_{2}, \cdots, a_{n}\right)$ 和 $b=\left(b_{1}, b_{2}, \cdots, b_{n}\right)$ 是两个不成比例的实数序列, 又设 $x=\left(x_{1}, x_{2}, \cdots, x_{n}\right)$ 是使
$$
\sum_{i=1}^{n} a_{i} x_{i}=0, \sum_{i=1}^{n} b_{i} x_{i}=1
$$

成立的任意实数序列. 求证:
$$
\sum_{i=1}^{n} x_{i}^{2} \geqslant \frac{A}{A B-C^{2}}
$$

其中 $A=\sum_{i=1}^{n} a_{i}^{2}, B=\sum_{i=1}^{n} b_{i}^{2}, C=\sum_{i=1}^{n} a_{i} b_{i}$.

证明 对任意实数 $\lambda$, 由柯西不等式, 得
$$
\left(\sum_{i=1}^{n} x_{i}^{2}\right) \sum_{i=1}^{n}\left(a_{i} \lambda-b_{i}\right)^{2} \geqslant\left(\lambda \sum_{i=1}^{n} a_{i} x_{i}-\sum_{i=1}^{n} b_{i} x_{i}\right)^{2}=1
$$

从而
$$
\left(\sum_{i=1}^{n} x_{i}^{2}\right)\left(A \lambda^{2}-2 C \lambda+B\right) \geqslant 1
$$

即对任意实数 $\lambda$, 有
$$
A \lambda^{2}-2 C \lambda+B-\frac{1}{\sum_{i=1}^{n} x_{i}^{2}} \leqslant 0
$$

于是
$$
\Delta=4 C^{2}-4 A B+\frac{4 A}{\sum_{i=1}^{n} x_{i}^{2}} \leqslant 0
$$

故命题成立.

注 该不等式的证明,也可通过构造一个新的序列 $\left\{y_{i}\right\}$ :
$$
y_{i}=\frac{A b_{i}-C a_{i}}{A B-C^{2}}, i \geqslant 1
$$

则 $\left\{y_{i}\right\}$ 满足条件
$$
\begin{gathered}
\sum_{i=1}^{n} x_{i} y_{i}=\frac{A}{A B-C^{2}}, \sum_{i=1}^{n} y_{i}^{2}=\frac{A}{A B-C^{2}} \\
\sum_{i=1}^{n} x_{i}^{2}-\sum_{i=1}^{n} y_{i}^{2}=\sum_{i=1}^{n}\left(x_{i}-y_{i}\right)^{2}
\end{gathered}
$$

从而命题成立.

例 5 设 $a_{i}>0,1 \leqslant i \leqslant n$. 求证:
$$
\sum_{k=1}^{n} \frac{k}{\sum_{i=1}^{k} a_{i}} \leqslant 2 \sum_{i=1}^{n} \frac{1}{a_{i}}
$$

证明 由柯西不等式得
$$
\left(\sum_{i=1}^{k} a_{i}\right)\left(\sum_{i=1}^{k} \frac{i^{2}}{a_{i}}\right) \geqslant\left(\sum_{i=1}^{k} i\right)^{2}=\left(\frac{k(k+1)}{2}\right)^{2}
$$

于是 $\sum_{k=1}^{n} \frac{k}{\sum_{i=1}^{k} a_{i}} \leqslant \sum_{k=1}^{n}\left(\frac{4}{k(k+1)^{2}} \sum_{i=1}^{k} \frac{i^{2}}{a_{i}}\right)$
$$
\begin{aligned}
& <2 \sum_{i=1}^{n}\left(\frac{i^{2}}{a_{i}} \sum_{k=i}^{n} \frac{2 k+1}{k^{2}(k+1)^{2}}\right) \\
& =2 \sum_{i=1}^{n}\left(\frac{i^{2}}{a_{i}} \sum_{k=i}^{n}\left(\frac{1}{k^{2}}-\frac{1}{(k+1)^{2}}\right)\right. \\
& =2 \sum_{i=1}^{n} \frac{i^{2}}{a_{i}}\left(\frac{1}{i^{2}}-\frac{1}{(n+1)^{2}}\right)<2 \sum_{i=1}^{n} \frac{1}{a_{i}}
\end{aligned}
$$

从而命题成立.

例 6 设 $n \in \mathbf{Z}_{+}$, 求最小实数 $t=t(n)$, 使得对 $x_{i} \in \mathbf{R}, 1 \leqslant i \leqslant n$,
$$
\sum_{k=1}^{n}\left(x_{1}+x_{2}+\cdots+x_{k}\right)^{2} \leqslant t(n) \sum_{i=1}^{n} x_{i}^{2}
$$

解 令 $\alpha=\frac{\pi}{2 n+1}$, 则 $\sin n \alpha=\sin (n+1) \alpha$.

记 $c_{i}=\sin i \alpha-\sin (i-1) \alpha, 1 \leqslant i \leqslant n$. 则 $c_{i}>0,1 \leqslant i \leqslant n$.

令 $s_{k}=\sum_{i=1}^{k} c_{i}=\sin k \alpha, 1 \leqslant k \leqslant n$.

由柯西不等式得
$$
\left(x_{1}+\cdots+x_{k}\right)^{2} \leqslant\left(c_{1}+\cdots+c_{k}\right)\left(\frac{x_{1}^{2}}{c_{1}}+\cdots+\frac{x_{k}^{2}}{c_{k}}\right)
$$

于是
$$
\begin{aligned}
\sum_{k=1}^{n}\left(x_{1}+\cdots+x_{k}\right)^{2} & \leqslant \sum_{k=1}^{n} s_{k}\left(\frac{x_{1}^{2}}{c_{1}}+\cdots+\frac{x_{k}^{2}}{c_{k}}\right) \\
& =\sum_{k=1}^{n}\left(s_{k} \sum_{i=1}^{k} \frac{x_{i}^{2}}{c_{i}}\right)=\sum_{i=1}^{n} \sum_{k=i}^{n} s_{k} \frac{x_{i}^{2}}{c_{i}} \\
& =\sum_{i=1}^{n} \frac{s_{i}+s_{i+1}+\cdots+s_{n}}{c_{i}} x_{i}^{2}
\end{aligned}
$$

下面证明 $\quad \frac{s_{1}+\cdots+s_{n}}{c_{1}}=\frac{s_{2}+\cdots+s_{n}}{c_{2}}=\cdots=\frac{s_{n}}{c_{n}}=\frac{1}{4 \sin ^{2} \frac{\alpha}{2}}$.

事实上 $\quad \frac{s_{k}+\cdots+s_{n}}{c_{k}}=\frac{\sin k \alpha+\cdots+\sin n \alpha}{\sin k \alpha-\sin (k-1) \alpha}$
$$
\begin{aligned}
& =\frac{2 \sin \frac{\alpha}{2}(\sin k \alpha+\cdots+\sin n \alpha)}{2 \sin \frac{\alpha}{2} \cdot 2 \sin \frac{\alpha}{2} \cos \left(k-\frac{1}{2}\right) \alpha}=\frac{\cos \left(k-\frac{1}{2}\right) \alpha-\cos \left(n+\frac{1}{2}\right) \alpha}{4 \sin ^{2} \frac{\alpha}{2} \cos \left(k-\frac{1}{2}\right) \alpha} \\
& =\frac{\cos \left(k-\frac{1}{2}\right) \alpha}{4 \sin ^{2} \frac{\alpha}{2} \cos \left(k-\frac{1}{2}\right) \alpha}=\frac{1}{4 \sin ^{2} \frac{\alpha}{2}}
\end{aligned}
$$

且等式成立的充要条件是 $\quad \frac{x_{1}}{c_{1}}=\frac{x_{2}}{c_{2}}=\cdots=\frac{x_{n}}{c_{n}}$.

故 $t=t(n)=\frac{1}{4 \sin ^{2} \frac{\alpha}{2}}$.

例 7 设 $a_{i} \in\left[\frac{1}{\sqrt{3}}, \sqrt{3}\right], 1 \leqslant i \leqslant 6$. 求证:
$$
\sum_{i=1}^{6} \frac{a_{i}-a_{i+1}}{a_{i+1}+a_{i+2}} \geqslant 0
$$

其中 $a_{7}=a_{1}, a_{8}=a_{2}$.

证明 由于 $2 a_{i}+a_{i+2}-a_{i+1} \geqslant \frac{3}{\sqrt{3}}-\sqrt{3}=0$. 由柯西不等式得
$$
\sum_{i=1}^{6} \frac{2 a_{i}-a_{i+1}+a_{i+2}}{a_{i+1}+a_{i+2}} \geqslant \frac{\left(\sum_{i=1}^{6}\left(2 a_{i}-a_{i+1}+a_{i+2}\right)\right)^{2}}{\sum_{i=1}^{6}\left(2 a_{i}-a_{i+1}+a_{i+2}\right)\left(a_{i+1}+a_{i+2}\right)}
$$
$$
=\frac{2\left(\sum_{i=1}^{6} a_{i}\right)^{2}}{\sum_{i=1}^{6} a_{i} a_{i+1}+\sum_{i=1}^{6} a_{i} a_{i+2}}
$$
$$
\begin{gathered}
\text { 又因为 } \begin{aligned}
& \sum_{i=1}^{6} \frac{2 a_{i}-a_{i+1}+a_{i+2}}{a_{i+1}+a_{i+2}}=\sum_{i=1}^{6} \frac{2\left(a_{i}-a_{i+1}\right)+a_{i+1}+a_{i+2}}{a_{i+1}+a_{i+2}} \\
&=2 \sum_{i=1}^{6} \frac{a_{i}-a_{i+1}}{a_{i+1}+a_{i+2}}+6
\end{aligned} \\
2 \sum_{i=1}^{6} \frac{a_{i}-a_{i+1}}{a_{i+1}+a_{i+2}} \geqslant \frac{2\left(\sum_{i=1}^{6} a_{i}\right)^{2}}{\sum_{i=1}^{6} a_{i} a_{i+1}+\sum_{i=1}^{6} a_{i} a_{i+2}}-6 \\
=2 \frac{\left(\sum_{i=1}^{6} a_{i}\right)^{2}-3\left(\sum_{i=1}^{6} a_{i} a_{i+1}+\sum_{i=1}^{6} a_{i} a_{i+2}\right)}{\sum_{i=1}^{6} a_{i} a_{i+1}+\sum_{i=1}^{6} a_{i} a_{i+2}}
\end{gathered}
$$

所以

由于
$$
\begin{aligned}
& \left(\sum_{i=1}^{6} a_{i}\right)^{2} \geqslant 3\left(\sum_{i=1}^{6} a_{i} a_{i+1}+\sum_{i=1}^{6} a_{i} a_{i+2}\right) \\
\Leftrightarrow & \sum_{i=1}^{6} a_{i}^{2}+2 a_{1} a_{4}+2 a_{2} a_{5}+2 a_{3} a_{6} \geqslant \sum_{i=1}^{6} a_{i} a_{i+1}+\sum_{i=1}^{6} a_{i} a_{i+2} \\
\Leftrightarrow & \left(a_{1}+a_{4}\right)^{2}+\left(a_{2}+a_{5}\right)^{2}+\left(a_{3}+a_{6}\right)^{2} \\
\geqslant & \left(a_{1}+a_{4}\right)\left(a_{2}+a_{5}\right)+\left(a_{2}+a_{5}\right)\left(a_{3}+a_{6}\right) \\
& +\left(a_{3}+a_{6}\right)\left(a_{1}+a_{4}\right)
\end{aligned}
$$

利用平均值不等式, 最后不等式成立.

故命题成立.

例 8 设 $a 、 b 、 c 、 d$ 为正实数, 满足 $a b+c d=1, p_{i}\left(x_{i}, y_{i}\right), i=1,2$, 3,4 为 以原点为圆心的单位圆上的四点. 求证:
$$
\begin{aligned}
& \left(a y_{1}+b y_{2}+c y_{3}+d y_{4}\right)^{2}+\left(a x_{4}+b x_{3}+c x_{2}+d x_{1}\right)^{2} \\
\leqslant & 2\left(\frac{a^{2}+b^{2}}{a b}+\frac{c^{2}+d^{2}}{c d}\right) .
\end{aligned}
$$

证明 令 $\alpha=a y_{1}+b y_{2}+c y_{3}+d y_{4}, \beta=a x_{4}+b x_{3}+c x_{2}+d x_{1}$, 由柯西不等式, 得
$$
\begin{aligned}
\alpha^{2} & =\left(a y_{1}+b y_{2}+c y_{3}+d y_{4}\right)^{2} \\
& \leqslant\left[\left(\sqrt{a d} y_{1}\right)^{2}+\left(\sqrt{b c} y_{2}\right)^{2}+\left(\sqrt{b c} y_{3}\right)^{2}+\left(\sqrt{a d} y_{4}\right)^{2}\right]
\end{aligned}
$$
$$
\begin{aligned}
& \cdot\left[\left(\sqrt{\frac{a}{d}}\right)^{2}+\left(\sqrt{\frac{b}{c}}\right)^{2}+\left(\sqrt{\frac{c}{b}}\right)^{2}+\left(\sqrt{\frac{d}{a}}\right)^{2}\right] \\
= & \left(a d y_{1}^{2}+b c y_{2}^{2}+b c y_{3}^{2}+a d y_{4}^{2}\right)\left(\frac{a}{d}+\frac{b}{c}+\frac{c}{b}+\frac{d}{a}\right)
\end{aligned}
$$

同理, $\beta^{2} \leqslant\left(a d x_{4}^{2}+b c x_{3}^{2}+b c x_{2}^{2}+a d x_{1}^{2}\right)\left(\frac{d}{a}+\frac{c}{b}+\frac{b}{c}+\frac{a}{d}\right)$.

对它们相加, 并利用 $x_{i}^{2}+y_{i}^{2}=1, i=1,2,3,4, a b+c d=1$, 得
$$
\begin{aligned}
\alpha^{2}+\beta^{2} & \leqslant(2 a d+2 b c)\left(\frac{a}{d}+\frac{b}{c}+\frac{c}{b}+\frac{d}{a}\right) \\
& =2(a d+b c)\left(\frac{a b+c d}{b d}+\frac{a b+c d}{a c}\right) \\
& =2(a d+b c)\left(\frac{1}{b d}+\frac{1}{a c}\right) \\
& =2\left(\frac{a^{2}+b^{2}}{a b}+\frac{c^{2}+d^{2}}{c d}\right)
\end{aligned}
$$

故命题成立.

\section*{4. 6 利用平均值不等式与柯西不等式解题}
例 1 设 $a 、 b 、 c$ 为实数, 满足 $a^{2}+2 b^{2}+3 c^{2}=\frac{3}{2}$, 求证:
$$
3^{-a}+9^{-b}+27^{-c} \geqslant 1
$$

证明 由平均值不等式, 得
$$
3^{-a}+9^{-b}+27^{-c} \geqslant 3 \sqrt[3]{3^{-a-2 b-3 c}}=3^{\frac{3-a-2 b-3 c}{3}}
$$

再由柯西不等式, 得
$$
\begin{aligned}
(a+2 b+3 c)^{2} & =(a+\sqrt{2} \cdot \sqrt{2} b+\sqrt{3} \cdot \sqrt{3} c)^{2} \\
& \leqslant(1+2+3)\left(a^{2}+2 b^{2}+3 c^{2}\right) \\
& =6 \cdot \frac{3}{2}=9
\end{aligned}
$$

从而 $a+2 b+3 c \leqslant 3,3-a-2 b-3 c \geqslant 0,3^{\frac{3-a-2 b-3 c}{3}} \geqslant 3^{0}=1$. 故命题成立.

例 2 求 $x \sqrt{1-y^{2}}+y \sqrt{1-x^{2}}$ 的最大值.\\
解 由柯西不等式, 得
$$
\left|x \sqrt{1-y^{2}}+y \sqrt{1-x^{2}}\right|^{2} \leqslant\left(x^{2}+y^{2}\right)\left(2-x^{2}-y^{2}\right)
$$

再由平均值不等式, 得
$$
\left|x \sqrt{1-y^{2}}+y \sqrt{1-x^{2}}\right| \leqslant \frac{x^{2}+y^{2}+2-x^{2}-y^{2}}{2}=1
$$

若 $x=\frac{1}{2}, y=\frac{\sqrt{3}}{2}$, 则
$$
x \sqrt{1-y^{2}}+y \sqrt{1-x^{2}}=1
$$

于是所求的最大值为 1 .

例 3 设 $a 、 b 、 c$ 为正数, 且满足 $a b c=1$, 求证:
$$
\frac{1}{a^{3}(b+c)}+\frac{1}{b^{3}(a+c)}+\frac{1}{c^{3}(a+b)} \geqslant \frac{3}{2}
$$

证明 由柯西不等式, 得
$$
\begin{aligned}
& {\left[\frac{1}{a^{3}(b+c)}+\frac{1}{b^{3}(a+c)}+\frac{1}{c^{3}(a+b)}\right] \cdot[a(b+c)+b(a+c)+c(a+b)] } \\
\geqslant & \left(\frac{1}{a}+\frac{1}{b}+\frac{1}{c}\right)^{2}=(a b+b c+a c)^{2}
\end{aligned}
$$

所以由平均值不等式, 得
$$
\begin{aligned}
& \frac{1}{a^{3}(b+c)}+\frac{1}{b^{3}(a+c)}+\frac{1}{c^{3}(a+b)} \\
\geqslant & \frac{1}{2}(a b+b c+c a) \\
\geqslant & \frac{1}{2} \cdot 3 \cdot \sqrt[3]{a^{2} b^{2} c^{2}}=\frac{3}{2}
\end{aligned}
$$

例 4 设 $x_{i}, i=1,2, \cdots, n$ 为正数, 且满足 $\sum_{i=1}^{n} x_{i}=a, a \in \mathbf{R}_{+}, m, n \in$ $\mathbf{N}^{*}, n \geqslant 2$, 求证:
$$
\sum_{i=1}^{n} \frac{x_{i}^{m}}{a-x_{i}} \geqslant \frac{a^{m-1}}{(n-1) n^{m-2}}
$$

证明 当 $m=1$ 时, 即证明
$$
\sum_{i=1}^{n} \frac{x_{i}}{a-x_{i}} \geqslant \frac{n}{n-1}
$$

由于
$$
\sum_{i=1}^{n} \frac{x_{i}}{a-x_{i}}=\sum_{i=1}^{n}\left[\left(\frac{a}{a-x_{i}}\right)-1\right]=\sum_{i=1}^{n} \frac{a}{a-x_{i}}-n
$$

由柯西不等式, 得

即
$$
\begin{gathered}
\sum_{i=1}^{n} \frac{a}{a-x_{i}} \cdot \sum_{i=1}^{n}\left(a-x_{i}\right) \geqslant a n^{2}, \\
\sum_{i=1}^{n} \frac{a}{a-x_{i}} \geqslant \frac{a n^{2}}{\sum_{i=1}^{n}\left(a-x_{i}\right)}=\frac{a n^{2}}{(n-1) a}
\end{gathered}
$$

所以
$$
\sum_{i=1}^{n} \frac{x_{i}}{a-x_{i}} \geqslant \frac{a n^{2}}{n a-a}-n=\frac{n}{n-1}
$$

于是命题成立.

当 $m \geqslant 2$ 时, 由柯西不等式, 得
$$
\sum_{i=1}^{n} \frac{x_{i}^{m}}{a-x_{i}} \cdot \sum_{i=1}^{n}\left(a-x_{i}\right) \geqslant\left(\sum_{i=1}^{n} x_{i}^{\frac{m}{2}}\right)^{2}
$$

再由幂平均值不等式, 得
$$
\left(\frac{1}{n} \sum_{i=1}^{n} x_{t}^{\frac{m}{2}}\right)^{2} \geqslant\left[\frac{1}{n}\left(\sum_{i=1}^{n} x_{i}\right)^{\frac{m}{2}}\right]^{2}=\frac{a^{m}}{n^{m}}
$$

由于 $\sum_{i=1}^{n}\left(a-x_{i}\right)=(n-1) a$, 于是
$$
\sum_{i=1}^{n} \frac{x_{i}^{m}}{a-x_{i}} \geqslant \frac{a^{m-1}}{(n-1) n^{m-2}}
$$

例 5 设实数 $x_{i}$ 满足 $\left|x_{i}\right|<1(i=1,2, \cdots, n), n \geqslant 2$, 求证:
$$
\sum_{i=1}^{n} \frac{1}{1-\left|x_{i}\right|^{n}} \geqslant \frac{n}{1-\prod_{i=1}^{n} x_{i}}
$$

证明 由柯西不等式, 得
$$
\sum_{i=1}^{n} \frac{1}{1-\left|x_{i}\right|^{n}} \cdot \sum_{i=1}^{n}\left(1-\left|x_{i}\right|^{n}\right) \geqslant n^{2}
$$

因此欲证原不等式只要证明
$$
\frac{n^{2}}{\sum_{i=1}^{n}\left(1-\left|x_{i}\right|^{n}\right)} \geqslant \frac{n}{1-\prod_{i=1}^{n} x_{i}}
$$

即证
$$
n-n \prod_{i=1}^{n} x_{i} \geqslant \sum_{i=1}^{n}\left(1-\left|x_{i}\right|^{n}\right)
$$

即
$$
\sum_{i=1}^{n}\left|x_{i}\right|^{n} \geqslant n \prod_{i=1}^{n} x_{i}
$$

由平均值不等式知上述不等式成立, 故原命题成立.

例 6 已知正数 $x_{i}$ 满足 $\sum_{i=1}^{n} \frac{1}{1+x_{i}}=1$, 证明:
$$
\prod_{i=1}^{n} x_{i} \geqslant(n-1)^{n}
$$

证明 由柯西不等式, 得
$$
\sum_{i=1}^{n} \frac{1}{1+x_{i}} \cdot \sum_{i=1}^{n} \frac{1+x_{i}}{x_{i}} \geqslant\left(\sum_{i=1}^{n} \frac{1}{\sqrt{x_{i}}}\right)^{2}
$$

即
$$
\sum_{i=1}^{n} \frac{1}{x_{i}}+n \geqslant \sum_{i=1}^{n} \frac{1}{x_{i}}+2 \sum_{1 \leqslant i<j \leqslant n} \frac{1}{\sqrt{x_{i} x_{j}}}
$$

再由平均值不等式, 得
$$
n \geqslant 2 \sum_{1 \leqslant i<j \leqslant n} \frac{1}{\sqrt{x_{i} x_{j}}} \geqslant 2 \cdot \frac{n(n-1)}{2} \cdot \sqrt[\frac{n(n-1)}{2}]{\prod_{i=1}^{n}\left(\frac{1}{\sqrt{x_{i}}}\right)^{n-1}}
$$

由此得到
$$
\prod_{i=1}^{n} x_{i} \geqslant(n-1)^{n}
$$

例 7 设 $x, y, z \geqslant 0$, 且 $x^{2}+y^{2}+z^{2}=1$, 求证:
$$
\frac{x}{1-y z}+\frac{y}{1-x z}+\frac{z}{1-x y} \leqslant \frac{3 \sqrt{3}}{2} .
$$

证明 设 $S=\frac{x}{1-y z}+\frac{y}{1-x z}+\frac{z}{1-x y}$, 如果 $x=0$ (或 $y=0$ 或 $z=0$ ),则
$$
S=y+z<2<\frac{3}{2} \sqrt{3}
$$

所以设 $x y z \neq 0$, 使得 $x, y, z \in(0,1)$. 因为
$$
\frac{x}{1-y z}=x+\frac{z y x}{1-y z}
$$

所以
$$
S=x+y+z+x y z\left(\frac{1}{1-y z}+\frac{1}{1-z x}+\frac{1}{1-x y}\right)
$$

因为
$$
\begin{aligned}
1-y z & \geqslant 1-\frac{1}{2}\left(y^{2}+z^{2}\right) \\
& =\frac{1}{2}\left(1+x^{2}\right)=\frac{1}{2}\left(2 x^{2}+y^{2}+z^{2}\right) \\
& \geqslant 2 \sqrt[4]{x^{2} x^{2} y^{2} z^{2}}=2 x \sqrt{y z}
\end{aligned}
$$

由平均值不等式, 得
$$
\begin{aligned}
& x y z\left(\frac{1}{1-y z}+\frac{1}{1-z x}+\frac{1}{1-y x}\right) \\
\leqslant & \frac{x y z}{2}\left(\frac{1}{x \sqrt{y z}}+\frac{1}{y \sqrt{z x}}+\frac{1}{z \sqrt{x y}}\right) \\
= & \frac{1}{2}(\sqrt{y z}+\sqrt{z x}+\sqrt{x y}) \\
\leqslant & \frac{1}{2}\left(\frac{y+z}{2}+\frac{z+x}{2}+\frac{x+y}{2}\right)=\frac{1}{2}(x+y+z)
\end{aligned}
$$

再由柯西不等式, 得
$$
S \leqslant \frac{3}{2}(x+y+z) \leqslant \frac{3}{2}\left(1^{2}+1^{2}+1^{2}\right)^{\frac{1}{2}}\left(x^{2}+y^{2}+z^{2}\right)^{\frac{1}{2}}=\frac{3}{2} \sqrt{3}
$$

故命题成立.

例 8 设 $a, b, c>0$, 求证:
$$
\sum_{\text {cyc }} \sqrt{\frac{5 a^{2}+8 b^{2}+5 c^{2}}{4 a c}} \geqslant 3 \sqrt[9]{\frac{8(a+b)^{2}(b+c)^{2}(c+a)^{2}}{(a b c)^{2}}}
$$

证明 由柯西不等式及均值不等式有
$$
\begin{aligned}
5 a^{2}+8 b^{2}+5 c^{2} & >4\left(a^{2}+b^{2}\right)+4\left(b^{2}+c^{2}\right) \\
& \geqslant 2(a+b)^{2}+2(b+c)^{2} \\
& \geqslant 4(a+b)(b+c)
\end{aligned}
$$

所以
$$
\begin{aligned}
\sum_{\mathrm{cyc}} \sqrt{\frac{5 a^{2}+8 b^{2}+5 c^{2}}{4 a c}} & \geqslant \sum_{\text {cyc }} \sqrt{\frac{(a+b)(b+c)}{a c}} \\
& \geqslant 3 \sqrt[6]{\frac{(a+b)^{2}(b+c)^{2}(c+a)^{2}}{(a b c)^{2}}}
\end{aligned}
$$

只需证明
$$
\sqrt[6]{\frac{(a+b)^{2}(b+c)^{2}(c+a)^{2}}{(a b c)^{2}}} \geqslant \sqrt[9]{\frac{8(a+b)^{2}(b+c)^{2}(c+a)^{2}}{(a b c)^{2}}}
$$

等价于 $(a+b)(b+c)(c+a) \geqslant 8 a b c$, 即 $\sum_{\text {cyc }} a(b-c)^{2} \geqslant 0$, 明显成立.

例 9 已知数列 $\left\{a_{n}\right\}$ 满足 $a_{1}>0, a_{2}>0, a_{n+2}=\frac{2}{a_{n}+a_{n+1}} . M_{n}=$ $\max \left\{a_{n}, \frac{1}{a_{n}}, \frac{1}{a_{n+1}}, a_{n+1}\right\}$. 求证:
$$
M_{n+3} \leqslant \frac{3}{4} M_{n}+\frac{1}{4}
$$

证明 由于
$$
M_{n+3}=\max \left\{a_{n+3}, a_{n+4}, \frac{1}{a_{n+3}}, \frac{1}{a_{n+4}}\right\}
$$

我们需证
$$
\begin{aligned}
& a_{n+3} \leqslant \frac{3}{4} M_{n}+\frac{1}{4} \\
& a_{n+4} \leqslant \frac{3}{4} M_{n}+\frac{1}{4} \\
& \frac{1}{a_{n+3}} \leqslant \frac{3}{4} M_{n}+\frac{1}{4} \\
& \frac{1}{a_{n+4}} \leqslant \frac{3}{4} M_{n}+\frac{1}{4}
\end{aligned}
$$

由于
$$
\begin{aligned}
a_{n+3} & =\frac{2}{a_{n+1}+a_{n+2}} \leqslant \frac{\frac{1}{a_{n+1}}+\frac{1}{a_{n+2}}}{2} \\
& =\frac{1}{2}\left(\frac{1}{a_{n+1}}+\frac{a_{n}+a_{n+1}}{2}\right) \\
& =\frac{1}{4}\left(a_{n+1}+\frac{1}{a_{n+1}}\right)+\frac{1}{4} \cdot \frac{1}{a_{n+1}}+\frac{1}{4} a_{n}
\end{aligned}
$$
$$
\begin{aligned}
& \leqslant \frac{1}{4}\left[\min \left\{a_{n+1}, \frac{1}{a_{n+1}}\right\}+\max \left\{a_{n+1}, \frac{1}{a_{n+1}}\right\}\right]+\frac{1}{4} M_{n}+\frac{1}{4} M_{n} \\
& \leqslant \frac{1}{4}\left(1+M_{n}\right)+\frac{1}{2} M_{n} \\
&= \frac{3}{4} M_{n}+\frac{1}{4} ; \\
& \frac{1}{a_{n+3}}= \frac{a_{n+1}+a_{n+2}}{2}=\frac{1}{2} \cdot \frac{1}{a_{n+1}}+\frac{1}{a_{n}+a_{n+1}} \\
& \leqslant \frac{1}{2}+\frac{\frac{1}{a_{n}}+\frac{1}{a_{n+1}}}{4} \\
& \leqslant \frac{1}{4}\left(a_{n+1}+\frac{1}{a_{n+1}}\right)+\frac{1}{4}\left(\frac{1}{a_{n}}+\frac{1}{a_{n+1}}\right) \\
&= \frac{1}{4}\left[\max \left\{a_{n+1}, \frac{1}{a_{n+1}}\right\}+\min \left\{a_{n+1}, \frac{1}{a_{n+1}}\right\}\right]+\frac{1}{4}\left(\frac{1}{a_{n}}+\frac{1}{a_{n+1}}\right) \\
& \leqslant \frac{1}{4}\left(M_{n}+1\right)+\frac{1}{4} \cdot 2 M_{n} \\
&= \frac{3}{4} M_{n}+\frac{1}{4} ; \\
&+\frac{1}{8}\left[\max \left\{a_{n+1}, \frac{1}{a_{n+1}}\right\}+\min \left\{a_{n+1}, \frac{1}{a_{n+1}}\right\}\right]+\frac{1}{8} a_{n}+\frac{3}{8} a_{n+1} \\
&= \frac{1}{8}\left(M_{n}+1\right)+\frac{1}{8}\left(M_{n}+1\right)+\frac{1}{8} M_{n}+\frac{3}{8} M_{n} \\
&= \frac{3}{4} M_{n}+\frac{1}{4} ; \\
&= \frac{a_{n}+a_{n+1}}{4}+\frac{a_{n+1}+a_{n+2}}{4} \\
&= \frac{1}{4} a_{n}+\frac{1}{2} a_{n+1}+\frac{1}{2} \cdot \frac{1}{a_{n}+a_{n+1}} \\
& \leqslant \frac{1}{4} a_{n}+\frac{1}{2} a_{n+1}+\frac{1}{8}\left(\frac{1}{a_{n}}+\frac{1}{a_{n+1}}\right) \\
&= \frac{1}{8}\left(a_{n}+\frac{1}{a_{n}}\right)+\frac{1}{8}\left(a_{n+1}+\frac{1}{a_{n+1}}\right)+\frac{1}{8} a_{n}+\frac{3}{8} a_{n+1} \\
&\left.\max \left\{a_{n}, \frac{1}{a_{n}}\right\}+\min \left\{a_{n}, \frac{1}{a_{n}}\right\}\right] \\
& a_{n+4}
\end{aligned}
$$
$$
\begin{aligned}
\frac{1}{a_{n+4}}= & \frac{a_{n+2}+a_{n+3}}{2}=\frac{1}{a_{n}+a_{n+1}}+\frac{1}{a_{n+1}+a_{n+2}} \\
\leqslant & \frac{\frac{1}{a_{n}}+\frac{1}{a_{n+1}}}{4}+\frac{\frac{1}{a_{n+1}}+\frac{1}{a_{n+2}}}{4} \\
= & \frac{1}{4} \cdot \frac{1}{a_{n}}+\frac{1}{2} \cdot \frac{1}{a_{n+1}}+\frac{1}{4} \cdot \frac{1}{a_{n+2}} \\
= & \frac{1}{4} \cdot \frac{1}{a_{n}}+\frac{1}{2} \cdot \frac{1}{a_{n+1}}+\frac{1}{8}\left(a_{n}+a_{n+1}\right) \\
= & \frac{1}{8}\left(a_{n}+\frac{1}{a_{n}}\right)+\frac{1}{8}\left(a_{n+1}+\frac{1}{a_{n+1}}\right)+\frac{1}{8} \cdot \frac{1}{a_{n}}+\frac{3}{8} \cdot \frac{1}{a_{n+1}} \\
= & \frac{1}{8}\left[\max \left\{a_{n}, \frac{1}{a_{n}}\right\}+\min \left\{a_{n}, \frac{1}{a_{n}}\right\}\right] \\
& +\frac{1}{8}\left[\max \left\{a_{n+1}, \frac{1}{a_{n+1}}\right\}+\min \left\{a_{n+1}, \frac{1}{a_{n+1}}\right\}\right]+\frac{1}{8} \cdot \frac{1}{a_{n}}+\frac{3}{8} \cdot \frac{1}{a_{n+1}} \\
\leqslant & \frac{1}{8}\left(M_{n}+1\right)+\frac{1}{8}\left(M_{n}+1\right)+\frac{1}{8} M_{n}+\frac{3}{8} M_{n} \\
= & \frac{3}{4} M_{n}+\frac{1}{4} .
\end{aligned}
$$

因此, $M_{n+3} \leqslant \frac{3}{4} M_{n}+\frac{1}{4}$.

注 当 $x, y>0$ 时, $x+y=\max \{x, y\}+\min \{x, y\}$; 当 $x>0$ 时, $\min \left\{x, \frac{1}{x}\right\} \leqslant 1$.

例 10 已知正实数 $x 、 y 、 z$ 满足 $\sqrt{x}+\sqrt{y}+\sqrt{z}=1$. 求证:
$$
\frac{x^{2}+y z}{\sqrt{2 x^{2}(y+z)}}+\frac{y^{2}+z x}{\sqrt{2 y^{2}(z+x)}}+\frac{z^{2}+x y}{\sqrt{2 z^{2}(x+y)}} \geqslant 1
$$

证法 1 注意到
$$
\begin{aligned}
\frac{x^{2}+y z}{\sqrt{2 x^{2}(y+z)}} & =\frac{x^{2}-x(y+z)+y z}{\sqrt{2 x^{2}(y+z)}}+\frac{x(y+z)}{\sqrt{2 x^{2}(y+z)}} \\
& =\frac{(x-y)(x-z)}{\sqrt{2 x^{2}(y+z)}}+\sqrt{\frac{y+z}{2}} \\
& \geqslant \frac{(x-y)(x-z)}{\sqrt{2 x^{2}(y+z)}}+\frac{\sqrt{y}+\sqrt{z}}{2}
\end{aligned}
$$

同理,
$$
\begin{aligned}
& \frac{y^{2}+z x}{\sqrt{2 y^{2}(z+x)}} \geqslant \frac{(y-z)(y-x)}{\sqrt{2 y^{2}(z+x)}}+\frac{\sqrt{z}+\sqrt{x}}{2} \\
& \frac{z^{2}+x y}{\sqrt{2 z^{2}(x+y)}} \geqslant \frac{(z-x)(z-y)}{\sqrt{2 z^{2}(x+y)}}+\frac{\sqrt{x}+\sqrt{y}}{2}
\end{aligned}
$$

以上三式相加得
$$
\begin{aligned}
& \frac{x^{2}+y z}{\sqrt{2 x^{2}(y+z)}}+\frac{y^{2}+z x}{\sqrt{2 y^{2}(z+x)}}+\frac{z^{2}+x y}{\sqrt{2 z^{2}(x+y)}} \\
\geqslant & \frac{(x-y)(x-z)}{\sqrt{2 x^{2}(y+z)}}+\frac{(y-z)(y-x)}{\sqrt{2 y^{2}(z+x)}}+\frac{(z-x)(z-y)}{\sqrt{2 z^{2}(x+y)}}+\sqrt{x}+\sqrt{y}+\sqrt{z} \\
= & \frac{(x-y)(x-z)}{\sqrt{2 x^{2}(y+z)}}+\frac{(y-z)(y-x)}{\sqrt{2 y^{2}(z+x)}}+\frac{(z-x)(z-y)}{\sqrt{2 z^{2}(x+y)}}+1
\end{aligned}
$$

从而, 只需证明
$$
\frac{(x-y)(x-z)}{\sqrt{2 x^{2}(y+z)}}+\frac{(y-z)(y-x)}{\sqrt{2 y^{2}(z+x)}}+\frac{(z-x)(z-y)}{\sqrt{2 z^{2}(x+y)}} \geqslant 0
$$

不失一般性,设 $x \geqslant y \geqslant z$. 于是,
$$
\frac{(x-y)(x-z)}{\sqrt{2 x^{2}(y+z)}} \geqslant 0
$$

且


\begin{align*}
& \frac{(y-z)(y-x)}{\sqrt{2 y^{2}(z+x)}}+\frac{(z-x)(z-y)}{\sqrt{2 z^{2}(x+y)}} \\
= & \frac{(y-z)(x-z)}{\sqrt{2 z^{2}(x+y)}}-\frac{(y-z)(x-y)}{\sqrt{2 y^{2}(z+x)}} \\
\geqslant & \frac{(y-z)(x-y)}{\sqrt{2 z^{2}(x+y)}}-\frac{(y-z)(x-y)}{\sqrt{2 y^{2}(z+x)}} \\
= & (y-z)(x-y) \cdot\left[\frac{1}{\sqrt{2 z^{2}(x+y)}}-\frac{1}{\sqrt{2 y^{2}(z+x)}}\right] \tag{1}
\end{align*}


事实上,由
$$
y^{2}(z+x)=y^{2} z+y^{2} x \geqslant y z^{2}+z^{2} x=z^{2}(x+y)
$$

可知式(1)非负.

从而, 题中不等式成立.

证法 2 根据柯西不等式得
$$
\begin{aligned}
& {\left[\frac{x^{2}}{\sqrt{2 x^{2}(y+z)}}+\frac{y^{2}}{\sqrt{2 y^{2}(z+x)}}+\frac{z^{2}}{\sqrt{2 z^{2}(x+y)}}\right] } \\
& {[\sqrt{2(y+z)}+\sqrt{2(z+x)}+\sqrt{2(x+y)}] } \\
\geqslant & (\sqrt{x}+\sqrt{y}+\sqrt{z})^{2}=1
\end{aligned}
$$

和
$$
\begin{aligned}
& {\left[\frac{y z}{\sqrt{2 x^{2}(y+z)}}+\frac{z x}{\sqrt{2 y^{2}(z+x)}}+\frac{x y}{\sqrt{2 z^{2}(x+y)}}\right] } \\
& {[\sqrt{2(y+z)}+\sqrt{2(z+x)}+\sqrt{2(x+y)}] } \\
\geqslant & \left(\sqrt{\frac{y z}{x}}+\sqrt{\frac{z x}{y}}+\sqrt{\frac{x y}{z}}\right)^{2}
\end{aligned}
$$

以上两式相加得
$$
\begin{aligned}
& {\left[\frac{x^{2}+y z}{\sqrt{2 x^{2}(y+z)}}+\frac{y^{2}+z x}{\sqrt{2 y^{2}(z+x)}}+\frac{z^{2}+x y}{\sqrt{2 z^{2}(x+y)}}\right] } \\
& {[\sqrt{2(y+z)}+\sqrt{2(z+x)}+\sqrt{2(x+y)}] } \\
\geqslant & 1+\left(\sqrt{\frac{y z}{x}}+\sqrt{\frac{z x}{y}}+\sqrt{\frac{x y}{z}}\right)^{2} \\
\geqslant & 2\left(\sqrt{\frac{y z}{x}}+\sqrt{\frac{z x}{y}}+\sqrt{\frac{x y}{z}}\right)
\end{aligned}
$$

从而, 只需证明
$$
2\left(\sqrt{\frac{y z}{x}}+\sqrt{\frac{z x}{y}}+\sqrt{\frac{x y}{z}}\right) \geqslant \sqrt{2(y+z)}+\sqrt{2(z+x)}+\sqrt{2(x+y)}
$$

根据均值不等式得
$$
\begin{aligned}
& {\left[\sqrt{\frac{y z}{x}}+\left(\frac{1}{2} \sqrt{\frac{z x}{y}}+\frac{1}{2} \sqrt{\frac{x y}{z}}\right)\right]^{2} } \\
& \geqslant 4 \sqrt{\frac{y z}{x}}\left(\frac{1}{2} \sqrt{\frac{z x}{y}}+\frac{1}{2} \sqrt{\frac{x y}{z}}\right)=2(y+z) \\
& \sqrt{\frac{y z}{x}}+\left(\frac{1}{2} \sqrt{\frac{z x}{y}}+\frac{1}{2} \sqrt{\frac{x y}{2}}\right) \geqslant \sqrt{2(y+z)}
\end{aligned}
$$

即

同理,
$$
\begin{aligned}
& \sqrt{\frac{z x}{y}}+\left(\frac{1}{2} \sqrt{\frac{x y}{z}}+\frac{1}{2} \sqrt{\frac{y z}{x}}\right) \geqslant \sqrt{2(z+x)} \\
& \sqrt{\frac{x y}{2}}+\left(\frac{1}{2} \sqrt{\frac{y z}{x}}+\frac{1}{2} \sqrt{\frac{z x}{y}}\right) \geqslant \sqrt{2(x+y)}
\end{aligned}
$$

以上三式相加得
$$
2\left(\sqrt{\frac{y z}{x}}+\sqrt{\frac{z x}{y}}+\sqrt{\frac{x y}{z}}\right) \geqslant \sqrt{2(y+z)}+\sqrt{2(z+x)}+\sqrt{2(x+y)}
$$

从而, 题中不等式成立.

例 11 设正整数 $n \geqslant 2$. 求常数 $C(n)$ 的最大值, 使得对于所有满足 $x_{i} \in$ $(0,1)(i=1,2, \cdots, n)$, 且 $\left(1-x_{i}\right)\left(1-x_{j}\right) \geqslant \frac{1}{4}(1 \leqslant i<j \leqslant n)$ 的实数 $x_{1}$, $x_{2}, \cdots, x_{n}$, 均有


\begin{equation*}
\sum_{i=1}^{n} x_{i} \geqslant C(n) \sum_{1 \leqslant i<j \leqslant n}\left(2 x_{i} x_{j}+\sqrt{x_{i} x_{j}}\right) \tag{1}
\end{equation*}


解 首先, 取 $x_{i}=\frac{1}{2}(i=1,2, \cdots, n)$. 代人式(1)有
$$
\frac{n}{2} \geqslant C(n) \mathrm{C}_{n}^{2}\left(\frac{1}{2}+\frac{1}{2}\right)
$$

于是, $C(n) \leqslant \frac{1}{n-1}$.

下面证明: $C(n)=\frac{1}{n-1}$ 满足条件.

由 $1-x_{i}+1-x_{j} \geqslant 2 \sqrt{\left(1-x_{i}\right)\left(1-x_{j}\right)} \geqslant 1(1 \leqslant i<j \leqslant n)$, 得 $x_{i}+$ $x_{j} \leqslant 1$.

取和得 $(n-1) \sum_{k=1}^{n} x_{k} \leqslant \mathrm{C}_{n}^{2}$, 即 $\sum_{k=1}^{n} x_{k} \leqslant \frac{n}{2}$.

故
$$
\begin{aligned}
& \frac{1}{n-1} \sum_{1 \leqslant i<j \leqslant n}\left(2 x_{i} x_{j}+\sqrt{x_{i} x_{j}}\right) \\
= & \frac{1}{n-1}\left(2 \sum_{1 \leqslant i<j \leqslant n} x_{i} x_{j}+\sum_{1 \leqslant i<j \leqslant n} \sqrt{x_{i} x_{j}}\right) \\
= & \frac{1}{n-1}\left[\left(\sum_{k=1}^{n} x_{k}\right)^{2}-\sum_{k=1}^{n} x_{k}^{2}+\sum_{1 \leqslant i<j \leqslant n} \sqrt{x_{i} x_{j}}\right] \\
\leqslant & \frac{1}{n-1}\left[\left(\sum_{k=1}^{n} x_{k}\right)^{2}-\frac{1}{n}\left(\sum_{k=1}^{n} x_{k}\right)^{2}+\sum_{1 \leqslant i<j \leqslant n} \frac{x_{i}+x_{j}}{2}\right] \\
= & \frac{1}{n-1}\left[\frac{n-1}{n}\left(\sum_{k=1}^{n} x_{k}\right)^{2}+\frac{n-1}{2} \sum_{k=1}^{n} x_{k}\right] \\
= & \frac{1}{n}\left(\sum_{k=1}^{n} x_{k}\right)^{2}+\frac{1}{2} \sum_{k=1}^{n} x_{k}
\end{aligned}
$$
$$
\begin{aligned}
& \leqslant \frac{1}{n}\left(\sum_{k=1}^{n} x_{k}\right) \cdot \frac{n}{2}+\frac{1}{2} \sum_{k=1}^{n} x_{k} \\
& =\sum_{k=1}^{n} x_{k}
\end{aligned}
$$

从而,原不等式成立.

因此, $C(n)$ 的最大值为 $\frac{1}{n-1}$.

例 12 给定整数 $n \geqslant 2$ 和正实数 $a$, 正实数 $x_{1}, x_{2}, \cdots, x_{n}$ 满足 $x_{1} x_{2} \cdots$ $x_{n}=1$. 求最小的实数 $M=M(n, a)$, 使得
$$
\sum_{i=1}^{n} \frac{1}{a+S-x_{i}} \leqslant M
$$

恒成立,其中 $S=x_{1}+x_{2}+\cdots+x_{n}$.

解 首先考虑 $a \geqslant 1$ 的情况, 令 $x_{i}=y_{i}^{n}, y_{i}>0$, 于是 $y_{1} y_{2} \cdots y_{n}=1$, 我们有
$$
\begin{aligned}
& S-x_{i}=\sum_{j \neq i} y_{j}^{n} \geqslant(n-1)\left(\frac{\sum_{j \neq i} y_{j}}{n-1}\right)^{n}(\text { 幂平均不等式 }) \\
& \geqslant(n-1)\left(\frac{\sum_{j \neq i} y_{j}}{n-1}\right) \cdot \prod_{j \neq i} y_{j}(\text { 算术平均 } \geqslant \text { 几何平均 }) \\
& =\frac{\sum_{j \neq i} y_{j}}{y_{i}} .
\end{aligned}
$$

于是


\begin{equation*}
\sum_{i=1}^{n} \frac{1}{a+S-x_{i}} \leqslant \sum_{i=1}^{n} \frac{y_{i}}{a y_{i}+\sum_{j \neq i} y_{j}} \tag{1}
\end{equation*}


当 $a=1$ 时,
$$
\sum_{i=1}^{n} \frac{y_{i}}{a y_{i}+\sum_{j \neq i} y_{j}}=\sum_{i=1}^{n} \frac{y_{i}}{\sum_{j=1}^{n} y_{j}}=1
$$

且当 $x_{1}=x_{2}=\cdots=x_{n}=1$ 时, $\sum_{i=1}^{n} \frac{1}{a+S-x_{i}}=1$, 此时 $M=1$.\\
下面假设 $a>1$. 令 $z_{i}=\frac{y_{i}}{\sum_{j=1}^{n} y_{j}}, i=1,2, \cdots, n$, 有 $\sum_{i=1}^{n} z_{i}=1$.


\begin{align*}
\frac{y_{i}}{a y_{i}+\sum_{j \neq i} y_{j}} & =\frac{y_{i}}{(a-1) y_{i}+\sum_{j=1}^{n} y_{j}}  \tag{2}\\
& =\frac{z_{i}}{(a-1) z_{i}+1} \\
& =\frac{1}{a-1}\left[1-\frac{1}{(a-1) z_{i}+1}\right]
\end{align*}


由柯西不等式
$$
\sum_{i=1}^{n}\left[(a-1) z_{i}+1\right] \cdot \sum_{i=1}^{n} \frac{1}{(a-1) z_{i}+1} \geqslant n^{2}
$$

而
$$
\sum_{i=1}^{n}\left[(a-1) z_{i}+1\right]=a-1+n
$$

故


\begin{equation*}
\sum_{i=1}^{n} \frac{1}{(a-1) z_{i}+1} \geqslant \frac{n^{2}}{a-1+n} \tag{3}
\end{equation*}


结合(1)、(2)、(3), 我们有
$$
\begin{aligned}
\sum_{i=1}^{n} \frac{1}{a+S-x_{i}} & \leqslant \sum_{i=1}^{n}\left[\frac{1}{a-1}\left(1-\frac{1}{(a-1) z_{i}+1}\right)\right] \\
& \leqslant \frac{n}{a-1}-\frac{1}{a-1} \cdot \frac{n^{2}}{a-1+n} \\
& =\frac{n}{a-1+n}
\end{aligned}
$$

当 $x_{1}=x_{2}=\cdots=x_{n}=1$ 时, 有
$$
\sum_{i=1}^{n} \frac{1}{a+S-x_{i}}=\frac{n}{a-1+n}
$$

故 $M=\frac{n}{a-1+n}$.

下面考虑 $a<1$ 的情况: 对任何常数 $\lambda>0$, 函数
$$
f(x)=\frac{x}{x+\lambda}=1-\frac{\lambda}{x+\lambda}
$$

在区间 $(0,+\infty)$ 上严格单调递增, 故 $f(a)<f(1)$, 即 $\frac{a}{a+\lambda}<\frac{1}{1+\lambda}$. 于是由 $a=1$ 时的结论,
$$
\sum_{i=1}^{n} \frac{1}{a+S-x_{i}}=\frac{1}{a} \sum_{i=1}^{n} \frac{a}{a+S-x_{i}}<\frac{1}{a} \sum_{i=1}^{n} \frac{1}{1+S-x_{i}} \leqslant \frac{1}{a}
$$

当 $x_{1}=x_{2}=\cdots=x_{n-1}=\varepsilon \rightarrow 0^{+}$, 而 $x_{n}=\varepsilon^{1-n} \rightarrow+\infty$ 时,
$$
\begin{aligned}
& \lim _{\varepsilon \rightarrow 0^{+}} \sum_{i=1}^{n} \frac{1}{a+S-x_{i}} \\
&= \lim _{\varepsilon \rightarrow 0^{+}}\left[\frac{n-1}{a+\varepsilon^{1-n}+(n-2) \varepsilon}+\frac{1}{a+(n-1) \varepsilon}\right] \\
&=\frac{1}{a}
\end{aligned}
$$

故 $M=\frac{1}{a}$, 综上所述,
$$
M= \begin{cases}\frac{n}{a-1+n}, & \text { 若 } a \geqslant 1 \\ \frac{1}{a}, & \text { 若 } 0<a<1\end{cases}
$$

\section*{习 题 4}
1 设 $a, b, c \in \mathbf{R}_{+}$满足 $a b c=1$, 求证:
$$
\frac{a^{5}}{a^{3}+1}+\frac{b^{5}}{b^{3}+1}+\frac{c^{5}}{c^{3}+1} \geqslant \frac{3}{2}
$$

2 已知 $a 、 b 、 c$ 为正数,证明:
$$
\frac{a^{3}+3 b^{3}}{5 a+b}+\frac{b^{3}+3 c^{3}}{5 b+c}+\frac{c^{3}+3 a^{3}}{5 c+a} \geqslant \frac{2}{3}\left(a^{2}+b^{2}+c^{2}\right)
$$

3 已知: $a 、 b 、 c 、 d>0, a+b+c+d=3$, 求证:
$$
\frac{a}{10-3 a}+\frac{b}{10-3 b}+\frac{c}{10-3 c}+\frac{d}{10-3 d} \geqslant \frac{12}{31}
$$

4 设 $a 、 b 、 c$ 为某三角形三边之长. 求证:
$$
\frac{a+b}{b+c-a}+\frac{b+c}{c+a-b}+\frac{c+a}{a+b-c} \geqslant 6
$$

5 设 $a_{1}, a_{2}, \cdots, a_{n}$ 为大于等于 1 的实数, $n \geqslant 1, A=1+a_{1}+a_{2}+\cdots+$ $a_{n}$. 定义 $x_{0}=1, x_{k}=\frac{1}{1+a_{k} x_{k-1}}, 1 \leqslant k \leqslant n$. 证明:
$$
x_{1}+x_{2}+\cdots+x_{n}>\frac{n^{2} A}{n^{2}+A^{2}}
$$

6 设 $x, y, z \in \mathbf{R}_{+}$, 且 $x+y+z \geqslant 6$. 求
$$
M=\sum_{\text {cyc }} x^{2}+\sum_{\text {cyc }} \frac{x}{y^{2}+z+1}
$$

的最小值, 其中, “ $\sum_{\text {cyc }} ”$ 表示轮换对称和.

7 设 $x, y, z \in \mathbf{R}_{+}$. 证明:
$$
\sum_{\text {cyc }} \frac{x}{\sqrt{2\left(x^{2}+y^{2}\right)}}<\sum_{\text {cyc }} \frac{4 x^{2}+y^{2}}{x^{2}+4 y^{2}}<9
$$

其中, “ $\sum_{\text {cyc }} ”$ 表示轮换对称和.

8 已知 $\lambda$ 为正实数. 求 $\lambda$ 的最大值, 使得对于所有满足条件
$$
u \sqrt{v w}+v \sqrt{w u}+w \sqrt{u v} \geqslant 1
$$

的正实数 $u 、 v 、 w$, 均有 $u+v+w \geqslant \lambda$.

9 设 $x_{1}, x_{2}, \cdots, x_{n}$ 为正实数, $x_{n+1}=x_{1}+x_{2}+\cdots+x_{n}$, 证明:
$$
x_{n+1} \sum_{i=1}^{n}\left(x_{n+1}-x_{i}\right) \geqslant\left(\sum_{i=1}^{n} \sqrt{x_{i}\left(x_{n+1}-x_{i}\right)}\right)^{2}
$$

10 设 $x, y, z, w \in \mathbf{R}_{+}, \alpha 、 \beta 、 \gamma 、 \theta$ 满足 $\alpha+\beta+\gamma+\theta=(2 k+1) \pi, k \in \mathbf{Z}$.求证:

$(x \sin \alpha+y \sin \beta+z \sin \gamma+w \sin \theta)^{2} \leqslant \frac{(x y+z w)(x z+y w)(x w+y z)}{x y z w}$,

当且仅当 $x \cos \alpha=y \cos \beta=z \cos \gamma=w \cos \theta$ 时等号成立.

11 已知 $0<a_{1}<a_{2}<\cdots<a_{n}$, 对于 $a_{1}, a_{2}, \cdots, a_{n}$ 的任意排列 $b_{1}, b_{2}, \cdots$, $b_{n}$. 令 $M=\prod_{i=1}^{n}\left(a_{i}+\frac{1}{b_{i}}\right)$, 求使 $M$ 取值最大的排列 $b_{1}, b_{2}, \cdots, b_{n}$.

12 设复数 $z_{k}=x_{k}+\mathrm{i} y_{k}, k=1,2, \cdots, n, x_{i}$ 和 $y_{i}$ 为实数, $\mathrm{i}=\sqrt{-1}$. 令 $r$表示 $\sqrt{z_{1}^{2}+z_{2}^{2}+\cdots+z_{n}^{2}}$ 的实部的绝对值,求证:
$$
r \leqslant\left|x_{1}\right|+\left|x_{2}\right|+\cdots+\left|x_{n}\right|
$$

13 设 $A_{n}=\frac{a_{1}+a_{2}+\cdots+a_{n}}{n}, a_{i}>0, i=1,2, \cdots, n$. 求证:
$$
\left(A_{n}-\frac{1}{A_{n}}\right)^{2} \leqslant \frac{1}{n} \sum_{i=1}^{n}\left(a_{i}-\frac{1}{a_{i}}\right)^{2}
$$

14 对满足 $1 \leqslant r \leqslant s \leqslant t$ 的一切实数 $r 、 s 、 t$. 求
$$
w=(r-1)^{2}+\left(\frac{s}{r}-1\right)^{2}+\left(\frac{t}{s}-1\right)^{2}+\left(\frac{4}{t}-1\right)^{2}
$$

的最小值.

15 设 $a_{i}>0, b_{i}>0, a_{i} b_{i}-c_{i}^{2}>0(i=1,2, \cdots, n)$, 则
$$
\frac{n^{3}}{\left(\sum_{i=1}^{n} a_{i}\right)\left(\sum_{i=1}^{n} b_{i}\right)-\left(\sum_{i=1}^{n} c_{i}\right)^{2}} \leqslant \sum_{i=1}^{n} \frac{1}{a_{i} b_{i}-c_{i}^{2}}
$$

16 设 $\frac{1}{2} \leqslant p \leqslant 1, a_{i} \geqslant 0,0 \leqslant b_{i} \leqslant p$ 且 $\sum_{i=1}^{n} a_{i}=\sum_{i=1}^{n} b_{i}=1$, 求证:
$$
\sum_{i=1}^{n} b_{i} \prod_{\substack{1 \leqslant j \leqslant n \\ j \neq i}} a_{j} \leqslant \frac{p}{(n-1)^{n-1}}
$$

17 已给自然数 $n \geqslant 2$, 求最小正数 $\lambda$, 使得对任意正数 $a_{1}, a_{2}, \cdots, a_{n}$, 及 $\left[0, \frac{1}{2}\right]$ 中任意 $n$ 个数 $b_{1}, b_{2}, \cdots, b_{n}$, 只要

就有
$$
\begin{aligned}
& \sum_{i=1}^{n} a_{i}=\sum_{i=1}^{n} b_{i}=1 \\
& \prod_{i=1}^{n} a_{i} \leqslant \lambda \sum_{i=1}^{n} a_{i} b_{i}
\end{aligned}
$$

18 已给两个大于 1 的自然数 $n$ 和 $m$, 求所有的自然数 $l$, 使得对任意正数 $a_{1}$, $a_{2}, \cdots, a_{n}$, 都有
$$
\sum_{k=1}^{n} \frac{1}{S_{k}}\left(l k+\frac{1}{4} l^{2}\right)<m^{2} \sum_{k=1}^{n} \frac{1}{a_{k}}
$$

其中 $S_{k}=\sum_{i=1}^{k} a_{i}$.

19 设 $u 、 v$ 是正实数, 对于给定的正整数 $n$, 求: $u 、 v$ 满足的充分必要条件, 使\\
得存在实数 $a_{1} \geqslant a_{2} \geqslant \cdots \geqslant a_{n}>0$ 满足
$$
\sum_{i=1}^{n} a_{i}=u, \sum_{i=1}^{n} a_{i}^{2}=v
$$

当这些数存在时,求 $a_{1}$ 的最大值与最小值.

20 给定实数 $r \in(0,1)$. 证明: 若 $n$ 个复数 $z_{1}, z_{2}, \cdots, z_{n}$ 满足 $\left|z_{k}-1\right|=$ $r, 1 \leqslant k \leqslant n$, 则
$$
\left|z_{1}+\cdots+z_{n}\right| \cdot\left|\frac{1}{z_{1}}+\cdots+\frac{1}{z_{n}}\right| \geqslant n^{2}\left(1-r^{2}\right)
$$

21 设 $a_{i} \in \mathbf{R}, 1 \leqslant i \leqslant n$ 满足 $\sum_{i=1}^{n}\left|a_{i}\right|+\left|\sum_{i=1}^{n} a_{i}\right|=1 . n \geqslant 2$.

求 $\sum_{i=1}^{n} a_{i}^{2}$ 的最大值和最小值.

22 设 $a_{i}>i, 1 \leqslant i \leqslant n, n \geqslant 2$. 求证
$$
\frac{\left(a_{1}+\cdots+a_{n}\right)^{2}}{\sqrt{a_{1}^{2}-1}+\sqrt{a_{2}^{2}-2^{2}}+\cdots+\sqrt{a_{n}^{2}-n^{2}}} \geqslant n(n+1)
$$

23 求最大实数 $\lambda$, 使得对任意 $0=x_{0}<x_{1}<x_{2}<\cdots<x_{n}, n \geqslant 1$, 有
$$
\frac{1}{x_{1}-x_{0}}+\frac{1}{x_{2}-x_{1}}+\cdots+\frac{1}{x_{n}-x_{n-1}} \geqslant \lambda\left(\frac{2}{x_{1}}+\frac{3}{x_{2}}+\cdots+\frac{n+1}{x_{n}}\right)
$$

24 设 $x_{i} \in \mathbf{R}(i=1,2, \cdots, n)$ 且 $\sum_{i=1}^{n} x_{i}^{2}=1$, 求证: 对任一整数 $k \geqslant 3$ 存在不全为零的整数 $a_{i},\left|a_{i}\right| \leqslant k-1$. 使得
$$
\left|\sum_{i=1}^{n} a_{i} x_{i}\right| \leqslant \frac{(k-1) \sqrt{n}}{k^{n}-1}
$$

25 设 $s, t, u, v \in\left(0, \frac{\pi}{2}\right)$, 满足 $s+t+u+v=\pi$, 证明:
$$
\frac{\sqrt{2} \sin s-1}{\cos s}+\frac{\sqrt{2} \sin t-1}{\cos t}+\frac{\sqrt{2} \sin u-1}{\cos u}+\frac{\sqrt{2} \sin v-1}{\cos v} \geqslant 0
$$

26 证明:
$$
\sqrt{\frac{A B_{1}}{A B}}+\sqrt{\frac{B C_{1}}{B C}}+\sqrt{\frac{C A_{1}}{C A}} \leqslant \frac{3}{\sqrt{2}}
$$

其中, $A_{1} 、 B_{1} 、 C_{1}$ 分别为 $\triangle A B C$ 的内切圆与边 $B C 、 A C 、 A B$ 的切点.\\
$27 a_{1} 、 a_{2} 、 a_{3} 、 a_{4}$ 是周长为 $2 s$ 的四边形的四边长, 证明:
$$
\sum_{i=1}^{4} \frac{1}{a_{i}+s} \leqslant \frac{2}{9} \sum_{1 \leqslant i<j \leqslant 4} \frac{1}{\sqrt{\left(s-a_{i}\right)\left(s-a_{j}\right)}}
$$

28 设 $a 、 b 、 c$ 是一个三角形的三边长. 证明:
$$
\frac{\sqrt{b+c-a}}{\sqrt{b}+\sqrt{c}-\sqrt{a}}+\frac{\sqrt{c+a-b}}{\sqrt{c}+\sqrt{a}-\sqrt{b}}+\frac{\sqrt{a+b-c}}{\sqrt{a}+\sqrt{b}-\sqrt{c}} \leqslant 3
$$

29 已知 $a, b, c>0, k$ 为正常数, 求
$$
\frac{a}{a+k b}+\frac{b}{b+k c}+\frac{c}{c+k a}
$$

的取值范围.

\section*{习 题 1}
\begin{enumerate}
  \item 由平均值不等式, $\sqrt{a}+\sqrt{b}+\sqrt{c}=\sqrt{\frac{1}{b c}}+\sqrt{\frac{1}{c a}}+\sqrt{\frac{1}{a b}} \leqslant$ $\frac{1}{2}\left(\frac{1}{b}+\frac{1}{c}\right)+\frac{1}{2}\left(\frac{1}{a}+\frac{1}{c}\right)+\frac{1}{2}\left(\frac{1}{a}+\frac{1}{b}\right)=\frac{1}{a}+\frac{1}{b}+\frac{1}{c}$.

  \item 由于 $\frac{(a+b)^{3}(b+c)^{2}(c+a)}{(a+b+c)^{6}}=108 \cdot \frac{\left(\frac{a+b}{3}\right)^{3}\left(\frac{b+c}{2}\right)^{2}(c+a)}{(a+b+c)^{6}} \leqslant$ $108 \cdot \frac{\left(\frac{a+b+c}{3}\right)^{6}}{(a+b+c)^{6}}=\frac{4}{27}$. 且当 $c=0, b=2 a>0$ 时, 等式成立.

  \item 设 $A=a\left(1-a^{2}\right)$, 则 $A^{2}=\frac{1}{2} \cdot 2 a^{2}\left(1-a^{2}\right)\left(1-a^{2}\right) \leqslant \frac{4}{27}$, 则 $A \leqslant$ $\frac{2}{3 \sqrt{3}}$, 所以 $\frac{a}{1-a^{2}}=\frac{a^{2}}{a\left(1-a^{2}\right)} \geqslant \frac{3 \sqrt{3}}{2} a^{2}$, 同理可得其他二式, 则左边 $\geqslant$ $\frac{3 \sqrt{3}}{2}\left(a^{2}+b^{2}+c^{2}\right) \geqslant \frac{3 \sqrt{3}}{2}(a b+b c+c a)=\frac{3 \sqrt{3}}{2}$.

  \item 由于 $a=\sqrt[4]{\frac{a^{4}}{a b c d}}=\sqrt[4]{\frac{a}{b} \cdot \frac{a}{b} \cdot \frac{b}{c} \cdot \frac{a}{d}} \leqslant \frac{1}{4}\left(\frac{a}{b}+\frac{a}{b}+\frac{b}{c}+\frac{a}{d}\right)$, 同理可得 $b \leqslant \frac{1}{4}\left(\frac{b}{c}+\frac{b}{c}+\frac{c}{d}+\frac{b}{a}\right), c \leqslant \frac{1}{4}\left(\frac{c}{d}+\frac{c}{d}+\frac{d}{a}+\frac{c}{b}\right), d \leqslant$ $\frac{1}{4}\left(\frac{d}{a}+\frac{d}{a}+\frac{a}{b}+\frac{d}{c}\right)$, 将它们相加, 并利用条件, 得到命题成立.

  \item 对任意 $i .1 \leqslant i \leqslant n$, 有 $\frac{a_{i}^{3}}{a_{i}^{2}+a_{i} a_{i+1}+a_{i+1}^{2}}=a_{i}-\frac{a_{i}^{2} a_{i+1}+a_{i} a_{i+1}^{2}}{a_{i}^{2}+a_{i} a_{i+1}+a_{i+1}^{2}} \geqslant$ $a_{i}-\frac{a_{i}^{2} a_{i+1}+a_{i} a_{i+1}^{2}}{3 a_{i} a_{i+1}}=\frac{2 a_{i}-a_{i+1}}{3}$. 对 $i, 1 \leqslant i \leqslant n$ 求和, 得 $\sum_{i=1}^{n} \frac{a_{i}^{3}+a_{i+1}^{3}}{a_{i}^{2}+a_{i} a_{i+1}+a_{i+1}^{2}}$\\
$\geqslant \frac{2}{3} \sum_{i=1}^{n} a_{i}$.

  \item 因为 $a_{1}+a_{2}+\cdots+a_{n}=1$, 所以由平均值不等式可得 $1+a_{i}=a_{1}+a_{2}$ $+\cdots+a_{n}+a_{i} \geqslant(n+1)\left(a_{1} a_{2} \cdots a_{n} a_{i}\right)^{1 /(n+1)}, 1-a_{i}=a_{1}+a_{2}+\cdots+a_{n}-a_{i} \geqslant$ $(n-1)\left(a_{1} a_{2} \cdots a_{n} / a_{i}\right)^{1 /(n-1)}$. 取 $i=1,2, \cdots, n$ 再将之分别累积后得 $\prod_{i=1}^{n}(1-$ $\left.a_{i}^{2}\right) \geqslant\left(n^{2}-1\right)^{n} \prod_{i=1}^{n} a_{i}^{2}$, 从而 $\left(\frac{1}{a_{1}^{2}}-1\right)\left(\frac{1}{a_{2}^{2}}-1\right) \cdots\left(\frac{1}{a_{n}^{2}}-1\right) \geqslant\left(n^{2}-1\right)^{n}$.

  \item 由于 $a+b+c+d+\frac{1}{a b c d} \geqslant 36 \cdot \sqrt[36]{a b c d\left(\frac{1}{32 a b c d}\right)^{32}}=$ $\frac{36}{\sqrt[36]{32^{32}\left(a^{2} b^{2} c^{2} d^{2}\right)^{31 / 2}}} \geqslant \frac{36}{\sqrt[36]{32^{32}\left(\frac{a^{2}+b^{2}+c^{2}+d^{2}}{4}\right)^{62}}}=\frac{36}{\sqrt[36]{32^{32}\left(\frac{1}{4}\right)^{62}}}=18$. 当且仅当 $a=b=c=d=\frac{1}{2}$ 时, 等式成立.

  \item 由平均值不等式 $\frac{x_{1}^{2}+x_{2}+\cdots+x_{n-1}+x_{n}}{x_{1}+x_{2}+\cdots+x_{n-1}+x_{n}^{2}}=\frac{x_{1}^{2}-x_{1}+1}{x_{n}^{2}-x_{n}+1}=$ $\frac{x_{1}^{2}+\frac{1}{4}+\frac{3}{4}-x_{1}}{x_{n}^{2}-x_{n}+1} \geqslant \frac{x_{1}+\frac{3}{4}-x_{1}}{x_{n}-x_{n}+1}=\frac{3}{4}$. 当 $x_{1}=\frac{1}{2}, x_{2}+\cdots+x_{n-1}=\frac{1}{2}$, $x_{n}=0$ 时, 等号成立. 另一方面, $\frac{x_{1}^{2}-x_{1}+1}{x_{n}^{2}-x_{n}+1}=\frac{x_{1}^{2}-x_{1}+1}{x_{n}^{2}+\frac{1}{4}+\frac{3}{4}-x_{n}} \leqslant \frac{4}{3}\left(x_{1}^{2}-\right.$ $\left.x_{1}+1\right) \leqslant \frac{4}{3}$. 当 $x_{1}=0, x_{2}+\cdots+x_{n-1}=\frac{1}{2}, x_{n}=\frac{1}{2}$ 时, 等号成立.

  \item 显然原不等式等价于 $\frac{1+x_{1}+x_{1}^{2}}{x_{1}\left(1+x_{1}\right)}+\frac{1+x_{2}+x_{2}^{2}}{x_{2}\left(1+x_{2}\right)}+\cdots+\frac{1+x_{n}+x_{n}^{2}}{x_{n}\left(1+x_{n}\right)} \geqslant$ $\frac{3 n}{2}$. 注意到 $4\left(1+x_{i}+x_{i}^{2}\right) \geqslant 3\left(1+x_{i}\right)^{2}$ 对任意的 $i=1,2, \cdots, n$ 都成立, 因此要证明上式只需证明 $\frac{3}{4}\left(\frac{1+x_{1}}{x_{1}}+\frac{1+x_{2}}{x_{2}}+\cdots+\frac{1+x_{n}}{x_{n}}\right) \geqslant \frac{3 n}{2}$, 即 $\frac{1}{x_{1}}+\frac{1}{x_{2}}$ $+\cdots+\frac{1}{x_{n}} \geqslant n \cdots$ (3). 由 $x_{1} x_{2} \cdots x_{n}=1$ 及平均值不等式易知(3)成立.

  \item 由 $a^{2}+b^{2}+c^{2}+(a+b+c)^{2} \leqslant 4$ 可知 $a^{2}+b^{2}+c^{2}+a b+b c+c a \leqslant 2$,因此 $\frac{2(a b+1)}{(a+b)^{2}} \geqslant \frac{2 a b+a^{2}+b^{2}+c^{2}+a b+b c+c a}{(a+b)^{2}}=\frac{(a+b)^{2}+(c+a)(c+b)}{(a+b)^{2}}$即 $\frac{a b+1}{(a+b)^{2}} \geqslant \frac{1}{2}\left(1+\frac{(c+a)(c+b)}{(a+b)^{2}}\right) \cdots$

\end{enumerate}

(4). 同理可得 $\frac{b c+1}{(b+c)^{2}} \geqslant$\\
$\frac{1}{2}\left(1+\frac{(a+b)(a+c)}{(b+c)^{2}}\right), \frac{c a+1}{(c+a)^{2}} \geqslant \frac{1}{2}\left(1+\frac{(b+c)(b+a)}{(c+a)^{2}}\right) \cdots$ (5). 另外由平均值不等式显然有 $\frac{(c+a)(c+b)}{(a+b)^{2}}+\frac{(a+b)(a+c)}{(b+c)^{2}}+\frac{(b+c)(b+a)}{(c+a)^{2}} \geqslant$ $3 \cdots$ (6). 综合(4)(5)(6)可得 $\frac{a b+1}{(a+b)^{2}}+\frac{b c+1}{(b+c)^{2}}+\frac{c a+1}{(c+a)^{2}} \geqslant 3$.

\begin{enumerate}
  \setcounter{enumi}{10}
  \item 注意到 $\frac{a^{2}+2}{2}=\frac{\left(a^{2}-a+1\right)+(a+1)}{2} \geqslant \sqrt{\left(a^{2}-a+1\right) \cdot(a+1)}=$ $\sqrt{1+a^{3}}$. 要证原不等式只需证明 $\frac{a^{2}}{\left(a^{2}+2\right)\left(b^{2}+2\right)}+\frac{b^{2}}{\left(b^{2}+2\right)\left(c^{2}+2\right)}+$ $\frac{c^{2}}{\left(c^{2}+2\right)\left(a^{2}+2\right)} \geqslant \frac{1}{3}$. 而上式等价于 $3 a^{2}\left(c^{2}+2\right)+3 b^{2}\left(a^{2}+2\right)+3 c^{2}\left(b^{2}+\right.$ 2) $\geqslant\left(a^{2}+2\right)\left(b^{2}+2\right)\left(c^{2}+2\right)$. 即 $\left(a^{2} b^{2}+b^{2} c^{2}+c^{2} a^{2}\right)+2\left(a^{2}+b^{2}+c^{2}\right) \geqslant$ $a^{2} b^{2} c^{2}+8=64+8=72$. 而 $a^{2} b^{2}+b^{2} c^{2}+c^{2} a^{2} \geqslant 3(a b c)^{\frac{4}{3}}=48, a^{2}+b^{2}+$ $c^{2} \geqslant 3(a b c)^{\frac{2}{3}}=12$, 则上式显然成立. 故原不等式得证, 当且仅当 $a=b=c=$ 2 时取等号.

  \item $n=1$ 时显然成立. 假设 $n=k$ 时, 有 $(a+b)^{k}-a^{k}-b^{k} \geqslant 2^{2 k}-2^{k+1}$.则对 $n=k+1$, 由 $\frac{1}{a}+\frac{1}{b}=1$, 有 $a+b=a b$, 于是 $a b=a+b \geqslant 2 \sqrt{a b}$ 即 $a b=a+b \geqslant 4$. 从而, $(a+b)^{k+1}-a^{k+1}-b^{k+1}=(a+b)\left[(a+b)^{k}-a^{k}-\right.$ $\left.b^{k}\right]+a^{k} b+a b^{k} \geqslant 4\left(2^{2 k}-2^{k+1}\right)+2 \sqrt{a^{k+1} b^{k+1}} \geqslant 2^{2 k+2}-2^{k+3}+2^{k+2}=2^{2(k+1)}-$ $2^{(k+1)+1}$.

  \item 已知 $a b>0, x-1>0$, 则 $a x+\frac{x}{x-1}=\left[a(x-1)+\frac{1}{x-1}\right]+a+$ $1 \geqslant 2 \sqrt{a}+a+1=(\sqrt{a}+1)^{2}$. 当且仅当 $a(x-1)=\frac{1}{x-1}$, 即 $x=1+\frac{1}{\sqrt{a}}$时, $a x+\frac{x}{x-1}$ 的最小值为 $(\sqrt{a}+1)^{2}$, 于是 $a x+\frac{x}{x-1}>b$ 对任意 $x>1$ 成立的充要条件是 $(\sqrt{a}+1)^{2}>b$, 即 $\sqrt{a}+1>\sqrt{b}$.

  \item 若 $y_{1}^{2}+y_{2}^{2}-1 \geqslant 0$, 不等式显然成立. 若 $y_{1}^{2}+y_{2}^{2}-1<0$, 则由平均值不等式, 得 $x_{1} y_{1} \leqslant \frac{x_{1}^{2}+y_{1}^{2}}{2}, x_{2} y_{2} \leqslant \frac{x_{2}^{2}+y_{2}^{2}}{2}$. 则 $x_{1} y_{1}+x_{2} y_{2} \leqslant \frac{1}{2}\left(x_{1}^{2}+x_{2}^{2}+y_{1}^{2}\right.$ $\left.+y_{2}^{2}\right) \leqslant 1$. 因为 $1-x_{1} y_{1}-x_{2} y_{2} \geqslant 1-\frac{x_{1}^{2}+y_{1}^{2}}{2}-\frac{x_{2}^{2}+y_{2}^{2}}{2}=\frac{1-x_{1}^{2}-x_{2}^{2}+1-y_{1}^{2}-y_{2}^{2}}{2}$ $>0$, 所以, $\left(1-x_{1} y_{1}-x_{2} y_{2}\right)^{2} \geqslant\left(\frac{1-x_{1}^{2}-x_{2}^{2}+1-y_{1}^{2}-y_{2}^{2}}{2}\right)^{2} \geqslant\left(x_{1}^{2}+x_{2}^{2}-\right.$ 1) $\left(y_{1}^{2}+y_{2}^{2}-1\right)$.

  \item 由于 $\left(1+\frac{a}{b}\right)\left(1+\frac{b}{c}\right)\left(1+\frac{c}{a}\right)=2+\left(\frac{a}{c}+\frac{c}{b}+\frac{b}{a}\right)+$ $\left(\frac{a}{b}+\frac{b}{c}+\frac{c}{a}\right)=2+\left(\frac{a}{c}+\frac{a}{b}+\frac{a}{a}\right)+\left(\frac{b}{a}+\frac{b}{c}+\frac{b}{b}\right)+\left(\frac{c}{b}+\frac{c}{a}+\frac{c}{c}\right)-$ $3 \geqslant-1+3 \frac{a+b+c}{\sqrt[3]{a b c}} \geqslant 2\left(1+\frac{a+b+c}{\sqrt[3]{a b c}}\right)$.

  \item 由平均值不等式, 得 $\frac{x_{2}}{x_{1}}=\frac{x_{2}}{x_{3}} \cdot \frac{x_{3}}{x_{1}} \leqslant \frac{1}{2}\left[\left(\frac{x_{2}}{x_{3}}\right)^{2}+\left(\frac{x_{3}}{x_{1}}\right)^{2}\right], \frac{x_{3}}{x_{2}}=\frac{x_{3}}{x_{1}} \cdot$ $\frac{x_{1}}{x_{2}} \leqslant \frac{1}{2}\left[\left(\frac{x_{3}}{x_{1}}\right)^{2}+\left(\frac{x_{1}}{x_{2}}\right)^{2}\right], \frac{x_{1}}{x_{3}}=\frac{x_{1}}{x_{2}} \cdot \frac{x_{2}}{x_{3}} \leqslant \frac{1}{2}\left[\left(\frac{x_{1}}{x_{2}}\right)^{2}+\left(\frac{x_{2}}{x_{3}}\right)^{2}\right]$. 将它们相加, 便得到命题成立.

  \item 由于 $1+a=2-b-c=1-b+1-c \geqslant 2 \sqrt{(1-b)(1-c)}$, 同理可得 $1+b \geqslant 2 \sqrt{(1-a)(1-c)}, 1+c \geqslant 2 \sqrt{(1-a)(1-b)}$. 将以上三式相乘便可以.

\end{enumerate}
$$
\begin{array}{r}
\text { 18. } \frac{x^{2} y}{z}+\frac{y^{2} z}{x}+\frac{z^{2} x}{y}-x^{2}-y^{2}-z^{2}=\frac{x^{2}}{z}(y-z)+\frac{y^{2} z}{x}+\frac{z^{2} x}{y}-y^{2}-z^{2} \geqslant \\
\frac{y^{2}}{z}(y-z)+2 z \sqrt{y z}-y^{2}-z^{2}=\frac{y-z}{z}\left(y^{2}-y z+z^{2}-\frac{2 z^{2} \sqrt{y}}{\sqrt{y}+\sqrt{z}}\right)=
\end{array}
$$

$\frac{(y-z)(\sqrt{y}-\sqrt{z})}{z(\sqrt{y}+\sqrt{z})}\left[y(\sqrt{y}+\sqrt{z})^{2}-z^{2}\right] \geqslant 0$.

\begin{enumerate}
  \setcounter{enumi}{18}
  \item 因 $\left(\frac{a b}{c}+\frac{b c}{a}+\frac{c a}{b}\right)^{2}=\frac{a^{2} b^{2}}{c^{2}}+\frac{b^{2} c^{2}}{a^{2}}+\frac{c^{2} a^{2}}{b^{2}}+2\left(a^{2}+b^{2}+c^{2}\right)=$ $\frac{1}{2}\left(\frac{a^{2} b^{2}}{c^{2}}+\frac{c^{2} a^{2}}{b^{2}}\right)+\frac{1}{2}\left(\frac{b^{2} c^{2}}{a^{2}}+\frac{a^{2} b^{2}}{c^{2}}\right)+\frac{1}{2}\left(\frac{c^{2} a^{2}}{b^{2}}+\frac{b^{2} c^{2}}{a^{2}}\right)+2 \geqslant a^{2}+b^{2}+c^{2}+2=$ 3. 所以, 命题成立.

  \item 因为 $\frac{a^{3}}{b+c+d}+\frac{b+c+d}{18}+\frac{1}{12} \geqslant 3 \sqrt[3]{\frac{a^{3}}{a+c+d} \cdot \frac{b+c+d}{18} \cdot \frac{1}{12}}=$ $\frac{a}{2}$, 即 $\frac{a^{3}}{b+c+d} \geqslant \frac{a}{2}-\frac{b+c+d}{18}-\frac{1}{12}$, 所以左 $\geqslant \frac{a+b+c+d}{2}-\frac{1}{18}(3 a+$ $3 b+3 c+3 d)-\frac{4}{12}=\frac{1}{3}(a+b+c+d)-\frac{1}{3}$. 又由假设知 $a b+b c+c d+$ $d a=1$, 即 $(a+c)(b+d)=1$, 所以, $a+b+c+d=a+c+\frac{1}{a+c} \geqslant 2$. 故 $\frac{a^{3}}{b+c+d}+\frac{b^{3}}{a+c+d}+\frac{c^{3}}{a+b+d}+\frac{d^{3}}{a+b+c} \geqslant \frac{1}{3}$.

  \item 不妨设 $a_{1} \geqslant a_{2} \geqslant \cdots \geqslant a_{n}$, 则 $a_{i}-a_{j} \geqslant(i-j) m . \sum_{1 \leqslant i<j \leqslant n}\left(a_{i}-a_{j}\right)^{2}=$\\
$(n-1) \sum_{i=1}^{n} a_{i}^{2}-2 \sum_{1 \leqslant i<j \leqslant n} a_{i} a_{j}=(n-1) \sum_{i=1}^{n} a_{i}^{2}-\left[\left(\sum_{i=1}^{n} a_{i}\right)^{2}-\sum_{i=1}^{n} a_{i}^{2}\right] \leqslant n \sum_{i=1}^{n} a_{i}^{2}=$ n. 另一方面, $\sum_{1 \leqslant i<j \leqslant n}\left(a_{i}-a_{j}\right)^{2} \geqslant m^{2} \sum_{1 \leqslant i<j \leqslant n}(i-j)^{2}=m^{2} \sum_{k=1}^{n-1}(n-k) k^{2}=$ $\frac{m^{2} n^{2}\left(n^{2}-1\right)}{12}$. 所以 $n \geqslant \frac{m^{2} n^{2}\left(n^{2}-1\right)}{12}$, 即 $m \leqslant \sqrt{\frac{12}{n\left(n^{2}-1\right)}}$. 且当 $\left|a_{i}\right|$ 成等差数列, $\sum_{i=1}^{n} a_{i}=0$ 时等号成立. 故 $m$ 的最大值为 $\sqrt{\frac{12}{n\left(n^{2}-1\right)}}$.

  \item 当 $y>1-\frac{1}{100}$ 时, 显然成立. 当 $y \leqslant 1-\frac{1}{100}$ 时, 由于 $x>1-y^{2016}$ $\geqslant 1-\left(1-\frac{1}{100}\right)^{2016}$ 由伯努利不等式, $x^{2016}>1-2016\left(1-\frac{1}{100}\right)^{2016}$ 于是, 只要证 明 $1-2016\left(1-\frac{1}{100}\right)^{2016}>1-\frac{1}{100} \Leftrightarrow\left(1-\frac{1}{100}\right)^{2016}$ $\frac{1}{100 \times 2016} \Leftrightarrow\left(\frac{100}{99}\right)^{2016}>100 \times 2016$. 再次利用伯努利不等式, $\left(\frac{100}{99}\right) 2016>$ $\left(\left(1+\frac{1}{99}\right)^{100}\right)^{20}>\left(1+\frac{100}{99}\right)^{20}>2^{20}=2^{9} \times 2^{11}=512 \times 2048>100 \times 2016$.

  \item 由平均值不等式 $1=\frac{1}{n} \sum_{i=1}^{n} \frac{\sqrt{2}+x_{i}}{\sqrt{2}+x_{i}}=\frac{1}{n} \sum_{i=1}^{n} \frac{\sqrt{2}}{\sqrt{2}+x_{i}}+\frac{1}{n} \sum_{i=1}^{n} \frac{x_{i}}{\sqrt{2}+x_{i}}$ $\geqslant \sqrt[n]{\prod_{i=1}^{n} \frac{\sqrt{2}}{\sqrt{2}+x_{i}}}+\sqrt[n]{\prod_{i=1}^{n} \frac{x_{i}}{\sqrt{2}+x_{i}}}=\frac{\sqrt{2}+1}{\sqrt[n]{\prod_{i=1}^{n}\left(\sqrt{2}+x_{i}\right)}}$, 从而 $\prod_{i=1}^{n}\left(\sqrt{2}+x_{i}\right) \geqslant$ $(1+\sqrt{2})^{n}$.

  \item 由平均值不等式 $\sum_{\text {cyc }} a^{4} \geqslant a^{4}+b^{4}+2 c^{2} d^{2}=\frac{1}{2}(a-b)^{2}(a+b)^{2}+$ $\frac{1}{2}\left(a^{2}+b^{2}\right)^{2}+2 c^{2} d^{2} \geqslant 2 a b(a-b)^{2}+2 c d\left(a^{2}+b^{2}\right)=4 a b c d+2 a b(a-b)^{2}+$ $2 c d(a-b)^{2} \geqslant 4 a b c d+4(a-b)^{2} \sqrt{a b c d}$.

  \item 不妨设 $0<a \leqslant b \leqslant c \leqslant d$; 则 $a^{3} b^{3} c^{3} \leqslant a^{3} b^{3} d^{3} \leqslant a^{2} b c^{3} d^{3} \leqslant a b^{2} c^{3} d^{3}$,以及 $a^{3} b^{3} d^{3}+a^{3} b^{3} c^{3} \leqslant 3 a^{2} b c^{3} d^{3}+3 a b^{2} c^{3} d^{3}$, 由平均值不等式, $b^{3} c^{3} d^{3}+a^{3} c^{3} d^{3}$ $+a^{3} b^{3} d^{3}+a^{3} b^{3} c^{3} \leqslant\left(a^{3}+3 a^{2} b+3 a b^{2}+b^{3}\right) c^{3} d^{3}=(a+b)^{3} c^{3} d^{3} \leqslant$ $\left(\frac{(a+b)+c+d}{3}\right)^{9}=1$

  \item 不等式左边 $=\sum_{k=1}^{n} \frac{x_{k}^{2}}{k}-\frac{1}{\sum_{k=1}^{n} k x_{k}^{2}}$. 于是, 只需要证明 $1-\frac{1}{\sum_{k=1}^{n} k x_{k}^{2} \sum_{k=1}^{n} \frac{x_{k}^{2}}{k}}$\\
$\leqslant\left(\frac{n-1}{n+1}\right)^{2}$ 等价于 $1-\left(\frac{n-1}{n+1}\right)^{2} \leqslant \frac{1}{\sum_{k=1}^{n} k x_{k}^{2} \cdot \sum_{k=1}^{n} \frac{x_{k}^{2}}{k}}$ 等价于 $\sum_{k=1}^{n} k x_{k}^{2} \cdot \sum_{k=1}^{n} \frac{x_{k}^{2}}{k}$ $\leqslant \frac{(n+1)^{2}}{4 n}$. 由平均值不等式, $\sum_{k=1}^{n} k x_{k}^{2} \cdot \sum_{k=1}^{n} \frac{x_{k}^{2}}{k}=\frac{1}{n}\left[\left(\sum_{k=1}^{n} k x_{k}^{2}\right) \cdot\left(n \sum_{k=1}^{n} \frac{x_{k}^{2}}{k}\right)\right] \leqslant$ $\frac{1}{n} \cdot \frac{1}{4}\left(\sum_{k=1}^{n} k x_{k}^{2}+n \sum_{k=1}^{n} \frac{x_{k}^{2}}{k}\right)^{2}=\frac{1}{4 n}\left(\sum_{k=1}^{n}\left(k+\frac{n}{k}\right) x k^{2}\right)^{2} \leqslant \frac{1}{4 n}\left(\sum_{k=1}^{n}(n+1) x_{k}^{2}\right)^{2}=$ $\frac{(n+1)^{2}}{4 n}$. 如果等式成立, 则 $\sum_{k=1}^{n}\left(k+\frac{n}{k}\right) x_{k}^{2}=\sum_{k=1}^{n}(n+1) x_{k}^{2}$. 由于 当 $k=2$, $\cdots, n-1$ 时, $k+\frac{n}{k}<n+1$. 所以 $x_{2}^{2}+\cdots+x_{n-1}^{2}=0$. 又由 $\sum_{k=1}^{n} k x_{k}^{2}\left(n \sum_{k=1}^{n} \frac{x_{k}^{2}}{k}\right)=$ $\frac{1}{4}\left(\sum_{k=1}^{n} k x_{k}^{2}+n \sum_{k=1}^{n} \frac{x_{k}^{2}}{k}\right)^{2}$, 得到 $x_{1}^{2}+n x_{n}^{2}=n x_{1}^{2}+x_{n}^{2}$, 即 $x_{1}^{2}=x_{n}^{2}$. 从而, $x_{1}^{2}=x_{n}^{2}$ $=\frac{1}{2}$. 于是, 等式成立的充要条件是 $x_{1}= \pm \frac{\sqrt{2}}{2}, x_{2}=\cdots=x_{n-1}=0, x_{n}=$ $\pm \frac{\sqrt{2}}{2}$

\end{enumerate}

\section*{习 题 2}
\begin{enumerate}
  \item 由平均值不等式 $u \geqslant \frac{2}{\sqrt{(1-a)(1-b)}} \geqslant \frac{2}{\frac{1-a+1-b}{2}}=$ $\frac{4}{2-(a+b)} \geqslant \frac{4}{2-2 \sqrt{a b}}=\frac{4}{2-\frac{1}{3}}=\frac{12}{5}$. 当 $a=b=\frac{1}{6}$ 时,上式取等号,故 $u$ 的最小值为 $\frac{12}{5}$.

  \item 不失一般性, 令 $a \geqslant b \geqslant c$. 则 $\frac{(c-a)(c-b)}{3(c+a b)} \geqslant 0$ 及 $\frac{(a-b)(a-c)}{3(a+b c)}+$ $\frac{(b-a)(b-c)}{3(b+a c)}=\frac{c(a-b)^{2}}{3}\left[\frac{1+a+b-c}{(a+b c)(b+a c)}\right] \geqslant 0$. 故 $\sum_{\text {cyc }} \frac{(a-b)(a-c)}{3(a+b c)}$ $\geqslant 0$. 而 $\sum_{\text {cyc }} \frac{1}{a+b c} \leqslant \frac{9}{2(a b+b c+a c)} \Leftrightarrow \sum_{\text {cyc }} \frac{1}{a(a+b+c)+3 b c} \leqslant$ $\frac{3}{2(a b+b c+a c)} \Leftrightarrow \sum_{\text {cyc }}\left[\frac{1}{2(a b+b c+a c)}-\frac{1}{a(a+b+c)+3 b c}\right] \geqslant 0 \Leftrightarrow$ $\sum_{\text {cyc }} \frac{(a-b)(a-c)}{a(a+b+c)+3 b c}=\sum_{\text {cyc }} \frac{(a-b)(a-c)}{3(a+b c)} \geqslant 0$. 由平均值不等式知 $\frac{1}{a \sqrt{2\left(a^{2}+b c\right)}}=\frac{\sqrt{b+c}}{\sqrt{2 a} \cdot \sqrt{(a b+a c)\left(a^{2}+b c\right)}} \geqslant \frac{\sqrt{2(b+c)}}{\sqrt{a}(a+c)(a+b)}$. 只需证 $\sum_{\text {cyc }} \sqrt{\frac{b+c}{2 a}} \cdot \frac{1}{(a+c)(a+b)} \geqslant \frac{9}{4(a b+b c+a c)}$. 又 $\sqrt{\frac{b+c}{2 a}} \leqslant$ $\sqrt{\frac{a+c}{2 b}} \leqslant \sqrt{\frac{a+b}{2 c}}, \frac{1}{(a+c)(a+b)} \leqslant \frac{1}{(b+c)(a+b)} \leqslant \frac{1}{(a+c)(c+b)}$,则由切比雪夫不等式知 $\sum_{\mathrm{cyc}} \sqrt{\frac{b+c}{2 a}} \cdot \frac{1}{(a+c)(a+b)} \geqslant \frac{1}{3}\left(\sum_{\mathrm{cyc}} \sqrt{\frac{b+c}{2 a}}\right) \cdot$ $\sum_{\text {cyc }} \frac{1}{(a+c)(a+b)}=\frac{2}{(a+b)(b+c)(a+c)} \sum_{\text {cyc }} \sqrt{\frac{b+c}{2 a}}$. 只需证 $\sum_{\text {cyc }} \sqrt{\frac{b+c}{2 a}}$ $\geqslant \frac{9(a+b)(b+c)(a+c)}{8(a b+b c+a c)}$. 令 $t=\sqrt[6]{\frac{(a+b)(b+c)(a+c)}{8 a b c}} \geqslant 1$. 则 $\frac{9(a+b)(b+c)(a+c)}{8(a b+b c+a c)}=\frac{27 t^{6}}{8 t^{6}+1}$. 由平均值不等式知 $\sum_{\text {cyc }} \sqrt{\frac{b+c}{2 a}} \geqslant 3 t$. 故 $3 t \geqslant \frac{27 t^{6}}{8 t^{6}+1} \Leftrightarrow 8 t^{6}-9 t^{5}+1 \geqslant 0$. 而当 $t \geqslant 1$ 时, 上述不等式恒成立.

  \item 原不等式 $\Leftrightarrow \frac{x_{1} x_{2} \cdots x_{n}}{\left(1+x_{1}\right)^{2}\left(1+x_{1}+x_{2}\right)^{2} \cdots\left(1+x_{1}+\cdots+x_{n}\right)^{2}} \leqslant$\\
$\left(\frac{1}{1+n}\right)^{n+1}$. 令 $y_{1}=\frac{x_{1}}{1+x_{1}}, y_{2}=\frac{x_{2}}{\left(1+x_{1}\right)\left(1+x_{1}+x_{2}\right)}, \cdots, y_{n}=$ $\frac{x_{n}}{\left(1+x_{1}+\cdots+x_{n-1}\right)\left(1+x_{1}+\cdots+x_{n}\right)}, y_{n+1}=\frac{1}{1+x_{1}+\cdots+x_{n}}$. 则 $\sum_{i=1}^{n+1} y_{i}$ $=1$. 由平均值不等式得 $\frac{x_{1} x_{2} \cdots x_{n}}{\left(1+x_{1}\right)^{2}\left(1+x_{1}+x_{2}\right)^{2} \cdots\left(1+x_{1}+\cdots+x_{n}\right)^{2}}=$ $y_{1} y_{2} \cdots y_{n+1} \leqslant\left(\frac{1}{n+1} \sum_{i=1}^{n+1} y_{i}\right)^{n+1}=\left(\frac{1}{1+n}\right)^{n+1}$. 从而, 命题成立.

  \item 设 $a_{n+1}=1-\left(a_{1}+a_{2}+\cdots+a_{n}\right)$, 则 $a_{n+1} \geqslant 0, a_{1}+\cdots+a_{n+1}=1$.则原不等式等价于 $n^{n+1} a_{1} a_{2} \cdots a_{n+1} \leqslant\left(1-a_{1}\right)\left(1-a_{2}\right) \cdots\left(1-a_{n}\right)\left(1-a_{n+1}\right)$. 对 $i=1,2, \cdots, n+1$, 由平均值不等式, 得 $1-a_{i}=a_{1}+\cdots+a_{i-1}+$ $a_{i+1}+\cdots+a_{n+1} \geqslant n \sqrt[n]{a_{1} \cdots a_{i-1} a_{i+1} \cdots a_{n+1}}$. 将 $n+1$ 个不等式相乘, 得 $(1-$ $\left.a_{1}\right)\left(1-a_{2}\right) \cdots\left(1-a_{n+1}\right) \geqslant n^{n+1} \sqrt[n]{a_{1}^{n} a_{2}^{n} \cdots a_{n+1}^{n}}=n^{n+1} a_{1} a_{2} \cdots a_{n+1}$, 当且仅当 $a_{1}=a_{2}=\cdots=a_{n+1}=\frac{1}{n+1}$ 时等号成立.

  \item 右 $=\left(\prod_{i=1}^{n} \frac{a_{i}}{a_{i}+b_{i}}\right)^{\frac{1}{n}}+\left(\prod_{i=1}^{n} \frac{b_{i}}{a_{i}+b_{i}}\right)^{\frac{1}{n}} \leqslant \frac{1}{n}\left(\sum_{i=1}^{n} \frac{a_{i}}{a_{i}+b_{i}}+\sum_{i=1}^{n} \frac{b_{i}}{a_{i}+b_{i}}\right)=$ 1. 所以左 $\geqslant$ 右.

  \item $\frac{x+y}{2}-\sqrt{x y}>\sqrt{\frac{x^{2}+y^{2}}{2}}-\frac{x+y}{2} \Leftrightarrow x+y>\sqrt{\frac{x^{2}+y^{2}}{2}}+\sqrt{x y} \Leftrightarrow$ $(x+y)^{2}>\frac{x^{2}+y^{2}}{2}+x y+\sqrt{2 x y\left(x^{2}+y^{2}\right)} \Leftrightarrow \frac{1}{2}(x+y)^{2}>\sqrt{2 x y\left(x^{2}+y^{2}\right)} \Leftrightarrow$ $(x+y)^{4}>8 x y\left(x^{2}+y^{2}\right) \Leftrightarrow x^{4}+4 x^{3} y+6 x^{2} y^{2}+4 x y^{3}+y^{4} \geqslant 8 x^{3} y+8 x y^{3} \Leftrightarrow$ $(x-y)^{4}>0$. 因为 $x \neq y$, 所以上面最后一个不等式成立, 即 $A-G>Q-A$.下面证明 $Q-A>G-H$. 因为 $\sqrt{\frac{x^{2}+y^{2}}{2}}-\frac{x+y}{2}>\sqrt{x y}-\frac{2 x y}{x+y} \Leftrightarrow$ $\sqrt{\frac{x^{2}+y^{2}}{2}}-\sqrt{x y}>\frac{x+y}{2}-\frac{2 x y}{x+y} \Leftrightarrow \frac{\frac{1}{2}\left(x^{2}+y^{2}\right)-x y}{\sqrt{\frac{x^{2}+y^{2}}{2}}+\sqrt{x y}}>\frac{(x-y)^{2}}{2(x+y)} \Leftrightarrow x+$ $y>\sqrt{\frac{x^{2}+y^{2}}{2}}+\sqrt{x y}$. 此不等式前面已经证明成立, 所以 $Q-A>G-H$.

  \item $\sum_{k=1}^{n} \frac{d}{a_{k}}=\sum_{k=1}^{n} \frac{a_{k+1}-a_{k}}{a_{k}}=\sum_{k=1}^{n} \frac{a_{k+1}}{a_{k}}-n \geqslant n\left(\sqrt[n]{\frac{a_{2}}{a_{1}} \cdot \frac{a_{3}}{a_{2}} \cdot \cdots \cdot \frac{a_{n+1}}{a_{n}}}-1\right)=$ $n\left(\sqrt[n]{\frac{a_{n+1}}{a_{1}}}-1\right)$. 又 $\sum_{k=1}^{n} \frac{d}{a_{k}}=\frac{d}{a_{1}}+\sum_{k=2}^{n}\left(1-\frac{a_{k-1}}{a_{k}}\right) \leqslant \frac{d}{a_{1}}+(n-1)-(n-1) \cdot$\\
$\sqrt[n-1]{\frac{a_{1}}{a_{n}}}=\frac{d}{a_{1}}+(n-1)\left(1-\sqrt[n-1]{\frac{a_{1}}{a_{n}}}\right)$.

  \item 首先设 $x_{i} \in \mathbf{N}$. 因为 $x_{i}$ 互不相同, 所以左边 $=$ $[\frac{\sum_{i=1}^{n} \overbrace{x_{i}+\cdots+x_{i}}^{x_{i} \text { 个 }}}{x_{1}+x_{2}+\cdots+x_{n}}]^{x_{1}+x_{2}+\cdots+x_{n}}>x_{1}{ }^{x_{1}} \cdot x_{2}{ }^{x_{2}} \cdots \cdots x_{n}^{{ }^{x_{n}}}$. 下面设 $x_{i}$ 为正有理数.记 $m$ 为 $x_{1}, x_{2}, \cdots, x_{n}$ 的各分母的最小公倍数, 则 $m x_{1}, m x_{2}, \cdots, m x_{n} \in \mathbf{Z}_{+}$,于是 $\left[\frac{\left(m x_{1}\right)^{2}+\cdots+\left(m x_{n}\right)^{2}}{m x_{1}+m x_{2}+\cdots+m x_{n}}\right]^{m x_{1}+m x_{2}+\cdots+m x_{n}}>\left(m x_{1}\right)^{m x_{1}} \cdot\left(m x_{2}\right)^{m x_{2}} \cdot \cdots$ ・ $\left(m x_{n}\right)^{m x_{n}}$. 两边开 $m$ 次方并除以 $m^{\left(x_{1}+\cdots+x_{n}\right)}$, 即得到原不等式.

  \item 设 $x_{i}>1(i=1,2, \cdots, n)$, 则 $\prod_{i=1}^{n}\left(x_{i}+1\right)^{\frac{1}{n}} \geqslant\left(\prod_{i=1}^{n} x_{i}\right)^{\frac{1}{n}}+1 \cdots$ (1). $\left(\prod_{i=1}^{n} x_{i}\right)^{\frac{1}{n}} \geqslant \prod_{i=1}^{n}\left(x_{i}-1\right)^{\frac{1}{n}}+1$, 即 $0<\left[\prod_{i=1}^{n}\left(x_{i}-1\right)\right]^{\frac{1}{n}} \leqslant\left(\prod_{i=1}^{n} x_{i}\right)^{\frac{1}{n}}-1 \cdots$ (2).由(1)、 (2), 得 $\prod_{i=1}^{n} \frac{x_{i}+1}{x_{i}-1} \geqslant\left(\frac{\sqrt[n]{x_{1} \cdots x_{n}}+1}{\sqrt[n]{x_{1} \cdots x_{n}}-1}\right)^{n} \cdots$ (3). 又函数 $f(x)=\frac{x+1}{x-1}=$ $1+\frac{2}{x-1}$ 在 $x>1$ 时是减函数. 令 $x_{i}=r_{i} s_{i} t_{i} u_{i} v_{i}$, 由平均值不等式, 得 $\left(\prod_{i=1}^{n} x_{i}\right)^{\frac{1}{n}}=\sqrt[n]{r_{1} \cdots r_{n}} \cdots \sqrt[n]{v_{1} \cdots v_{n}} \leqslant R S T U V$. 代人(3)式, 得 $\prod_{i=1}^{n} \frac{r_{i} s_{i} t_{i} u_{i} v_{i}+1}{r_{i} s_{i} t_{i} u_{i} v_{i}-1} \geqslant$ $\left(\frac{R S T U V+1}{R S T U V-1}\right)^{n}$

  \item 由平均值不等式得 $\frac{1}{x} \cdot x a^{x}+\frac{1}{y} \cdot y b^{y}+\frac{1}{z} \cdot z c^{z} \geqslant x^{\frac{1}{x}} y^{\frac{1}{y}} z^{\frac{1}{z}} a b c$. 只要证明 $x^{\frac{1}{x}} y^{\frac{1}{y}} z^{\frac{1}{z}} \geqslant \frac{4 x y z}{(x+y+z-3)^{2}}$. 由 加权平均值不 等 式, $2 x^{\frac{1}{2}\left(1-\frac{1}{x}\right)} y^{\frac{1}{2}\left(1-\frac{1}{y}\right)} z^{\frac{1}{2}\left(1-\frac{1}{z}\right)} \leqslant 2 \cdot \frac{1}{2}(x-1+y-1+z-1)=x+y+z-3$.从而, $x^{\frac{1}{x}} y^{\frac{1}{y}} z^{\frac{1}{z}} \geqslant \frac{4 x y z}{(x+y+z-3)^{2}}$, 故命题成立.

  \item 由 $f(x, y, z)=\frac{1}{x^{2}}+\frac{1}{z^{2}}+2\left(\frac{1}{y^{2}}+\frac{1}{z^{2}}\right)$, 以及 $\sqrt{2} \geqslant \frac{1}{z}>\frac{\sqrt{2}}{x}$, 有 $\frac{1}{z^{2}}$ $\leqslant 2, \frac{z^{2}}{x^{2}}<\frac{1}{2}$. 又由 $x^{2}+3 z^{2} \geqslant 3$, 有 $\frac{2}{3}+\frac{2 z^{2}}{x^{2}} \geqslant \frac{2}{x^{2}}$. 于是 $\frac{1}{x^{2}}+\frac{1}{z^{2}}=\frac{2}{x^{2}}+\frac{1}{z^{2}}$ $-\frac{1}{x^{2}} \leqslant \frac{2}{3}+\frac{2 z^{2}}{x^{2}}+\frac{1}{z^{2}}\left(1-\frac{z^{2}}{x^{2}}\right) \leqslant \frac{2}{3}+\frac{2 z^{2}}{x^{2}}+2\left(1-\frac{z^{2}}{x^{2}}\right)=\frac{8}{3}$. 同理, $\frac{1}{y^{2}}+\frac{1}{z^{2}}$\\
$\leqslant \frac{13}{5}$. 于是 $f(x, y, z) \leqslant \frac{8}{3}+\frac{26}{5}=\frac{118}{15}$. 当 $x=\sqrt{\frac{3}{2}}, y=\sqrt{\frac{5}{3}}, z=\sqrt{\frac{1}{2}}$时, $f(x, y, z)=\frac{118}{15}$. 知 $x, y, z$ 满足条件 (1), (2), (3). 故 $f(x, y, z)$的最大值为 $\frac{118}{15}$.

  \item 令 $b_{i}=S-a_{i}$, 则 $\sum_{i=1}^{n} b_{i}=(n-1) S$. 由平均值不等式, 得 $\sum_{i=1}^{n} \frac{a_{i}}{S-a_{i}}=$ $\sum_{i=1}^{n}\left(\frac{a_{i}}{b_{i}}+1\right)-n=S \sum_{i=1}^{n} \frac{1}{b_{i}}-n \geqslant \frac{n S}{\sqrt[n]{b_{1} \cdots b_{n}}}-n \geqslant \frac{n^{2} S}{b_{1}+\cdots+b_{n}}-n=\frac{n}{n-1}$.当 $a_{1}=a_{2}=\cdots=a_{n}=1$ 时, 等号成立, 故最小值为 $\frac{n}{n-1}$.

  \item 令 $a_{i}=1+x_{1}+\cdots+x_{i}, 1 \leqslant i \leqslant n, a_{0}=1$. 则 $x_{i}=a_{i}-a_{i-1}$, $1 \leqslant i \leqslant n$. 以及 $f=\min \left\{\frac{a_{1}-1}{a_{1}}, \frac{a_{2}-a_{1}}{a_{2}}, \cdots, \frac{a_{n}-a_{n-1}}{a_{n}}\right\}=1-$ $\max \left\{\frac{1}{a_{1}}, \frac{a_{1}}{a_{2}}, \cdots, \frac{a_{n-1}}{a_{n}}\right\}$. 由于 $a_{n}=2$, 利用平均值不等式得 $\frac{1}{a_{1}}+\frac{a_{1}}{a_{2}}+\cdots+$ $\frac{a_{n-1}}{a_{n}} \geqslant n \sqrt[n]{\frac{1}{2}}$. 于是 $\max \left\{\frac{1}{a_{1}}, \frac{a_{1}}{a_{2}}, \cdots, \frac{a_{n-1}}{a_{n}}\right\} \geqslant \sqrt[n]{\frac{1}{2}}$, 从而, $f \leqslant 1-\sqrt[n]{\frac{1}{2}}$. 且当 $\frac{1}{a_{1}}=\frac{a_{1}}{a_{2}}=\cdots=\frac{a_{n-1}}{a_{n}}$, 即 $x_{i}=2^{\frac{i}{n}}-2^{\frac{i-1}{n}}, 1 \leqslant i \leqslant n$ 时, 等式成立. 故 $f$ 的最大值为 $1-\sqrt[n]{\frac{1}{2}}$.

  \item 不妨设 $x \geqslant y \geqslant z$. 如果 $256 y^{3} \geqslant x^{2} z$, 则 $\frac{x^{2}+256 y z}{y^{2}+z^{2}}-\frac{x^{2}}{y^{2}}=$ $\frac{z\left(256 y^{3}-x^{2} z\right)}{\left(y^{2}+z^{2}\right) y^{2}} \geqslant 0$. 以及 $\frac{y^{2}+256 z x}{z^{2}+x^{2}} \geqslant \frac{y^{2}}{x^{2}}, \frac{z^{2}+256 x y}{x^{2}+y^{2}} \geqslant \frac{256 x y}{x^{2}+y^{2}}$. 从而 $\sum_{\text {cyc }} \sqrt{\frac{x^{2}+256 y z}{y^{2}+z^{2}}} \geqslant \frac{x}{y}+\frac{y}{x}+16 \sqrt{\frac{x y}{x^{2}+y^{2}}}=\frac{x^{2}+y^{2}}{x y}+8 \sqrt{\frac{x y}{x^{2}+y^{2}}}+8$ $\sqrt{\frac{x y}{x^{2}+y^{2}}} \geqslant 12$. 且当 $x: y=(2+\sqrt{3}): 1, z=0$ 时, 等式成立. 如果 $256 y^{3}<$ $x^{2} z$. 由于 $x^{2} z \leqslant x^{2} y$, 所以, $256 y^{3}<x^{2} y, 256 y^{2}<x^{2}$. 从而可知 $\sum_{\text {cyc }} \sqrt{\frac{x^{2}+256 y z}{y^{2}+z^{2}}} \geqslant \sqrt{\frac{x^{2}+256 y z}{y^{2}+z^{2}}} \geqslant \sqrt{\frac{256 y^{2}+256 y z}{y^{2}+z^{2}}}=16 \sqrt{\frac{y^{2}+y z}{y^{2}+z^{2}}} \geqslant 16$ $>12$. 综上所述, $\sum_{\text {cyc }} \sqrt{\frac{x^{2}+256 y z}{y^{2}+z^{2}}} \geqslant 12$.

  \item 令 $b_{i}=2-a_{i} \geqslant 0(i=1,2, \cdots, n)$, 则 $\sum_{i=1}^{n} b_{i}=2 n-1$. 由平均值不等式, 得 $\sum_{i=1}^{n} \frac{a_{i}}{b_{i}}=\sum_{i=1}^{n}\left(\frac{a_{i}}{b_{i}}-1\right)-n=2 \sum_{i=1}^{n} \frac{1}{b_{i}}-n \geqslant \frac{2 n}{\sqrt[n]{b_{1} \cdots b_{n}}}-n \geqslant$ $\frac{2 n^{2}}{b_{1}+\cdots+b_{n}}-n=\frac{2 n^{2}}{2 n-1}-n=\frac{n}{2 n-1}$. 当 $a_{1}=\cdots=a_{n}=\frac{1}{n}$ 时, $\sum_{i=1}^{n} \frac{a_{i}}{2-a_{i}}=$ $\frac{n}{2 n-1}$, 故最小值为 $\frac{n}{2 n-1}$.

  \item 由平均值不等式, 得 $\left(x_{1} x_{2} \cdots, x_{n}\right)^{\frac{1}{2 n}}=\left[\left(a-x_{1}\right)\left(a-x_{2}\right) \cdots(a-\right.$ $\left.\left.x_{n}\right)\right]^{\frac{1}{n}} \leqslant a-\frac{x_{1}+\cdots+x_{n}}{n} \leqslant a-\left(x_{1} \cdots x_{n}\right)^{\frac{1}{n}}$. 令 $y=\left(x_{1} x_{2} \cdots x_{n}\right)^{\frac{1}{2 n}} \geqslant 0$, 则有 $y \leqslant a-y^{2}$, 即 $y^{2}+y-a \leqslant 0$. 解不等式得 $0 \leqslant y \leqslant \frac{-1+\sqrt{4 a+1}}{2}$, 故 $x_{1} x_{2} \cdots x_{n}$ 的最大值为 $\left(\frac{-1+\sqrt{4 a+1}}{2}\right)^{2 n}$.

  \item 由已知条件, 得 $2 \sum_{i=1}^{n} x_{i}^{2}+2 \sum_{i=1}^{n-1} x_{i} x_{i+1}=2$. 即 $x_{1}^{2}+\left(x_{1}+x_{2}\right)^{2}+$ $\left(x_{2}+x_{3}\right)^{2}+\cdots+\left(x_{n-2}+x_{n-1}\right)^{2}+\left(x_{n-1}+x_{n}\right)^{2}+x_{n}^{2}=2$. 对给定的正整数 $k$, $1 \leqslant k \leqslant n$, 由平均值不等式, 得 $\sqrt{\frac{x_{1}^{2}+\left(x_{1}+x_{2}\right)^{2}+\cdots+\left(x_{k-1}+x_{k}\right)^{2}}{k}} \geqslant$ $\frac{\left|x_{1}\right|+\left|x_{1}+x_{2}\right|+\cdots+\left|x_{k-1}+x_{k}\right|}{k} \geqslant \frac{\left|x_{1}-\left(x_{1}+x_{2}\right)+\cdots+(-1)^{k-1}\left(x_{k-1}+x_{k}\right)\right|}{k}=$ $\frac{\left|x_{k}\right|}{k}$. 所以 $\frac{x_{1}^{2}+\left(x_{1}+x_{2}\right)^{2}+\cdots+\left(x_{k-1}+x_{k}\right)^{2}}{k} \geqslant \frac{x_{k}^{2}}{k^{2}}$, 即 $x_{1}^{2}+\left(x_{1}+x_{2}\right)^{2}$ $+\cdots+\left(x_{k-1}+x_{k}\right)^{2} \geqslant \frac{x_{k}^{2}}{k}$. 同理, 可得 $\left(x_{k}+x_{k+1}\right)^{2}+\cdots+\left(x_{n-1}+x_{n}\right)^{2}+x_{n}^{2} \geqslant$ $\frac{x_{k}^{2}}{n-k+1}$. 将以上两式相加, 得 $\left|x_{k}\right| \leqslant \sqrt{\frac{2 k(n+1-k)}{n+1}}(k=1,2, \cdots, n)$,当且仅当 $x_{1}=-\left(x_{1}+x_{2}\right)=\left(x_{2}+x_{3}\right)=\cdots=(-1)^{k-1}\left(x_{k-1}+x_{k}\right)$ 及 $x_{k}+x_{k+1}=-\left(x_{k+1}+x_{k+2}\right)=\cdots=(-1)^{n-k} x_{n}$ 时, 等号成立, 即当且仅当 $x_{i}=$ $x_{k}(-1)^{i-k} \frac{i}{k}(i=1,2, \cdots, k-1)$ 且 $x_{j}=x_{k}(-1)^{j-k} \frac{n+1-j}{n-k+1}(j=k+1$, $k+2, \cdots, n)$ 时, $\left|x_{k}\right|=\sqrt{\frac{2 k(n+1-k)}{n+1}}$. 于是 $\left|x_{k}\right|_{\text {max }}=$ $\sqrt{\frac{2 k(n+1-k)}{n+1}}$

\end{enumerate}

\begin{center}
此处有图片 % \includegraphics[max width=\textwidth]{2024_05_22_4ff05a14ba9ad07b725fg-201}
\end{center}

(第 18 题)

\begin{enumerate}
  \setcounter{enumi}{17}
  \item 如图所示, 在 $\triangle A B C$ 中, 设 $A B=c, A C$ $=b, B C=a$, 且 $\angle B A C=\alpha$, 考虑 $\triangle A B C$ 的外接圆周上边 $B C$ 的对弧 $\overparen{B A C}$, 因为弧的中点 $D$ 是弧上离弦 $B C$ 最远的点, 所以对 $\triangle A B C$ 与 $\triangle D B C$ 的高 $A H=h$ 与 $D K$, 有 $h \leqslant D K=B K \cdot \cot \frac{\angle B D C}{2}$ $=\frac{a}{2} \cot \frac{\alpha}{2}$. 由平均值不等式, 得 $\frac{a b+a c+b c}{4 S} \geqslant$ $\frac{3}{4 S} \sqrt[3]{a^{2} b^{2} c^{2}}=\frac{3}{4} \sqrt[3]{\frac{a^{2} b^{2} c^{2}}{\left(\frac{1}{2} b c \sin \alpha\right)^{2} \cdot \frac{1}{2} a h}}=\frac{3}{2} \sqrt[3]{\frac{a}{h \sin ^{2} \alpha}} \geqslant \frac{3}{2} \sqrt[3]{\frac{2}{\sin ^{2} \alpha \cot \frac{\alpha}{2}}}$.令 $\cos \alpha=x$, 再由平均值不等式, 得 $\frac{1}{2} \sin ^{2} \alpha \cot \frac{\alpha}{2}=\sin \alpha \cos ^{2} \frac{\alpha}{2}=\frac{1}{2} \sin \alpha(1$ $+\cos \alpha)=\frac{1}{2} \sqrt{1-x^{2}}(1+x)=\frac{1}{2} \sqrt{(1+x)^{3}(1-x)}=$ $\frac{1}{2} \sqrt{27\left(\frac{1+x}{3}\right)^{3}(1-x)} \leqslant \frac{1}{2} \sqrt{27}\left[\frac{1}{4}\left(3 \cdot \frac{1+x}{3}+(1-x)\right)\right]^{2}=\frac{1}{2} \sqrt{27}\left(\frac{2}{4}\right)^{2}=$ $\left(\frac{\sqrt{3}}{2}\right)^{3}$. 从而 $\frac{a b+b c+c a}{4 S} \geqslant \frac{3}{2} \cdot \frac{2}{\sqrt{3}}=\sqrt{3}$.

  \item 记 $a=B C, b=A C, c=A B, S=S_{\triangle A B C}, S_{0}=S_{\triangle D E F}$. 由三角形平分线的性质, 得 $\frac{A F}{b}=\frac{B F}{a}=\frac{A F+B F}{b+a}=\frac{c}{a+b}$. 从而 $A F=\frac{b c}{a+b}$, 同理可得 $A E=\frac{b c}{a+c}$. 因此, $S_{\triangle A E F}=\frac{1}{2} A F \cdot A E \sin \angle B A C=\frac{1}{2} b c \sin \angle B A C \cdot$ $\frac{b c}{(a+b)(a+c)}=\frac{b c S}{(a+b)(a+c)}$. 同理可得 $S_{\triangle B D F}=\frac{a c S}{(a+b)(b+c)}, S_{\triangle C D E}=$ $\frac{a b S}{(a+c)(b+c)}$. 由平均值不等式, 得 $S-S_{0}=S_{\triangle A E F}+S_{\triangle B D F}+S_{\triangle C D E}=$ $\left[\frac{b c}{(a+b)(a+c)}+\frac{a c}{(b+a)(b+c)}+\frac{a b}{(c+a)(c+b)}\right] S \geqslant \frac{6 a b c S}{(a+b)(b+c)(c+a)}=$ $3\left[1-\frac{b c}{(a+b)(a+c)}-\frac{a c}{(a+b)(c+b)}-\frac{a b}{(a+c)(c+b)}\right] S=3\left(S-S_{\triangle A E F}-\right.$ $\left.S_{\triangle B D F}-S_{\triangle C D E}\right)=3 S_{0}$. 于是, $S_{0} \leqslant \frac{1}{4} S$, 即命题成立.

  \item 设 $\triangle A B C$ 的三个内角为 $\alpha, \beta, \gamma$, 则 $P=\frac{1}{2} R^{2}(\sin 2 \alpha+\sin 2 \beta+$ $\sin 2 \gamma)$. 由于 $\triangle A^{\prime} B^{\prime} C^{\prime}$ 的内角为 $\frac{\beta+\gamma}{2}, \frac{\alpha+\gamma}{2}, \frac{\alpha+\beta}{2}$, 所以 $Q=\frac{1}{2} R^{2}[\sin (\beta+$\\
$\gamma)+\sin (\alpha+\gamma)+\sin (\alpha+\beta)]$. 由平均值不等式, 得 $16 Q^{3}=2 R^{6}[\sin (\beta+\gamma)+$ $\sin (\alpha+\gamma)+\sin (\alpha+\beta)]^{3} \geqslant 2 R^{6} \cdot 27 \sin (\beta+\gamma) \sin (\alpha+\gamma) \sin (\alpha+\beta)=$ $27 R^{6}[\cos (\alpha-\beta)+\cos \gamma] \sin (\alpha+\beta)=\frac{27}{2} R^{6}[\sin (\alpha+\beta+\gamma)+\sin (\alpha+\beta-\gamma)+$ $\sin 2 \alpha+\sin 2 \beta]=\frac{27}{2} R^{6}(\sin 2 \alpha+\sin 2 \beta+\sin 2 \gamma)=27 R^{4} P$.

  \item 如图所示, 知 $S_{\triangle A B_{2} C_{1}}=S_{\triangle B C_{2} A_{1}}=S_{\triangle C A_{2} B_{1}}=S_{\triangle A B C}$, 所以 $\frac{G}{F}=$ $\frac{S_{A B C_{2} C_{1}}+S_{B C A_{2} A_{2}}+S_{A C B_{1} B_{2}}+4 F}{F}=\frac{S_{\triangle A A_{1} A_{2}}+S_{\triangle B B_{1} B_{2}}+S_{\triangle C C_{1} C_{2}}+F}{F}=1+$ $\frac{(b+a)(c+a)}{b c}+\frac{(a+b)(c+b)}{a c}+\frac{(a+c)(b+c)}{a b}=1+3+\frac{a}{b}+\frac{a}{c}+\frac{b}{a}+$ $\frac{b}{c}+\frac{c}{a}+\frac{c}{b}+\frac{a^{2}}{b c}+\frac{b^{2}}{a c}+\frac{c^{2}}{a b} \geqslant 4+9 \sqrt[9]{\frac{a}{b} \cdot \frac{a}{c} \cdot \frac{b}{a} \cdot \frac{b}{c} \cdot \frac{c}{a} \cdot \frac{c}{b} \cdot \frac{a^{2}}{b c} \cdot \frac{b^{2}}{a c} \cdot \frac{c^{2}}{a b}}$ $=13$.

\end{enumerate}

\begin{center}
此处有图片 % \includegraphics[max width=\textwidth]{2024_05_22_4ff05a14ba9ad07b725fg-202(1)}
\end{center}

(第 21 题)

\begin{center}
此处有图片 % \includegraphics[max width=\textwidth]{2024_05_22_4ff05a14ba9ad07b725fg-202}
\end{center}

(第 22 题)

\begin{enumerate}
  \setcounter{enumi}{21}
  \item 如图所示, 设 $A D$ 是较大的底边, $B H$ 是给定梯形 $A B C D$ 的高. 如果 $A B=C D=13$, 则 $A D+B C=2$, 且 $S_{\text {梯形 } A B C D}=B H \cdot \frac{A D+B C}{2} \leqslant 13 \cdot \frac{2}{2}=$ $13<27$, 不可能. 因此, $A D=13$. 记 $A B=x$, 则 $B C=28-13-2 x=15-2 x$, $A H=x-1, B H=\sqrt{2 x-1}$. 由平均值不等式, 得 $S_{\text {栏形 } A B C D}=\sqrt{2 x-1} \cdot \frac{28-2 x}{2}$ $=\sqrt{(2 x-1)(14-x)^{2}} \leqslant \sqrt{\left[\frac{(2 x-1)+(14-x)+(14-x)}{3}\right]^{3}}=27$. 当且仅当 $2 x-1=14-x$, 即 $x=5$, 也是 $A B=B C=C D=5$ 时, $S_{\text {栏形 } A B C D}=27$,而等式 $S_{\text {㭢形 } A B C D}=27.001$ 是不可能成立的.

  \item 设 $M$ 为弦 $C D$ 的中点. 连结 $C D 、 M O$, 则 $\triangle P O M$ 为等腰直角三角形, 且 $M P=M O$. 不难得到 $P C^{2}+P D^{2}=(M C-M P)^{2}+(M C+M P)^{2}=$ $20 O^{2}=\frac{1}{2}$. 同理 $Q E^{2}+Q F^{2}=\frac{1}{2}$. 由平均值不等式得 $P C \cdot Q E+P D \cdot Q F \leqslant$

\end{enumerate}
$$
\frac{P C^{2}+Q E^{2}}{2}+\frac{P D^{2}+Q F^{2}}{2}=\frac{1}{2} \text {. 故 } 2 P C \cdot Q E+2 P D \cdot Q F \leqslant 1 \text {. }
$$

\begin{enumerate}
  \setcounter{enumi}{23}
  \item 因为 $a 、 b 、 c$ 为正有理数,故存在 $m \in \mathbf{N}$, 使 $m a, m b, m c$ 为正整数,又 $a 、 b 、 c$ 为三边之长, 有 $1+\frac{b-c}{a}>0,1+\frac{c-a}{b}>0,1+\frac{a-b}{c}>0$. 由平均值不 等 式, 得 $\left[\left(1+\frac{b-c}{a}\right)^{m a}\left(1+\frac{c-a}{b}\right)^{m b}\left(1+\frac{a-b}{c}\right)^{m c}\right]^{\frac{1}{m a+m b+m c}} \leqslant$ $\frac{m a\left(1+\frac{b-c}{a}\right)+m b\left(1+\frac{c-a}{b}\right)+m c\left(1+\frac{a-b}{c}\right)}{m a+m b+m c}=1$.

  \item 首先求最大值, 由 平均值不 等式, 得 $\sqrt[n]{x_{1}^{2} x_{2}^{2} \cdots x_{n}^{2}} \leqslant$ $\frac{x_{1}^{2}+x_{2}^{2}+\cdots+x_{n}^{2}}{n}=\frac{1}{n}$. 当 $x_{1}=x_{2}=\cdots=x_{n}=\frac{1}{\sqrt{n}}>\frac{1}{n}(n \geqslant 2)$ 时等号成立. 所以最大值为 $n^{-\frac{n}{2}}$. 再求最小值. 令 $y_{1}=x_{1}, \cdots, y_{n-2}=x_{n-2}, y_{n-1}=$ $\sqrt{x_{n-1}^{2}+x_{n}^{2}-\frac{1}{n^{2}}}, y_{n}=\frac{1}{n}$, 则 $y_{i} \geqslant \frac{1}{n}, i=1,2, \cdots, n$, 且 $y_{1}^{2}+\cdots+y_{n}^{2}=$ $x_{1}^{2}+\cdots+x_{n}^{2}=1$. 由于 $y_{n-1}^{2} y_{n}^{2}-x_{n-1}^{2} x_{n}^{2}=-\left(x_{n-1}^{2}-\frac{1}{n^{2}}\right)\left(x_{n}^{2}-\frac{1}{n^{2}}\right) \leqslant 0$, 所以 $y_{1} y_{2} \cdots y_{n-2} y_{n-1} y_{n} \leqslant x_{1} x_{2} \cdots x_{n-2} x_{n-1} x_{n}$. 重复这个过程 $n-1$ 次, 得 $x_{1} x_{2} \cdots x_{n} \geqslant\left(\frac{1}{n}\right)^{n-1} \sqrt{1-\frac{n-1}{n^{2}}}=\frac{\sqrt{n^{2}-n+1}}{n^{n}}$. 当 $x_{1}=\cdots=x_{n-1}=\frac{1}{n}$, $x_{n}=\frac{\sqrt{n^{2}-n+1}}{n}$ 时, 等号成立. 故最小值为 $\frac{\sqrt{n^{2}-n+1}}{n^{n}}$.

  \item 易知 $a^{2}+b^{2}+3 c^{2}+2 a b-4 b c=(a+b-c)^{2}+2 c^{2}+2 a c-2 b c=$ $(a+b-c)^{2}+2 c(a+c-b)$. 令 $I=\frac{(a+b-c)^{2}+2 c(a+c-b)}{2 c(a+b-c)}=\frac{a+b-c}{2 c}$ $+\frac{a+c-b}{a+b-c}$, 由于 $a \geqslant \frac{1}{3}(b+c)$, 所以 $a \geqslant \frac{1}{4}(a+b-c)+\frac{c}{2}$. 于是 $a+c-$ $b=2 a-(a+b-c) \geqslant-\frac{1}{2}(a+b-c)+c$. 由此可知 $I \geqslant \frac{a+b-c}{2 c}-$ $\frac{\frac{1}{2}(a+b-c)}{a+b-c}+\frac{c}{a+b-c}=-\frac{1}{2}+\frac{a+b-c}{2 c}+\frac{c}{a+b-c} \geqslant-\frac{1}{2}+2 \sqrt{\frac{1}{2}}=$ $\sqrt{2}-\frac{1}{2}$. 即 $\frac{a c+b c-c^{2}}{a^{2}+b^{2}+3 c^{2}+2 a b-4 b c}=\frac{1}{2 I} \leqslant \frac{1}{2 \sqrt{2}-1}=\frac{2 \sqrt{2}+1}{7}$. 所以, $\lambda \geqslant \frac{2 \sqrt{2}+1}{7}$. 另一方面, 当 $a=\frac{\sqrt{2}}{4}+\frac{1}{2}, b=\frac{3 \sqrt{2}}{4}+\frac{1}{2}, c=1$, 则 $a c+b c-$\\
$c^{2}=\sqrt{2}, a^{2}+b^{2}+3 c^{2}+2 a b-4 b c=4-\sqrt{2}$, 所以 $\frac{1}{2 I}=\frac{\sqrt{2}}{4-\sqrt{2}}=\frac{2 \sqrt{2}+1}{7}$.故 $\lambda=\frac{2 \sqrt{2}+1}{7}$.

  \item 显然 $2 j+1=(j+1)^{2}-j^{2}$, 由平均值不等式, 得 $\sum_{j=1}^{n} \frac{2 j+1}{j^{2}}=$ $\sum_{j=1}^{n}\left[\frac{(j+1)^{2}}{j^{2}}-1\right]=\sum_{j=1}^{n} \frac{(j+1)^{2}}{j^{2}}-n \geqslant n\left[\frac{2^{2}}{1^{2}} \cdot \frac{3^{2}}{2^{2}} \cdot \cdots \cdot \frac{(n+1)^{2}}{n^{2}}\right]^{\frac{1}{n}}-n=$ $n\left[(n+1)^{\frac{2}{n}}-1\right]$.

  \item 不妨设 $A \geqslant 60^{\circ}$, 则 $B+C \leqslant 180^{\circ}-60^{\circ}=120^{\circ} . \sin 3 A+\sin 3 B+$ $\sin 3 C=\sin 3 A+2 \sin \frac{3}{2}(B+C) \cos \frac{3}{2}(B-C) \leqslant \sin 3 A+2 \sin \frac{3}{2}(B+C)$. 记 $\alpha=$ $\frac{3}{2}(B+C)$, 则 $0 \leqslant \alpha \leqslant 180^{\circ}$, 且 $A=180^{\circ}-(B+C)=180^{\circ}-\frac{2}{3} \alpha$. 于是 $\sin 3 A+$ $\sin 3 B+\sin 3 C \leqslant \sin \left(3 \times 180^{\circ}-2 \alpha\right)+2 \sin \alpha=\sin 2 \alpha+2 \sin \alpha=2 \sin \alpha(1+$ $\cos \alpha)=8 \sin \frac{\alpha}{2} \cos ^{3} \frac{\alpha}{2}$. 由平均值不等式, 得 $\sin \frac{\alpha}{2} \cos ^{3} \frac{\alpha}{2}=\sqrt{\sin ^{2} \frac{\alpha}{2} \cos ^{6} \frac{\alpha}{2}}=$ $\sqrt{\frac{1}{3} \cdot 3 \sin ^{2} \frac{\alpha}{2} \cos ^{6} \frac{\alpha}{2}} \leqslant \sqrt{\frac{1}{3}\left[\frac{3 \sin ^{2} \frac{\alpha}{2}+\cos ^{2} \frac{\alpha}{2}+\cos ^{2} \frac{\alpha}{2}+\cos ^{2} \frac{\alpha}{2}}{4}\right]^{4}} \leqslant \frac{3 \sqrt{3}}{16}$. 所以 $\sin 3 A+\sin 3 B+\sin 3 C \leqslant \frac{3}{2} \sqrt{3}$, 且当且仅当 $A=140^{\circ}, B=C=20^{\circ}$ 时, 等号成立.

  \item 由平均值不等式, 得 $\csc ^{2} \frac{\alpha}{2}+\csc ^{2} \frac{\beta}{2}+\csc ^{2} \frac{\gamma}{2} \geqslant 3\left(\csc \frac{\alpha}{2} \csc \frac{\beta}{2} \csc \right.$ $\left.\frac{\gamma}{2}\right)^{\frac{2}{3}}$, 当且仅当 $\alpha=\beta=\gamma$ 时等号成立. 再由平均值不等式及凸函数性质, 得 $\left(\sin \frac{\alpha}{2} \sin \frac{\beta}{2} \sin \frac{\gamma}{2}\right)^{\frac{1}{3}} \leqslant \frac{\sin \frac{\alpha}{2}+\sin \frac{\beta}{2}+\sin \frac{\gamma}{2}}{3} \leqslant \sin \frac{\frac{\alpha}{2}+\frac{\beta}{2}+\frac{\gamma}{2}}{3}=\sin \frac{\pi}{6}=$ $\frac{1}{2}$, 因此 $\csc ^{2} \frac{\alpha}{2}+\csc ^{2} \frac{\beta}{2}+\csc ^{2} \frac{\gamma}{2} \geqslant 3\left(\sin \frac{\alpha}{2} \sin \frac{\beta}{2} \sin \frac{\gamma}{2}\right)^{-\frac{2}{3}} \geqslant 3\left(\frac{1}{2}\right)^{-2}=$ 12. 当且仅当 $\alpha=\beta=\gamma$ 时, 等号成立.

  \item 令 $x=\tan \frac{A}{2}, y=\tan \frac{B}{2}, z=\tan \frac{C}{2}$. 这里 $A, B, C \in[0, \pi)$. 由于 $\tan \left(\frac{A}{2}+\frac{B}{2}+\frac{C}{2}\right)=\frac{\tan \frac{A}{2}+\tan \frac{B}{2}+\tan \frac{C}{2}-\tan \frac{A}{2} \tan \frac{B}{2} \tan \frac{C}{2}}{1-\tan \frac{A}{2} \tan \frac{B}{2}-\tan \frac{A}{2} \tan \frac{C}{2}-\tan \frac{B}{2} \tan \frac{C}{2}}$, 所\\
以 $\cot \left(\frac{A}{2}+\frac{B}{2}+\frac{C}{2}\right)=\frac{1-(x y+y z+z x)}{x+y+z-x y z}$. 由已知条件知 $y 、 z$ 不全为 0 且 $0 \leqslant y z \leqslant 1$, 从而 $x+y+z>x \geqslant x y z \geqslant 0$, 于是 $\cot \left(\frac{A}{2}+\frac{B}{2}+\frac{C}{2}\right)=0$. 所以 $0<\frac{1}{2}(A+B+C)<\frac{3 \pi}{2}$, 得 $\frac{1}{2}(A+B+C)=\frac{\pi}{2}$, 即 $A+B+C=\pi$. 又 $x\left(1-y^{2}\right)\left(1-z^{2}\right)+y\left(1-z^{2}\right)\left(1-x^{2}\right)+z\left(1-x^{2}\right)\left(1-y^{2}\right)=$ $\frac{\sin A \cos B \cos C+\sin B \cos C \cos A+\sin C \cos A \cos B}{2 \cos ^{2} \frac{A}{2} \cos ^{2} \frac{B}{2} \cos ^{2} \frac{C}{2}}=4 \tan \frac{A}{2} \tan \frac{B}{2} \tan \frac{C}{2}$ $=4 x y z$, 由平均值不等式, 得 $(x y z)^{2} \leqslant\left(\frac{x y+y z+z x}{3}\right)^{3}=\frac{1}{27}$, 从而 $x y z \leqslant$ $\frac{1}{9} \sqrt{3}$

  \item 已知 $\left(1+\frac{1}{k}\right)^{k}$ 单调增加且收玫于 e. 对任意 $i \in \mathbf{N}$, 有 $i\left(1+\frac{1}{i}\right)^{i} \leqslant$ $i \mathrm{e}$, 记 $b_{i}=i\left(1+\frac{1}{i}\right)^{i}$. 则 $\frac{b_{i}}{i} \leqslant \mathrm{e}$. 由 $b_{1} b_{2} \cdots b_{k}=(1+k)^{k}$, 得 $\sqrt[k]{a_{1} \cdots a_{k}}=$ $\frac{1}{1+k} \sqrt[k]{\left(a_{1} b_{1}\right) \cdots\left(a_{k} b_{k}\right)}$. 由平均值不等式, 得 $\sqrt[k]{a_{1} \cdots a_{k}} \leqslant \frac{1}{k(k+1)} \sum_{i=1}^{k} a_{i} b_{i}=$ $\left(\frac{1}{k}-\frac{1}{k+1}\right) \sum_{i=1}^{k} a_{i} b_{i}, \quad \sum_{k=1}^{n} \sqrt[k]{a_{1} \cdots a_{k}} \leqslant \sum_{k=1}^{n}\left(\frac{1}{k}-\frac{1}{k+1}\right) \sum_{i=1}^{k} a_{i} b_{i}=$ $\sum_{i=1}^{n} a_{i} b_{i} \sum_{k=i}^{n}\left(\frac{1}{k}-\frac{1}{k+1}\right)=\sum_{i=1}^{n}\left(\frac{1}{i}-\frac{1}{n+1}\right) b_{i} a_{i}<\sum_{i=1}^{n} \frac{b_{i}}{i} a_{i}<\mathrm{e} \sum_{i=1}^{n} a_{i}$.

  \item (1) 由于 $(n-1) p^{2}-2 n q=(n-1)\left(\sum_{i=1}^{n} x_{i}\right)^{2}-2 n \sum_{1 \leqslant i<j \leqslant n} x_{i} x_{j}=$ $(n-1) \sum_{i=1}^{n} x_{i}^{2}-2 \sum_{1 \leqslant i<j \leqslant n} x_{i} x_{j}=\sum_{1 \leqslant i<j \leqslant n}\left(x_{i}-x_{j}\right)^{2}$, 所以 $\frac{n-1}{n} p^{2}-2 q \geqslant 0$.

\end{enumerate}

(2) $\left|x_{i}-\frac{p}{n}\right|=\frac{n-1}{n}\left|\frac{1}{n-1} \sum_{k=1}^{n}\left(x_{i}-x_{k}\right)\right|$. 由幂平均不等式, 得 $\left|x_{i}-\frac{p}{n}\right| \leqslant \frac{n-1}{n} \sqrt{\frac{1}{n-1} \sum_{k=1}^{n}\left(x_{i}-x_{k}\right)^{2}} \leqslant \frac{n-1}{n} \sqrt{\frac{1}{n-1} \sum_{1 \leqslant i<j \leqslant n}\left(x_{i}-x_{k}\right)^{2}}$.由(1)的结果, 得 $\left|x_{i}-\frac{p}{n}\right| \leqslant \frac{n-1}{n} \sqrt{p^{2}-\frac{2 n}{n-1} q}$.

\begin{enumerate}
  \setcounter{enumi}{32}
  \item 设 $f(x)$ 的三个根为 $\alpha, \beta, \gamma$, 并设 $0 \leqslant \alpha \leqslant \beta \leqslant \gamma$, 则 $x-a=x+\alpha+$ $\beta+\gamma, f(x)=(x-\alpha)(x-\beta)(x-\gamma)$. (1) 当 $0 \leqslant x \leqslant \alpha$ 时, 则有 $-f(x)=$ $(\alpha-x)(\beta-x)(\gamma-x) \leqslant\left(\frac{\alpha+\beta+\gamma-3 x}{3}\right)^{3} \leqslant \frac{1}{27}(x+\alpha+\beta+\gamma)^{3}=\frac{1}{27}(x-$\\
$a)^{3}$, 所以 $f(x) \geqslant-\frac{1}{27}(x-a)^{3}$. 当 $x=0, \alpha=\beta=\gamma$ 时,等号成立; (2) 当 $\alpha \leqslant x \leqslant \beta$ 或 $x>\gamma$ 时, $f(x)=(x-\alpha)(x-\beta)(x-\gamma)>0>-\frac{1}{27}(x-a)^{3}$. (3) 当 $\beta \leqslant x \leqslant \gamma$ 时, $-f(x)=(x-\alpha)(x-\beta)(\gamma-x) \leqslant\left(\frac{x+\gamma-\alpha-\beta}{3}\right)^{3} \leqslant$ $\frac{1}{27}(x+\alpha+\beta+\gamma)^{3}=\frac{1}{27}(x-a)^{3}$. 所以 $f(x) \geqslant-\frac{1}{27}(x-a)^{3}$. 当 $\alpha=\beta=0$, $\gamma=2 x$ 时, 等号成立. 综上所述, $\lambda$ 的最大值 $-\frac{1}{27}$.

  \item 令 $a=\frac{1+y z}{y-z}, b=\frac{1+x y}{x-y}$, 则 $\frac{1-a b}{a+b}=\frac{1+x z}{z-x}$. 由于 $\left|\frac{1+y z}{y-z}+\frac{1+z x}{z-x}+\frac{1+x y}{x-y}\right|=\left|a+b+\frac{1-a b}{a+b}\right|=\left|\frac{(a+b)^{2}+1-a b}{a+b}\right| \geqslant$ $\left|\frac{\frac{3}{4}(a+b)^{2}+1}{a+b}\right| \geqslant\left|\frac{2 \sqrt{\frac{3}{4}(a+b)^{2}}}{a+b}\right|=\sqrt{3}$. 当 $a=b, \frac{\sqrt{3}}{2}(a+b)=1$, 即 $a=$ $b=\frac{\sqrt{3}}{3}$ 时, 等式成立. 即当 $\frac{1+x y}{x-y}=\frac{1+y z}{y-z}=\frac{1+z x}{z-x}=\frac{1}{\sqrt{3}}$ 时, 等式成立. 取 $x=0, y=-\sqrt{3}, z=\sqrt{3}$, 则 $x 、 y 、 z$ 满足条件, 故最小值为 $\sqrt{3}$.

  \item 取 $a=b=c=\frac{1}{3}$, 则 $\sum_{\mathrm{cyc}} \sqrt{a+b^{2}}=\sum_{\mathrm{cyc}} \sqrt{\frac{4}{9}}=\sum_{\mathrm{cyc}} \frac{2}{3}=2$. 下面证明 $\sum_{\text {cyc }} \sqrt{a+b^{2}} \geqslant 2 \Leftrightarrow \sum_{\text {cyc }}\left(\sqrt{a+b^{2}}-b\right) \geqslant 1 \Leftrightarrow \sum_{\text {cyc }} \frac{a}{b+\sqrt{a+b^{2}}} \geqslant 1$, 由平均值 不 等 式 得 $\frac{a}{b+\sqrt{a+b^{2}}}=\frac{a(a+b)}{b(a+b)+(a+b) \sqrt{a+b^{2}}} \geqslant$ $\frac{2 a(a+b)}{2 b(a+b)+(a+b)^{2}+a+b^{2}}=\frac{a(a+b)}{2 a^{2}+5 a b+4 b^{2}+c a}$. 只需要证明 $\sum_{\text {cyc }}$ $\frac{a(a+b)}{2 a^{2}+5 a b+4 b^{2}+c a} \geqslant 1 \cdots$ (1) $\Leftrightarrow 4 \sum_{\text {cyc }} a^{4} b^{2}+3 \sum_{\text {cyc }} a^{3} b^{2} c-19 \sum_{\text {cyc }} a^{2} b^{3} c+$ $16 \sum_{\text {cyc }} a^{4} b c-12 a^{2} b^{2} c^{2} \geqslant 0 \Leftrightarrow 4\left(\sum_{\text {cyc }} a^{4} b^{2}-\sum_{\text {cyc }} a^{2} b^{3} c\right)+3\left(\sum_{\text {cyc }} a^{3} b^{2} c-3 a^{2} b^{2} c^{2}\right)+$ $15\left(\sum_{\text {cyc }} a^{4} b c-\sum_{\text {cyc }} a^{2} b^{3} c\right)+\left(\sum_{\text {cyc }} a^{4} b c-3 a^{2} b^{2} c^{2}\right) \geqslant 0$. 由平均值不等式, 最后一个不等式成立. 从而(1)成立. 故 $\sum_{\mathrm{cyc}} \sqrt{a+b^{2}}$ 时最小值为 2 .

  \item 由于 $\left(\sum_{\text {cyc }} a^{\frac{15}{8}}\right)^{2} \geqslant\left(a^{15 / 8}+3 b^{5 / 8} c^{5 / 8} d^{5 / 8}\right)^{2}=a^{\frac{15}{4}}+6 a^{15 / 8} b^{5 / 8} c^{5 / 8} d^{5 / 8}+$ $9 b^{5 / 4} c^{5 / 4} d^{5 / 4} \geqslant a^{15 / 4}+15 \sqrt[15]{a^{45 / 4} b^{15} c^{15} d^{15}}=a^{15 / 4}+15 a^{3 / 4} b c d$. 所以 $\left(\sum_{\mathrm{cyc}} a^{15 / 8}\right)^{2} a^{3}$\\
$\geqslant a^{\frac{27}{4}}+15 a^{\frac{15}{4}} b c d=a^{15 / 4}\left(a^{3}+15 b c d\right)$, 即 $\frac{a^{3}}{a^{3}+15 b c d} \geqslant \frac{a^{15 / 4}}{\left(\sum_{\mathrm{cyc}} a^{15 / 8}\right)^{2}}$, 从而, $\sqrt{\frac{a^{3}}{a^{3}+15 b c d}} \geqslant \frac{a^{15 / 8}}{\sum_{\mathrm{cyc}} a^{15 / 8}}$. 关于 $b 、 c 、 d$ 有类似的不等式, 求和得到 $\sum_{\mathrm{cyc}}$ $\sqrt{\frac{a^{3}}{a^{3}+15 b c d}} \geqslant \sum_{\text {cyc }} \frac{a^{15 / 8}}{\sum_{\text {cyc }} a^{15 / 8}}=1$. 故命题成立.

  \item 不妨设 $c \geqslant b \geqslant a>0$. 则 $\sqrt[n]{b+c} \geqslant \sqrt[n]{c+a} \geqslant \sqrt[n]{a+b}$ 以及 $\frac{a}{\sqrt[n]{b+c}}$ $\leqslant \frac{b}{\sqrt[n]{c+a}} \leqslant \frac{c}{\sqrt[n]{a+b}}$. 由切比雪夫不等式得 $\sum_{\text {cyc }} a=\sum_{\text {cyc }}\left(\sqrt[n]{b+c} \cdot \frac{a}{\sqrt[n]{b+c}}\right)$ $\leqslant \frac{1}{3} \sum_{\text {cyc }} \sqrt[n]{b+c} \cdot \sum_{\text {cyc }} \frac{a}{\sqrt[n]{b+c}}$. 再由幂平均不等式得 $\frac{1}{3} \sum_{\mathrm{cyc}} \sqrt[n]{b+c} \leqslant$ $\sqrt[n]{\frac{1}{3} \sum_{\text {cyc }}(b+c)}=\sqrt[n]{\frac{2}{3} \sum_{\text {cyc }} a}$, 于是, $\sum_{\text {cyc }} a \leqslant \sqrt[n]{\frac{2}{3} \sum_{\text {cyc }} a} \cdot \sum_{\text {cyc }} \frac{a}{\sqrt[n]{b+c}}$, 从而, $\sum_{\text {cyc }} \frac{a}{\sqrt[n]{b+c}} \geqslant \frac{\sum_{\text {cyc }} a}{\sqrt[n]{\frac{2}{3} \sum_{\text {cyc }} a}}=\sqrt[n]{\frac{3}{2}\left(\sum_{\text {cyc }} a\right)^{n-1}}$ 由于 $\sum_{\text {cyc }} a \geqslant 3 \sqrt[3]{a b c}=3$, 所以 $\sum_{\text {cyc }} \frac{a}{\sqrt[n]{b+c}} \geqslant \frac{3}{\sqrt[n]{2}}$

\end{enumerate}

\section*{习 题 3}
\begin{enumerate}
  \item 由于关于 $a 、 b 、 c$ 的对称性, 不妨设 $a \geqslant b \geqslant c$, 则 $\frac{a^{2 a} b^{2 b} c^{2 c}}{a^{b+c} b^{a+c} c^{a+b}}=$ $\left(\frac{a}{b}\right)^{a-b}\left(\frac{b}{c}\right)^{b-c}\left(\frac{a}{c}\right)^{a-c} \geqslant 1$, 所以 $a^{2 a} b^{2 b} c^{2 c} \geqslant a^{b+c} b^{c+a} c^{a+b}$.

  \item 由柯西不等式, 可得原式的左边 $\geqslant$

\end{enumerate}

$\frac{16}{(4 a+3 b+c)+(3 a+b+4 d)+(a+4 c+3 d)+(4 b+3 c+d)}=2$.

\begin{enumerate}
  \setcounter{enumi}{2}
  \item 由柯 西不 等 式, 有 $\left(\sum_{i=1}^{n} \frac{1}{a+i b}\right)^{2} \leqslant n \sum_{i=1}^{n}\left(\frac{1}{a+i b}\right)^{2}<$ $n\left\{\frac{1}{a(a+b)}+\frac{1}{(a+b)(a+2 b)}+\cdots+\frac{1}{[a+(n-1) b](a+n b)}\right\}=$ $\frac{n}{b}\left[\left(\frac{1}{a}-\frac{1}{a+b}\right)+\left(\frac{1}{a+b}-\frac{1}{a+2 b}\right)+\cdots+\left(\frac{1}{a+(n-1) b}-\frac{1}{a+n b}\right)\right]=$ $\frac{n}{b}\left(\frac{1}{a}-\frac{1}{a+n b}\right)=\frac{n^{2}}{a(a+n b)}$, 即 $\sum_{i=1}^{n} \frac{1}{a+i b}<\frac{n}{\sqrt{a(a+n b)}}$.

  \item 因为 $a b+b c+c a=3$, 所以 $(a+b+c)^{2} \geqslant 3(a b+b c+c a)=(a b+$ $b c+c a)^{2}$, 即 $a+b+c \geqslant a b+b c+c a$. 由柯西不等式 $[a(b+c)+b(c+a)+c(a$ $+b)]\left[\frac{a}{b+c}+\frac{b}{c+a}+\frac{c}{a+b}\right] \geqslant(a+b+c)^{2} \geqslant(a+b+c)(a b+b c+c a)$ 故 $\frac{a}{b+c}$ $+\frac{b}{c+a}+\frac{c}{a+b} \geqslant \frac{(a+b+c)(a b+b c+c a)}{2(a b+b c+c a)}=\frac{a+b+c}{2} \cdots$ (1). 又 $\frac{a+b}{4}+\frac{a^{2} b^{2}}{a+b}$ $\geqslant 2 \sqrt{\frac{a^{2} b^{2}}{4}}=a b \cdots$ (2), $\frac{b+c}{4}+\frac{b^{2} c^{2}}{b+c} \geqslant 2 \sqrt{\frac{b^{2} c^{2}}{4}}=b c \cdots$ (3), $\frac{c+a}{4}+\frac{c^{2} a^{2}}{c+a} \geqslant$ $2 \sqrt{\frac{c^{2} a^{2}}{4}}=c a \cdots$ (4). (2)+(3)+(4)得: $\frac{a+b+c}{2}+\frac{a^{2} b^{2}}{a+b}+\frac{b^{2} c^{2}}{b+c}+\frac{c^{2} a^{2}}{c+a} \geqslant a b+b c$ $+c a=3$. 结合(1)得 $\frac{a}{b+c}+\frac{b}{c+a}+\frac{c}{a+b}+\frac{a^{2} b^{2}}{a+b}+\frac{b^{2} c^{2}}{b+c}+\frac{c^{2} a^{2}}{c+a} \geqslant 3$.

  \item 等式左边的分母显然为正数. 由柯西不等式得 $\frac{a}{a^{2}-b c+1}+$ $\frac{b}{b^{2}-c a+1}+\frac{c}{c^{2}-a b+1}=\frac{a^{2}}{a^{3}-a b c+a}+\frac{b^{2}}{b^{3}-a b c+b}+\frac{c^{2}}{c^{3}-a b c+c} \geqslant$ $\frac{(a+b+c)^{2}}{a^{3}+b^{3}+c^{3}+a+b+c-3 a b c}=\frac{(a+b+c)^{2}}{(a+b+c)\left(a^{2}+b^{2}+c^{2}-a b-b c-c a\right)+(a+b+c)}$ $=\frac{a+b+c}{a^{2}+b^{2}+c^{2}-a b-b c-c a+1}=\frac{a+b+c}{a^{2}+b^{2}+c^{2}+2(a b+b c+c a)}=$ $\frac{1}{a+b+c}$. 命题得证.

  \item 由柯西不等式得 $\left(1+\frac{y}{x}+\frac{z}{x}\right)(1+x y+x z) \geqslant(1+y+z)^{2} \Rightarrow$ $\frac{1+x y+x z}{(1+y+z)^{2}} \geqslant \frac{x}{x+y+z}$. 同理, $\frac{1+y z+y x}{(1+z+x)^{2}} \geqslant \frac{y}{x+y+z} \cdot \frac{1+z x+z y}{(1+x+y)^{2}} \geqslant$ $\frac{z}{x+y+z}$. 上述三个不等式相加即得 $\frac{1+x y+x z}{(1+y+z)^{2}}+\frac{1+y z+y x}{(1+z+x)^{2}}+$ $\frac{1+z x+z y}{(1+x+y)^{2}} \geqslant 1$.

  \item 原不等式等价于 $\sum_{\mathrm{cyc}} \frac{1}{x^{5}+y^{2}+z^{2}} \leqslant \frac{3}{x^{2}+y^{2}+z^{2}}$. 利用 $x y z \geqslant 1$ 及柯西不等式得 $\left(x^{5}+y^{2}+z^{2}\right) \cdot\left(y z+y^{2}+z^{2}\right) \geqslant\left(\sum_{\mathrm{cyc}} x^{2}\right)^{2}$. 而 $\sum_{\mathrm{cyc}}\left(y z+y^{2}+z^{2}\right) \leqslant$ $\sum_{\text {cyc }}\left(\frac{y^{2}+z^{2}}{2}+y^{2}+z^{2}\right)=3 \sum_{\text {cyc }} x^{2}$. 代人即得结果.

  \item 由 柯 西 不 等 式 知 $\left(b^{2}+c+c^{2}+a+a^{2}+b\right) \cdot$ $\left(\frac{a^{4}}{b^{2}+c}+\frac{b^{4}}{c^{2}+a}+\frac{c^{4}}{c^{2}+b}\right) \geqslant\left(a^{2}+b^{2}+c^{2}\right)^{2}$. 故 $\frac{a^{4}}{b^{2}+c}+\frac{b^{4}}{c^{2}+a}+\frac{c^{4}}{a^{2}+b} \geqslant$ $\frac{\left(a^{2}+b^{2}+c^{2}\right)^{2}}{a^{2}+b^{2}+c^{2}+3}$. 令 $a^{2}+b^{2}+c^{2}=x$. 易证 $x \geqslant 3$. 故 $\frac{x^{2}}{3+x} \geqslant \frac{3}{2} \Leftrightarrow 2 x^{2} \geqslant$ $9+3 x \Leftrightarrow 2 x^{2}-3 x-9 \geqslant 0 \Leftrightarrow(2 x+3)(x-3) \geqslant 0$. 显然成立. 于是, $\frac{a^{4}}{b^{2}+c}+$ $\frac{b^{4}}{c^{2}+a}+\frac{c^{4}}{a^{2}+b} \geqslant \frac{3}{2}$.

  \item 记 $A=\frac{a^{2}}{a b^{2}(4-a b)}+\frac{b^{2}}{b c^{2}(4-b c)}+\frac{c^{2}}{c a^{2}(4-c a)}, B=\frac{b^{2}}{a b^{2}(4-a b)}$ $+\frac{c^{2}}{b c^{2}(4-b c)}+\frac{a^{2}}{c a^{2}(4-c a)}$. 欲证明原不等式, 只需证明 $A \geqslant 1, B \geqslant 1$. 由柯西-施瓦兹不等式得 $\left(\frac{4-a b}{a}+\frac{4-b c}{b}+\frac{4-a c}{c}\right) A \geqslant\left(\frac{1}{a}+\frac{1}{b}+\frac{1}{c}\right)^{2}$. 设 $k=\frac{1}{a}+\frac{1}{b}+\frac{1}{c}$, 则 $A \geqslant \frac{k^{2}}{4 k-3}$. 由 $(a+b+c)\left(\frac{1}{a}+\frac{1}{b}+\frac{1}{c}\right) \geqslant 3^{2} \Rightarrow k=$ $\frac{1}{a}+\frac{1}{b}+\frac{1}{c} \geqslant 3 \Rightarrow(k-3)(k-1) \geqslant 0 \Rightarrow k^{2}-4 k+3 \geqslant 0 \Rightarrow A=\frac{k^{2}}{4 k-3} \geqslant 1$.又因为 $B=\frac{1}{a(4-a b)}+\frac{1}{b(4-b c)}+\frac{1}{c(4-c a)}$, 则 $\left(\frac{4-a b}{a}+\frac{4-b c}{b}+\right.$ $\left.\frac{4-c a}{c}\right) B \geqslant\left(\frac{1}{a}+\frac{1}{b}+\frac{1}{c}\right)^{2}$. 故 $B \geqslant \frac{k^{2}}{4 k-3} \geqslant 1$. 因此, $A+3 B \geqslant 4$.

  \item 令 $a=\frac{x}{y}, b=\frac{y}{z}, c=\frac{z}{x}$, 则原不等式等价于 $\frac{y}{2 x+y}+\frac{z}{2 y+z}+$\\
$\frac{x}{2 z+x} \geqslant 1$. 由柯西不等式, 得 $[x(x+2 z)+y(y+2 x)+z(z+$ $2 y)]\left(\frac{x}{x+2 z}+\frac{y}{y+2 x}+\frac{z}{z+2 y}\right) \geqslant(x+y+z)^{2}$, 即 $(x+y+$ $z)^{2}\left(\frac{x}{x+2 z}+\frac{y}{y+2 x}+\frac{z}{z+2 y}\right) \geqslant(x+y+z)^{2}$.

  \item 由 $\left(\sum_{i=1}^{n} a_{i}^{3}\right)^{2} \leqslant \sum_{i=1}^{n} a_{i}^{2} \sum_{i=1}^{n} a_{i}^{4} \leqslant \sum_{i=1}^{n} a_{i}^{2}\left(\sum_{i=1}^{n} a_{i}^{2}\right)^{2}=\left(\sum_{i=1}^{n} a_{i}^{2}\right)^{3}$, 则 $\left(\sum_{i=1}^{n} a_{i}^{3}\right)^{\frac{1}{3}} \leqslant\left(\sum_{i=1}^{n} a_{i}^{2}\right)^{\frac{1}{2}}$

  \item 由柯西不等式, 得 $2(a+b+c)\left(\frac{1}{a+b}+\frac{1}{b+c}+\frac{1}{c+a}\right)=[(a+b)+$ $(b+c)+(c+a)]\left(\frac{1}{a+b}+\frac{1}{b+c}+\frac{1}{c+a}\right) \geqslant 9$, 故命题成立.

  \item 由柯西不等式, 得 $\frac{k^{2}(k+1)^{2}}{4}=\left(\sum_{i=1}^{k} \frac{i}{\sqrt{a_{i}}} \cdot \sqrt{a_{i}}\right)^{2} \leqslant \sum_{i=1}^{k} \frac{i^{2}}{a_{i}} \sum_{i=1}^{k} a_{i}$.所以 $\frac{k}{\sum_{i=1}^{k} a_{i}} \leqslant \frac{4}{k(k+1)^{2}} \sum_{i=1}^{k} \frac{i^{2}}{a_{i}}$. 求和, 得 $\sum_{k=1}^{n} \frac{k}{\sum_{i=1}^{k} a_{i}} \leqslant \sum_{k=1}^{n}\left[\frac{4}{k(k+1)^{2}} \sum_{i=1}^{k} \frac{i^{2}}{a_{i}}\right]<$ $2 \sum_{i=1}^{n}\left[\frac{i^{2}}{a_{i}} \sum_{k=i}^{n} \frac{2 k+1}{k^{2}(k+1)^{2}}\right]=2 \sum_{i=1}^{n}\left[\frac{i^{2}}{a_{i}} \sum_{k=i}^{n}\left(\frac{1}{k^{2}}-\frac{1}{(k+1)^{2}}\right)\right]=2 \sum_{i=1}^{n} \frac{i^{2}}{a_{i}}\left(\frac{1}{i^{2}}-\frac{1}{(n+1)^{2}}\right)<$ $2 \cdot \sum_{i=1}^{n} \frac{i^{2}}{a_{i}} \cdot \frac{1}{i^{2}}=2 \sum_{i=1}^{n} \frac{1}{a_{i}}$.

  \item $\left(\sum_{i=1}^{n} a_{i} b_{i} c_{i} d_{i}\right)^{4} \leqslant\left[\sum_{i=1}^{n}\left(a_{i} b_{i}\right)^{2}\right]^{2}\left[\sum_{i=1}^{n}\left(c_{i} d_{i}\right)^{2}\right]^{2} \leqslant \sum_{i=1}^{n} a_{i}^{4} \sum_{i=1}^{n} b_{i}^{4} \sum_{i=1}^{n} c_{i}^{4} \sum_{i=1}^{n} d_{i}^{4}$.

  \item 显然 $x_{i} \in(0,1)$, 由柯西不等式和平均值不等式, 有 $\sum_{k=1}^{n} \frac{x_{k}}{x_{k+1}-x_{k+1}^{2}}$

\end{enumerate}
$$
=\sum_{k=1}^{n} \frac{\frac{x_{k}}{x_{k+1}}}{1-x_{k+1}} \geqslant \frac{\left(\sum_{k=1}^{n} \sqrt{\frac{x_{k}}{x_{k+1}}}\right)^{2}}{n-\sum_{k=1}^{n} x_{k+1}} \geqslant \frac{\left[n\left(\prod_{k=1}^{n} \sqrt{\frac{x_{k}}{x_{k+1}}}\right)^{\frac{1}{n}}\right]^{2}}{n-\frac{1}{n}\left(\sum_{k=1}^{n} x_{k}\right)^{2}}=\frac{n^{2}}{n-\frac{1}{n}}=\frac{n^{3}}{n^{2}-1}
$$

\begin{enumerate}
  \setcounter{enumi}{15}
  \item 令 $a_{n+1}=a_{1}, a_{n+2}=a_{2}$, 则 $\sum_{i=1}^{n} a_{i}\left(a_{i+1}+a_{i+2}\right) \sum_{i=1}^{n} \frac{a_{i}}{a_{i+1}+a_{i+2}} \geqslant$
\end{enumerate}

$\left(\sum_{i=1}^{n} a_{i}\right)^{2}$. 于是 $\sum_{i=1}^{n} \frac{a_{i}}{a_{i+1}+a_{i+2}} \geqslant \frac{\left(\sum_{i=1}^{n} a_{i}\right)^{2}}{\sum_{i=1}^{n} a_{i}\left(a_{i+1}+a_{i+2}\right)}$. 只需证 $2 \sum_{i=1}^{n} a_{i}^{2} \geqslant$\\
$\sum_{i=1}^{n} a_{i}\left(a_{i+1}+a_{i+2}\right)$, 即 $\frac{1}{2} \sum_{i=1}^{n}\left[\left(a_{i}^{2}+a_{i+1}^{2}\right)+\left(a_{i}^{2}+a_{i+2}^{2}\right)\right] \geqslant \sum_{i=1}^{n} a_{i}\left(a_{i+1}+a_{i+2}\right)$.由 $a_{i}^{2}+a_{i+1}^{2} \geqslant 2 a_{i} a_{i+1}$, 便得到命题成立.\\
17. $\frac{1}{4}(a b c+b c d+c d a+d a b)=\frac{1}{4}[b c(a+d)+d a(b+c)] \leqslant$ $\frac{1}{4}\left[\left(\frac{b+c}{2}\right)^{2}(a+d)+\left(\frac{a+d}{2}\right)^{2}(b+c)\right]=\frac{1}{16}(b+c)(a+d)(a+b+c+d)$ $\leqslant \frac{1}{64}(a+b+c+d)^{3}=\left(\frac{a+b+c+d}{4}\right)^{3} \leqslant\left(\sqrt{\frac{a^{3}+b^{3}+c^{3}+d^{3}}{4}}\right)^{3}$.

\begin{enumerate}
  \setcounter{enumi}{17}
  \item 当 $n=2$ 时, 则 $\sqrt{2}<2$. 命题成立. 当 $n=3$ 时, 则 $1<\sqrt{3}$. 所以可设 $n \geqslant 4$. 由柯西不等式, 得 $1 \cdot \sqrt{\mathrm{C}_{n}^{1}}+2 \cdot \sqrt{\mathrm{C}_{n}^{2}}+\cdots+n \cdot \sqrt{\mathrm{C}_{n}^{n}} \leqslant\left(1^{2}+\right.$ $\left.2^{2}+\cdots+n^{2}\right)^{\frac{1}{2}}\left(\mathrm{C}_{n}^{1}+\mathrm{C}_{n}^{2}+\cdots+\mathrm{C}_{n}^{n}\right)^{\frac{1}{2}}=\left[\frac{n(n+1)(2 n+1)}{6}\right]^{\frac{1}{2}} \cdot\left(2^{n}-1\right)^{\frac{1}{2}}$. 即证明: $\frac{n(n+1)(2 n+1)}{6} \cdot\left(2^{n}-1\right)<2^{n-1} \cdot n^{3}$ 便可. 等价于 $\left(2 n^{2}+3 n+1\right)$ ・ $\left(2^{n}-1\right)<3 n^{2} \cdot 2^{n}$. 因为 $n \geqslant 4$, 故 $n^{2}>3 n, n^{2} \geqslant 3 n+1$, 进而 $3 n^{2} \geqslant 2 n^{2}+$ $3 n+1$. 所以 $\left(2 n^{2}+3 n+1\right)\left(2^{n}-1\right)<3 n^{2} \cdot 2^{n}$. 从而, 命题成立.

  \item 令 $x_{i}=\frac{a_{i}}{\sum_{k=1}^{n} a_{k}}$, 则 $\sum_{i=1}^{n} x_{i}=1$, 以及 $\sum_{i=1}^{n} \frac{a_{i} x_{i}}{x_{i}+y_{i}}=\sum_{i=1}^{n} a_{i} \sum_{i=1}^{n} \frac{x_{i}^{2}}{x_{i}+y_{i}}$. 对任意 $y_{i} \in \mathbf{R}_{+}, 1 \leqslant i \leqslant n, \sum_{i=1}^{n} y_{i}=1$, 由柯西不等式得 $2 \sum_{i=1}^{n} \frac{x_{i}^{2}}{x_{i}+y_{i}}=\sum_{i=1}^{n}\left(x_{i}\right.$ $\left.+y_{i}\right) \sum_{i=1}^{n} \frac{x_{i}^{2}}{x_{i}+y_{i}} \geqslant\left(\sum_{i=1}^{n} x_{i}\right)^{2}=1$. 于是 $\sum_{i=1}^{n} \frac{a_{i} x_{i}}{x_{i}+y_{i}}=\sum_{i=1}^{n} a_{i} \sum_{i=1}^{n} \frac{x_{i}^{2}}{x_{i}+y_{i}} \geqslant$ $\frac{1}{2} \sum_{i=1}^{n} a_{i}$. 故命题成立.

  \item 当 $n=2$ 时, 由柯西不等式得 $\frac{\sqrt{1-x_{1}}}{x_{1}}+\frac{\sqrt{1-x_{2}}}{x_{2}}=$ $\frac{x_{2} \sqrt{1-x_{1}}+x_{1} \sqrt{1-x_{2}}}{x_{1} x_{2}} \leqslant \frac{\sqrt{1-x_{1}+x_{1}^{2}} \sqrt{1-x_{2}+x_{2}^{2}}}{x_{1} x_{2}}<\frac{1}{x_{1} x_{2}}$, 当 $n \geqslant 3$时, 由归纳假设及柯西不等式, 得 $\frac{\sqrt{1-x_{1}}}{x_{1}}+\cdots+\frac{\sqrt{1-x_{n}}}{x_{n}}<\frac{\sqrt{n-2}}{x_{1} \cdots x_{n-1}}+$ $\frac{\sqrt{1-x_{n}}}{x_{n}}=\frac{\sqrt{n-2} x_{n}+x_{1} \cdots x_{n-1} \sqrt{1-x_{n}}}{x_{1} x_{2} \cdots x_{n}} \leqslant$ $\frac{\sqrt{n-2+\left(x_{1} \cdots x_{n-1}\right)^{2}} \sqrt{1-x_{n}+x_{n}^{2}}}{x_{1} x_{2} \cdots x_{n}}<\frac{\sqrt{n-1}}{x_{1} \cdots x_{n}}$.

  \item 由 $1+2 a b c \geqslant a^{2}+b^{2}+c^{2}$, 有 $(a-b c)^{2} \leqslant\left(1-b^{2}\right)\left(1-c^{2}\right)$, 由柯西不等式得 $\left(a^{n-1}+a^{n-2} b c+\cdots+(b c)^{n-1}\right)^{2} \leqslant\left(|a|^{n-1}+|a|^{n-2}|b c|+\cdots+\right.$ $\left.|b c|^{n-1}\right)^{2} \leqslant\left(1+|b c|+\cdots+|b c|^{n-1}\right)^{2} \leqslant\left(1+b^{2}+\cdots+b^{2(n-1)}\right)\left(1+c^{2}+\cdots\right.$ $\left.+c^{2(n-1)}\right)$, 于是 $(a-b c)^{2}\left(a^{n-1}+\cdots+(b c)^{n-1}\right)^{2} \leqslant\left(1-b^{2}\right)\left(1+b^{2}+\cdots+\right.$ $\left.b^{2(n-1)}\right) \cdot\left(1-c^{2}\right) \cdot\left(1+c^{2}+\cdots+c^{2(n-1)}\right)$. 即 $\left(a^{n}-b^{n} c^{n}\right)^{2} \leqslant\left(1-b^{2 n}\right)\left(1-c^{2 n}\right)$.从而, $a^{2 n}+b^{2 n}+c^{2 n} \leqslant 1+2 a^{n} b^{n} c^{n}$.

  \item 由柯西不等式 $\sum_{i=1}^{n} \frac{1}{a_{i}^{2}}+\frac{1}{\left(\sum_{i=1}^{n} a_{i}\right)^{2}}=\frac{n^{2}}{n^{3}+1}\left(\sum_{i=1}^{n} 1+\right.$ $\left.\frac{1}{n^{2}}\right)\left(\sum_{i=1}^{n} \frac{1}{a_{i}^{2}}+\frac{1}{\left(\sum_{i=1}^{n} a_{i}\right)^{2}}\right) \geqslant \frac{n^{2}}{n^{3}+1}\left(\sum_{i=1}^{n} \frac{1}{a_{i}}+\frac{1}{n \sum_{i=1}^{n} a_{i}}\right)^{2}$. 令 $A=\sum_{i=1}^{n} \frac{1}{a_{i}}, B=$ $\sum_{i=1}^{n} a_{i}$. 则由柯西不等式, $A B \geqslant n^{2}$. 为证原不等式, 只要证明 $\frac{n^{2}}{n^{3}+1}\left(A+\frac{1}{n B}\right)^{2} \geqslant \frac{n^{3}+1}{\left(n^{2}+1\right)^{2}}\left(A+\frac{1}{B}\right)^{2} \Leftrightarrow n\left(n^{2}+1\right)\left(A+\frac{1}{n B}\right) \geqslant\left(n^{2}+\right.$ 1) $\left(A+\frac{1}{B}\right) \Leftrightarrow A B \geqslant n^{2}$. 从而命题成立.

  \item 由柯西不等式得 $\left(\sum_{i=1}^{n} a_{i}\right)^{2}=\left[n a_{1}+(n-1)\left(a_{2}-a_{1}\right)+\cdots+\left(a_{n}-\right.\right.$ $\left.\left.a_{n-1}\right)\right]^{2} \leqslant\left[n^{2}+(n-1)^{2}+\cdots+2^{2}+1^{2}\right]\left[a_{1}^{2}+\left(a_{2}-a_{1}\right)^{2}+\cdots+\left(a_{n}-a_{n-1}\right)^{2}\right]$ $=\frac{n(n+1)(2 n+1)}{6}\left(2 a_{1}^{2}+\cdots+2 a_{n-1}^{2}+a_{n}^{2}-2 a_{1} a_{2}-\cdots-2 a_{n-1} a_{n}\right) \leqslant$ $\frac{n(n+1)(2 n+1)}{3}\left(\sum_{i=1}^{n} a_{i}^{2}-\sum_{i=1}^{n-1} a_{i} a_{i+1}\right)$, 从 而 $\sum_{i=1}^{n} a_{i}^{2} \geqslant \sum_{i=1}^{n-1} a_{i} a_{i+1}+$ $\frac{3}{n(n+1)(2 n+1)}\left(\sum_{i=1}^{n} a_{i}\right)^{2}$.

  \item 由于 $\sum_{i=1}^{n} \frac{x_{i}^{\frac{3}{2}}}{y_{i}^{\frac{1}{2}}}=\sum_{i=1}^{n} \frac{x_{i}^{2}}{x_{i}^{\frac{1}{2}} y_{i}^{\frac{1}{2}}} \geqslant \frac{\left(\sum_{i=1}^{n} x_{i}\right)^{2}}{\sum_{i=1}^{n} x_{i}^{\frac{1}{2}} y_{i}^{\frac{1}{2}}}$ 以及 $\sum_{i=1}^{n} x_{i}^{\frac{1}{2}} y_{i}^{\frac{1}{2}} \leqslant$ $\left(\sum_{i=1}^{n} x_{i}\right)^{\frac{1}{2}}\left(\sum_{i=1}^{n} y_{i}\right)^{\frac{1}{2}}$, 于是 $\sum_{i=1}^{n} \frac{x_{i}^{\frac{3}{2}}}{y_{i}^{\frac{1}{2}}} \geqslant \frac{\left(\sum_{i=1}^{n} x_{i}\right)^{2}}{\left(\sum_{i=1}^{n} x_{i}\right)^{\frac{1}{2}}\left(\sum_{i=1}^{n} y_{i}\right)^{\frac{1}{2}}}=\frac{\left(\sum_{i=1}^{n} x_{i}\right)^{\frac{3}{2}}}{\left(\sum_{i=1}^{n} y_{i}\right)^{\frac{1}{2}}}$.

\end{enumerate}

\section*{习 题 4}
\begin{enumerate}
  \item 因为 $\frac{a^{5}}{a^{3}+1}=\frac{a^{5}}{a^{3}+a b c}=\frac{a^{4}}{a^{2}+b c}$. 所以 $\sum_{\mathrm{cyc}} \frac{a^{5}}{a^{3}+1}=\sum_{\mathrm{cyc}} \frac{a^{4}}{a^{2}+b c} \geqslant$ $\frac{\left(\sum_{\text {cyc }} a^{2}\right)^{2}}{\sum_{\text {cyc }}\left(a^{2}+b c\right)}($ 柯西不等式 $) \geqslant \frac{1}{2} \frac{\left(\sum_{\text {cyc }} a^{2}\right)^{2}}{\sum_{\text {cyc }} a^{2}}=\frac{1}{2} \sum_{\text {cyc }} a^{2}$ (平均值不等式 $) \geqslant \frac{3}{2}$ $\sqrt[3]{a^{2} b^{2} c^{2}}=\frac{3}{2}$

  \item 由柯西不等式, 可得 $\sum_{\text {cyc }} \frac{a^{3}}{5 a+b}+\sum_{\text {cyc }} \frac{3 a^{3}}{5 c+a}=\sum_{\text {cyc }} \frac{a^{4}}{5 a^{2}+a b}+$ $3 \sum_{\text {cyc }} \frac{a^{4}}{5 a c+a^{2}} \geqslant \frac{\left(a^{2}+b^{2}+c^{2}\right)^{2}}{5\left(a^{2}+b^{2}+c^{2}\right)+a b+b c+c a}+3$. $\frac{\left(a^{2}+b^{2}+c^{2}\right)^{2}}{a^{2}+b^{2}+c^{2}+5(a b+b c+c a)} \geqslant \frac{a^{2}+b^{2}+c^{2}}{6}+\frac{a^{2}+b^{2}+c^{2}}{2}=\frac{2}{3}\left(a^{2}+b^{2}+\right.$ $\left.c^{2}\right)$.

  \item $\frac{a}{10-3 a}+\frac{b}{10-3 b}+\frac{c}{10-3 c}+\frac{d}{10-3 d}=\left(-\frac{1}{3}+\frac{10}{3} \cdot \frac{1}{10-3 a}\right)+$ $\left(-\frac{1}{3}+\frac{10}{3} \cdot \frac{1}{10-3 b}\right)+\left(-\frac{1}{3}+\frac{10}{3} \cdot \frac{1}{10-3 c}\right)+\left(-\frac{1}{3}+\frac{10}{3} \cdot \frac{1}{10-3 d}\right)=$ $-\frac{4}{3}+\frac{10}{3}\left(\frac{1}{10-3 a}+\frac{1}{10-3 b}+\frac{1}{10-3 c}+\frac{1}{10-3 d}\right)$, 由柯西不等式变式知 $\frac{1}{10-3 a}+\frac{1}{10-3 b}+\frac{1}{10-3 c}+\frac{1}{10-3 d} \geqslant$

\end{enumerate}

$\frac{(1+1+1+1)^{2}}{(10-3 a)+(10-3 b)+(10-3 c)+(10-3 d)}=\frac{16}{31}$. 所以, $\frac{a}{10-3 a}+$ $\frac{b}{10-3 b}+\frac{c}{10-3 c}+\frac{d}{10-3 d} \geqslant-\frac{4}{3}+\frac{10}{3} \cdot \frac{16}{31}=\frac{12}{31}$.

\begin{enumerate}
  \setcounter{enumi}{3}
  \item 由柯西不等式得 $\sum_{\text {cyc }} \frac{a+b}{b+c-a}=\sum_{\text {cyc }} \frac{(a+b)^{2}}{(a+b)(b+c-a)} \geqslant$ $\frac{\left(\sum_{\mathrm{cyc}}(a+b)\right)^{2}}{\sum_{\mathrm{cyc}}[(a+b)(b+c-a)]}=\frac{4\left(\sum_{\text {cyc }} a\right)^{2}}{\sum_{\mathrm{cyc}}(a+b) c+\sum_{\text {cyc }}\left(b^{2}-a^{2}\right)}=\frac{4\left(\sum_{\text {cyc }} a\right)^{2}}{2 \sum_{\text {cyc }} a b}$. 又因为 $\left(\sum_{\text {cyc }} a\right)^{2} \geqslant 3 \sum_{\text {cyc }} a b$, 所以 $\sum_{\text {cyc }} \frac{a+b}{b+c-a} \geqslant \frac{4\left(\sum_{\text {cyc }} a\right)^{2}}{2 / 3\left(\sum_{\text {cyc }} a\right)^{2}}=6$.

  \item 设 $y_{k}=\frac{1}{x_{k}}$, 从而, $\frac{1}{y_{k}}=\frac{1}{1+\frac{a_{k}}{y_{k-1}}} \Leftrightarrow y_{k}=1+\frac{a_{k}}{y_{k-1}}$. 由 $y_{k-1} \geqslant 1, a_{k} \geqslant$

\end{enumerate}

1 可得 $\left(\frac{1}{y_{k-1}}-1\right)\left(a_{k}-1\right) \leqslant 0 \Leftrightarrow 1+\frac{a_{k}}{y_{k-1}} \leqslant a_{k}+\frac{1}{y_{k-1}}$. 所以, $y_{k}=1+\frac{a_{k}}{y_{k-1}} \leqslant$ $a_{k}+\frac{1}{y_{k-1}}$. 故 $\sum_{k=1}^{n} y_{k} \leqslant \sum_{k=1}^{n} a_{k}+\sum_{k=1}^{n-1} \frac{1}{y_{k-1}}=\sum_{k=1}^{n} a_{k}+\frac{1}{y_{0}}+\sum_{k=1}^{n} \frac{1}{y_{k}}=A+\sum_{k=1}^{n-1} \frac{1}{y_{k}}$ $<A+\sum_{k=1}^{n} \frac{1}{y_{k}}$. 令 $t=\sum_{k=1}^{n} \frac{1}{y_{k}}$, 由柯西不等式有 $\sum_{k=1}^{n} y_{k} \geqslant \frac{n^{2}}{t}$. 因此, 对 $t>0$, 有 $\frac{n^{2}}{t}<A+t \Leftrightarrow t^{2}+A t-n^{2}>0 \Leftrightarrow t>\frac{-A+\sqrt{A^{2}+4 n^{2}}}{2}=\frac{2 n^{2}}{A+\sqrt{A^{2}+4 n^{2}}} \geqslant$ $\frac{2 n^{2}}{A+A+\frac{2 n^{2}}{A}}=\frac{n^{2} A}{n^{2}+A^{2}}$

\begin{enumerate}
  \setcounter{enumi}{5}
  \item 由均值不等式有 $\frac{x^{2}}{14}+\frac{x}{y^{2}+z+1}+\frac{2}{49}\left(y^{2}+z+1\right) \geqslant 3 \sqrt[3]{\frac{x^{3}}{7^{3}}}=\frac{3}{7} x$.则 $\frac{1}{14} \sum_{\text {cyc }} x^{2}+\sum_{\text {cyc }} \frac{x}{y^{2}+z+1}+\frac{2}{49} \sum_{\text {cyc }} x^{2}+\frac{2}{49} \sum_{\text {cyc }} x+\frac{6}{49} \geqslant \frac{3}{7} \sum_{\text {cyc }} x$. 故 $\frac{11}{98} \sum_{\text {cyc }} x^{2}+\sum_{\text {cyc }} \frac{x}{y^{2}+z+1}+\frac{6}{49} \geqslant\left(\frac{3}{7}-\frac{2}{49}\right) \sum_{\text {cyc }} x=\frac{19}{49} \sum_{\text {cyc }} x$. 又 $\sum_{\text {cyc }} x^{2} \geqslant$ $\frac{1}{3}\left(\sum_{\text {cyc }} x\right)^{2} \geqslant 12$, 故 $\sum_{\text {cyc }} x^{2}+\sum_{\text {cyc }} \frac{x}{y^{2}+z+1}=\frac{87}{98} \sum_{\text {cyc }} x^{2}+\frac{11}{98} \sum_{\text {cyc }} x^{2}-\frac{6}{49} \geqslant$ $\frac{87}{98} \sum_{\text {cyc }} x^{2}+\frac{19}{49} \sum_{\text {cyc }} x-\frac{6}{49} \geqslant \frac{87 \times 6+19 \times 6-6}{49}=\frac{90}{7}$. 从而, $M_{\min }=\frac{90}{7}$. 此时, $(x, y, z)=(2,2,2)$.

  \item 令 $x \geqslant y 、 z$. 则 $\frac{4 z^{2}+x^{2}}{z^{2}+4 x^{2}} \leqslant 1, \frac{4 x^{2}+y^{2}}{x^{2}+4 y^{2}}<4, \frac{4 y^{2}+z^{2}}{y^{2}+4 z^{2}}<4$. 三式相加, 知右边的不等式成立. 由平均值不等式知 $4 x y^{2} \leqslant y^{3}+4 x^{2} y$. 则 $y^{3}+4 x^{2} y$ $+4 x^{3}+x y^{2}>4 x y^{2}+x^{3} \Rightarrow \frac{4 x^{2}+y^{2}}{x^{2}+4 y^{2}}>\frac{x}{x+y}$. 故 $\sum_{\mathrm{cyc}} \frac{x}{x+y}<\sum_{\text {cyc }} \frac{4 x^{2}+y^{2}}{x^{2}+4 y^{2}}$.由柯西一施瓦兹不等式知 $\sum_{\text {cyc }} \frac{x}{\sqrt{2\left(x^{2}+y^{2}\right)}} \leqslant \sum_{\text {cyc }} \frac{x}{x+y}<\sum_{\text {cyc }} \frac{4 x^{2}+y^{2}}{x^{2}+4 y^{2}}$. 故命题得证.

  \item 首先, 易观察出当 $u=v=w=\frac{\sqrt{3}}{3}$ 时, $u \sqrt{v w}+v \sqrt{w u}+w \sqrt{u v}=$ 1 及 $u+v+w=\sqrt{3}$. 因此, $\lambda$ 的最大值不超过 $\sqrt{3}$. 下面证明: 对于所有 $u, v$, $w>0$, 且满足 $u \sqrt{v w}+v \sqrt{w u}+w \sqrt{u v} \geqslant 1$, 均有 $u+v+w \geqslant \sqrt{3}$. 由平均值不等式及柯西不等式有 $\frac{(u+v+w)^{4}}{9}=\left(\frac{u+v+w}{3}\right)^{3} \cdot 3(u+v+w) \geqslant$\\
$3 u v w(u+v+w)=(u v w+v w u+w u v)(u+v+w) \geqslant(u \sqrt{v w}+v \sqrt{w u}+$ $w \sqrt{u v})^{2} \geqslant 1$. 因此, $u+v+w \geqslant \sqrt{3}$. 当且仅当 $u=v=w=\frac{\sqrt{3}}{3}$ 时, 上式等号成立. 综上,所求 $\lambda$ 的最大值为 $\sqrt{3}$.

  \item 由于 $\sum_{i=1}^{n}\left(x_{n+1}-x_{i}\right)=n x_{n+1}-\sum_{i=1}^{n} x_{i}=(n-1) x_{n+1}$, 于是, 只需证明, $x_{n+1} \sqrt{n-1} \geqslant \sum_{i=1}^{n} \sqrt{x_{i}\left(x_{n+1}-x_{i}\right)}$, 即证 $\sum_{i=1}^{n} \sqrt{\frac{x_{i}}{x_{n+1}}\left(1-\frac{x_{i}}{x_{n+1}}\right)} \leqslant \sqrt{n-1}$.由柯西不等式, 得 $\left[\sum_{i=1}^{n} \sqrt{\frac{x_{i}}{x_{n+1}} \cdot\left(1-\frac{x_{i}}{x_{n+1}}\right)}\right]^{2} \leqslant\left(\sum_{i=1}^{n} \frac{x_{i}}{x_{n+1}}\right)\left[\sum_{i=1}^{n}\left(1-\frac{x_{i}}{x_{n+1}}\right)\right]=$ $\left(\frac{1}{x_{n+1}} \sum_{i=1}^{n} x_{i}\right)\left(n-\frac{1}{x_{n+1}} \sum_{i=1}^{n} x_{i}\right)=n-1$.

  \item 设 $u=x \sin \alpha+y \sin \beta, v=z \sin \gamma+w \sin \theta$, 则 $u^{2}=$ $(x \sin \alpha+y \sin \beta)^{2} \leqslant(x \sin \alpha+y \sin \beta)^{2}+(x \cos \alpha-y \cos \beta)^{2}=x^{2}+y^{2}-$ $2 x y \cos (\alpha+\beta)$. 所以 $\cos (\alpha+\beta) \leqslant \frac{x^{2}+y^{2}-u^{2}}{2 x y}$. 同理 $\cos (\gamma+\theta) \leqslant$ $\frac{z^{2}+w^{2}-v^{2}}{2 z w}$. 由假设 $\cos (\alpha+\beta)+\cos (\gamma+\theta)=0$, 则 $\frac{u^{2}}{x y}+\frac{v^{2}}{z w} \leqslant \frac{x^{2}+y^{2}}{x y}+$ $\frac{z^{2}+w^{2}}{z w}$. 于是 $(u+v)^{2}=\left(u \cdot \frac{\sqrt{x y}}{\sqrt{x y}}+v \cdot \frac{\sqrt{z w}}{\sqrt{z w}}\right)^{2} \leqslant\left(\frac{u^{2}}{x y}+\frac{v^{2}}{z w}\right)(x y+$ $z w) \leqslant(x y+z w)\left(\frac{x^{2}+y^{2}}{x y}+\frac{z^{2}+w^{2}}{z w}\right)$. 等号成立 $\Leftrightarrow x \cos \alpha=y \cos \beta$, $z \cos \gamma=w \cos \theta, \frac{u}{x y}=\frac{v}{z w} \Leftrightarrow x \cos \alpha=y \cos \beta=z \cos \gamma=w \cos \theta$.

  \item 令 $A=a_{1} a_{2} \cdots a_{n}$, 则 $M=\frac{1}{A} \prod_{i=1}^{n}\left(a_{i} b_{i}+1\right)$. 由 $\left(a_{i} b_{i}+1\right)^{2} \leqslant\left(a_{i}^{2}+1\right)$ $\left(b_{i}^{2}+1\right)$ 知等号成立 $\Leftrightarrow a_{i}=b_{i}$. 由此得到 $M \leqslant \frac{1}{A} \prod_{i=1}^{n}\left(1+a_{i}^{2}\right)$ , 且等号成立 $\Leftrightarrow$ $a_{i}=b_{i}(i=1,2, \cdots, n)$. 故 $b_{1}=a_{1}, b_{2}=a_{2}, \cdots, b_{n}=a_{n}$ 时,$M$ 取值最大.

  \item 设 $a+\mathrm{i} b=\sqrt{\sum_{i=1}^{n} z_{i}^{2}}, a, b \in \mathbf{R}$, 则 $a^{2}-b^{2}=\sum_{k=1}^{n} x_{k}^{2}-\sum_{k=1}^{n} y_{k}^{2}, a b=$ $\sum_{k=1}^{n} x_{k} y_{k}$. 若 $r=|a|>\sum_{k=1}^{n}\left|x_{k}\right|$, 由于 $\sum_{k=1}^{n}\left|x_{k}\right| \geqslant\left(\sum_{k=1}^{n} x_{k}^{2}\right)^{\frac{1}{2}}$, 则 $|a|>$ $\left(\sum_{k=1}^{n} x_{k}^{2}\right)^{\frac{1}{2}}$. 由柯西不等式, 得 $|a| \cdot|b| \leqslant\left(\sum_{k=1}^{n} x_{k}^{2}\right)^{\frac{1}{2}}\left(\sum_{k=1}^{n} y_{k}^{2}\right)^{\frac{1}{2}}$, 从而 $|b| \leqslant$\\
$\left(\sum_{k=1}^{n} y_{k}^{2}\right)^{\frac{1}{2}}$, 于是 $a^{2}=\sum_{k=1}^{n} x_{k}^{2}+b^{2}-\sum_{k=1}^{n} y_{k}^{2} \leqslant \sum_{k=1}^{n} x_{k}^{2}$ 与 $|a|>\left(\sum_{k=1}^{n} x_{k}^{2}\right)^{\frac{1}{2}}$ 矛盾.

  \item $n \sum_{i=1}^{n}\left(a_{i}-\frac{1}{a_{i}}\right)^{2}=n \sum_{i=1}^{n} a_{i}^{2}+n \sum_{i=1}^{n} \frac{1}{a_{i}^{2}}-2 n^{2} \geqslant\left(\sum_{i=1}^{n} a_{i}\right)^{2}+\left(\sum_{i=1}^{n} \frac{1}{a_{i}}\right)^{2}-$ $2 n^{2} \geqslant n^{2}\left(A_{n}^{2}+\frac{1}{A_{n}^{2}}-2\right)=n^{2}\left(A_{n}-\frac{1}{A_{n}}\right)^{2}$.

  \item 由柯西不等式, 得 $w \geqslant \frac{1}{4}\left[(r-1)+\left(\frac{s}{r}-1\right)+\left(\frac{t}{s}-1\right)+\left(\frac{4}{t}-1\right)\right]^{2}$ $=\frac{1}{4}\left(r+\frac{s}{r}+\frac{t}{s}+\frac{4}{t}-4\right)^{2}$. 又 $r+\frac{s}{r}+\frac{t}{s}+\frac{4}{t} \geqslant 4 \sqrt[4]{r \cdot \frac{s}{r} \cdot \frac{t}{s} \cdot \frac{4}{t}}=$ $4 \sqrt{2}$, 所以 $w \geqslant 4(\sqrt{2}-1)^{2}$. 当且仅当 $r=\sqrt{2}, s=2, t=2 \sqrt{2}$ 时, 取等号. 故 $w_{\text {min }}=4(\sqrt{2}-1)^{2}$.

  \item 令 $a_{i} b_{i}-c_{i}^{2}=d_{i}^{2}>0$, 则由柯西不等式, 得 $\left(\sum a_{i}\right)\left(\sum b_{i}\right) \geqslant$ $\left(\sum \sqrt{a_{i} b_{i}}\right)^{2}=\sum_{i=1}^{n} \sum_{j=1}^{n} \sqrt{a_{i} b_{i}} \sqrt{a_{j} b_{j}}=\sum_{i=1}^{n} \sum_{j=1}^{n} \sqrt{c_{i}^{2}+d_{i}^{2}} \sqrt{c_{j}^{2}+d_{j}^{2}} \geqslant$ $\sum_{i=1}^{n} \sum_{j=1}^{n}\left(c_{i} c_{j}+d_{i} d_{j}\right)=\left(\sum_{i=1}^{n} c_{i}\right)^{2}+\left(\sum_{i=1}^{n} d_{i}\right)^{2}$, 又因为 $\left(\sum_{i=1}^{n} d_{i}\right)^{2}\left(\sum_{i=1}^{n} d_{i}^{-2}\right) \geqslant$ $n^{3}$, 故左边 $\leqslant \frac{n^{3}}{\left(\sum d_{i}\right)^{2}} \leqslant \sum d_{i}^{-2}$. 等号成立当且仅当 $a_{1}=a_{2}=\cdots=a_{n}, b_{1}$ $=b_{2}=\cdots=b_{n}, c_{1}=c_{2}=\cdots=c_{n}$.

  \item 不妨设 $a_{1} \leqslant a_{2} \leqslant \cdots \leqslant a_{n}, b_{1} \geqslant b_{2} \geqslant \cdots \geqslant b_{n}$. 令 $A_{i}=\prod_{\substack{j=1 \\ j \neq i}}^{n} a_{j}$, 则 $A_{1} \geqslant A_{2} \geqslant \cdots \geqslant A_{n} \geqslant 0$. 由排序不等式, 得 $\sum_{i=1}^{n} b_{i} A_{i} \leqslant b_{1} A_{1}+\left(1-b_{1}\right) A_{2}=$ $p A_{1}+p A_{2}+\left(-p+1-b_{1}\right) A_{2}+\left(b_{1}-p\right) A_{1}=p\left(A_{1}+A_{2}\right)-\left(p-b_{1}\right)\left(A_{1}-\right.$ $\left.A_{2}\right)+(1-2 p) A_{2} \leqslant p\left(A_{1}+A_{2}\right)$ (因为 $\frac{1}{2} \leqslant p \leqslant 1$ ). 由平均值不等式, 得 $A_{1}+A_{2}=a_{3} a_{4} \cdots a_{n}\left(a_{2}+a_{1}\right) \leqslant\left(\frac{1}{n-1} \sum_{i=1}^{n} a_{i}\right)^{n-1}=\frac{1}{(n-1)^{n-1}}$, 所以 $\sum_{i=1}^{n} b_{i} A_{i} \leqslant$ $\frac{p}{(n-1)^{n-1}}$

  \item 由柯西不等式, 得 $1=\sum_{i=1}^{n} b_{i} \leqslant\left(\sum_{i=1}^{n} \frac{b_{i}}{a_{i}}\right)^{\frac{1}{2}}\left(\sum_{i=1}^{n} a_{i} b_{i}\right)^{\frac{1}{2}}$, 从而 $\frac{1}{\sum_{i=1}^{n} a_{i} b_{i}}$ $\leqslant \sum_{i=1}^{n} \frac{b_{i}}{a_{i}}$. 令 $M=\prod_{i=1}^{n} a_{i}, A_{i}=\frac{M}{a_{i}}, i=1,2, \cdots, n$, 则 $\frac{M}{\sum_{i=1}^{n} a_{i} b_{i}} \leqslant \sum_{i=1}^{n} b_{i} A_{i}$.\\
不妨设 $b_{1} \geqslant b_{2} \geqslant \cdots \geqslant b_{n}, A_{1} \geqslant A_{2} \geqslant \cdots \geqslant A_{n}$, 由排序不等式, 得 $\sum_{i=1}^{n} b_{i} A_{i} \leqslant$ $b_{1} A_{1}+\left(1-b_{1}\right) A_{2}$. 由于 $0 \leqslant b_{1} \leqslant \frac{1}{2}, A_{1} \geqslant A_{2}$, 所以 $\sum_{i=1}^{n} b_{i} A_{i} \leqslant$ $\frac{1}{2}\left(A_{1}+A_{2}\right)=\frac{1}{2}\left(a_{1}+a_{2}\right) a_{3} \cdots a_{n}$. 由平均值不等式, 得 $\sum_{i=1}^{n} b_{i} A_{i} \leqslant$ $\frac{1}{2}\left(\frac{1}{n-1}\right)^{n-1}$, 所以 $\lambda \leqslant \frac{1}{2}\left(\frac{1}{n-1}\right)^{n-1}$. 另一方面, 当 $a_{1}=a_{2}=\frac{1}{2(n-1)}$, $a_{3}=\cdots=a_{n}=\frac{1}{n-1}, b_{1}=b_{2}=\frac{1}{2}, b_{3}=\cdots=b_{n}=0$ 时, $\prod_{i=1}^{n} a_{i}=$ $\frac{1}{2}\left(\frac{1}{n-1}\right)^{n-1} \sum_{i=1}^{n} a_{i} b_{i}$, 所以 $\lambda \geqslant \frac{1}{2}\left(\frac{1}{n-1}\right)^{n-1}$. 故 $\lambda_{\min }=\frac{1}{2}\left(\frac{1}{n-1}\right)^{n-1}$.

  \item $\sum_{k=1}^{n} \frac{1}{S_{k}}\left(l k+\frac{1}{4} l^{2}\right)=\sum_{k=1}^{n}\left[\frac{1}{S_{k}}\left(\frac{l}{2}+k\right)^{2}-\frac{k^{2}}{S_{k}}\right]=\left(\frac{l}{2}+1\right)^{2} \frac{1}{S_{1}}-\frac{n^{2}}{S_{n}}+$ $\sum_{k=2}^{n}\left[\frac{1}{S_{k}}\left(\frac{l}{2}+k\right)^{2}-\frac{(k-1)^{2}}{S_{k-1}}\right]$. 由于当 $k=2,3, \cdots, n$ 时, 有 $\frac{1}{S_{k}}\left(\frac{l}{2}+k\right)^{2}-$ $\frac{(k-1)^{2}}{S_{k-1}}=\frac{1}{S_{k} S_{k-1}}\left[\left(\frac{l}{2}+1\right)^{2} S_{k-1}+(l+2)(k-1) S_{k-1}+(k-1)^{2}\left(S_{k-1}-S_{k}\right)\right]=$

\end{enumerate}

$\frac{1}{S_{k} S_{k-1}}\left[\left(\frac{l}{2}+1\right)^{2} S_{k-1}-\left(\sqrt{a_{k}}(k-1)-\left(\frac{l}{2}+1\right) \frac{S_{k-1}}{\sqrt{a_{k}}}\right)^{2}+\left(\frac{l}{2}+1\right)^{2} \frac{S_{k-1}^{2}}{a_{k}}\right] \leqslant$ $\frac{1}{S_{k} S_{k-1}}\left(\frac{l}{2}+1\right)^{2}\left(S_{k-1}+\frac{S_{k-1}^{2}}{a_{k}}\right)=\left(\frac{l}{2}+1\right)^{2} \frac{1}{a_{k}}$, 所以 $\sum_{k=1}^{n} \frac{1}{S_{k}}\left(l k+\frac{1}{4} l^{2}\right) \leqslant$ $\left(\frac{l}{2}+1\right)^{2} \sum_{k=1}^{n} \frac{1}{a_{k}}-\frac{n^{2}}{S_{n}}<\left(\frac{l}{2}+1\right)^{2} \sum_{k=1}^{n} \frac{1}{a_{k}}$. 显然, $\frac{l}{2}+1 \leqslant m$, 即 $l \leqslant 2(m-$ $1)$ 满足所要之条件. 另一方面, 当 $l>2(m-1)$, 即 $l \geqslant 2 m-1$ 时, 任意给定 $a_{1}>0$. 令 $a_{k}=\frac{l+2}{2(k-1)} S_{k-1}, k=2,3, \cdots, n$, 则 $\sum_{k=1}^{n} \frac{1}{S_{k}}\left(l k+\frac{1}{4} l^{2}\right)=$ $\left(\frac{l}{2}+1\right)^{2} \sum_{i=1}^{n} \frac{1}{a_{i}}-\frac{n^{2}}{S_{n}}=\left[\left(\frac{l}{2}+1\right)^{2}-1\right] \sum_{k=1}^{n} \frac{1}{a_{k}}+\sum_{k=1}^{n} \frac{1}{a_{k}}-\frac{n^{2}}{S_{n}}$. 由 $l \geqslant 2 m-$ 1 , 可推出 $\left(\frac{l}{2}+1\right)^{2}-1 \geqslant\left(m+\frac{1}{2}\right)^{2}-1=m^{2}+m+\frac{1}{4}-1>m^{2}$. 由柯西不等式, 得 $n^{2} \leqslant\left(\sum_{k=1}^{n} a_{k}\right)\left(\sum_{k=1}^{n} \frac{1}{a_{k}}\right)=S_{n} \sum_{k=1}^{n} \frac{1}{a_{k}}$, 即 $\sum_{k=1}^{n} \frac{1}{a_{k}}-\frac{n^{2}}{S_{n}} \geqslant 0$. 从而 $\sum_{k=1}^{n} \frac{1}{S_{k}}\left(l k+\frac{1}{4} l^{2}\right)>m^{2} \sum_{k=1}^{n} \frac{1}{a_{k}}$, 于是 $1,2, \cdots, 2(m-1)$ 是满足要求的所有自然数 $l$.

\begin{enumerate}
  \setcounter{enumi}{18}
  \item 若存在 $a_{1}, a_{2}, \cdots, a_{n}$, 由柯西不等式, 得 $\left(\sum_{i=1}^{n} a_{i}\right)^{2} \leqslant n \sum_{i=1}^{n} a_{i}^{2}$. 又 $\left(\sum_{i=1}^{n} a_{i}\right)^{2} \geqslant \sum_{i=1}^{n} a_{i}^{2}$, 所以 $u 、 v$ 满足的必要条件是 $v \leqslant u^{2} \leqslant n v \cdots$ (1). 可以证明, 以上也为充分条件. 若(1)成立, 取正数 $a_{1}=\frac{u+\sqrt{(n-1)\left(n v-u^{2}\right)}}{n}$, 则 $a_{1} \leqslant u$. 若 $n$ $>1$, 再取 $a_{2}=a_{3}=\cdots=a_{n}=\frac{u-a_{1}}{n-1}$, 则 $\sum_{i=1}^{n} a_{i}=u, \sum_{i=1}^{n} a_{i}^{2}=v, a_{1} \geqslant \frac{u-a_{1}}{n-1}$.可以证明, $a_{1}$ 的最大值为 $\frac{u+\sqrt{(n-1)\left(n v-u^{2}\right)}}{n}$. 事实上, 若 $a_{1}>$ $\frac{u+\sqrt{(n-1)\left(n v-u^{2}\right)}}{n}$, 则 $n>1$, 且 $n a_{1}^{2}-2 u a_{1}+u^{2}-(n-1) v>0$, 即 $(n-$ 1) $\left(v-a_{1}^{2}\right)<\left(u-a_{1}\right)^{2} \cdots$ (2). 若有 $a_{2} \geqslant a_{3} \geqslant \cdots \geqslant a_{n} \geqslant 0$, 使 $\sum_{k=2}^{n} a_{i}=u-a_{1}$, $\sum_{k=2}^{n} a_{k}^{2}=v-a_{1}^{2}$, 则由柯西不等式, 得 $\left(u-a_{1}\right)^{2} \leqslant(n-1)\left(v-a_{1}^{2}\right)$, 矛盾. 以下求 $a_{1}$ 的最小值, 设 $a_{1}, a_{2}, \cdots, a_{n}$ 满足所要之条件, 则对于任何 $1 \leqslant i, j \leqslant n$,有 $a_{i}^{2}+a_{j}^{2} \leqslant\left(a_{i}+a_{j}\right)^{2} \cdots$ (3). $a_{i}^{2}+a_{j}^{2} \leqslant a_{i}^{2}+a_{j}^{2}+2\left(a_{1}-a_{i}\right)\left(a_{1}-a_{j}\right)=a_{1}^{2}+$ $\left(a_{i}+a_{j}-a_{1}\right)^{2} \cdots$ (4). 若 $n=1$, 显然 $u^{2}=v$, 且 $a_{1}=u$, 对 $n \geqslant 2$, 显然 $\frac{u}{n} \leqslant$ $a_{1} \leqslant u$. 若存在 $k \in\{1,2, \cdots, n-1\}$, 使得 $a_{1} \leqslant \frac{u}{k}$, 则当 $a_{i}+a_{j} \leqslant a_{1}$ 时,使用(3), $a_{i}+a_{j}>a_{1}$ 时, 使用(4). 重复上述步骤有限次, 得 $v=\sum_{i=1}^{n} a_{i}^{2} \leqslant k a_{1}^{2}+$ $\left(u-k a_{1}\right)^{2} \cdots$ (5). 进一步, 若 $\frac{u}{k+1} \leqslant a_{1} \leqslant \frac{u}{k}$, 由(5)可得 $v=\sum_{i=1}^{n} a_{i}^{2} \leqslant k a_{1}^{2}+$ $\left(u-k a_{1}\right)^{2} \leqslant \frac{u^{2}}{k} \cdots$ (6). 由(1)知, $\frac{u^{2}}{n} \leqslant v \leqslant u^{2}$, 显然 $v=\frac{u^{2}}{n}$ 的充要条件为 $a_{1}=$ $a_{2}=\cdots=a_{n}=\frac{u}{n}$. 若 $v>\frac{u^{2}}{n}$, 则存在 $k \in\{1,2, \cdots, n-1\}$, 使得 $\frac{u^{2}}{k+1}<$ $v \leqslant \frac{u^{2}}{k}$. 可以证明 $a_{1} \geqslant \frac{k u+\sqrt{k\left[(k+1) v-u^{2}\right]}}{k(k+1)} \cdots$ (7). 如果(7)不成立, 则存在 $a_{1}, a_{2}, \cdots, a_{n}$ 满足题设中的条件, 且 $a_{1}<\frac{k u+\sqrt{k\left[(k+1) v-u^{2}\right]}}{k(k+1)} \cdots$ (8). 由于 $0<(k+1) v-u^{2} \leqslant \frac{u^{2}}{k}$, 所以 $a_{1} \leqslant \frac{u}{k}$. 由(5)可知 $v \leqslant k a_{1}^{2}+\left(u-k a_{1}\right)^{2}$, 即 $k(k+1) a_{1}^{2}-2 k u a_{1}+u^{2}-v \geqslant 0$. 再由 $k^{2} u^{2}-k(k+1)\left(u^{2}-v\right)=k[(k+1) v-$\\
$\left.u^{2}\right]>0$ 和(8)可推出 $a_{1} \leqslant \frac{k u-\sqrt{k\left[(k+1) v-u^{2}\right]}}{k(k+1)}<\frac{u}{k+1}$. 于是存在 $k+$ $1 \leqslant m \leqslant n-1$, 使得 $\frac{u}{m+1} \leqslant a_{1}<\frac{u}{m}$. 由(6)可得 $v \leqslant \frac{u^{2}}{m} \leqslant \frac{u^{2}}{k+1}$ 与 $v>\frac{u^{2}}{k+1}$矛盾. 所以 (7)成立. 另一方面, 在 $\frac{u^{2}}{k+1}<v \leqslant \frac{u^{2}}{k}$ 的条件下, 若 $a_{1}=$ $\frac{k u+\sqrt{k\left[(k+1) v-u^{2}\right]}}{k(k+1)}$, 则 $\frac{u}{k+1}<a_{1} \leqslant \frac{u}{k}$ 且 $k(k+1) a_{1}^{2}-2 k u a_{1}+u^{2}-$ $v=0$, 即 $v=k a_{1}^{2}+\left(u-k a_{1}\right)^{2}$. 令 $a_{1}=\cdots=a_{k}=\frac{u}{k}, a_{k+1}=u-k a_{1}$, $a_{k+2}=\cdots=a_{n}=0$, 则 $a_{1}, a_{2}, \cdots, a_{n}$ 满足所要条件. 故当 $\frac{u^{2}}{k+1}<v \leqslant \frac{u^{2}}{k}(k \in$ $\{1,2, \cdots, n-1\})$ 时,$a_{1}$ 的最小值为 $\frac{k n+\sqrt{k\left[(k+1) v-u^{2}\right]}}{k(k+1)}$.

  \item 设 $z_{k}=x_{k}+y_{k} \mathrm{i}, x_{k}, y_{k} \in \mathbf{R}, 1 \leqslant k \leqslant n$. 首先证明 $\frac{x_{k}^{2}}{x_{k}^{2}+y_{k}^{2}} \geqslant 1-$ $r^{2}, 1 \leqslant k \leqslant n \cdots$ (1). 令 $n=\frac{x_{k}^{2}}{x_{k}^{2}+y_{k}^{2}}$. 由 $\left|x_{k}-1\right| \leqslant r<1$ 知 $x_{k}>0$. 于是 $n$ $>0$, 且 $y_{k}^{2}=\left(\frac{1}{n}-1\right) x_{k}^{2}$, 从而 $r^{2} \geqslant\left|z_{n}-1\right|^{2}=\left(x_{k}-1\right)^{2}+\left(\frac{1}{n}-1\right) x_{k}^{2}=$ $\frac{1}{n}\left(x_{k}-n\right)^{2}+1-n \geqslant 1-n$. 即 (1) 成立. 由于 $\left|z_{1}+\cdots+z_{n}\right| \geqslant \mid \operatorname{Re}\left(z_{1}+\cdots\right.$ $\left.+z_{n}\right) \mid=\sum_{k=1}^{n} x_{k} \cdots$ (2). 以及 $\frac{1}{z_{k}}=\frac{x_{k}-y_{k} \mathrm{i}}{x_{k}^{2}+y_{k}^{2}}, 1 \leqslant k \leqslant n$. 所以 $\left|\frac{1}{z_{1}}+\cdots+\frac{1}{z_{n}}\right| \geqslant$ $\left|\operatorname{Re}\left(\frac{1}{z_{1}}+\cdots+\frac{1}{z_{n}}\right)\right|=\sum_{k=1}^{n} \frac{x_{k}}{x_{k}^{2}+y_{k}^{2}} \cdots$ (3). 注意到 $x_{k}>0,1 \leqslant k \leqslant n$. 由 (2)、 (3), 以及柯西不等式, 可得 $\left|z_{1}+\cdots+z_{n}\right| \cdot\left|\frac{1}{z_{1}}+\cdots+\frac{1}{z_{n}}\right| \geqslant$ $\left(\sum_{k=1}^{n} x_{k}\right)\left(\sum_{k=1}^{n} \frac{x_{k}}{x_{k}^{2}+y_{k}^{2}}\right) \geqslant\left(\sum_{k=1}^{n} \sqrt{\frac{x_{k}^{2}}{x_{k}^{2}+y_{k}^{2}}}\right)^{2} \geqslant\left(n \sqrt{1-r^{2}}\right)^{2}=n^{2}\left(1-r^{2}\right)$.

  \item 由柯 西不等式 得 $\left|\sum_{i=1}^{n} a_{i}\right| \leqslant \sum_{i=1}^{n}\left|a_{i}\right| \leqslant\left(n \sum_{i=1}^{n} a_{i}^{2}\right)^{\frac{1}{2}}$. 于 是 $2\left(n \sum_{i=1}^{n} a_{i}^{2}\right)^{\frac{1}{2}} \geqslant \sum_{i=1}^{n}\left|a_{i}\right|+\left|\sum_{i=1}^{n} a_{i}\right|=1$, 即 $\sum_{i=1}^{n} a_{i}^{2} \geqslant \frac{1}{4 n}$, 当 $a_{1}=\cdots=a_{n}=$ $\frac{1}{2 n}$ 时, 等式成 立. 另 一 方面, $1=\sum_{i=1}^{n}\left|a_{i}\right|+\left|\sum_{i=1}^{n} a_{i}\right| \geqslant$ $\left|\sum_{\substack{i=1 \\ i \neq j}}^{n}\left(a_{i}-a_{j}\right)-\sum_{i=1}^{n} a_{i}\right|=2\left|a_{j}\right|$, 即 $\left|a_{j}\right| \leqslant \frac{1}{2}, 1 \leqslant j \leqslant n$. 从而 $\sum_{i=1}^{n} a_{i}^{2}=$\\
$\sum_{i=1}^{n}\left|a_{i}\right| \cdot\left|a_{i}\right| \leqslant \frac{1}{2} \sum_{i=1}^{n}\left|a_{i}\right|=\frac{1}{2}$, 当 $a_{1}=\frac{1}{2}, a_{2}=-\frac{1}{2}, a_{3}=\cdots=a_{n}=$ 0 时, 等式成立. 故 $\sum_{i=1}^{n} a_{i}^{2}$ 的最大值为 $\frac{1}{2}$, 最小值为 $\frac{1}{4 n}$.

  \item 由柯西不等式得 $\left(\sum_{i=1}^{n} \sqrt{a_{i}^{2}-i^{2}}\right)^{2} \leqslant \sum_{i=1}^{n}\left(a_{i}+i\right) \cdot \sum_{i=1}^{n}\left(a_{i}-i\right)=$ $\left(\sum_{i=1}^{n} a_{i}+\frac{n(n+1)}{2}\right)\left(\sum_{i=1}^{n} a_{i}-\frac{n(n+1)}{2}\right)=\left(\sum_{i=1}^{n} a_{i}\right)^{2}-\frac{n^{2}(n+1)^{2}}{4}$. 于是, 由平均值不等式得 $\frac{\left(\sum_{i=1}^{n} a_{i}\right)^{2}}{\sum_{i=1}^{n} \sqrt{a_{i}^{2}-i^{2}}} \geqslant \frac{\left(\sum_{i=1}^{n} a_{i}\right)^{2}}{\sqrt{\left(\sum_{i=1}^{n} a_{i}\right)^{2}-\frac{n^{2}(n+1)^{2}}{4}}}=$ $\sqrt{\left(\sum_{i=1}^{n} a_{i}\right)^{2}-\frac{n^{2}(n+1)^{2}}{4}}+\frac{n^{2}(n+1)^{2}}{4 \sqrt{\left(\sum_{i=1}^{n} a_{i}\right)^{2}-\frac{n^{2}(n+1)^{2}}{4}}} \geqslant n(n+1)$, 且当 $a_{i}$ $=\sqrt{2} i, 1 \leqslant i \leqslant n$ 时等式成立. 故命题成立.

  \item 对 $n \geqslant 1,2 \leqslant k \leqslant n$, 由柯西不 等式 得 $\left(x_{k-1}+\left(x_{k}-\right.\right.$ $\left.\left.x_{k-1}\right)\right)\left(\frac{(k-1)^{2}}{x_{k-1}}+\frac{3^{2}}{x_{k}-x_{k-1}}\right) \geqslant(k-1+3)^{2}$, 即 $\frac{9}{x_{k}-x_{k-1}} \geqslant \frac{(k+2)^{2}}{x_{k}}-$ $\frac{(k-1)^{2}}{x_{k-1}}$. 对 $k=2,3, \cdots, n$ 求和, 并两边同时加 $\frac{9}{x_{1}}$, 则 $9 \sum_{k=1}^{n} \frac{1}{x_{k}-x_{k-1}} \geqslant$ $4 \sum_{k=1}^{n} \frac{k+1}{x_{k}}+\frac{n^{2}}{x_{n}}>4 \sum_{k=1}^{n} \frac{k+1}{x_{k}}$, 从而 $\lambda \geqslant \frac{4}{9}$. 再令 $x_{0}=0, x_{k}=x_{k-1}+k(k+$ 1), $k \geqslant 1$. 即 $x_{k}=\frac{1}{3} k(k+1)(k+2), k \geqslant 1$, 于是 $\sum_{k=1}^{n} \frac{1}{k(k+1)}=1-\frac{1}{n+1}$, $\lambda \sum_{k=1}^{n} \frac{k+1}{x_{k}}=3 \lambda \sum_{k=1}^{n} \frac{1}{k(k+2)}=\frac{3}{2} \lambda \sum_{k=1}^{n}\left(\frac{1}{k}-\frac{1}{k+2}\right)=\frac{3}{2} \lambda\left(1+\frac{1}{2}-\frac{1}{n+1}-\right.$ $\left.\frac{1}{n+2}\right)$ 令 $n \rightarrow+\infty$, 则 $\lambda \leqslant \frac{4}{9}$. 故 $\lambda$ 的最大值为 $\frac{4}{9}$.

  \item 不妨设 $x_{i} \geqslant 0$, 考虑 $A=\left\{\sum_{i=1}^{n} e_{i} x_{i} \mid e_{i} \in\{0,1,2, \cdots, k-1\}\right\}$. 若 $A$ 中的数均不相同, 则 $|A|=k^{n}$. 由柯西不等式, 得 $0 \leqslant \sum_{i=1}^{n} e_{i} x_{i} \leqslant(k-$ 1) $\sum_{i=1}^{n} x_{i} \leqslant(k-1) \sqrt{\sum_{i=1}^{n} x_{i}^{2}} \cdot \sqrt{n}=\sqrt{n}(k-1)$. 所以 $A$ 中的 $k^{n}$ 个数落在 $[0$, $(k-1) \sqrt{n}]$ 内, 将它分成 $k^{n}-1$ 个小区间, 每个小区间长 $\frac{(k-1) \sqrt{n}}{k^{n}-1}$, 依抽屉原\\
理知存在 $\sum_{i=1}^{n} e_{i} x_{i} 、 \sum_{i=1}^{n} d_{i} x_{i}$ 落在同一小区间上 (包括端点), 令 $a_{i}=e_{i}-d_{i}$, 则 $\left|a_{i}\right| \leqslant k-1$, 满足要求.

  \item 令 $a=\tan s, b=\tan t, c=\tan u, d=\tan v$, 则 $a, b, c, d \in \mathbf{R}_{+}$,由 $s+t+u+v=\pi$, 得 $\tan (s+t)+\tan (u+v)=0$. 即 $\frac{a+b}{1-a b}+\frac{c+d}{1-c d}=$ 0 . 两边乘以 $(1-a b)(1-c d)$, 得 $a+b+c+d=a b c+b c d+c d a+d a b$. 推出 $(a+b)(a+c)(a+d)=\left(a^{2}+1\right)(a+b+c+d)$, 即 $\frac{a^{2}+1}{a+b}=$ $\frac{(a+c)(a+d)}{a+b+c+d}$. 类似, 得到 $\frac{a^{2}+1}{a+b}+\frac{b^{2}+1}{b+c}+\frac{c^{2}+1}{c+d}+\frac{d^{2}+1}{d+a}=a+b+c+$ $d$. 由柯西不 等式, 得 $2(a+b+c+d)^{2}=2(a+b+c+$ d) $\left(\frac{a^{2}+1}{a+b}+\frac{b^{2}+1}{b+c}+\frac{c^{2}+1}{c+d}+\frac{d^{2}+1}{d+a}\right) \geqslant\left(\sqrt{a^{2}+1}+\sqrt{b^{2}+1}+\sqrt{c^{2}+1}+\right.$ $\left.\sqrt{d^{2}+1}\right)^{2}$, 即 $\sqrt{a^{2}+1}+\sqrt{b^{2}+1}+\sqrt{c^{2}+1}+\sqrt{d^{2}+1} \leqslant \sqrt{2}(a+b+c+$ d), 等价于 $\frac{1}{\cos s}+\frac{1}{\cos t}+\frac{1}{\cos u}+\frac{1}{\cos v} \leqslant \sqrt{2}\left(\frac{\sin s}{\cos s}+\frac{\sin t}{\cos t}+\frac{\sin u}{\cos u}+\frac{\sin v}{\cos v}\right)$.

  \item 设 $x=A B_{1}, y=B C_{1}, z=C A_{1}$. 要证 $\sqrt{\frac{x}{x+y}}+\sqrt{\frac{y}{y+z}}+\sqrt{\frac{z}{z+x}}$ $\leqslant \frac{3}{\sqrt{2}}$, 即证 $\frac{1}{\sqrt{1+a^{2}}}+\frac{1}{\sqrt{1+b^{2}}}+\frac{1}{\sqrt{1+c^{2}}} \leqslant \frac{3}{\sqrt{2}}$, 其中 $a 、 b 、 c$ 为正实数, 且 $a b c=1$. 不妨设 $a b \leqslant 1$. 由柯西一施瓦兹不等式得 $\frac{1}{\sqrt{1+a^{2}}}+$ $\frac{1}{\sqrt{1+b^{2}}} \leqslant \sqrt{2\left(\frac{1}{1+a^{2}}+\frac{1}{1+b^{2}}\right)}, \frac{1}{1+a^{2}}+\frac{1}{1+b^{2}}=1+\frac{1-a^{2} b^{2}}{\left(1+a^{2}\right)\left(1+b^{2}\right)} \leqslant 1+$ $\frac{1-a^{2} b^{2}}{(1+a b)^{2}}=\frac{2}{1+a b}, \frac{1}{\sqrt{1+c^{2}}} \leqslant \frac{\sqrt{2}}{1+c}$. 由算术一几何均值不等式得 $\frac{1}{\sqrt{1+a^{2}}}+$ $\frac{1}{\sqrt{1+b^{2}}}+\frac{1}{\sqrt{1+c^{2}}} \leqslant 2 \sqrt{\frac{c}{1+c}}+\frac{\sqrt{2}}{1+c}=\frac{\sqrt{2}}{1+c}[\sqrt{2 c(c+1)}+1] \leqslant$ $\frac{\sqrt{2}}{1+c}\left(\frac{2 c+c+1}{2}+1\right)=\frac{3}{\sqrt{2}}$.

  \item 因为 $\frac{2}{9} \sum_{1 \leqslant i<j \leqslant 4} \frac{1}{\sqrt{\left(s-a_{i}\right)\left(s-a_{j}\right)}} \geqslant \frac{4}{9} \sum_{1 \leqslant i<j \leqslant 4} \frac{1}{\left(s-a_{i}\right)+\left(s-a_{j}\right)} \cdots$ (1).所以只要证明: $\sum_{i=1}^{4} \frac{1}{a_{i}+s} \leqslant \frac{4}{9} \sum_{1 \leqslant i<j \leqslant 4} \frac{1}{\sqrt{\left(s-a_{i}\right)+\left(s-a_{j}\right)}}$. 记 $a_{1}=a, a_{2}=b$,\\
$a_{3}=c, a_{4}=d$. 上式等价于 $\frac{2}{9}\left(\frac{1}{a+b}+\frac{1}{a+c}+\frac{1}{a+d}+\frac{1}{b+c}+\frac{1}{b+d}+\frac{1}{c+d}\right) \geqslant$ $\frac{1}{3 a+b+c+d}+\frac{1}{a+3 b+c+d}+\frac{1}{a+b+3 c+d}+\frac{1}{a+b+c+3 d}$. 由柯西不等式, 可得 $(3 a+b+c+d)\left(\frac{1}{a+b}+\frac{1}{a+c}+\frac{1}{a+d}\right) \geqslant 9$, 即 $\frac{1}{9}\left(\frac{1}{a+b}+\frac{1}{a+c}+\frac{1}{a+d}\right) \geqslant \frac{1}{3 a+b+c+d}$. 轮换相加即得(1). 证毕.

  \item 不妨设 $a \geqslant b \geqslant c$. 于是, $\sqrt{a+b-c}-\sqrt{a}=\frac{(a+b-c)-a}{\sqrt{a+b-c}+\sqrt{a}} \leqslant$ $\frac{b-c}{\sqrt{b}+\sqrt{c}}=\sqrt{b}-\sqrt{c}$. 因此, $\frac{\sqrt{a+b-c}}{\sqrt{a}+\sqrt{b}-\sqrt{c}} \leqslant 1 \cdots$ (1). 设 $p=\sqrt{a}+\sqrt{b}, q=\sqrt{a}-\sqrt{b}$.则 $a-b=p q, p \geqslant 2 \sqrt{c}$. 由柯西不等式有 $\left(\frac{\sqrt{b+c-a}}{\sqrt{b}+\sqrt{c}-\sqrt{a}}+\frac{\sqrt{c+a-b}}{\sqrt{c}+\sqrt{a}-\sqrt{b}}\right)^{2}=$ $\left(\frac{\sqrt{c-p q}}{\sqrt{c}-q}+\frac{\sqrt{c+p q}}{\sqrt{c}+q}\right)^{2} \leqslant\left(\frac{c-p q}{\sqrt{c}-q}+\frac{c+p q}{\sqrt{c}+q}\right)\left(\frac{1}{\sqrt{c}-q}+\frac{1}{\sqrt{c}+q}\right)=\frac{2\left(c \sqrt{c}-p q^{2}\right)}{c-q^{2}} \cdot$ $\frac{2 \sqrt{c}}{c-q^{2}}=4 \times \frac{c^{2}-\sqrt{c} p q^{2}}{\left(c-q^{2}\right)^{2}} \leqslant 4 \times \frac{c^{2}-2 c q^{2}}{\left(c-q^{2}\right)^{2}} \leqslant 4$. 从而, $\frac{\sqrt{b+c-a}}{\sqrt{b}+\sqrt{c}-\sqrt{a}}+\frac{\sqrt{c+a-b}}{\sqrt{c}+\sqrt{a}-\sqrt{b}}$ $\leqslant 2$. 结合式(1)即得所证不等式.

  \item 令 $f(a, b, c, k)=\frac{a}{a+k b}+\frac{b}{b+k c}+\frac{c}{c+k a}$, 注意到 $3-f(a, b, c$, $k)=3-\left(\frac{a}{a+k b}+\frac{b}{b+k c}+\frac{c}{c+k a}\right)=\frac{k b}{a+k b}+\frac{k c}{b+k c}+\frac{k a}{c+k a}=\frac{b}{b+\frac{a}{k}}+$ $\frac{a}{a+\frac{c}{k}}+\frac{c}{c+\frac{b}{k}}=f\left(b, a, c, \frac{1}{k}\right)$. 如果 $k \geqslant 2$, 此时由柯西不等式有 $\left(\frac{a}{a+k b}+\frac{b}{b+k c}+\frac{c}{c+k a}\right)(a(a+k b)+b(b+k c)+c(c+k a)) \geqslant(a+b+c)^{2}$,所以 $\frac{a}{a+k b}+\frac{b}{b+k c}+\frac{c}{c+k a} \geqslant \frac{(a+b+c)^{2}}{(a+b+c)^{2}+(k-2)(a b+b c+c a)} \cdots$于 $(a+b+c)^{2} \geqslant 3(a b+b c+c a)$, 所以 $\frac{(a+b+c)^{2}}{(a+b+c)^{2}+(k-2)(a b+b c+c a)}$ $\geqslant \frac{1}{1+\frac{k-2}{3}}=\frac{3}{k+1}$. 如果 $k<2$, 由不等式 (1) 有 $\frac{a}{a+k b}+\frac{b}{b+k c}+\frac{c}{c+k a}>$

  \item 综上所述. (1) 当 $k \geqslant 2$ 时, $\frac{a}{a+k b}+\frac{b}{b+k c}+\frac{c}{c+k a} \geqslant \frac{3}{1+k} \cdot \frac{a}{a+k b}+$\\
$\frac{b}{b+k c}+\frac{c}{c+k a}=3-f\left(b, a, c, \frac{1}{k}\right)<2$. 令 $a=n b=n^{2} c=n^{2}$, 令 $n \rightarrow$ $+\infty$, 则 $\frac{a}{a+k b}+\frac{b}{b+k c}+\frac{c}{c+k a} \rightarrow 2$. 令 $a=b=c$, 则 $\frac{a}{a+k b}+\frac{b}{b+k c}+\frac{c}{c+k a}$ $=\frac{3}{1+k}$. 因此, $\frac{a}{a+k b}+\frac{b}{b+k c}+\frac{c}{c+k a}$ 的取值范围为 $\left[\frac{3}{1+k}, 2\right)$. 类似 (1)这样分析, 可得 (2) 当 $\frac{1}{2}<k<2$ 时, $\frac{a}{a+k b}+\frac{b}{b+k c}+\frac{c}{c+k a}$ 的取值范围为 $(1,2)$. (3) 当 $k \leqslant \frac{1}{2}$ 时, $\frac{a}{a+k b}+\frac{b}{b+k c}+\frac{c}{c+k a}$ 的取值范围为 $\left(1, \frac{3}{1+k}\right]$.

\end{enumerate}

[1] 李胜宏, 冯祖鸣. 高中数学竞赛培训教材 $[\mathrm{M}]$. 杭州: 浙江大学出版社, 2004.

[2] Titu Andreescu, Zuming Feng. USA and International Mathematical Oilmpiads. 2002, 2003. Mathematical Association of America, 2003.

[3] 熊斌, 刘诗雄. 高中竞赛数学教程 [M]. 武汉: 武汉大学出版社, 2003.

[4] 常庚哲等. 中学数学竞赛导引[M]. 上海: 上海教育出版社, 1997.

[5] 单墫.数学竞赛研究教程 $[\mathrm{M}]$. 南京: 江苏教育出版社, 1993.

[]李胜宏, 李名德. 高中数学竞赛培优教程 (专题讲座) [M]. 杭州: 浙江大学出版社. 2003.

[7] 唐立华. 金牌奥赛兵法・高中数学 $[\mathrm{M}]$. 上海: 文汇出版社, 2002 .

[8] 严镇军. 数学奥林匹克高中版新版 - 竞赛篇 $[\mathrm{M}]$. 北京: 北京大学出版社, 1993.

$[9]$ 冷岗松等. 奥林匹克数学中的代数问题 $[\mathrm{M}]$. 长沙: 湖南师范大学出版社,2004.

[10] 李胜宏. 数学分析 [M]. 杭州: 浙江大学出版社, 2009.

[11] 李名德, 李胜宏. 高中数学竞赛培优教程 $[\mathrm{M}]$. 杭州: 浙江大学出版社,2011.

[12] 熊斌, 冷岗松. 赛前集训: 高中数学联赛专题辅导 $[\mathrm{M}]$. 上海:华东师范大学出版社, 2004.\\
算术平均值 $A_{n}=\frac{a_{1}+a_{2}+\cdots+a_{n}}{n}$.

几何平均值 $G_{n}=\sqrt[n]{a_{1} a_{2} \cdots a_{n}}$.

调和平均值 $H_{n}=\frac{n}{\frac{1}{a_{1}}+\frac{1}{a_{2}}+\cdots+\frac{1}{a_{n}}}$.

平方平均值 $Q_{n}=\sqrt{\frac{a_{1}^{2}+a_{2}^{2}+\cdots+a_{n}^{2}}{n}}$.

平均值不等式 设 $a_{1}, a_{2}, \cdots, a_{n}$ 为非负实数, 则
$$
H_{n} \leqslant G_{n} \leqslant A_{n} \leqslant Q_{n}
$$

等号成立的充分必要条件是 $a_{1}=a_{2}=\cdots=a_{n}$.

Jacobsthai 不等式 设 $x, y$ 为正实数, $n$ 为正整数, 则
$$
x^{n+1}+n y^{n+1} \geqslant(n+1) y^{n} x
$$

伯努利 (Bernoulli)不等式

设实数 $x>-1$, 则当实数 $r \leqslant 0$ 或 $r \geqslant 1$ 时, $(1+x)^{r} \geqslant 1+r x$;当 $0 \leqslant r \leqslant 1$ 时,$(1+x)^{r} \leqslant 1+r x$;

设实数 $x_{i} \geqslant-1,1 \leqslant i \leqslant n$, 且 $x_{i}$ 与 $x_{j}$ 同号 $, 1 \leqslant i, j \leqslant n$, 则
$$
\left(1+x_{1}\right)\left(1+x_{2}\right) \cdots\left(1+x_{n}\right) \geqslant 1+x_{1}+\cdots+x_{n}
$$

排序不等式 设两个实数组 $a_{1}, a_{2}, \cdots, a_{n}$ 和 $b_{1}, b_{2}, \cdots, b_{n}$, 满足 $a_{1}$ $\leqslant a_{2} \leqslant \cdots \leqslant a_{n} ; b_{1} \leqslant b_{2} \leqslant \cdots \leqslant b_{n}$.

则
$$
\begin{aligned}
& a_{1} b_{1}+a_{2} b_{2}+\cdots+a_{n} b_{n}(\text { 同序乘积之和 }) \\
\geqslant & a_{1} b_{j_{1}}+a_{2} b_{j_{2}}+\cdots+a_{n} b_{j_{n}} \text { (乱序乘积之和) } \\
\geqslant & \left.a_{1} b_{n}+a_{2} b_{n-1}+\cdots+a_{n} b_{1} \text { (反序乘积之和 }\right),
\end{aligned}
$$

其中 $j_{1}, j_{2}, \cdots, j_{n}$ 是 $1,2, \cdots, n$ 的一个排列.

切比雪夫(Chebyshev) 不等式 设 $a_{1}, a_{2}, \cdots, a_{n}, b_{1}, b_{2}, \cdots, b_{n}$ 满足 $a_{1} \leqslant a_{2} \leqslant \cdots \leqslant a_{n}, b_{1} \leqslant b_{2} \leqslant \cdots \leqslant b_{n}$, 则
$$
n \sum_{k=1}^{n} a_{k} b_{n-k+1} \leqslant \sum_{k=1}^{n} a_{k} \sum_{k=1}^{n} b_{k} \leqslant n \sum_{k=1}^{n} a_{k} b_{k} .
$$

凹函数 $\quad$ 设 $y=f(x), x \in(a, b)$ 满足
$$
f\left(\frac{x_{1}+x_{2}}{2}\right) \geqslant \frac{f\left(x_{1}\right)+f\left(x_{2}\right)}{2}, x_{1}, x_{2} \in(a, b)
$$

则称 $y=f(x)$ 在 $(a, b)$ 上为凹函数.

琴生(Jensen)不等式 设 $y=f(x), x \in(a, b)$ 为凹函数,则
$$
f\left(\frac{x_{1}+x_{2}+\cdots+x_{n}}{n}\right) \geqslant \frac{f\left(x_{1}\right)+f\left(x_{2}\right)+\cdots+f\left(x_{n}\right)}{n}
$$

其中 $x_{1}, x_{2}, \cdots, x_{n} \in(a, b)$.

加权琴生不等式 设 $y=f(x), x \in(a, b)$ 为凹函数, 则

对 $x_{1}, x_{2}, \cdots, x_{n} \in(a, b)$, 有
$$
\frac{1}{p_{1}} f\left(x_{1}\right)+\frac{1}{p_{2}} f\left(x_{2}\right)+\cdots+\frac{1}{p_{n}} f\left(x_{n}\right) \leqslant f\left(\frac{x_{1}}{p_{1}}+\frac{x_{2}}{p_{2}}+\cdots+\frac{x_{n}}{p_{n}}\right)
$$

其中 $p_{i}>0,1 \leqslant i \leqslant n$, 且 $\sum_{i=1}^{n} \frac{1}{p_{i}}=1$.

舒尔(Schur)不等式 设 $x, y, z \geqslant 0, r \in \mathbf{R}$, 则
$$
x^{r}(x-y)(x-z)+y^{r}(y-z)(y-x)+z^{r}(z-x)(z-y) \geqslant 0
$$

阿贝尔(Abel)求和 设 $a_{1}, a_{2}, \cdots, a_{n} ; b_{1}, b_{2}, \cdots, b_{n} ; A_{k}=a_{1}+a_{2}+$ $\cdots+a_{k}, 1 \leqslant k \leqslant n, A_{0}=0$. 则 $\sum_{k=1}^{n} a_{k} b_{k}=A_{n} b_{n}+\sum_{i=1}^{n-1} A_{i}\left(b_{i}-b_{i+1}\right)$.

幂平均不等式 设 $a_{1}, a_{2}, \cdots, a_{n}$ 为正实数, $\alpha>\beta>0$, 则
$$
\left(\frac{a_{1}^{\beta}+a_{2}^{\beta}+\cdots+a_{n}^{\beta}}{n}\right)^{\frac{1}{\beta}} \leqslant\left(\frac{a_{1}^{\alpha}+a_{2}^{\alpha}+\cdots+a_{n}^{\alpha}}{n}\right)^{\frac{1}{\alpha}}
$$

加权平均值不等式 设 $a_{i}>0, \alpha_{i}>0,1 \leqslant i \leqslant n, \sum_{i=1}^{n} \alpha_{i}=1$, 则
$$
a_{1}^{\alpha_{1}} a_{2}^{\alpha_{2}} \cdots a_{n}^{\alpha_{n}} \leqslant \alpha_{1} a_{1}+\alpha_{2} a_{2}+\cdots+\alpha_{n} a_{n}
$$

卡尔松(Carlson)不等式 设 $a_{i j} \geqslant 0,1 \leqslant i \leqslant n, 1 \leqslant j \leqslant m, \alpha_{i} \geqslant 0$, $1 \leqslant i \leqslant m, \sum_{i=1}^{m} \alpha_{i}=1$, 则
$$
\prod_{j=1}^{m}\left(\sum_{i=1}^{n} a_{i j}\right)^{a_{j}} \geqslant \sum_{i=1}^{n}\left(\prod_{j=1}^{m} a_{i j}^{a_{j}}\right)
$$

柯西(Cauchy)不等式 设 $a_{i}, b_{i} \in \mathbf{R}, 1 \leqslant i \leqslant n$, 则
$$
\left(a_{1} b_{1}+a_{2} b_{2}+\cdots+a_{n} b_{n}\right)^{2} \leqslant\left(a_{1}^{2}+a_{2}^{2}+\cdots+a_{n}^{2}\right)\left(b_{1}^{2}+b_{2}^{2}+\cdots+b_{n}^{2}\right)
$$

当且仅当 $\frac{a_{1}}{b_{1}}=\frac{a_{2}}{b_{2}}=\cdots=\frac{a_{n}}{b_{n}}$ (规定 $a_{i}=0$ 时, $b_{i}=0$ ) 时等号成立.

拉格朗日(Lagrange)恒等式 设 $a_{i}, b_{i} \in \mathbf{R}, 1 \leqslant i \leqslant n$, 则
$$
\sum_{i=1}^{n} a_{i}^{2} \cdot \sum_{i=1}^{n} b_{i}^{2}-\left(\sum_{i=1}^{n} a_{i} b_{i}\right)^{2}=\sum_{1 \leqslant i<j \leqslant n}\left(a_{i} b_{j}-a_{j} b_{i}\right)^{2}
$$

Aczel 不等式 设 $a_{i}, b_{i} \in \mathbf{R}, 1 \leqslant i \leqslant n$, 满足 $a_{1}^{2}-a_{2}^{2}-\cdots-a_{n}^{2}>0$ 或 $b_{1}^{2}-$ $b_{2}^{2}-\cdots-b_{n}^{2}>0$, 则
$$
\left(a_{1} b_{1}-a_{2} b_{2}-\cdots-a_{n} b_{n}\right)^{2} \geqslant\left(a_{1}^{2}-a_{2}^{2}-\cdots-a_{n}^{2}\right)\left(b_{1}^{2}-b_{2}^{2}-\cdots-b_{n}^{2}\right)
$$

赫尔德(Hölder)不等式 设 $a_{i}, b_{i}>0,1 \leqslant i \leqslant n, p>0, q>0$

满足 $\frac{1}{p}+\frac{1}{q}=1$, 则
$$
\sum_{i=1}^{n} a_{i} b_{i} \leqslant\left(\sum_{i=1}^{n} a_{i}^{p}\right)^{\frac{1}{p}}\left(\sum_{i=1}^{n} b_{i}^{q}\right)^{\frac{1}{q}}
$$

当且仅当 $a_{i}^{p}=\lambda b_{i}^{q} \quad 1 \leqslant i \leqslant n, \lambda>0$ 时, 等号成立.

闵可夫斯基(Minkowski)不等式 设 $a_{i}, b_{i}>0,1 \leqslant i \leqslant n, k>1$, 则
$$
\left[\sum_{i=1}^{n}\left(a_{i}+b_{i}\right)^{k}\right]^{\frac{1}{k}} \leqslant\left(\sum_{i=1}^{n} a_{i}^{k}\right)^{\frac{1}{k}}+\left(\sum_{i=1}^{n} b_{i}^{k}\right)^{\frac{1}{k}},
$$

当且仅当 $\frac{a_{1}}{b_{1}}=\frac{a_{2}}{b_{2}}=\cdots=\frac{a_{n}}{b_{n}}$ 时, 等号成立.
\end{comment}