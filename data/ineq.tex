\chapter{不等式}
\begin{comment}
\section{平均值不等式及其证明}
平均值不等式是最基本的重要不等式之一, 在不等式理论研究和证明中占有重要的位置. 平均值不等式的证明有许多种方法. 这里, 我们选了部分具有代表意义的证明方法, 其中用来证明平均值不等式的许多结论, 其本身又具有重要的意义. 特别是, 在许多竞赛的书籍中, 都有专门的章节介绍和讨论, 如数学归纳法、变量替换、恒等变形和分析综合方法等, 这些也是证明不等式的常用方法和技巧. 希望大家能认真思考和好好掌握, 熟悉不等式的证明.

\section*{1. 1 平均值不等式}
对任意非负实数 $a 、 b$, 有

$$
(\sqrt{a}-\sqrt{b})^{2} \geqslant 0
$$

于是, 得

$$
\frac{a+b}{2} \geqslant \sqrt{a b}
$$

一般地, 假设 $a_{1}, a_{2}, \cdots, a_{n}$ 为 $n$ 个非负实数, 它们的算术平均值记为

$$
A_{n}=\frac{a_{1}+a_{2}+\cdots+a_{n}}{n}
$$

几何平均值记为

$$
G_{n}=\left(a_{1} a_{2} \cdots a_{n}\right)^{\frac{1}{n}}=\sqrt[n]{a_{1} a_{2} \cdots a_{n}}
$$

算术平均值与几何平均值之间有如下的关系

$$
\frac{a_{1}+a_{2}+\cdots+a_{n}}{n} \geqslant \sqrt[n]{a_{1} a_{2} \cdots a_{n}}
$$

即

$$
A_{n} \geqslant G_{n}
$$

当且仅当 $a_{1}=a_{2}=\cdots=a_{n}$ 时, 等号成立.

上述不等式称为平均值不等式,或简称为均值不等式.

平均值不等式的表达形式简单,容易记住,但它的证明和应用非常灵活、广泛, 其证明有多种不同的方法. 为使大家理解和掌握, 这里我们选择了其中的几种典型的证明方法. 当然, 有些方法是几个知识点的结合, 很难将它们归类,有些大体相同或相似,但选择的变量不同,或处理的方式不同,导致证明的难易不同,所以,我们将它们看作是不同的方法.

\section{1. 2 平均值不等式的证明}
\section*{证法一(归纳法)}
(1) 当 $n=2$ 时,已知结论成立.

(2)假设对 $n=k$ (正整数 $k \geqslant 2$ )时命题成立, 即对于 $a_{i}>0, i=1$, $2, \cdots, k$, 有

$$
\left(a_{1} a_{2} \cdots a_{k}\right)^{\frac{1}{k}} \leqslant \frac{a_{1}+a_{2}+\cdots+a_{k}}{k}
$$

那么, 当 $n=k+1$ 时, 由于

$$
A_{k+1}=\frac{a_{1}+a_{2}+\cdots+a_{k+1}}{k+1}, G_{k+1}=\sqrt[k+1]{a_{1} a_{2} \cdots a_{k} a_{k+1}}
$$

关于 $a_{1}, a_{2}, \cdots, a_{k+1}$ 是对称的, 任意对调 $a_{i}$ 与 $a_{j}(i \neq j)$, 即将 $a_{i}$ 写成 $a_{j}, a_{j}$写成 $a_{i}, A_{k+1}$ 和 $G_{k+1}$ 的值不改变,因此不妨设 $a_{1}=\min \left\{a_{1}, a_{2}, \cdots, a_{k+1}\right\}$, $a_{k+1}=\max \left\{a_{1}, a_{2}, \cdots, a_{k+1}\right\}$, 显然 $a_{1} \leqslant A_{k+1} \leqslant a_{k+1}$, 以及

$$
A_{k+1}\left(a_{1}+a_{k+1}-A_{k+1}\right)-a_{1} a_{k+1}=\left(a_{1}-A_{k+1}\right)\left(A_{k+1}-a_{k+1}\right) \geqslant 0
$$

即

$$
A_{k+1}\left(a_{1}+a_{k+1}-A_{k+1}\right) \geqslant a_{1} a_{k+1}
$$

对 $k$ 个正数 $a_{2}, a_{3}, \cdots, a_{k}, a_{1}+a_{k+1}-A_{k+1}$, 由归纳假设, 得

$$
\frac{a_{2}+a_{3}+\cdots+a_{k}+\left(a_{1}+a_{k+1}-A_{k+1}\right)}{k} \geqslant \sqrt[k]{a_{2} a_{3} \cdots a_{k}\left(a_{1}+a_{k+1}-A_{k+1}\right)}
$$

而

$$
\frac{a_{2}+a_{3}+\cdots+a_{k}+\left(a_{1}+a_{k+1}-A_{k+1}\right)}{k}=\frac{(k+1) A_{k+1}-A_{k+1}}{k}=A_{k+1},
$$

于是

$$
A_{k+1}^{k} \geqslant a_{2} a_{3} \cdots a_{k}\left(a_{1}+a_{k+1}-A_{k+1}\right)
$$

两边乘以 $A_{k+1}$, 得

$$
\begin{aligned}
A_{k+1}^{k+1} & \geqslant a_{2} a_{3} \cdots a_{k} A_{k+1}\left(a_{1}+a_{k+1}-A_{k+1}\right) \\
& \geqslant a_{2} a_{3} \cdots a_{k}\left(a_{1} a_{k+1}\right)=G_{k+1}^{k+1}
\end{aligned}
$$

从而, 有 $A_{k+1} \geqslant G_{k+1}$.

直接验证可知, 当且仅当所有的 $a_{i}$ 相等时等号成立, 故命题成立.

说明 这里,利用了证明与正整数有关的命题的常用方法,即数学归纳法.数学归纳法证题技巧的应用, 可以说是五彩缤纷, 千姿百态. 应用数学归纳法,除了需要验证当 $n=1$ 或 $n=n_{0}$ (这里 $n_{0}$ 为某个固定的正整数) 外, 其关键是要在 $n=k$ 时成立的假设之下, 导出当 $n=k+1$ 时命题也成立,要完成这一步,需要一定的技巧和处理问题的能力, 只有通过多做练习来实现理解和掌握.

证法二(归纳法,与证法一的不同处理)

(1) 当 $n=2$ 时,已知结论成立.

(2)假设对 $n=k$ (正整数 $k \geqslant 2$ ) 时命题成立, 即对于 $a_{i}>0, i=1$, $2, \cdots, k$, 有

$$
\left(a_{1} a_{2} \cdots a_{k}\right)^{\frac{1}{k}} \leqslant \frac{a_{1}+a_{2}+\cdots+a_{k}}{k}
$$

那么,当 $n=k+1$ 时, 由归纳假设得


\begin{align*}
& a_{1}+a_{2}+\cdots+a_{k}+a_{k+1}  \tag{1}\\
& =a_{1}+a_{2}+\cdots+a_{k}+(a_{k+1}+\overbrace{G_{k+1}+\cdots+G_{k+1}}^{(k-1) \uparrow G_{k+1}})-(k-1) G_{k+1}  \tag{2}\\
& \geqslant k \sqrt[k]{a_{1} a_{2} \cdots a_{k}}+k \sqrt[k]{a_{k+1} G_{k+1}^{k-1}}-(k-1) G_{k+1}  \tag{3}\\
& \geqslant 2 k \sqrt{\sqrt[k]{a_{1} a_{2} \cdots a_{k}} \sqrt[k]{a_{k+1} G_{k+1}^{k-1}}}-(k-1) G_{k+1}  \tag{4}\\
& =2 k \sqrt[2 k]{a_{1} a_{2} \cdots a_{k+1} G_{k+1}^{k-1}}-(k-1) G_{k+1}  \tag{5}\\
& =2 k \sqrt[2 k]{G_{k+1}^{k+1} G_{k+1}^{k-1}}-(k-1) G_{k+1}  \tag{6}\\
& =(k+1) G_{k+1} \tag{7}
\end{align*}


于是 $A_{k+1} \geqslant G_{k+1}$.

不难看出, 当且仅当所有的 $a_{i}$ 相等时等号成立, 故命题成立.

说明 在这个证明中, 为了利用归纳假设, 将(1)写成(2)的形式. 由归纳假设, 从(2)得到(3),由于当 $n=2$ 时,不等式成立,则由 (3)得到了 (4).\\
证法三(归纳法,另一种处理方式)

(1) 当 $n=2$ 时,已知结论成立.

(2)假设对 $n=k$ (正整数 $k \geqslant 2$ ) 时命题成立, 即对于 $a_{i}>0, i=1$, $2, \cdots, k$, 有

$$
\left(a_{1} a_{2} \cdots a_{k}\right)^{\frac{1}{k}} \leqslant \frac{a_{1}+a_{2}+\cdots+a_{k}}{k}
$$

那么, 当 $n=k+1$ 时, 由归纳假设得

$$
\begin{aligned}
A_{k+1} & =\frac{1}{2 k}\left[(k+1) A_{k+1}+(k-1) A_{k+1}\right] \\
& =\frac{1}{2 k}(a_{1}+a_{2}+\cdots+a_{k+1}+\underbrace{A_{k+1}+A_{k+1}+\cdots+A_{k+1}}_{\text {共k-1 }}) \\
& \geqslant \frac{1}{2 k}\left(k \sqrt[k]{a_{1} a_{2} \cdots a_{k}}+k \sqrt[k]{a_{k+1} A_{k+1}^{k-1}}\right) \\
& \geqslant \sqrt[2 k]{a_{1} a_{2} \cdots a_{k} a_{k+1} A_{k+1}^{k-1}}
\end{aligned}
$$

所以 $A_{k+1}^{2 k} \geqslant a_{1} a_{2} \cdots a_{k+1} A_{k+1}^{k-1}$, 故得 $A_{k+1} \geqslant G_{k+1}$.

说明 (1) 在上面的证明中, 将 $A_{k+1}$ 表示为 $A_{k+1}=\frac{1}{2 k}\left[(k+1) A_{k+1}+\right.$ $\left.(k-1) A_{k+1}\right]$ 是一步较为关键和重要的变形技巧.

(2)我们也可以从 $G_{n+1}$ 出发进行处理,由归纳假设,得到

$$
\begin{aligned}
G_{n+1}= & {\left[\left(G_{n+1}\right)^{\frac{n+1}{n}}\left(G_{n+1}\right)^{\frac{n-1}{n}}\right]^{\frac{1}{2}} } \\
= & {\left[\left(a_{1} \cdots a_{n} a_{n+1}\right)^{\frac{1}{n}}\left(G_{n+1}\right)^{\frac{n-1}{n}}\right]^{\frac{1}{2}} } \\
= & G_{n}^{\frac{1}{2}}\left(a_{n+1}^{\frac{1}{n}+1} G_{n+1}^{\frac{n-1}{n}}\right)^{\frac{1}{2}} \leqslant \frac{1}{2}\left(G_{n}+a_{n^{\frac{1}{n}+1}}^{{ }^{\frac{n}{2}} G_{n+1}^{n}}\right) \\
\leqslant & \frac{1}{2}\left(G_{n}+\frac{a_{n+1}+(n-1) G_{n}}{n}\right) \\
\leqslant & \frac{1}{2}\left(A_{n}+\frac{a_{n+1}+(n-1) G_{n}}{n}\right) \\
= & \frac{n A_{n}+a_{n+1}}{2 n}+\frac{(n-1) G_{n+1}}{2 n} \\
= & \frac{(n+1) A_{n+1}}{2 n}+\frac{(n-1) G_{n+1}}{2 n} \\
& \frac{(n+1) A_{n+1}}{2 n} \geqslant \frac{(n+1) G_{n+1}}{2 n}
\end{aligned}
$$

即

故 $A_{n+1} \geqslant G_{n+1}$.

\section*{证法四(归纳法和变换)}
在证明原命题之前,首先令

$$
y_{1}=\frac{a_{1}}{G_{n}}, y_{2}=\frac{a_{2}}{G_{n}}, \cdots, y_{n}=\frac{a_{n}}{G_{n}}
$$

其中 $G_{n}=\sqrt[n]{a_{1} a_{2} \cdots a_{n}}$, 则 $y_{1} y_{2} \cdots y_{n}=1\left(y_{i}>0\right)$, 且平均值不等式等价于

$$
y_{1}+y_{2}+\cdots+y_{n} \geqslant n
$$

即在条件 $y_{1} y_{2} \cdots y_{n}=1\left(y_{i}>0\right)$ 之下, 证明 $y_{1}+y_{2}+\cdots+y_{n} \geqslant n$.

我们用归纳法证明上述不等式.

(1) 当 $n=1$ 时, $y_{1}=1 \geqslant 1$, 显然成立.

(2)假设当 $n=k$ 时不等式成立,则对于 $n=k+1$, 由于 $y_{1} y_{2} \cdots y_{n}=1$ $\left(y_{i}>0\right)$, 那么 $y_{i}$ 中必有大于或等于 1 者, 也有小于或等于 1 者, 不妨设 $y_{k} \geqslant$ $1, y_{k+1} \leqslant 1$, 并令 $y=y_{k} y_{k+1}$, 则 $y_{1} y_{2} \cdots y_{k-1} y=1$, 从而由归纳假设,得

$$
y_{1}+y_{2}+\cdots+y_{k-1}+y \geqslant k
$$

于是

$$
\begin{aligned}
& y_{1}+y_{2}+\cdots+y_{k-1}+y_{k}+y_{k+1} \\
\geqslant & k+y_{k}+y_{k+1}-y_{k} y_{k+1} \\
= & k+1+\left(y_{k}-1\right)\left(1-y_{k+1}\right) \\
\geqslant & k+1 .
\end{aligned}
$$

不难看出, 当且仅当 $y_{1}=y_{2}=\cdots=y_{n}=1$, 从而 $a_{1}=a_{2}=\cdots=a_{n}$ 时,等号成立.

故当 $n=k+1$ 时, 命题也成立.

说明 通过变量替换,将原问题化为一个与正整数有关的形式简单的不等式, 在证明中运用了我们比较熟悉的手段和技巧.

证法五 (归纳法和二项展开式)

(1) 当 $n=2$ 时,已知结论成立.

(2)假设对 $n=k$ (正整数 $k \geqslant 2$ ) 时命题成立, 即对于 $a_{i}>0, i=1$, $2, \cdots, k$, 有

$$
\left(a_{1} a_{2} \cdots a_{k}\right)^{\frac{1}{k}} \leqslant \frac{a_{1}+a_{2}+\cdots+a_{k}}{k}
$$

那么, 当 $n=k+1$ 时, 不妨假设 $a_{k+1}=\max \left\{a_{1}, a_{2}, \cdots, a_{k+1}\right\}$, 于是由归纳假设, 得

$$
a_{k+1} \geqslant \frac{a_{1}+a_{2}+\cdots+a_{k}}{k}=A_{k} \geqslant G_{k}=\sqrt[k]{a_{1} a_{2} \cdots a_{k}}
$$

从而, 得


\begin{align*}
A_{k+1}^{k+1} & =\left(\frac{a_{1}+a_{2}+\cdots+a_{k}+a_{k+1}}{k+1}\right)^{k+1} \\
& =\left(\frac{k A_{k}+a_{k+1}}{k+1}\right)^{k+1}=\left(A_{k}+\frac{a_{k+1}-A_{k}}{k+1}\right)^{k+1}  \tag{8}\\
& =A_{k}^{k+1}+(k+1) A_{k}^{k}\left(\frac{a_{k+1}-A_{k}}{k+1}\right)+\cdots+\left(\frac{a_{k+1}-A_{k}}{k+1}\right)^{k+1}  \tag{9}\\
& \geqslant A_{k}^{k+1}+(k+1) A_{k}^{k}\left(\frac{a_{k+1}-A_{k}}{k+1}\right)=A_{k}^{k+1}+A_{k}^{k}\left(a_{k+1}-A_{k}\right)  \tag{1}\\
& =A_{k}^{k} a_{k+1} \geqslant G_{k}^{k} a_{k+1}=a_{1} a_{2} \cdots a_{k} a_{k+1}  \tag{1}\\
& =G_{k+1}^{k+1} \tag{12}
\end{align*}


所以 $A_{k+1} \geqslant G_{k+1}$.

不难看出, 当且仅当所有的 $a_{i}$ 相等时等号成立, 故命题成立.

说明 在证明过程中, 考虑 $A_{k+1}^{k+1}$, 并通过一定的处理和运算, 导出所需要的结果. 有时候可能利用到其他的有用结论.

\section*{证法六 (归纳法和函数)}
(1)当 $n=2$ 时,易知结论成立.

(2)假设 $n=k$ (正整数 $k \geqslant 2$ ) 时命题成立,即 $A_{k} \geqslant G_{k}$. 那么, 当 $n=k+$ 1 时, 作函数 $f(x)=\left(\frac{a_{1}+\cdots+a_{n}+x}{n+1}\right)^{n+1}-a_{1} \cdots a_{n} x, x \in \mathbf{R}$, 并令

$$
f^{\prime}(x)=\left(\frac{a_{1}+\cdots+a_{n}+x}{n+1}\right)^{n}-a_{1} \cdots a_{n}=0
$$

解之得 $x_{0}=-\left(a_{1}+\cdots+a_{n}\right)+(n+1) \sqrt[n]{a_{1} \cdots a_{n}}$

$$
=-n A_{n}+(n+1) G_{n}
$$

不难验证, $x_{0}$ 为 $f(x)$ 的唯一极小值点, 且为最小值点, 以及

$$
f\left(x_{0}\right)=n\left(G_{n}\right)^{n}\left(A_{n}-G_{n}\right)
$$

由归纳假设, 得 $f(x) \geqslant f\left(x_{0}\right) \geqslant 0$, 即

$$
\left(\frac{a_{1}+\cdots+a_{n}+x}{n+1}\right)^{n+1} \geqslant a_{1} \cdots a_{n} x
$$

令 $x=a_{n+1} \geqslant 0$, 则 $A_{n+1}^{n+1} \geqslant G_{n+1}^{n+1}$.\\
故 $A_{n+1} \geqslant G_{n+1}$.

证法七(归纳法与 Jacobsthai 不等式)

为了证明平均值不等式, 需要证明一个引理.

引理 1 假设 $x 、 y$ 为正实数, $n$ 为正整数,则

$$
x^{n+1}+n y^{n+1} \geqslant(n+1) y^{n} x
$$

引理的证明: 由于 $x 、 y$ 与 $x^{k} 、 y^{k}(1 \leqslant k \leqslant n)$ 同序, 所以

$$
(x-y)\left(x^{k}-y^{k}\right) \geqslant 0
$$

于是

$$
\begin{aligned}
& x^{n+1}+n y^{n+1}-(n+1) x y^{n} \\
= & x\left(x^{n}-y^{n}\right)-n y^{n}(x-y) \\
= & (x-y)\left[x\left(x^{n-1}+x^{n-2} y+\cdots+y^{n-1}\right)-n y^{n}\right] \\
= & (x-y)\left[\left(x^{n}-y^{n}\right)+\left(x^{n-1}-y^{n-1}\right) y+\cdots+(x-y) y^{n-1}\right] \\
\geqslant & 0
\end{aligned}
$$

故引理 1 成立. 现在, 我们利用引理 1 和数学归纳法证明平均值不等式.

(1)当 $n=2$ 时,已知结论成立.

(2)假设对 $n=k$ (正整数 $k \geqslant 2$ ) 时命题成立. 那么,当 $n=k+1$ 时,令 $a_{1} a_{2} \cdots a_{k}=y^{k(k+1)}, a_{k+1}=x^{k+1}, x, y \geqslant 0$, 则由归纳假设和引理 1 , 得

$$
\begin{aligned}
& a_{1}+a_{2}+\cdots+a_{k}+a_{k+1}-(k+1) G_{k+1} \\
\geqslant & k \sqrt[k]{a_{1} a_{2} \cdots a_{k}}+a_{k+1}-(k+1) G_{k+1} \\
= & k y^{k+1}+x^{k+1}-(k+1) y^{k} x \geqslant 0 .
\end{aligned}
$$

不难看出,当且仅当所有的 $a_{i}$ 相等时等式成立,故命题成立.

说明 (1) 引理 1 中的不等式称为 Jacobsthai 不等式.

(2) 在 Jacobsthai 不等式中, 取 $y=1$, 得到

伯努利不等式 $x^{n} \geqslant 1+n(x-1), x>0, n \geqslant 1$.

关于伯努利(Bernoulli)不等式和平均值不等式, 我们有如下的结论.

\section*{定理 伯努利不等式与平均值不等式等价.}
事实上, 如果假设伯努利不等式成立, 则对 $\frac{A_{n}}{A_{n-1}}>0$, 有

$$
\left(\frac{A_{n}}{A_{n-1}}\right)^{n} \geqslant 1+n\left(\frac{A_{n}}{A_{n-1}}-1\right)=\frac{n A_{n}-(n-1) A_{n-1}}{A_{n-1}}=\frac{a_{n}}{A_{n-1}}, n \geqslant 2
$$

于是

$$
A_{n}^{n} \geqslant a_{n} A_{n-1}^{n-1}, \quad n \geqslant 2
$$

从而, $A_{n}^{n} \geqslant a_{n} A_{n-1}^{n-1} \geqslant a_{n} a_{n-1} A_{n-2}^{n-2} \geqslant \cdots \geqslant a_{n} a_{n-1} \cdots a_{2} a_{1}=G_{n}^{n}$.故 $A_{n} \geqslant G_{n}$.\\
反之, 如果平均值不等式成立, 则当 $n=1$ 时, 伯努利不等式成立.

当 $n \geqslant 2$ 时, 若 $0<x \leqslant 1-\frac{1}{n}$, 则伯努利不等式成立.

若 $x>1-\frac{1}{n}$, 则 $1+n(x-1)>0$, 由平均值不等式, 得

$$
\begin{aligned}
x^{n} & =(\frac{(1+n(x-1))+\overbrace{1+\cdots+1}^{n-1 \uparrow 1}}{n})^{n} \\
& \geqslant(1+n(x-1)) \cdot 1 \cdot \cdots \cdot 1=1+n(x-1)
\end{aligned}
$$

从而, 伯努利不等式成立.

注 伯努利不等式的一般形式:

设 $x>-1$, 则实数 $r \leqslant 0$ 或 $r \geqslant 1$ 时, $(1+x)^{r} \geqslant 1+r x$;

若 $0 \leqslant r \leqslant 1$, 则 $(1+x)^{r} \leqslant 1+r x$;

设 $x_{i} \geqslant-1,1 \leqslant i \leqslant n$, 且 $x_{i}$ 与 $x_{j}$ 同号 $, 1 \leqslant i, j \leqslant n$, 则

$$
\left(1+x_{1}\right)\left(1+x_{2}\right) \cdots\left(1+x_{n}\right) \geqslant 1+x_{1}+\cdots+x_{n}
$$

证法八(数列与 Jacobsthai 不等式)

令 $f(n)=n\left(\frac{a_{1}+\cdots+a_{n}}{n}-\sqrt[n]{a_{1} a_{2} \cdots a_{n}}\right)$, 如果能证明 $f(n)$ 关于 $n$ 单调不减, 即 $f(n) \leqslant f(n+1), n \geqslant 2$. 那么, 由 $f(2) \geqslant 0$, 得到 $f(n) \geqslant f(2) \geqslant$ 0 , 则平均值不等式成立.

下面利用 Jacobsthai 不等式证明 $f(n)$ 的单调性.

令 $a_{1} a_{2} \cdots a_{n}=y^{n(n+1)}, a_{n+1}=x^{n+1}, x, y \geqslant 0$, 则由引理 1 , 得

$$
\begin{aligned}
& f(n+1)-f(n) \\
= & (n+1)\left(\frac{a_{1}+\cdots+a_{n+1}}{n+1}-\sqrt[n+1]{a_{1} \cdots a_{n+1}}\right) \\
& -n\left(\frac{a_{1}+\cdots+a_{n}}{n}-\sqrt[n]{a_{1} \cdots a_{n}}\right) \\
= & a_{n+1}-(n+1) \sqrt[n+1]{a_{1} \cdots a_{n+1}}+n \sqrt[n]{a_{1} \cdots a_{n}} \\
= & x^{n+1}-(n+1) y^{n} x+n y^{n+1} \geqslant 0 .
\end{aligned}
$$

这表明 $f(n+1) \geqslant f(n)$.

另外, 由于 $f(2) \geqslant 0$, 则对任意 $n \geqslant 2$, 得

$$
f(n) \geqslant f(n-1) \geqslant \cdots \geqslant f(2) \geqslant 0
$$

不难看出, 当且仅当所有的 $a_{i}$ 相等时等号成立, 故平均值不等式成立.

\section*{证法九(倒向归纳法)}
倒向归纳法, 也称 “留空回填” 法. 基本思想是先对自然数的一个子列 $\left\{n_{m}\right\}$ 证明命题成立, 然后再回过来证明 $\{n\} \backslash\left\{n_{m}\right\}$ 相应的命题成立.

首先证明当 $n=2^{m}$ ( $m$ 为正整数) 时, 平均值不等式成立. 为此, 对 $m$ 用数学归纳法.

当 $m=1$ 时, 显然有 $\sqrt{a_{1} a_{2}} \leqslant \frac{a_{1}+a_{2}}{2}$.

假设 $m=k$ 时命题成立, 则当 $m=k+1$ 时,

$$
\begin{aligned}
& \sqrt[2^{k+1}]{a_{1} a_{2} \cdots a_{2^{k}} a_{2^{k}+1} \cdots a_{2^{k+1}}}
\end{aligned}
$$

%\begin{center}
%\includegraphics[max width=\textwidth]{2024_05_22_4ff05a14ba9ad07b725fg-010}
%\end{center}

$$
\begin{aligned}
& \leqslant \frac{1}{2}\left(\sqrt[2^{k}]{a_{1} a_{2} \cdots a_{2^{k}}}+\sqrt[2^{k}]{a_{2^{k}+1} \cdots a_{2^{k+1}}}\right) \\
& \leqslant \frac{1}{2}\left(\frac{a_{1}+a_{2}+\cdots+a_{2^{k}}}{2^{k}}+\frac{a_{2^{k}+1}+\cdots+a_{2^{k+1}}}{2^{k}}\right) \\
& =\frac{a_{1}+a_{2}+\cdots+a_{2^{k}}+a_{2^{k}+1}+\cdots+a_{2^{k+1}}}{2^{k+1}}
\end{aligned}
$$

所以对于具有 $n=2^{m}$ 形式的正整数 $n$, 平均值不等式成立, 即对无穷多个正整数 $2,4,8, \cdots, 2^{m}, \cdots$, 平均值不等式成立.

现假设 $n=k+1$ 时, 平均值不等式成立.

当 $n=k$ 时, $A_{k}=\frac{a_{1}+a_{2}+\cdots+a_{k}}{k}$, 则由假设, 得

$$
\sqrt[k+1]{a_{1} a_{2} \cdots a_{k} A_{k}} \leqslant \frac{a_{1}+a_{2}+\cdots+a_{k}+A_{k}}{k+1}=\frac{k A_{k}+A_{k}}{k+1}=A_{k},
$$

所以 $G_{k} \leqslant A_{k}$, 也就是说当 $n=k$ 时命题也成立.

综上可知, 对一切正整数 $n$, 平均值不等式成立. 不难看出, 当且仅当所有的 $a_{i}$ 相等时等号成立,故命题成立.

注 由上述证明知, 对任意整数 $n \geqslant 1$, 有

$$
\frac{a_{1}+a_{2}+\cdots+a_{2^{n}}}{2^{n}} \geqslant \sqrt[2^{n}]{a_{1} a_{2} \cdots a_{2^{n}}}
$$

如果取 $a_{1}, a_{2}, \cdots, a_{n}, a_{n+1}=a_{n+2}=\cdots=a_{2^{n}}=A_{n}$, 则

$$
\begin{aligned}
A_{n} & =\frac{n A_{n}+\left(2^{n}-n\right) A_{n}}{2^{n}} \geqslant \sqrt[2^{n}]{a_{1} a_{2} \cdots a_{n} A_{n}^{2^{n}-n}} \\
& =\left(a_{1} a_{2} \cdots a_{n}\right)^{\frac{1}{2^{n}}} \cdot A_{n}^{1-\frac{n}{2^{n}}}
\end{aligned}
$$

从而 $A_{n} \geqslant G_{n}$. 故平均值不等式成立.

证法十(利用排序不等式)

为了利用与上面不同的方法证明平均值不等式, 我们首先介绍和证明另一个重要的结论, 即排序不等式.

引理 2 (排序不等式) $\quad$ 设两个实数组 $a_{1}, a_{2}, \cdots, a_{n}$ 和 $b_{1}, b_{2}, \cdots, b_{n}$, 满足

$$
a_{1} \leqslant a_{2} \leqslant \cdots \leqslant a_{n} ; b_{1} \leqslant b_{2} \leqslant \cdots \leqslant b_{n}
$$

则

$$
a_{1} b_{1}+a_{2} b_{2}+\cdots+a_{n} b_{n}(\text { 同序乘积之和 })
$$

$$
\begin{aligned}
& \geqslant a_{1} b_{j_{1}}+a_{2} b_{j_{2}}+\cdots+a_{n} b_{j_{n}} \text { (乱序乘积之和) } \\
& \geqslant a_{1} b_{n}+a_{2} b_{n-1}+\cdots+a_{n} b_{1} \text { (反序乘积之和) }
\end{aligned}
$$

其中 $j_{1}, j_{2}, \cdots, j_{n}$ 是 $1,2, \cdots, n$ 的一个排列, 并且等号同时成立的充分必要条件是 $a_{1}=a_{2}=\cdots=a_{n}$ 或 $b_{1}=b_{2}=\cdots=b_{n}$ 成立.

引理的证明: 令 $A=a_{1} b_{j_{1}}+a_{2} b_{j_{2}}+\cdots+a_{n} b_{j_{n}}$. 如果 $j_{n} \neq n$, 且假设此时 $b_{n}$ 所在的项是 $a_{j_{m}} b_{n}$, 则由 $\left(b_{n}-b_{j_{n}}\right)\left(a_{n}-a_{j_{m}}\right) \geqslant 0$, 得

$$
a_{n} b_{n}+a_{j_{m}} b_{j_{n}} \geqslant a_{j_{m}} b_{n}+a_{n} b_{j_{n}}
$$

也就是说, $j_{n} \neq n$ 时, 调换 $A$ 中 $b_{n}$ 与 $b_{j_{n}}$ 的位置, 其余都不动, 则得到 $a_{n} b_{n}$ 项,并使 $A$ 变为 $A_{1}$, 且 $A_{1} \geqslant A$. 用同样的方法, 可以再得到 $a_{n-1} b_{n-1}$ 项, 并使 $A_{1}$变为 $A_{2}$, 且 $A_{2} \geqslant A_{1}$.

继续这个过程, 至多经过 $n-1$ 次调换, 得 $a_{1} b_{1}+a_{2} b_{2}+\cdots+a_{n} b_{n}$, 故

$$
a_{1} b_{1}+a_{2} b_{2}+\cdots+a_{n} b_{n} \geqslant A
$$

同样可以证明 $A \geqslant a_{1} b_{n}+a_{2} b_{n-1}+\cdots+a_{n} b_{1}$.

显然当 $a_{1}=a_{2}=\cdots=a_{n}$ 或 $b_{1}=b_{2}=\cdots=b_{n}$ 时,两个等号同时成立.反之, 如果 $\left\{a_{1}, a_{2}, \cdots, a_{n}\right\}$ 及 $\left\{b_{1}, b_{2}, \cdots, b_{n}\right\}$ 中的数都不全相同时, 则必有 $a_{1} \neq a_{n}, b_{1} \neq b_{n}$. 于是 $a_{1} b_{1}+a_{n} b_{n}>a_{1} b_{n}+a_{n} b_{1}$, 且

$$
a_{2} b_{2}+\cdots+a_{n-1} b_{n-1} \geqslant a_{2} b_{n-1}+\cdots+a_{n-1} b_{2}
$$

从而有

$$
a_{1} b_{n}+a_{2} b_{2}+\cdots+a_{n} b_{n}>a_{1} b_{n}+a_{2} b_{n-1}+\cdots+a_{n} b_{1}
$$

故这两个等式中至少有一个不成立.

现在,利用引理 2 证明平均值不等式.

令 $y_{k}=\frac{a_{1} a_{2} \cdots a_{k}}{G_{n}^{k}}, k=1,2, \cdots, n$. 由排序不等式, 得

$$
\begin{aligned}
& y_{1} \times \frac{1}{y_{1}}+y_{2} \times \frac{1}{y_{2}}+\cdots+y_{n} \times \frac{1}{y_{n}} \\
\leqslant & y_{1} \times \frac{1}{y_{n}}+y_{2} \times \frac{1}{y_{1}}+\cdots+y_{n} \times \frac{1}{y_{n-1}} \\
= & \frac{a_{1}+a_{2}+\cdots+a_{n}}{G_{n}},
\end{aligned}
$$

所以 $A_{n} \geqslant G_{n}$.

显然当 $a_{1}=a_{2}=\cdots=a_{n}$ 时, $A_{n}=G_{n}$. 如果 $a_{1}, a_{2}, \cdots, a_{n}$ 不全相等,不妨设 $a_{1} \neq a_{2}$, 令 $b=\frac{a_{1}+a_{2}}{2}$, 则 $a_{1} a_{2}<b^{2}$, 且 $b+b=a_{1}+a_{2}$,

$$
G_{n}<\sqrt[n]{b \cdot b \cdot a_{3} \cdots a_{n}} \leqslant \frac{b+b+a_{3}+\cdots+a_{n}}{n}=A_{n}
$$

故当 $A_{n}=G_{n}$ 时必有 $a_{1}=a_{2}=\cdots=a_{n}$. 反之亦然.

注 (1) 我们可以类似于证法四, 由 $G_{n}=\sqrt[n]{a_{1} a_{2} \cdots a_{n}}$, 令

$$
y_{1}=\frac{a_{1}}{G_{n}}, y_{2}=\frac{a_{2}}{G_{n}}, \cdots, y_{n}=\frac{a_{n}}{G_{n}}
$$

则 $y_{1} y_{2} \cdots y_{n}=1\left(y_{i}>0\right)$, 且平均值不等式等价于

$$
y_{1}+y_{2}+\cdots+y_{n} \geqslant n
$$

下面利用排序不等式证明这个不等式.

任取 $x_{1}>0$, 再取 $x_{2}>0$, 使得 $y_{1}=\frac{x_{1}}{x_{2}}$, 再取 $x_{3}>0$, 使得 $y_{2}=\frac{x_{2}}{x_{3}}, \cdots$,最后取 $x_{n}>0$, 使得 $y_{n-1}=\frac{x_{n-1}}{x_{n}}$. 所以

$$
y_{n}=\frac{1}{y_{1} y_{2} \cdots y_{n-1}}=\frac{1}{\frac{x_{1}}{x_{2}} \frac{x_{2}}{x_{3}} \cdots \frac{x_{n-1}}{x_{n}}}=\frac{x_{n}}{x_{1}}
$$

由引理 2 , 得

$$
y_{1}+y_{2}+\cdots+y_{n}=\frac{x_{1}}{x_{2}}+\frac{x_{2}}{x_{3}}+\cdots+\frac{x_{n-1}}{x_{n}}+\frac{x_{n}}{x_{1}} \geqslant n
$$

当且仅当 $x_{1}=x_{2}=\cdots=x_{n}$ 时等号成立, 从而当且仅当 $y_{1}=y_{2}=\cdots=y_{n}$时等号成立,所以当且仅当 $a_{1}=a_{2}=\cdots=a_{n}$ 时等号成立.

(2)排序不等式是一个重要的基本的不等式, 可以利用排序不等式直接证明许多其他有关的不等式. 例如:\\
切比雪夫(Chebyshev)不等式 设 $a_{1}, a_{2}, \cdots, a_{n}, b_{1}, b_{2}, \cdots, b_{n}$ 满足 $a_{1}$ $\leqslant a_{2} \leqslant \cdots \leqslant a_{n}, b_{1} \leqslant b_{2} \leqslant \cdots \leqslant b_{n}$, 则

$$
n \sum_{k=1}^{n} a_{k} b_{n-k+1} \leqslant \sum_{k=1}^{n} a_{k} \sum_{k=1}^{n} b_{k} \leqslant n \sum_{k=1}^{n} a_{k} b_{k}
$$

当且仅当 $a_{1}=a_{2}=\cdots=a_{n}$ 或 $b_{1}=b_{2}=\cdots=b_{n}$ 时等号成立.

证明 显然

$$
\begin{aligned}
& n \sum_{k=1}^{n} a_{k} b_{k}-\sum_{k=1}^{n} a_{k} \sum_{k=1}^{n} b_{k} \\
= & \sum_{k=1}^{n} \sum_{j=1}^{n}\left(a_{k} b_{k}-a_{k} b_{j}\right)=\sum_{j=1}^{n} \sum_{k=1}^{n}\left(a_{j} b_{j}-a_{j} b_{k}\right) \\
= & \frac{1}{2} \sum_{k=1}^{n} \sum_{j=1}^{n}\left(a_{k} b_{k}+a_{j} b_{j}-a_{k} b_{j}-a_{j} b_{k}\right) \\
= & \frac{1}{2} \sum_{k=1}^{n} \sum_{j=1}^{n}\left(a_{k}-a_{j}\right)\left(b_{k}-b_{j}\right) \geqslant 0
\end{aligned}
$$

故命题成立.

\section*{证法十一(调整法)}
(1)首先,如果 $a_{1}=a_{2}=\cdots=a_{n}$, 那么必有 $A_{n}=G_{n}$. 下设这些数不全等,不妨设 $a_{1}=\min \left\{a_{1}, a_{2}, \cdots, a_{n}\right\}, a_{2}=\max \left\{a_{1}, a_{2}, \cdots, a_{n}\right\}$, 则 $a_{1}<$ $A_{n}<a_{2}, a_{1}<G_{n}<a_{2}$. 令 $b_{1}=A_{n}, b_{2}=a_{1}+a_{2}-A_{n}, b_{i}=a_{i}, i \geqslant 3$. 并记 $A_{n}^{1}=\frac{b_{1}+b_{2}+\cdots+b_{n}}{n}=\frac{a_{1}+a_{2}+\cdots+a_{n}}{n}$, 则 $A_{n}^{1}=A_{n}$, 且由于

$$
\begin{aligned}
b_{1} b_{2}-a_{1} a_{2} & =A_{n}\left(a_{1}+a_{2}-A_{n}\right)-a_{1} a_{2} \\
& =\left(A_{n}-a_{1}\right)\left(a_{2}-A_{n}\right)>0
\end{aligned}
$$

则 $G_{n} \leqslant G_{n}^{1}=\sqrt[n]{b_{1} b_{2} \cdots b_{n}}$.

(2) 如果 $b_{1}=b_{2}=\cdots=b_{n}$, 则命题成立. 若不全等,则必有最大和最小者, 而且它们都不等于 $A_{n}$, 仿照上面作法, 可以得到 $c_{1}, c_{2}, \cdots, c_{n}$, 这组数中,有两个数为 $A_{n}$, 且 $A_{n}^{2}=\frac{c_{1}+c_{2}+\cdots+c_{n}}{n}=\frac{b_{1}+b_{2}+\cdots+b_{n}}{n}=A_{n}^{1}=A_{n}$, $G_{n}^{2}=\sqrt[n]{c_{1} c_{2} \cdots c_{n}} \geqslant G_{n}^{1} \geqslant G_{n}$. 如果 $c_{1}=c_{2}=\cdots=c_{n}$, 那么 $A_{n}^{2}=G_{n}^{2}$, 从而 $A_{n}=A_{n}^{2} \geqslant G_{n}$. 如果 $c_{1}, c_{2}, \cdots, c_{n}$ 仍然不全相等, 再按上述方法, 进行第三次变换, 所得到的新的数组中必有 3 个数都为 $A_{n}$. 这样下去, 一定存在某个数 $m(2 \leqslant m \leqslant n)$ 使得

$$
A_{n}=A_{n}^{1}=\cdots=A_{n}^{m}, G_{n} \leqslant G_{n}^{1} \leqslant G_{n}^{2} \leqslant \cdots \leqslant G_{n}^{m}, A_{n}^{m}=G_{n}^{m},
$$

从而得 $A_{n} \geqslant G_{n}$, 且只要 $a_{1}, a_{2}, \cdots, a_{n}$ 不全相等, 必有 $A_{n}>G_{n}$. 故命题成立.

注 调整法是证明不等式或求最值的一种有效方法, 特别是对那些当变量相等时取等号或取到最值的有关问题.

\section*{证法十二 (利用辅助命题)}
为了证明平均值不等式, 首先证明另一个不等式, 即

引理 3 如果 $x_{k} \geqslant 0$, 且 $x_{k} \geqslant x_{k-1}(k=2,3, \cdots, n)$, 则

$$
x_{n}^{n} \geqslant x_{1}\left(2 x_{2}-x_{1}\right)\left(3 x_{3}-2 x_{2}\right) \cdots\left[n x_{n}-(n-1) x_{n-1}\right]
$$

当且仅当 $x_{1}=x_{2}=\cdots=x_{n}$ 时等号成立.

引理的证明: 因为 $x_{k} \geqslant x_{k-1}$, 则

$$
x_{k}^{k-1}+x_{k}^{k-2} x_{k-1}+\cdots+x_{k-1}^{k-1} \geqslant k x_{k-1}^{k-1},
$$

所以

$$
\begin{aligned}
x_{k}^{k}-x_{k-1}^{k} & =\left(x_{k}-x_{k-1}\right)\left(x_{k}^{k-1}+x_{k}^{k-2} x_{k-1}+\cdots+x_{k-1}^{k-1}\right) \\
& \geqslant k x_{k-1}^{k-1}\left(x_{k}-x_{k-1}\right)
\end{aligned}
$$

即

$$
x_{k}^{k} \geqslant x_{k-1}^{k-1}\left[k x_{k}-(k-1) x_{k-1}\right](k=1,2, \cdots, n)
$$

当且仅当 $x_{k}=x_{k-1}$ 时等号成立.

所以

$$
x_{n}^{n}=x_{1} \frac{x_{2}^{2}}{x_{1}} \frac{x_{3}^{3}}{x_{2}^{2}} \cdots \frac{x_{n}^{n}}{x_{n-1}^{n-1}} \geqslant x_{1}\left(2 x_{2}-x_{1}\right)\left(3 x_{3}-2 x_{2}\right) \cdots\left[n x_{n}-(n-1) x_{n-1}\right]
$$

现在利用引理 3 证明平均值不等式.

不妨假设 $a_{n} \geqslant a_{n-1} \geqslant \cdots \geqslant a_{2} \geqslant a_{1}>0$. 由 $A_{k}=\frac{a_{1}+a_{2}+\cdots+a_{k}}{k}$, 则 $A_{k} \geqslant A_{k-1}>0(k=2,3, \cdots, n)$, 且 $k A_{k}-(k-1) A_{k-1}=a_{k}$. 由引理 3 , 得

$$
A_{n}^{n} \geqslant a_{1} a_{2} \cdots a_{n}
$$

即 $A_{n} \geqslant G_{n}$. 当且仅当 $A_{1}=A_{2}=\cdots=A_{n}$, 即 $a_{1}=a_{2}=\cdots=a_{n}$ 时等号成立.

\section*{证法十三 (函数方法)}
引理 4 如果函数 $f(x):(a, b) \rightarrow \mathbf{R}$ 满足


\begin{equation*}
f\left(\frac{x+y}{2}\right) \geqslant \frac{f(x)+f(y)}{2}, x, y \in(a, b) \tag{13}
\end{equation*}


那么


\begin{equation*}
f\left(\frac{x_{1}+x_{2}+\cdots+x_{n}}{n}\right) \geqslant \frac{f\left(x_{1}\right)+f\left(x_{2}\right)+\cdots+f\left(x_{n}\right)}{n} \tag{14}
\end{equation*}


其中 $x_{i} \in(a, b)$.

引理的证明: 对 $n$ 用归纳法.

当 $n=1,2$ 时,结论显然成立.

设当 $n=k$ 时结论成立. 对于 $n=k+1$, 有

并记

$$
\begin{gathered}
A_{k+1}=\frac{a_{1}+a_{2}+\cdots+a_{k}}{2 k}+\frac{a_{k+1}+(k-1) A_{k+1}}{2 k} \\
B=\frac{a_{k+1}+(k-1) A_{k+1}}{k}
\end{gathered}
$$

则

$$
\begin{aligned}
f\left(A_{k+1}\right)= & f\left(\frac{A_{k}+B}{2}\right) \\
\geqslant & \frac{1}{2}\left[f\left(A_{k}\right)+f(B)\right] \\
\geqslant & \frac{1}{2}\left\{\frac{1}{k}\left[f\left(a_{1}\right)+f\left(a_{2}\right)+\cdots+f\left(a_{k}\right)\right]\right. \\
& \left.+\frac{1}{k}\left[f\left(a_{k+1}\right)+(k-1) f\left(A_{k+1}\right)\right]\right\}
\end{aligned}
$$

所以

$$
f\left(\frac{a_{1}+a_{2}+\cdots+a_{k+1}}{k+1}\right) \geqslant \frac{f\left(a_{1}\right)+f\left(a_{2}\right)+\cdots+f\left(a_{k+1}\right)}{k+1}
$$

我们称满足(13)式的函数为凹函数 (可以证明, 如果函数 $f$ 二阶可导, 则当 $f^{\prime \prime}(x) \leqslant 0$ 时, $f$ 为凹函数). 特别的, 不难验证函数 $f(x)=\ln x$ 在 $(0$, $+\infty)$ 上是凹函数, 于是, 对 $a_{i} \in(0,+\infty), i=1,2, \cdots, n$, 我们有

$$
f\left(\frac{a_{1}+a_{2}+\cdots+a_{n}}{n}\right) \geqslant \frac{f\left(a_{1}\right)+f\left(a_{2}\right)+\cdots+f\left(a_{n}\right)}{n}
$$

从而

$$
\ln \frac{a_{1}+a_{2}+\cdots+a_{n}}{n} \geqslant \ln \left(a_{1} a_{2} \cdots a_{n}\right)^{\frac{1}{n}}
$$

由对数函数的单调性, 得

$$
\frac{a_{1}+a_{2}+\cdots+a_{n}}{n} \geqslant\left(a_{1} a_{2} \cdots a_{n}\right)^{\frac{1}{n}}
$$

故命题成立.\\
下面验证 $\ln x$ 为凹函数. 对任意 $x, y$, 要使得:

$$
f\left(\frac{x+y}{2}\right) \geqslant \frac{f(x)+f(y)}{2}
$$

即

$$
\ln \frac{x+y}{2} \geqslant \frac{\ln (x)+\ln (y)}{2}
$$

等价于

$$
\ln \frac{x+y}{2} \geqslant \ln (x y)^{\frac{1}{2}}
$$

由函数的单调性, 等价于

$$
\frac{x+y}{2} \geqslant(x y)^{\frac{1}{2}}
$$

这个可以由 $(\sqrt{x}-\sqrt{y})^{2}>0$ 直接导出.

另外, 设 $p>0, q>0$, 且 $\frac{1}{p}+\frac{1}{q}=1$, 由于函数 $f(x)=\ln x, x \in \mathbf{R}_{+}$为凹函数, 则对 $x, y>0$, 有

即

$$
\begin{gathered}
\frac{1}{p} \ln x+\frac{1}{q} \ln y \leqslant \ln \left(\frac{1}{p} x+\frac{1}{q} y\right) \\
x^{\frac{1}{p}} y^{\frac{1}{q}} \leqslant \frac{1}{p} x+\frac{1}{q} y
\end{gathered}
$$

等号成立的充分必要条件是 $x=y$.

这个不等式称为 Young 不等式.

注 引理 4 中的不等式(44, 称为琴生 (Jensen)不等式, 它的一般形式为设 $y=f(x), x \in(a, b)$ 为凹函数, 则对任意 $x_{i} \in(a, b)(i=1,2, \cdots$, $n)$, 我们有加权的琴生不等式

$$
\frac{1}{p_{1}} f\left(x_{1}\right)+\frac{1}{p_{2}} f\left(x_{2}\right)+\cdots+\frac{1}{p_{n}} f\left(x_{n}\right) \leqslant f\left(\frac{x_{1}}{p_{1}}+\frac{x_{2}}{p_{2}}+\cdots+\frac{x_{n}}{p_{n}}\right),
$$

其中 $p_{i}>0(i=1,2, \cdots, n)$ 且 $\sum_{i=1}^{n} \frac{1}{p_{i}}=1$.

证法十四(平均值不等式与函数不等式)

利用函数 $f(x)=\mathrm{e}^{x}-1-x$ 的性质,不难得到\\
引理 $5 \mathrm{e}^{x} \geqslant 1+x, x \in \mathbf{R}$, 当且仅当 $x=0$ 时, 等号成立.

设 $a_{1}, a_{2}, \cdots, a_{n} \geqslant 0$. 令 $a_{k}=\left(1+x_{k}\right) A_{n}$, 其中 $\sum_{i=1}^{n} x_{i}=0$.

由引理 $5, \mathrm{e}^{x_{k}} \geqslant 1+x_{k}$, 于是

$$
\begin{aligned}
G_{n} & =\sqrt[n]{a_{1} a_{2} \cdots a_{n}}=\left(\prod_{k=1}^{n}\left(1+x_{k}\right) A_{n}\right)^{\frac{1}{n}} \\
& =A_{n}\left(\prod_{k=1}^{n}\left(1+x_{k}\right)\right)^{\frac{1}{n}} \leqslant A_{n}\left(\prod_{k=1}^{n} \mathrm{e}^{x_{k}}\right)^{\frac{1}{n}} \\
& =A_{n} \mathrm{e}^{\frac{1}{n} \sum_{k=1}^{n} x_{k}}=A_{n}
\end{aligned}
$$

从而 $A_{n} \geqslant G_{n}$, 故平均值不等式成立.

证法十五(几何方法)

作函数 $y=\mathrm{e}^{\frac{x}{G_{n}}}$ 的图象, 并过点 $\left(G_{n}, \mathrm{e}\right)$ 作该曲线的切线 $y=\frac{\mathrm{e}}{G_{n}} x$. 易知 $\mathrm{e}^{\frac{x}{G_{n}}} \geqslant \frac{\mathrm{e}}{G_{n}} x, x \geqslant 0$

当且仅当 $x=G_{n}$ 时, 等号成立.

对 $a_{i} \geqslant 0,1 \leqslant i \leqslant n$, 有 $\mathrm{e}^{\frac{a_{i}}{G_{n}}} \geqslant \frac{\mathrm{e}}{G_{n}} a_{i}, 1 \leqslant i \leqslant n$, 将这 $n$ 个不等式相乘,得 $\mathrm{e}^{\frac{a_{1}+\cdots+a_{n}}{G_{n}}} \geqslant \frac{\mathrm{e}^{n}}{G_{n}^{n}} a_{1} a_{2} \cdots a_{n}$, 即 $\mathrm{e}^{\frac{a_{1}+\cdots+a_{n}}{G_{n}}} \geqslant \mathrm{e}^{n}$.

由函数 $y=\mathrm{e}^{x}$ 的单调性, 得 $\frac{a_{1}+\cdots+a_{n}}{G_{n}} \geqslant n$.

从而 $A_{n} \geqslant G_{n}$, 且当且仅当 $a_{1}=\cdots=a_{n}$ 时等号成立. 故平均值不等式成立.

在这部分, 我们利用不同的方法证明了平均值不等式成立. 在证明过程中,利用了各种技巧和方法.

\section*{习 题 1}
$\square$ 设 $a, b, c>0, a b c=1$. 求证:

$$
\frac{1}{a}+\frac{1}{b}+\frac{1}{c} \geqslant \sqrt{a}+\sqrt{b}+\sqrt{c}
$$

22 设 $a, b, c \geqslant 0, a+b+c>0$, 求证: $\frac{(a+b)^{3}(b+c)^{2}(c+a)}{(a+b+c)^{6}} \leqslant \frac{4}{27}$, 并指\\
出等号成立的条件.

3 已知 $0<a, b, c<1$, 并且 $a b+b c+c a=1$. 证明:

$$
\frac{a}{1-a^{2}}+\frac{b}{1-b^{2}}+\frac{c}{1-c^{2}} \geqslant \frac{3 \sqrt{3}}{2}
$$

4设 $a, b, c, d \in \mathbf{R}_{+}$满足 $a b c d=1, a+b+c+d>\frac{a}{b}+\frac{b}{c}+\frac{c}{d}+\frac{d}{a}$.

求证: $a+b+c+d<\frac{b}{a}+\frac{c}{b}+\frac{d}{c}+\frac{a}{d}$.

5 设 $a_{i}>0,1 \leqslant i \leqslant n$. 求证: $\sum_{i=1}^{n} \frac{a_{i}^{3}+a_{i+1}^{3}}{a_{i}^{2}+a_{i} a_{i+1}+a_{i+1}^{2}} \geqslant \frac{2}{3} \sum_{i=1}^{n} a_{i}$, 其中, $a_{n+1}=$ $a_{1}$.

6 设 $a_{1}, a_{2}, \cdots, a_{n}>0$ 且 $a_{1}+a_{2}+\cdots+a_{n}=1$. 求证:

$$
\left(\frac{1}{a_{1}^{2}}-1\right)\left(\frac{1}{a_{2}^{2}}-1\right) \cdots\left(\frac{1}{a_{n}^{2}}-1\right) \geqslant\left(n^{2}-1\right)^{n}
$$

7 设 $a, b, c, d>0$, 满足 $a^{2}+b^{2}+c^{2}+d^{2}=1$. 求证:

$$
a+b+c+d+\frac{1}{a b c d} \geqslant 18
$$

8 设 $x_{i} \geqslant 0,1 \leqslant i \leqslant n, n \geqslant 3$ 满足 $x_{1}+\cdots+x_{n}=1$, 求证:

$$
\frac{3}{4} \leqslant \frac{x_{1}^{2}+x_{2}+\cdots+x_{n-1}+x_{n}}{x_{1}+x_{2}+\cdots+x_{n-1}+x_{n}^{2}} \leqslant \frac{4}{3}
$$

9 设 $x_{1}, x_{2}, \cdots, x_{n}>0$, 且 $x_{1} x_{2} \cdots x_{n}=1$. 求证:

$$
\frac{1}{x_{1}\left(1+x_{1}\right)}+\frac{1}{x_{2}\left(1+x_{2}\right)}+\cdots+\frac{1}{x_{n}\left(1+x_{n}\right)} \geqslant \frac{n}{2}
$$

10 设 $a, b, c>0$ 且 $a^{2}+b^{2}+c^{2}+(a+b+c)^{2} \leqslant 4$. 求证:

$$
\frac{a b+1}{(a+b)^{2}}+\frac{b c+1}{(b+c)^{2}}+\frac{c a+1}{(c+a)^{2}} \geqslant 3
$$

11 已知 $a 、 b 、 c$ 为正实数, 且 $a b c=8$, 求证:

$$
\frac{a^{2}}{\sqrt{\left(1+a^{3}\right)\left(1+b^{3}\right)}}+\frac{b^{2}}{\sqrt{\left(1+b^{3}\right)\left(1+c^{3}\right)}}+\frac{c^{2}}{\sqrt{\left(1+c^{3}\right)\left(1+a^{3}\right)}} \geqslant \frac{4}{3}
$$

12 设 $a, b \in \mathbf{R}, \frac{1}{a}+\frac{1}{b}=1$. 求证: 对一切正整数 $n$, 有

$$
(a+b)^{n}-a^{n}-b^{n} \geqslant 2^{2 n}-2^{n+1}
$$

13 设 $a, b \in \mathbf{R}_{+}$, 求证: $\sqrt{a}+1>\sqrt{b}$ 成立的充要条件是对任意 $x>1$, 有 $a x+$ $\frac{x}{x-1}>b$.

14 设 $x_{1}, x_{2} \in \mathbf{R}$, 且 $x_{1}^{2}+x_{2}^{2} \leqslant 1$. 求证:对任意 $y_{1}, y_{2} \in \mathbf{R}$, 有

$$
\left(x_{1} y_{1}+x_{2} y_{2}-1\right)^{2} \geqslant\left(x_{1}^{2}+x_{2}^{2}-1\right)\left(y_{1}^{2}+y_{2}^{2}-1\right)
$$

15 设 $a 、 b 、 c$ 为正实数, 求证:

$$
\left(1+\frac{a}{b}\right)\left(1+\frac{b}{c}\right)\left(1+\frac{c}{a}\right) \geqslant 2\left(1+\frac{a+b+c}{\sqrt[3]{a b c}}\right)
$$

16 设 $x_{1}, x_{2}, x_{3} \in \mathbf{R}_{+}$, 证明:

$$
\frac{x_{2}}{x_{1}}+\frac{x_{3}}{x_{2}}+\frac{x_{1}}{x_{3}} \leqslant\left(\frac{x_{1}}{x_{2}}\right)^{2}+\left(\frac{x_{2}}{x_{3}}\right)^{2}+\left(\frac{x_{3}}{x_{1}}\right)^{2}
$$

17 设 $a 、 b 、 c$ 为正实数,且 $a+b+c=1$. 求证:

$$
(1+a)(1+b)(1+c) \geqslant 8(1-a)(1-b)(1-c)
$$

18 设 $x 、 y 、 z$ 为正实数,且 $x \geqslant y \geqslant z$. 求证:

$$
\frac{x^{2} y}{z}+\frac{y^{2} z}{x}+\frac{z^{2} x}{y} \geqslant x^{2}+y^{2}+z^{2}
$$

19 设 $a 、 b 、 c$ 为正实数,满足 $a^{2}+b^{2}+c^{2}=1$. 求证:

$$
\frac{a b}{c}+\frac{b c}{a}+\frac{c a}{b} \geqslant \sqrt{3}
$$

20 设 $a 、 b 、 c 、 d$ 是非负实数,满足 $a b+b c+c d+d a=1$. 求证:

$$
\frac{a^{3}}{b+c+d}+\frac{b^{3}}{a+c+d}+\frac{c^{3}}{a+d+b}+\frac{d^{3}}{a+b+c} \geqslant \frac{1}{3}
$$

21 设 $n$ 为给定的自然数, $n \geqslant 3$, 对于 $n$ 个给定的实数 $a_{1}, a_{2}, \cdots, a_{n}$, 记 $\left|a_{i}-a_{j}\right|(1 \leqslant i<j \leqslant n)$ 的最小值为 $m$, 求在 $a_{1}^{2}+\cdots+a_{n}^{2}=1$ 时, $m$ 的最大值.

22 设 $x, y \in \mathbf{R}_{+}, x+y^{2016}>1$, 求证:

$$
x^{2016}+y>1-\frac{1}{100}
$$

23 设 $x_{i}>0,1 \leqslant i \leqslant n, x_{1} x_{2} \cdots x_{n}=1$. 求证:

$$
\prod_{i=1}^{n}\left(\sqrt{2}+x_{i}\right) \geqslant(1+\sqrt{2})^{n}
$$

24 设 $a, b, c, d>0$. 求证:

$$
a^{4}+b^{4}+c^{4}+d^{4} \geqslant 4 a b c d+4(a-b)^{2} \sqrt{a b c d}
$$

25 设 $a, b, c, d>0$ 满足 $a+b+c+d=3$. 求证:

$$
\frac{1}{a^{3}}+\frac{1}{b^{3}}+\frac{1}{c^{3}}+\frac{1}{d^{3}} \leqslant \frac{1}{(a b c d)^{3}}
$$

26 设 $x_{i} \in \mathbf{R}$, 满足 $\sum_{i=1}^{n} x_{i}^{2}=1, h \geqslant 2$. 求证:

$$
\sum_{k=1}^{n}\left(1-\frac{k}{\sum_{i=1}^{n} i x_{i}^{2}}\right)^{2} \frac{x_{k}^{2}}{k} \leqslant\left(\frac{n-1}{n+1}\right)^{2} \sum_{k=1}^{n} \frac{x_{k}^{2}}{k}
$$

并确定等式成立的条件.

\section*{平均值不等式的应用}
\section*{2.1 平均值不等式在不等式证明中的应用}
下面举例说明平均值不等式在证明各种竞赛问题中的应用. 在证明过程中,应用灵活,具有较高的技巧性.

\begin{example}
	设 $f(x)=\frac{a}{a^{2}-1}\left(a^{x}-a^{-x}\right)(a>0, a \neq 1)$, 证明: 对正整数 $n \geqslant 2$,有
	
	$$
	f(n)>n
	$$
\end{example}
\begin{proof}
	当 $n \geqslant 2$ 时, 由平均值不等式, 得
	
	$$
	\begin{aligned}
	f(n) & =\frac{a}{a^{2}-1}\left(a^{n}-a^{n}\right)=\frac{a}{a^{2}-1}\left(a^{n}-\frac{1}{a^{n}}\right) \\
	& =\frac{a}{a^{2}-1}\left(a-\frac{1}{a}\right)\left(a^{n-1}+a^{n-2} \frac{1}{a}+a^{n-3} \frac{1}{a^{2}}+\cdots+a \frac{1}{a^{n-2}}+\frac{1}{a^{n-1}}\right) \\
	& \geqslant \frac{a}{a^{2}-1}\left(a-\frac{1}{a}\right) n \sqrt[n]{a^{n-1} a^{n-2} \cdots a^{2} a \frac{1}{a} \frac{1}{a^{2}} \cdots \frac{1}{a^{n-1}}}=n
	\end{aligned}
	$$
	
	当且仅当 $a=1$ 时等号成立,故命题成立.
\end{proof}
\begin{note}
	此题也可以用归纳法证明.
	
	当 $n=1$ 时, 则 $a_{1}=1$, 显然成立. 假定当 $n=k$ 时成立, 那么, 对于 $n=$ $k+1$, 由于 $a_{1} a_{2} \cdots a_{k} a_{k+1}=1$, 如果有某个 $a_{i}=1$, 则由归纳假设, 命题成立. 如果 $a_{i}$ 都不为 1 , 则必有大于 1 的, 且必有小于 1 的, 不妨设 $a_{k}>1, a_{k+1}<1$. 则由归纳假设, 得
	
	$$
	\left(2+a_{1}\right)\left(2+a_{2}\right) \cdots\left(2+a_{k-1}\right)\left(2+a_{k} a_{k+1}\right) \geqslant 3^{k}
	$$
	
	于是, 为了证明命题, 只要证明
	
	$$
	\left(2+a_{k}\right)\left(2+a_{k+1}\right) \geqslant 3\left(2+a_{k} a_{k+1}\right)
	$$
	
	便可.
	
	因为
	
	$$
	\left(2+a_{k}\right)\left(2+a_{k+1}\right) \geqslant 3\left(2+a_{k} a_{k+1}\right)
	$$
	
	等价于
	
	$$
	4+2 a_{k}+2 a_{k+1}+a_{k} a_{k+1} \geqslant 6+3 a_{k} a_{k+1}
	$$
	
	等价于
	
	$$
	a_{k}+a_{k+1}-a_{k} a_{k+1}-1 \geqslant 0
	$$
	
	等价于
	
	$$
	\left(a_{k}-1\right)\left(1-a_{k+1}\right) \geqslant 0
	$$
	
	由假设最后不等式成立,故命题成立.
	
	注 这里, 选取 $a_{k}>1, a_{k+1}<1$, 在平均值不等式的证明方法四中有过类似的考虑.
\end{note}

\begin{example}
	设 $x>0$, 证明: $2^{12 \sqrt{x}}+2^{\sqrt[4]{x}} \geqslant 2 \cdot 2^{6^{x}}$.
\end{example}
\begin{proof}
	由该不等式的外形, 很容易想到平均值不等式. 由平均值不等式, 得
	
	$$
	2^{12 \sqrt{x}}+2^{4 \sqrt[4]{x}} \geqslant 2 \cdot \sqrt{2^{12 \sqrt{x}} 2^{\sqrt[4]{x}}}=2 \cdot 2^{\frac{12 \sqrt{x}+\sqrt{x}}{2}}
	$$
	
	又
	
	$$
	\frac{\sqrt[12]{x}+\sqrt[4]{x}}{2} \geqslant\left(x^{\frac{1}{12}} x^{\frac{1}{4}}\right)^{\frac{1}{2}}=x^{\frac{1}{6}}
	$$
	
	所以
	
	$$
	2^{12 \sqrt{x}}+2^{\sqrt[4]{x}} \geqslant 2 \cdot 2^{\sqrt[6]{x}}
	$$
\end{proof}
\begin{note}
	此题也可以用归纳法证明.
	
	当 $n=1$ 时, 则 $a_{1}=1$, 显然成立. 假定当 $n=k$ 时成立, 那么, 对于 $n=$ $k+1$, 由于 $a_{1} a_{2} \cdots a_{k} a_{k+1}=1$, 如果有某个 $a_{i}=1$, 则由归纳假设, 命题成立. 如果 $a_{i}$ 都不为 1 , 则必有大于 1 的, 且必有小于 1 的, 不妨设 $a_{k}>1, a_{k+1}<1$. 则由归纳假设, 得
	
	$$
	\left(2+a_{1}\right)\left(2+a_{2}\right) \cdots\left(2+a_{k-1}\right)\left(2+a_{k} a_{k+1}\right) \geqslant 3^{k}
	$$
	
	于是, 为了证明命题, 只要证明
	
	$$
	\left(2+a_{k}\right)\left(2+a_{k+1}\right) \geqslant 3\left(2+a_{k} a_{k+1}\right)
	$$
	
	便可.
	
	因为
	
	$$
	\left(2+a_{k}\right)\left(2+a_{k+1}\right) \geqslant 3\left(2+a_{k} a_{k+1}\right)
	$$
	
	等价于
	
	$$
	4+2 a_{k}+2 a_{k+1}+a_{k} a_{k+1} \geqslant 6+3 a_{k} a_{k+1}
	$$
	
	等价于
	
	$$
	a_{k}+a_{k+1}-a_{k} a_{k+1}-1 \geqslant 0
	$$
	
	等价于
	
	$$
	\left(a_{k}-1\right)\left(1-a_{k+1}\right) \geqslant 0
	$$
	
	由假设最后不等式成立,故命题成立.
	
	注 这里, 选取 $a_{k}>1, a_{k+1}<1$, 在平均值不等式的证明方法四中有过类似的考虑.
\end{note}

\begin{example}
	设 $a_{i}>0, i=1,2, \cdots, n$ 满足 $a_{1} a_{2} \cdots a_{n}=1$. 证明:
	
	$$
	\left(2+a_{1}\right)\left(2+a_{2}\right) \cdots\left(2+a_{n}\right) \geqslant 3^{n}
	$$
\end{example}
\begin{proof}
	由于对任意的 $i$,
	
	$$
	\begin{gathered}
	2+a_{i}=1+1+a_{i} \geqslant 3 \sqrt[3]{a_{i}} \\
	\left(2+a_{1}\right)\left(2+a_{2}\right) \cdots\left(2+a_{n}\right) \geqslant 3^{n} \sqrt[3]{a_{1} a_{2} \cdots a_{n}}=3^{n}
	\end{gathered}
	$$
	
	故
\end{proof}
\begin{note}
	此题也可以用归纳法证明.
	
	当 $n=1$ 时, 则 $a_{1}=1$, 显然成立. 假定当 $n=k$ 时成立, 那么, 对于 $n=$ $k+1$, 由于 $a_{1} a_{2} \cdots a_{k} a_{k+1}=1$, 如果有某个 $a_{i}=1$, 则由归纳假设, 命题成立. 如果 $a_{i}$ 都不为 1 , 则必有大于 1 的, 且必有小于 1 的, 不妨设 $a_{k}>1, a_{k+1}<1$. 则由归纳假设, 得
	
	$$
	\left(2+a_{1}\right)\left(2+a_{2}\right) \cdots\left(2+a_{k-1}\right)\left(2+a_{k} a_{k+1}\right) \geqslant 3^{k}
	$$
	
	于是, 为了证明命题, 只要证明
	
	$$
	\left(2+a_{k}\right)\left(2+a_{k+1}\right) \geqslant 3\left(2+a_{k} a_{k+1}\right)
	$$
	
	便可.
	
	因为
	
	$$
	\left(2+a_{k}\right)\left(2+a_{k+1}\right) \geqslant 3\left(2+a_{k} a_{k+1}\right)
	$$
	
	等价于
	
	$$
	4+2 a_{k}+2 a_{k+1}+a_{k} a_{k+1} \geqslant 6+3 a_{k} a_{k+1}
	$$
	
	等价于
	
	$$
	a_{k}+a_{k+1}-a_{k} a_{k+1}-1 \geqslant 0
	$$
	
	等价于
	
	$$
	\left(a_{k}-1\right)\left(1-a_{k+1}\right) \geqslant 0
	$$
	
	由假设最后不等式成立,故命题成立.
	
	注 这里, 选取 $a_{k}>1, a_{k+1}<1$, 在平均值不等式的证明方法四中有过类似的考虑.
\end{note}

\begin{example}
	设 $a>b>0$, 求证: $\sqrt{2} a^{3}+\frac{3}{a b-b^{2}} \geqslant 10$.
\end{example}
\begin{proof}
	因为 $a b-b^{2}=b(a-b) \leqslant \frac{[b+(a-b)]^{2}}{4}=\frac{a^{2}}{4}$, 所以
	
	$$
	\begin{aligned}
	& \sqrt{2} a^{3}+\frac{3}{a b-b^{2}} \geqslant \sqrt{2} a^{3}+\frac{12}{a^{2}} \\
	= & \frac{\sqrt{2}}{2} a^{3}+\frac{\sqrt{2}}{2} a^{3}+\frac{4}{a^{2}}+\frac{4}{a^{2}}+\frac{4}{a^{2}} \\
	\geqslant & 5 \sqrt[5]{\frac{\sqrt{2}}{2} a^{3} \cdot \frac{\sqrt{2}}{2} a^{3} \cdot \frac{4}{a^{2}} \cdot \frac{4}{a^{2}} \cdot \frac{4}{a^{2}}}=10,
	\end{aligned}
	$$
	
	即命题成立.
\end{proof}
\begin{note}
	为了消去 $a$, 将 $\sqrt{2} a^{3}$ 写成两项, $\frac{12}{a^{2}}$ 写成三项. 这样, 利用平均值不等式,它们的乘积为一个常数.
	
	例 $\mathbf{5}$ 设 $a, b, c \in \mathbf{R}_{+}$. 证明:
	
	$$
	\frac{1+a^{2}}{1+b}+\frac{1+b^{2}}{1+c}+\frac{1+c^{2}}{1+a} \geqslant 6(\sqrt{2}-1)
	$$
	
	解 由平均值不等式得
	
	
	\begin{equation*}
	\frac{1+a^{2}}{1+b}+\frac{1+b^{2}}{1+c}+\frac{1+c^{2}}{1+a} \geqslant 3 \cdot \sqrt[3]{\frac{1+a^{2}}{1+a} \cdot \frac{1+b^{2}}{1+b} \cdot \frac{1+c^{2}}{1+c}} \tag{1}
	\end{equation*}
	
	
	首先证明: 对任意 $x \in \mathbf{R}_{+}$, 有
	
	
	\begin{equation*}
	\frac{1+x^{2}}{1+x} \geqslant 2(\sqrt{2}-1) \tag{2}
	\end{equation*}
	
	
	事实上,
	
	$$
	\text { (2) } \begin{aligned}
	& \Leftrightarrow x^{2}-2(\sqrt{2}-1) x+1-(2 \sqrt{2}-2) \geqslant 0 \\
	& \Leftrightarrow x^{2}-2(\sqrt{2}-1) x+(\sqrt{2}-1)^{2} \geqslant 0 \\
	& \Leftrightarrow(x-(\sqrt{2}-1))^{2} \geqslant 0
	\end{aligned}
	$$
	
	即(2)成立.
	
	由(1)、(2), 得到
	
	$$
	\frac{1+a^{2}}{1+b}+\frac{1+b^{2}}{1+c}+\frac{1+c^{2}}{1+a} \geqslant 3 \cdot \sqrt[3]{8(\sqrt{2}-1)^{3}}=6(\sqrt{2}-1)
	$$
	
	从而命题成立.
	
	注 (1)不难验证, 当且仅当 $a=b=c=\sqrt{2}-1$ 时, 等式成立.
	
	(2) 由于 $\left(1+a^{2}, 1+b^{2}, 1+c^{2}\right)$ 与 $\left(\frac{1}{1+a}, \frac{1}{1+b}, \frac{1}{1+c}\right)$ 为反序三元组.所以,由排序不等式和不等式 (2), 得到
	
	$$
	\frac{1+a^{2}}{1+b}+\frac{1+b^{2}}{1+c}+\frac{1+c^{2}}{1+a} \geqslant \frac{1+a^{2}}{1+a}+\frac{1+b^{2}}{1+b}+\frac{1+c^{2}}{1+c} \geqslant 6(\sqrt{2}-1)
	$$
	
	即命题成立.
\end{note}

\begin{example}
	设 $n$ 为正整数, $x_{i} \in \mathbf{R}_{+}, 1 \leqslant i \leqslant n$, 满足 $x_{1} \cdots x_{n}=1$. 求证:
	
	$$
	\sum_{i=1}^{n} x_{i} \sqrt{x_{1}^{2}+\cdots+x_{i}^{2}} \geqslant \frac{n+1}{2} \sqrt{n}
	$$
\end{example}
\begin{proof}
	由平均值不等式, 得到
	
	$$
	\begin{aligned}
	& \sum_{i=1}^{n} x_{i} \sqrt{x_{1}^{2}+\cdots+x_{i}^{2}} \geqslant \sum_{i=1}^{n} x_{i} \frac{x_{1}+\cdots+x_{i}}{\sqrt{i}} \\
	= & \sum_{i=1}^{n} \sum_{j=1}^{i} \frac{x_{i} x_{j}}{\sqrt{i}} \geqslant \frac{1}{\sqrt{n}} \sum_{i=1}^{n} \sum_{j=1}^{i} x_{i} x_{j} \\
	= & \frac{1}{2 \sqrt{n}} \sum_{1 \leqslant j \leqslant i \leqslant n} 2 x_{i} x_{j} \\
	= & \frac{1}{2 \sqrt{n}}\left(\sum_{i=1}^{n} x_{i}^{2}+\left(\sum_{i=1}^{n} x_{i}\right)^{2}\right) \\
	\geqslant & \frac{1}{2 \sqrt{n}}\left[n \cdot \sqrt[n]{x_{1}^{2} \cdots x_{n}^{2}}+\left(n \cdot \sqrt[n]{x_{1} \cdots x_{n}}\right)^{2}\right] \\
	= & \frac{n+n^{2}}{2 \sqrt{n}}=\frac{n+1}{2} \sqrt{n} .
	\end{aligned}
	$$
	
	即命题成立.
\end{proof}
\begin{note}
	这里用到了双求和符号及有关性质.
\end{note}

\begin{example}
	设 $a_{1}, a_{2}, \cdots, a_{n} \in \mathbf{R}_{+}, S=a_{1}+a_{2}+\cdots+a_{n}$. 求证:
	
	$$
	\left(1+a_{1}\right)\left(1+a_{2}\right) \cdots\left(1+a_{n}\right) \leqslant 1+S+\frac{S^{2}}{2!}+\cdots+\frac{S^{n}}{n!}
	$$
\end{example}
\begin{proof}
	由于 $G_{n} \leqslant A_{n}$, 得
	
	$$
	\begin{aligned}
	& \left(1+a_{1}\right)\left(1+a_{2}\right) \cdots\left(1+a_{n}\right) \\
	\leqslant & \left(\frac{n+a_{1}+a_{2}+\cdots+a_{n}}{n}\right)^{n}=\left(1+\frac{S}{n}\right)^{n} \\
	= & 1+\mathrm{C}_{n}^{1}\left(\frac{S}{n}\right)+\mathrm{C}_{n}^{2}\left(\frac{S}{n}\right)^{2}+\cdots+\mathrm{C}_{n}^{m}\left(\frac{S}{n}\right)^{m}+\cdots+\mathrm{C}_{n}^{n}\left(\frac{S}{n}\right)^{n}
	\end{aligned}
	$$
	
	因为 $n!=(n-m)!(n-m+1) \cdots n \leqslant(n-m)!n^{m}$,
	
	所以
	
	$$
	\mathrm{C}_{n}^{m}\left(\frac{S}{n}\right)^{m}=\frac{n!}{m!(n-m)!} \cdot \frac{1}{n^{m}} S^{m} \leqslant \frac{S^{m}}{m!}
	$$
	
	从而命题成立.
\end{proof}
\begin{note}
	应用平均值不等式时, 通常要将乘幂看作连乘积, 有时还要巧妙地添上数 1.
\end{note}

\begin{example}
	设 $k 、 n$ 为正整数, 且 $1 \leqslant k \leqslant n, a_{i} \in \mathbf{R}_{+}$, 满足 $a_{1}+a_{2}+\cdots+$ $a_{k}=a_{1} a_{2} \cdots a_{k}$. 求证:
	
	$$
	a_{1}^{n-1}+a_{2}^{n-1}+\cdots+a_{k}^{n-1} \geqslant k n,
	$$
	
	并确定等号成立的充要条件.
\end{example}
\begin{proof}
	令 $a=a_{1}+a_{2}+\cdots+a_{k}=a_{1} a_{2} \cdots a_{k}$. 由平均值不等式, 得
	
	$$
	a \geqslant k a^{\frac{1}{k}} \text {, 即 } a \geqslant k^{\frac{k}{k-1}} .
	$$
	
	又因为
	
	$$
	a_{1}^{n-1}+a_{2}^{n-1}+\cdots+a_{k}^{n-1} \geqslant k\left(a_{1} a_{2} \cdots a_{k}\right)^{\frac{n-1}{k}}=k a^{\frac{n-1}{k}} \geqslant k \cdot k^{\frac{n-1}{k-1}},
	$$
	
	于是只需证明
	
	$$
	k^{\frac{n-1}{k-1}} \geqslant n
	$$
	
	再由平均值不等式,得
	
	$$
	k=\frac{(k-1) n+(n-k) \times 1}{n-1} \geqslant n^{\frac{k-1}{n-1}},
	$$
	
	从而不等式成立.
	
	不难看出, 当 $k=n$ 且 $a_{1}=a_{2}=\cdots=a_{k}$ 时等号成立.
\end{proof}
\begin{note}
	应用平均值不等式时, 通常要将乘幂看作连乘积, 有时还要巧妙地添上数 1.
\end{note}

\begin{example}
	设 $a_{i}>0(i=1,2, \cdots, n)$, 求证:
	
	$$
	\sum_{k=1}^{n} k a_{k} \leqslant \frac{n(n-1)}{2}+\sum_{k=1}^{n} a_{k}^{k} .
	$$
\end{example}
\begin{proof}
	因为 $\frac{n(n-1)}{2}=\sum_{k=1}^{n}(k-1)$, 所以由平均值不等式, 得
	
	$$
	\begin{aligned}
	\frac{n(n-1)}{2}+\sum_{k=1}^{n} a_{k}^{k} & =\sum_{k=1}^{n}\left[(k-1)+a_{k}^{k}\right] \\
	& =\sum_{k=1}^{n}\left(1+1+\cdots+1+a_{k}^{k}\right) \\
	& \geqslant \sum_{k=1}^{n} k \sqrt[k]{1^{k-1} \cdot a_{k}^{k}}=\sum_{k=1}^{n} k a_{k}
	\end{aligned}
	$$
	
	故命题成立.
\end{proof}
\begin{note}
	应用平均值不等式时, 通常要将乘幂看作连乘积, 有时还要巧妙地添上数 1.
\end{note}

\begin{example}
	设 $a_{i}>0, b_{i}>0$ 且满足 $a_{1}+a_{2}+\cdots+a_{n} \leqslant 1, b_{1}+b_{2}+\cdots+b_{n} \leqslant n$.求证:
	
	$$
	\left(\frac{1}{a_{1}}+\frac{1}{b_{1}}\right)\left(\frac{1}{a_{2}}+\frac{1}{b_{2}}\right) \cdots\left(\frac{1}{a_{n}}+\frac{1}{b_{n}}\right) \geqslant(n+1)^{n}
	$$
\end{example}
\begin{proof}
	由已知条件和平均值不等式, 得
	
	$$
	\begin{aligned}
	& a_{1} a_{2} \cdots a_{n} \leqslant\left(\frac{a_{1}+a_{2}+\cdots+a_{n}}{n}\right)^{n} \leqslant \frac{1}{n^{n}} \\
	& b_{1} b_{2} \cdots b_{n} \leqslant\left(\frac{b_{1}+b_{2}+\cdots+b_{n}}{n}\right)^{n} \leqslant 1 . \\
	& \quad \frac{1}{a_{i}}+\frac{1}{b_{i}}=\frac{1}{n a_{i}}+\cdots+\frac{1}{n a_{i}}+\frac{1}{b_{i}} \geqslant(n+1) \sqrt[n+1]{\left(\frac{1}{n a_{i}}\right)^{n}\left(\frac{1}{b_{i}}\right)}
	\end{aligned}
	$$
	
	从而
	
	$$
	\begin{aligned}
	& \left(\frac{1}{a_{1}}+\frac{1}{b_{1}}\right)\left(\frac{1}{a_{2}}+\frac{1}{b_{2}}\right) \cdots\left(\frac{1}{a_{n}}+\frac{1}{b_{n}}\right) \\
	\geqslant & (n+1)^{n^{n+1}} \sqrt{\frac{1}{\left(n^{n}\right)^{n}} \frac{1}{\left(a_{1} a_{2} \cdots a_{n}\right)^{n}} \frac{1}{b_{1} b_{2} \cdots b_{n}}} \\
	\geqslant & (n+1)^{n} .
	\end{aligned}
	$$
	
	故命题成立.
\end{proof}
\begin{note}
	此题证明的关键是将 $\frac{1}{a_{i}}$ 写成 $\frac{1}{n a_{i}}+\cdots+\frac{1}{n a_{i}}$.
\end{note}

\begin{example}
	假设 $a 、 b 、 c$ 都是正数,证明:
	
	$$
	a b c \geqslant(a+b-c)(b+c-a)(c+a-b)
	$$
\end{example}
\begin{proof}
	如果 $a+b-c, b+c-a, c+a-b$ 中有负数, 不妨设 $a+b-c<0$, 则 $c>a+b$. 故 $b+c-a$ 与 $c+a-b$ 均为正数, 则结论显然成立.
	
	若 $a+b-c, b+c-a, c+a-b$ 均非负, 则由平均值不等式, 得
	
	$$
	\sqrt{(a+b-c)(b+c-a)} \leqslant \frac{(a+b-c)+(b+c-a)}{2}=b
	$$
	
	同理可得
	
	$$
	\begin{aligned}
	& \sqrt{(b+c-a)(c+a-b)} \leqslant \frac{(b+c-a)+(c+a-b)}{2}=c \\
	& \sqrt{(c+a-b)(a+b-c)} \leqslant \frac{(c+a-b)+(a+b-c)}{2}=a
	\end{aligned}
	$$
	
	将三式相乘,即得到我们要证明的问题,故命题成立.
\end{proof}
\begin{note}
	通过对部分变量应用平均值不等式, 而且轮换使用, 从而得到结论的证明.
\end{note}

\begin{example}
	设整数 $n \geqslant 2, x_{i} \in \mathbf{R}_{+}, 1 \leqslant i \leqslant n$, 满足 $\sum_{i=1}^{n} x_{i}=1$.\\
	求证: $\left(\sum_{i=1}^{n} \frac{1}{1-x_{i}}\right)\left(\sum_{1 \leqslant i<j \leqslant n} x_{i} x_{j}\right) \leqslant \frac{n}{2}$.
\end{example}
\begin{proof}
	首先证局部不等式, 即对每个 $1 \leqslant k \leqslant n$, 有
	
	
	\begin{equation*}
	\left(2 \sum_{1 \leqslant i<j \leqslant n} x_{i} x_{j}\right) \frac{1}{1-x_{k}} \leqslant 2 x_{k}+\frac{n-2}{n-1} \sum_{i \neq k} x_{i} \tag{1}
	\end{equation*}
	
	
	事实上,由平均值不等式
	
	$$
	\sum_{i \neq k} x_{i}^{2} \geqslant \frac{2}{n-2} \sum_{i, j \neq k} x_{i} x_{j}
	$$
	
	从而,
	
	$$
	2 \sum_{\substack{1 \leqslant i<j \leqslant n \\ i, j \neq k}} x_{i} x_{j} \leqslant \frac{n-2}{n-1}\left(\sum_{i \neq k} x_{i}\right)^{2}
	$$
	
	于是
	
	$$
	\begin{aligned}
	& \left(2 \sum_{1 \leqslant i<j \leqslant n} x_{i} x_{j}\right) \frac{1}{1-x_{k}} \\
	= & \left(2 x_{k}\left(1-x_{k}\right)+2 \sum_{\substack{1 \leqslant i<j \leqslant n \\
	i, j \neq k}} x_{i} x_{j}\right) \frac{1}{1-x_{k}} \\
	= & 2 x_{k}+\frac{2 \sum_{i \neq j \neq k} x_{i} x_{j}}{\sum_{i \neq k} x_{i}} \\
	\leqslant & 2 x_{k}+\frac{n-2}{n-1} \sum_{i \neq k} x_{i} .
	\end{aligned}
	$$
	
	所以(1)成立.
	
	由不等式(1)关于 $1 \leqslant k \leqslant n$ 求和, 得到
	
	$$
	2 \sum_{1 \leqslant i<j \leqslant n} x_{i} x_{j} \sum_{k=1}^{n} \frac{1}{1-x_{k}} \leqslant 2 \sum_{k=1}^{n} x_{k}+\frac{n-2}{n-1} \sum_{k=1}^{n} \sum_{i \neq k} x_{i}=n
	$$
	
	即
	
	$$
	\sum_{1 \leqslant i<j \leqslant n} x_{i} x_{j} \cdot \sum_{i=1}^{n} \frac{1}{1-x_{i}} \leqslant \frac{n}{2}
	$$
	
	故命题成立.
	
	另外,我们也可以用切比雪夫不等式证明.
	
	由于
	
	$$
	2 \sum_{1 \leqslant i<j \leqslant n} x_{i} x_{j}=\sum_{i=1}^{n} x_{i} \sum_{1 \leqslant i<j \leqslant n} x_{j}=\sum_{i=1}^{n} x_{i}\left(1-x_{i}\right)
	$$
	
	所以,原不等式等价于
	
	
	\begin{equation*}
	\left(\sum_{i=1}^{n} \frac{1}{1-x_{i}}\right)\left(\sum_{i=1}^{n} x_{i}\left(1-x_{i}\right)\right) \leqslant n \tag{2}
	\end{equation*}
	
	
	不妨设 $0<x_{1} \leqslant x_{2} \leqslant \cdots \leqslant x_{n} \leqslant 1$.
	
	由于对任意 $1 \leqslant i<j \leqslant n$, 有 $x_{i}+x_{j} \leqslant 1,0<x_{i}<x_{j} \leqslant 1$.
	
	从而, $\left(x_{i}-x_{j}\right)\left(1-x_{i}-x_{j}\right) \leqslant 0$, 即
	
	$$
	x_{i}\left(1-x_{i}\right) \leqslant x_{j}\left(1-x_{j}\right)
	$$
	
	于是 $x_{1}\left(1-x_{1}\right) \leqslant x_{2}\left(1-x_{2}\right) \leqslant \cdots \leqslant x_{n}\left(1-x_{n}\right)$, 以及
	
	$$
	\frac{1}{1-x_{1}} \leqslant \frac{1}{1-x_{2}} \leqslant \cdots \leqslant \frac{1}{1-x_{n}}
	$$
	
	由切比雪夫不等式,得
	
	$$
	\frac{1}{n}\left(\sum_{i=1}^{n} \frac{1}{1-x_{i}}\right)\left(\sum_{i=1}^{n} x_{i}\left(1-x_{i}\right)\right) \leqslant\left(\sum_{i=1}^{n} \frac{1}{1-x_{i}} \cdot x_{i}\left(1-x_{i}\right)\right)=1
	$$
	
	所以(2)成立. 故命题成立.
\end{proof}
\begin{note}
	由于分子之和 $a^{2}+b^{2}+c^{2}=1$, 所以当各分母被控制在某个常数之内时,便可以推出命题成立. 这个方法在分式不等式证明中常常使用.
\end{note}

\begin{example}
	设 $n$ 为正整数,证明:
	
	$$
	n\left[(n+1)^{\frac{1}{n}}-1\right] \leqslant 1+\frac{1}{2}+\frac{1}{3}+\cdots+\frac{1}{n} \leqslant n-(n-1)\left(\frac{1}{n}\right)^{\frac{1}{n-1}}
	$$
\end{example}
\begin{proof}
	只证明不等式的左边,不等式的右边可同样处理.
	
	令 $A=\frac{1+\frac{1}{2}+\frac{1}{3}+\cdots+\frac{1}{n}+n}{n}$, 则左边的不等式等价于
	
	$$
	A \geqslant(n+1)^{\frac{1}{n}}
	$$
	
	由平均值不等式, 得
	
	$$
	\begin{aligned}
	A & =\frac{(1+1)+\left(1+\frac{1}{2}\right)+\cdots+\left(1+\frac{1}{n}\right)}{n} \\
	& =\frac{2+\frac{3}{2}+\frac{4}{3}+\cdots+\frac{n+1}{n}}{n} \\
	& \geqslant \sqrt[n]{2 \cdot \frac{3}{2} \cdot \frac{4}{3} \cdot \cdots \cdot \frac{n+1}{n}}=(n+1)^{\frac{1}{n}}
	\end{aligned}
	$$
	
	从而得
	
	$$
	1+\frac{1}{2}+\frac{1}{3}+\cdots+\frac{1}{n} \geqslant n\left[(n+1)^{\frac{1}{n}}-1\right] .
	$$
	
	不难看出, 当 $n=1$ 时等号成立.\\
\end{proof}
\begin{note}
	由于分子之和 $a^{2}+b^{2}+c^{2}=1$, 所以当各分母被控制在某个常数之内时,便可以推出命题成立. 这个方法在分式不等式证明中常常使用.
\end{note}

\begin{example}
	设 $a_{i}>0, i=1,2, \cdots, n, m>0$ 且满足 $\sum_{i=1}^{n} \frac{1}{1+a_{i}^{m}}=1$. 求证:
	
	$$
	a_{1} a_{2} \cdots a_{n} \geqslant(n-1)^{\frac{n}{m}}
	$$
	
	证法一 令 $x_{i}=\frac{1}{1+a_{i}^{m}}$, 则 $a_{i}^{m}=\frac{1-x_{i}}{x_{i}}$, 且 $x_{i}>0, i=1,2, \cdots, n$, $\sum_{i=1}^{n} x_{i}=1$. 于是
	
	$$
	\begin{aligned}
	a_{1}^{m} a_{2}^{m} \cdots a_{n}^{m} & =\frac{\left(x_{2}+\cdots+x_{n}\right) \cdots\left(x_{1}+x_{2}+\cdots+x_{n-1}\right)}{x_{1} x_{2} \cdots x_{n}} \\
	& \geqslant \frac{(n-1) \sqrt[n-1]{x_{2} x_{3} \cdots x_{n}} \cdots(n-1) \sqrt[n-1]{x_{1} x_{2} \cdots x_{n-1}}}{x_{1} x_{2} \cdots x_{n}} \\
	& =(n-1)^{n} .
	\end{aligned}
	$$
	
	故命题成立.
	
	证法二 令 $a_{i}^{m}=\tan ^{2} \alpha_{i}$, 则 $\sum_{i=1}^{n} \frac{1}{1+a_{i}^{m}}=1$ 等价于
	
	$$
	\sum_{i=1}^{n} \cos ^{2} \alpha_{i}=1
	$$
	
	其结论等价于
	
	$$
	\tan ^{2} \alpha_{1} \tan ^{2} \alpha_{2} \cdots \tan ^{2} \alpha_{n} \geqslant(n-1)^{n}
	$$
	
	即
	
	$$
	\sin ^{2} \alpha_{1} \sin ^{2} \alpha_{2} \cdots \sin ^{2} \alpha_{n} \geqslant(n-1)^{n} \cos ^{2} \alpha_{1} \cos ^{2} \alpha_{2} \cdots \cos ^{2} \alpha_{n}
	$$
	
	由平均值不等式, 得
	
	$$
	\begin{aligned}
	\sin ^{2} \alpha_{1} & =1-\cos ^{2} \alpha_{1}=\cos ^{2} \alpha_{2}+\cdots+\cos ^{2} \alpha_{n} \\
	& \geqslant(n-1) \sqrt[n-1]{\cos ^{2} \alpha_{2} \cos ^{2} \alpha_{3} \cdots \cos ^{2} \alpha_{n}}
	\end{aligned}
	$$
	
	一般地,
	
	$$
	\begin{aligned}
	\sin ^{2} \alpha_{i} & =1-\cos ^{2} \alpha_{i} \\
	& =\cos ^{2} \alpha_{1}+\cdots+\cos ^{2} \alpha_{i-1}+\cos ^{2} \alpha_{i+1}+\cdots+\cos ^{2} \alpha_{n} \\
	& \geqslant(n-1) \sqrt[n-1]{\cos ^{2} \alpha_{1} \cdots \cos ^{2} \alpha_{i-1} \cos ^{2} \alpha_{i+1} \cdots \cos ^{2} \alpha_{n}}
	\end{aligned}
	$$
	
	将它们相乘, 得
	
	$$
	\sin ^{2} \alpha_{1} \sin ^{2} \alpha_{2} \cdots \sin ^{2} \alpha_{n} \geqslant(n-1)^{n} \cos ^{2} \alpha_{1} \cos ^{2} \alpha_{2} \cdots \cos ^{2} \alpha_{n}
	$$
	
	故命题成立.\\
	例 15 设 $a, b, c \in \mathbf{R}_{+}$, 且 $a^{2}+b^{2}+c^{2}=1$. 求证:
	
	$$
	\frac{a}{1-a^{2}}+\frac{b}{1-b^{2}}+\frac{c}{1-c^{2}} \geqslant \frac{3 \sqrt{3}}{2}
	$$
\end{example}
\begin{proof}
	原不等式
	
	$$
	\frac{a}{1-a^{2}}+\frac{b}{1-b^{2}}+\frac{c}{1-c^{2}} \geqslant \frac{3 \sqrt{3}}{2}
	$$
	
	等价于
	
	$$
	\frac{a^{2}}{a\left(1-a^{2}\right)}+\frac{b^{2}}{b\left(1-b^{2}\right)}+\frac{c^{2}}{c\left(1-c^{2}\right)} \geqslant \frac{3 \sqrt{3}}{2}
	$$
	
	由于 $a^{2}+b^{2}+c^{2}=1$, 如果能证明 $x\left(1-x^{2}\right) \leqslant \frac{2}{3 \sqrt{3}}$, 则上述不等式成立.由平均值不等式, 得
	
	$$
	\begin{aligned}
	x\left(1-x^{2}\right) & =\sqrt{\frac{2 x^{2}\left(1-x^{2}\right)\left(1-x^{2}\right)}{2}} \\
	& \leqslant \sqrt{\frac{1}{2}\left[\frac{2 x^{2}+\left(1-x^{2}\right)+\left(1-x^{2}\right)}{3}\right]^{3}} \\
	& =\sqrt{\frac{1}{2} \cdot\left(\frac{2}{3}\right)^{3}}=\frac{2}{3 \sqrt{3}}
	\end{aligned}
	$$
	
	故不等式成立.
\end{proof}
\begin{note}
	由于分子之和 $a^{2}+b^{2}+c^{2}=1$, 所以当各分母被控制在某个常数之内时,便可以推出命题成立. 这个方法在分式不等式证明中常常使用.
\end{note}

\begin{example}
	设 $a_{1}, a_{2}, \cdots, a_{n}$ 是 $1,2, \cdots, n$ 的一个排列. 求证:
	
	$$
	\frac{1}{2}+\frac{2}{3}+\cdots+\frac{n-1}{n} \leqslant \frac{a_{1}}{a_{2}}+\frac{a_{2}}{a_{3}}+\cdots+\frac{a_{n-1}}{a_{n}}
	$$
\end{example}
\begin{proof}
	因为 $a_{1}, a_{2}, \cdots, a_{n}$ 是 $1,2, \cdots, n$ 的一个排列, 所以
	
	$$
	\begin{aligned}
	& \left(1+a_{1}\right)\left(1+a_{2}\right) \cdots\left(1+a_{n-1}\right) \\
	\geqslant & (1+1)(1+2) \cdots[1+(n-1)] \\
	= & a_{1} a_{2} \cdots a_{n}
	\end{aligned}
	$$
	
	于是
	
	$$
	\begin{aligned}
	& \frac{a_{1}}{a_{2}}+\frac{a_{2}}{a_{3}}+\cdots+\frac{a_{n-1}}{a_{n}}+\frac{1}{1}+\frac{1}{2}+\cdots+\frac{1}{n} \\
	= & \frac{a_{1}}{a_{2}}+\frac{a_{2}}{a_{3}}+\cdots+\frac{a_{n-1}}{a_{n}}+\frac{1}{a_{1}}+\frac{1}{a_{2}}+\cdots+\frac{1}{a_{n}}
	\end{aligned}
	$$
	
	$$
	\begin{aligned}
	& =\frac{1}{a_{1}}+\frac{1+a_{1}}{a_{2}}+\frac{1+a_{2}}{a_{3}}+\cdots+\frac{1+a_{n-1}}{a_{n}} \\
	& \geqslant n \sqrt[n]{\frac{\left(1+a_{1}\right)\left(1+a_{2}\right) \cdots\left(1+a_{n-1}\right)}{a_{1} a_{2} \cdots a_{n}}} \geqslant n
	\end{aligned}
	$$
	
	又因为 $n=\left(1+\frac{1}{2}+\frac{1}{3}+\cdots+\frac{1}{n}\right)+\left(\frac{1}{2}+\frac{2}{3}+\cdots+\frac{n-1}{n}\right)$, 所以
	
	$$
	\frac{a_{1}}{a_{2}}+\frac{a_{2}}{a_{3}}+\cdots+\frac{a_{n-1}}{a_{n}} \geqslant \frac{1}{2}+\frac{2}{3}+\cdots+\frac{n-1}{n}
	$$
\end{proof}
\begin{note}
	对于该不等式的证明,首先要充分理解 $a_{1}, a_{2}, \cdots, a_{n}$ 是 $1,2, \cdots, n$的一个排列, 此外, 两边同时相加 $1+\frac{1}{2}+\frac{1}{3}+\cdots+\frac{1}{n}\left(\right.$ 即 $\left.\frac{1}{a_{1}}+\frac{1}{a_{2}}+\cdots+\frac{1}{a_{n}}\right)$ 也是很重要的一步.
\end{note}

\begin{example}
	设 $a 、 b 、 c$ 为正实数, 求证:
	
	$$
	\frac{a}{\sqrt{a^{2}+8 b c}}+\frac{b}{\sqrt{b^{2}+8 a c}}+\frac{c}{\sqrt{c^{2}+8 a b}} \geqslant 1
	$$
\end{example}
\begin{proof}
	容易看出, 如果我们能证明 $\frac{a}{\sqrt{a^{2}+8 b c}} \geqslant \frac{a^{\frac{4}{3}}}{a^{\frac{4}{3}}+b^{\frac{4}{3}}+c^{\frac{4}{3}}}$, 那么,将它们相加便得到所要证明的不等式. 因为
	
	$$
	\frac{a}{\sqrt{a^{2}+8 b c}} \geqslant \frac{a^{\frac{4}{3}}}{a^{\frac{4}{3}}+b^{\frac{4}{3}}+c^{\frac{4}{3}}}
	$$
	
	等价于
	
	$$
	\left(a^{\frac{4}{3}}+b^{\frac{4}{3}}+c^{\frac{4}{3}}\right)^{2} \geqslant a^{\frac{2}{3}}\left(a^{2}+8 b c\right)
	$$
	
	再由平均值不等式, 得
	
	$$
	\begin{aligned}
	\left(a^{\frac{4}{3}}+b^{\frac{4}{3}}+c^{\frac{4}{3}}\right)^{2}-\left(a^{\frac{4}{3}}\right)^{2} & =\left(b^{\frac{4}{3}}+c^{\frac{4}{3}}\right)\left(a^{\frac{4}{3}}+a^{\frac{4}{3}}+b^{\frac{4}{3}}+c^{\frac{4}{3}}\right) \\
	& \geqslant 2 b^{\frac{2}{3}} c^{\frac{2}{3}} \cdot 4 a^{\frac{2}{3}} b^{\frac{1}{3}} c^{\frac{1}{3}}=8 a^{\frac{2}{3}} b c
	\end{aligned}
	$$
	
	于是
	
	$$
	\left(a^{\frac{4}{3}}+b^{\frac{4}{3}}+c^{\frac{4}{3}}\right)^{2} \geqslant\left(a^{\frac{4}{3}}\right)^{2}+8 a^{\frac{2}{3}} b c=a^{\frac{2}{3}}\left(a^{2}+8 b c\right)
	$$
	
	从而
	
	$$
	\frac{a}{\sqrt{a^{2}+8 b c}} \geqslant \frac{a^{\frac{4}{3}}}{a^{\frac{4}{3}}+b^{\frac{4}{3}}+c^{\frac{4}{3}}}
	$$
	
	同理可得
	
	$$
	\frac{b}{\sqrt{b^{2}+8 a c}} \geqslant \frac{b^{\frac{4}{3}}}{a^{\frac{4}{3}}+b^{\frac{4}{3}}+c^{\frac{4}{3}}}, \frac{c}{\sqrt{c^{2}+8 a b}} \geqslant \frac{c^{\frac{4}{3}}}{a^{\frac{4}{3}}+b^{\frac{4}{3}}+c^{\frac{4}{3}}}
	$$
	
	于是
	
	$$
	\frac{a}{\sqrt{a^{2}+8 b c}}+\frac{b}{\sqrt{b^{2}+8 a c}}+\frac{c}{\sqrt{c^{2}+8 a b}} \geqslant 1
	$$
\end{proof}
\begin{note}
	这是一道 IMO 试题, 有多种不同的证明方法, 后面将再次遇到. 这里的指数 $\frac{4}{3}$ 是这样得到的: 取 $x$ 为待定常数.
	
	设
	
	$$
	\frac{a}{\sqrt{a^{2}+8 b c}} \geqslant \frac{a^{x}}{a^{x}+b^{x}+c^{x}}
	$$
	
	上式等价于
	
	$$
	\begin{aligned}
	& a^{2}\left(a^{x}+b^{x}+c^{x}\right)^{2} \geqslant a^{2 x}\left(a^{2}+8 b c\right) \\
	\Leftrightarrow & \left(a^{x}+b^{x}+c^{x}\right)^{2} \geqslant a^{2 x-2}\left(a^{2}+8 b c\right) \\
	\Leftrightarrow & a^{2 x}+2 a^{x}\left(b^{x}+c^{x}\right)+\left(b^{x}+c^{x}\right)^{2} \geqslant a^{2 x}+8 a^{2 x-2} b c \\
	\Leftrightarrow & 2 a^{x}\left(b^{x}+c^{x}\right)+\left(b^{x}+c^{x}\right)^{2} \geqslant 8 a^{2 x-2} b c .
	\end{aligned}
	$$
	
	由于
	
	$$
	b^{x}+c^{x} \geqslant 2 b^{\frac{x}{2}} c^{\frac{x}{2}}
	$$
	
	只需
	
	$$
	2 a^{x} \cdot 2 b^{\frac{x}{2}} c^{\frac{x}{2}}+\left(2 b^{\frac{x}{2}} c^{\frac{x}{2}}\right)^{2} \geqslant 8 a^{2 x-2} b c
	$$
	
	即
	
	$$
	a^{x} b^{\frac{x}{2}} c^{\frac{x}{2}}+b^{x} c^{x} \geqslant 2 a^{2 x-2} b c
	$$
	
	由于
	
	$$
	a^{x} b^{\frac{x}{2}} c^{\frac{x}{2}}+b^{x} c^{x} \geqslant 2 \sqrt{a^{x} b^{\frac{3}{2} x} c^{\frac{3}{2} x}}=2 a^{\frac{x}{2}} b^{\frac{3}{4} x} c^{\frac{3}{4} x}
	$$
	
	所以只需
	
	$$
	a^{\frac{x}{2}} b^{\frac{3}{4} x} c^{\frac{3}{4} x} \geqslant a^{2 x-2} b c
	$$
	
	显然取 $x=\frac{4}{3}$ 满足要求.
\end{note}

\begin{example}
	已知正整数 $n \geqslant 2$, 实数 $a_{i} 、 b_{i}$, 满足
	
	$$
	a_{1} \geqslant a_{2} \geqslant \cdots \geqslant a_{n}>0, b_{1} \geqslant b_{2} \geqslant \cdots \geqslant b_{n}>0
	$$
	
	并且
	
	$$
	a_{1} a_{2} \cdots a_{n}=b_{1} b_{2} \cdots b_{n},
	$$
	
	$$
	\sum_{1 \leqslant i<j \leqslant n}\left(a_{i}-a_{j}\right) \leqslant \sum_{1 \leqslant i<j \leqslant n}\left(b_{i}-b_{j}\right)
	$$
	
	求证: $\sum_{i=1}^{n} a_{i} \leqslant(n-1) \sum_{i=1}^{n} b_{i}$.
\end{example}
\begin{proof}
	当 $n=2$ 时,
	
	$$
	\left(a_{1}+a_{2}\right)^{2}-\left(a_{1}-a_{2}\right)^{2}=4 a_{1} a_{2}=4 b_{1} b_{2}=\left(b_{1}+b_{2}\right)^{2}-\left(b_{1}-b_{2}\right)^{2},
	$$
	
	由假设得 $a_{1}-a_{2} \leqslant b_{1}-b_{2}$, 所以 $a_{1}+a_{2} \leqslant b_{1}+b_{2}$.
	
	当 $n \geqslant 3$ 时, 不妨设 $b_{1} b_{2} \cdots b_{n}=1$ (否则用 $a^{\prime}{ }_{i}=\frac{a_{i}}{\sqrt[n]{a_{1} a_{2} \cdots a_{n}}}, b_{i}^{\prime}=$ $\frac{b_{i}}{\sqrt[n]{b_{1} b_{2} \cdots b_{n}}}$ 代替 $\left.a_{i}, b_{i}(1 \leqslant i \leqslant n)\right)$.
	
	如果 $a_{1} \leqslant n-1$, 则由平均值不等式, 得
	
	$$
	\sum_{i=1}^{n} a_{i} \leqslant n(n-1)=(n-1) n \sqrt[n]{b_{1} b_{2} \cdots b_{n}} \leqslant(n-1) \sum_{i=1}^{n} b_{i}
	$$
	
	下设 $a_{1}>n-1$, 因为
	
	$$
	\begin{aligned}
	\sum_{1 \leqslant i<j \leqslant n}\left(a_{i}-a_{j}\right) \geqslant & {\left[\left(a_{1}-a_{n}\right)+\left(a_{2}-a_{n}\right)+\cdots+\left(a_{n-1}-a_{n}\right)\right] } \\
	& +\left[\left(a_{1}-a_{2}\right)+\left(a_{2}-a_{3}\right)+\cdots+\left(a_{n-2}-a_{n-1}\right)\right] \\
	= & \sum_{i=1}^{n} a_{i}+\left(a_{1}-a_{n-1}\right)-n a_{n} \\
	\sum_{1 \leqslant i<j \leqslant n}\left(b_{i}-b_{j}\right)= & \sum_{i=1}^{n}(n-2 i+1) b_{i} \\
	= & \sum_{i=1}^{n}\left[(n-1) b_{i}+(2-2 i) b_{i}\right] \\
	= & (n-1) \sum_{i=1}^{n} b_{i}+\sum_{i=1}^{n}(2-2 i) b_{i} \\
	\leqslant & (n-1) \sum_{i=1}^{n} b_{i}-2 b_{2}-2(n-1) b_{n}
	\end{aligned}
	$$
	
	所以, 当 $a_{1}-a_{n-1}-n a_{n}+2 b_{2}+2(n-1) b_{n} \geqslant 0$ 时, 结论成立.
	
	当 $a_{1}-a_{n-1}-n a_{n}+2 b_{2}+2(n-1) b_{n}<0$ 时,得
	
	$$
	n a_{n}>2(n-1) b_{n}+2 b_{2}+a_{1}-a_{n-1} \geqslant 2(n-1) b_{n}+2 b_{2} \geqslant 2 n b_{n}
	$$
	
	即 $a_{n}>2 b_{n}$.
	
	又由 $a_{1} a_{2} \cdots a_{n}=1$, 得 $a_{n} \leqslant 1$, 所以
	
	$$
	a_{1}-(n-1) a_{n}>n-1-(n-1)=0
	$$
	
	于是 $\quad 2 b_{2}<a_{n-1}+n a_{n}-a_{1}-2(n-1) b_{n}<a_{n-1}+a_{n} \leqslant 2 a_{n-1}$,
	
	即 $b_{2}<a_{n-1}$. 从而可得
	
	$$
	b_{1} b_{2} \cdots b_{n}=a_{1} a_{2} \cdots a_{n}>2 b_{n} b_{2} a_{1} a_{2} \cdots a_{n-2}
	$$
	
	即
	
	$$
	b_{1} b_{3} \cdots b_{n-1}>2 a_{1} a_{2} \cdots a_{n-2}
	$$
	
	而
	
	$$
	\begin{aligned}
	& b_{3} \leqslant b_{2}<a_{n-1} \leqslant a_{n-2} \\
	& b_{4} \leqslant b_{3}<a_{n-2} \leqslant a_{n-3} \\
	& \cdots \cdots \\
	& b_{n-1} \leqslant b_{n-2}<a_{3} \leqslant a_{2}
	\end{aligned}
	$$
	
	所以
	
	$$
	b_{1}>2 a_{1},(n-1) \sum_{i=1}^{n} b_{i}>2(n-1) a_{1}>n a_{1} \geqslant \sum_{i=1}^{n} a_{i}
	$$
\end{proof}
\begin{note}
	这个不等式可以看作舒尔 (Schur)不等式的一种弱形式.
	
	舒尔(Schur)不等式 设 $x, y, z \geqslant 0, r \in \mathbf{R}_{+}$. 则
	
	$$
	x^{r}(x-y)(x-z)+y^{r}(y-z)(y-x)+z^{r}(z-x)(z-y) \geqslant 0
	$$
	
	特别, 当 $r=1$ 时, 有
	
	
	\begin{equation*}
	x^{3}+y^{3}+z^{3}+3 x y z \geqslant x y(x+y)+y z(y+z)+z x(z+x) \tag{3}
	\end{equation*}
	
	
	由 (3), 有 $x^{3}+y^{3}+z^{3}+3 x y z \geqslant 2\left(x^{3 / 2} y^{3 / 2}+y^{3 / 2} z^{3 / 2}+z^{3 / 2} x^{3 / 2}\right)$.\\
	当 $a, b, c>0$ 时, 令 $x=a^{2 / 3}, y=b^{2 / 3}, z=c^{2 / 3}$, 由 (4) 式, 得到
	
	$$
	a^{2}+b^{2}+c^{2}+3(a b c)^{2 / 3} \geqslant 2(a b+b c+c a)
	$$
	
	即
	
	$$
	(a+b+c)^{2}-2(a b+c b+c a)+3(a b c)^{2 / 3} \geqslant 2(a b+b c+c a)
	$$
	
	于是
	
	
	\begin{equation*}
	a b+b c+c a \leqslant \frac{3}{4}(a b c)^{2 / 3}+\frac{1}{4} \tag{5}
	\end{equation*}
	
	
	由(5)知, 为证明原不等式, 只需证明
	
	
	\begin{gather*}
	3(a b c)^{2 / 3}<2 \sqrt{a b c} \\
	a b c<\left(\frac{2}{3}\right)^{6} \tag{6}
	\end{gather*}
	
	
	由平均值不等式,$a b c \leqslant\left(\frac{a+b+c}{3}\right)^{3}=\left(\frac{1}{3}\right)^{3}<\left(\frac{2}{3}\right)^{6}$.
	
	故不等式(6)成立.
	
	注 关于舒尔不等式的证明和应用, 可以参考有关文献.
\end{note}

\begin{example}
	给定 $n \geqslant 2, n \in \mathbf{Z}_{+}$, 求所有 $m \in \mathbf{Z}_{+}$, 使得对 $a_{i} \in \mathbf{R}_{+}, i=1,2$, $\cdots, n$, 满足 $a_{1} a_{2} \cdots a_{n}=1$, 则
	
	$$
	a_{1}^{m}+a_{2}^{m}+\cdots+a_{n}^{m} \geqslant \frac{1}{a_{1}}+\frac{1}{a_{2}}+\cdots+\frac{1}{a_{n}}
	$$
\end{example}
\begin{solution}
	取 $x=a_{1}=a_{2}=\cdots=a_{n-1}>0, a_{n}=\frac{1}{x^{n-1}}$, 则
	
	$$
	(n-1) x^{m}+\frac{1}{x^{(n-1) m}} \geqslant \frac{n-1}{x}+x^{n-1}
	$$
	
	由此得到 $m \geqslant n-1$. 现在, 假设 $m \geqslant n-1$, 则
	
	$$
	\begin{aligned}
	& (n-1)\left(a_{1}^{m}+a_{2}^{m}+\cdots+a_{n}^{m}\right)+n(m-n+1) \\
	= & \left(a_{1}^{m}+a_{2}^{m}+\cdots+a_{n-1}^{m}+1+1+\cdots+1\right)(\text { 共 } m-n+1 \text { 个 } 1) \\
	& +\left(a_{1}^{m}+a_{2}^{m}+\cdots+a_{n-2}^{m}+a_{n}^{m}+1+1+\cdots+1\right)+\cdots \\
	& +\left(a_{2}^{m}+a_{3}^{m}+\cdots+a_{n}^{m}+1+1+\cdots+1\right) \\
	\geqslant & m \sqrt[m]{\left(a_{1} a_{2} \cdots a_{n-1}\right)^{m}}+\cdots+m \sqrt[m]{\left(a_{2} \cdots a_{n}\right)^{m}} \\
	= & m\left(a_{1} a_{2} \cdots a_{n-1}+\cdots+a_{2} a_{3} \cdots a_{n}\right) \\
	= & m\left(\frac{1}{a_{1}}+\frac{1}{a_{2}}+\cdots+\frac{1}{a_{n}}\right) .
	\end{aligned}
	$$
	
	所以
	
	$$
	a_{1}^{m}+a_{2}^{m}+\cdots+a_{n}^{m} \geqslant \frac{m}{n-1}\left(\frac{1}{a_{1}}+\frac{1}{a_{2}}+\cdots+\frac{1}{a_{n}}\right)-\frac{n}{n-1}(m-n+1)
	$$
	
	于是, 只要证明
	
	$$
	\frac{m}{n-1}\left(\frac{1}{a_{1}}+\frac{1}{a_{2}}+\cdots+\frac{1}{a_{n}}\right)-\frac{n}{n-1}(m-n+1) \geqslant \frac{1}{a_{1}}+\frac{1}{a_{2}}+\cdots+\frac{1}{a_{n}}
	$$
	
	即
	
	$$
	(m-n+1)\left(\frac{1}{a_{1}}+\frac{1}{a_{2}}+\cdots+\frac{1}{a_{n}}-n\right) \geqslant 0
	$$
	
	由假设以及平均值不等式, 得
	
	$$
	\left(\frac{1}{a_{1}}+\frac{1}{a_{2}}+\cdots+\frac{1}{a_{n}}-n\right) \geqslant n \sqrt[n]{\frac{1}{a_{1}} \cdot \frac{1}{a_{2}} \cdots \cdots \cdot \frac{1}{a_{n}}}-n=0
	$$
	
	所以原不等式成立.
	
	故对所有满足 $m \geqslant n-1$ 的 $m \in \mathbf{Z}_{+}$均可.
\end{solution}
\begin{note}
	这个不等式可以看作舒尔 (Schur)不等式的一种弱形式.
	
	舒尔(Schur)不等式 设 $x, y, z \geqslant 0, r \in \mathbf{R}_{+}$. 则
	
	$$
	x^{r}(x-y)(x-z)+y^{r}(y-z)(y-x)+z^{r}(z-x)(z-y) \geqslant 0
	$$
	
	特别, 当 $r=1$ 时, 有
	
	
	\begin{equation*}
	x^{3}+y^{3}+z^{3}+3 x y z \geqslant x y(x+y)+y z(y+z)+z x(z+x) \tag{3}
	\end{equation*}
	
	
	由 (3), 有 $x^{3}+y^{3}+z^{3}+3 x y z \geqslant 2\left(x^{3 / 2} y^{3 / 2}+y^{3 / 2} z^{3 / 2}+z^{3 / 2} x^{3 / 2}\right)$.\\
	当 $a, b, c>0$ 时, 令 $x=a^{2 / 3}, y=b^{2 / 3}, z=c^{2 / 3}$, 由 (4) 式, 得到
	
	$$
	a^{2}+b^{2}+c^{2}+3(a b c)^{2 / 3} \geqslant 2(a b+b c+c a)
	$$
	
	即
	
	$$
	(a+b+c)^{2}-2(a b+c b+c a)+3(a b c)^{2 / 3} \geqslant 2(a b+b c+c a)
	$$
	
	于是
	
	
	\begin{equation*}
	a b+b c+c a \leqslant \frac{3}{4}(a b c)^{2 / 3}+\frac{1}{4} \tag{5}
	\end{equation*}
	
	
	由(5)知, 为证明原不等式, 只需证明
	
	
	\begin{gather*}
	3(a b c)^{2 / 3}<2 \sqrt{a b c} \\
	a b c<\left(\frac{2}{3}\right)^{6} \tag{6}
	\end{gather*}
	
	
	由平均值不等式,$a b c \leqslant\left(\frac{a+b+c}{3}\right)^{3}=\left(\frac{1}{3}\right)^{3}<\left(\frac{2}{3}\right)^{6}$.
	
	故不等式(6)成立.
	
	注 关于舒尔不等式的证明和应用, 可以参考有关文献.
\end{note}

\begin{example}
	设 $n(n \geqslant 2)$ 是整数, $a_{1}, a_{2}, \cdots, a_{n} \in \mathbf{R}_{+}$, 求证:
	
	$$
	\left(a_{1}^{3}+1\right)\left(a_{2}^{3}+1\right) \cdots\left(a_{n}^{3}+1\right) \geqslant\left(a_{1}^{2} a_{2}+1\right)\left(a_{2}^{2} a_{3}+1\right) \cdots\left(a_{n}^{2} a_{1}+1\right) .
	$$
\end{example}
\begin{proof}
	先证对于正实数 $x_{i}, y_{i}(i=1,2,3)$, 有
	
	$$
	\Pi\left(x_{i}^{3}+y_{i}^{3}\right) \geqslant\left(\prod x_{i}+\prod y_{i}\right)^{3}
	$$
	
	实际上, 由平均值不等式, 得
	
	$$
	\begin{aligned}
	& \sqrt[3]{\frac{x_{1}^{3} x_{2}^{3} x_{3}^{3}}{\prod\left(x_{i}^{3}+y_{i}^{3}\right)}} \leqslant \frac{1}{3}\left(\sum_{i=1}^{3} \frac{x_{i}^{3}}{x_{i}^{3}+y_{i}^{3}}\right) \\
	& \sqrt[3]{\frac{y_{1}^{3} y_{2}^{3} y_{3}^{3}}{\prod\left(x_{i}^{3}+y_{i}^{3}\right)}} \leqslant \frac{1}{3}\left(\sum_{i=1}^{3} \frac{y_{i}^{3}}{x_{i}^{3}+y_{i}^{3}}\right)
	\end{aligned}
	$$
	
	所以
	
	即
	
	$$
	\begin{aligned}
	& \sqrt[3]{\frac{x_{1}^{3} x_{3}^{3} x_{3}^{3}}{\prod^{3}\left(x_{i}^{3}+y_{i}^{3}\right)}}+\sqrt[3]{\frac{y_{1}^{3} y_{2}^{3} y_{3}^{3}}{\Pi\left(x_{i}^{3}+y_{i}^{3}\right)}} \leqslant 1, \\
	& \prod\left(x_{i}^{3}+y_{i}^{3}\right) \geqslant\left(\Pi x_{i}+\Pi y_{i}\right)^{3} .
	\end{aligned}
	$$
	
	令 $x_{1}=x_{2}=a_{k}, x_{3}=a_{k+1}, a_{n+1}=a_{1}, y_{1}=y_{2}=y_{3}=1, k=1,2, \cdots$, $n$, 则
	
	$$
	\left(a_{k}^{3}+1\right)^{2}\left(a_{k+1}^{3}+1\right) \geqslant\left(a_{k}^{2} a_{k+1}+1\right)^{3}, k=1,2, \cdots, n
	$$
	
	将它们相乘,则
	
	$$
	\prod\left(a_{i}^{3}+1\right)^{3} \geqslant \prod\left(a_{i}^{2} a_{i+1}+1\right)^{3}
	$$
	
	故
	
	$\left(a_{1}^{3}+1\right)\left(a_{2}^{3}+1\right) \cdots\left(a_{n}^{3}+1\right) \geqslant\left(a_{1}^{2} a_{2}+1\right)\left(a_{2}^{2} a_{3}+1\right) \cdots\left(a_{n}^{2} a_{1}+1\right)$.
\end{proof}
\begin{note}
	这个不等式可以看作舒尔 (Schur)不等式的一种弱形式.
	
	舒尔(Schur)不等式 设 $x, y, z \geqslant 0, r \in \mathbf{R}_{+}$. 则
	
	$$
	x^{r}(x-y)(x-z)+y^{r}(y-z)(y-x)+z^{r}(z-x)(z-y) \geqslant 0
	$$
	
	特别, 当 $r=1$ 时, 有
	
	
	\begin{equation*}
	x^{3}+y^{3}+z^{3}+3 x y z \geqslant x y(x+y)+y z(y+z)+z x(z+x) \tag{3}
	\end{equation*}
	
	
	由 (3), 有 $x^{3}+y^{3}+z^{3}+3 x y z \geqslant 2\left(x^{3 / 2} y^{3 / 2}+y^{3 / 2} z^{3 / 2}+z^{3 / 2} x^{3 / 2}\right)$.\\
	当 $a, b, c>0$ 时, 令 $x=a^{2 / 3}, y=b^{2 / 3}, z=c^{2 / 3}$, 由 (4) 式, 得到
	
	$$
	a^{2}+b^{2}+c^{2}+3(a b c)^{2 / 3} \geqslant 2(a b+b c+c a)
	$$
	
	即
	
	$$
	(a+b+c)^{2}-2(a b+c b+c a)+3(a b c)^{2 / 3} \geqslant 2(a b+b c+c a)
	$$
	
	于是
	
	
	\begin{equation*}
	a b+b c+c a \leqslant \frac{3}{4}(a b c)^{2 / 3}+\frac{1}{4} \tag{5}
	\end{equation*}
	
	
	由(5)知, 为证明原不等式, 只需证明
	
	
	\begin{gather*}
	3(a b c)^{2 / 3}<2 \sqrt{a b c} \\
	a b c<\left(\frac{2}{3}\right)^{6} \tag{6}
	\end{gather*}
	
	
	由平均值不等式,$a b c \leqslant\left(\frac{a+b+c}{3}\right)^{3}=\left(\frac{1}{3}\right)^{3}<\left(\frac{2}{3}\right)^{6}$.
	
	故不等式(6)成立.
	
	注 关于舒尔不等式的证明和应用, 可以参考有关文献.
\end{note}

\begin{example}
	给定整数 $n \geqslant 2$. 设 $0<a_{1} \leqslant a_{2} \leqslant \cdots \leqslant a_{n}$, 以及 $a_{1} \geqslant \frac{a_{2}}{2} \geqslant \cdots \geqslant$ $\frac{a_{n}}{n}$
	
	求证: $\frac{A_{n}}{G_{n}} \leqslant \frac{n+1}{2 \cdot \sqrt[n]{n!}}$. 其中 $A_{1}=\frac{a_{1}+\cdots+a_{n}}{n}, G_{n}=\sqrt[n]{a_{1} \cdots a_{n}}$.
\end{example}
\begin{proof}
	由假设 $0<a_{1} \leqslant a_{2} \leqslant \cdots \leqslant a_{n}, \frac{1}{a_{1}} \leqslant \frac{2}{a_{2}} \leqslant \cdots \leqslant \frac{n}{a_{n}}$, 以及切比雪夫不等式, 得到
	
	
	\begin{equation*}
	\left(\frac{1}{n} \sum_{i=1}^{n} a_{i}\right)\left(\frac{1}{n} \sum_{i=1}^{n} \frac{i}{a_{i}}\right) \leqslant \frac{1}{n} \sum_{i=1}^{n} a_{i} \cdot \frac{i}{a_{i}}=\frac{n+1}{2} \tag{1}
	\end{equation*}
	
	
	又由平均值不等式得
	
	
	\begin{equation*}
	\frac{1}{n} \sum_{i=1}^{n} \frac{i}{a_{i}} \geqslant \sqrt[n]{\frac{n!}{a_{1} \cdots a_{n}}}=\frac{\sqrt[n]{n!}}{G_{n}} \tag{2}
	\end{equation*}
	
	
	由(1)、(2)得
	
	$$
	\frac{A_{n}}{G_{n}} \leqslant \frac{n+1}{2 \cdot \sqrt[n]{n!}}
	$$
	
	从而, 命题成立.
\end{proof}
\begin{note}
	这个不等式可以看作舒尔 (Schur)不等式的一种弱形式.
	
	舒尔(Schur)不等式 设 $x, y, z \geqslant 0, r \in \mathbf{R}_{+}$. 则
	
	$$
	x^{r}(x-y)(x-z)+y^{r}(y-z)(y-x)+z^{r}(z-x)(z-y) \geqslant 0
	$$
	
	特别, 当 $r=1$ 时, 有
	
	
	\begin{equation*}
	x^{3}+y^{3}+z^{3}+3 x y z \geqslant x y(x+y)+y z(y+z)+z x(z+x) \tag{3}
	\end{equation*}
	
	
	由 (3), 有 $x^{3}+y^{3}+z^{3}+3 x y z \geqslant 2\left(x^{3 / 2} y^{3 / 2}+y^{3 / 2} z^{3 / 2}+z^{3 / 2} x^{3 / 2}\right)$.\\
	当 $a, b, c>0$ 时, 令 $x=a^{2 / 3}, y=b^{2 / 3}, z=c^{2 / 3}$, 由 (4) 式, 得到
	
	$$
	a^{2}+b^{2}+c^{2}+3(a b c)^{2 / 3} \geqslant 2(a b+b c+c a)
	$$
	
	即
	
	$$
	(a+b+c)^{2}-2(a b+c b+c a)+3(a b c)^{2 / 3} \geqslant 2(a b+b c+c a)
	$$
	
	于是
	
	
	\begin{equation*}
	a b+b c+c a \leqslant \frac{3}{4}(a b c)^{2 / 3}+\frac{1}{4} \tag{5}
	\end{equation*}
	
	
	由(5)知, 为证明原不等式, 只需证明
	
	
	\begin{gather*}
	3(a b c)^{2 / 3}<2 \sqrt{a b c} \\
	a b c<\left(\frac{2}{3}\right)^{6} \tag{6}
	\end{gather*}
	
	
	由平均值不等式,$a b c \leqslant\left(\frac{a+b+c}{3}\right)^{3}=\left(\frac{1}{3}\right)^{3}<\left(\frac{2}{3}\right)^{6}$.
	
	故不等式(6)成立.
	
	注 关于舒尔不等式的证明和应用, 可以参考有关文献.
\end{note}

\begin{example}
	已知 $x, y, z \in \mathbf{R}_{+} \cup\{0\}$, 且 $x+y+z=2$.
\end{example}
\begin{proof}
	注意到
	
	
	\begin{align*}
	& x^{2} y^{2}+y^{2} z^{2}+z^{2} x^{2}+x y z \\
	= & \frac{1}{2}\left(2 x^{2} y^{2}+2 y^{2} z^{2}+2 z^{2} x^{2}+2 x y z\right) \\
	= & \frac{1}{2}(x y \cdot 2 x y+y z \cdot 2 y z+z x \cdot 2 z x+2 x y z) \\
	\leqslant & \frac{1}{2}\left[x y\left(x^{2}+y^{2}\right)+y z\left(y^{2}+z^{2}\right)+z x\left(z^{2}+x^{2}\right)+2 x y z\right]  \tag{1}\\
	= & \frac{1}{2}\left[(x y+y z+z x)\left(x^{2}+y^{2}+z^{2}\right)-x y z^{2}-y z x^{2}-z x y^{2}+2 x y z\right] \\
	= & \frac{1}{2}\left[(x y+y z+z x)\left(x^{2}+y^{2}+z^{2}\right)-x y z(x+y+z-2)\right]
	\end{align*}
	
	
	$$
	=\frac{1}{2}(x y+y z+z x)\left(x^{2}+y^{2}+z^{2}\right)
	$$
	
	由此得到
	
	
	\begin{equation*}
	x^{2} y^{2}+y^{2} z^{2}+z^{2} x^{2}+x y z \leqslant \frac{1}{2}\left[(x y+y z+z x)\left(x^{2}+y^{2}+z^{2}\right)\right] \tag{2}
	\end{equation*}
	
	
	由式(1)知, 当 $x=y=z$ 或 $x=y, z=0$ 或 $y=z, x=0$ 或 $z=x, y=0$时,式(2)取等号.
	
	又 $x+y+z=2$, 因此, 当
	
	$$
	(x, y, z)=\left(\frac{2}{3}, \frac{2}{3}, \frac{2}{3}\right) \text { 或 }(1,1,0) \text { 或 }(1,0,1) \text { 或 }(0,1,1)
	$$
	
	时,式(2)取等号.
	
	运用常见不等式
	
	$$
	\alpha \beta \leqslant\left(\frac{\alpha+\beta}{2}\right)^{2}(\alpha, \beta \in \mathbf{R})
	$$
	
	令 $\alpha=2 x y+2 y z+2 z x, \beta=x^{2}+y^{2}+z^{2}$. 则
	
	$$
	\frac{1}{2}(x y+y z+z x)\left(x^{2}+y^{2}+z^{2}\right)
	$$
	
	
	\begin{align*}
	& =\frac{1}{4}(2 x y+2 y z+2 z x)\left(x^{2}+y^{2}+z^{2}\right) \\
	& \leqslant \frac{1}{4}\left(\frac{2 x y+2 y z+2 z x+x^{2}+y^{2}+z^{2}}{2}\right)^{2} \\
	& =\frac{1}{16}(x+y+z)^{4}=1 \tag{3}
	\end{align*}
	
	
	结合式(2)得
	
	
	\begin{equation*}
	x^{2} y^{2}+y^{2} z^{2}+z^{2} x^{2}+x y z \leqslant 1 \tag{4}
	\end{equation*}
	
	
	由式(3)取等号的条件知, 当
	
	$$
	\alpha=\beta \Leftrightarrow 2 x y+2 y z+2 z x=x^{2}+y^{2}+z^{2}
	$$
	
	时,式(4)等号成立.
	
	故 $(x, y, z)=(1,1,0)$ 或 $(1,0,1)$ 或 $(0,1,1)$.
\end{proof}
\begin{note}
	这个不等式可以看作舒尔 (Schur)不等式的一种弱形式.
	
	舒尔(Schur)不等式 设 $x, y, z \geqslant 0, r \in \mathbf{R}_{+}$. 则
	
	$$
	x^{r}(x-y)(x-z)+y^{r}(y-z)(y-x)+z^{r}(z-x)(z-y) \geqslant 0
	$$
	
	特别, 当 $r=1$ 时, 有
	
	
	\begin{equation*}
	x^{3}+y^{3}+z^{3}+3 x y z \geqslant x y(x+y)+y z(y+z)+z x(z+x) \tag{3}
	\end{equation*}
	
	
	由 (3), 有 $x^{3}+y^{3}+z^{3}+3 x y z \geqslant 2\left(x^{3 / 2} y^{3 / 2}+y^{3 / 2} z^{3 / 2}+z^{3 / 2} x^{3 / 2}\right)$.\\
	当 $a, b, c>0$ 时, 令 $x=a^{2 / 3}, y=b^{2 / 3}, z=c^{2 / 3}$, 由 (4) 式, 得到
	
	$$
	a^{2}+b^{2}+c^{2}+3(a b c)^{2 / 3} \geqslant 2(a b+b c+c a)
	$$
	
	即
	
	$$
	(a+b+c)^{2}-2(a b+c b+c a)+3(a b c)^{2 / 3} \geqslant 2(a b+b c+c a)
	$$
	
	于是
	
	
	\begin{equation*}
	a b+b c+c a \leqslant \frac{3}{4}(a b c)^{2 / 3}+\frac{1}{4} \tag{5}
	\end{equation*}
	
	
	由(5)知, 为证明原不等式, 只需证明
	
	
	\begin{gather*}
	3(a b c)^{2 / 3}<2 \sqrt{a b c} \\
	a b c<\left(\frac{2}{3}\right)^{6} \tag{6}
	\end{gather*}
	
	
	由平均值不等式,$a b c \leqslant\left(\frac{a+b+c}{3}\right)^{3}=\left(\frac{1}{3}\right)^{3}<\left(\frac{2}{3}\right)^{6}$.
	
	故不等式(6)成立.
	
	注 关于舒尔不等式的证明和应用, 可以参考有关文献.
\end{note}

\begin{example}
	已知 $a 、 b 、 c$ 为正实数. 证明:
	
	$$
	\frac{a^{2} b(b-c)}{a+b}+\frac{b^{2} c(c-a)}{b+c}+\frac{c^{2} a(a-b)}{c+a} \geqslant 0
	$$
\end{example}
	\begin{proof}
	$$
	\begin{aligned}
	\text { 原式 } & \Leftrightarrow \frac{a^{2} b^{2}}{a+b}+\frac{b^{2} c^{2}}{b+c}+\frac{c^{2} a^{2}}{c+a} \geqslant a b c\left(\frac{a}{a+b}+\frac{b}{b+c}+\frac{c}{c+a}\right) \\
	& \Leftrightarrow \frac{a b}{c(a+b)}+\frac{b c}{a(b+c)}+\frac{a c}{b(c+a)} \geqslant \frac{a}{a+b}+\frac{b}{b+c}+\frac{c}{c+a} \\
	& \Leftrightarrow(a b+b c+a c)\left(\frac{1}{a c+b c}+\frac{1}{a b+a c}+\frac{1}{b c+a b}\right) \\
	& \geqslant \frac{a c}{a c+b c}+\frac{a b}{a b+a c}+\frac{b c}{b c+a b}+3 .
	\end{aligned}
	$$
	
	下面进行换元. 令
	
	$$
	\left\{\begin{array}{l}
	x=a b+a c \\
	y=b c+b a \\
	z=c a+c b
	\end{array}, \Rightarrow\left\{\begin{array}{l}
	a c=\frac{x+z-y}{2} \\
	a b=\frac{x+y-z}{2}, \Rightarrow a b+b c+c a=\frac{x+y+z}{2} \\
	b c=\frac{y+z-x}{2}
	\end{array}\right.\right.
	$$
	
	故
	
	$$
	\begin{aligned}
	\text { 原式 } & \Leftrightarrow \frac{1}{2}(x+y+z)\left(\frac{1}{x}+\frac{1}{y}+\frac{1}{z}\right) \geqslant \frac{x+z-y}{2 z}+\frac{x+y-z}{2 x}+\frac{y+z-}{2 y} \\
	& \Leftrightarrow(x+y+z)\left(\frac{1}{x}+\frac{1}{y}+\frac{1}{z}\right) \geqslant \frac{x-y}{z}+\frac{y-z}{x}+\frac{z-x}{y}+9 \\
	& \Leftrightarrow 3+\frac{y}{x}+\frac{z}{x}+\frac{x}{y}+\frac{z}{y}+\frac{x}{z}+\frac{y}{z} \geqslant \frac{x-y}{z}+\frac{y-z}{x}+\frac{z-x}{y}+9 \\
	& \Leftrightarrow \frac{2 y}{z}+\frac{2 z}{x}+\frac{2 x}{y} \geqslant 6 \\
	& \Leftrightarrow \frac{y}{z}+\frac{z}{x}+\frac{x}{y} \geqslant 3 .
	\end{aligned}
	$$
	
	由均值不等式即知结论成立.
\end{proof}
\begin{example}
	设正实数 $x_{1}, x_{2}, \cdots, x_{n}$ 满足 $x_{1} x_{2} \cdots x_{n}=1$. 证明:
	$$
	\sum_{i=1}^{n} \frac{1}{n-1+x_{i}} \leqslant 1
	$$
\end{example}
\begin{proof}
	用反证法.
	
	假设 $\quad \sum_{i=1}^{n} \frac{1}{n-1+x_{i}}>1$.\\
	则对任意的 $k(k \in\{1,2, \cdots, n\})$, 由式(1)有
	
	$$
	\begin{aligned}
	\frac{1}{n-1+x_{k}} & >1-\sum_{\substack{1 \leqslant i \leqslant n \\
	i \neq k}} \frac{1}{n-1+x_{i}} \\
	& =\sum_{\substack{1 \leqslant i \leqslant n \\
	i \neq k}}\left(\frac{1}{n-1}-\frac{1}{n-1+x_{i}}\right) \\
	& =\sum_{\substack{1 \leqslant i \leqslant n \\
	i \neq k}} \frac{x_{i}}{(n-1)\left(n-1+x_{i}\right)} \\
	& \geqslant(n-1)\left[\prod_{\substack{\leqslant i \leqslant n \\
	i \neq k}} \frac{x_{i}}{(n-1)\left(n-1+x_{i}\right)}\right]^{\frac{1}{n-1}} \\
	& =\left(\prod_{\substack{1 \leq i \leq n \\
	i \neq k}} \frac{x_{i}}{n-1+x_{i}}\right)^{\frac{1}{n-1}}
	\end{aligned}
	$$
	
	即对 $1 \leqslant k \leqslant n ,$ 均有
	
	$$
	\frac{1}{n-1+x_{k}}>\left(\prod_{\substack{1 \leq i \leq n \\ i \neq k}} \frac{x_{i}}{n-1+x_{i}}\right)^{\frac{1}{n-1}}
	$$
	
	取积得
	
	$$
	\prod_{k=1}^{n} \frac{1}{n-1+x_{k}}>\prod_{k=1}^{n}\left(\prod_{\substack{1 \leqslant i \leqslant n \\ i \neq k}} \frac{x_{i}}{n-1+x_{i}}\right)^{\frac{1}{n-1}}=\prod_{k=1}^{n} \frac{x_{k}}{n-1+x_{k}}
	$$
	
	则 $\prod_{k=1}^{n} x_{k}<1$, 这与 $\prod_{k=1}^{n} x_{k}=1$ 矛盾.
	
	所以, 假设不成立,必有
	
	$$
	\sum_{i=1}^{n} \frac{1}{n-1+x_{i}} \leqslant 1
	$$
\end{proof}
\begin{note}
	这个不等式可以看作舒尔 (Schur)不等式的一种弱形式.
	
	舒尔(Schur)不等式 设 $x, y, z \geqslant 0, r \in \mathbf{R}_{+}$. 则
	
	$$
	x^{r}(x-y)(x-z)+y^{r}(y-z)(y-x)+z^{r}(z-x)(z-y) \geqslant 0
	$$
	
	特别, 当 $r=1$ 时, 有
	
	
	\begin{equation*}
	x^{3}+y^{3}+z^{3}+3 x y z \geqslant x y(x+y)+y z(y+z)+z x(z+x) \tag{3}
	\end{equation*}
	
	
	由 (3), 有 $x^{3}+y^{3}+z^{3}+3 x y z \geqslant 2\left(x^{3 / 2} y^{3 / 2}+y^{3 / 2} z^{3 / 2}+z^{3 / 2} x^{3 / 2}\right)$.\\
	当 $a, b, c>0$ 时, 令 $x=a^{2 / 3}, y=b^{2 / 3}, z=c^{2 / 3}$, 由 (4) 式, 得到
	
	$$
	a^{2}+b^{2}+c^{2}+3(a b c)^{2 / 3} \geqslant 2(a b+b c+c a)
	$$
	
	即
	
	$$
	(a+b+c)^{2}-2(a b+c b+c a)+3(a b c)^{2 / 3} \geqslant 2(a b+b c+c a)
	$$
	
	于是
	
	
	\begin{equation*}
	a b+b c+c a \leqslant \frac{3}{4}(a b c)^{2 / 3}+\frac{1}{4} \tag{5}
	\end{equation*}
	
	
	由(5)知, 为证明原不等式, 只需证明
	
	
	\begin{gather*}
	3(a b c)^{2 / 3}<2 \sqrt{a b c} \\
	a b c<\left(\frac{2}{3}\right)^{6} \tag{6}
	\end{gather*}
	
	
	由平均值不等式,$a b c \leqslant\left(\frac{a+b+c}{3}\right)^{3}=\left(\frac{1}{3}\right)^{3}<\left(\frac{2}{3}\right)^{6}$.
	
	故不等式(6)成立.
	
	注 关于舒尔不等式的证明和应用, 可以参考有关文献.
\end{note}

\begin{example}
	设实数 $a 、 b 、 c$ 满足 $a+b+c=1, a b c>0$. 求证:
	
	$$
	a b+b c+c a<\frac{\sqrt{a b c}}{2}+\frac{1}{4}
	$$
\end{example}
\begin{proof}
	由于 $a b c>0$, 则 $a 、 b 、 c$ 中或者一个正数, 两个负数; 或者三个都是正数.
	
	如果一个为正数,两个为负数, 不妨设 $a>0, b, c<0$. 则
	
	$$
	a b+b c+c a=b(a+c)+c a=b(1-b)+c a<0<\frac{\sqrt{a b c}}{2}+\frac{1}{4}
	$$
	
	结论成立.
	
	下面设 $a, b, c>0$, 且不妨设 $a \geqslant b \geqslant c$. 则 $a \geqslant \frac{1}{3}, 0<c \leqslant \frac{1}{3}$.
	
	于是 $\quad a b+b c+c a-\frac{\sqrt{a b c}}{2}=c(a+b)+\sqrt{a b}\left(\sqrt{a b}-\frac{\sqrt{c}}{2}\right)$
	
	$$
	=c(1-c)+\sqrt{a b}\left(\sqrt{a b}-\frac{\sqrt{c}}{2}\right)
	$$
	
	由于 $\sqrt{a b} \geqslant \sqrt{\frac{b}{3}} \geqslant \sqrt{\frac{c}{3}}>\frac{\sqrt{c}}{2}$, 且 $\sqrt{a b} \leqslant \frac{a+b}{2}=\frac{1-c}{2}$, 所以,
	
	$$
	\begin{aligned}
	& c(1-c)+\sqrt{a b}\left(\sqrt{a b}-\frac{\sqrt{c}}{2}\right) \\
	\leqslant & c(1-c)+\frac{1-c}{2}\left(\frac{1-c}{2}-\frac{\sqrt{c}}{2}\right) \\
	= & \frac{1}{4}-\frac{3 c^{2}}{4}+\frac{c \sqrt{c}}{4}+\frac{c}{2}-\frac{\sqrt{c}}{4}
	\end{aligned}
	$$
	
	于是, 只要证明 $\frac{3 c^{2}}{4}-\frac{c \sqrt{c}}{4}-\frac{c}{2}+\frac{\sqrt{c}}{4}>0$, 即
	
	
	\begin{equation*}
	3 c \sqrt{c}-c-2 \sqrt{c}+1>0 \tag{1}
	\end{equation*}
	
	
	由于 $0<c \leqslant \frac{1}{3}$, 所以 $\frac{1}{3}-c \geqslant 0$.
	
	又由平均值不等式
	
	
	\begin{equation*}
	3 c \sqrt{c}+\frac{1}{3}+\frac{1}{3} \geqslant 3\left(3 c \sqrt{c} \cdot \frac{1}{3} \cdot \frac{1}{3}\right)^{\frac{1}{3}}=\sqrt[3]{9} \sqrt{c}>2 \sqrt{c} \tag{3}
	\end{equation*}
	
	
	将(2)、(3)两式相加即得(1)式成立. 因此, 原不等式成立.
\end{proof}
\begin{note}
	这个不等式可以看作舒尔 (Schur)不等式的一种弱形式.
	
	舒尔(Schur)不等式 设 $x, y, z \geqslant 0, r \in \mathbf{R}_{+}$. 则
	
	$$
	x^{r}(x-y)(x-z)+y^{r}(y-z)(y-x)+z^{r}(z-x)(z-y) \geqslant 0
	$$
	
	特别, 当 $r=1$ 时, 有
	
	
	\begin{equation*}
	x^{3}+y^{3}+z^{3}+3 x y z \geqslant x y(x+y)+y z(y+z)+z x(z+x) \tag{3}
	\end{equation*}
	
	
	由 (3), 有 $x^{3}+y^{3}+z^{3}+3 x y z \geqslant 2\left(x^{3 / 2} y^{3 / 2}+y^{3 / 2} z^{3 / 2}+z^{3 / 2} x^{3 / 2}\right)$.\\
	当 $a, b, c>0$ 时, 令 $x=a^{2 / 3}, y=b^{2 / 3}, z=c^{2 / 3}$, 由 (4) 式, 得到
	
	$$
	a^{2}+b^{2}+c^{2}+3(a b c)^{2 / 3} \geqslant 2(a b+b c+c a)
	$$
	
	即
	
	$$
	(a+b+c)^{2}-2(a b+c b+c a)+3(a b c)^{2 / 3} \geqslant 2(a b+b c+c a)
	$$
	
	于是
	
	
	\begin{equation*}
	a b+b c+c a \leqslant \frac{3}{4}(a b c)^{2 / 3}+\frac{1}{4} \tag{5}
	\end{equation*}
	
	
	由(5)知, 为证明原不等式, 只需证明
	
	
	\begin{gather*}
	3(a b c)^{2 / 3}<2 \sqrt{a b c} \\
	a b c<\left(\frac{2}{3}\right)^{6} \tag{6}
	\end{gather*}
	
	
	由平均值不等式,$a b c \leqslant\left(\frac{a+b+c}{3}\right)^{3}=\left(\frac{1}{3}\right)^{3}<\left(\frac{2}{3}\right)^{6}$.
	
	故不等式(6)成立.
	
	注 关于舒尔不等式的证明和应用, 可以参考有关文献.
\end{note}

\begin{example}
	设 $x, y, z \in \mathbf{R}_{+}$, 求证:
	
	$$
	\frac{x y}{z}+\frac{y z}{x}+\frac{z x}{y}>2 \sqrt[3]{x^{3}+y^{3}+z^{3}}
	$$
\end{example}
\begin{proof}
	欲证的不等式等价于
	
	$$
	\begin{aligned}
	& \left(\frac{x y}{z}+\frac{y z}{x}+\frac{z x}{y}\right)^{3}>8\left(x^{3}+y^{3}+z^{3}\right) \\
	\Leftrightarrow & \left(\frac{x y}{z}\right)^{3}+\left(\frac{y z}{x}\right)^{3}+\left(\frac{z x}{y}\right)^{3}+6 x y z+3 x^{3}\left(\frac{y}{z}+\frac{z}{y}\right) \\
	& +3 y^{3}\left(\frac{x}{z}+\frac{z}{x}\right)+3 z^{3}\left(\frac{y}{x}+\frac{x}{y}\right) \\
	> & 8\left(x^{3}+y^{3}+z^{3}\right)
	\end{aligned}
	$$
	
	因为 $\frac{y}{z}+\frac{z}{y} \geqslant 2, \frac{x}{z}+\frac{z}{x} \geqslant 2, \frac{y}{x}+\frac{x}{y} \geqslant 2$, 所以只需证
	
	
	\begin{equation*}
	\left(\frac{x y}{z}\right)^{3}+\left(\frac{y z}{x}\right)^{3}+\left(\frac{z x}{y}\right)^{3}+6 x y z>2\left(x^{3}+y^{3}+z^{3}\right) \tag{1}
	\end{equation*}
	
	
	不妨设 $x \geqslant y \geqslant z$, 记
	
	$$
	f(x, y, z)=\left(\frac{x y}{z}\right)^{3}+\left(\frac{y z}{x}\right)^{3}+\left(\frac{z x}{y}\right)^{3}+6 x y z-2\left(x^{3}+y^{3}+z^{3}\right)
	$$
	
	下证 $f(x, y, z)-f(y, y, z) \geqslant 0, f(y, y, z) \geqslant 0$.
	
	事实上,
	
	$$
	\begin{aligned}
	& f(x, y, z)-f(y, y, z) \\
	= & \left(\frac{x y}{z}\right)^{3}+\left(\frac{y z}{x}\right)^{3}+\left(\frac{z x}{y}\right)^{3}+6 x y z-2\left(x^{3}+y^{3}+z^{3}\right) \\
	& -\left[\left(\frac{y^{2}}{z}\right)^{3}+z^{3}+z^{3}+6 y^{2} z-2\left(y^{3}+y^{3}+z^{3}\right)\right] \\
	= & \left(\frac{x y}{z}\right)^{3}-\frac{y^{6}}{z^{3}}+\left(\frac{y z}{x}\right)^{3}+\left(\frac{z x}{y}\right)^{3}-2 z^{3}+6 y z(x-y)-2\left(x^{3}-y^{3}\right) \\
	= & \left(x^{3}-y^{3}\right)\left(\frac{y^{3}}{z^{3}}+\frac{z^{3}}{y^{3}}-2+\frac{6 y z}{x^{2}+x y+y^{2}}-\frac{z^{3}}{x^{3}}\right),
	\end{aligned}
	$$
	
	而 $x^{3}-y^{3} \geqslant 0, \frac{y^{3}}{z^{3}}+\frac{z^{3}}{y^{3}} \geqslant 2$,
	
	$$
	\frac{6 y z}{x^{2}+x y+y^{2}}-\frac{z^{3}}{x^{3}} \geqslant \frac{2 y z}{x^{2}}-\frac{z^{3}}{x^{3}}=\frac{z\left(2 x y-z^{2}\right)}{x^{3}}>0
	$$
	
	所以 $f(x, y, z)-f(y, y, z) \geqslant 0$.
	
	又
	
	$$
	\begin{aligned}
	f(y, y, z) & =\left(\frac{y^{2}}{z}\right)^{3}+z^{3}+z^{3}+6 y^{2} z-2\left(y^{3}+y^{3}+z^{3}\right) \\
	& =\frac{y^{6}}{z^{3}}+2 y^{2} z+2 y^{2} z+2 y^{2} z-4 y^{3} \\
	& \geqslant 4 \sqrt[4]{2^{3} y^{12}}-4 y^{3}=4(\sqrt[4]{8}-1) y^{3}>0
	\end{aligned}
	$$
	
	从而(1)式得证,原命题得证.
\end{proof}
\begin{note}
	本题所用的方法叫调整法,在各级竞赛中偶尔出现时, 因难度大而得分率极低. 此题的解答由第 50 届 IMO 金牌获得者郑志伟给出.
\end{note}

\begin{example}
	设 $x 、 y 、 z$ 为非负实数, 且 $x+y+z=1$, 求证:
	
	$$
	x y+y z+z x-2 x y z \leqslant \frac{7}{27}
	$$
\end{example}
\begin{proof}
	不妨设 $x \geqslant y \geqslant z$.
	
	当 $x \geqslant \frac{1}{2}$ 时, 则 $y z-2 x y z \leqslant 0$, 所以
	
	$$
	x y+y z+z x-2 x y z \leqslant x y+z x=x(1-x) \leqslant \frac{1}{4}<\frac{7}{27}
	$$
	
	当 $x<\frac{1}{2}$ 时, 则 $y \leqslant \frac{1}{2}, z \leqslant \frac{1}{2}$.
	
	$$
	(1-2 x)(1-2 y)(1-2 z)=1-2+4(x y+y z+z x)-8 x y z
	$$
	
	又由平均值不等式, 得
	
	$$
	(1-2 x)(1-2 y)(1-2 z) \leqslant\left[\frac{3-2(x+y+z)}{3}\right]^{3}=\frac{1}{27}
	$$
	
	从而
	
	$$
	x y+y z+z x-2 x y z \leqslant \frac{1}{4}\left(\frac{1}{27}+1\right)=\frac{7}{27}
	$$
\end{proof}
\begin{note}
	在证明的过程中, 其实平均值不等式的作用是较小的. 本题的证明有一定的难度, 是因为像这样证明不等式的方法和处理技巧并不多见. 此解答由第 51 届 IMO 金牌获得者李嘉伦给出.
\end{note}

\begin{example}
	设 $n$ 为正整数, $\left(x_{1}, x_{2}, \cdots, x_{n}\right),\left(y_{1}, y_{2}, \cdots, y_{n}\right)$ 为两个正数数列. 假设正实数列 $\left(z_{1}, z_{2}, \cdots, z_{2 n}\right)$, 满足
	
	$$
	z_{i+j}^{2} \geqslant x_{i} y_{j}, 1 \leqslant i, j \leqslant n
	$$
	
	令 $M=\max \left\{z_{1}, z_{2}, \cdots, z_{2 n}\right\}$, 证明:
	
	$$
	\left(\frac{M+z_{2}+z_{3}+\cdots+z_{2 n}}{2 n}\right)^{2} \geqslant\left(\frac{x_{1}+x_{2}+\cdots+x_{n}}{n}\right)\left(\frac{y_{1}+y_{2}+\cdots+y_{n}}{n}\right)
	$$
\end{example}
\begin{proof}
	$\quad$ 令 $X=\max \left\{x_{1}, x_{2}, \cdots, x_{n}\right\}, Y=\max \left\{y_{1}, y_{2}, \cdots, y_{n}\right\}$, 不妨假设 $X=Y=1$ (否则用 $a_{i}=\frac{x_{i}}{X}, b_{i}=\frac{y_{i}}{Y}, c_{i}=\frac{z_{i}}{\sqrt{X Y}}$ 代替 $)$.
	
	我们将证明
	
	$$
	M+z_{2}+z_{3}+\cdots+z_{2 n} \geqslant x_{1}+x_{2}+\cdots+x_{n}+y_{1}+y_{2}+\cdots+y_{n}
	$$
	
	于是
	
	$$
	\frac{M+z_{2}+z_{3}+\cdots+z_{2 n}}{2 n} \geqslant \frac{1}{2}\left(\frac{x_{1}+x_{2}+\cdots+x_{n}}{n}+\frac{y_{1}+y_{2}+\cdots+y_{n}}{n}\right)
	$$
	
	由平均值不等式得到原不等式成立.
	
	为了证明上述不等式, 我们将证明: 对任意 $r>0$, 左边大于 $r$ 的项数不小于右边相应的项数, 那么, 对每个 $k$, 左边第 $k$ 个最大的项大于或等于右边第 $k$个最大的项, 这样就证明了上述不等式成立. 证明如下:
	
	如果 $r \geqslant 1$, 则右边没有项大于 $r$, 所以只考虑 $r<1$.
	
	令 $A=\left\{x_{i} \mid x_{i}>r, 1 \leqslant i \leqslant n\right\}, a=|A|, B=\left\{y_{i} \mid y_{i}>r, 1 \leqslant i \leqslant\right.$ $n\}, b=|B|$. 由于 $X=Y=1$, 所以 $a, b$ 大于 0 .\\
	由于 $x_{i}>r, y_{j}>r$ 推出 $z_{i+j} \geqslant \sqrt{x_{i} y_{j}}>r$. 于是
	
	$$
	A+B=\{\alpha+\beta \mid \alpha \in A, \beta \in B\} \subseteq C=\left\{z_{i} \mid z_{i}>r, 2 \leqslant i \leqslant 2 n\right\}
	$$
	
	但是, 由于如果 $A=\left\{i_{1}, i_{2}, \cdots, i_{a}\right\}, i_{1}<i_{2}<\cdots<i_{a}, B=\left\{j_{1}, j_{2}, \cdots\right.$, $\left.j_{b}\right\}, j_{1}<j_{2}<\cdots<j_{b}$, 则 $a+b-1$ 个数 $i_{1}+j_{1}, i_{1}+j_{2}, \cdots, i_{1}+j_{b}, i_{2}+j_{b}, \cdots$, $i_{a}+j_{b}$ 互不相同, 且属于 $A+B$. 所以 $|A+B| \geqslant|A|+|B|-1$. 因此 $|C| \geqslant$ $a+b-1$. 特别, $|C| \geqslant 1$, 于是对某个 $k, z_{k}>r$, 那么 $M>r$. 所以上式的左边至少有 $a+b$ 项大于 $r$, 由于 $a+b$ 为右边大于 $r$ 的项数, 于是, 上述不等式成立.
\end{proof}
\begin{note}
	在证明的过程中, 其实平均值不等式的作用是较小的. 本题的证明有一定的难度, 是因为像这样证明不等式的方法和处理技巧并不多见. 此解答由第 51 届 IMO 金牌获得者李嘉伦给出.
\end{note}

\begin{example}
	已知 $a \geqslant b \geqslant c>0$, 求证: $\frac{b}{a}+\frac{c}{b}+\frac{a}{c}+a b c \geqslant a+b+c+1$.
\end{example}	
	
	$$
	b^{2} c+c^{2} a+a^{2} b+a^{2} b^{2} c^{2} \geqslant a^{2} b c+a b^{2} c+a b c^{2}+a b c
	$$
	
	而
	
	$$
	\begin{aligned}
	& \frac{a^{2} b^{2} c^{2}+a^{2} b+a^{2} c}{3} \geqslant a^{2} b c \\
	& \frac{a^{2} b^{2} c^{2}+b^{2} a+b^{2} c}{3} \geqslant a b^{2} c \\
	& \frac{a^{2} b^{2} c^{2}+c^{2} a+c^{2} b}{3} \geqslant a b c^{2}
	\end{aligned}
	$$
	
	三式相加得
	
	$$
	a^{2} b^{2} c^{2}+\frac{1}{3}\left(a^{2} b+a^{2} c+b^{2} a+b^{2} c+c^{2} a+c^{2} b\right) \geqslant a^{2} b c+a b^{2} c+a b c^{2}
	$$
	
	只需证明
	
	$$
	\frac{2}{3}\left(a^{2} b+b^{2} c+c^{2} a\right) \geqslant \frac{1}{3}\left(a b^{2}+b c^{2}+c a^{2}\right)+a b c
	$$
	
	而
	
	$$
	\frac{1}{3}\left(a^{2} b+b^{2} c+c^{2} a\right) \geqslant a b c
	$$
	
	只需证明
	
	$$
	a^{2} b+b^{2} c+c^{2} a \geqslant a b^{2}+b c^{2}+c a^{2}
	$$
	
	而 $a^{2} b+b^{2} c+c^{2} a-\left(a b^{2}+b c^{2}+c a^{2}\right)=(a-b)(b-c)(a-c) \geqslant 0$,得证.
	
	\section*{2. 2 平均值不等式在求极值中的应用}
	不等式在求极值中起着重要的作用,在利用平均值不等式求极值的过程中, 要注意“缩”或“放”的结果是否为常数 (通常是和与积), 同时必须指出等号成立的条件.
	
\begin{example}
	设 $a 、 b 、 c$ 为正实数, 求
	$$
	\frac{a+3 c}{a+2 b+c}+\frac{4 b}{a+b+2 c}-\frac{8 c}{a+b+3 c}
	$$
	
	的最小值.
\end{example}
\begin{proof}
	原不等式等价于
	$$
	b^{2} c+c^{2} a+a^{2} b+a^{2} b^{2} c^{2} \geqslant a^{2} b c+a b^{2} c+a b c^{2}+a b c
	$$
	
	而
	$$
	\begin{aligned}
	& \frac{a^{2} b^{2} c^{2}+a^{2} b+a^{2} c}{3} \geqslant a^{2} b c \\
	& \frac{a^{2} b^{2} c^{2}+b^{2} a+b^{2} c}{3} \geqslant a b^{2} c \\
	& \frac{a^{2} b^{2} c^{2}+c^{2} a+c^{2} b}{3} \geqslant a b c^{2}
	\end{aligned}
	$$
	
	三式相加得
	$$
	a^{2} b^{2} c^{2}+\frac{1}{3}\left(a^{2} b+a^{2} c+b^{2} a+b^{2} c+c^{2} a+c^{2} b\right) \geqslant a^{2} b c+a b^{2} c+a b c^{2}
	$$
	
	只需证明
	
	$$
	\frac{2}{3}\left(a^{2} b+b^{2} c+c^{2} a\right) \geqslant \frac{1}{3}\left(a b^{2}+b c^{2}+c a^{2}\right)+a b c
	$$
	
	而
	
	$$
	\frac{1}{3}\left(a^{2} b+b^{2} c+c^{2} a\right) \geqslant a b c
	$$
	
	只需证明
	
	$$
	a^{2} b+b^{2} c+c^{2} a \geqslant a b^{2}+b c^{2}+c a^{2}
	$$
	
	而 $a^{2} b+b^{2} c+c^{2} a-\left(a b^{2}+b c^{2}+c a^{2}\right)=(a-b)(b-c)(a-c) \geqslant 0$,得证.
	

	\section*{2.3 平均值不等式在几何不等式中的应用}
	对于几何中出现的不等式证明, 常用的方法有: 几何方法、代数方法和三角方法, 当然, 我们不能将它们截然地分开, 常常是要综合地运用各种知识.如果采用代数方法证明几何命题, 那么, 灵活运用平均值不等式和柯西不等式, 对解决问题将有极大的帮助.

\begin{note}
	此题还有多种不同的证明方法, 感兴趣的读者不妨自己试试.
	上式等号当且仅当 $a x=b y=c z$ 时成立. 这就是说, $S_{1}=S_{2}=S_{3}=\frac{1}{3} S$使得 $x y z$ 取最大. 这时 $P$ 为 $\triangle A B C$ 的重心.
\end{note}

	\section*{2.4 平均值不等式的变形及应用}
	对于平均值不等式, 有各种不同的变形和推广, 由于这些问题可以包括在命题的证明和讨论中, 这里就不展开讨论了.

	对任意正数 $a_{1}, a_{2}, \cdots, a_{n}$, 由平均值不等式,得
	
	$$
	\sum_{i=1}^{n} a_{i} \cdot \sum_{i=1}^{n} \frac{1}{a_{i}} \geqslant n \sqrt[n]{a_{1} a_{2} \cdots a_{n}} \cdot n \sqrt[n]{\frac{1}{a_{1}} \cdot \frac{1}{a_{2}} \cdot \cdots \cdot \frac{1}{a_{n}}}=n^{2}
	$$
	
	从而
	
	$$
	\frac{\sum_{i=1}^{n} a_{i}}{n} \geqslant \frac{n}{\sum_{i=1}^{n} \frac{1}{a_{i}}}
	$$
	
	令 $H_{n}=\frac{n}{\frac{1}{a_{1}}+\frac{1}{a_{2}}+\cdots+\frac{1}{a_{n}}}$, 则称 $H_{n}$ 为 $n$ 个正实数 $a_{1}, a_{2}, \cdots, a_{n}$ 的调和平均值.
	
	由于 $\frac{x_{1}+x_{2}+\cdots+x_{n}}{n} \geqslant \sqrt[n]{x_{1} x_{2} \cdots x_{n}}$, 令 $x_{i}=\frac{1}{a_{i}}$, 则
	
	$$
	\frac{n}{\frac{1}{a_{1}}+\frac{1}{a_{2}}+\cdots+\frac{1}{a_{n}}} \leqslant \sqrt[n]{a_{1} a_{2} \cdots a_{n}} .
	$$
	
	即 $H_{n} \leqslant G_{n}$, 调和平均值不大于几何平均值.
	
	对任意实数 $a_{1}, a_{2}, \cdots, a_{n}$, 有
	
	$$
	n \sum_{i=1}^{n} a_{i}^{2}-\left(\sum_{i=1}^{n} a_{i}\right)^{2}=\sum_{1 \leqslant i<j \leqslant n}\left(a_{i}-a_{j}\right)^{2} \geqslant 0
	$$
	
	故得到
	
	$$
	\frac{a_{1}+a_{2}+\cdots+a_{n}}{n} \leqslant \sqrt{\frac{a_{1}^{2}+a_{2}^{2}+\cdots+a_{n}^{2}}{n}}
	$$
	
	令 $Q_{n}=\sqrt{\frac{a_{1}^{2}+a_{2}^{2}+\cdots+a_{n}^{2}}{n}}$, 称 $Q_{n}$ 为 $n$ 个实数 $a_{1}, a_{2}, \cdots, a_{n}$ 的平方平均值.
	
	所以, 对任意实数 $a_{1}, a_{2}, \cdots, a_{n}, A_{n} \leqslant Q_{n}$, 即算术平均值不大于平方平均值.
	
	于是, 对任意正实数 $a_{1}, a_{2}, \cdots, a_{n}$, 得到四个平均值有如下的关系
	
	$$
	H_{n} \leqslant G_{n} \leqslant A_{n} \leqslant Q_{n},
	$$
	
	且等式成立的充分必要条件是 $a_{1}=a_{2}=\cdots=a_{n}$.


	
	\section*{2. 5 带参数的平均值不等式}
	引进适当的参数, 是解决不等式问题的重要技巧.
	
	一般地, 当 $a_{i}>0, \lambda_{i}>0(i=1,2, \cdots, n)$, 且 $\prod_{i=1}^{n} \lambda_{i}=1$ 时, 我们有
	
	$$
	\frac{1}{n} \sum_{i=1}^{n} \lambda_{i} a_{i} \geqslant \sqrt[n]{a_{1} a_{2} \cdots a_{n}}
	$$

\begin{example}
	设 $a 、 b 、 c$ 是正实数, 且满足 $a^{2}+b^{2}+c^{2}=3$. 证明:
	
	$$
	\frac{1}{1+2 a b}+\frac{1}{1+2 b c}+\frac{1}{1+2 c a} \geqslant 1
	$$
\end{example}
\begin{proof}
	由算术平均值大于或等于几何平均值及算术平均值大于或等于调和平均值可得
	
	$$
	\begin{aligned}
	& \frac{1}{1+2 a b}+\frac{1}{1+2 b c}+\frac{1}{1+2 c a} \\
	\geqslant & \frac{1}{1+a^{2}+b^{2}}+\frac{1}{1+b^{2}+c^{2}}+\frac{1}{1+c^{2}+a^{2}} \\
	\geqslant & 3 \cdot \frac{3}{\left(1+a^{2}+b^{2}\right)+\left(1+b^{2}+c^{2}\right)+\left(1+c^{2}+a^{2}\right)} \\
	= & \frac{9}{3+2\left(a^{2}+b^{2}+c^{2}\right)}=1
	\end{aligned}
	$$
\end{proof}

	

	\section*{加权平均值不等式}
	设 $a_{i}>0, \alpha_{i}>0,1 \leqslant i \leqslant n, \sum_{i=1}^{n} \alpha_{i}=1$. 由加权琴生不等式和函数 $f(x)=\ln x, x \in \mathbf{R}_{+}$的凹凸性, 得到加权平均值不等式:
	
	$$
	a_{1}^{\alpha_{1}} a_{2}^{\alpha_{2}} \cdots a_{n}^{\alpha_{n}} \leqslant \alpha_{1} a_{1}+\alpha_{2} a_{2}+\cdots+\alpha_{n} a_{n} .
	$$
	
	由加权平均值不等式得
	
	卡尔松(Carlson)不等式 设 $a_{i j} \geqslant 0,1 \leqslant i \leqslant n, 1 \leqslant j \leqslant m, \alpha_{i} \geqslant 0,1$ $\leqslant i \leqslant m$ 满足 $\alpha_{1}+\alpha_{2}+\cdots+\alpha_{m}=1$, 则
	
	$$
	\prod_{j=1}^{m}\left(\sum_{i=1}^{n} a_{i j}\right)^{\alpha_{j}} \geqslant \sum_{i=1}^{n}\left(\prod_{j=1}^{m} a_{i j}^{\alpha_{j}}\right)
	$$
	
	事实上, 由齐次性, 不妨设 $a_{1 j}+a_{2 j}+\cdots+a_{n j}=1,1 \leqslant j \leqslant m$, 则需要证明
	
	$$
	\sum_{i=1}^{n} \prod_{j=1}^{m} a_{i j}^{a_{j}} \leqslant 1
	$$
	
	由加权平均值不等式得
	
	$$
	\prod_{j=1}^{m} a_{i j}^{a_{j}} \leqslant \sum_{j=1}^{m} \alpha_{j} a_{i j}, \quad 1 \leqslant i \leqslant n
	$$
	
	于是
	
	$$
	\sum_{i=1}^{n} \prod_{j=1}^{m} a_{i j}^{a_{j}} \leqslant \sum_{i=1}^{n} \sum_{j=1}^{m} \alpha_{j} a_{i j}=\sum_{j=1}^{m} \alpha_{j} \sum_{i=1}^{n} a_{i j}=\sum_{j=1}^{m} \alpha_{j}=1
	$$
	
	我们将在第三章中,进一步讨论卡尔松不等式.
\begin{note}
	此题还有多种不同的证明方法, 感兴趣的读者不妨自己试试.
\end{note}	

	\section*{证法四(归纳法)}
	众所周知, 归纳法是证明不等式的一种强有力和常用的方法, 这里, 利用归纳法证明一个更强的结论, 即
	
	$$
	\sum_{i=1}^{n}\left|a_{i} b_{i}\right| \leqslant \sqrt{\sum_{i=1}^{n} a_{i}^{2}} \sqrt{\sum_{i=1}^{n} b_{i}^{2}}
	$$
	
	(1) 当 $n=2$ 时,
	
	$$
	\begin{aligned}
	\left(a_{1} b_{1}+a_{2} b_{2}\right)^{2} & =a_{1}^{2} b_{1}^{2}+2 a_{1} b_{1} a_{2} b_{2}+a_{2}^{2} b_{2}^{2} \\
	& \leqslant a_{1}^{2} b_{1}^{2}+a_{1}^{2} b_{2}^{2}+a_{2}^{2} b_{1}^{2}+a_{2}^{2} b_{2}^{2} \\
	& =\left(a_{1}^{2}+a_{2}^{2}\right)\left(b_{1}^{2}+b_{2}^{2}\right)
	\end{aligned}
	$$
	
	且等号成立当且仅当 $\frac{a_{1}}{b_{1}}=\frac{a_{2}}{b_{2}}$, 命题成立.
	
	(2)假设当 $n=k$ 时命题成立, 那么对于 $n=k+1$, 由归纳假设,
	
	$$
	\begin{aligned}
	& \sqrt{\sum_{i=1}^{k+1} a_{i}^{2}} \cdot \sqrt{\sum_{i=1}^{k+1} b_{i}^{2}} \\
	= & \sqrt{\sum_{i=1}^{k} a_{i}^{2}+a_{k+1}^{2}} \cdot \sqrt{\sum_{i=1}^{k} b_{i}^{2}+b_{k+1}^{2}}
	\end{aligned}
	$$
	
	$$
	\begin{aligned}
	& \geqslant \sqrt{\sum_{i=1}^{k} a_{i}^{2}} \cdot \sqrt{\sum_{i=1}^{k} b_{i}^{2}}+\left|a_{k+1} b_{k+1}\right| \\
	& \geqslant \sum_{i=1}^{k}\left|a_{i} b_{i}\right|+\left|a_{k+1} b_{k+1}\right|=\sum_{i=1}^{k+1}\left|a_{i} b_{i}\right|
	\end{aligned}
	$$
	
	所以对一切的 $n$ 命题成立.
	
	不难得到等号成立的充分必要条件.
	
	证法五(归纳与综合法)
	
	(1) 当 $n=2$ 时,有
	
	$$
	\begin{aligned}
	\left(a_{1} b_{1}+a_{2} b_{2}\right)^{2} & =a_{1}^{2} b_{1}^{2}+2 a_{1} b_{1} a_{2} b_{2}+a_{2}^{2} b_{2}^{2} \\
	& \leqslant a_{1}^{2} b_{1}^{2}+a_{1}^{2} b_{2}^{2}+a_{2}^{2} b_{1}^{2}+a_{2}^{2} b_{2}^{2} \\
	& =\left(a_{1}^{2}+a_{2}^{2}\right)\left(b_{1}^{2}+b_{2}^{2}\right)
	\end{aligned}
	$$
	
	且等号成立当且仅当 $\frac{a_{1}}{b_{1}}=\frac{a_{2}}{b_{2}}$, 命题成立.
	
	(2)假设当 $n=k$ 时命题成立. 对于 $n=k+1$, 令 $A_{k}=a_{1}^{2}+a_{2}^{2}+\cdots+a_{k}^{2}$, $B_{k}=a_{1} b_{1}+a_{2} b_{2}+\cdots+a_{k} b_{k}, C_{k}=b_{1}^{2}+b_{2}^{2}+\cdots+b_{k}^{2}$, 则由归纳假设
	
	$$
	B_{k}^{2} \leqslant A_{k} C_{k}
	$$
	
	由于我们要证明
	
	$$
	\begin{aligned}
	& \left(a_{1} b_{1}+a_{2} b_{2}+\cdots+a_{k} b_{k}+a_{k+1} b_{k+1}\right)^{2} \\
	\leqslant & \left(a_{1}^{2}+a_{2}^{2}+\cdots+a_{k}^{2}+a_{k+1}^{2}\right)\left(b_{1}^{2}+b_{2}^{2}+\cdots+b_{k}^{2}+b_{k+1}^{2}\right)
	\end{aligned}
	$$
	
	等价于证明
	
	$$
	\begin{aligned}
	& \left(B_{k}+a_{k+1} b_{k+1}\right)^{2} \leqslant\left(A_{k}+a_{k+1}^{2}\right)\left(C_{k}+b_{k+1}^{2}\right) \\
	\Leftrightarrow & B_{k}^{2}+2 B_{k} a_{k+1} b_{k+1} \leqslant A_{k} C_{k}+A_{k} b_{k+1}^{2}+C_{k} a_{k+1}^{2} \\
	\Leftrightarrow & A_{k} C_{k}-B_{k}^{2}+A_{k} b_{k+1}^{2}+C_{k} a_{k+1}^{2}-2 B_{k} a_{k+1} b_{k+1} \geqslant 0 \\
	\Leftrightarrow & A_{k} C_{k}-B_{k}^{2}+\left(\sqrt{A_{k}} b_{k+1}-\sqrt{C_{k}} a_{k+1}\right)^{2}+2\left(\sqrt{A_{k}} \sqrt{C_{k}}-B_{k}\right) a_{k+1} b_{k+1} \geqslant 0
	\end{aligned}
	$$
	
	由归纳假设, 上述不等式成立, 且等式成立当且仅当 $\frac{a_{1}}{b_{1}}=\frac{a_{2}}{b_{2}}=\cdots=$ $\frac{a_{k+1}}{b_{k+1}}$, 故对任意 $n \geqslant 1$, 命题成立.
	
	证法六(归纳法和平均值不等式)
	
	(1) 当 $n=2$ 时, 有
	
	$$
	\begin{aligned}
	\left(a_{1} b_{1}+a_{2} b_{2}\right)^{2} & =a_{1}^{2} b_{1}^{2}+2 a_{1} b_{1} a_{2} b_{2}+a_{2}^{2} b_{2}^{2} \\
	& \leqslant a_{1}^{2} b_{1}^{2}+a_{1}^{2} b_{2}^{2}+a_{2}^{2} b_{1}^{2}+a_{2}^{2} b_{2}^{2} \\
	& =\left(a_{1}^{2}+a_{2}^{2}\right)\left(b_{1}^{2}+b_{2}^{2}\right)
	\end{aligned}
	$$
	
	即命题成立.
	
	(2)假设当 $n=k$ 时命题成立. 对于 $n=k+1$, 由于
	
	$$
	\begin{aligned}
	& \left(a_{1} b_{1}+a_{2} b_{2}+\cdots+a_{k} b_{k}+a_{k+1} b_{k+1}\right)^{2} \\
	= & \left(a_{1} b_{1}+a_{2} b_{2}+\cdots+a_{k} b_{k}\right)^{2} \\
	& +2\left(a_{1} b_{1}+a_{2} b_{2}+\cdots+a_{k} b_{k}\right) a_{k+1} b_{k+1}+a_{k+1}^{2} b_{k+1}^{2}
	\end{aligned}
	$$
	
	由平均值不等式, 得
	
	$$
	\begin{aligned}
	& 2\left(a_{1} b_{1}+a_{2} b_{2}+\cdots+a_{k} b_{k}\right) a_{k+1} b_{k+1} \\
	\leqslant & a_{k+1}^{2}\left(b_{1}^{2}+b_{2}^{2}+\cdots+b_{k}^{2}\right)+b_{k+1}^{2}\left(a_{1}^{2}+a_{2}^{2}+\cdots+a_{k}^{2}\right)
	\end{aligned}
	$$
	
	由归纳假设, 得
	
	$$
	\begin{aligned}
	& \left(a_{1} b_{1}+a_{2} b_{2}+\cdots+a_{k} b_{k}+a_{k+1} b_{k+1}\right)^{2} \\
	= & \left(a_{1} b_{1}+a_{2} b_{2}+\cdots+a_{k} b_{k}\right)^{2}+2\left(a_{1} b_{1}+a_{2} b_{2}+\cdots+a_{k} b_{k}\right) a_{k+1} b_{k+1}+a_{k+1}^{2} b_{k+1}^{2} \\
	\leqslant & \left(a_{1} b_{1}+a_{2} b_{2}+\cdots+a_{k} b_{k}\right)^{2}+a_{k+1}^{2}\left(b_{1}^{2}+b_{2}^{2}+\cdots+b_{k}^{2}\right) \\
	& +b_{k+1}^{2}\left(a_{1}^{2}+a_{2}^{2}+\cdots+a_{k}^{2}\right)+a_{k+1}^{2} b_{k+1}^{2} \\
	= & \left(a_{1}^{2}+a_{2}^{2}+\cdots+a_{k+1}^{2}\right)\left(b_{1}^{2}+b_{2}^{2}+\cdots+b_{k+1}^{2}\right)
	\end{aligned}
	$$
	
	结合平均值不等式等号成立的条件, 不难得到柯西不等式等号成立的充要条件, 故命题成立.
	
	注 (1)在上述的证明中, 我们反复利用了平均值不等式.
	
	(2)上述几种证明均用归纳法, 由于证明过程中,对表达式的处理的不同, 所以难易程度也就不同.
	
	证法七(利用排序不等式) 由于
	
	$$
	\sum_{i=1}^{n} a_{i}^{2} \sum_{i=1}^{n} b_{i}^{2}=a_{1}^{2} \sum_{i=1}^{n} b_{i}^{2}+a_{2}^{2} \sum_{i=1}^{n} b_{i}^{2}+\cdots+a_{n}^{2} \sum_{i=1}^{n} b_{i}^{2}
	$$
	
	则有
	
	$$
	\begin{gathered}
	a_{1} b_{1}, \cdots, a_{1} b_{n}, a_{2} b_{1}, \cdots, a_{2} b_{n}, \cdots, a_{n} b_{1}, \cdots, a_{n} b_{n}, \\
	a_{1} b_{1}, \cdots, a_{1} b_{n}, a_{2} b_{1}, \cdots, a_{2} b_{n}, \cdots, a_{n} b_{1}, \cdots, a_{n} b_{n}
	\end{gathered}
	$$
	
	两行相同, 共 $n^{2}$ 列, 且是同序的.
	
	另一方面, 有乱序
	
	$$
	\begin{aligned}
	& a_{1} b_{1}, \cdots, a_{1} b_{n}, a_{2} b_{1}, \cdots, a_{2} b_{n}, \cdots, a_{n} b_{1}, \cdots, a_{n} b_{n} \\
	& a_{1} b_{1}, \cdots, a_{n} b_{1}, a_{1} b_{2}, \cdots, a_{n} b_{2}, \cdots, a_{1} b_{n}, \cdots, a_{n} b_{n}
	\end{aligned}
	$$
	
	两行, 共 $n^{2}$ 列, 且两行为乱序, 其乘积为
	
	$$
	\sum_{i=1}^{n} \sum_{j=1}^{n}\left(a_{i} b_{j}\right)\left(a_{j} b_{i}\right)=\left(\sum_{i=1}^{n} a_{i} b_{i}\right)^{2}
	$$
	
	由引理 1 , 得
	
	$$
	\left(\sum_{i=1}^{n} a_{i} b_{i}\right)^{2} \leqslant \sum_{i=1}^{n} a_{i}^{2} \sum_{i=1}^{n} b_{i}^{2}
	$$
	
	当且仅当 $\frac{a_{1}}{b_{1}}=\frac{a_{2}}{b_{2}}=\cdots=\frac{a_{n}}{b_{n}}$ 时等号成立.
	
	\section*{证法八(利用参数平均值不等式)}
	由于对 $m \in \mathbf{R}_{+}$, 得
	
	$$
	a_{i} b_{i} \leqslant \frac{1}{2}\left(m^{2} a_{i}^{2}+\frac{b_{i}^{2}}{m^{2}}\right)
	$$
	
	令 $m^{2}=\sqrt{\frac{\sum_{i=1}^{n} b_{i}^{2}}{\sum_{i=1}^{n} a_{i}^{2}}}$, 则
	
	%\begin{center}
	%\includegraphics[max width=\textwidth]{2024_05_22_4ff05a14ba9ad07b725fg-096}
	%\end{center}
	
	从而
	
	$$
	\sum_{i=1}^{n}\left|a_{i} b_{i}\right| \leqslant \frac{1}{2}\left(\sqrt{\frac{\sum_{i=1}^{n} b_{i}^{2}}{\sum_{i=1}^{n} a_{i}^{2}}} \sum_{i=1}^{n} a_{i}^{2}+\sqrt{\frac{\sum_{i=1}^{n} a_{i}^{2}}{\sum_{i=1}^{n} b_{i}^{2}}} \sum_{i=1}^{n} b_{i}^{2}\right)
	$$
	
	故
	
	$$
	\begin{aligned}
	\sum_{i=1}^{n} a_{i} b_{i} & \leqslant \sum_{i=1}^{n}\left|a_{i} b_{i}\right| \leqslant \frac{1}{2}\left(\sqrt{\sum_{i=1}^{n} b_{i}^{2} \sum_{i=1}^{n} a_{i}^{2}}+\sqrt{\sum_{i=1}^{n} a_{i}^{2} \sum_{i=1}^{n} b_{i}^{2}}\right) \\
	& =\left(\sum_{i=1}^{n} a_{i}^{2}\right)^{\frac{1}{2}}\left(\sum_{i=1}^{n} b_{i}^{2}\right)^{\frac{1}{2}}
	\end{aligned}
	$$
	
	注 利用含参数的基本不等式来证明不等式, 具有较高的灵活性和技巧, 为了让大家熟悉这种证明方法, 后面, 我们将专门介绍.
	
	\section*{证法九(利用行列式性质)}
	$$
	\begin{aligned}
	S & =\sum_{i=1}^{n} a_{i}^{2} \cdot \sum_{i=1}^{n} b_{i}^{2}-\left(\sum_{i=1}^{n} a_{i} b_{i}\right)^{2} \\
	& =\left|\begin{array}{cc}
	a_{1}^{2}+a_{2}^{2}+\cdots+a_{n}^{2} & a_{1} b_{1}+a_{2} b_{2}+\cdots+a_{n} b_{n} \\
	a_{1} b_{1}+a_{2} b_{2}+\cdots+a_{n} b_{n} & b_{1}^{2}+b_{2}^{2}+\cdots+b_{n}^{2}
	\end{array}\right| \\
	& =\sum_{i=1}^{n}\left|\begin{array}{cc}
	a_{1}^{2}+a_{2}^{2}+\cdots+a_{n}^{2} & a_{i} b_{i} \\
	a_{1} b_{1}+a_{2} b_{2}+\cdots+a_{n} b_{n} & b_{i}^{2}
	\end{array}\right| \\
	& =\sum_{i=1}^{n} \sum_{j=1}^{n}\left|\begin{array}{cc}
	a_{j}^{2} & a_{i} b_{i} \\
	a_{j} b_{j} & b_{i}^{2}
	\end{array}\right| \\
	& =\sum_{i=1}^{n} \sum_{j=1}^{n} a_{j} b_{i}\left|\begin{array}{cc}
	a_{j} & a_{i} \\
	b_{j} & b_{i}
	\end{array}\right|
	\end{aligned}
	$$
	
	又
	
	$$
	\begin{aligned}
	S & =\sum_{j=1}^{n} \sum_{i=1}^{n} a_{i} b_{j}\left|\begin{array}{cc}
	a_{i} & a_{j} \\
	b_{i} & b_{j}
	\end{array}\right| \\
	& =\sum_{j=1}^{n} \sum_{i=1}^{n} a_{i} b_{j}(-1)\left|\begin{array}{cc}
	a_{j} & a_{i} \\
	b_{j} & b_{i}
	\end{array}\right| \\
	& =\sum_{i=1}^{n} \sum_{j=1}^{n} a_{i} b_{j}(-1)\left|\begin{array}{cc}
	a_{j} & a_{i} \\
	b_{j} & b_{i}
	\end{array}\right|
	\end{aligned}
	$$
	
	所以
	
	$$
	\begin{aligned}
	2 S & =\sum_{i=1}^{n} \sum_{j=1}^{n}\left(a_{j} b_{i}-a_{i} b_{j}\right)\left|\begin{array}{ll}
	a_{j} & a_{i} \\
	b_{j} & b_{i}
	\end{array}\right| \\
	& =\sum_{i=1}^{n} \sum_{j=1}^{n}\left(a_{j} b_{i}-a_{i} b_{j}\right)^{2} \geqslant 0
	\end{aligned}
	$$
	
	即 $S \geqslant 0$, 故不等式成立.
	
	证法十(利用拉格朗日 (Lagrange)恒等式)
	
	对 $a_{1}, a_{2}, \cdots, a_{n}$ 与 $b_{1}, b_{2}, \cdots, b_{n}$, 我们有如下的拉格朗日恒等式
	
	$$
	\sum_{i=1}^{n} a_{i}^{2} \cdot \sum_{i=1}^{n} b_{i}^{2}-\left(\sum_{i=1}^{n} a_{i} b_{i}\right)^{2}=\sum_{1 \leqslant i<j \leqslant n}\left(a_{i} b_{j}-a_{j} b_{i}\right)^{2} \geqslant 0
	$$
	
	不难看出命题成立.
	
	注 实际上,证法十是证法九的一种特殊情况,但在证明不等式中,拉格朗日恒等式往往作为已知的结果使用, 此外, 拉格朗日恒等式也可以用其他方法来证明.
	
	证法十一 (内积法)
	
	令 $\boldsymbol{\alpha}=\left(a_{1}, a_{2}, \cdots, a_{n}\right), \boldsymbol{\beta}=\left(b_{1}, b_{2}, \cdots, b_{n}\right)$, 对任意实数 $t$, 我们有
	
	$$
	0 \leqslant(\boldsymbol{\alpha}+t \boldsymbol{\beta}, \boldsymbol{\alpha}+t \boldsymbol{\beta})=(\boldsymbol{\alpha}, \boldsymbol{\alpha})+2(\boldsymbol{\alpha}, \boldsymbol{\beta}) t+(\boldsymbol{\beta}, \boldsymbol{\beta}) t^{2}
	$$
	
	于是
	
	$$
	\sum_{i=1}^{n} a_{i}^{2}+2 t \sum_{i=1}^{n} a_{i} b_{i}+\left(\sum_{i=1}^{n} b_{i}^{2}\right) t^{2} \geqslant 0
	$$
	
	由 $t$ 的任意性, 得
	
	$$
	4\left[\left(\sum_{i=1}^{n} a_{i} b_{i}\right)^{2}-\sum_{i=1}^{n} a_{i}^{2} \sum_{i=1}^{n} b_{i}^{2}\right] \leqslant 0
	$$
	
	故命题成立.
	
	证法十二 (向量法)
	
	令 $\boldsymbol{\alpha}=\left(a_{1}, a_{2}, \cdots, a_{n}\right), \boldsymbol{\beta}=\left(b_{1}, b_{2}, \cdots, b_{n}\right)$, 则对向量 $\boldsymbol{\alpha} 、 \boldsymbol{\beta}$, 我们有
	
	$$
	\cos \langle\boldsymbol{\alpha}, \boldsymbol{\beta}\rangle=\frac{\boldsymbol{\alpha} \cdot \boldsymbol{\beta}}{|\boldsymbol{\alpha}| \cdot|\boldsymbol{\beta}|}
	$$
	
	从而
	
	$$
	\frac{\boldsymbol{\alpha} \cdot \boldsymbol{\beta}}{|\boldsymbol{\alpha}| \cdot|\boldsymbol{\beta}|}=\cos (\boldsymbol{\alpha}, \boldsymbol{\beta}) \leqslant 1
	$$
	
	由 $\boldsymbol{\alpha} \cdot \boldsymbol{\beta}=a_{1} b_{1}+a_{2} b_{2}+\cdots+a_{n} b_{n},|\boldsymbol{\alpha}|^{2}=\sum_{i=1}^{n} a_{i}^{2},|\boldsymbol{\beta}|^{2}=\sum_{i=1}^{n} b_{i}^{2}$, 且等号成立当且仅当 $\cos \langle\boldsymbol{\alpha}, \boldsymbol{\beta}\rangle=1$, 即 $\boldsymbol{\alpha}$ 与 $\boldsymbol{\beta}$ 平行. 故命题成立.
	
	注 内积法和向量法有着密切的联系, 内积亦称为点积, 其定义为: 对任意两个向量 $\boldsymbol{\alpha} 、 \boldsymbol{\beta}$, 它们的内积为
	
	$$
	(\boldsymbol{\alpha}, \boldsymbol{\beta})=\boldsymbol{\alpha} \cdot \boldsymbol{\beta}=\sum_{i=1}^{n} a_{i} b_{i}
	$$
	
	容易验证,对任意向量 $\boldsymbol{\alpha} \neq \overrightarrow{0} ,$
	
	$$
	(\boldsymbol{\alpha}, \boldsymbol{\alpha})=\sum_{i=1}^{n} a_{i}^{2}>0
	$$
	
	在证法十一中, 就是利用了这个性质.
	
	证法十三(构造单调数列)
	
	构造数列 $\left\{S_{n}\right\}$, 其中
	
	$$
	S_{n}=\left(a_{1} b_{1}+a_{2} b_{2}+\cdots+a_{n} b_{n}\right)^{2}-\left(a_{1}^{2}+a_{2}^{2}+\cdots+a_{n}^{2}\right)\left(b_{1}^{2}+b_{2}^{2}+\cdots+b_{n}^{2}\right),
	$$
	
	则
	
	$$
	S_{1}=\left(a_{1} b_{1}\right)^{2}-a_{1}^{2} b_{1}^{2}=0
	$$
	
	$$
	\begin{aligned}
	S_{n+1}-S_{n}= & {\left[\left(a_{1} b_{1}+a_{2} b_{2}+\cdots+a_{n+1} b_{n+1}\right)^{2}\right.} \\
	& \left.-\left(a_{1}^{2}+a_{2}^{2}+\cdots+a_{n+1}^{2}\right)\left(b_{1}^{2}+b_{2}^{2}+\cdots+b_{n+1}^{2}\right)\right] \\
	& -\left[\left(a_{1} b_{1}+a_{2} b_{2}+\cdots+a_{n} b_{n}\right)^{2}-\left(a_{1}^{2}+a_{2}^{2}+\cdots+a_{n}^{2}\right) \cdot\right. \\
	& \left.\left(b_{1}^{2}+b_{2}^{2}+\cdots+b_{n}^{2}\right)\right]
	\end{aligned}
	$$
	
	$$
	\begin{aligned}
	= & 2\left(a_{1} b_{1}+a_{2} b_{2}+\cdots+a_{n} b_{n}\right) a_{n+1} b_{n+1}+a_{n+1}^{2} b_{n+1}^{2} \\
	& -\left(a_{1}^{2}+a_{2}^{2}+\cdots+a_{n}^{2}\right) b_{n+1}^{2} \\
	& -a_{n+1}^{2}\left(b_{1}^{2}+b_{2}^{2}+\cdots+b_{n}^{2}\right)-a_{n+1}^{2} b_{n+1}^{2} \\
	= & -\left[\left(a_{1} b_{n+1}-b_{1} a_{n+1}\right)^{2}+\left(a_{2} b_{n+1}-b_{2} a_{n+1}\right)^{2}\right. \\
	& \left.+\cdots+\left(a_{n} b_{n+1}-b_{n} a_{n+1}\right)^{2}\right] \leqslant 0
	\end{aligned}
	$$
	
	即 $S_{n+1} \leqslant S_{n}$, 所以数列 $\left\{S_{n}\right\}$ 单调减少, 从而对一切 $n \geqslant 1$, 有 $S_{n} \leqslant S_{1}=0$, 故命题成立.
	
	\section*{证法十四 (二次函数的判别式)}
	令 $A_{n}=a_{1}^{2}+a_{2}^{2}+\cdots+a_{n}^{2}, B_{n}=a_{1} b_{1}+a_{2} b_{2}+\cdots+a_{n} b_{n}, C_{n}=b_{1}^{2}+$ $b_{2}^{2}+\cdots+b_{n}^{2}$, 作二次函数 $f(x)=A_{n} x^{2}+2 B_{n} x+C_{n}=\sum_{i=1}^{n}\left(a_{i} x+b_{i}\right)^{2} \geqslant 0$, 且 $f(x)=0$ 的充要条件是 $\frac{a_{i}}{b_{i}}=\lambda$ 为常数.
	
	由于 $A_{n}>0, f(x) \geqslant 0$, 则它的判别式 $\Delta=4\left(B_{n}^{2}-A_{n} C_{n}\right) \leqslant 0$, 即
	
	$$
	B_{n}^{2} \leqslant A_{n} C_{n} .
	$$
	
	等号成立当且仅当 $\frac{a_{1}}{b_{1}}=\frac{a_{2}}{b_{2}}=\cdots=\frac{a_{n}}{b_{n}}$ 为常数.
	
	用类似的方法,可以证明下列不等式:
	
	Aczel 不等式 设 $a_{i}, b_{i} \in \mathbf{R}, 1 \leqslant i \leqslant n$, 满足 $a_{1}^{2}-a_{2}^{2}-\cdots-a_{n}^{2}>0$ 或 $b_{1}^{2}-b_{2}^{2}-\cdots-b_{n}^{2}>0$, 求证:
	
	$$
	\left(a_{1} b_{1}-a_{2} b_{2}-\cdots-a_{n} b_{n}\right)^{2} \geqslant\left(a_{1}^{2}-a_{2}^{2}-\cdots-a_{n}^{2}\right)\left(b_{1}^{2}-b_{2}^{2}-\cdots-b_{n}^{2}\right)
	$$
	
	证明 按上述记号, 不妨设 $A_{n}>0$, 考虑函数
	
	$$
	g(x)=A_{n} x^{2}+2 B_{n} x+C_{n}=\left(a_{1} x+b_{1}\right)^{2}-\sum_{i=2}^{n}\left(a_{i} x+b_{i}\right)^{2}
	$$
	
	则存在 $x_{0}=-\frac{b_{1}}{a_{1}}, a_{1} \neq 0$, 使得 $g\left(x_{0}\right) \leqslant 0$, 由于二次函数开口向上, 从而存在 $x_{1}$ 充分大, 使得 $g\left(x_{1}\right)>0$. 则它的判别式 $\Delta=4\left(B_{n}^{2}-A_{n} C_{n}\right) \geqslant 0$, 即
	
	$$
	B_{n}^{2} \geqslant A_{n} C_{n}
	$$
	
	等号成立当且仅当 $\frac{a_{1}}{b_{1}}=\frac{a_{2}}{b_{2}}=\cdots=\frac{a_{n}}{b_{n}}$ 为常数.
	
	\section*{证法十五(凹函数方法)}
	令 $A_{n}=a_{1}^{2}+a_{2}^{2}+\cdots+a_{n}^{2}, B_{n}=a_{1} b_{1}+a_{2} b_{2}+\cdots+a_{n} b_{n}, C_{n}=b_{1}^{2}+$\\
	$b_{2}^{2}+\cdots+b_{n}^{2}$, 且不妨假设 $a_{i}>0, b_{i}>0$, 由前面的引理 4, 对凹函数 $f(x)=$ $\ln x$, 有
	
	$$
	\begin{aligned}
	& \frac{1}{2} \ln \frac{a_{i}^{2}}{A_{n}}+\frac{1}{2} \ln \frac{b_{i}^{2}}{C_{n}} \leqslant \ln \frac{\frac{a_{i}^{2}}{A_{n}}+\frac{b_{i}^{2}}{C_{n}}}{2} \\
	\Leftrightarrow & \ln \left(\frac{a_{i}^{2}}{A_{n}} \frac{b_{i}^{2}}{C_{n}}\right)^{\frac{1}{2}} \leqslant \ln \frac{\frac{a_{i}^{2}}{A_{n}}+\frac{b_{i}^{2}}{C_{n}}}{2} \\
	\Leftrightarrow & \left(\frac{a_{i}^{2}}{A_{n}} \frac{b_{i}^{2}}{C_{n}}\right)^{\frac{1}{2}} \leqslant \frac{\frac{a_{i}^{2}}{A_{n}}+\frac{b_{i}^{2}}{C_{n}}}{2}
	\end{aligned}
	$$
	
	于是
	
	$$
	\begin{aligned}
	& \sum_{i=1}^{n} \frac{a_{i}}{A_{n}^{\frac{1}{2}}} \frac{b_{i}}{C_{n}^{\frac{1}{2}}} \leqslant \frac{1}{2}\left(\frac{1}{A_{n}} \sum_{i=1}^{n} a_{i}^{2}+\frac{1}{C_{n}} \sum_{i=1}^{n} b_{i}^{2}\right)=1 \\
	\Leftrightarrow & \sum_{i=1}^{n} a_{i} b_{i} \leqslant A_{n}^{\frac{1}{2}} C_{n}^{\frac{1}{2}}
	\end{aligned}
	$$
	
	不难得到, 等式成立的充要条件是 $\frac{a_{1}}{b_{1}}=\frac{a_{2}}{b_{2}}=\cdots=\frac{a_{n}}{b_{n}}$.
	
	另外, 如果令 $x=\frac{a_{i}^{2}}{A_{n}}, y=\frac{b_{i}^{2}}{C_{n}}, p=q=2$, 则由 Young 不等式, 容易得到柯西不等式.
	
	\section*{3.2 柯西不等式的变形和推广}
	变形 $\mathbf{1}$ 设 $a_{i} \in \mathbf{R}, b_{i}>0(i=1,2, \cdots, n)$, 则
	
	$$
	\sum_{i=1}^{n} \frac{a_{i}^{2}}{b_{i}} \geqslant \frac{\left(\sum_{i=1}^{n} a_{i}\right)^{2}}{\sum_{i=1}^{n} b_{i}}
	$$
	
	等号成立的充分必要条件是 $a_{i}=\lambda b_{i}(i=1,2, \cdots, n)$.
	
	变形 2 设 $a_{i}, b_{i}(i=1,2, \cdots, n)$ 同号且不为零, 则
	
	$$
	\sum_{i=1}^{n} \frac{a_{i}}{b_{i}} \geqslant \frac{\left(\sum_{i=1}^{n} a_{i}\right)^{2}}{\sum_{i=1}^{n} a_{i} b_{i}}
	$$
	
	等号成立的充分必要条件是 $b_{1}=b_{2}=\cdots=b_{n}$.\\
	柯西不等式的推广为赫尔德 (Hölder)不等式,即
	
	赫尔德不等式 设 $a_{i}>0, b_{i}>0(i=1,2, \cdots, n), p>0, q>0$, 满足 $\frac{1}{p}+\frac{1}{q}=1$, 则
	
	$$
	\sum_{i=1}^{n} a_{i} b_{i} \leqslant\left(\sum_{i=1}^{n} a_{i}^{p}\right)^{\frac{1}{p}}\left(\sum_{i=1}^{n} b_{i}^{q}\right)^{\frac{1}{q}}
	$$
	
	等号成立的充分必要条件是 $a_{i}^{p}=\lambda b_{i}^{q}(i=1,2, \cdots, n, \lambda>0)$.
	
	证明 由 Young 不等式, 得
	
	$$
	\begin{aligned}
	& \sum_{i=1}^{n}\left[\frac{a_{i}^{p}}{\sum_{i=1}^{n} a_{i}^{p}}\right]^{\frac{1}{p}} \cdot\left[\frac{b_{i}^{q}}{\sum_{i=1}^{n} b_{i}^{q}}\right]^{\frac{1}{q}} \\
	\leqslant & \sum_{i=1}^{n}\left[\frac{1}{p} \frac{a_{i}^{p}}{\sum_{i=1}^{n} a_{i}^{p}}\right]+\sum_{i=1}^{n}\left[\frac{1}{q} \frac{b_{i}^{q}}{\sum_{i=1}^{n} b_{i}^{q}}\right] \\
	= & \frac{1}{p}+\frac{1}{q}=1
	\end{aligned}
	$$
	
	等号成立的充分必要条件是
	
	$$
	\frac{a_{i}^{p}}{\sum_{i=1}^{n} a_{i}^{p}}=\frac{b_{i}^{q}}{\sum_{i=1}^{n} b_{i}^{q}}
	$$
	
	即 $a_{i}^{p}=\lambda b_{i}^{q}(i=1,2, \cdots, n, \lambda>0)$.
	
	赫尔德不等式也可以变形为
	
	$$
	\sum_{i=1}^{n} \frac{a_{i}^{m+1}}{b_{i}^{m}} \geqslant \frac{\left(\sum_{i=1}^{n} a_{i}\right)^{m+1}}{\left(\sum_{i=1}^{n} b_{i}\right)^{m}}
	$$
	
	等号成立的充分必要条件是 $a_{i}=\lambda b_{i}(i=1,2, \cdots, n)$. 其中 $a_{i}>0, b_{i}>0$ $(i=1,2, \cdots, n), m>0$ 或 $m<-1$.
	
	证明 当 $m>0$ 时, 由赫尔德不等式, 得
	
	$$
	\begin{aligned}
	\sum_{i=1}^{n} a_{i} & =\sum_{i=1}^{n}\left(\frac{a_{i}}{b_{i}^{\frac{m}{m+1}}}\right) \cdot b_{i}^{\frac{m}{m+1}} \\
	& \leqslant\left[\sum_{i=1}^{n}\left(\frac{a_{i}}{b_{i}^{\frac{m}{m+1}}}\right)^{m+1}\right]^{\frac{1}{m+1}} \cdot\left[\sum_{i=1}^{n}\left(b_{i}^{\frac{m}{m+1}}\right)^{\frac{m+1}{m}}\right]^{\frac{m}{m+1}}
	\end{aligned}
	$$
	
	$$
	=\left(\sum_{i=1}^{n} \frac{a_{i}^{m+1}}{b_{i}^{m}}\right)^{\frac{1}{m+1}} \cdot\left(\sum_{i=1}^{n} b_{i}\right)^{\frac{m}{m+1}}
	$$
	
	故
	
	$$
	\sum_{i=1}^{n} \frac{a_{i}^{m+1}}{b_{i}^{m}} \geqslant \frac{\left(\sum_{i=1}^{n} a_{i}\right)^{m+1}}{\left(\sum_{i=1}^{n} b_{i}\right)^{m}}
	$$
	
	当 $m<-1$ 时, $-(m+1)>0$, 对于数组 $\left(b_{1}, b_{2}, \cdots, b_{n}\right)$ 和 $\left(a_{1}, a_{2}, \cdots\right.$, $\left.a_{n}\right)$ 有
	
	即
	
	$$
	\begin{gathered}
	\sum_{i=1}^{n} \frac{b_{i}^{-(m+1)+1}}{a_{i}^{-(m+1)}} \geqslant \frac{\left(\sum_{i=1}^{n} b_{i}\right)^{-(m+1)+1}}{\left(\sum_{i=1}^{n} a_{i}\right)^{-(m+1)}} \\
	\sum_{i=1}^{n} \frac{a_{i}^{m+1}}{b_{i}^{m}} \geqslant \frac{\left(\sum_{i=1}^{n} a_{i}\right)^{m+1}}{\left(\sum_{i=1}^{n} b_{i}\right)^{m}}
	\end{gathered}
	$$
	
	等号成立当且仅当 $\left(\frac{a_{i}}{b_{i}^{m+1}}\right)^{m+1}=\mu\left(b_{i}^{\frac{m}{m+1}}\right)^{\frac{m+1}{m}}$, 即 $a_{i}=\lambda b_{i}(i=1$, $2, \cdots, n)$.
	
	由赫尔德不等式可以推出另一个重要的不等式, 即
	
	闵可夫斯基(Minkowski)不等式 对 $a_{i}, b_{i} \in \mathbf{R}_{+}, 1 \leqslant i \leqslant n, k>1$, 则
	
	$$
	\left[\sum_{i=1}^{n}\left(a_{i}+b_{i}\right)^{k}\right]^{\frac{1}{k}} \leqslant\left(\sum_{i=1}^{n} a_{i}^{k}\right)^{\frac{1}{k}}+\left(\sum_{i=1}^{n} b_{i}^{k}\right)^{\frac{1}{k}}
	$$
	
	当且仅当 $\frac{a_{1}}{b_{1}}=\frac{a_{2}}{b_{2}}=\cdots=\frac{a_{n}}{b_{n}}$ 时, 等号成立.
	
	证明 $\quad$ 由赫尔德不等式, 得
	
	$$
	\begin{aligned}
	\sum_{i=1}^{n}\left(a_{i}+b_{i}\right)^{k} & =\sum_{i=1}^{n} a_{i}\left(a_{i}+b_{i}\right)^{k-1}+\sum_{i=1}^{n} b_{i}\left(a_{i}+b_{i}\right)^{k-1} \\
	& \leqslant\left(\sum_{i=1}^{n} a_{i}^{k}\right)^{\frac{1}{k}}\left[\sum_{i=1}^{n}\left(a_{i}+b_{i}\right)^{k}\right]^{\frac{k-1}{k}}+\left(\sum_{i=1}^{n} b_{i}^{k}\right)^{\frac{1}{k}}\left[\sum_{i=1}^{n}\left(a_{i}+b_{i}\right)^{k}\right]^{\frac{k-1}{k}}
	\end{aligned}
	$$
	
	所以
	
	$$
	\left[\sum_{i=1}^{n}\left(a_{i}+b_{i}\right)^{)^{1}}\right]^{\frac{1}{k}} \leqslant\left(\sum_{i=1}^{n} a_{i}^{k}\right)^{\frac{1}{k}}+\left(\sum_{i=1}^{n} b_{i}^{k}\right)^{\frac{1}{k}}
	$$
	
	不难知, 当且仅当 $\frac{a_{1}}{b_{1}}=\frac{a_{2}}{b_{2}}=\cdots=\frac{a_{n}}{b_{n}}$ 时, 等号成立.

\begin{example}
	设 $a 、 b 、 c$ 为某三角形三边之长. 令
	
	$$
	A=\sum_{\mathrm{cyc}} \frac{a^{2}+b c}{b+c}, B=\sum_{\mathrm{cyc}} \frac{1}{\sqrt{(a+b-c)(b+c-a)}}
	$$
	
	其中 “ $\sum_{\text {cyc }} ”$ 表示循环求和. 证明: $A B \geqslant 9$.
\end{example}
\begin{proof}
	设 $a=y+z, b=x+z, c=x+y, x, y, z \in \mathbf{R}_{+}$, 则
	
	$$
	\begin{aligned}
	B= & \sum_{\text {cyc }} \frac{1}{2 \sqrt{x y}}, \\
	A= & \sum_{\text {cyc }} \frac{x^{2}+y^{2}+z^{2}+x y+2 x+3 y z}{2 x+y+z} \\
	A B= & \left(\sum_{\text {cyc }} \frac{1}{2 \sqrt{y z}}\right)\left(\sum_{\text {cyc }} \frac{\sum_{\text {cyc }} x^{2}+\sum_{\text {cyc }} x y+2 y z}{2 x+y+z}\right) \\
	\geqslant & {\left[\sqrt{\sum_{\text {cyc }} \frac{1}{2 \sqrt{y z}} \frac{\sum_{\text {cyc }} x^{2}+\sum_{\text {cyc }} x y+2 y z}{2 x+y+z}}\right]^{2} } \\
	& \frac{1}{2 \sqrt{y z}} \cdot \frac{x^{2}+y^{2}+z^{2}+x y+z x+3 y z}{2 x+y+z} \geqslant 1 \\
	& \Leftrightarrow\left(x^{2}+y^{2}+z^{2}+x y+z x+3 y z\right)^{2} \geqslant 4 y z(2 x+y+z)^{2}
	\end{aligned}
	$$
	
	由于
	
	$$
	\Leftrightarrow \sum_{\mathrm{cyc}} x^{4}+3 \sum_{\mathrm{cyc}} x^{2} y^{2}+2 \sum_{\mathrm{sym}} x^{3} y \geqslant 8 x y^{2} z+8 x^{2} z+8 x y z^{2}
	$$
	
	由幂平均值不等式得
	
	$$
	\begin{aligned}
	x^{4}+y^{4}+z^{4} & \geqslant \frac{1}{27}(x+y+z)^{4} \\
	& \geqslant x y z(x+y+z) \\
	& =x y^{2} z+x^{2} y z+x y z^{2}
	\end{aligned}
	$$
	
	再由平均值不等式得
	
	$$
	\begin{aligned}
	& 3 x^{2} y^{2}+3 x^{2} z^{2}+3 y^{2} z^{2} \geqslant 3\left(x y^{2} z+x^{2} y z+x y z^{2}\right) \\
	& \quad x^{3} y+x y^{3}+x^{3} z+x z^{3}+y^{3} z+y z^{3}-2 x y z(x+y+z) \\
	& =\left(x^{3} y+y z^{3}-x y z^{2}-x^{2} y z\right)+\left(y^{3} z+x^{3} z-x^{2} y z-x y^{2} z\right) \\
	& \quad+\left(z^{3} x+x y^{3}-x y^{2} z-x y z^{2}\right) \\
	& =y(x+z)(x-z)^{2}+z(x+y)(x-y)^{2}+x(y+z)(y-z)^{2} \geqslant 0
	\end{aligned}
	$$
	
	从而
	
	$$
	A B \geqslant\left(\sqrt{\sum_{\text {cyc }} 1}\right)^{2}=3^{2}=9
	$$
	
	故命题成立.
\end{proof}	

	\section*{3. 1 柯西不等式及其证明}
	柯西(Cauchy)不等式 设 $a_{1}, a_{2}, \cdots, a_{n}$ 及 $b_{1}, b_{2}, \cdots, b_{n}$ 为任意实数, 则
	
	$$
	\left(a_{1} b_{1}+a_{2} b_{2}+\cdots+a_{n} b_{n}\right)^{2} \leqslant\left(a_{1}^{2}+a_{2}^{2}+\cdots+a_{n}^{2}\right)\left(b_{1}^{2}+b_{2}^{2}+\cdots+b_{n}^{2}\right)
	$$
	
	当且仅当 $\frac{a_{1}}{b_{1}}=\frac{a_{2}}{b_{2}}=\cdots=\frac{a_{n}}{b_{n}}$ (规定 $a_{i}=0$ 时, $b_{i}=0$ ) 时等号成立.
	
	柯西不等式的证明方法很多, 这里我们选择其中一些简单和具有一定技巧的证明.
	
	证法一
	
	不妨假设 $A_{n}=\sum_{i=1}^{n} a_{i}^{2} \neq 0, C_{n}=\sum_{i=1}^{n} b_{i}^{2} \neq 0$, 令 $x_{i}=\frac{a_{i}}{\sqrt{A_{n}}}, y_{i}=\frac{b_{i}}{\sqrt{C_{n}}}$,则
	
	$$
	\sum_{i=1}^{n} x_{i}^{2}=\sum_{i=1}^{n} y_{i}^{2}=1
	$$
	
	则原不等式等价于
	
	$$
	x_{1} y_{1}+x_{2} y_{2}+\cdots+x_{n} y_{n} \leqslant 1
	$$
	
	即
	
	$$
	2\left(x_{1} y_{1}+x_{2} y_{2}+\cdots+x_{n} y_{n}\right) \leqslant x_{1}^{2}+x_{2}^{2}+\cdots+x_{n}^{2}+y_{1}^{2}+y_{2}^{2}+\cdots+y_{n}^{2}
	$$
	
	又等价于
	
	$$
	\left(x_{1}-y_{1}\right)^{2}+\left(x_{2}-y_{2}\right)^{2}+\cdots+\left(x_{n}-y_{n}\right)^{2} \geqslant 0
	$$
	
	这个不等式显然成立, 且等号成立的充要条件为 $x_{i}=y_{i}(i=1,2, \cdots$, $n)$, 从而原不等式成立, 且等号成立的充要条件是
	
	$$
	b_{i}=k a_{i}\left(k=\frac{\sqrt{C_{n}}}{\sqrt{A_{n}}}\right)
	$$
	
	\section*{证法二 (比值法)}
	按上述证明方法和记号, 不妨假设 $A_{n} \neq 0, C_{n} \neq 0$, 令 $x_{i}=\frac{\left|a_{i}\right|}{\sqrt{A_{n}}}, y_{i}=$ $\frac{\left|b_{i}\right|}{\sqrt{C_{n}}}$, 则
	
	$$
	\sum_{i=1}^{n} x_{i}^{2}=\sum_{i=1}^{n} y_{i}^{2}=1
	$$
	
	由于 $\frac{\left|\sum_{i=1}^{n} a_{i} b_{i}\right|}{\sqrt{A_{n}} \cdot \sqrt{C_{n}}} \leqslant \sum_{i=1}^{n} x_{i} y_{i} \leqslant \sum_{i=1}^{n} \frac{1}{2}\left(x_{i}^{2}+y_{i}^{2}\right)$
	
	$$
	=\frac{1}{2}\left(\sum_{i=1}^{n} x_{i}^{2}+\sum_{i=1}^{n} y_{i}^{2}\right)=1
	$$
	
	且等号成立当且仅当
	
	$$
	\begin{aligned}
	\left|\sum_{i=1}^{n} a_{i} b_{i}\right| & =\sum_{i=1}^{n}\left|a_{i} b_{i}\right| \\
	\frac{a_{i}^{2}}{\sum_{i=1}^{n} a_{i}^{2}} & =\frac{b_{i}^{2}}{\sum_{i=1}^{n} b_{i}^{2}}
	\end{aligned}
	$$
	
	由第一个条件表明 $a_{1} b_{1}, a_{2} b_{2}, \cdots, a_{n} b_{n}$ 同号. 第二个条件成立的充分必要条件是 $\frac{a_{i}^{2}}{b_{i}^{2}}=\frac{A_{n}}{C_{n}}$, 即 $\frac{\left|a_{i}\right|}{\left|b_{i}\right|}$ 为常数.
	
	由于 $a_{1} b_{1}, a_{2} b_{2}, \cdots, a_{n} b_{n}$ 同号, 从而命题成立.
	
	证法三 (比值法, 类似证法二)
	
	令 $A_{n}=a_{1}^{2}+a_{2}^{2}+\cdots+a_{n}^{2}, B_{n}=a_{1} b_{1}+a_{2} b_{2}+\cdots+a_{n} b_{n}, C_{n}=b_{1}^{2}+$ $b_{2}^{2}+\cdots+b_{n}^{2}$, 则
	
	$$
	\begin{aligned}
	\frac{A_{n} C_{n}}{B_{n}^{2}}+1 & =\sum_{i=1}^{n} \frac{a_{i}^{2} C_{n}}{B_{n}^{2}}+\sum_{i=1}^{n} \frac{b_{i}^{2}}{C_{n}} \\
	& =\sum_{i=1}^{n}\left(\frac{a_{i}^{2} C_{n}}{B_{n}^{2}}+\frac{b_{i}^{2}}{C_{n}}\right) \\
	& \geqslant \sum_{i=1}^{n} 2 \cdot \frac{a_{i} b_{i}}{B_{n}}=2
	\end{aligned}
	$$
	
	所以
	
	$$
	\frac{A_{n} C_{n}}{B_{n}^{2}}+1 \geqslant 2
	$$
	
	即
	
	$$
	B_{n}^{2} \leqslant A_{n} C_{n}
	$$
	
	等号成立当且仅当 $\frac{a_{i}}{b_{i}}(i=1,2, \cdots, n)$ 为一个常数.

	(1) 这两个证明方法比较简单, 但是对于不等式的证明来讲, 怎样人手是十分重要的. 比值法是证明不等式的一种常用、基本的方法.
	
	(2)上述两种方法也称为标准化方法, 这个方法可以简化许多不等式的证明. 在前面我们也使用过. 如为了证明 $G_{n} \leqslant A_{n}$, 令 $y_{i}=\frac{a_{i}}{G_{n}}$, 则问题化为在条件 $y_{1} y_{2} \cdots y_{n}=1\left(y_{i}>0\right)$ 下, 证明 $\sum_{i=1}^{n} y_{i} \geqslant n$.
	
	\section*{证法四(归纳法)}
	众所周知, 归纳法是证明不等式的一种强有力和常用的方法, 这里, 利用归纳法证明一个更强的结论, 即
	
	$$
	\sum_{i=1}^{n}\left|a_{i} b_{i}\right| \leqslant \sqrt{\sum_{i=1}^{n} a_{i}^{2}} \sqrt{\sum_{i=1}^{n} b_{i}^{2}}
	$$
	
	(1) 当 $n=2$ 时,
	
	$$
	\begin{aligned}
	\left(a_{1} b_{1}+a_{2} b_{2}\right)^{2} & =a_{1}^{2} b_{1}^{2}+2 a_{1} b_{1} a_{2} b_{2}+a_{2}^{2} b_{2}^{2} \\
	& \leqslant a_{1}^{2} b_{1}^{2}+a_{1}^{2} b_{2}^{2}+a_{2}^{2} b_{1}^{2}+a_{2}^{2} b_{2}^{2} \\
	& =\left(a_{1}^{2}+a_{2}^{2}\right)\left(b_{1}^{2}+b_{2}^{2}\right)
	\end{aligned}
	$$
	
	且等号成立当且仅当 $\frac{a_{1}}{b_{1}}=\frac{a_{2}}{b_{2}}$, 命题成立.
	
	(2)假设当 $n=k$ 时命题成立, 那么对于 $n=k+1$, 由归纳假设,
	
	$$
	\begin{aligned}
	& \sqrt{\sum_{i=1}^{k+1} a_{i}^{2}} \cdot \sqrt{\sum_{i=1}^{k+1} b_{i}^{2}} \\
	= & \sqrt{\sum_{i=1}^{k} a_{i}^{2}+a_{k+1}^{2}} \cdot \sqrt{\sum_{i=1}^{k} b_{i}^{2}+b_{k+1}^{2}}
	\end{aligned}
	$$
	
	$$
	\begin{aligned}
	& \geqslant \sqrt{\sum_{i=1}^{k} a_{i}^{2}} \cdot \sqrt{\sum_{i=1}^{k} b_{i}^{2}}+\left|a_{k+1} b_{k+1}\right| \\
	& \geqslant \sum_{i=1}^{k}\left|a_{i} b_{i}\right|+\left|a_{k+1} b_{k+1}\right|=\sum_{i=1}^{k+1}\left|a_{i} b_{i}\right|
	\end{aligned}
	$$
	
	所以对一切的 $n$ 命题成立.
	
	不难得到等号成立的充分必要条件.
	
	证法五(归纳与综合法)
	
	(1) 当 $n=2$ 时,有
	
	$$
	\begin{aligned}
	\left(a_{1} b_{1}+a_{2} b_{2}\right)^{2} & =a_{1}^{2} b_{1}^{2}+2 a_{1} b_{1} a_{2} b_{2}+a_{2}^{2} b_{2}^{2} \\
	& \leqslant a_{1}^{2} b_{1}^{2}+a_{1}^{2} b_{2}^{2}+a_{2}^{2} b_{1}^{2}+a_{2}^{2} b_{2}^{2} \\
	& =\left(a_{1}^{2}+a_{2}^{2}\right)\left(b_{1}^{2}+b_{2}^{2}\right)
	\end{aligned}
	$$
	
	且等号成立当且仅当 $\frac{a_{1}}{b_{1}}=\frac{a_{2}}{b_{2}}$, 命题成立.
	
	(2)假设当 $n=k$ 时命题成立. 对于 $n=k+1$, 令 $A_{k}=a_{1}^{2}+a_{2}^{2}+\cdots+a_{k}^{2}$, $B_{k}=a_{1} b_{1}+a_{2} b_{2}+\cdots+a_{k} b_{k}, C_{k}=b_{1}^{2}+b_{2}^{2}+\cdots+b_{k}^{2}$, 则由归纳假设
	
	$$
	B_{k}^{2} \leqslant A_{k} C_{k}
	$$
	
	由于我们要证明
	
	$$
	\begin{aligned}
	& \left(a_{1} b_{1}+a_{2} b_{2}+\cdots+a_{k} b_{k}+a_{k+1} b_{k+1}\right)^{2} \\
	\leqslant & \left(a_{1}^{2}+a_{2}^{2}+\cdots+a_{k}^{2}+a_{k+1}^{2}\right)\left(b_{1}^{2}+b_{2}^{2}+\cdots+b_{k}^{2}+b_{k+1}^{2}\right)
	\end{aligned}
	$$
	
	等价于证明
	
	$$
	\begin{aligned}
	& \left(B_{k}+a_{k+1} b_{k+1}\right)^{2} \leqslant\left(A_{k}+a_{k+1}^{2}\right)\left(C_{k}+b_{k+1}^{2}\right) \\
	\Leftrightarrow & B_{k}^{2}+2 B_{k} a_{k+1} b_{k+1} \leqslant A_{k} C_{k}+A_{k} b_{k+1}^{2}+C_{k} a_{k+1}^{2} \\
	\Leftrightarrow & A_{k} C_{k}-B_{k}^{2}+A_{k} b_{k+1}^{2}+C_{k} a_{k+1}^{2}-2 B_{k} a_{k+1} b_{k+1} \geqslant 0 \\
	\Leftrightarrow & A_{k} C_{k}-B_{k}^{2}+\left(\sqrt{A_{k}} b_{k+1}-\sqrt{C_{k}} a_{k+1}\right)^{2}+2\left(\sqrt{A_{k}} \sqrt{C_{k}}-B_{k}\right) a_{k+1} b_{k+1} \geqslant 0
	\end{aligned}
	$$
	
	由归纳假设, 上述不等式成立, 且等式成立当且仅当 $\frac{a_{1}}{b_{1}}=\frac{a_{2}}{b_{2}}=\cdots=$ $\frac{a_{k+1}}{b_{k+1}}$, 故对任意 $n \geqslant 1$, 命题成立.
	
	证法六(归纳法和平均值不等式)
	
	(1) 当 $n=2$ 时, 有
	
	$$
	\begin{aligned}
	\left(a_{1} b_{1}+a_{2} b_{2}\right)^{2} & =a_{1}^{2} b_{1}^{2}+2 a_{1} b_{1} a_{2} b_{2}+a_{2}^{2} b_{2}^{2} \\
	& \leqslant a_{1}^{2} b_{1}^{2}+a_{1}^{2} b_{2}^{2}+a_{2}^{2} b_{1}^{2}+a_{2}^{2} b_{2}^{2} \\
	& =\left(a_{1}^{2}+a_{2}^{2}\right)\left(b_{1}^{2}+b_{2}^{2}\right)
	\end{aligned}
	$$
	
	即命题成立.
	
	(2)假设当 $n=k$ 时命题成立. 对于 $n=k+1$, 由于
	
	$$
	\begin{aligned}
	& \left(a_{1} b_{1}+a_{2} b_{2}+\cdots+a_{k} b_{k}+a_{k+1} b_{k+1}\right)^{2} \\
	= & \left(a_{1} b_{1}+a_{2} b_{2}+\cdots+a_{k} b_{k}\right)^{2} \\
	& +2\left(a_{1} b_{1}+a_{2} b_{2}+\cdots+a_{k} b_{k}\right) a_{k+1} b_{k+1}+a_{k+1}^{2} b_{k+1}^{2}
	\end{aligned}
	$$
	
	由平均值不等式, 得
	
	$$
	\begin{aligned}
	& 2\left(a_{1} b_{1}+a_{2} b_{2}+\cdots+a_{k} b_{k}\right) a_{k+1} b_{k+1} \\
	\leqslant & a_{k+1}^{2}\left(b_{1}^{2}+b_{2}^{2}+\cdots+b_{k}^{2}\right)+b_{k+1}^{2}\left(a_{1}^{2}+a_{2}^{2}+\cdots+a_{k}^{2}\right)
	\end{aligned}
	$$
	
	由归纳假设, 得
	
	$$
	\begin{aligned}
	& \left(a_{1} b_{1}+a_{2} b_{2}+\cdots+a_{k} b_{k}+a_{k+1} b_{k+1}\right)^{2} \\
	= & \left(a_{1} b_{1}+a_{2} b_{2}+\cdots+a_{k} b_{k}\right)^{2}+2\left(a_{1} b_{1}+a_{2} b_{2}+\cdots+a_{k} b_{k}\right) a_{k+1} b_{k+1}+a_{k+1}^{2} b_{k+1}^{2} \\
	\leqslant & \left(a_{1} b_{1}+a_{2} b_{2}+\cdots+a_{k} b_{k}\right)^{2}+a_{k+1}^{2}\left(b_{1}^{2}+b_{2}^{2}+\cdots+b_{k}^{2}\right) \\
	& +b_{k+1}^{2}\left(a_{1}^{2}+a_{2}^{2}+\cdots+a_{k}^{2}\right)+a_{k+1}^{2} b_{k+1}^{2} \\
	= & \left(a_{1}^{2}+a_{2}^{2}+\cdots+a_{k+1}^{2}\right)\left(b_{1}^{2}+b_{2}^{2}+\cdots+b_{k+1}^{2}\right)
	\end{aligned}
	$$
	
	结合平均值不等式等号成立的条件, 不难得到柯西不等式等号成立的充要条件, 故命题成立.
	
	注 (1)在上述的证明中, 我们反复利用了平均值不等式.
	
	(2)上述几种证明均用归纳法, 由于证明过程中,对表达式的处理的不同, 所以难易程度也就不同.
	
	证法七(利用排序不等式) 由于
	
	$$
	\sum_{i=1}^{n} a_{i}^{2} \sum_{i=1}^{n} b_{i}^{2}=a_{1}^{2} \sum_{i=1}^{n} b_{i}^{2}+a_{2}^{2} \sum_{i=1}^{n} b_{i}^{2}+\cdots+a_{n}^{2} \sum_{i=1}^{n} b_{i}^{2}
	$$
	
	则有
	
	$$
	\begin{gathered}
	a_{1} b_{1}, \cdots, a_{1} b_{n}, a_{2} b_{1}, \cdots, a_{2} b_{n}, \cdots, a_{n} b_{1}, \cdots, a_{n} b_{n}, \\
	a_{1} b_{1}, \cdots, a_{1} b_{n}, a_{2} b_{1}, \cdots, a_{2} b_{n}, \cdots, a_{n} b_{1}, \cdots, a_{n} b_{n}
	\end{gathered}
	$$
	
	两行相同, 共 $n^{2}$ 列, 且是同序的.
	
	另一方面, 有乱序
	
	$$
	\begin{aligned}
	& a_{1} b_{1}, \cdots, a_{1} b_{n}, a_{2} b_{1}, \cdots, a_{2} b_{n}, \cdots, a_{n} b_{1}, \cdots, a_{n} b_{n} \\
	& a_{1} b_{1}, \cdots, a_{n} b_{1}, a_{1} b_{2}, \cdots, a_{n} b_{2}, \cdots, a_{1} b_{n}, \cdots, a_{n} b_{n}
	\end{aligned}
	$$
	
	两行, 共 $n^{2}$ 列, 且两行为乱序, 其乘积为
	
	$$
	\sum_{i=1}^{n} \sum_{j=1}^{n}\left(a_{i} b_{j}\right)\left(a_{j} b_{i}\right)=\left(\sum_{i=1}^{n} a_{i} b_{i}\right)^{2}
	$$
	
	由引理 1 , 得
	
	$$
	\left(\sum_{i=1}^{n} a_{i} b_{i}\right)^{2} \leqslant \sum_{i=1}^{n} a_{i}^{2} \sum_{i=1}^{n} b_{i}^{2}
	$$
	
	当且仅当 $\frac{a_{1}}{b_{1}}=\frac{a_{2}}{b_{2}}=\cdots=\frac{a_{n}}{b_{n}}$ 时等号成立.
	
	\section*{证法八(利用参数平均值不等式)}
	由于对 $m \in \mathbf{R}_{+}$, 得
	
	$$
	a_{i} b_{i} \leqslant \frac{1}{2}\left(m^{2} a_{i}^{2}+\frac{b_{i}^{2}}{m^{2}}\right)
	$$
	
	令 $m^{2}=\sqrt{\frac{\sum_{i=1}^{n} b_{i}^{2}}{\sum_{i=1}^{n} a_{i}^{2}}}$, 则
	
	%\begin{center}
	%\includegraphics[max width=\textwidth]{2024_05_22_4ff05a14ba9ad07b725fg-096}
	%\end{center}
	
	从而
	
	$$
	\sum_{i=1}^{n}\left|a_{i} b_{i}\right| \leqslant \frac{1}{2}\left(\sqrt{\frac{\sum_{i=1}^{n} b_{i}^{2}}{\sum_{i=1}^{n} a_{i}^{2}}} \sum_{i=1}^{n} a_{i}^{2}+\sqrt{\frac{\sum_{i=1}^{n} a_{i}^{2}}{\sum_{i=1}^{n} b_{i}^{2}}} \sum_{i=1}^{n} b_{i}^{2}\right)
	$$
	
	故
	
	$$
	\begin{aligned}
	\sum_{i=1}^{n} a_{i} b_{i} & \leqslant \sum_{i=1}^{n}\left|a_{i} b_{i}\right| \leqslant \frac{1}{2}\left(\sqrt{\sum_{i=1}^{n} b_{i}^{2} \sum_{i=1}^{n} a_{i}^{2}}+\sqrt{\sum_{i=1}^{n} a_{i}^{2} \sum_{i=1}^{n} b_{i}^{2}}\right) \\
	& =\left(\sum_{i=1}^{n} a_{i}^{2}\right)^{\frac{1}{2}}\left(\sum_{i=1}^{n} b_{i}^{2}\right)^{\frac{1}{2}}
	\end{aligned}
	$$
	
	注 利用含参数的基本不等式来证明不等式, 具有较高的灵活性和技巧, 为了让大家熟悉这种证明方法, 后面, 我们将专门介绍.
	
	\section*{证法九(利用行列式性质)}
	$$
	\begin{aligned}
	S & =\sum_{i=1}^{n} a_{i}^{2} \cdot \sum_{i=1}^{n} b_{i}^{2}-\left(\sum_{i=1}^{n} a_{i} b_{i}\right)^{2} \\
	& =\left|\begin{array}{cc}
	a_{1}^{2}+a_{2}^{2}+\cdots+a_{n}^{2} & a_{1} b_{1}+a_{2} b_{2}+\cdots+a_{n} b_{n} \\
	a_{1} b_{1}+a_{2} b_{2}+\cdots+a_{n} b_{n} & b_{1}^{2}+b_{2}^{2}+\cdots+b_{n}^{2}
	\end{array}\right| \\
	& =\sum_{i=1}^{n}\left|\begin{array}{cc}
	a_{1}^{2}+a_{2}^{2}+\cdots+a_{n}^{2} & a_{i} b_{i} \\
	a_{1} b_{1}+a_{2} b_{2}+\cdots+a_{n} b_{n} & b_{i}^{2}
	\end{array}\right| \\
	& =\sum_{i=1}^{n} \sum_{j=1}^{n}\left|\begin{array}{cc}
	a_{j}^{2} & a_{i} b_{i} \\
	a_{j} b_{j} & b_{i}^{2}
	\end{array}\right| \\
	& =\sum_{i=1}^{n} \sum_{j=1}^{n} a_{j} b_{i}\left|\begin{array}{cc}
	a_{j} & a_{i} \\
	b_{j} & b_{i}
	\end{array}\right|
	\end{aligned}
	$$
	
	又
	
	$$
	\begin{aligned}
	S & =\sum_{j=1}^{n} \sum_{i=1}^{n} a_{i} b_{j}\left|\begin{array}{cc}
	a_{i} & a_{j} \\
	b_{i} & b_{j}
	\end{array}\right| \\
	& =\sum_{j=1}^{n} \sum_{i=1}^{n} a_{i} b_{j}(-1)\left|\begin{array}{cc}
	a_{j} & a_{i} \\
	b_{j} & b_{i}
	\end{array}\right| \\
	& =\sum_{i=1}^{n} \sum_{j=1}^{n} a_{i} b_{j}(-1)\left|\begin{array}{cc}
	a_{j} & a_{i} \\
	b_{j} & b_{i}
	\end{array}\right|
	\end{aligned}
	$$
	
	所以
	
	$$
	\begin{aligned}
	2 S & =\sum_{i=1}^{n} \sum_{j=1}^{n}\left(a_{j} b_{i}-a_{i} b_{j}\right)\left|\begin{array}{ll}
	a_{j} & a_{i} \\
	b_{j} & b_{i}
	\end{array}\right| \\
	& =\sum_{i=1}^{n} \sum_{j=1}^{n}\left(a_{j} b_{i}-a_{i} b_{j}\right)^{2} \geqslant 0
	\end{aligned}
	$$
	
	即 $S \geqslant 0$, 故不等式成立.
	
	证法十(利用拉格朗日 (Lagrange)恒等式)
	
	对 $a_{1}, a_{2}, \cdots, a_{n}$ 与 $b_{1}, b_{2}, \cdots, b_{n}$, 我们有如下的拉格朗日恒等式
	
	$$
	\sum_{i=1}^{n} a_{i}^{2} \cdot \sum_{i=1}^{n} b_{i}^{2}-\left(\sum_{i=1}^{n} a_{i} b_{i}\right)^{2}=\sum_{1 \leqslant i<j \leqslant n}\left(a_{i} b_{j}-a_{j} b_{i}\right)^{2} \geqslant 0
	$$
	
	不难看出命题成立.
	
	注 实际上,证法十是证法九的一种特殊情况,但在证明不等式中,拉格朗日恒等式往往作为已知的结果使用, 此外, 拉格朗日恒等式也可以用其他方法来证明.
	
	证法十一 (内积法)
	
	令 $\boldsymbol{\alpha}=\left(a_{1}, a_{2}, \cdots, a_{n}\right), \boldsymbol{\beta}=\left(b_{1}, b_{2}, \cdots, b_{n}\right)$, 对任意实数 $t$, 我们有
	
	$$
	0 \leqslant(\boldsymbol{\alpha}+t \boldsymbol{\beta}, \boldsymbol{\alpha}+t \boldsymbol{\beta})=(\boldsymbol{\alpha}, \boldsymbol{\alpha})+2(\boldsymbol{\alpha}, \boldsymbol{\beta}) t+(\boldsymbol{\beta}, \boldsymbol{\beta}) t^{2}
	$$
	
	于是
	
	$$
	\sum_{i=1}^{n} a_{i}^{2}+2 t \sum_{i=1}^{n} a_{i} b_{i}+\left(\sum_{i=1}^{n} b_{i}^{2}\right) t^{2} \geqslant 0
	$$
	
	由 $t$ 的任意性, 得
	
	$$
	4\left[\left(\sum_{i=1}^{n} a_{i} b_{i}\right)^{2}-\sum_{i=1}^{n} a_{i}^{2} \sum_{i=1}^{n} b_{i}^{2}\right] \leqslant 0
	$$
	
	故命题成立.
	
	证法十二 (向量法)
	
	令 $\boldsymbol{\alpha}=\left(a_{1}, a_{2}, \cdots, a_{n}\right), \boldsymbol{\beta}=\left(b_{1}, b_{2}, \cdots, b_{n}\right)$, 则对向量 $\boldsymbol{\alpha} 、 \boldsymbol{\beta}$, 我们有
	
	$$
	\cos \langle\boldsymbol{\alpha}, \boldsymbol{\beta}\rangle=\frac{\boldsymbol{\alpha} \cdot \boldsymbol{\beta}}{|\boldsymbol{\alpha}| \cdot|\boldsymbol{\beta}|}
	$$
	
	从而
	
	$$
	\frac{\boldsymbol{\alpha} \cdot \boldsymbol{\beta}}{|\boldsymbol{\alpha}| \cdot|\boldsymbol{\beta}|}=\cos (\boldsymbol{\alpha}, \boldsymbol{\beta}) \leqslant 1
	$$
	
	由 $\boldsymbol{\alpha} \cdot \boldsymbol{\beta}=a_{1} b_{1}+a_{2} b_{2}+\cdots+a_{n} b_{n},|\boldsymbol{\alpha}|^{2}=\sum_{i=1}^{n} a_{i}^{2},|\boldsymbol{\beta}|^{2}=\sum_{i=1}^{n} b_{i}^{2}$, 且等号成立当且仅当 $\cos \langle\boldsymbol{\alpha}, \boldsymbol{\beta}\rangle=1$, 即 $\boldsymbol{\alpha}$ 与 $\boldsymbol{\beta}$ 平行. 故命题成立.
	
	注 内积法和向量法有着密切的联系, 内积亦称为点积, 其定义为: 对任意两个向量 $\boldsymbol{\alpha} 、 \boldsymbol{\beta}$, 它们的内积为
	
	$$
	(\boldsymbol{\alpha}, \boldsymbol{\beta})=\boldsymbol{\alpha} \cdot \boldsymbol{\beta}=\sum_{i=1}^{n} a_{i} b_{i}
	$$
	
	容易验证,对任意向量 $\boldsymbol{\alpha} \neq \overrightarrow{0} ,$
	
	$$
	(\boldsymbol{\alpha}, \boldsymbol{\alpha})=\sum_{i=1}^{n} a_{i}^{2}>0
	$$
	
	在证法十一中, 就是利用了这个性质.
	
	证法十三(构造单调数列)
	
	构造数列 $\left\{S_{n}\right\}$, 其中
	
	$$
	S_{n}=\left(a_{1} b_{1}+a_{2} b_{2}+\cdots+a_{n} b_{n}\right)^{2}-\left(a_{1}^{2}+a_{2}^{2}+\cdots+a_{n}^{2}\right)\left(b_{1}^{2}+b_{2}^{2}+\cdots+b_{n}^{2}\right),
	$$
	
	则
	
	$$
	S_{1}=\left(a_{1} b_{1}\right)^{2}-a_{1}^{2} b_{1}^{2}=0
	$$
	
	$$
	\begin{aligned}
	S_{n+1}-S_{n}= & {\left[\left(a_{1} b_{1}+a_{2} b_{2}+\cdots+a_{n+1} b_{n+1}\right)^{2}\right.} \\
	& \left.-\left(a_{1}^{2}+a_{2}^{2}+\cdots+a_{n+1}^{2}\right)\left(b_{1}^{2}+b_{2}^{2}+\cdots+b_{n+1}^{2}\right)\right] \\
	& -\left[\left(a_{1} b_{1}+a_{2} b_{2}+\cdots+a_{n} b_{n}\right)^{2}-\left(a_{1}^{2}+a_{2}^{2}+\cdots+a_{n}^{2}\right) \cdot\right. \\
	& \left.\left(b_{1}^{2}+b_{2}^{2}+\cdots+b_{n}^{2}\right)\right]
	\end{aligned}
	$$
	
	$$
	\begin{aligned}
	= & 2\left(a_{1} b_{1}+a_{2} b_{2}+\cdots+a_{n} b_{n}\right) a_{n+1} b_{n+1}+a_{n+1}^{2} b_{n+1}^{2} \\
	& -\left(a_{1}^{2}+a_{2}^{2}+\cdots+a_{n}^{2}\right) b_{n+1}^{2} \\
	& -a_{n+1}^{2}\left(b_{1}^{2}+b_{2}^{2}+\cdots+b_{n}^{2}\right)-a_{n+1}^{2} b_{n+1}^{2} \\
	= & -\left[\left(a_{1} b_{n+1}-b_{1} a_{n+1}\right)^{2}+\left(a_{2} b_{n+1}-b_{2} a_{n+1}\right)^{2}\right. \\
	& \left.+\cdots+\left(a_{n} b_{n+1}-b_{n} a_{n+1}\right)^{2}\right] \leqslant 0
	\end{aligned}
	$$
	
	即 $S_{n+1} \leqslant S_{n}$, 所以数列 $\left\{S_{n}\right\}$ 单调减少, 从而对一切 $n \geqslant 1$, 有 $S_{n} \leqslant S_{1}=0$, 故命题成立.
	
	\section*{证法十四 (二次函数的判别式)}
	令 $A_{n}=a_{1}^{2}+a_{2}^{2}+\cdots+a_{n}^{2}, B_{n}=a_{1} b_{1}+a_{2} b_{2}+\cdots+a_{n} b_{n}, C_{n}=b_{1}^{2}+$ $b_{2}^{2}+\cdots+b_{n}^{2}$, 作二次函数 $f(x)=A_{n} x^{2}+2 B_{n} x+C_{n}=\sum_{i=1}^{n}\left(a_{i} x+b_{i}\right)^{2} \geqslant 0$, 且 $f(x)=0$ 的充要条件是 $\frac{a_{i}}{b_{i}}=\lambda$ 为常数.
	
	由于 $A_{n}>0, f(x) \geqslant 0$, 则它的判别式 $\Delta=4\left(B_{n}^{2}-A_{n} C_{n}\right) \leqslant 0$, 即
	
	$$
	B_{n}^{2} \leqslant A_{n} C_{n} .
	$$
	
	等号成立当且仅当 $\frac{a_{1}}{b_{1}}=\frac{a_{2}}{b_{2}}=\cdots=\frac{a_{n}}{b_{n}}$ 为常数.
	
	用类似的方法,可以证明下列不等式:
	
	Aczel 不等式 设 $a_{i}, b_{i} \in \mathbf{R}, 1 \leqslant i \leqslant n$, 满足 $a_{1}^{2}-a_{2}^{2}-\cdots-a_{n}^{2}>0$ 或 $b_{1}^{2}-b_{2}^{2}-\cdots-b_{n}^{2}>0$, 求证:
	
	$$
	\left(a_{1} b_{1}-a_{2} b_{2}-\cdots-a_{n} b_{n}\right)^{2} \geqslant\left(a_{1}^{2}-a_{2}^{2}-\cdots-a_{n}^{2}\right)\left(b_{1}^{2}-b_{2}^{2}-\cdots-b_{n}^{2}\right)
	$$
	
	证明 按上述记号, 不妨设 $A_{n}>0$, 考虑函数
	
	$$
	g(x)=A_{n} x^{2}+2 B_{n} x+C_{n}=\left(a_{1} x+b_{1}\right)^{2}-\sum_{i=2}^{n}\left(a_{i} x+b_{i}\right)^{2}
	$$
	
	则存在 $x_{0}=-\frac{b_{1}}{a_{1}}, a_{1} \neq 0$, 使得 $g\left(x_{0}\right) \leqslant 0$, 由于二次函数开口向上, 从而存在 $x_{1}$ 充分大, 使得 $g\left(x_{1}\right)>0$. 则它的判别式 $\Delta=4\left(B_{n}^{2}-A_{n} C_{n}\right) \geqslant 0$, 即
	
	$$
	B_{n}^{2} \geqslant A_{n} C_{n}
	$$
	
	等号成立当且仅当 $\frac{a_{1}}{b_{1}}=\frac{a_{2}}{b_{2}}=\cdots=\frac{a_{n}}{b_{n}}$ 为常数.
	
	\section*{证法十五(凹函数方法)}
	令 $A_{n}=a_{1}^{2}+a_{2}^{2}+\cdots+a_{n}^{2}, B_{n}=a_{1} b_{1}+a_{2} b_{2}+\cdots+a_{n} b_{n}, C_{n}=b_{1}^{2}+$\\
	$b_{2}^{2}+\cdots+b_{n}^{2}$, 且不妨假设 $a_{i}>0, b_{i}>0$, 由前面的引理 4, 对凹函数 $f(x)=$ $\ln x$, 有
	
	$$
	\begin{aligned}
	& \frac{1}{2} \ln \frac{a_{i}^{2}}{A_{n}}+\frac{1}{2} \ln \frac{b_{i}^{2}}{C_{n}} \leqslant \ln \frac{\frac{a_{i}^{2}}{A_{n}}+\frac{b_{i}^{2}}{C_{n}}}{2} \\
	\Leftrightarrow & \ln \left(\frac{a_{i}^{2}}{A_{n}} \frac{b_{i}^{2}}{C_{n}}\right)^{\frac{1}{2}} \leqslant \ln \frac{\frac{a_{i}^{2}}{A_{n}}+\frac{b_{i}^{2}}{C_{n}}}{2} \\
	\Leftrightarrow & \left(\frac{a_{i}^{2}}{A_{n}} \frac{b_{i}^{2}}{C_{n}}\right)^{\frac{1}{2}} \leqslant \frac{\frac{a_{i}^{2}}{A_{n}}+\frac{b_{i}^{2}}{C_{n}}}{2}
	\end{aligned}
	$$
	
	于是
	
	$$
	\begin{aligned}
	& \sum_{i=1}^{n} \frac{a_{i}}{A_{n}^{\frac{1}{2}}} \frac{b_{i}}{C_{n}^{\frac{1}{2}}} \leqslant \frac{1}{2}\left(\frac{1}{A_{n}} \sum_{i=1}^{n} a_{i}^{2}+\frac{1}{C_{n}} \sum_{i=1}^{n} b_{i}^{2}\right)=1 \\
	\Leftrightarrow & \sum_{i=1}^{n} a_{i} b_{i} \leqslant A_{n}^{\frac{1}{2}} C_{n}^{\frac{1}{2}}
	\end{aligned}
	$$
	
	不难得到, 等式成立的充要条件是 $\frac{a_{1}}{b_{1}}=\frac{a_{2}}{b_{2}}=\cdots=\frac{a_{n}}{b_{n}}$.
	
	另外, 如果令 $x=\frac{a_{i}^{2}}{A_{n}}, y=\frac{b_{i}^{2}}{C_{n}}, p=q=2$, 则由 Young 不等式, 容易得到柯西不等式.
	
	\section*{3.2 柯西不等式的变形和推广}
	变形 $\mathbf{1}$ 设 $a_{i} \in \mathbf{R}, b_{i}>0(i=1,2, \cdots, n)$, 则
	
	$$
	\sum_{i=1}^{n} \frac{a_{i}^{2}}{b_{i}} \geqslant \frac{\left(\sum_{i=1}^{n} a_{i}\right)^{2}}{\sum_{i=1}^{n} b_{i}}
	$$
	
	等号成立的充分必要条件是 $a_{i}=\lambda b_{i}(i=1,2, \cdots, n)$.
	
	变形 2 设 $a_{i}, b_{i}(i=1,2, \cdots, n)$ 同号且不为零, 则
	
	$$
	\sum_{i=1}^{n} \frac{a_{i}}{b_{i}} \geqslant \frac{\left(\sum_{i=1}^{n} a_{i}\right)^{2}}{\sum_{i=1}^{n} a_{i} b_{i}}
	$$
	
	等号成立的充分必要条件是 $b_{1}=b_{2}=\cdots=b_{n}$.\\
	柯西不等式的推广为赫尔德 (Hölder)不等式,即
	
	赫尔德不等式 设 $a_{i}>0, b_{i}>0(i=1,2, \cdots, n), p>0, q>0$, 满足 $\frac{1}{p}+\frac{1}{q}=1$, 则
	
	$$
	\sum_{i=1}^{n} a_{i} b_{i} \leqslant\left(\sum_{i=1}^{n} a_{i}^{p}\right)^{\frac{1}{p}}\left(\sum_{i=1}^{n} b_{i}^{q}\right)^{\frac{1}{q}}
	$$
	
	等号成立的充分必要条件是 $a_{i}^{p}=\lambda b_{i}^{q}(i=1,2, \cdots, n, \lambda>0)$.
	
	证明 由 Young 不等式, 得
	
	$$
	\begin{aligned}
	& \sum_{i=1}^{n}\left[\frac{a_{i}^{p}}{\sum_{i=1}^{n} a_{i}^{p}}\right]^{\frac{1}{p}} \cdot\left[\frac{b_{i}^{q}}{\sum_{i=1}^{n} b_{i}^{q}}\right]^{\frac{1}{q}} \\
	\leqslant & \sum_{i=1}^{n}\left[\frac{1}{p} \frac{a_{i}^{p}}{\sum_{i=1}^{n} a_{i}^{p}}\right]+\sum_{i=1}^{n}\left[\frac{1}{q} \frac{b_{i}^{q}}{\sum_{i=1}^{n} b_{i}^{q}}\right] \\
	= & \frac{1}{p}+\frac{1}{q}=1
	\end{aligned}
	$$
	
	等号成立的充分必要条件是
	
	$$
	\frac{a_{i}^{p}}{\sum_{i=1}^{n} a_{i}^{p}}=\frac{b_{i}^{q}}{\sum_{i=1}^{n} b_{i}^{q}}
	$$
	
	即 $a_{i}^{p}=\lambda b_{i}^{q}(i=1,2, \cdots, n, \lambda>0)$.
	
	赫尔德不等式也可以变形为
	
	$$
	\sum_{i=1}^{n} \frac{a_{i}^{m+1}}{b_{i}^{m}} \geqslant \frac{\left(\sum_{i=1}^{n} a_{i}\right)^{m+1}}{\left(\sum_{i=1}^{n} b_{i}\right)^{m}}
	$$
	
	等号成立的充分必要条件是 $a_{i}=\lambda b_{i}(i=1,2, \cdots, n)$. 其中 $a_{i}>0, b_{i}>0$ $(i=1,2, \cdots, n), m>0$ 或 $m<-1$.
	
	证明 当 $m>0$ 时, 由赫尔德不等式, 得
	
	$$
	\begin{aligned}
	\sum_{i=1}^{n} a_{i} & =\sum_{i=1}^{n}\left(\frac{a_{i}}{b_{i}^{\frac{m}{m+1}}}\right) \cdot b_{i}^{\frac{m}{m+1}} \\
	& \leqslant\left[\sum_{i=1}^{n}\left(\frac{a_{i}}{b_{i}^{\frac{m}{m+1}}}\right)^{m+1}\right]^{\frac{1}{m+1}} \cdot\left[\sum_{i=1}^{n}\left(b_{i}^{\frac{m}{m+1}}\right)^{\frac{m+1}{m}}\right]^{\frac{m}{m+1}}
	\end{aligned}
	$$
	
	$$
	=\left(\sum_{i=1}^{n} \frac{a_{i}^{m+1}}{b_{i}^{m}}\right)^{\frac{1}{m+1}} \cdot\left(\sum_{i=1}^{n} b_{i}\right)^{\frac{m}{m+1}}
	$$
	
	故
	
	$$
	\sum_{i=1}^{n} \frac{a_{i}^{m+1}}{b_{i}^{m}} \geqslant \frac{\left(\sum_{i=1}^{n} a_{i}\right)^{m+1}}{\left(\sum_{i=1}^{n} b_{i}\right)^{m}}
	$$
	
	当 $m<-1$ 时, $-(m+1)>0$, 对于数组 $\left(b_{1}, b_{2}, \cdots, b_{n}\right)$ 和 $\left(a_{1}, a_{2}, \cdots\right.$, $\left.a_{n}\right)$ 有
	
	即
	
	$$
	\begin{gathered}
	\sum_{i=1}^{n} \frac{b_{i}^{-(m+1)+1}}{a_{i}^{-(m+1)}} \geqslant \frac{\left(\sum_{i=1}^{n} b_{i}\right)^{-(m+1)+1}}{\left(\sum_{i=1}^{n} a_{i}\right)^{-(m+1)}} \\
	\sum_{i=1}^{n} \frac{a_{i}^{m+1}}{b_{i}^{m}} \geqslant \frac{\left(\sum_{i=1}^{n} a_{i}\right)^{m+1}}{\left(\sum_{i=1}^{n} b_{i}\right)^{m}}
	\end{gathered}
	$$
	
	等号成立当且仅当 $\left(\frac{a_{i}}{b_{i}^{m+1}}\right)^{m+1}=\mu\left(b_{i}^{\frac{m}{m+1}}\right)^{\frac{m+1}{m}}$, 即 $a_{i}=\lambda b_{i}(i=1$, $2, \cdots, n)$.
	
	由赫尔德不等式可以推出另一个重要的不等式, 即
	
	闵可夫斯基(Minkowski)不等式 对 $a_{i}, b_{i} \in \mathbf{R}_{+}, 1 \leqslant i \leqslant n, k>1$, 则
	
	$$
	\left[\sum_{i=1}^{n}\left(a_{i}+b_{i}\right)^{k}\right]^{\frac{1}{k}} \leqslant\left(\sum_{i=1}^{n} a_{i}^{k}\right)^{\frac{1}{k}}+\left(\sum_{i=1}^{n} b_{i}^{k}\right)^{\frac{1}{k}}
	$$
	
	当且仅当 $\frac{a_{1}}{b_{1}}=\frac{a_{2}}{b_{2}}=\cdots=\frac{a_{n}}{b_{n}}$ 时, 等号成立.
	
	证明 $\quad$ 由赫尔德不等式, 得
	
	$$
	\begin{aligned}
	\sum_{i=1}^{n}\left(a_{i}+b_{i}\right)^{k} & =\sum_{i=1}^{n} a_{i}\left(a_{i}+b_{i}\right)^{k-1}+\sum_{i=1}^{n} b_{i}\left(a_{i}+b_{i}\right)^{k-1} \\
	& \leqslant\left(\sum_{i=1}^{n} a_{i}^{k}\right)^{\frac{1}{k}}\left[\sum_{i=1}^{n}\left(a_{i}+b_{i}\right)^{k}\right]^{\frac{k-1}{k}}+\left(\sum_{i=1}^{n} b_{i}^{k}\right)^{\frac{1}{k}}\left[\sum_{i=1}^{n}\left(a_{i}+b_{i}\right)^{k}\right]^{\frac{k-1}{k}}
	\end{aligned}
	$$
	
	所以
	
	$$
	\left[\sum_{i=1}^{n}\left(a_{i}+b_{i}\right)^{)^{1}}\right]^{\frac{1}{k}} \leqslant\left(\sum_{i=1}^{n} a_{i}^{k}\right)^{\frac{1}{k}}+\left(\sum_{i=1}^{n} b_{i}^{k}\right)^{\frac{1}{k}}
	$$
	
	不难知, 当且仅当 $\frac{a_{1}}{b_{1}}=\frac{a_{2}}{b_{2}}=\cdots=\frac{a_{n}}{b_{n}}$ 时, 等号成立.
\end{proof}

\begin{example}
	对一切正数 $x_{1}, x_{2}, \cdots, x_{n}$, 求证:
	
	$$
	\begin{aligned}
	& x_{1}^{3}+\left(\frac{x_{1}+x_{2}}{2}\right)^{3}+\left(\frac{x_{1}+x_{2}+x_{3}}{3}\right)^{3}+\cdots+\left(\frac{x_{1}+x_{2}+\cdots+x_{n}}{n}\right)^{3} \\
	< & \frac{27}{8}\left(x_{1}^{3}+x_{2}^{3}+\cdots+x_{n}^{3}\right) .
	\end{aligned}
	$$
\end{example}
\begin{proof}
	记 $A_{k}=\frac{x_{1}+x_{2}+\cdots+x_{k}}{k}$, 并约定 $A_{0}=0$, 则 $x_{k}=k A_{k}-(k-1) A_{k-1}$ $(k=1,2, \cdots, n)$
	
	先证: $\quad \sum_{k=1}^{n} A_{k}^{3} \leqslant \frac{3}{2} \sum_{k=1}^{n} x_{k} A_{k}^{2}$
	
	$$
	\Leftrightarrow \sum_{k=1}^{n} A_{k}^{3} \leqslant \frac{3}{2} \sum_{k=1}^{n}\left[k A_{k}-(k-1) A_{k-1}\right] A_{k}^{2}
	$$
	
	而右式 $=\frac{3}{2} \sum_{k=1}^{n} k A_{k}^{3}-\frac{3}{2} \sum_{k=1}^{n} A_{k-1} A_{k}^{2} \cdot(k-1)$
	
	$$
	\begin{aligned}
	& \geqslant \frac{3}{2} \sum_{k=1}^{n} k A_{k}^{3}-\frac{3}{2} \sum_{k=1}^{n} \frac{A_{k-1}^{3}+2 A_{k}^{3}}{3} \cdot(k-1) \\
	& =\sum_{k=1}^{n} A_{k}^{3}+\frac{n}{2} A_{n}^{3} \geqslant \sum_{k=1}^{n} A_{k}^{3}
	\end{aligned}
	$$
	
	$\circledast$ 得证.
	
	原不等式 $\Leftrightarrow \sum_{k=1}^{n} A_{k}^{3} \leqslant \frac{27}{8} \sum_{k=1}^{n} x_{k}^{3}$.
	
	由赫尔德不等式:
	
	
	\begin{gather*}
	\left(\sum_{k=1}^{n} x_{k}^{3}\right) \cdot\left(\sum_{k=1}^{n} A_{k}^{3}\right)^{2} \geqslant\left(\sum_{k=1}^{n} x_{k} A_{k}^{2}\right)^{3} \geqslant \frac{8}{27} \cdot\left(\sum_{k=1}^{n} A_{k}^{3}\right)^{3} \\
	\Rightarrow \sum_{k=1}^{n} A_{k}^{3} \leqslant \frac{27}{8} \sum_{k=1}^{n} x_{k}^{3} .
	\end{gather*}
	
	
	综上, 命题得证.
	
	关于赫尔德不等式的推广, 有如下定理.
	
	定理 对于 $n \times m$ 矩阵
	
	$$
	\left(\begin{array}{cccc}
	a_{11} & a_{12} & \cdots & a_{1 m} \\
	a_{21} & a_{22} & \cdots & a_{2 m} \\
	\cdots & \cdots & \cdots & \cdots \\
	a_{n 1} & a_{n 2} & \cdots & a_{n m}
	\end{array}\right)
	$$
	
	其中, $a_{i j} \geqslant 0(i=1,2, \cdots, n, j=1,2, \cdots, m)$, 则
	
	$$
	\left[\prod_{j=1}^{m}\left(\sum_{i=1}^{n} a_{i j}\right)\right]^{\frac{1}{m}} \geqslant \sum_{i=1}^{n}\left(\prod_{j=1}^{m} a_{i j}\right)^{\frac{1}{m}}
	$$
	
	其中, 等号成立的充要条件是至少有一列数都是 0 或所有行中的数对应成比例.
	
	这个不等式称为卡尔松不等式.
	
	证明 记
	
	$$
	\begin{aligned}
	A_{j} & =\sum_{i=1}^{n} a_{i j}(j=1,2, \cdots, m) \\
	G_{i} & =\prod_{j=1}^{m} a_{i j}(i=1,2, \cdots, n)
	\end{aligned}
	$$
	
	若某个 $A_{j}=0$, 则由 $a_{i j} \geqslant 0(i=1,2, \cdots, n)$, 得 $a_{1 j}=a_{2 j}=\cdots=$ $a_{n j}=0$.
	
	此时, $G_{1}=G_{2}=\cdots=G_{n}=0$,
	
	$$
	\left[\prod_{j=1}^{m}\left(\sum_{i=1}^{n} a_{i j}\right)\right]^{\frac{1}{m}}=\sum_{i=1}^{n}\left(\prod_{j=1}^{m} a_{i j}\right)^{\frac{1}{m}}=0
	$$
	
	从而, 不等式成立.
	
	若所有的 $A_{j}>0$, 由均值不等式得
	
	$$
	\frac{a_{i 1}}{A_{1}}+\frac{a_{i 2}}{A_{2}}+\cdots+\frac{a_{i n}}{A_{m}} \geqslant m\left(\frac{\prod_{j=1}^{m} a_{i j}}{\prod_{j=1}^{m} A_{j}}\right)^{\frac{1}{m}}(i=1,2, \cdots, n)
	$$
	
	将以上 $n$ 个不等式相加得
	
	$$
	m \geqslant m \sum_{i=1}^{n}\left(\frac{\prod_{j=1}^{m} a_{i j}}{\prod_{j=1}^{m} A_{j}}\right)^{\frac{1}{m}}=m \frac{\sum_{i=1}^{n} G_{i}^{\frac{1}{m}}}{\left(\prod_{j=1}^{m} A_{j}\right)^{\frac{1}{m}}}
	$$
	
	故 $\left[\prod_{j=1}^{m}\left(\sum_{i=1}^{n} a_{i j}\right)\right]^{\frac{1}{m}} \geqslant \sum_{i=1}^{n}\left(\prod_{j=1}^{m} a_{i j}\right)^{\frac{1}{m}}$.
	
	等号成立的充要条件是至少有一列数都是 0 或 $\frac{a_{i 1}}{A_{1}}=\frac{a_{i 2}}{A_{2}}=\cdots=\frac{a_{\text {in }}}{A_{m}}$, 即所有行中的数对应成比例.
	
	利用卡尔松不等式可以推证柯西不等式、均值不等式及幂平均不等式.
	
	(1)构造 $n \times 2$ 矩阵 $\left(\begin{array}{cc}a_{1}^{2} & b_{1}^{2} \\ a_{2}^{2} & b_{2}^{2} \\ \cdots & \cdots \\ a_{n}^{2} & b_{n}^{2}\end{array}\right)$.
	
	利用卡尔松不等式得柯西不等式
	
	$$
	\left[\left(\sum_{i=1}^{n} a_{i}^{2}\right) \sum_{i=1}^{n} b_{i}^{2}\right]^{\frac{1}{2}} \geqslant \sum_{i=1}^{n} a_{i} b_{i}
	$$
	
	(2) 构造 $n \times n$ 矩阵
	
	$$
	\left(\begin{array}{cccc}
	x_{1} & x_{2} & \cdots & x_{n} \\
	x_{2} & x_{3} & \cdots & x_{1} \\
	\cdots & \cdots & \cdots & \cdots \\
	x_{n} & x_{1} & \cdots & x_{n-1}
	\end{array}\right)
	$$
	
	利用卡尔松不等式得
	
	$$
	\left[\left(\sum_{i=1}^{n} x_{i}\right)^{n}\right]^{\frac{1}{n}} \geqslant n\left(\prod_{i=1}^{n} x_{i}\right)^{\frac{1}{n}}
	$$
	
	即均值不等式
	
	$$
	\frac{\sum_{i=1}^{n} x_{i}}{n} \geqslant\left(\prod_{i=1}^{n} x_{i}\right)^{\frac{1}{n}}
	$$
	
	(3)构造 $n \times \alpha$ 矩阵
	
	$$
	\left(\begin{array}{ccccccc}
	x_{1}^{\alpha} & x_{1}^{\alpha} & \cdots & x_{1}^{\alpha} & 1 & \cdots & 1 \\
	x_{2}^{\alpha} & x_{2}^{\alpha} & \cdots & x_{2}^{\alpha} & 1 & \cdots & 1 \\
	\cdots & \cdots & \cdots & \cdots & \cdots & \cdots & \cdots \\
	x_{n}^{\alpha} & x_{n}^{\alpha} & \cdots & x_{n}^{\alpha} & 1 & \cdots & 1
	\end{array}\right),
	$$
	
	其中, $x_{i}^{x}$ 共有 $\beta$ 列, 1 共有 $\alpha-\beta$ 列.
	
	利用卡尔松不等式得
	
	$$
	\begin{aligned}
	& {\left[\left(x_{1}^{\alpha}+x_{2}^{\alpha}+\cdots+x_{n}^{\alpha}\right)^{\beta} n^{\alpha-\beta}\right]^{\frac{1}{\alpha}} } \\
	\geqslant & {\left[\left(x_{1}^{\alpha}\right)^{\beta} \cdot 1^{\alpha-\beta}\right]^{\frac{1}{\alpha}}+\left[\left(x_{2}^{\alpha}\right)^{\beta} \cdot 1^{\alpha-\beta}\right]^{\frac{1}{\alpha}}+\cdots+\left[\left(x_{n}^{\alpha}\right)^{\beta} \cdot 1^{\alpha \beta}\right]^{\frac{1}{\alpha}} } \\
	= & x_{1}^{\beta}+x_{2}^{\beta}+\cdots+x_{n}^{\beta},
	\end{aligned}
	$$
	
	即幂平均不等式
	
	$$
	\left(\frac{x_{1}^{\alpha}+x_{2}^{\alpha}+\cdots+x_{n}^{\alpha}}{n}\right)^{\frac{1}{\alpha}} \geqslant\left(\frac{x_{1}^{\beta}+x_{2}^{\beta}+\cdots+x_{n}^{\beta}}{n}\right)^{\frac{1}{\beta}}\left(\alpha, \beta \in \mathbf{N}_{+}, \alpha>\beta\right)
	$$
	
	【说明】(1) 卡尔松不等式和均值不等式是等价的, 柯西不等式是卡尔松不等式的一种特殊形式, 即 $n \times m$ 矩阵中 $m=2$ 的情形.
	
	(2)利用卡尔松不等式证明不等式的关键是构造矩阵, 充分利用条件和结论提供的信息, 注意取等号的条件是构造矩阵的关键.
	

	\section*{柯西不等式的应用}
	\section*{4. 1 柯西不等式在证明不等式中的应用}
	运用柯西不等式, 证明其他不等式的关键是构造两组数, 并按照柯西不等式形式进行探索, 巧妙选取两组数.

\begin{note}
	在证明过程中, 注意条件的利用和不等式的变形.
\end{note}

\begin{example}
	已知 $a, b, c \in \mathbf{R}_{+}$, 且 $a+b+c=1$, 求证:
	
	$$
	36 \leqslant \frac{1}{a}+\frac{4}{b}+\frac{9}{c}
	$$
\end{example}
\begin{proof}
	由柯西不等式,得
	
	$$
	\begin{aligned}
	\frac{1}{a}+\frac{4}{b}+\frac{9}{c}= & \left(\frac{1}{a}+\frac{4}{b}+\frac{9}{c}\right) \cdot(a+b+c) \\
	\geqslant & \left(\sqrt{a} \cdot \frac{1}{\sqrt{a}}+\sqrt{b} \cdot \frac{2}{\sqrt{b}}+\sqrt{c} \cdot \frac{3}{\sqrt{c}}\right)^{2}=36 \\
	& \frac{1}{a}+\frac{4}{b}+\frac{9}{c} \geqslant 36
	\end{aligned}
	$$
	
	所以
\end{proof}
\begin{note}
	在证明过程中, 注意条件的利用和不等式的变形.
\end{note}

\begin{example}
	设 $a, b, c \in \mathbf{R}_{+}$, 满足 $a \cos ^{2} \alpha+b \sin ^{2} \alpha<c$, 求证:
	
	$$
	\sqrt{a} \cos ^{2} \alpha+\sqrt{b} \sin ^{2} \alpha<\sqrt{c}
	$$
\end{example}
\begin{proof}
	由柯西不等式, 得
	
	$$
	\begin{aligned}
	\sqrt{a} \cos ^{2} \alpha+\sqrt{b} \sin ^{2} \alpha & =\sqrt{a} \cos \alpha \cdot \cos \alpha+\sqrt{b} \sin \alpha \cdot \sin \alpha \\
	& \leqslant\left[(\sqrt{a} \cos \alpha)^{2}+(\sqrt{b} \sin \alpha)^{2}\right]^{\frac{1}{2}} \cdot\left(\cos ^{2} \alpha+\sin ^{2} \alpha\right)^{\frac{1}{2}} \\
	& =\left(a \cos ^{2} \alpha+b \sin ^{2} \alpha\right)^{\frac{1}{2}}<\sqrt{c},
	\end{aligned}
	$$
	
	故命题成立.\\
\end{proof}
\begin{note}
	在证明过程中, 注意条件的利用和不等式的变形.
\end{note}

\begin{example}
	设 $a_{i}>0(i=1,2, \cdots, n)$ 满足 $\sum_{i=1}^{n} a_{i}=1$, 求证:
	
	$$
	\frac{a_{1}^{2}}{a_{1}+a_{2}}+\frac{a_{2}^{2}}{a_{2}+a_{3}}+\cdots+\frac{a_{n}^{2}}{a_{n}+a_{1}} \geqslant \frac{1}{2}
	$$
\end{example}
\begin{proof}
	令 $a_{n+1}=a_{1}$, 由柯西不等式, 得
	
	$$
	\begin{aligned}
	\left(\sum_{i=1}^{n} a_{i}\right)^{2} & =\left(\sum_{i=1}^{n} \frac{a_{i}}{\sqrt{a_{i}+a_{i+1}}} \cdot \sqrt{a_{i}+a_{i+1}}\right)^{2} \\
	& \leqslant \sum_{i=1}^{n} \frac{a_{i}^{2}}{a_{i}+a_{i+1}} \cdot \sum_{i=1}^{n}\left(a_{i}+a_{i+1}\right) \\
	& =2 \sum_{i=1}^{n} \frac{a_{i}^{2}}{a_{i}+a_{i+1}} \cdot \sum_{i=1}^{n} a_{i}
	\end{aligned}
	$$
	
	于是
	
	$$
	\sum_{i=1}^{n} \frac{a_{i}^{2}}{a_{i}+a_{i+1}} \geqslant \frac{1}{2} \sum_{i=1}^{n} a_{i}=\frac{1}{2}
	$$
\end{proof}
\begin{note}
	在证明过程中, 注意条件的利用和不等式的变形.
\end{note}

\begin{example}
	设 $a 、 b 、 c$ 是正实数, 且满足 $a+b+c=1$. 证明:
	
	$$
	\frac{a-b c}{a+b c}+\frac{b-c a}{b+c a}+\frac{c-a b}{c+a b} \leqslant \frac{3}{2}
	$$
\end{example}
\begin{proof}
	注意到
	
	$$
	1-\frac{a-b c}{a+b c}=\frac{2 b c}{a+b c}=\frac{2 b c}{1-b-c+b c}=\frac{2 b c}{(1-b)(1-c)}
	$$
	
	同理, $1-\frac{b-c a}{b+c a}=\frac{2 c a}{(1-c)(1-a)}, 1-\frac{c-a b}{c+a b}=\frac{2 a b}{(1-a)(1-b)}$.故原不等式等价于
	
	$$
	\frac{2 b c}{(1-b)(1-c)}+\frac{2 c a}{(1-c)(1-a)}+\frac{2 a b}{(1-a)(1-b)} \geqslant \frac{3}{2}
	$$
	
	化简后得
	
	$$
	4(b c+c a+a b-3 a b c) \geqslant 3(b c+c a+a b+1-a-b-c-a b c)
	$$
	
	即
	
	$$
	a b+b c+c a \geqslant 9 a b c
	$$
	
	从而要证 $\frac{1}{a}+\frac{1}{b}+\frac{1}{c} \geqslant 9$.
	
	而 $\frac{1}{a}+\frac{1}{b}+\frac{1}{c}=(a+b+c)\left(\frac{1}{a}+\frac{1}{b}+\frac{1}{c}\right) \geqslant 9$, 因此, 原不等式成立.\\
\end{proof}
\begin{note}
	此题的证明方法较多,不妨自己试一试.
\end{note}

\begin{example}
	设正实数 $a 、 b 、 c$ 满足 $a+b+c=3$. 证明:
	
	$$
	\frac{1}{2+a^{2}+b^{2}}+\frac{1}{2+b^{2}+c^{2}}+\frac{1}{2+c^{2}+a^{2}} \leqslant \frac{3}{4}
	$$
\end{example}
\begin{proof}
	用符号 $\sum_{\mathrm{cyc}}$ 表示循环和, 即证明:
	
	
	\begin{equation*}
	\sum_{\mathrm{cyc}} \frac{1}{2+a^{2}+b^{2}} \leqslant \frac{3}{4} \tag{1}
	\end{equation*}
	
	
	由柯西不等式得
	
	$$
	\left(\sum_{\mathrm{cyc}} \frac{a^{2}+b^{2}}{2+a^{2}+b^{2}}\right) \sum_{\mathrm{cyc}}\left(2+a^{2}+b^{2}\right) \geqslant\left(\sum_{\mathrm{cyc}} \sqrt{a^{2}+b^{2}}\right)^{2}
	$$
	
	又
	
	$$
	\left(\sum_{\text {cyc }} \sqrt{a^{2}+b^{2}}\right)^{2}=2 \sum_{\text {cyc }} a^{2}+2 \sum_{\text {cyc }} \sqrt{\left(a^{2}+b^{2}\right)\left(a^{2}+c^{2}\right)}
	$$
	
	及
	
	$$
	\sqrt{\left(a^{2}+b^{2}\right)\left(a^{2}+c^{2}\right)} \geqslant a^{2}+b c
	$$
	
	则
	
	$$
	\begin{aligned}
	& \left(\sum_{\text {cyc }} \sqrt{a^{2}+b^{2}}\right)^{2} \\
	\geqslant & 2 \sum_{\text {cyc }} a^{2}+2 \sum_{\text {cyc }} a^{2}+2 \sum_{\text {cyc }} a b \\
	= & 3 \sum_{\text {cyc }} a^{2}+(a+b+c)^{2}=9+3 \sum_{\text {cyc }} a^{2} \\
	= & \frac{3}{2}\left(6+2 \sum_{\text {cyc }} a^{2}\right)=\frac{3}{2} \sum_{\text {cyc }}\left(2+a^{2}+b^{2}\right)
	\end{aligned}
	$$
	
	故 $\left(\sum_{\text {cyc }} \frac{a^{2}+b^{2}}{2+a^{2}+b^{2}}\right) \sum_{\text {сус }}\left(2+a^{2}+b^{2}\right) \geqslant \frac{3}{2} \sum_{\text {сус }}\left(2+a^{2}+b^{2}\right)$.
	
	所以,
	
	
	\begin{equation*}
	\sum_{\text {сус }} \frac{a^{2}+b^{2}}{2+a^{2}+b^{2}} \geqslant \frac{3}{2} \tag{2}
	\end{equation*}
	
	
	式(2)两边乘以 -1 ,再加 3 , 再除以 2 即得式(1).
\end{proof}
\begin{note}
	此题的证明方法较多,不妨自己试一试.
\end{note}

\begin{example}
	已知 $x, y, z>0$, 且 $x y z=1$. 求证:
	
	$\frac{(x+y-1)^{2}}{z}+\frac{(y+z-1)^{2}}{x}+\frac{(z+x-1)^{2}}{y} \geqslant 4(x+y+z)-12+\frac{9}{x+y+z}$.
\end{example}
\begin{proof}
	因为
	
	$$
	(a-b)^{2}=a^{2}-2 a b+b^{2},
	$$
	
	所以
	
	$$
	a^{2}=2 a b-b^{2}+(a-b)^{2},
	$$
	
	在 $b>0$ 时有
	
	$$
	\frac{a^{2}}{b}=2 a-b+\frac{(a-b)^{2}}{b}
	$$
	
	利用上式及柯西不等式, 可知
	
	$$
	\begin{aligned}
	& \frac{(x+y-1)^{2}}{z}+\frac{(y+z-1)^{2}}{x}+\frac{(z+x-1)^{2}}{y} \\
	= & 2(x+y-1)-z+\frac{(x+y-z-1)^{2}}{z} \\
	& +2(y+z-1)-x+\frac{(y+z-x-1)^{2}}{x} \\
	& +2(z+x-1)-y+\frac{(z+x-y-1)^{2}}{y} \\
	= & 3(x+y+z)-6+\frac{(x+y-z-1)^{2}}{z}+\frac{(y+z-x-1)^{2}}{x}+\frac{(z+x-y-1)^{2}}{y} \\
	\geqslant & 3(x+y+z)-6+\frac{(x+y+z-3)^{2}}{x+y+z} \\
	= & 3(x+y+z)-6+\frac{(x+y+z)^{2}-6(x+y+z)+9}{x+y+z} \\
	= & 4(x+y+z)-12+\frac{9}{x+y+z} .
	\end{aligned}
	$$
\end{proof}
\begin{note}
	此题的证明方法较多,不妨自己试一试.
\end{note}

\begin{example}
	设非负实数 $a_{1}, a_{2}, \cdots, a_{n}$ 与 $b_{1}, b_{2}, \cdots, b_{n}$ 同时满足以下条件:
	
	(1) $\sum_{i=1}^{n}\left(a_{i}+b_{i}\right)=1$;
	
	(2) $\sum_{i=1}^{n} i\left(a_{i}-b_{i}\right)=0$;
	
	(3) $\sum_{i=1}^{n} i^{2}\left(a_{i}+b_{i}\right)=10$.
	
	求证:对任意 $1 \leqslant k \leqslant n$, 都有 $\max \left\{a_{k}, b_{k}\right\} \leqslant \frac{10}{10+k^{2}}$.
\end{example}
\begin{proof}
	对任意 $1 \leqslant k \leqslant n$, 有
	
	$$
	\begin{aligned}
	\left(k a_{k}\right)^{2} & \leqslant\left(\sum_{i=1}^{n} i a_{i}\right)^{2}=\left(\sum_{i=1}^{n} i b_{i}\right)^{2} \\
	& \leqslant\left(\sum_{i=1}^{n} i^{2} b_{i}\right)\left(\sum_{i=1}^{n} b_{i}\right) \quad \text { (柯西不等式) } \\
	& =\left(10-\sum_{i=1}^{n} i^{2} a_{i}\right)\left(1-\sum_{i=1}^{n} a_{i}\right) \\
	& \leqslant\left(10-k^{2} a_{k}\right)\left(1-a_{k}\right) \\
	& =10-\left(10+k^{2}\right) a_{k}+k^{2} a_{k}^{2},
	\end{aligned}
	$$
	
	从而 $a_{k} \leqslant \frac{10}{10+k^{2}}$.
	
	同理有 $b_{k} \leqslant \frac{10}{10+k^{2}}$, 所以 $\max \left\{a_{k}, b_{k}\right\} \leqslant \frac{10}{10+k^{2}}$.
\end{proof}
\begin{note}
	此题的证明方法较多,不妨自己试一试.
\end{note}

\begin{example}
	设 $a_{i} \in \mathbf{R}_{+}(i=1,2, \cdots, n)$, 如果对任意 $x_{i} \geqslant 0$,
	
	$$
	\sum_{i=1}^{n} r_{i}\left(x_{i}-a_{i}\right) \leqslant \sqrt{\sum_{i=1}^{n} x_{i}^{2}}-\sqrt{\sum_{i=1}^{n} a_{i}^{2}}
	$$
	
	求 $r_{i}(i=1,2, \cdots, n)$.
\end{example}
\begin{solution}
	令 $x_{i}=0$, 则 $\sum_{i=1}^{n} r_{i} a_{i} \geqslant \sqrt{\sum_{i=1}^{n} a_{i}^{2}}$.
	
	再令 $x_{i}=2 a_{i}$, 则 $\sum_{i=1}^{n} r_{i} a_{i} \leqslant \sqrt{\sum_{i=1}^{n} a_{i}^{2}}$.
	
	于是
	
	$$
	\sum_{i=1}^{n} r_{i} a_{i}=\sqrt{\sum_{i=1}^{n} a_{i}^{2}}
	$$
	
	令 $x_{i}=r_{i}$, 则 $\sum_{i=1}^{n} r_{i}\left(r_{i}-a_{i}\right) \leqslant \sqrt{\sum_{i=1}^{n} r_{i}^{2}}-\sqrt{\sum_{i=1}^{n} a_{i}^{2}}$,
	
	推出
	
	$$
	\sum_{i=1}^{n} r_{i}^{2} \leqslant \sqrt{\sum_{i=1}^{n} r_{i}^{2}}
	$$
	
	即 $\sum_{i=1}^{n} r_{i}^{2} \leqslant 1$. 由柯西不等式, 得
	
	$$
	\left(\sum_{i=1}^{n} r_{i} a_{i}\right)^{2} \leqslant\left(\sum_{i=1}^{n} r_{i}^{2}\right)\left(\sum_{i=1}^{n} a_{i}^{2}\right)
	$$
	
	等号成立充要条件是 $r_{i}=\lambda a_{i}$.
	
	从而 $\sum_{i=1}^{n} r_{i}^{2} \geqslant 1$, 于是
	
	$$
	\sum_{i=1}^{n} r_{i}^{2}=1, \lambda=\frac{1}{\sqrt{\sum_{i=1}^{n} a_{i}^{2}}}, r_{i}=\frac{a_{i}}{\sqrt{\sum_{i=1}^{n} a_{i}^{2}}}
	$$
	
	经验证, $r_{i}=\frac{a_{i}}{\sqrt{\sum_{i=1}^{n} a_{i}^{2}}}(i=1,2, \cdots, n)$ 为所求.\\
	上面的条件可以改为一般的形式:
	
	$$
	\sum_{i=1}^{n} r_{i}\left(x_{i}-a_{i}\right) \leqslant\left(\sum_{i=1}^{n} x_{i}^{m}\right)^{\frac{1}{m}}-\left(\sum_{i=1}^{n} a_{i}^{m}\right)^{\frac{1}{m}}
	$$
	
	其中 $m>1$ 为给定的常数.
	
	利用赫尔德不等式, 得
	
	$$
	r_{i}=\left[\frac{a_{i}^{m}}{\sum_{i=1}^{n} a_{i}^{m}}\right]^{\frac{m-1}{m}}(i=1,2, \cdots, n)
	$$
\end{solution}
\begin{note}
	此题的证明方法较多,不妨自己试一试.
\end{note}

\begin{example}
	设 $x_{i}, y_{i}, \cdots, z_{i} \in \mathbf{R}(i=1,2, \cdots, n)$, 求证:
	
	$$
	\sum_{i=1}^{n} \sqrt{x_{i}^{2}+y_{i}^{2}+\cdots+z_{i}^{2}} \geqslant \sqrt{\left(\sum_{i=1}^{n} x_{i}\right)^{2}+\left(\sum_{i=1}^{n} y_{i}\right)^{2}+\cdots+\left(\sum_{i=1}^{n} z_{i}\right)^{2}}
	$$
\end{example}
\begin{proof}
	令 $a=\sum_{i=1}^{n} x_{i}, b=\sum_{i=1}^{n} y_{i}, \cdots, c=\sum_{i=1}^{n} z_{i}$. 不妨设 $a^{2}+b^{2}+\cdots+$ $c^{2} \neq 0$, 则由柯西不等式, 得
	
	$$
	\left(a^{2}+b^{2}+\cdots+c^{2}\right)\left(x_{i}^{2}+y_{i}^{2}+\cdots+z_{i}^{2}\right) \geqslant\left(a x_{i}+b y_{i}+\cdots+c z_{i}\right)^{2}
	$$
	
	即
	
	$a x_{i}+b y_{i}+\cdots+c z_{i} \leqslant \sqrt{a^{2}+b^{2}+\cdots+c^{2}} \cdot \sqrt{x_{i}^{2}+y_{i}^{2}+\cdots+z_{i}^{2}}$.
	
	求和, 得
	
	$a^{2}+b^{2}+\cdots+c^{2} \leqslant \sqrt{a^{2}+b^{2}+\cdots+c^{2}} \sum_{i=1}^{n} \sqrt{x_{i}^{2}+y_{i}^{2}+\cdots+z_{i}^{2}}$.
	
	故
	
	$$
	\sum_{i=1}^{n} \sqrt{x_{i}^{2}+y_{i}^{2}+\cdots+z_{i}^{2}} \geqslant \sqrt{a^{2}+b^{2}+\cdots+c^{2}}
	$$
	
	本例如果用向量方法证明, 会更简洁.
\end{proof}
\begin{note}
	此题的证明方法较多,不妨自己试一试.
\end{note}

\begin{example}
	设 $a_{i} \in \mathbf{R}_{+}, 1 \leqslant i \leqslant n$. 证明:
	
	$$
	\frac{1}{\frac{1}{1+a_{1}}+\frac{1}{1+a_{2}}+\cdots+\frac{1}{1+a_{n}}}-\frac{1}{\frac{1}{a_{1}}+\frac{1}{a_{2}}+\cdots+\frac{1}{a_{n}}} \geqslant \frac{1}{n}
	$$
\end{example}
\begin{proof}
	令 $\sum_{i=1}^{n} \frac{1}{a_{i}}=a$, 则 $\sum_{i=1}^{n} \frac{1+a_{i}}{a_{i}}=n+a$. 由柯西不等式, 得
	
	$$
	\sum_{i=1}^{n} \frac{a_{i}}{1+a_{i}} \cdot \sum_{i=1}^{n} \frac{1+a_{i}}{a_{i}} \geqslant n^{2}
	$$
	
	所以 $\sum_{i=1}^{n} \frac{a_{i}}{a_{i}+1} \geqslant \frac{n^{2}}{n+a}$, 以及
	
	$$
	\begin{aligned}
	\sum_{i=1}^{n} \frac{1}{a_{i}+1} & =\sum_{i=1}^{n}\left(1-\frac{a_{i}}{a_{i}+1}\right)=n-\sum_{i=1}^{n} \frac{a_{i}}{a_{i}+1} \\
	& \leqslant n-\frac{n^{2}}{n+a}=\frac{n a}{n+a}
	\end{aligned}
	$$
	
	于是
	
	$$
	\begin{aligned}
	& \frac{1}{\frac{1}{1+a_{1}}+\frac{1}{1+a_{2}}+\cdots+\frac{1}{1+a_{n}}}-\frac{1}{\frac{1}{a_{1}}+\frac{1}{a_{2}}+\cdots+\frac{1}{a_{n}}} \\
	\geqslant & \frac{1}{\frac{n a}{n+a}}-\frac{1}{a}=\frac{n+a}{n a}-\frac{1}{a}=\frac{a}{n a}=\frac{1}{n}
	\end{aligned}
	$$
	
	故命题成立.
\end{proof}
\begin{note}
	此题的证明方法较多,不妨自己试一试.
\end{note}

\begin{example}
	设 $n$ 为正整数, $x_{1} \leqslant x_{2} \leqslant \cdots \leqslant x_{n}$ 为实数, 证明:
	
	(1) $\left(\sum_{i, j=1}^{n}\left|x_{i}-x_{j}\right|\right)^{2} \leqslant \frac{2\left(n^{2}-1\right)}{3} \sum_{i, j=1}^{n}\left(x_{i}-x_{j}\right)^{2}$;
	
	(2)第 (1) 小题等号成立的充要条件是 $x_{1}, x_{2}, \cdots, x_{n}$ 为等差数列.
\end{example}
\begin{proof}
	(1) 不失一般性, 可设 $\sum_{i=1}^{n} x_{i}=0$, 得
	
	$$
	\sum_{i, j=1}^{n}\left|x_{i}-x_{j}\right|=2 \sum_{i<j}\left(x_{j}-x_{i}\right)=2 \sum_{i=1}^{n}(2 i-n-1) x_{i}
	$$
	
	由柯西不等式, 得
	
	$$
	\begin{aligned}
	\left(\sum_{i, j=1}^{n}\left|x_{i}-x_{j}\right|\right)^{2} & \leqslant 4 \sum_{i=1}^{n}(2 i-n-1)^{2} \sum_{i=1}^{n} x_{i}^{2} \\
	& =4 \times \frac{n(n-1)(n+1)}{3} \sum_{i=1}^{n} x_{i}^{2}
	\end{aligned}
	$$
	
	另一方面,
	
	$$
	\sum_{i, j=1}^{n}\left(x_{i}-x_{j}\right)^{2}=n \sum_{i=1}^{n} x_{i}^{2}-2 \sum_{i=1}^{n} x_{i} \sum_{j=1}^{n} x_{j}+n \sum_{j=1}^{n} x_{j}^{2}=2 n \sum_{i=1}^{n} x_{i}^{2}
	$$
	
	从而
	
	$$
	\left(\sum_{i, j=1}^{n}\left|x_{i}-x_{j}\right|\right)^{2} \leqslant \frac{2\left(n^{2}-1\right)}{3} \sum_{i, j=1}\left(x_{i}-x_{j}\right)^{2}
	$$
	
	(2)如果等号成立, 则对某个 $k, x_{i}=k(2 i-n-1)$, 则 $x_{1}, x_{2}, \cdots, x_{n}$ 为等差数列. 另一方面, 如果 $x_{1}, x_{2}, \cdots, x_{n}$ 为等差数列, 公差为 $d$, 则
	
	$$
	x_{i}=\frac{d}{2}(2 i-n-1)+\frac{x_{1}+x_{n}}{2}
	$$
	
	将每个 $x_{i}$ 减去 $\frac{x_{1}+x_{n}}{2}$, 就有 $x_{i}=\frac{d}{2}(2 i-n-1)$, 且 $\sum_{i=1}^{n} x_{i}=0$, 这时等号成立.
\end{proof}
\begin{note}
	在证明过程中, 常常进行一些恒等的变形.
\end{note}

\begin{example}
	证明: 满足条件
	
	(1) $a_{1}+a_{2}+\cdots+a_{n} \geqslant n^{2}$
	
	(2) $a_{1}^{2}+a_{2}^{2}+\cdots+a_{n}^{2} \leqslant n^{3}+1$
	
	的整数只有 $\left(a_{1}, a_{2}, \cdots, a_{n}\right)=(n, n, \cdots, n)$.
\end{example}
\begin{proof}
	设 $\left(a_{1}, a_{2}, \cdots, a_{n}\right)$ 是满足条件的整数组, 则由柯西不等式, 得
	
	$$
	a_{1}^{2}+a_{2}^{2}+\cdots+a_{n}^{2} \geqslant \frac{1}{n}\left(a_{1}+a_{2}+\cdots+a_{n}\right)^{2} \geqslant n^{3}
	$$
	
	结合 $a_{1}^{2}+a_{2}^{2}+\cdots+a_{n}^{2} \leqslant n^{3}+1$, 可知只能 $\sum_{i=1}^{n} a_{i}^{2}=n^{3}$ 或者 $\sum_{i=1}^{n} a_{i}^{2}=$ $n^{3}+1$.
	
	当 $\sum_{i=1}^{n} a_{i}^{2}=n^{3}$ 时, 由柯西不等式取等号得 $a_{1}=a_{2}=\cdots=a_{n}$, 即 $a_{i}^{2}=n^{2}$, $1 \leqslant i \leqslant n$. 再由 $\sum_{i=1}^{n} a_{i} \geqslant n^{2}$, 则只有 $a_{1}=a_{2}=\cdots=a_{n}=n$.
	
	当 $\sum_{i=1}^{n} a_{i}=n^{3}+1$ 时, 则令 $b_{i}=a_{i}-n$, 得
	
	$$
	\sum_{i=1}^{n} b_{i}^{2}=\sum_{i=1}^{n} a_{i}^{2}-2 n \sum_{i=1}^{n} a_{i}+n^{3} \leqslant 2 n^{3}+1-2 n \sum_{i=1}^{n} a_{i} \leqslant 1
	$$
	
	于是 $b_{i}^{2}$ 只能是 0 或者 1 , 且 $b_{1}^{2}, b_{2}^{2}, \cdots, b_{n}^{2}$ 中至多有一个为 1 . 如果都为零, 则 $a_{i}=n, \sum_{i=1}^{n} a_{i}^{2}=n^{3} \neq n^{3}+1$, 矛盾. 如果 $b_{1}^{2}, b_{2}^{2}, \cdots, b_{n}^{2}$ 中有一个为 1 ,则 $\sum_{i=1}^{n} a_{i}^{2}=n^{3} \pm 2 n+1 \neq n^{3}+1$, 也矛盾. 故只有 $\left(a_{1}, a_{2}, \cdots, a_{n}\right)=(n$,\\
	$n, \cdots, n)$ 为唯一一组整数解.
\end{proof}
\begin{note}
	在证明过程中, 常常进行一些恒等的变形.
\end{note}

\begin{example}
	证明:关于两个三角形的匹窦不等式
	
	$$
	a^{2}\left(b_{1}^{2}+c_{1}^{2}-a_{1}^{2}\right)+b^{2}\left(c_{1}^{2}+a_{1}^{2}-b_{1}^{2}\right)+c^{2}\left(a_{1}^{2}+b_{1}^{2}-c_{1}^{2}\right) \geqslant 16 S S_{1},
	$$
	
	这里 $a 、 b 、 c 、 S ; a_{1} 、 b_{1} 、 c_{1} 、 S_{1}$ 分别为两个三角形的边长和面积.
\end{example}
\begin{proof}
	由柯西不等式, 得
	
	$$
	\begin{aligned}
	& 16 S S_{1}+2 a^{2} a_{1}^{2}+2 b^{2} b_{1}^{2}+2 c^{2} c_{1}^{2} \\
	\leqslant & \left(16 S_{1}^{2}+2 a_{1}^{4}+2 b_{1}^{4}+2 c^{4}\right)^{\frac{1}{2}}\left(16 S^{2}+2 a^{4}+2 b^{4}+2 c^{4}\right)^{\frac{1}{2}} \\
	= & \left(a_{1}^{2}+b_{1}^{2}+c_{1}^{2}\right)\left(a^{2}+b^{2}+c^{2}\right),
	\end{aligned}
	$$
	
	所以
	
	$$
	a^{2}\left(b_{1}^{2}+c_{1}^{2}-a_{1}^{2}\right)+b^{2}\left(c_{1}^{2}+a_{1}^{2}-b_{1}^{2}\right)+c^{2}\left(a_{1}^{2}+b_{1}^{2}-c_{1}^{2}\right) \geqslant 16 S S_{1} .
	$$
\end{proof}
\begin{note}
	在证明过程中, 常常进行一些恒等的变形.
\end{note}

\begin{example}
	设 $a 、 b 、 c$ 是三角形的三边长,求证:
	
	$$
	a^{2} b(a-b)+b^{2} c(b-c)+c^{2} a(c-a) \geqslant 0
	$$
\end{example}
\begin{proof}
	显然存在正数 $x 、 y 、 z$ 使得 $a=y+z, b=z+x, c=x+y$. 由于
	
	$$
	\begin{aligned}
	& a^{2} b(a-b)=(y+z)^{2}(z+x)(y-x) \\
	= & (y+z)(z+x)\left(y^{2}-z^{2}\right)+(y+z)^{2}\left(z^{2}-x^{2}\right)
	\end{aligned}
	$$
	
	同样处理 $b^{2} c(b-c), c^{2} a(c-a)$, 所以
	
	$$
	\begin{aligned}
	& a^{2} b(a-b)+b^{2} c(b-c)+c^{2} a(c-a) \\
	= & 2 x(y-z) y^{2}+2 y(z-x) z^{2}+2 z(x-y) x^{2}
	\end{aligned}
	$$
	
	原不等式等价于
	
	$$
	x y z(x+y+z) \leqslant x y^{3}+y z^{3}+z x^{3} .
	$$
	
	由柯西不等式, 得
	
	则
	
	$$
	\begin{gathered}
	x+y+z \leqslant\left(\frac{x^{2}}{y}+\frac{y^{2}}{z}+\frac{z^{2}}{x}\right)^{\frac{1}{2}}(x+y+z)^{\frac{1}{2}} \\
	x+y+z \leqslant \frac{x^{2}}{y}+\frac{y^{2}}{z}+\frac{z^{2}}{x}
	\end{gathered}
	$$
	
	于是原不等式成立, 且当 $a=b=c$ 时等号成立.\\
\end{proof}
\begin{note}
	该例题也可以直接用柯西不等式证明, 关于它的推广, 可以参见引文或练习。
\end{note}

\begin{example}
	设 $x_{i}>0, x_{i} y_{i}-z_{i}^{2}>0, i=1,2$, 求证:
	
	$$
	\frac{8}{\left(x_{1}+x_{2}\right)\left(y_{1}+y_{2}\right)-\left(z_{1}+z_{2}\right)^{2}} \leqslant \frac{1}{x_{1} y_{1}-z_{1}^{2}}+\frac{1}{x_{2} y_{2}-z_{2}^{2}}
	$$
\end{example}
\begin{proof}
	注意到不等式的右边 $\geqslant \frac{2}{\left[\left(x_{1} y_{1}-z_{1}^{2}\right)\left(x_{2} y_{2}-z_{2}^{2}\right)\right]^{\frac{1}{2}}}$, 考虑证明一个更强的结论:
	
	$$
	\begin{aligned}
	& \left(x_{1}+x_{2}\right)\left(y_{1}+y_{2}\right)-\left(z_{1}+z_{2}\right)^{2} \geqslant 4\left[\left(x_{1} y_{1}-z_{1}^{2}\right)\left(x_{2} y_{2}-z_{2}^{2}\right)\right]^{\frac{1}{2}} . \\
	& \text { 令 } u_{i}=\sqrt{x_{i} y_{i}-z_{i}^{2}}, i=1,2 \text {, 由于 } 4 u_{1} u_{2} \leqslant\left(u_{1}+u_{2}\right)^{2} \text {, 则只要证明 } \\
	& \quad\left(x_{1}+x_{2}\right)\left(y_{1}+y_{2}\right)-\left(z_{1}+z_{2}\right)^{2} \geqslant\left(u_{1}+u_{2}\right)^{2},
	\end{aligned}
	$$
	
	等价于
	
	$$
	\left(x_{1}+x_{2}\right)\left(y_{1}+y_{2}\right) \geqslant\left(u_{1}+u_{2}\right)^{2}+\left(z_{1}+z_{2}\right)^{2}
	$$
	
	由柯西不等式, 得
	
	$$
	\begin{aligned}
	\left(x_{1}+x_{2}\right)\left(y_{1}+y_{2}\right) & \geqslant\left(\sqrt{x_{1} y_{1}}+\sqrt{x_{2} y_{2}}\right)^{2} \\
	& =\left(\sqrt{u_{1}^{2}+z_{1}^{2}}+\sqrt{u_{2}^{2}+z_{2}^{2}}\right)^{2} \\
	& =\left(u_{1}^{2}+z_{1}^{2}\right)+2 \sqrt{u_{1}^{2}+z_{1}^{2}} \sqrt{u_{2}^{2}+z_{2}^{2}}+\left(u_{2}^{2}+z_{2}^{2}\right) \\
	& \geqslant\left(u_{1}^{2}+z_{1}^{2}\right)+2\left(u_{1} u_{2}+z_{1} z_{2}\right)+\left(u_{2}^{2}+z_{2}^{2}\right) \\
	& =\left(u_{1}+u_{2}\right)^{2}+\left(z_{1}+z_{2}\right)^{2} .
	\end{aligned}
	$$
	
	从而原不等式成立, 且等号成立的充分必要条件为 $x_{1}=x_{2}, y_{1}=y_{2}$, $z_{1}=z_{2}$.
\end{proof}
\begin{note}
	该例题也可以直接用柯西不等式证明, 关于它的推广, 可以参见引文或练习。
\end{note}

\begin{example}
	设 $a_{i}, b_{i}, c_{i}, d_{i} \in \mathbf{R}_{+}(i=1,2, \cdots, n)$, 求证:
	
	$$
	\left(\sum a_{i} b_{i} c_{i} d_{i}\right)^{4} \leqslant \sum a_{i}^{4} \cdot \sum b_{i}^{4} \cdot \sum c_{i}^{4} \cdot \sum d_{i}^{4}
	$$
\end{example}
\begin{proof}
	两次利用柯西不等式, 得
	
	$$
	\begin{aligned}
	\text { 左边 } & =\left[\sum\left(a_{i} b_{i}\right)\left(c_{i} d_{i}\right)\right]^{4} \\
	& \leqslant\left[\sum\left(a_{i} b_{i}\right)^{2}\right]^{2}\left[\sum\left(c_{i} d_{i}\right)^{2}\right]^{2} \\
	& =\left[\sum a_{i}^{2} b_{i}^{2}\right]^{2}\left[\sum c_{i}^{2} d_{i}^{2}\right]^{2} \\
	& \leqslant \sum a_{i}^{4} \cdot \sum b_{i}^{4} \cdot \sum c_{i}^{4} \cdot \sum d_{i}^{4}
	\end{aligned}
	$$
	
	故命题成立.
\end{proof}
\begin{note}
	前面, 我们用平均值不等式证明了这个不等式, 读者还可以用其他方法证明.
\end{note}

\begin{example}
	设 $t_{a} 、 t_{b} 、 t_{c}$ 分别是 $\triangle A B C$ 的 $\angle A 、 \angle B 、 \angle C$ 的角平分线的长,证明:
	
	$$
	\sum_{\text {cyc }} \frac{b c}{t_{a}^{2}} \geqslant 4
	$$
\end{example}
\begin{proof}
	不难求得 $t_{a}^{2}=\frac{b c\left[(b+c)^{2}-a^{2}\right]}{(b+c)^{2}}$, 则 $\frac{b c}{t_{a}^{2}}=\frac{(b+c)^{2}}{(b+c)^{2}-a^{2}}$.
	
	同理可得
	
	$$
	\frac{a c}{t_{b}^{2}}=\frac{(a+c)^{2}}{(a+c)^{2}-b^{2}}, \frac{a b}{t_{c}^{2}}=\frac{(a+b)^{2}}{(a+b)^{2}-c^{2}}
	$$
	
	则
	
	$$
	\sum_{\mathrm{cyc}} \frac{b c}{t_{a}^{2}}=\sum_{\mathrm{cyc}} \frac{(a+b)^{2}}{(a+b)^{2}-c^{2}} \geqslant \frac{4(a+b+c)^{2}}{\sum_{\mathrm{cyc}}\left[(a+b)^{2}-c^{2}\right]}=\frac{4(a+b+c)^{2}}{(a+b+c)^{2}}=4
	$$
	
	所以原不等式成立.
\end{proof}
\begin{note}
	前面, 我们用平均值不等式证明了这个不等式, 读者还可以用其他方法证明.
\end{note}

\begin{example}
	设 $x, y, z, w \in \mathbf{R}_{+}, \alpha, \beta, \gamma, \theta$ 满足 $\alpha+\beta+\gamma+\theta=(2 k+1) \pi$, $k \in \mathbf{Z}$.
	
	求证 $\quad(x \sin \alpha+y \sin \beta+z \sin \gamma+w \sin \theta)^{2}$
	
	$$
	\leqslant \frac{(x y+z w)(x z+y w)(x w+y z)}{x y z w}
	$$
\end{example}
\begin{proof}
	令 $u=x \sin \alpha+y \sin \beta, v=z \sin \gamma+w \sin \theta$, 则
	
	$$
	\begin{aligned}
	u^{2} & =(x \sin \alpha+y \sin \beta)^{2} \\
	& \leqslant(x \sin \alpha+y \sin \beta)^{2}+(x \cos \alpha-y \cos \beta)^{2} \\
	& =x^{2}+y^{2}-2 x y \cos (\alpha+\beta)
	\end{aligned}
	$$
	
	于是
	
	$$
	\cos (\alpha+\beta) \leqslant \frac{x^{2}+y^{2}-u^{2}}{2 x y}
	$$
	
	同理
	
	$$
	\cos (\gamma+\theta) \leqslant \frac{z^{2}+w^{2}-v^{2}}{2 z w}
	$$
	
	两式相加, 并由假设得
	
	$$
	0=\cos (\alpha+\beta)+\cos (\gamma+\theta) \leqslant \frac{x^{2}+y^{2}-u^{2}}{2 x y}+\frac{z^{2}+w^{2}-v^{2}}{2 z w}
	$$
	
	即
	
	$$
	\frac{u^{2}}{x y}+\frac{v^{2}}{z w} \leqslant \frac{x^{2}+y^{2}}{x y}+\frac{z^{2}+w^{2}}{z w}
	$$
	
	从而
	
	$$
	(x \sin \alpha+y \sin \beta+z \sin \gamma+w \sin \theta)^{2}=(u+v)^{2}
	$$
	
	$$
	\begin{aligned}
	& \leqslant(x y+z w)\left(\frac{u^{2}}{x y}+\frac{v^{2}}{z w}\right) \\
	& \leqslant(x y+z w)\left(\frac{x^{2}+y^{2}}{x y}+\frac{z^{2}+w^{2}}{z w}\right) \\
	& =\frac{(x y+z w)(x z+y w)(x w+y z)}{x y z w} .
	\end{aligned}
	$$
	
	故命题成立.
\end{proof}
\begin{note}
	前面, 我们用平均值不等式证明了这个不等式, 读者还可以用其他方法证明.
\end{note}

\begin{example}
	给定正整数 $n \geqslant 2$, 设正整数 $a_{i}(i=1,2, \cdots, n)$ 满足 $a_{1}<$ $a_{2}<\cdots<a_{n}$ 以及 $\sum_{i=1}^{n} \frac{1}{a_{i}} \leqslant 1$. 求证: 对任意实数 $x$, 有
	
	$$
	\left(\sum_{i=1}^{n} \frac{1}{a_{i}^{2}+x^{2}}\right)^{2} \leqslant \frac{1}{2} \cdot \frac{1}{a_{1}\left(a_{1}-1\right)+x^{2}}
	$$
\end{example}
\begin{proof}
	当 $x^{2} \geqslant a_{1}\left(a_{1}-1\right)$ 时, 由于 $\sum \frac{1}{a_{i}} \leqslant 1$, 得
	
	$$
	\begin{aligned}
	\left(\sum_{i=1}^{n} \frac{1}{a_{i}^{2}+x^{2}}\right)^{2} & \leqslant\left(\sum_{i=1}^{n} \frac{1}{2 a_{i}|x|}\right)^{2}=\frac{1}{4 x^{2}}\left(\sum_{i=1}^{n} \frac{1}{a_{i}}\right)^{2} \\
	& \leqslant \frac{1}{4 x^{2}} \leqslant \frac{1}{2} \cdot \frac{1}{a_{1}\left(a_{1}-1\right)+x^{2}}
	\end{aligned}
	$$
	
	当 $x^{2}<a_{1}\left(a_{1}-1\right)$ 时, 由柯西不等式, 得
	
	$$
	\left(\sum_{i=1}^{n} \frac{1}{a_{i}^{2}+x^{2}}\right)^{2} \leqslant\left(\sum_{i=1}^{n} \frac{1}{a_{i}}\right) \sum_{i=1}^{n} \frac{a_{i}}{\left(a_{i}^{2}+x^{2}\right)^{2}} \leqslant \sum_{i=1}^{n} \frac{a_{i}}{\left(a_{i}^{2}+x^{2}\right)^{2}}
	$$
	
	对于正整数 $a_{1}<a_{2}<\cdots<a_{n}$, 有 $a_{i+1} \geqslant a_{i}+1, i=1,2, \cdots, n-1$,且.
	
	$$
	\begin{aligned}
	\frac{2 a_{i}}{\left(a_{i}^{2}+x^{2}\right)^{2}} & \leqslant \frac{2 a_{i}}{\left(a_{i}^{2}+x^{2}+\frac{1}{4}\right)^{2}-a_{i}^{2}} \\
	& =\frac{1}{\left(a_{i}-\frac{1}{2}\right)^{2}+x^{2}}-\frac{1}{\left(a_{i}+\frac{1}{2}\right)^{2}+x^{2}} \\
	& \leqslant \frac{1}{\left(a_{i}-\frac{1}{2}\right)^{2}+x^{2}}-\frac{1}{\left(a_{i+1}-\frac{1}{2}\right)^{2}+x^{2}}, i=1,2, \cdots, n-1
	\end{aligned}
	$$
	
	同理
	
	$$
	\begin{aligned}
	\frac{2 a_{n}}{\left(a_{n}^{2}+x^{2}\right)^{2}} & \leqslant \frac{1}{\left(a_{n}-\frac{1}{2}\right)^{2}+x^{2}}-\frac{1}{\left(a_{n}+\frac{1}{2}\right)^{2}+x^{2}} \\
	& \leqslant \frac{1}{\left(a_{n}-\frac{1}{2}\right)^{2}+x^{2}}
	\end{aligned}
	$$
	
	所以
	
	$$
	\begin{aligned}
	\sum_{i=1}^{n} \frac{a_{i}}{\left(a_{i}^{2}+x^{2}\right)^{2}} \leqslant & \frac{1}{2} \sum_{i=1}^{n-1}\left[\frac{1}{\left(a_{i}-\frac{1}{2}\right)^{2}+x^{2}}-\frac{1}{\left(a_{i+1}-\frac{1}{2}\right)^{2}+x^{2}}\right] \\
	& +\frac{1}{\left(a_{n}-\frac{1}{2}\right)^{2}+x^{2}} \\
	\leqslant & \frac{1}{2} \cdot \frac{1}{\left(a_{1}-\frac{1}{2}\right)^{2}+x^{2}} \leqslant \frac{1}{2} \cdot \frac{1}{a_{1}\left(a_{1}-1\right)+x^{2}}
	\end{aligned}
	$$
	
	故命题成立.
\end{proof}
\begin{note}
	前面, 我们用平均值不等式证明了这个不等式, 读者还可以用其他方法证明.
\end{note}

\begin{example}
	设 $x \in\left(0, \frac{\pi}{2}\right), n \in \mathbf{N}$, 求证:
	
	$$
	\left(\frac{1-\sin ^{2 n} x}{\sin ^{2 n} x}\right)\left(\frac{1-\cos ^{2 n} x}{\cos ^{2 n} x}\right) \geqslant\left(2^{n}-1\right)^{2}
	$$
	
	
	$$
	\begin{aligned}
	1-\sin ^{2 n} x & =\left(1-\sin ^{2} x\right)\left(1+\sin ^{2} x+\sin ^{4} x+\cdots+\sin ^{2(n-1)} x\right) \\
	& =\cos ^{2} x\left(1+\sin ^{2} x+\sin ^{4} x+\cdots+\sin ^{2(n-1)} x\right) \\
	1-\cos ^{2 n} x & =\left(1-\cos ^{2} x\right)\left(1+\cos ^{2} x+\cos ^{4} x+\cdots+\cos ^{2(n-1)} x\right) \\
	& =\sin ^{2} x\left(1+\cos ^{2} x+\cos ^{4} x+\cdots+\cos ^{2(n-1)} x\right)
	\end{aligned}
	$$
	
	所以由柯西不等式, 得
	
	$$
	\begin{aligned}
	& \left(\frac{1-\sin ^{2 n} x}{\sin ^{2 n} x}\right)\left(\frac{1-\cos ^{2 n} x}{\cos ^{2 n} x}\right) \\
	= & \frac{1}{\sin ^{2 n-2} x}\left(1+\sin ^{2} x+\sin ^{4} x+\cdots+\sin ^{2 n-2} x\right) \\
	& \cdot \frac{1}{\cos ^{2 n-2} x}\left(1+\cos ^{2} x+\cos ^{4} x+\cdots+\cos ^{2 n-2} x\right) \\
	= & \left(1+\frac{1}{\sin ^{2} x}+\frac{1}{\sin ^{4} x}+\cdots+\frac{1}{\sin ^{2 n-2} x}\right)
	\end{aligned}
	$$
	
	$$
	\begin{aligned}
	& \cdot\left(1+\frac{1}{\cos ^{2} x}+\frac{1}{\cos ^{4} x}+\cdots+\frac{1}{\cos ^{2 n-2} x}\right) \\
	\geqslant & {\left[1+\frac{1}{\sin x \cos x}+\frac{1}{(\sin x \cos x)^{2}}+\cdots+\frac{1}{(\sin x \cos x)^{n-1}}\right]^{2} } \\
	= & {\left[1+\frac{2}{\sin 2 x}+\frac{4}{(\sin 2 x)^{2}}+\cdots+\frac{2^{n-1}}{(\sin 2 x)^{n-1}}\right]^{2} } \\
	\geqslant & \left(1+2+2^{2}+\cdots+2^{n-1}\right)^{2}=\left(2^{n}-1\right)^{2}
	\end{aligned}
	$$
\begin{example}	
	例 21 设 $a_{1}, a_{2}, \cdots, a_{n}$ 是一个有无穷项的实数列, 对于所有正整数 $i$,存在一个实数 $c$, 使得 $0 \leqslant a_{i} \leqslant c$, 且 $\left|a_{i}-a_{j}\right| \geqslant \frac{1}{i+j}$ 对所有正整数 $i 、 j$ $(i \neq j)$ 成立. 证明: $c \geqslant 1$.
\end{example}
\begin{proof}
	因为
	$$
	\begin{aligned}
	1-\sin ^{2 n} x & =\left(1-\sin ^{2} x\right)\left(1+\sin ^{2} x+\sin ^{4} x+\cdots+\sin ^{2(n-1)} x\right) \\
	& =\cos ^{2} x\left(1+\sin ^{2} x+\sin ^{4} x+\cdots+\sin ^{2(n-1)} x\right) \\
	1-\cos ^{2 n} x & =\left(1-\cos ^{2} x\right)\left(1+\cos ^{2} x+\cos ^{4} x+\cdots+\cos ^{2(n-1)} x\right) \\
	& =\sin ^{2} x\left(1+\cos ^{2} x+\cos ^{4} x+\cdots+\cos ^{2(n-1)} x\right)
	\end{aligned}
	$$
	
	所以由柯西不等式, 得
	
	$$
	\begin{aligned}
	& \left(\frac{1-\sin ^{2 n} x}{\sin ^{2 n} x}\right)\left(\frac{1-\cos ^{2 n} x}{\cos ^{2 n} x}\right) \\
	= & \frac{1}{\sin ^{2 n-2} x}\left(1+\sin ^{2} x+\sin ^{4} x+\cdots+\sin ^{2 n-2} x\right) \\
	& \cdot \frac{1}{\cos ^{2 n-2} x}\left(1+\cos ^{2} x+\cos ^{4} x+\cdots+\cos ^{2 n-2} x\right) \\
	= & \left(1+\frac{1}{\sin ^{2} x}+\frac{1}{\sin ^{4} x}+\cdots+\frac{1}{\sin ^{2 n-2} x}\right)
	\end{aligned}
	$$
	
	$$
	\begin{aligned}
	& \cdot\left(1+\frac{1}{\cos ^{2} x}+\frac{1}{\cos ^{4} x}+\cdots+\frac{1}{\cos ^{2 n-2} x}\right) \\
	\geqslant & {\left[1+\frac{1}{\sin x \cos x}+\frac{1}{(\sin x \cos x)^{2}}+\cdots+\frac{1}{(\sin x \cos x)^{n-1}}\right]^{2} } \\
	= & {\left[1+\frac{2}{\sin 2 x}+\frac{4}{(\sin 2 x)^{2}}+\cdots+\frac{2^{n-1}}{(\sin 2 x)^{n-1}}\right]^{2} } \\
	\geqslant & \left(1+2+2^{2}+\cdots+2^{n-1}\right)^{2}=\left(2^{n}-1\right)^{2}
	\end{aligned}
	$$
\end{proof}
\begin{note}
	前面, 我们用平均值不等式证明了这个不等式, 读者还可以用其他方法证明.
\end{note}

\begin{example}
	设 $a 、 b 、 c$ 为正实数. 求证:
	
	$$
	\frac{(2 a+b+c)^{2}}{2 a^{2}+(b+c)^{2}}+\frac{(a+2 b+c)^{2}}{2 b^{2}+(a+c)^{2}}+\frac{(a+b+2 c)^{2}}{2 c^{2}+(b+a)^{2}} \leqslant 8
	$$
\end{example}
\begin{proof}
	在第二章中, 我们用了两种不同的方法证明了这个不等式, 这里,用柯西不等式给出另一种新的证明.
	
	由柯西不等式, 得
	
	$$
	\begin{aligned}
	\sqrt{\frac{2 a^{2}+\frac{(b+c)^{2}}{2}+\frac{(b+c)^{2}}{2}}{3}} & \geqslant \frac{\sqrt{2} a+\frac{\sqrt{2}}{2}(b+c)+\frac{\sqrt{2}}{2}(b+c)}{3} \\
	& =\frac{\sqrt{2}(a+b+c)}{3} .
	\end{aligned}
	$$
	
	于是 $2 a^{2}+(b+c)^{2} \geqslant \frac{2(a+b+c)^{2}}{3}$. 同理可得
	
	$$
	2 b^{2}+(c+a)^{2} \geqslant \frac{2(a+b+c)^{2}}{3}, 2 c^{2}+(a+b)^{2} \geqslant \frac{2(a+b+c)^{2}}{3}
	$$
	
	如果 $4 a \geqslant b+c, 4 b \geqslant c+a, 4 c \geqslant a+b$, 则
	
	$$
	\begin{aligned}
	\frac{(2 a+b+c)^{2}}{2 a^{2}+(b+c)^{2}} & =2+\frac{(4 a-b-c)(b+c)}{2 a^{2}+(b+c)^{2}} \\
	& \leqslant 2+\frac{3\left(4 a b+4 a c-b^{2}-2 b c-c^{2}\right)}{2(a+b+c)^{2}}
	\end{aligned}
	$$
	
	同理可得
	
	$$
	\begin{aligned}
	& \frac{(a+2 b+c)^{2}}{2 b^{2}+(a+c)^{2}} \leqslant 2+\frac{3\left(4 b c+4 b a-a^{2}-2 a c-c^{2}\right)}{2(a+b+c)^{2}} \\
	& \frac{(a+b+2 c)^{2}}{2 c^{2}+(a+b)^{2}} \leqslant 2+\frac{3\left(4 c b+4 c a-a^{2}-2 b a-b^{2}\right)}{2(a+b+c)^{2}}
	\end{aligned}
	$$
	
	三式相加, 得
	
	$$
	\begin{aligned}
	& \frac{(2 a+b+c)^{2}}{2 a^{2}+(b+c)^{2}}+\frac{(a+2 b+c)^{2}}{2 b^{2}+(a+c)^{2}}+\frac{(a+b+2 c)^{2}}{2 c^{2}+(a+b)^{2}} \\
	\leqslant & 6+\frac{3\left(6 a b+6 b c+6 c a-2 a^{2}-2 b^{2}-2 c^{2}\right)}{2(a+b+c)^{2}} \\
	= & \frac{21}{2}-\frac{15}{2} \cdot \frac{a^{2}+b^{2}+c^{2}}{(a+b+c)^{2}} \\
	\leqslant & \frac{21}{2}-\frac{15}{2} \times \frac{1}{3}=8 .
	\end{aligned}
	$$
	
	当上述假设不成立时, 不妨设 $4 a<b+c$, 则
	
	$$
	\frac{(2 a+b+c)^{2}}{2 a^{2}+(b+c)^{2}}<2
	$$
	
	由柯西不等式, 得
	
	$$
	[b+b+(c+a)]^{2} \leqslant\left(b^{2}+b^{2}+(c+a)^{2}\right)(1+1+1)
	$$
	
	于是
	
	$$
	\frac{(2 b+a+c)^{2}}{2 b^{2}+(a+c)^{2}} \leqslant 3
	$$
	
	同理可得
	
	$$
	\frac{(2 c+b+a)^{2}}{2 c^{2}+(a+b)^{2}} \leqslant 3
	$$
	
	所以
	
	$$
	\frac{(2 a+b+c)^{2}}{2 a^{2}+(b+c)^{2}}+\frac{(a+2 b+c)^{2}}{2 b^{2}+(a+c)^{2}}+\frac{(a+b+2 c)^{2}}{2 c^{2}+(a+b)^{2}} \leqslant 8
	$$
	
	综上可知原不等式成立. 当且仅当 $a=b=c$ 时等号成立.
	
	对于三种不同的证明方法,希望大家能好好理解.
\end{proof}
\begin{note}
	前面, 我们用平均值不等式证明了这个不等式, 读者还可以用其他方法证明.
\end{note}

\begin{example}
	设正整数 $n \geqslant 2, a_{i} \in \mathbf{R}, 1 \leqslant i \leqslant n$. 证明: 可以选择 $\varepsilon_{i} \in\{-1$, $1\}, 1 \leqslant i \leqslant n$. 使得
	
	$$
	\left(\sum_{i=1}^{n} a_{i}\right)^{2}+\left(\sum_{i=1}^{n} \varepsilon_{i} a_{i}\right)^{2} \leqslant(n+1) \sum_{i=1}^{n} a_{i}^{2}
	$$
	
	证法一 取 $\varepsilon_{i}=1,1 \leqslant i \leqslant\left[\frac{n}{2}\right] ; \varepsilon_{i}=-1,\left[\frac{n}{2}\right]+1 \leqslant i \leqslant n$, 这里 $[x]$为实数 $x$ 的整数部分. 则
	
	$$
	\begin{aligned}
	& \left(\sum_{i=1}^{n} a_{i}\right)^{2}+\left(\sum_{i=1}^{n} \varepsilon_{i} a_{i}\right)^{2} \\
	= & \left(\sum_{i=1}^{\left[\frac{n}{2}\right]} a_{i}+\sum_{j=\left[\frac{n}{2}\right]+1}^{n} a_{j}\right)^{2}+\left(\sum_{i=1}^{\left[\frac{n}{2}\right]} a_{i}-\sum_{j=\left[\frac{n}{2}\right]+1}^{n} a_{j}\right)^{2} \\
	= & 2\left(\sum_{i=1}^{\left[\frac{n}{2}\right]} a_{i}\right)^{2}+2\left(\sum_{j=\left[\frac{n}{2}\right]+1}^{n} a_{j}\right)^{2} \\
	\leqslant & 2\left[\frac{n}{2}\right]\left(\sum_{i=1}^{\left[\frac{n}{2}\right]} a_{i}^{2}\right)+2\left(n-\left[\frac{n}{2}\right]\right)\left(\sum_{j=\left[\frac{n}{2}\right]+1}^{n} a_{j}^{2}\right)(\text { 柯西不等式 }) \\
	= & 2\left[\frac{n}{2}\right]\left(\sum_{i=1}^{\left[\frac{n}{2}\right]} a_{i}^{2}\right)+2\left(\left[\frac{n+1}{2}\right]\right)\left(\sum_{j=\left[\frac{n}{2}\right]+1}^{n} a_{j}^{2}\right)\left(n-\left[\frac{n}{2}\right]=\left[\frac{n+1}{2}\right]\right)
	\end{aligned}
	$$
	
	$\leqslant n \sum_{i=1}^{\left[\frac{n}{2}\right]} a_{i}^{2}+(n+1) \sum_{j=\left[\frac{n}{2}\right]+1}^{n} a_{j}^{2}$
	
	$\leqslant(n+1) \sum_{i=1}^{n} a_{i}^{2}$.
	
	从而, 命题成立.
	
	证法二 由对称性, 不妨设 $a_{1} \geqslant a_{2} \geqslant \cdots \geqslant a_{n}$. 此外, 若将 $a_{1}, \cdots, a_{n}$ 中负数均改变符号, 由于 $\left(\sum_{i=1}^{n} a_{i}\right)^{2}$ 不减, $\sum_{i=1}^{n} a_{i}^{2}$ 不变, 且不影响 $\varepsilon_{i}= \pm 1$ 的选取.所以,可进一步假设 $a_{1} \geqslant a_{2} \geqslant \cdots \geqslant a_{n} \geqslant 0$.
	
	引理 $\quad$ 设 $a_{1} \geqslant a_{2} \geqslant \cdots \geqslant a_{n} \geqslant 0$, 则 $0 \leqslant \sum_{i=1}^{n}(-1)^{i-1} a_{i} \leqslant a_{1}$.
	
	事实上, 由于 $a_{i} \geqslant a_{i+1}, 1 \leqslant i \leqslant n-1$. 则当 $n$ 为偶数时,
	
	$$
	\begin{aligned}
	& \sum_{i=1}^{n}(-1)^{i-1} a_{i}=\left(a_{1}-a_{2}\right)+\cdots+\left(a_{n-1}-a_{n}\right) \geqslant 0 \\
	& \sum_{i=1}^{n}(-1)^{i-1} a_{i}=a_{1}-\left(a_{2}-a_{3}\right)-\cdots-\left(a_{n-2}-a_{n-1}\right)-a_{n} \leqslant a_{1}
	\end{aligned}
	$$
	
	当 $n$ 为奇数时
	
	$$
	\begin{aligned}
	& \sum_{i=1}^{n}(-1)^{i-1} a_{i}=\left(a_{1}-a_{2}\right)+\cdots+\left(a_{n-2}-a_{n-1}\right)+a_{n} \geqslant 0 \\
	& \sum_{i=1}^{n}(-1)^{i-1} a_{i}=a_{1}-\left(a_{2}-a_{3}\right)-\cdots-\left(a_{n-1}-a_{n}\right) \leqslant a_{1}
	\end{aligned}
	$$
	
	所以,引理成立.
	
	由柯西不等式和引理得
	
	$$
	\left(\sum_{i=1}^{n} a_{i}\right)^{2}+\left(\sum_{i=1}^{n}(-1)^{i-1} a_{i}\right)^{2} \leqslant n \sum_{i=1}^{n} a_{i}^{2}+a_{1}^{2} \leqslant(n+1) \sum_{i=1}^{n} a_{i}^{2}
	$$
	
	故命题成立.
	
	利用变形的赫尔德不等式可以证明下列不等式.
	
	例 24 证明: 对正实数 $a 、 b 、 c$, 有
	
	$$
	\frac{a}{\sqrt{a^{2}+8 b c}}+\frac{b}{\sqrt{b^{2}+8 a c}}+\frac{c}{\sqrt{c^{2}+8 a b}} \geqslant 1
	$$
\end{example}
\begin{proof}
	由变形的柯西不等式,
	
	$$
	\text { 左边 }=\sum_{\text {cyc }} \frac{a}{\sqrt{a^{2}+8 b c}}=\sum_{\text {cyc }} \frac{a^{\frac{3}{2}}}{\sqrt{a^{3}+8 a b c}} \geqslant \frac{\left(\sum_{\mathrm{cyc}} a\right)^{\frac{3}{2}}}{\left[\sum_{\mathrm{cyc}}\left(a^{3}+8 a b c\right)\right]^{\frac{1}{2}}} \text {. }
	$$
	
	所以要证明原不等式, 只需要证明
	
	$$
	\frac{\left(\sum_{\mathrm{cyc}} a\right)^{\frac{3}{2}}}{\left[\sum_{\text {cyc }}\left(a^{3}+8 a b c\right)\right]^{\frac{1}{2}}} \geqslant 1
	$$
	
	等价于
	
	$$
	\left(\sum_{\text {cyc }} a\right)^{3} \geqslant \sum_{\text {cyc }} a^{3}+24 a b c
	$$
	
	等价于
	
	$$
	\sum_{\text {cyc }} a^{3}+3 \sum_{\text {cyc }}\left(a^{2} b+a b^{2}\right)+6 a b c \geqslant \sum_{\text {cyc }} a^{3}+24 a b c
	$$
	
	等价于
	
	$$
	\sum_{c y c}\left(a^{2} b+a b^{2}\right) \geqslant 6 a b c
	$$
	
	易知该不等式成立, 故原不等式成立.
\end{proof}
\begin{note}
	前面, 我们用平均值不等式证明了这个不等式, 读者还可以用其他方法证明.
\end{note}

\begin{example}
	已知 $x 、 y 、 z$ 是正实数, 且 $x y z=x+y+z+2$. 求证:
	
	$$
	2(\sqrt{x y}+\sqrt{y z}+\sqrt{z x}) \leqslant x+y+z+6 .
	$$
\end{example}
\begin{proof}
	由于 $x y z=x+y+z+2, z=\frac{x+y+2}{x y-1}, x y>1$.
	
	设 $\frac{x+1}{y+1}=\frac{b}{a}, a, b \in \mathbf{R}_{+}$, 则 $(x+1) a=(y+1) b$.
	
	令 $(x+1) a=(y+1) b=S$, 则 $S>a, S>b$,
	
	$$
	x=\frac{S-a}{a}, y=\frac{S-b}{b}
	$$
	
	由 $x y>1$, 得 $(S-a)(S-b)>a b$,
	
	$$
	\begin{aligned}
	& S^{2}>(a+b) S \\
	& S>a+b
	\end{aligned}
	$$
	
	设 $c=S-(a+b), c>0$, 于是
	
	$$
	\begin{aligned}
	& x=\frac{b+c}{a}, y=\frac{c+a}{b} \\
	& z=\frac{\frac{b+c}{a}+\frac{c+a}{b}+2}{\frac{(b+c)(c+a)}{a b}-1}=\frac{b^{2}+b c+a^{2}+a c+2 a b}{c^{2}+a c+b c}=\frac{a+b}{c}
	\end{aligned}
	$$
	
	因为 $(\sqrt{x}+\sqrt{y}+\sqrt{z})^{2}=x+y+z+2(\sqrt{x y}+\sqrt{y z}+\sqrt{z x})$,所以 $2(\sqrt{x y}+\sqrt{y z}+\sqrt{z x})=(\sqrt{x}+\sqrt{y}+\sqrt{z})^{2}-(x+y+z)$.原不等式等价于
	
	$$
	\begin{aligned}
	& (\sqrt{x}+\sqrt{y}+\sqrt{z})^{2} \leqslant 2(x+y+z+3) \\
	\Leftrightarrow & \sqrt{x}+\sqrt{y}+\sqrt{z} \leqslant \sqrt{2(x+y+z+3)} \\
	\Leftrightarrow & \sqrt{\frac{b+c}{a}}+\sqrt{\frac{c+a}{b}}+\sqrt{\frac{a+b}{c}} \leqslant \sqrt{2\left(\frac{b+c}{a}+\frac{c+a}{b}+\frac{a+b}{c}+3\right)}
	\end{aligned}
	$$
	
	$$
	\text { 而 } 2\left(\frac{b+c}{a}+\frac{c+a}{b}+\frac{a+b}{c}+3\right)=2(a+b+c)\left(\frac{1}{a}+\frac{1}{b}+\frac{1}{c}\right) \text { , }
	$$
	
	因此只需证
	
	$$
	\sqrt{\frac{b+c}{a}}+\sqrt{\frac{c+a}{b}}+\sqrt{\frac{a+b}{c}} \leqslant \sqrt{2(a+b+c)\left(\frac{1}{a}+\frac{1}{b}+\frac{1}{c}\right)}
	$$
	
	由于
	
	$$
	\begin{aligned}
	& 2(a+b+c)\left(\frac{1}{a}+\frac{1}{b}+\frac{1}{c}\right) \\
	= & {[(a+b)+(b+c)+(c+a)] \cdot\left(\frac{1}{c}+\frac{1}{a}+\frac{1}{b}\right) } \\
	\geqslant & \left(\sqrt{\frac{a+b}{c}}+\sqrt{\frac{b+c}{a}}+\sqrt{\frac{c+a}{b}}\right)^{2}
	\end{aligned}
	$$
	
	故原不等式成立.
\end{proof}
\begin{note}
	本题的条件式和结论式都非齐次, 用 $x=\frac{b+c}{a}, y=\frac{c+a}{b}, z=$ $\frac{a+b}{c}$ 将条件与结论都变成了齐次式.
	
	例 $26 a 、 b 、 c$ 是非负实数, 但不全为 0 ,求证:
	
	$$
	\frac{a}{a+b+7 c}+\frac{b}{b+c+7 a}+\frac{c}{c+a+7 b}+\frac{2}{3} \cdot \frac{a b+b c+c a}{a^{2}+b^{2}+c^{2}} \leqslant 1
	$$
	
	证明 由于 $\frac{a}{a+b+c}-\frac{a}{a+b+7 c}=\frac{6 c a}{(a+b+c)(a+b+7 c)}$, 故
	
	$$
	\frac{a}{a+b+7 c}=\frac{a}{a+b+c}-\frac{6 c a}{(a+b+c)(a+b+7 c)}
	$$
	
	同理 $\frac{b}{b+c+7 a}=\frac{b}{a+b c}-\frac{6 a b}{(b+c+a)(b+c+7 a)}$,
	
	$$
	\frac{c}{c+a+7 b}=\frac{c}{c+a+b}-\frac{6 b c}{(c+a+b)(c+a+7 b)}
	$$
	
	三式相加, 得
	
	$$
	\begin{aligned}
	& \frac{a}{a+b+7 c}+\frac{b}{b+c+7 a}+\frac{c}{c+a+7 b} \\
	= & 1-\frac{6}{a+b+c}\left(\frac{c a}{a+b+7 c}+\frac{a b}{b+c+7 a}+\frac{b c}{c+a+7 b}\right)
	\end{aligned}
	$$
	
	因此, 要证原不等式, 只需证明
	
	$$
	\begin{aligned}
	& \frac{2}{3} \cdot \frac{a b+b c+c a}{a^{2}+b^{2}+c^{2}} \leqslant \frac{6}{a+b+c}\left(\frac{c a}{a+b+7 c}+\frac{a b}{b+c+7 a}+\frac{b c}{c+a+7 b}\right) \\
	& \Leftrightarrow \frac{c a}{a+b+7 c}+\frac{a b}{b+c+7 a}+\frac{b c}{c+a+7 b} \geqslant \frac{(a+b+c)(a b+b c+c a)}{9\left(a^{2}+b^{2}+c^{2}\right)}
	\end{aligned}
	$$
	
	由柯西不等式
	
	$$
	\begin{aligned}
	& {[c a(a+b+7 c)+a b(b+c+7 a)+b c(c+a+7 b)]} \\
	& \quad\left[\frac{c a}{a+b+7 c}+\frac{a b}{b+c+7 a}+\frac{b c}{c+a+7 b}\right] \\
	& \geqslant(c a+a b+b c)^{2}
	\end{aligned}
	$$
	
	故
	
	$$
	\begin{aligned}
	& \frac{c a}{a+b+7 c}+\frac{a b}{b+c+7 a}+\frac{b c}{c+a+7 b} \\
	\geqslant & \frac{(c a+a b+b c)^{2}}{a b^{2}+b c^{2}+c a^{2}+7\left(c^{2} a+a^{2} b+b^{2} c\right)+3 a b c}
	\end{aligned}
	$$
	
	因此, 只需证明
	
	$$
	\begin{aligned}
	& \quad \frac{(a b+b c+c a)^{2}}{a b^{2}+b c^{2}+c a^{2}+7\left(a^{2} b+b^{2} c+c^{2} a\right)+3 a b c} \geqslant \frac{(a+b+c)(a b+b c+c a)}{9\left(a^{2}+b^{2}+c^{2}\right)} \\
	& \Leftrightarrow 9\left(a^{2}+b^{2}+c^{2}\right)(a b+b c+c a) \geqslant(a+b+c)\left[a b^{2}+b c^{2}+c a^{2}+7\left(a^{2} b+b^{2} c\right.\right. \\
	& \left.\left.\quad+c^{2} a\right)+3 a b c\right] \\
	& \Leftrightarrow 9\left[a^{3} b+a^{3} c+b^{3} a+b^{3} c+c^{3} a+c^{3} b+a b c(a+b+c)\right] \\
	& \quad \geqslant\left[a^{3} c+b^{3} a+c^{3} b+a^{2} b^{2}+b^{2} c^{2}+c^{2} a^{2}+a b c(a+b+c)\right]+7\left[a^{3} b+b^{3} c\right. \\
	& \left.\quad+c^{3} a+a^{2} c^{2}+b^{2} c^{2}+a^{2} b^{2}+a b c(a+b+c)\right]+3 a b c(a+b+c) \\
	& \Leftrightarrow 2\left(a^{3} b+b^{3} c+c^{3} a\right)+8\left(a^{3} c+b^{3} a+c^{3} b\right) \geqslant 8\left(a^{2} c^{2}+b^{2} c^{2}+a^{2} b^{2}\right)+2 a b c(a
	\end{aligned}
	$$
	
	$+b+c)$
	
	$\Leftrightarrow a^{3} b+b^{3} c+c^{3} a+4\left(a^{3} c+b^{3} a+c^{3} b\right) \geqslant 4\left(a^{2} c^{2}+b^{2} c^{2}+a^{2} b^{2}\right)+a b c(a+b$ $+c)$
	
	$\Leftrightarrow\left(a^{3} b+4 b^{3} a-4 a^{2} b^{2}\right)+\left(b^{3} c+4 c^{3} b-4 b^{2} c^{2}\right)+\left(c^{3} a+4 a^{3} c-4 a^{2} c^{2}\right) \geqslant a b c(a$
	
	$+b+c)$
	
	$\Leftrightarrow a b(a-2 b)^{2}+b c(b-2 c)^{2}+c a(c-2 a)^{2} \geqslant a b c(a+b+c)$
	
	由于 $\left[a b(a-2 b)^{2}+b c(b-2 c)^{2}+c a(c-2 a)^{2}\right](c+a+b)$
	
	$\geqslant[\sqrt{a b c}(a-2 b)+\sqrt{a b c}(b-2 c)+\sqrt{a b c}(c-2 a)]^{2}=a b c(a+b+c)^{2}$故原不等式成立.
\end{note}

\begin{example}
	对一切正实数 $a 、 b 、 c$. 求证:
	
	$$
	\frac{a^{2}}{b}+\frac{b^{2}}{c}+\frac{c^{2}}{a}+a+b+c \geqslant \frac{6\left(a^{2}+b^{2}+c^{2}\right)}{a+b+c}
	$$
\end{example}
\begin{proof}
	不妨设 $a \geqslant b \geqslant c$.
	
	原不等式等价于
	
	$$
	\begin{aligned}
	& \left(\frac{a^{2}}{b}+b-2 a\right)+\left(\frac{b^{2}}{c}+c-2 b\right)+\left(\frac{c^{2}}{a}+a-2 c\right) \geqslant \frac{6\left(a^{2}+b^{2}+c^{2}\right)}{a+b+c}-2(a+b+c) \\
	\Leftrightarrow & \frac{(a-b)^{2}}{b}+\frac{(b-c)^{2}}{c}+\frac{(c-a)^{2}}{a} \geqslant 2 \cdot \frac{(a-b)^{2}+(b-c)^{2}+(c-a)^{2}}{a+b+c} \\
	\Leftrightarrow & (a-b)^{2}\left(\frac{1}{b}-\frac{2}{a+b+c}\right)+(b-c)^{2}\left(\frac{1}{c}-\frac{2}{a+b+c}\right)+(c-a)^{2}\left(\frac{1}{a}-\frac{2}{a+b+c}\right) \geqslant 0
	\end{aligned}
	$$
	
	设 $S_{a}=\frac{1}{a}-\frac{2}{a+b+c}, S_{b}=\frac{1}{b}-\frac{2}{a+b+c}, S_{c}=\frac{1}{c}-\frac{2}{a+b+c}$.
	
	又设 $x=a-b, y=b-c$, 则原不等式等价于
	
	$x^{2} S_{b}+y^{2} S_{c}+(x+y)^{2} S_{a}=\left(S_{a}+S_{b}\right) x^{2}+\left(S_{a}+S_{c}\right) y^{2}+2 x y S_{a}$.
	
	显然 $S_{b}>0, S_{c}>0$, 由于 $\frac{1}{a}+\frac{1}{b}+\frac{1}{c} \geqslant \frac{9}{a+b+c}$, 故 $S_{a}+S_{b}+S_{c}>0$.
	
	$$
	\begin{aligned}
	& S_{a}+S_{c}=\frac{1}{a}+\frac{1}{c}-\frac{4}{a+b+c} \geqslant \frac{4}{a+c}-\frac{4}{a+b+c}>0 \\
	& S_{a}+S_{b}=\frac{1}{a}+\frac{1}{b}-\frac{4}{a+b+c} \geqslant \frac{4}{a+b}-\frac{4}{a+b+c}>0, \\
	& \Delta=\left(2 S_{a}\right)^{2}-4\left(S_{a}+S_{b}\right)\left(S_{a}+S_{c}\right)=-4\left(S_{a} S_{b}+S_{b} S_{c}+S_{c} S_{a}\right)
	\end{aligned}
	$$
	
	下证: $S_{a} S_{b}+S_{b} S_{c}+S_{a} S_{c} \geqslant 0$.
	
	即证: $\left(\frac{1}{a}-\frac{2}{a+b+c}\right)\left(\frac{1}{b}-\frac{2}{a+b+c}\right)+\left(\frac{1}{b}-\frac{2}{a+b+c}\right)\left(\frac{1}{c}-\right.$
	
	$$
	\begin{aligned}
	& \left.\frac{2}{a+b+c}\right)+\left(\frac{1}{c}-\frac{2}{a+b+c}\right)\left(\frac{1}{a}-\frac{2}{a+b+c}\right) \geqslant 0 \\
	& \quad \text { 左边 }=\frac{1}{a b c(a+b+c)^{2}}\left[a^{3}+b^{3}+c^{3}-\left(a^{2} b+a^{2} c+b^{2} a+b^{2} c+c^{2} a+c^{2} b\right)\right. \\
	& +6 a b c] \geqslant 0
	\end{aligned}
	$$
\end{proof}
\begin{note}
	用类似的方法可以证明下面的命题:
	
	设 $n(\geqslant 3)$ 为正整数, $a 、 b$ 为给定的实数,实数 $x_{0}, x_{1}, x_{2}, \cdots, x_{n}$ 满足
	
	$$
	\begin{aligned}
	& x_{0}+x_{1}+x_{2}+\cdots+x_{n}=a, \\
	& x_{0}^{2}+x_{1}^{2}+x_{2}^{2}+\cdots+x_{n}^{2}=b,
	\end{aligned}
	$$
	
	则当 $b<\frac{a^{2}}{n+1}$ 时, $x_{0}$ 不存在;
	
	$$
	\begin{aligned}
	& \text { 当 } b=\frac{a^{2}}{n+1} \text { 时, } x_{0}=\frac{a}{n+1} \text {; } \\
	& \text { 当 } b>\frac{a^{2}}{n+1} \text { 时, } x_{0} \text { 满足 } \\
	& \frac{a-\frac{1}{2} \sqrt{\delta}}{n+1} \leqslant x_{0} \leqslant \frac{a+\frac{1}{2} \sqrt{\delta}}{n+1}
	\end{aligned}
	$$
	
	其中 $\delta$ 为二次方程 $(n+1) x_{0}^{2}-2 a x_{0}+a^{2}-n b=0$ 的判别式.
\end{note}

\begin{example}
	设非负实数 $a_{1}, a_{2}, \cdots, a_{n}$ 满足 $a_{1}^{2}+a_{2}^{2}+\cdots+a_{n}^{2}=n$, 求证
	
	$$
	\sum_{i=1}^{n} \frac{1}{a_{i}^{2}+1} \leqslant \frac{n^{3}}{2\left(\sum_{i=1}^{n} a_{i}\right)^{2}}
	$$
\end{example}
\begin{proof}
	由柯西不等式
	
	$$
	\begin{aligned}
	& \frac{a_{1}^{2}}{a_{1}^{2}+a_{1}^{2}}+\frac{a_{2}^{2}}{a_{1}^{2}+a_{2}^{2}}+\frac{a_{3}^{2}}{a_{1}^{2}+a_{3}^{2}}+\cdots+\frac{a_{n}^{2}}{a_{1}^{2}+a_{n}^{2}} \\
	\geqslant & \frac{\left(a_{1}+a_{2}+\cdots+a_{n}\right)^{2}}{n a_{1}^{2}+\left(a_{1}^{2}+a_{2}^{2}+\cdots+a_{n}^{2}\right)} \\
	= & \frac{\left(\sum_{i=1}^{n} a_{i}\right)^{2}}{n a_{1}^{2}+n}
	\end{aligned}
	$$
	
	将上式中的 $a_{1}$ 换成 $a_{k}$, 这样的 $n$ 个式子相加, 得
	
	$$
	\begin{aligned}
	& \frac{n}{2}+\mathrm{C}_{n}^{2} \geqslant \sum_{i=1}^{n} \frac{\left(\sum_{i=1}^{n} a_{i}\right)^{2}}{n a_{i}^{2}+n} \\
	& \frac{n^{2}}{2} \geqslant \sum_{i=1}^{n} \frac{\left(\sum_{i=1}^{n} a_{i}\right)^{2}}{n a_{i}^{2}+n}
	\end{aligned}
	$$
	
	即 $\frac{n^{3}}{2\left(\sum_{i=1}^{n} a_{i}\right)^{2}} \geqslant \sum_{i=1}^{n} \frac{1}{a_{i}^{2}+1}$.
\end{proof}
\begin{note}
	用类似的方法可以证明下面的命题:
	
	设 $n(\geqslant 3)$ 为正整数, $a 、 b$ 为给定的实数,实数 $x_{0}, x_{1}, x_{2}, \cdots, x_{n}$ 满足
	
	$$
	\begin{aligned}
	& x_{0}+x_{1}+x_{2}+\cdots+x_{n}=a, \\
	& x_{0}^{2}+x_{1}^{2}+x_{2}^{2}+\cdots+x_{n}^{2}=b,
	\end{aligned}
	$$
	
	则当 $b<\frac{a^{2}}{n+1}$ 时, $x_{0}$ 不存在;
	
	$$
	\begin{aligned}
	& \text { 当 } b=\frac{a^{2}}{n+1} \text { 时, } x_{0}=\frac{a}{n+1} \text {; } \\
	& \text { 当 } b>\frac{a^{2}}{n+1} \text { 时, } x_{0} \text { 满足 } \\
	& \frac{a-\frac{1}{2} \sqrt{\delta}}{n+1} \leqslant x_{0} \leqslant \frac{a+\frac{1}{2} \sqrt{\delta}}{n+1}
	\end{aligned}
	$$
	
	其中 $\delta$ 为二次方程 $(n+1) x_{0}^{2}-2 a x_{0}+a^{2}-n b=0$ 的判别式.
\end{note}

\begin{example}
	设 $x, y, z>0$ 且 $x^{2}+y^{2}+z^{2}=1$. 求证:
	
	$$
	x y z+\sqrt{x^{2} y^{2}+y^{2} z^{2}+z^{2} x^{2}} \geqslant \frac{4}{3} \sqrt{x y z(x+y+z)}
	$$
\end{example}
\begin{proof}
	将原不等式两边除以 $x y z$,则欲证的不等式为
	
	$$
	1+\sqrt{\frac{1}{z^{2}}+\frac{1}{x^{2}}+\frac{1}{y^{2}}} \geqslant \frac{4}{3} \sqrt{\frac{1}{y z}+\frac{1}{z x}+\frac{1}{x y}}
	$$
	
	将 $x 、 y 、 z$ 分别换成 $\frac{1}{a} 、 \frac{1}{b} 、 \frac{1}{c}$, 则 $\frac{1}{a^{2}}+\frac{1}{b^{2}}+\frac{1}{c^{2}}=1$.
	
	原不等式即为 $1+\sqrt{a^{2}+b^{2}+c^{2}} \geqslant \frac{4}{3} \sqrt{a b+b c+c a}$
	
	$$
	\begin{aligned}
	& \Leftrightarrow 1+\sqrt{\left(a^{2}+b^{2}+c^{2}\right)\left(a^{-2}+b^{-2}+c^{-2}\right)} \\
	& \quad \geqslant \frac{4}{3} \sqrt{(a b+b c+c a)\left(a^{-2}+b^{-2}+c^{-2}\right)}
	\end{aligned}
	$$
	
	这时候, 由于上式是齐次式, 不妨 $a+b+c=1$, 则
	
	$$
	\begin{aligned}
	& a^{2}+b^{2}+c^{2}=a^{2}+b^{2}+c^{2}-\frac{4}{3}(a+b+c)+\frac{4}{9} \cdot 3 \\
	&=\left(\frac{2}{3}-a\right)^{2}+\left(\frac{2}{3}-b\right)^{2}+\left(\frac{2}{3}-c\right)^{2} . \\
	& \sqrt{\left(a^{2}+b^{2}+c^{2}\right)\left(a^{-2}+b^{-2}+c^{-2}\right)} \\
	&= \sqrt{\left[\left(\frac{2}{3}-a\right)^{2}+\left(\frac{2}{3}-b\right)^{2}+\left(\frac{2}{3}-c\right)^{2}\right]\left(a^{-2}+b^{-2}+c^{-2}\right)} \\
	& \geqslant \sqrt{\left[\left(\frac{2}{3}-a\right) \cdot \frac{1}{a}+\left(\frac{2}{3}-b\right) \cdot \frac{1}{b}+\left(\frac{2}{3}-c\right) \cdot \frac{1}{c}\right]^{2}} \\
	&=\left|\left(\frac{2}{3}-a\right) \cdot \frac{1}{a}+\left(\frac{2}{3}-b\right) \cdot \frac{1}{b}+\left(\frac{2}{3}-c\right) \cdot \frac{1}{c}\right| \\
	&=\left|\frac{2}{3}\left(\frac{1}{a}+\frac{1}{b}+\frac{1}{c}\right)-3\right| \\
	& \geqslant \frac{2}{3}\left(\frac{1}{a}+\frac{1}{b}+\frac{1}{c}\right)-3 \\
	&= \frac{2}{3}\left(\frac{1}{a}+\frac{1}{b}+\frac{1}{c}\right)(a+b+c)-3 \\
	&= \frac{2}{3}\left(\frac{b+c}{a}+\frac{c+a}{b}+\frac{a+b}{c}\right)-1 .
	\end{aligned}
	$$
	
	因此, 只需证明
	
	$$
	\begin{aligned}
	& \frac{b+c}{a}+\frac{c+a}{b}+\frac{a+b}{c} \geqslant 2 \sqrt{(a b+b c+c a)\left(a^{-2}+b^{-2}+c^{-2}\right)} \\
	& \Leftrightarrow\left(\frac{b+c}{a}\right)^{2}+\left(\frac{c+a}{b}\right)^{2}+\left(\frac{a+b}{c}\right)^{2} \\
	& \quad+2\left(\frac{c^{2}+a b+b c+c a}{a b}+\frac{a^{2}+a b+a c+b c}{b c}+\frac{b^{2}+a b+a c+b c}{c a}\right)
	\end{aligned}
	$$
	
	$$
	\begin{aligned}
	& \geqslant 4\left(\frac{b+c}{a}+\frac{c+a}{b}+\frac{a+b}{c}+\frac{b c}{a^{2}}+\frac{c a}{b^{2}}+\frac{a b}{c^{2}}\right) \\
	& \Leftrightarrow \frac{c^{2}+a b+b c+c a}{a b}+\frac{a^{2}+a b+b c+c a}{b c}+\frac{b^{2}+b c+a b+c a}{c a} \\
	& \geqslant\left(\frac{b+c}{a}+\frac{c+a}{b}\right)+\left(\frac{c+a}{b}+\frac{a+b}{c}\right)+\left(\frac{a+b}{c}+\frac{b+c}{a}\right) \\
	& \Leftrightarrow \frac{c^{2}+a b}{a b}+\frac{a^{2}+b c}{b c}+\frac{b^{2}+c a}{c a} \geqslant \frac{a^{2}+b^{2}}{a b}+\frac{b^{2}+c^{2}}{b c}+\frac{a^{2}+c^{2}}{c a} \\
	& \Leftrightarrow c^{3}+a^{3}+b^{3}+3 a b c \geqslant a^{2} c+a^{2} b+b^{2} c+b^{2} a+c^{2} a+c^{2} b \\
	& \Leftrightarrow a(a-b)(a-c)+b(b-a)(b-c)+c(c-a)(c-b) \geqslant 0
	\end{aligned}
	$$
	
	也可以设 $a \geqslant b \geqslant c$, 则
	
	$$
	\begin{aligned}
	& a(a-b)(a-c)+b(b-a)(b-c) \\
	= & (a-b)[a(a-c)-b(b-c)] \\
	= & (a-b)^{2}(a+b-c) \geqslant 0
	\end{aligned}
	$$
	
	而 $c(c-a)(c-b) \geqslant 0$, 故原不等式成立.
\end{proof}
\begin{note}
	用类似的方法可以证明下面的命题:
	
	设 $n(\geqslant 3)$ 为正整数, $a 、 b$ 为给定的实数,实数 $x_{0}, x_{1}, x_{2}, \cdots, x_{n}$ 满足
	
	$$
	\begin{aligned}
	& x_{0}+x_{1}+x_{2}+\cdots+x_{n}=a, \\
	& x_{0}^{2}+x_{1}^{2}+x_{2}^{2}+\cdots+x_{n}^{2}=b,
	\end{aligned}
	$$
	
	则当 $b<\frac{a^{2}}{n+1}$ 时, $x_{0}$ 不存在;
	
	$$
	\begin{aligned}
	& \text { 当 } b=\frac{a^{2}}{n+1} \text { 时, } x_{0}=\frac{a}{n+1} \text {; } \\
	& \text { 当 } b>\frac{a^{2}}{n+1} \text { 时, } x_{0} \text { 满足 } \\
	& \frac{a-\frac{1}{2} \sqrt{\delta}}{n+1} \leqslant x_{0} \leqslant \frac{a+\frac{1}{2} \sqrt{\delta}}{n+1}
	\end{aligned}
	$$
	
	其中 $\delta$ 为二次方程 $(n+1) x_{0}^{2}-2 a x_{0}+a^{2}-n b=0$ 的判别式.
\end{note}

\begin{example}
	设 $x_{1}, x_{2}, \cdots, x_{2019}$ 为实数, 且 $x_{12}=1$. 求 $\sum_{i, j=1}^{2019} \min \{i, j\} x_{i} x_{j}$ 的最小值.
\end{example}
\begin{solution}
	记 $S_{n}=\sum_{i, j=1}^{n} \min \left\{x_{i}, x_{j}\right\} x_{i} x_{j}$. 则
	
	$$
	\begin{aligned}
	& S_{1}=x_{1}^{2} \text {, } \\
	& \begin{aligned}
	S_{2} & =\sum_{i, j=1}^{2} \min \left\{x_{i}, x_{j}\right\} x_{i} x_{j} \\
	& =x_{1}^{2}+x_{1} x_{2}+x_{2} x_{1}+2 x_{2}^{2}=\left(x_{1}+x_{2}\right)^{2}+x_{2}^{2}, \\
	S_{3} & =\sum_{i, j=1}^{3} \min \left\{x_{i}, x_{j}\right\} x_{i} x_{j} \\
	& =\left(x_{1}^{2}+x_{1} x_{2}+x_{1} x_{3}\right)+\left(x_{2} x_{1}+2 x_{2}^{2}+2 x_{2} x_{3}\right)+\left(x_{3} x_{1}+2 x_{3} x_{2}+3 x_{3}^{2}\right) \\
	& =\left(x_{1}+x_{2}+x_{3}\right)^{2}+\left(x_{2}+x_{3}\right)^{2}+x_{3}^{2}, \\
	& \quad \text { 假设 } S_{n}=\sum_{i=1}^{n}\left(x_{i}+x_{i+1}+\cdots+x_{n}\right)^{2} \text {. 则 } \\
	& \quad S_{n+1}=\sum_{i, j=1}^{n+1} \min \{i, j\} x_{i} x_{j} \\
	& =\sum_{i, j=1}^{n} \min \{i, j\} x_{i} x_{j}+\sum_{j=1}^{n+1} j x_{n+1} x_{j}+\sum_{i=1}^{n} i x_{i} x_{n+1}
	\end{aligned}
	\end{aligned}
	$$
	
	$$
	\begin{aligned}
	= & \sum_{i=1}^{n}\left(x_{i}+x_{i+1}+\cdots+x_{n}\right)^{2}+2 x_{n+1}\left(x_{1}+2 x_{2}+\cdots+n x_{n}\right)+(n \\
	& +1) x_{n+1}^{2} \\
	= & \sum_{i=1}^{n}\left(x_{i}+x_{i+1}+\cdots+x_{n}\right)^{2}+2 \cdot \sum_{i=1}^{n}\left(x_{i}+x_{i+1}+\cdots+x_{n}\right) \cdot x_{n+1}+ \\
	& n x_{n+1}^{2}+x_{n+1}^{2} \\
	= & \sum_{i=1}^{n}\left[\left(x_{i}+x_{i+1}+\cdots+x_{n}\right)^{2}+2 \cdot \sum_{i=1}^{n}\left(x_{i}+x_{i+1}+\cdots+x_{n}\right) \cdot x_{n+1}\right. \\
	& \left.+x_{n+1}^{2}\right]+x_{n+1}^{2} \\
	= & \sum_{i=1}^{n}\left(x_{i}+x_{i+1}+\cdots+x_{n+1}\right)^{2}+x_{n+1}^{2} \\
	= & \sum_{i=1}^{n}\left(x_{i}+x_{i+1}+\cdots+x_{n+1}\right)^{2} .
	\end{aligned}
	$$
	
	由归纳原理知 $S_{n}=\sum_{i=1}^{n}\left(x_{i}+x_{i+1}+\cdots+x_{n}\right)^{2}$.
	
	设 $T_{n, i}=x_{i}+x_{i+1}+\cdots+x_{n}$, 则 $S_{n}=\sum_{i=1}^{n} T_{n, i}^{2}$,
	
	$$
	\begin{aligned}
	& T_{n, 12}=1+x_{13}+x_{14}+\cdots+x_{n}, \\
	& T_{11,13}=x_{13}+x_{14}+\cdots+x_{n}, \\
	S_{n} \geqslant & T_{n, 12}^{2}+T_{n, 13}^{2}=\left(1+T_{n, 13}\right)^{2}+T_{n, 13}^{2} \\
	\geqslant & \frac{1}{2}\left[\left(1+T_{n, 13}\right)-T_{n, 13}\right]^{2}=\frac{1}{2} .
	\end{aligned}
	$$
	
	例如取 $x_{11}=-\frac{1}{2}, x_{13}=-\frac{1}{2}, x_{i}=0(i \neq 11,12,13)$ 时, $S_{n}=\frac{1}{2}$.故 $S_{2019}$ 的最小值为 $\frac{1}{2}$.
	
	\section*{4. 2 柯西不等式在解方程组和求极值中的应用}
	应用柯西不等式中等号成立的条件, 通过不等式夹逼, 求出方程组中各个未知数的值, 从而进一步求出有关代数式的值.
	
	极值问题往往是关于对称式的问题. 先根据条件, 在各个未知元相等时的值得出极值, 然后证明相应的不等式.\\
\begin{note}
	用类似的方法可以证明下面的命题:
	
	设 $n(\geqslant 3)$ 为正整数, $a 、 b$ 为给定的实数,实数 $x_{0}, x_{1}, x_{2}, \cdots, x_{n}$ 满足
	
	$$
	\begin{aligned}
	& x_{0}+x_{1}+x_{2}+\cdots+x_{n}=a, \\
	& x_{0}^{2}+x_{1}^{2}+x_{2}^{2}+\cdots+x_{n}^{2}=b,
	\end{aligned}
	$$
	
	则当 $b<\frac{a^{2}}{n+1}$ 时, $x_{0}$ 不存在;
	
	$$
	\begin{aligned}
	& \text { 当 } b=\frac{a^{2}}{n+1} \text { 时, } x_{0}=\frac{a}{n+1} \text {; } \\
	& \text { 当 } b>\frac{a^{2}}{n+1} \text { 时, } x_{0} \text { 满足 } \\
	& \frac{a-\frac{1}{2} \sqrt{\delta}}{n+1} \leqslant x_{0} \leqslant \frac{a+\frac{1}{2} \sqrt{\delta}}{n+1}
	\end{aligned}
	$$
	
	其中 $\delta$ 为二次方程 $(n+1) x_{0}^{2}-2 a x_{0}+a^{2}-n b=0$ 的判别式.
\end{note}

\begin{example}
	求方程组
	
	\[
	\left\{\begin{array}{l}
	a^{2}=\frac{\sqrt{b c} \sqrt[3]{b c d}}{(b+c)(b+c+d)}  \tag{1}\\
	b^{2}=\frac{\sqrt{c d} \sqrt[3]{c d a}}{(c+d)(c+d+a)} \\
	c^{2}=\frac{\sqrt{d a} \sqrt[3]{d a b}}{(d+a)(d+a+b)} \\
	d^{2}=\frac{\sqrt{a b} \sqrt[3]{a b c}}{(a+b)(a+b+c)}
	\end{array}\right.
	\]
	
	的实数解.
\end{example}
\begin{solution}
	首先, 注意到没有一个变量等于零. 不失一般性, 假设 $b=0$, 由 (1)得 $a=0$, 由(4)得 $d=0$, 由(3)得 $c=0$, 这就意味着所有值为零, 但这是不可能的, 因为分母会为零.
	
	其次, 注意到 $b c 、 c d 、 d a 、 a b$ 的平方根一定都存在, 这就表明 $a 、 b 、 c 、 d$一定都是负数或都是正数. 如果都是负数, 这些方程的右边是负的, 与它们是实数的平方相矛盾, 由此可得 4 个值一定都是正的.
	
	根据算术一几何平均值不等式, 有
	
	$$
	\sqrt{b c} \leqslant \frac{b+c}{2}, \text { 即 } \frac{\sqrt{b c}}{b+c} \leqslant \frac{1}{2}
	$$
	
	和 $\sqrt[3]{b c d} \leqslant \frac{b+c+d}{3}$, 即
	
	$$
	\frac{\sqrt[3]{b c d}}{b+c+d} \leqslant \frac{1}{3}
	$$
	
	因此 $\quad a^{2}=\frac{\sqrt{b c} \sqrt[3]{b c d}}{(b+c)(b+c+d)} \leqslant \frac{1}{2} \times \frac{1}{3}=\frac{1}{6}$.
	
	从而 $a \leqslant \frac{1}{\sqrt{6}}$.
	
	类似地, 有 $b \leqslant \frac{1}{\sqrt{6}}, c \leqslant \frac{1}{\sqrt{6}}, d \leqslant \frac{1}{\sqrt{6}}$.
	
	由此得 $(b+c)(b+c+d) \leqslant \frac{2}{\sqrt{6}} \times \frac{3}{\sqrt{6}}=1$.
	
	同样地, 有
	
	$$
	\begin{aligned}
	& (c+d)(c+d+a) \leqslant 1 \\
	& (d+a)(d+a+b) \leqslant 1 \\
	& (a+b)(a+b+c) \leqslant 1
	\end{aligned}
	$$
	
	由(1) $\times(2) \times(3) \times(4)$, 可得
	
	$$
	1=(b+c)(b+c+d)(c+d)(c+d+a)(d+a)(d+a+b)(a+b)(a+b+c)
	$$
	
	因为 4 个小于或等于 1 的表达式的积等于 1 , 那么, 这 4 个表达式一定都等于 1 .
	
	从而唯一的可能是每个变量取它的最大的可能值.
	
	因此 $a=b=c=d=\frac{\sqrt{6}}{6}$ 为给定方程组的唯一解.
\end{solution}
\begin{note}
	用类似的方法可以证明下面的命题:
	
	设 $n(\geqslant 3)$ 为正整数, $a 、 b$ 为给定的实数,实数 $x_{0}, x_{1}, x_{2}, \cdots, x_{n}$ 满足
	
	$$
	\begin{aligned}
	& x_{0}+x_{1}+x_{2}+\cdots+x_{n}=a, \\
	& x_{0}^{2}+x_{1}^{2}+x_{2}^{2}+\cdots+x_{n}^{2}=b,
	\end{aligned}
	$$
	
	则当 $b<\frac{a^{2}}{n+1}$ 时, $x_{0}$ 不存在;
	
	$$
	\begin{aligned}
	& \text { 当 } b=\frac{a^{2}}{n+1} \text { 时, } x_{0}=\frac{a}{n+1} \text {; } \\
	& \text { 当 } b>\frac{a^{2}}{n+1} \text { 时, } x_{0} \text { 满足 } \\
	& \frac{a-\frac{1}{2} \sqrt{\delta}}{n+1} \leqslant x_{0} \leqslant \frac{a+\frac{1}{2} \sqrt{\delta}}{n+1}
	\end{aligned}
	$$
	
	其中 $\delta$ 为二次方程 $(n+1) x_{0}^{2}-2 a x_{0}+a^{2}-n b=0$ 的判别式.
\end{note}

\begin{example}
	已知实数 $x, y, z>3$, 求方程
	
	$$
	\frac{(x+2)^{2}}{y+z-2}+\frac{(y+4)^{2}}{z+x-4}+\frac{(z+6)^{2}}{x+y-6}=36
	$$
	
	的所有实数解 $(x, y, z)$.
\end{example}
\begin{solution}
	$(x, y, z)$.
	
	解 由 $x, y, z>3$, 知
	
	$$
	y+z-2>0, z+x-4>0, x+y-6>0
	$$
	
	由柯西一施瓦兹不等式得
	
	$$
	\begin{aligned}
	& {\left[\frac{(x+2)^{2}}{y+z-2}+\frac{(y+4)^{2}}{x+z-4}+\frac{(z+6)^{2}}{x+y-6}\right] } \\
	& {[(y+z-2)+(x+z-4)+(x+y-6)] } \\
	\geqslant & (x+y+z+12)^{2} \\
	\Leftrightarrow & \frac{(x+2)^{2}}{y+z-2}+\frac{(y+4)^{2}}{x+z-4}+\frac{(z+6)^{2}}{x+y-6} \\
	\geqslant & \frac{1}{2} \cdot \frac{(x+y+z+12)^{2}}{x+y+z-6}
	\end{aligned}
	$$
	
	结合题设等式得
	
	
	\begin{equation*}
	\frac{(x+y+z+12)^{2}}{x+y+z-6} \leqslant 72 \tag{1}
	\end{equation*}
	
	
	当 $\frac{x+2}{y+z-2}=\frac{y+4}{x+z-4}=\frac{z+6}{x+y-6}=\lambda$, 即
	
	\[
	\left\{\begin{array}{l}
	\lambda(y+z)-x=2(\lambda+1)  \tag{2}\\
	\lambda(x+z)-y=4(\lambda+1) \\
	\lambda(x+y)-z=6(\lambda+1)
	\end{array}\right.
	\]
	
	时, 式(1)等号成立.
	
	设 $w=x+y+z+12$. 则
	
	$$
	\frac{(x+y+z+12)^{2}}{x+y+z-6}=\frac{w^{2}}{w-18}
	$$
	
	又
	
	$$
	\begin{aligned}
	& \frac{w^{2}}{w-18} \geqslant 4 \times 18=72 \\
	\Leftrightarrow & w^{2}-4 \times 18 w+4 \times 18^{2} \geqslant 0 \\
	\Leftrightarrow & (w-36)^{2} \geqslant 0
	\end{aligned}
	$$
	
	则
	
	
	\begin{equation*}
	\frac{(x+y+z+12)^{2}}{x+y+z-6} \geqslant 72 \tag{3}
	\end{equation*}
	
	
	当
	
	
	\begin{align*}
	& w=x+y+z+12=36 \\
	\Leftrightarrow & x+y+z=24 \tag{4}
	\end{align*}
	
	
	时, 式(3)等号成立.
	
	由式(1), (3)得
	
	$$
	\frac{(x+y+z+12)^{2}}{x+y+z-6}=72
	$$
	
	由方程组(2)与式(4)得
	
	$$
	\left\{\begin{array}{l}
	(2 \lambda-1)(x+y+z)=12(\lambda+1), \\
	x+y+z=24
	\end{array} \Rightarrow \lambda=1\right.
	$$
	
	将 $\lambda=1$ 代人方程组(2)得
	
	$$
	\left\{\begin{array}{l}
	y+z-x=4 \\
	x+z-y=8, \Rightarrow(x, y, z)=(10,8,6) \\
	x+y-z=12
	\end{array}\right.
	$$
	
	所以,所求唯一实数解为 $(x, y, z)=(10,8,6)$.
\end{solution}
\begin{note}
	用类似的方法可以证明下面的命题:
	
	设 $n(\geqslant 3)$ 为正整数, $a 、 b$ 为给定的实数,实数 $x_{0}, x_{1}, x_{2}, \cdots, x_{n}$ 满足
	
	$$
	\begin{aligned}
	& x_{0}+x_{1}+x_{2}+\cdots+x_{n}=a, \\
	& x_{0}^{2}+x_{1}^{2}+x_{2}^{2}+\cdots+x_{n}^{2}=b,
	\end{aligned}
	$$
	
	则当 $b<\frac{a^{2}}{n+1}$ 时, $x_{0}$ 不存在;
	
	$$
	\begin{aligned}
	& \text { 当 } b=\frac{a^{2}}{n+1} \text { 时, } x_{0}=\frac{a}{n+1} \text {; } \\
	& \text { 当 } b>\frac{a^{2}}{n+1} \text { 时, } x_{0} \text { 满足 } \\
	& \frac{a-\frac{1}{2} \sqrt{\delta}}{n+1} \leqslant x_{0} \leqslant \frac{a+\frac{1}{2} \sqrt{\delta}}{n+1}
	\end{aligned}
	$$
	
	其中 $\delta$ 为二次方程 $(n+1) x_{0}^{2}-2 a x_{0}+a^{2}-n b=0$ 的判别式.
\end{note}

\begin{example}
	设 $n$ 是一个正整数, $a_{1}, a_{2}, \cdots, a_{n}, b_{1}, b_{2}, \cdots, b_{n}$ 是 $2 n$ 个正实数, 满足 $a_{1}+a_{2}+\cdots+a_{n}=1, b_{1}+b_{2}+\cdots+b_{n}=1$, 求 $\frac{a_{1}^{2}}{a_{1}+b_{1}}+\frac{a_{2}^{2}}{a_{2}+b_{2}}+\cdots$\\
	$+\frac{a_{n}^{2}}{a_{n}+b_{n}}$ 的最小值.
\end{example}
\begin{solution}
	由柯西不等式知
	
	$$
	\begin{aligned}
	& \quad\left(a_{1}+a_{2}+\cdots+a_{n}+b_{1}+b_{2}+\cdots+b_{n}\right)\left(\frac{a_{1}^{2}}{a_{1}+b_{1}}+\frac{a_{2}^{2}}{a_{2}+b_{2}}+\cdots+\frac{a_{n}^{2}}{a_{n}+b_{n}}\right) \\
	& \geqslant\left(a_{1}+a_{2}+\cdots+a_{n}\right)^{2}=1, \\
	& \text { 且 } \quad a_{1}+a_{2}+\cdots+a_{n}+b_{1}+b_{2}+\cdots+b_{n}=2, \\
	& \text { 所以 } \quad \frac{a_{1}^{2}}{a_{1}+b_{1}}+\frac{a_{2}^{2}}{a_{2}+b_{2}}+\cdots+\frac{a_{n}^{2}}{a_{n}+b_{n}} \geqslant \frac{1}{2},
	\end{aligned}
	$$
	
	且当 $a_{1}=a_{2}=\cdots=a_{n}=b_{1}=b_{2}=\cdots=b_{n}=\frac{1}{n}$ 时取到.
	
	所以 $\frac{a_{1}^{2}}{a_{1}+b_{1}}+\frac{a_{2}^{2}}{a_{2}+b_{2}}+\cdots+\frac{a_{n}^{2}}{a_{n}+b_{n}}$ 的最小值为 $\frac{1}{2}$.
\end{solution}
\begin{note}
	用类似的方法可以证明下面的命题:
	
	设 $n(\geqslant 3)$ 为正整数, $a 、 b$ 为给定的实数,实数 $x_{0}, x_{1}, x_{2}, \cdots, x_{n}$ 满足
	
	$$
	\begin{aligned}
	& x_{0}+x_{1}+x_{2}+\cdots+x_{n}=a, \\
	& x_{0}^{2}+x_{1}^{2}+x_{2}^{2}+\cdots+x_{n}^{2}=b,
	\end{aligned}
	$$
	
	则当 $b<\frac{a^{2}}{n+1}$ 时, $x_{0}$ 不存在;
	
	$$
	\begin{aligned}
	& \text { 当 } b=\frac{a^{2}}{n+1} \text { 时, } x_{0}=\frac{a}{n+1} \text {; } \\
	& \text { 当 } b>\frac{a^{2}}{n+1} \text { 时, } x_{0} \text { 满足 } \\
	& \frac{a-\frac{1}{2} \sqrt{\delta}}{n+1} \leqslant x_{0} \leqslant \frac{a+\frac{1}{2} \sqrt{\delta}}{n+1}
	\end{aligned}
	$$
	
	其中 $\delta$ 为二次方程 $(n+1) x_{0}^{2}-2 a x_{0}+a^{2}-n b=0$ 的判别式.
\end{note}

\begin{example}
	已知 $x 、 y 、 z$ 为实数, 且满足
	
	$$
	x+y+z=x y+y z+z x
	$$
	
	求 $\frac{x}{x^{2}+1}+\frac{y}{y^{2}+1}+\frac{z}{z^{2}+1}$ 的最小值.
\end{example}
\begin{solution}
	令 $x=1, y=z=-1$. 则
	
	$$
	\frac{x}{x^{2}+1}+\frac{y}{y^{2}+1}+\frac{z}{z^{2}+1}=-\frac{1}{2}
	$$
	
	猜想最小值为 $-\frac{1}{2}$.
	
	只须证:
	
	
	\begin{align*}
	& \frac{x}{x^{2}+1}+\frac{y}{y^{2}+1}+\frac{z}{z^{2}+1} \geqslant-\frac{1}{2} \\
	\Leftrightarrow & \frac{(x+1)^{2}}{x^{2}+1}+\frac{(y+1)^{2}}{y^{2}+1} \geqslant \frac{(z-1)^{2}}{z^{2}+1} \tag{1}
	\end{align*}
\end{solution}
\begin{note}
	用类似的方法可以证明下面的命题:
	
	设 $n(\geqslant 3)$ 为正整数, $a 、 b$ 为给定的实数,实数 $x_{0}, x_{1}, x_{2}, \cdots, x_{n}$ 满足
	
	$$
	\begin{aligned}
	& x_{0}+x_{1}+x_{2}+\cdots+x_{n}=a, \\
	& x_{0}^{2}+x_{1}^{2}+x_{2}^{2}+\cdots+x_{n}^{2}=b,
	\end{aligned}
	$$
	
	则当 $b<\frac{a^{2}}{n+1}$ 时, $x_{0}$ 不存在;
	
	$$
	\begin{aligned}
	& \text { 当 } b=\frac{a^{2}}{n+1} \text { 时, } x_{0}=\frac{a}{n+1} \text {; } \\
	& \text { 当 } b>\frac{a^{2}}{n+1} \text { 时, } x_{0} \text { 满足 } \\
	& \frac{a-\frac{1}{2} \sqrt{\delta}}{n+1} \leqslant x_{0} \leqslant \frac{a+\frac{1}{2} \sqrt{\delta}}{n+1}
	\end{aligned}
	$$
	
	其中 $\delta$ 为二次方程 $(n+1) x_{0}^{2}-2 a x_{0}+a^{2}-n b=0$ 的判别式.
\end{note}

\begin{example}
	设 $a 、 b 、 c 、 x 、 y 、 z$ 为实数,且
	
	$$
	a^{2}+b^{2}+c^{2}=25, x^{2}+y^{2}+z^{2}=36, a x+b y+c z=30
	$$
	
	求 $\frac{a+b+c}{x+y+z}$ 的值.
\end{example}
\begin{solution}
	由柯西不等式, 得
	
	$25 \times 36=\left(a^{2}+b^{2}+c^{2}\right)\left(x^{2}+y^{2}+z^{2}\right) \geqslant(a x+b y+c z)^{2}=30^{2}$.上述不等式等号成立, 得
	
	$$
	\frac{a}{x}=\frac{b}{y}=\frac{c}{z}=k
	$$
	
	于是 $k^{2}\left(x^{2}+y^{2}+z^{2}\right)=25$, 所以 $k= \pm \frac{5}{6}$ (负的舍去). 从而
	
	$$
	\frac{a+b+c}{x+y+z}=k=\frac{5}{6}
	$$
\end{solution}
\begin{note}
	用类似的方法可以证明下面的命题:
	
	设 $n(\geqslant 3)$ 为正整数, $a 、 b$ 为给定的实数,实数 $x_{0}, x_{1}, x_{2}, \cdots, x_{n}$ 满足
	
	$$
	\begin{aligned}
	& x_{0}+x_{1}+x_{2}+\cdots+x_{n}=a, \\
	& x_{0}^{2}+x_{1}^{2}+x_{2}^{2}+\cdots+x_{n}^{2}=b,
	\end{aligned}
	$$
	
	则当 $b<\frac{a^{2}}{n+1}$ 时, $x_{0}$ 不存在;
	
	$$
	\begin{aligned}
	& \text { 当 } b=\frac{a^{2}}{n+1} \text { 时, } x_{0}=\frac{a}{n+1} \text {; } \\
	& \text { 当 } b>\frac{a^{2}}{n+1} \text { 时, } x_{0} \text { 满足 } \\
	& \frac{a-\frac{1}{2} \sqrt{\delta}}{n+1} \leqslant x_{0} \leqslant \frac{a+\frac{1}{2} \sqrt{\delta}}{n+1}
	\end{aligned}
	$$
	
	其中 $\delta$ 为二次方程 $(n+1) x_{0}^{2}-2 a x_{0}+a^{2}-n b=0$ 的判别式.
\end{note}

\begin{example}
	设实数 $a 、 b 、 c 、 d 、 e$ 满足
	
	$$
	a+b+c+d+e=8, a^{2}+b^{2}+c^{2}+d^{2}+e^{2}=16
	$$
	
	求 $e$ 的最大值.
\end{example}
\begin{solution}
	将条件改写为
	
	$$
	8-e=a+b+c+d, 16-e^{2}=a^{2}+b^{2}+c^{2}+d^{2}
	$$
	
	由此得到一个包含 $e$ 的不等式. 由柯西不等式, 得
	
	$$
	a+b+c+d \leqslant(1+1+1+1)^{\frac{1}{2}}\left(a^{2}+b^{2}+c^{2}+d^{2}\right)^{\frac{1}{2}}
	$$
	
	将条件代人并两边平方, 得
	
	$$
	\begin{gathered}
	(8-e)^{2} \leqslant 4\left(16-e^{2}\right) \\
	64-16 e+e^{2} \leqslant 64-4 e^{2} \\
	5 e^{2}-16 e \leqslant 0, e(5-16 e) \leqslant 0
	\end{gathered}
	$$
	
	从此得到 $0 \leqslant e \leqslant \frac{16}{5}$, 当 $a=b=c=d=\frac{6}{5}$ 时达到最大值 $\frac{16}{5}$.
\end{solution}
\begin{note}
	用类似的方法可以证明下面的命题:
	
	设 $n(\geqslant 3)$ 为正整数, $a 、 b$ 为给定的实数,实数 $x_{0}, x_{1}, x_{2}, \cdots, x_{n}$ 满足
	
	$$
	\begin{aligned}
	& x_{0}+x_{1}+x_{2}+\cdots+x_{n}=a, \\
	& x_{0}^{2}+x_{1}^{2}+x_{2}^{2}+\cdots+x_{n}^{2}=b,
	\end{aligned}
	$$
	
	则当 $b<\frac{a^{2}}{n+1}$ 时, $x_{0}$ 不存在;
	
	$$
	\begin{aligned}
	& \text { 当 } b=\frac{a^{2}}{n+1} \text { 时, } x_{0}=\frac{a}{n+1} \text {; } \\
	& \text { 当 } b>\frac{a^{2}}{n+1} \text { 时, } x_{0} \text { 满足 } \\
	& \frac{a-\frac{1}{2} \sqrt{\delta}}{n+1} \leqslant x_{0} \leqslant \frac{a+\frac{1}{2} \sqrt{\delta}}{n+1}
	\end{aligned}
	$$
	
	其中 $\delta$ 为二次方程 $(n+1) x_{0}^{2}-2 a x_{0}+a^{2}-n b=0$ 的判别式.
\end{note}

\begin{example}
	设 $x \geqslant 0, y \geqslant 0, z \geqslant 0, a 、 b 、 c 、 l 、 m 、 n$ 是给定的正数, 并且 $a x+b y+c z=\delta$ 为常数,求
	
	$$
	w=\frac{l}{x}+\frac{m}{y}+\frac{n}{z}
	$$
	
	的最小值.
\end{example}
\begin{solution}
	由柯西不等式, 得
	
	$$
	\begin{aligned}
	w \cdot \delta & =\left[\left(\sqrt{\frac{l}{x}}\right)^{2}+\left(\sqrt{\frac{m}{y}}\right)^{2}+\left(\sqrt{\frac{n}{z}}\right)^{2}\right] \\
	& \cdot\left[(\sqrt{a x})^{2}+(\sqrt{b y})^{2}+(\sqrt{c z})^{2}\right] \\
	& \geqslant(\sqrt{a l}+\sqrt{b m}+\sqrt{c n})^{2} \\
	& w \geqslant \frac{(\sqrt{a l}+\sqrt{b m}+\sqrt{c n})^{2}}{\delta}
	\end{aligned}
	$$
	
	利用柯西等式成立的条件, 得 $x=k \sqrt{\frac{l}{a}}, y=k \sqrt{\frac{m}{b}}, z=k \sqrt{\frac{n}{c}}$, 其中 $k=\frac{\delta}{\sqrt{a l}+\sqrt{b m}+\sqrt{c n}}$, 它们使得 $a x+b y+c z=\delta$, 且
	
	$$
	w=\frac{(\sqrt{a l}+\sqrt{b m}+\sqrt{c n})^{2}}{\delta}
	$$
	
	所以
	
	$$
	w_{\min }=\frac{(\sqrt{a l}+\sqrt{b m}+\sqrt{c n})^{2}}{\delta}
	$$
\end{solution}
\begin{note}
	这个题目的表达形式看起来很复杂,但通过变量代换后, 可以发现各项之间的关系,借助于柯西不等式, 估计出它的下界.
	
	\section*{4. 3 柯西不等式在证明分式不等式中的应用}
	在各种不等式中, 分式不等式的问题由于自身的特点, 证明它们需要有较灵活的技巧和方法. 对于分式型的不等式, 通常运用柯西不等式的一些变形.

\begin{example}
	对满足 $a+b=1$ 的正实数 $a 、 b$, 求
	
	$$
	\left(a+\frac{1}{a}\right)^{2}+\left(b+\frac{1}{b}\right)^{2}
	$$
	
	的最小值.
\end{example}
\begin{solution}
	当 $a=b=\frac{1}{2}$ 时, 我们有
	
	$$
	\left(a+\frac{1}{a}\right)^{2}+\left(b+\frac{1}{b}\right)^{2}=\frac{25}{2}
	$$
	
	下面证明
	
	$$
	\left(a+\frac{1}{a}\right)^{2}+\left(b+\frac{1}{b}\right)^{2} \geqslant \frac{25}{2}
	$$
	
	从而最小值为 $\frac{25}{2}$.
	
	$$
	\begin{aligned}
	& \text { 令 } x=a+\frac{1}{a}, y=b+\frac{1}{b} \text {, 由 } \\
	& \frac{x^{2}+y^{2}}{2} \geqslant\left(\frac{x+y}{2}\right)^{2}
	\end{aligned}
	$$
	
	于是
	
	$$
	\begin{aligned}
	\frac{1}{2}\left[\left(a+\frac{1}{a}\right)^{2}+\left(b+\frac{1}{b}\right)^{2}\right] & \geqslant\left\{\frac{1}{2}\left[\left(a+\frac{1}{a}\right)+\left(b+\frac{1}{b}\right)\right]\right\}^{2} \\
	& =\left[\frac{1}{2}\left(1+\frac{1}{a}+\frac{1}{b}\right)\right]^{2}
	\end{aligned}
	$$
	
	由柯西不等式, 得 $\left(\frac{1}{a}+\frac{1}{b}\right)(a+b) \geqslant(1+1)^{2}=4$. 则
	
	$$
	\left[\frac{1}{2}\left(1+\frac{1}{a}+\frac{1}{b}\right)\right]^{2} \geqslant\left[\frac{1}{2}\left(1+\frac{4}{a+b}\right)\right]^{2}=\left(\frac{1+4}{2}\right)^{2}=\frac{25}{4}
	$$
	
	从而命题成立.
\end{solution}

	这个题目的表达形式看起来很复杂,但通过变量代换后, 可以发现各项之间的关系,借助于柯西不等式, 估计出它的下界.


	\section*{4. 4 柯西不等式在组合计数估计中的应用}
	在研究组合, 特别是组合计数问题时, 常常需要由给定的条件, 对一些不	等式进行估计. 如果能灵活地应用, 柯西不等式在解决这些问题中能发挥很好的作用.

\begin{note}
	这里, 在两次利用柯西不等式时, 引进了参数 $n 、 m 、 a 、 b$.
\end{note}

\begin{example}
	将 1650 个学生排成 22 行, 75 列的方阵,已知任意给定的两列处于同一行的两个人中, 性别相同的学生不超过 11 对, 证明: 男生的人数不超过 928 .
\end{example}
\begin{solution}
	设第 $i$ 行的男生数为 $x_{i}$, 则女生数为 $75-x_{i}$, 依题意, 得
	
	$$
	\sum_{i=1}^{22}\left(\mathrm{C}_{x_{i}}^{2}+\mathrm{C}_{75-x_{i}}^{2}\right) \leqslant 11 \times \mathrm{C}_{75}^{2}
	$$
	
	于是
	
	$$
	\sum_{i=1}^{22}\left(x_{i}^{2}-75 x_{i}\right) \leqslant-30525
	$$
	
	即 $\sum_{i=1}^{22}\left(2 x_{i}-75\right)^{2} \leqslant 1650$. 由柯西不等式, 得
	
	$$
	\left[\sum_{i=1}^{22}\left(2 x_{i}-750\right)\right]^{2} \leqslant 22 \sum_{i=1}^{22}\left(2 x_{i}-75\right)^{2} \leqslant 36300
	$$
	
	因此 $\sum_{i=1}^{22}\left(2 x_{i}-75\right)<191$, 从而
	
	$$
	\sum_{i=1}^{22} x_{i}<\frac{191+1650}{2}<921
	$$
	
	故男生的人数不超过 928.
\end{solution}
\begin{note}
	这里, 在两次利用柯西不等式时, 引进了参数 $n 、 m 、 a 、 b$.
\end{note}

\begin{example}
	在一群数学家中, 每一个人都有一些朋友 (关系是互相的). 证明:存在一个数学家他所有的朋友的平均值不小于这群人的朋友的平均数.
\end{example}
\begin{proof}
	记 $M$ 为这群数学家的集合, $n=|M|, F(m)$ 表示数学家 $m$ 的朋友的集合, $f(m)$ 表示数学家 $m$ 的朋友数 $(f(m)=|F(m)|)$. 即命题等价于证明: 必有一个 $m_{0}$ 使
	
	$$
	\frac{1}{f\left(m_{0}\right)} \sum_{m \in F\left(m_{0}\right)} f(m) \geqslant \frac{1}{n} \sum_{m \in M} f(m)
	$$
	
	我们用反证法来证明这个命题, 如果不存在这样的数学家 $m_{0}$. 则对任意的 $m_{0}$, 有
	
	$$
	n \cdot \sum_{m \in F\left(m_{0}\right)} f(m)<f\left(m_{0}\right) \sum_{m \in M} f(m)
	$$
	
	对一切 $m_{0}$ 求和, 得
	
	$$
	n \cdot \sum_{m_{0}} \sum_{m \in F\left(m_{0}\right)} f(m)=n \sum_{m} \sum_{m \in F\left(m_{0}\right)} f(m)=n \sum_{m \in M} f^{2}(m)<\left(\sum_{m \in M} f(m)\right)^{2}
	$$
	
	这与柯西不等式矛盾, 故命题成立.
\end{proof}
\begin{note}
	这里, 在两次利用柯西不等式时, 引进了参数 $n 、 m 、 a 、 b$.
\end{note}

\begin{example}
	设空间中有 $2 n(n \geqslant 2)$ 个点, 其中任何 4 点都不共面. 在它们之间任意连接 $N$ 条线段, 这些线段都至少构成一个三角形. 求 $N$ 的最小值.
\end{example}
\begin{solution}
	将 $2 n$ 个已知点均分为 $A 、 B$ 两组:
	
	$$
	A=\left\{A_{1}, A_{2}, \cdots, A_{n}\right\}, B=\left\{B_{1}, B_{2}, \cdots, B_{n}\right\}
	$$
	
	现将每对点 $A_{i}$ 和 $B_{i}$ 之间都连接一条线段 $A_{i} B_{i}$, 而同组的任意两点之间不连线, 则共有 $n^{2}$ 条线段. 这时, $2 n$ 个已知点中的任何 3 点中至少有两点属于同一组, 两者之间没有连线. 因而这 $n^{2}$ 条线段不能构成任何三角形. 这表明 $N$ 的最小值必大于 $n^{2}$. 由于 $2 n$ 个点之间连有 $n^{2}+1$ 条线段, 平均每点引出 $n$条线段还多, 故可以猜想有一条线段的两个端点引出的线段之和不小于 $2 n+$ 1. 下面证明 $N$ 的最小值为 $2 n+1$.
	
	设从 $A_{1}, A_{2}, \cdots, A_{2 n}$ 引出的线段条数分别为 $a_{1}, a_{2}, \cdots, a_{2 n}$, 且对于任一线段 $A_{i} A_{j}$ 都有 $a_{i}+a_{j} \leqslant 2 n$. 于是, 所有线段的两端点所引出的线段条数之和不超过 $2 n\left(n^{2}+1\right)$. 但在此计数中, $A_{i}$ 点恰被计算了 $a_{i}$ 次, 故有
	
	$$
	\sum_{i=1}^{2 n} a_{i}^{2} \leqslant 2 n\left(n^{2}+1\right)
	$$
	
	另一方面, 显然有
	
	$$
	\sum_{i=1}^{2 n} a_{i}=2\left(n^{2}+1\right)
	$$
	
	故由柯西不等式, 得
	
	$$
	\left(\sum_{i=1}^{2 n} a_{i}\right)^{2} \leqslant 2 n\left(\sum_{i=1}^{2 n} a_{i}^{2}\right)
	$$
	
	即
	
	$$
	\sum_{i=1}^{2 n} a_{i}^{2} \geqslant \frac{1}{2 n} \cdot 4\left(n^{2}+1\right)^{2}>2 n\left(n^{2}+1\right)
	$$
	
	于是矛盾, 从而证明了必有一条线段, 从它的两端点引出的线段数之和不小于 $2 n+1$. 不妨设 $A_{1} A_{2}$ 是一条这样的线段, 从而又有 $A_{k}(k \geqslant 3)$, 使线段 $A_{1} A_{k}, A_{2} A_{k}$ 都存在, 于是 $\triangle A_{1} A_{2} A_{k}$ 即为所求.
\end{solution}
\begin{note}
	这里, 在两次利用柯西不等式时, 引进了参数 $n 、 m 、 a 、 b$.
\end{note}

\begin{example}
	在 $m \times m$ 方格纸中, 至少要挑出多少个小方格, 才能使得这些小方格中存在四个小方格, 它们的中心组成一个矩形的 4 个顶点, 而矩形的边平行于原正方形的边.
\end{example}
\begin{solution}
	所求的最小值为 $\left[\frac{m}{2}(1+\sqrt{4 m-3})-1\right]+1$. 设最多能挑出 $k$ 个小\\
	方格, 使得这些小方格中不存在任何四个小方格, 它们的中点组成一个矩形的 4 个顶点 (矩形的边平行于原正方形的边). 并假设位于第 $i$ 行的有 $k_{i}(i=$ $1,2, \cdots, m)$ 个, 则
	
	$$
	\sum_{i=1}^{n} k_{i}=k
	$$
	
	设第 $i$ 行的 $k_{i}$ 个小方格位于这行的第 $j_{1}, j_{2}, \cdots, j_{k_{i}}$ 列, $1 \leqslant j_{1}<$ $j_{2}<\cdots<j_{k_{i}} \leqslant m$. 如果第 $r$ 行的第 $j_{p}, j_{q}$ 列的两个方格已经挑出, 则任意的第 $s(s \neq r)$ 行的 $j_{p}, j_{q}$ 列的两个方格不能同时挑出, 否则将组成一个矩形的 4 个顶点. 所以对于每个 $i$, 考虑 $j_{1}, j_{2}, \cdots, j_{k_{i}}$ 中每两个的组合, 可得到 $\mathrm{C}_{k_{i}}^{2}$ 个组合. 对 $i=1,2, \cdots, m$, 可得 $\sum \mathrm{C}_{k_{i}}^{2}$ 个组合, 且其中任意两个不相同 (即无重复), 这些组合都是 $1,2, \cdots, m$ 中取两个的组合, 总数为 $\mathrm{C}_{m}^{2}$. 所以
	
	$$
	\sum_{i=1}^{m} \mathrm{C}_{k_{i}}^{2} \leqslant \mathrm{C}_{m}^{2}
	$$
	
	即
	
	$$
	\frac{1}{2} \sum_{i=1}^{m} k_{i}\left(k_{i}-1\right) \leqslant \frac{1}{2} m(m-1)
	$$
	
	由 $\sum_{i=1}^{m} k_{i}=k$, 得到 $\sum_{i=1}^{m} k_{i}^{2} \leqslant m(m-1)+k$. 由柯西不等式, 得
	
	$$
	\sum_{i=1}^{m} k_{i}^{2} \geqslant \frac{\left(\sum_{i=1}^{m} k_{i}\right)^{2}}{m}=\frac{k^{2}}{m}
	$$
	
	所以 $\frac{k^{2}}{m} \leqslant m(m-1)+k$, 故 $k \leqslant \frac{m}{2}(1+\sqrt{4 m-3})$.
	
	因此, 至少要挑出 $\left[\frac{m}{2}(1+\sqrt{4 m-3})-1\right]+1$ 个小方格.
\end{solution}
\begin{note}
	这里, 在两次利用柯西不等式时, 引进了参数 $n 、 m 、 a 、 b$.
\end{note}

\begin{example}
	设 $A_{1}, A_{2}, \cdots, A_{30}$ 是集 $\{1,2, \cdots, 2003\}$ 的子集,且 $\left|A_{i}\right| \geqslant 660$ $(i=1,2, \cdots, 30)$. 证明: 存在 $i, j \in\{1,2, \cdots, 30\}, i \neq j$, 使得
	
	$$
	\left|A_{i} \cap A_{j}\right| \geqslant 203
	$$
\end{example}
\begin{proof}
	不妨设每个 $A_{i}$ 的元素都为 660 个 (否则去除一些元素), 我们作一个集合、元素的关系表:表中每一行(除最上面的一行)表示 30 个集合,表的 $n$列(最左面一列除外)表示 2003 个元素 $1,2, \cdots, 2003$. 如果 $i \in A_{j}(i=1$, $2, \cdots, 2003,1 \leqslant j \leqslant 30)$, 则在 $i$ 所在的列与 $A_{j}$ 所在的交叉处填上 1 , 如果 $i \notin A_{j}$, 则写上 0 . 表中每一行有 660 个 1 , 因此共有 $30 \times 660$ 个 1 . 第 $j$ 列有\\
	$m_{j}$ 个 $1(j=1,2, \cdots, 2003)$, 则
	
	$$
	\sum_{j=1}^{2003} m_{j}=30 \times 660
	$$
	
	由于每个元素 $j$ 属于 $\mathrm{C}_{m_{j}}^{2}$ 个交集 $A_{s} \cap A_{t}$, 因此
	
	$$
	\sum_{j=1}^{2003} \mathrm{C}_{m_{j}}^{2}=\sum_{1 \leqslant s<t \leqslant 30}\left|A_{s} \cap A_{t}\right|
	$$
	
	由柯西不等式, 得
	
	$$
	\sum_{j=1}^{2003} \mathrm{C}_{m_{j}}^{2}=\frac{1}{2}\left(\sum_{j=1}^{2003} m_{j}^{2}-\sum_{j=1}^{2003} m_{j}\right) \geqslant \frac{1}{2}\left[\frac{1}{2003}\left(\sum_{j=1}^{2003} m_{i}\right)^{2}-\sum_{j=1}^{2003} m_{j}\right]
	$$
	
	所以,必有 $i \neq j$, 满足
	
	$$
	\begin{aligned}
	\left|A_{i} \cap A_{j}\right| & \geqslant \frac{1}{\mathrm{C}_{30}^{2}} \times \frac{1}{2}\left[\frac{1}{2003}\left(\sum_{j=1}^{2003} m_{j}\right)^{2}-\sum_{j=1}^{2003} m_{j}\right] \\
	& =\frac{660(30 \times 660-2003)}{29 \times 2003}>202
	\end{aligned}
	$$
	
	故 $\quad\left|A_{i} \cap A_{j}\right| \geqslant 203$.
\end{proof}
\begin{note}
	这里, 在两次利用柯西不等式时, 引进了参数 $n 、 m 、 a 、 b$.
\end{note}

\begin{example}
	给定平面上的 $n$ 个相异点. 证明: 其中距离为单位长的点对少于 $2 \sqrt{n^{3}}$ 对.
\end{example}
\begin{proof}
	对于平面上的点集 $\left\{P_{1}, P_{2}, \cdots, P_{n}\right\}$, 令 $a_{i}$ 为与 $P_{i}$ 相距为单位长的点 $P_{i}$ 的个数. 不妨设 $a_{i} \geqslant 1$, 则相距为单位长的点对的对数是
	
	$$
	A=\frac{a_{1}+a_{2}+\cdots+a_{n}}{2}
	$$
	
	设 $C_{i}$ 是以点 $P_{i}$ 为圆心, 以 1 为半径的圆.
	
	因为每对圆至多有 2 个交点, 故所有的 $C_{i}$ 至多有 $2 \mathrm{C}_{n}^{2}=n(n-1)$ 个交点.
	
	点 $P_{i}$ 作为 $C_{j}$ 的交点出现 $\mathrm{C}_{a_{j}}^{2}$ 次, 因此
	
	$$
	n(n-1) \geqslant \sum_{j=1}^{n} \mathrm{C}_{a_{j}}^{2}=\sum_{j=1}^{n} \frac{a_{j}\left(a_{j}-1\right)}{2} \geqslant \frac{1}{2} \sum_{j=1}^{n}\left(a_{j}-1\right)^{2}
	$$
	
	由柯西不等式, 得
	
	$$
	\left[\sum_{j=1}^{n}\left(a_{j}-1\right)\right]^{2} \leqslant n \cdot \sum_{j=1}^{n}\left(a_{j}-1\right)^{2} \leqslant n \cdot 2 n(n-1)<2 n^{3}
	$$
	
	于是
	
	$$
	\sum_{j=1}^{n}\left(a_{j}-1\right)<\sqrt{2} \cdot \sqrt{n^{3}}
	$$
	
	从而
	
	$$
	A=\frac{\sum_{j=1}^{n} a_{j}}{2}<\frac{n+\sqrt{2 n^{3}}}{2}<2 \sqrt{n^{3}}
	$$
	
	故命题成立.
\end{proof}
\begin{note}
	这里, 在两次利用柯西不等式时, 引进了参数 $n 、 m 、 a 、 b$.
\end{note}

\begin{example}
	在三维空间中给定一点 $O$ 以及由总长度为 1988 的若干条线段组成的有限集 $A$, 证明: 存在一个平面与集 $A$ 不相交且到点 $O$ 的距离不超过 574 .
\end{example}
\begin{proof}
	以点 $O$ 为原点建立直角坐标系, 并将所给的线段分别向 3 条坐标轴投影. 设 $A$ 中共有 $n$ 条线段且它们在 3 条轴上的投影长分别为
	
	$$
	\begin{aligned}
	& x_{i}, y_{i}, z_{i}, i=1,2, \cdots, n \\
	& \text { 记 } x=\sum x_{i}, y=\sum y_{i}, z=\sum z_{i} . \text { 于是, 由柯西不等式, 得 } \\
	& x^{2}+y^{2}+z^{2}=\left(\sum x_{i}\right)^{2}+\left(\sum y_{i}\right)^{2}+\left(\sum z_{i}\right)^{2} \\
	& =\sum_{i=1}^{n} \sum_{j=1}^{n}\left(x_{i} x_{j}+y_{i} y_{j}+z_{i} z_{j}\right) \\
	& \leqslant \sum_{i=1}^{n} \sum_{j=1}^{n} \sqrt{\left(x_{i}^{2}+y_{i}^{2}+z_{i}^{2}\right)\left(x_{j}^{2}+y_{j}^{2}+z_{j}^{2}\right)} \\
	& =\left(\sum_{i=1}^{n} \sqrt{x_{i}^{2}+y_{i}^{2}+z_{i}^{2}}\right)^{2}=1988^{2}
	\end{aligned}
	$$
	
	不妨设 $x=\min \{x, y, z\}$, 于是
	
	$$
	x \leqslant \frac{1988}{\sqrt{3}}<2 \times 574
	$$
	
	从而在 $x$ 轴上的区间 $[-574,574]$ 内必有一点不在 $n$ 条给定线段的投影上, 过这点作与 $x$ 轴垂直的平面便满足题中的要求.
\end{proof}
\begin{note}
	这里, 在两次利用柯西不等式时, 引进了参数 $n 、 m 、 a 、 b$.
\end{note}

\begin{example}
	设 $O x y z$ 是空间直角坐标系, $S$ 是空间中一个有限点集, $S_{x} 、 S_{y} 、 S_{z}$分别是 $S$ 中所有点在 $O y z$ 平面, $O z x$ 平面和 $O x y$ 平面上的正投影所成的集合. 求证:
	
	$$
	|S|^{2} \leqslant\left|S_{x}\right| \cdot\left|S_{y}\right| \cdot\left|S_{z}\right|
	$$
	
	说明:所谓一个点在一个平面上的正投影是指由点向平面所作垂线的\\
	垂足.
\end{example}
\begin{proof}
	设共有 $n$ 个平行于 $O x y$ 平面的平面上有 $S$ 中的点, 这些平面分别记为 $M_{1}, M_{2}, \cdots, M_{n}$. 对于平面 $M_{i}, 1 \leqslant i \leqslant n$, 设它与 $O z x 、 O z y$ 平面分别交于直线 $l_{y}$ 和 $l_{x}$, 并设 $M_{i}$ 上有 $m_{i}$ 个 $S$ 中的点. 显然, $m_{i} \leqslant\left|S_{z}\right|$.
	
	设 $M_{i}$ 上的点在 $l_{x} 、 l_{y}$ 上的正投影的集合分别为 $A_{i}$ 和 $B_{i}$, 记 $a_{i}=\left|A_{i}\right|$, $b_{i}=\left|B_{i}\right|$, 则有 $m_{i} \leqslant a_{i} b_{i}$. 又因为
	
	$$
	\sum_{i=1}^{n} a_{i}=\left|S_{y}\right|, \sum_{i=1}^{n} b_{i}=\left|S_{x}\right|, \sum_{i=1}^{n} m_{i}=|S|
	$$
	
	从而由柯西不等式, 得
	
	$$
	\begin{aligned}
	\left|S_{x}\right| \cdot\left|S_{y}\right| \cdot\left|S_{z}\right| & =\left(\sum_{i=1}^{n} b_{i}\right)\left(\sum_{i=1}^{n} a_{i}\right) \cdot\left|S_{z}\right| \\
	& \geqslant\left(\sum_{i=1}^{n} \sqrt{a_{i} b_{i}}\right)^{2} \cdot\left|S_{z}\right| \\
	& =\left(\sum_{i=1}^{n} \sqrt{a_{i} b_{i}\left|S_{z}\right|}\right)^{2} \\
	& \geqslant\left(\sum_{i=1}^{n} m_{i}\right)^{2}=|S|^{2}
	\end{aligned}
	$$
	
	得证.
\end{proof}
\begin{note}
	这里, 在两次利用柯西不等式时, 引进了参数 $n 、 m 、 a 、 b$.
\end{note}

\begin{example}
	某次考试共 $m$ 道试题, $n$ 个学生参加, 其中 $m, n \geqslant 2$ 为给定整数,每道题得分规则为: 若该题恰有 $x$ 个学生没有答对, 则每个答对该题的学生得 $x$ 分, 未答对的学生得零分. 每个学生的总分为其 $m$ 道题的得分总和. 将所有学生总分从高到低排列为 $p_{1} \geqslant p_{2} \geqslant \cdots \geqslant p_{n}$. 求 $p_{1}+p_{n}$ 的最大值.
\end{example}
\begin{solution}
	设第 $k$ 题没有答对者有 $x_{k}$ 人, $1 \leqslant k \leqslant m$, 则第 $k$ 题答对者有 $n-x_{k}$人, 由得分规则知, 这 $n-x_{k}$ 个人在第 $k$ 题均得 $x_{k}$ 分. 设 $n$ 个学生的得分之和为 $S$. 则有
	
	$$
	\sum_{i=1}^{n} p_{i}=S=\sum_{i=1}^{m} x_{i}\left(n-x_{i}\right)=n \sum_{i=1}^{m} x_{i}-\sum_{i=1}^{m} x_{i}^{2}
	$$
	
	因为每一个人在第 $k$ 道题上至多得 $x_{k}$ 分, 则
	
	$$
	p_{1} \leqslant \sum_{k=1}^{m} x_{k}
	$$
	
	由于 $p_{2} \geqslant \cdots \geqslant p_{n}$, 故有 $p_{n} \leqslant \frac{p_{2}+p_{3}+\cdots+p_{n}}{n-1}=\frac{S-p_{1}}{n-1}$.\\
	所以 $p_{1}+p_{n} \leqslant p_{1}+\frac{S-p_{1}}{n-1}=\frac{n-2}{n-1} p_{1}+\frac{S}{n-1}$
	
	$$
	\begin{aligned}
	& \leqslant \frac{n-2}{n-1} \sum_{k=1}^{m} x_{k}+\frac{1}{n-1}\left(n \sum_{k=1}^{m} x_{k}-\sum_{k=1}^{m} x_{k}^{2}\right) \\
	& =2 \sum_{k=1}^{m} x_{k}-\frac{1}{n-1} \sum_{k=1}^{m} x_{k}^{2}
	\end{aligned}
	$$
	
	由柯西不等式得 $\quad \sum_{k=1}^{m} x_{k}^{2} \geqslant \frac{1}{m}\left(\sum_{k=1}^{m} x_{k}\right)^{2}$.
	
	于是 $p_{1}+p_{n} \leqslant 2 \sum_{k=1}^{m} x_{k}-\frac{1}{m(n-1)}\left(\sum_{k=1}^{m} x_{k}\right)^{2}$
	
	$$
	=-\frac{1}{m(n-1)}\left(\sum_{k=1}^{m} x_{k}-m(n-1)\right)^{2}+m(n-1)
	$$
	
	$$
	\leqslant m(n-1)
	$$
	
	另一方面, 若有一个学生全部答对, 其他 $n-1$ 个学生全部答错, 则
	
	$$
	p_{1}+p_{n}=p_{1}=\sum_{k=1}^{m}(n-1)=m(n-1)
	$$
	
	故 $p_{1}+p_{n}$ 的最大值为 $m(n-1)$.
	
	\section*{4. 5 带参数的柯西不等式}
	如果 $a_{i}, b_{i} \in \mathbf{R}, \lambda_{i}>0, i=1,2, \cdots, n$, 则
	
	$$
	\left(\sum_{i=1}^{n} a_{i} b_{i}\right)^{2} \leqslant \sum_{i=1}^{n} \lambda_{i} a_{i}^{2} \cdot \sum_{i=1}^{n} \frac{1}{\lambda_{i}} b_{i}^{2}
	$$
\begin{note}
	这里, 在两次利用柯西不等式时, 引进了参数 $n 、 m 、 a 、 b$.
\end{note}

\begin{example}
	已知正实数 $a 、 b 、 c 、 d$ 满足
	
	$$
	a\left(c^{2}-1\right)=b\left(b^{2}+c^{2}\right)
	$$
	
	且 $d \leqslant 1$. 证明:
	
	$$
	d\left(a \sqrt{1-d^{2}}+b^{2} \sqrt{1+d^{2}}\right) \leqslant \frac{(a+b) c}{2}
	$$
\end{example}
\begin{proof}
	设参数 $\lambda>1$, 由柯西不等式得
	
	$$
	\begin{aligned}
	& d\left(a \sqrt{1-d^{2}}+b^{2} \sqrt{1+d^{2}}\right) \\
	\leqslant & d \sqrt{\left(\frac{a^{2}}{\lambda}+b^{4}\right)\left[\left(1-d^{2}\right) \lambda+\left(1+d^{2}\right)\right]}
	\end{aligned}
	$$
	
	$$
	\begin{aligned}
	& =\sqrt{\left(\frac{a^{2}}{\lambda}+b^{4}\right)\left[(1-\lambda) d^{4}+(\lambda+1) d^{2}\right]} \\
	& \leqslant \sqrt{\frac{1}{\lambda-1}\left(\frac{a^{2}}{\lambda}+b^{4}\right)} \cdot \frac{\lambda+1}{2}
	\end{aligned}
	$$
	
	由已知条件知 $c^{2}=\frac{a+b^{3}}{a-b}$. 故 $a>b$, 取 $\lambda=\frac{a}{b}$. 则
	
	$$
	\frac{\lambda+1}{2} \sqrt{\frac{1}{\lambda-1}\left(\frac{a^{2}}{\lambda}+b^{4}\right)}=\frac{a+b}{2} \sqrt{\frac{a+b^{3}}{a-b}}=\frac{(a+b) c}{2}
	$$
	
	所以, 命题得证.
\end{proof}
\begin{note}
	这里, 在两次利用柯西不等式时, 引进了参数 $n 、 m 、 a 、 b$.
\end{note}

\begin{example}
	设 $p, q \in \mathbf{R}_{+}, x \in\left(0, \frac{\pi}{2}\right)$, 试求
	
	$$
	\frac{p}{\sqrt{\sin x}}+\frac{q}{\sqrt{\cos x}}
	$$
	
	的最小值.
\end{example}
\begin{solution}
	由柯西不等式, 得
	
	$$
	(\sqrt{p m}+\sqrt{q n})^{2} \leqslant\left(\frac{p}{\sqrt{\sin x}}+\frac{q}{\sqrt{\cos x}}\right)(m \sqrt{\sin x}+n \sqrt{\cos x})
	$$
	
	当且仅当 $\frac{\frac{p}{\sqrt{\sin x}}}{m \sqrt{\sin x}}=\frac{\frac{q}{\sqrt{\cos x}}}{n \sqrt{\cos x}}$ 时, 等号成立. 故 $\tan x=\frac{n p}{m q}$.
	
	$$
	\text { 又 } \begin{aligned}
	(m \sqrt{\sin x}+n \sqrt{\cos x})^{2} & =\left(\frac{m}{a} \cdot a \sqrt{\sin x}+\frac{n}{b} \cdot b \sqrt{\cos x}\right)^{2} \\
	& \leqslant\left(\frac{m^{2}}{a^{2}}+\frac{n^{2}}{b^{2}}\right)\left(a^{2} \sin x+b^{2} \cos x\right) \\
	& \leqslant\left(\frac{m^{2}}{a^{2}}+\frac{n^{2}}{b^{2}}\right) \sqrt{a^{4}+b^{4}}
	\end{aligned}
	$$
	
	当且仅当 $\tan x=\frac{a^{2}}{b^{2}}, \frac{a^{2} \sin x}{\frac{m^{2}}{a^{2}}}=\frac{b^{2} \cos x}{\frac{n^{2}}{b^{2}}}$ 时, 即 $\tan x=\frac{b^{4} m^{2}}{a^{4} n^{2}}=\frac{a^{2}}{b^{2}}$ 时, 等号
	
	成立. 故
	
	且
	
	$$
	\begin{gathered}
	\frac{m}{n}=\frac{a^{3}}{b^{3}}, \tan x=\left(\frac{m}{n}\right)^{\frac{2}{3}} \\
	m \sqrt{\sin x}+n \sqrt{\cos x} \leqslant\left(m^{\frac{4}{3}}+n^{\frac{4}{3}}\right)^{\frac{3}{4}}
	\end{gathered}
	$$
	
	从而
	
	$$
	\begin{aligned}
	& \left(\frac{m}{n}\right)^{\frac{2}{3}}=\frac{n p}{m q} \\
	& \frac{m}{n}=\left(\frac{p}{q}\right)^{\frac{3}{5}}
	\end{aligned}
	$$
	
	即
	
	令 $m=p^{\frac{3}{5}}, n=q^{\frac{3}{5}}$, 得
	
	$$
	\frac{p}{\sqrt{\sin x}}+\frac{q}{\sqrt{\cos x}} \geqslant \frac{(\sqrt{p m}+\sqrt{n q})^{2}}{\left(m^{\frac{4}{3}}+n^{\frac{4}{3}}\right)^{\frac{3}{4}}}=\left(p^{\frac{4}{5}}+q^{\frac{4}{5}}\right)^{\frac{5}{4}}
	$$
	
	当且仅当 $\tan x=\left(\frac{m}{n}\right)^{\frac{2}{3}}=\left[\left(\frac{p}{q}\right)^{\frac{3}{5}}\right]^{\frac{2}{3}}=\left(\frac{p}{q}\right)^{\frac{2}{5}}$ 时, 等号成立.
\end{solution}
\begin{note}
	这里, 在两次利用柯西不等式时, 引进了参数 $n 、 m 、 a 、 b$.
\end{note}

\begin{example}
	(1) 设 3 个正实数 $a 、 b 、 c$ 满足
	
	$$
	\left(a^{2}+b^{2}+c^{2}\right)^{2}>2\left(a^{4}+b^{4}+c^{4}\right)
	$$
	
	求证: $a 、 b 、 c$ 一定是某个三角形的 3 条边长;
	
	(2) 设 $n$ 个正实数 $a_{1}, a_{2}, \cdots, a_{n}(n \geqslant 4)$ 满足
	
	$$
	\left(a_{1}^{2}+a_{2}^{2}+\cdots+a_{n}^{2}\right)^{2}>(n-1)\left(a_{1}^{4}+a_{2}^{4}+\cdots+a_{n}^{4}\right)
	$$
	
	求证: 这些数中任意 3 个一定是某个三角形的 3 条边长.
\end{example}
\begin{proof}
	(1) 不妨设 $a \geqslant b \geqslant c>0$, 由题设, 得
	
	$$
	\left(a^{2}+b^{2}+c^{2}\right)^{2}-2\left(a^{4}+b^{4}+c^{4}\right)>0
	$$
	
	分解因式, 得
	
	$$
	(a+b+c)(a+b-c)(a+c-b)(b+c-a)>0
	$$
	
	所以 $b+c-a>0$, 即 $b+c>a$, 从而 $a 、 b 、 c$ 是某个三角形的 3 条边长;
	
	(2) 在 $a_{1}, a_{2}, \cdots, a_{n}$ 中任取 3 个, 不妨设为 $a_{1} 、 a_{2} 、 a_{3}$. 由带参数的柯西不等式, 得
	
	$$
	\begin{aligned}
	(n-1)\left(\sum a_{i}^{4}\right) & <\left(\sum_{i=1}^{n} a_{i}^{2}\right)^{2} \\
	& =\left[\lambda\left(a_{1}^{2}+a_{2}^{2}+a_{3}^{2}\right) \cdot \frac{1}{\lambda}+\sum_{i=4}^{n} a_{i}^{2}\right]^{2} \\
	& \leqslant\left[\lambda^{2}\left(a_{1}^{2}+a_{2}^{2}+a_{3}^{2}\right)^{2}+\sum_{i=4}^{n} a_{i}^{4}\right]\left(\frac{1}{\lambda^{2}}+n-3\right)
	\end{aligned}
	$$
	
	令 $\frac{1}{\lambda^{2}}+n-3=n-1$, 即 $\lambda=\sqrt{\frac{1}{2}}$, 所以
	
	$$
	\left(a_{1}^{2}+a_{2}^{2}+a_{3}^{2}\right)^{2}>2\left(a_{1}^{4}+a_{2}^{4}+a_{3}^{4}\right) .
	$$
	
	由 (1) 知, $a_{1} 、 a_{2} 、 a_{3}$ 为某个三角形的三边长.
\end{proof}
\begin{note}
	该不等式的证明,也可通过构造一个新的序列 $\left\{y_{i}\right\}$ :
	
	$$
	y_{i}=\frac{A b_{i}-C a_{i}}{A B-C^{2}}, i \geqslant 1
	$$
	
	则 $\left\{y_{i}\right\}$ 满足条件
	
	$$
	\begin{gathered}
	\sum_{i=1}^{n} x_{i} y_{i}=\frac{A}{A B-C^{2}}, \sum_{i=1}^{n} y_{i}^{2}=\frac{A}{A B-C^{2}} \\
	\sum_{i=1}^{n} x_{i}^{2}-\sum_{i=1}^{n} y_{i}^{2}=\sum_{i=1}^{n}\left(x_{i}-y_{i}\right)^{2}
	\end{gathered}
	$$
	
	从而命题成立.
\end{note}

\begin{example}
	设 $a=\left(a_{1}, a_{2}, \cdots, a_{n}\right)$ 和 $b=\left(b_{1}, b_{2}, \cdots, b_{n}\right)$ 是两个不成比例的实数序列, 又设 $x=\left(x_{1}, x_{2}, \cdots, x_{n}\right)$ 是使
	
	$$
	\sum_{i=1}^{n} a_{i} x_{i}=0, \sum_{i=1}^{n} b_{i} x_{i}=1
	$$
	
	成立的任意实数序列. 求证:
	
	$$
	\sum_{i=1}^{n} x_{i}^{2} \geqslant \frac{A}{A B-C^{2}}
	$$
	
	其中 $A=\sum_{i=1}^{n} a_{i}^{2}, B=\sum_{i=1}^{n} b_{i}^{2}, C=\sum_{i=1}^{n} a_{i} b_{i}$.
\end{example}
\begin{proof}
	对任意实数 $\lambda$, 由柯西不等式, 得
	
	$$
	\left(\sum_{i=1}^{n} x_{i}^{2}\right) \sum_{i=1}^{n}\left(a_{i} \lambda-b_{i}\right)^{2} \geqslant\left(\lambda \sum_{i=1}^{n} a_{i} x_{i}-\sum_{i=1}^{n} b_{i} x_{i}\right)^{2}=1
	$$
	
	从而
	
	$$
	\left(\sum_{i=1}^{n} x_{i}^{2}\right)\left(A \lambda^{2}-2 C \lambda+B\right) \geqslant 1
	$$
	
	即对任意实数 $\lambda$, 有
	
	$$
	A \lambda^{2}-2 C \lambda+B-\frac{1}{\sum_{i=1}^{n} x_{i}^{2}} \leqslant 0
	$$
	
	于是
	
	$$
	\Delta=4 C^{2}-4 A B+\frac{4 A}{\sum_{i=1}^{n} x_{i}^{2}} \leqslant 0
	$$
	
	故命题成立.
\end{proof}
\begin{note}
	该不等式的证明,也可通过构造一个新的序列 $\left\{y_{i}\right\}$ :
	
	$$
	y_{i}=\frac{A b_{i}-C a_{i}}{A B-C^{2}}, i \geqslant 1
	$$
	
	则 $\left\{y_{i}\right\}$ 满足条件
	
	$$
	\begin{gathered}
	\sum_{i=1}^{n} x_{i} y_{i}=\frac{A}{A B-C^{2}}, \sum_{i=1}^{n} y_{i}^{2}=\frac{A}{A B-C^{2}} \\
	\sum_{i=1}^{n} x_{i}^{2}-\sum_{i=1}^{n} y_{i}^{2}=\sum_{i=1}^{n}\left(x_{i}-y_{i}\right)^{2}
	\end{gathered}
	$$
	
	从而命题成立.
\end{note}

\begin{example}
	设 $a_{i}>0,1 \leqslant i \leqslant n$. 求证:
	
	$$
	\sum_{k=1}^{n} \frac{k}{\sum_{i=1}^{k} a_{i}} \leqslant 2 \sum_{i=1}^{n} \frac{1}{a_{i}}
	$$
\end{example}
\begin{proof}
	由柯西不等式得
	
	$$
	\left(\sum_{i=1}^{k} a_{i}\right)\left(\sum_{i=1}^{k} \frac{i^{2}}{a_{i}}\right) \geqslant\left(\sum_{i=1}^{k} i\right)^{2}=\left(\frac{k(k+1)}{2}\right)^{2}
	$$
	
	于是 $\sum_{k=1}^{n} \frac{k}{\sum_{i=1}^{k} a_{i}} \leqslant \sum_{k=1}^{n}\left(\frac{4}{k(k+1)^{2}} \sum_{i=1}^{k} \frac{i^{2}}{a_{i}}\right)$
	
	$$
	\begin{aligned}
	& <2 \sum_{i=1}^{n}\left(\frac{i^{2}}{a_{i}} \sum_{k=i}^{n} \frac{2 k+1}{k^{2}(k+1)^{2}}\right) \\
	& =2 \sum_{i=1}^{n}\left(\frac{i^{2}}{a_{i}} \sum_{k=i}^{n}\left(\frac{1}{k^{2}}-\frac{1}{(k+1)^{2}}\right)\right. \\
	& =2 \sum_{i=1}^{n} \frac{i^{2}}{a_{i}}\left(\frac{1}{i^{2}}-\frac{1}{(n+1)^{2}}\right)<2 \sum_{i=1}^{n} \frac{1}{a_{i}}
	\end{aligned}
	$$
	
	从而命题成立.
\end{proof}
\begin{note}
	当 $x, y>0$ 时, $x+y=\max \{x, y\}+\min \{x, y\}$; 当 $x>0$ 时, $\min \left\{x, \frac{1}{x}\right\} \leqslant 1$.
\end{note}

\begin{example}
	设 $n \in \mathbf{Z}_{+}$, 求最小实数 $t=t(n)$, 使得对 $x_{i} \in \mathbf{R}, 1 \leqslant i \leqslant n$,
	
	$$
	\sum_{k=1}^{n}\left(x_{1}+x_{2}+\cdots+x_{k}\right)^{2} \leqslant t(n) \sum_{i=1}^{n} x_{i}^{2}
	$$
\end{example}
\begin{solution}
	令 $\alpha=\frac{\pi}{2 n+1}$, 则 $\sin n \alpha=\sin (n+1) \alpha$.
	
	记 $c_{i}=\sin i \alpha-\sin (i-1) \alpha, 1 \leqslant i \leqslant n$. 则 $c_{i}>0,1 \leqslant i \leqslant n$.
	
	令 $s_{k}=\sum_{i=1}^{k} c_{i}=\sin k \alpha, 1 \leqslant k \leqslant n$.
	
	由柯西不等式得
	
	$$
	\left(x_{1}+\cdots+x_{k}\right)^{2} \leqslant\left(c_{1}+\cdots+c_{k}\right)\left(\frac{x_{1}^{2}}{c_{1}}+\cdots+\frac{x_{k}^{2}}{c_{k}}\right)
	$$
	
	于是
	
	$$
	\begin{aligned}
	\sum_{k=1}^{n}\left(x_{1}+\cdots+x_{k}\right)^{2} & \leqslant \sum_{k=1}^{n} s_{k}\left(\frac{x_{1}^{2}}{c_{1}}+\cdots+\frac{x_{k}^{2}}{c_{k}}\right) \\
	& =\sum_{k=1}^{n}\left(s_{k} \sum_{i=1}^{k} \frac{x_{i}^{2}}{c_{i}}\right)=\sum_{i=1}^{n} \sum_{k=i}^{n} s_{k} \frac{x_{i}^{2}}{c_{i}} \\
	& =\sum_{i=1}^{n} \frac{s_{i}+s_{i+1}+\cdots+s_{n}}{c_{i}} x_{i}^{2}
	\end{aligned}
	$$
	
	下面证明 $\quad \frac{s_{1}+\cdots+s_{n}}{c_{1}}=\frac{s_{2}+\cdots+s_{n}}{c_{2}}=\cdots=\frac{s_{n}}{c_{n}}=\frac{1}{4 \sin ^{2} \frac{\alpha}{2}}$.
	
	事实上 $\quad \frac{s_{k}+\cdots+s_{n}}{c_{k}}=\frac{\sin k \alpha+\cdots+\sin n \alpha}{\sin k \alpha-\sin (k-1) \alpha}$
	
	$$
	\begin{aligned}
	& =\frac{2 \sin \frac{\alpha}{2}(\sin k \alpha+\cdots+\sin n \alpha)}{2 \sin \frac{\alpha}{2} \cdot 2 \sin \frac{\alpha}{2} \cos \left(k-\frac{1}{2}\right) \alpha}=\frac{\cos \left(k-\frac{1}{2}\right) \alpha-\cos \left(n+\frac{1}{2}\right) \alpha}{4 \sin ^{2} \frac{\alpha}{2} \cos \left(k-\frac{1}{2}\right) \alpha} \\
	& =\frac{\cos \left(k-\frac{1}{2}\right) \alpha}{4 \sin ^{2} \frac{\alpha}{2} \cos \left(k-\frac{1}{2}\right) \alpha}=\frac{1}{4 \sin ^{2} \frac{\alpha}{2}}
	\end{aligned}
	$$
	
	且等式成立的充要条件是 $\quad \frac{x_{1}}{c_{1}}=\frac{x_{2}}{c_{2}}=\cdots=\frac{x_{n}}{c_{n}}$.
	
	故 $t=t(n)=\frac{1}{4 \sin ^{2} \frac{\alpha}{2}}$.
\end{solution}
\begin{note}
	当 $x, y>0$ 时, $x+y=\max \{x, y\}+\min \{x, y\}$; 当 $x>0$ 时, $\min \left\{x, \frac{1}{x}\right\} \leqslant 1$.
\end{note}

\begin{example}
	设 $a_{i} \in\left[\frac{1}{\sqrt{3}}, \sqrt{3}\right], 1 \leqslant i \leqslant 6$. 求证:
	
	$$
	\sum_{i=1}^{6} \frac{a_{i}-a_{i+1}}{a_{i+1}+a_{i+2}} \geqslant 0
	$$
	
	其中 $a_{7}=a_{1}, a_{8}=a_{2}$.
\end{example}
\begin{proof}
	由于 $2 a_{i}+a_{i+2}-a_{i+1} \geqslant \frac{3}{\sqrt{3}}-\sqrt{3}=0$. 由柯西不等式得
	
	$$
	\sum_{i=1}^{6} \frac{2 a_{i}-a_{i+1}+a_{i+2}}{a_{i+1}+a_{i+2}} \geqslant \frac{\left(\sum_{i=1}^{6}\left(2 a_{i}-a_{i+1}+a_{i+2}\right)\right)^{2}}{\sum_{i=1}^{6}\left(2 a_{i}-a_{i+1}+a_{i+2}\right)\left(a_{i+1}+a_{i+2}\right)}
	$$
	
	$$
	=\frac{2\left(\sum_{i=1}^{6} a_{i}\right)^{2}}{\sum_{i=1}^{6} a_{i} a_{i+1}+\sum_{i=1}^{6} a_{i} a_{i+2}}
	$$
	
	$$
	\begin{gathered}
	\text { 又因为 } \begin{aligned}
	& \sum_{i=1}^{6} \frac{2 a_{i}-a_{i+1}+a_{i+2}}{a_{i+1}+a_{i+2}}=\sum_{i=1}^{6} \frac{2\left(a_{i}-a_{i+1}\right)+a_{i+1}+a_{i+2}}{a_{i+1}+a_{i+2}} \\
	&=2 \sum_{i=1}^{6} \frac{a_{i}-a_{i+1}}{a_{i+1}+a_{i+2}}+6
	\end{aligned} \\
	2 \sum_{i=1}^{6} \frac{a_{i}-a_{i+1}}{a_{i+1}+a_{i+2}} \geqslant \frac{2\left(\sum_{i=1}^{6} a_{i}\right)^{2}}{\sum_{i=1}^{6} a_{i} a_{i+1}+\sum_{i=1}^{6} a_{i} a_{i+2}}-6 \\
	=2 \frac{\left(\sum_{i=1}^{6} a_{i}\right)^{2}-3\left(\sum_{i=1}^{6} a_{i} a_{i+1}+\sum_{i=1}^{6} a_{i} a_{i+2}\right)}{\sum_{i=1}^{6} a_{i} a_{i+1}+\sum_{i=1}^{6} a_{i} a_{i+2}}
	\end{gathered}
	$$
	
	所以
	
	由于
	
	$$
	\begin{aligned}
	& \left(\sum_{i=1}^{6} a_{i}\right)^{2} \geqslant 3\left(\sum_{i=1}^{6} a_{i} a_{i+1}+\sum_{i=1}^{6} a_{i} a_{i+2}\right) \\
	\Leftrightarrow & \sum_{i=1}^{6} a_{i}^{2}+2 a_{1} a_{4}+2 a_{2} a_{5}+2 a_{3} a_{6} \geqslant \sum_{i=1}^{6} a_{i} a_{i+1}+\sum_{i=1}^{6} a_{i} a_{i+2} \\
	\Leftrightarrow & \left(a_{1}+a_{4}\right)^{2}+\left(a_{2}+a_{5}\right)^{2}+\left(a_{3}+a_{6}\right)^{2} \\
	\geqslant & \left(a_{1}+a_{4}\right)\left(a_{2}+a_{5}\right)+\left(a_{2}+a_{5}\right)\left(a_{3}+a_{6}\right) \\
	& +\left(a_{3}+a_{6}\right)\left(a_{1}+a_{4}\right)
	\end{aligned}
	$$
	
	利用平均值不等式, 最后不等式成立.
	
	故命题成立.
\end{proof}
\begin{note}
	当 $x, y>0$ 时, $x+y=\max \{x, y\}+\min \{x, y\}$; 当 $x>0$ 时, $\min \left\{x, \frac{1}{x}\right\} \leqslant 1$.
\end{note}

\begin{example}
	设 $a 、 b 、 c 、 d$ 为正实数, 满足 $a b+c d=1, p_{i}\left(x_{i}, y_{i}\right), i=1,2$, 3,4 为 以原点为圆心的单位圆上的四点. 求证:
	
	$$
	\begin{aligned}
	& \left(a y_{1}+b y_{2}+c y_{3}+d y_{4}\right)^{2}+\left(a x_{4}+b x_{3}+c x_{2}+d x_{1}\right)^{2} \\
	\leqslant & 2\left(\frac{a^{2}+b^{2}}{a b}+\frac{c^{2}+d^{2}}{c d}\right) .
	\end{aligned}
	$$
\end{example}
\begin{proof}
	令 $\alpha=a y_{1}+b y_{2}+c y_{3}+d y_{4}, \beta=a x_{4}+b x_{3}+c x_{2}+d x_{1}$, 由柯西不等式, 得
	
	$$
	\begin{aligned}
	\alpha^{2} & =\left(a y_{1}+b y_{2}+c y_{3}+d y_{4}\right)^{2} \\
	& \leqslant\left[\left(\sqrt{a d} y_{1}\right)^{2}+\left(\sqrt{b c} y_{2}\right)^{2}+\left(\sqrt{b c} y_{3}\right)^{2}+\left(\sqrt{a d} y_{4}\right)^{2}\right]
	\end{aligned}
	$$
	
	$$
	\begin{aligned}
	& \cdot\left[\left(\sqrt{\frac{a}{d}}\right)^{2}+\left(\sqrt{\frac{b}{c}}\right)^{2}+\left(\sqrt{\frac{c}{b}}\right)^{2}+\left(\sqrt{\frac{d}{a}}\right)^{2}\right] \\
	= & \left(a d y_{1}^{2}+b c y_{2}^{2}+b c y_{3}^{2}+a d y_{4}^{2}\right)\left(\frac{a}{d}+\frac{b}{c}+\frac{c}{b}+\frac{d}{a}\right)
	\end{aligned}
	$$
	
	同理, $\beta^{2} \leqslant\left(a d x_{4}^{2}+b c x_{3}^{2}+b c x_{2}^{2}+a d x_{1}^{2}\right)\left(\frac{d}{a}+\frac{c}{b}+\frac{b}{c}+\frac{a}{d}\right)$.
	
	对它们相加, 并利用 $x_{i}^{2}+y_{i}^{2}=1, i=1,2,3,4, a b+c d=1$, 得
	
	$$
	\begin{aligned}
	\alpha^{2}+\beta^{2} & \leqslant(2 a d+2 b c)\left(\frac{a}{d}+\frac{b}{c}+\frac{c}{b}+\frac{d}{a}\right) \\
	& =2(a d+b c)\left(\frac{a b+c d}{b d}+\frac{a b+c d}{a c}\right) \\
	& =2(a d+b c)\left(\frac{1}{b d}+\frac{1}{a c}\right) \\
	& =2\left(\frac{a^{2}+b^{2}}{a b}+\frac{c^{2}+d^{2}}{c d}\right)
	\end{aligned}
	$$
	
	故命题成立.
	
	\section*{4. 6 利用平均值不等式与柯西不等式解题}
\begin{note}
	当 $x, y>0$ 时, $x+y=\max \{x, y\}+\min \{x, y\}$; 当 $x>0$ 时, $\min \left\{x, \frac{1}{x}\right\} \leqslant 1$.
\end{note}

\begin{example}
	设 $a 、 b 、 c$ 为实数, 满足 $a^{2}+2 b^{2}+3 c^{2}=\frac{3}{2}$, 求证:
	
	$$
	3^{-a}+9^{-b}+27^{-c} \geqslant 1
	$$
\end{example}
\begin{proof}
	由平均值不等式, 得
	
	$$
	3^{-a}+9^{-b}+27^{-c} \geqslant 3 \sqrt[3]{3^{-a-2 b-3 c}}=3^{\frac{3-a-2 b-3 c}{3}}
	$$
	
	再由柯西不等式, 得
	
	$$
	\begin{aligned}
	(a+2 b+3 c)^{2} & =(a+\sqrt{2} \cdot \sqrt{2} b+\sqrt{3} \cdot \sqrt{3} c)^{2} \\
	& \leqslant(1+2+3)\left(a^{2}+2 b^{2}+3 c^{2}\right) \\
	& =6 \cdot \frac{3}{2}=9
	\end{aligned}
	$$
	
	从而 $a+2 b+3 c \leqslant 3,3-a-2 b-3 c \geqslant 0,3^{\frac{3-a-2 b-3 c}{3}} \geqslant 3^{0}=1$. 故命题成立.
\end{proof}
\begin{note}
	当 $x, y>0$ 时, $x+y=\max \{x, y\}+\min \{x, y\}$; 当 $x>0$ 时, $\min \left\{x, \frac{1}{x}\right\} \leqslant 1$.
\end{note}

\begin{example}
	求 $x \sqrt{1-y^{2}}+y \sqrt{1-x^{2}}$ 的最大值.\\
\end{example}
\begin{solution}
	由柯西不等式, 得
	
	$$
	\left|x \sqrt{1-y^{2}}+y \sqrt{1-x^{2}}\right|^{2} \leqslant\left(x^{2}+y^{2}\right)\left(2-x^{2}-y^{2}\right)
	$$
	
	再由平均值不等式, 得
	
	$$
	\left|x \sqrt{1-y^{2}}+y \sqrt{1-x^{2}}\right| \leqslant \frac{x^{2}+y^{2}+2-x^{2}-y^{2}}{2}=1
	$$
	
	若 $x=\frac{1}{2}, y=\frac{\sqrt{3}}{2}$, 则
	
	$$
	x \sqrt{1-y^{2}}+y \sqrt{1-x^{2}}=1
	$$
	
	于是所求的最大值为 1 .
\end{solution}
\begin{note}
	当 $x, y>0$ 时, $x+y=\max \{x, y\}+\min \{x, y\}$; 当 $x>0$ 时, $\min \left\{x, \frac{1}{x}\right\} \leqslant 1$.
\end{note}

\begin{example}
	设 $a 、 b 、 c$ 为正数, 且满足 $a b c=1$, 求证:
	
	$$
	\frac{1}{a^{3}(b+c)}+\frac{1}{b^{3}(a+c)}+\frac{1}{c^{3}(a+b)} \geqslant \frac{3}{2}
	$$
\end{example}
\begin{proof}
	由柯西不等式, 得
	
	$$
	\begin{aligned}
	& {\left[\frac{1}{a^{3}(b+c)}+\frac{1}{b^{3}(a+c)}+\frac{1}{c^{3}(a+b)}\right] \cdot[a(b+c)+b(a+c)+c(a+b)] } \\
	\geqslant & \left(\frac{1}{a}+\frac{1}{b}+\frac{1}{c}\right)^{2}=(a b+b c+a c)^{2}
	\end{aligned}
	$$
	
	所以由平均值不等式, 得
	
	$$
	\begin{aligned}
	& \frac{1}{a^{3}(b+c)}+\frac{1}{b^{3}(a+c)}+\frac{1}{c^{3}(a+b)} \\
	\geqslant & \frac{1}{2}(a b+b c+c a) \\
	\geqslant & \frac{1}{2} \cdot 3 \cdot \sqrt[3]{a^{2} b^{2} c^{2}}=\frac{3}{2}
	\end{aligned}
	$$
\end{proof}
\begin{note}
	当 $x, y>0$ 时, $x+y=\max \{x, y\}+\min \{x, y\}$; 当 $x>0$ 时, $\min \left\{x, \frac{1}{x}\right\} \leqslant 1$.
\end{note}

\begin{example}
	设 $x_{i}, i=1,2, \cdots, n$ 为正数, 且满足 $\sum_{i=1}^{n} x_{i}=a, a \in \mathbf{R}_{+}, m, n \in$ $\mathbf{N}^{*}, n \geqslant 2$, 求证:
	
	$$
	\sum_{i=1}^{n} \frac{x_{i}^{m}}{a-x_{i}} \geqslant \frac{a^{m-1}}{(n-1) n^{m-2}}
	$$
\end{example}
\begin{proof}
	当 $m=1$ 时, 即证明
	
	$$
	\sum_{i=1}^{n} \frac{x_{i}}{a-x_{i}} \geqslant \frac{n}{n-1}
	$$
	
	由于
	
	$$
	\sum_{i=1}^{n} \frac{x_{i}}{a-x_{i}}=\sum_{i=1}^{n}\left[\left(\frac{a}{a-x_{i}}\right)-1\right]=\sum_{i=1}^{n} \frac{a}{a-x_{i}}-n
	$$
	
	由柯西不等式, 得
	
	即
	
	$$
	\begin{gathered}
	\sum_{i=1}^{n} \frac{a}{a-x_{i}} \cdot \sum_{i=1}^{n}\left(a-x_{i}\right) \geqslant a n^{2}, \\
	\sum_{i=1}^{n} \frac{a}{a-x_{i}} \geqslant \frac{a n^{2}}{\sum_{i=1}^{n}\left(a-x_{i}\right)}=\frac{a n^{2}}{(n-1) a}
	\end{gathered}
	$$
	
	所以
	
	$$
	\sum_{i=1}^{n} \frac{x_{i}}{a-x_{i}} \geqslant \frac{a n^{2}}{n a-a}-n=\frac{n}{n-1}
	$$
	
	于是命题成立.
	
	当 $m \geqslant 2$ 时, 由柯西不等式, 得
	
	$$
	\sum_{i=1}^{n} \frac{x_{i}^{m}}{a-x_{i}} \cdot \sum_{i=1}^{n}\left(a-x_{i}\right) \geqslant\left(\sum_{i=1}^{n} x_{i}^{\frac{m}{2}}\right)^{2}
	$$
	
	再由幂平均值不等式, 得
	
	$$
	\left(\frac{1}{n} \sum_{i=1}^{n} x_{t}^{\frac{m}{2}}\right)^{2} \geqslant\left[\frac{1}{n}\left(\sum_{i=1}^{n} x_{i}\right)^{\frac{m}{2}}\right]^{2}=\frac{a^{m}}{n^{m}}
	$$
	
	由于 $\sum_{i=1}^{n}\left(a-x_{i}\right)=(n-1) a$, 于是
	
	$$
	\sum_{i=1}^{n} \frac{x_{i}^{m}}{a-x_{i}} \geqslant \frac{a^{m-1}}{(n-1) n^{m-2}}
	$$
\end{proof}
\begin{note}
	当 $x, y>0$ 时, $x+y=\max \{x, y\}+\min \{x, y\}$; 当 $x>0$ 时, $\min \left\{x, \frac{1}{x}\right\} \leqslant 1$.
\end{note}

\begin{example}
	设实数 $x_{i}$ 满足 $\left|x_{i}\right|<1(i=1,2, \cdots, n), n \geqslant 2$, 求证:
	
	$$
	\sum_{i=1}^{n} \frac{1}{1-\left|x_{i}\right|^{n}} \geqslant \frac{n}{1-\prod_{i=1}^{n} x_{i}}
	$$
\end{example}
\begin{proof}
	由柯西不等式, 得
	
	$$
	\sum_{i=1}^{n} \frac{1}{1-\left|x_{i}\right|^{n}} \cdot \sum_{i=1}^{n}\left(1-\left|x_{i}\right|^{n}\right) \geqslant n^{2}
	$$
	
	因此欲证原不等式只要证明
	
	$$
	\frac{n^{2}}{\sum_{i=1}^{n}\left(1-\left|x_{i}\right|^{n}\right)} \geqslant \frac{n}{1-\prod_{i=1}^{n} x_{i}}
	$$
	
	即证
	
	$$
	n-n \prod_{i=1}^{n} x_{i} \geqslant \sum_{i=1}^{n}\left(1-\left|x_{i}\right|^{n}\right)
	$$
	
	即
	
	$$
	\sum_{i=1}^{n}\left|x_{i}\right|^{n} \geqslant n \prod_{i=1}^{n} x_{i}
	$$
	
	由平均值不等式知上述不等式成立, 故原命题成立.
\end{proof}
\begin{note}
	当 $x, y>0$ 时, $x+y=\max \{x, y\}+\min \{x, y\}$; 当 $x>0$ 时, $\min \left\{x, \frac{1}{x}\right\} \leqslant 1$.
\end{note}

\begin{example}
	已知正数 $x_{i}$ 满足 $\sum_{i=1}^{n} \frac{1}{1+x_{i}}=1$, 证明:
	
	$$
	\prod_{i=1}^{n} x_{i} \geqslant(n-1)^{n}
	$$
\end{example}
\begin{proof}
	由柯西不等式, 得
	
	$$
	\sum_{i=1}^{n} \frac{1}{1+x_{i}} \cdot \sum_{i=1}^{n} \frac{1+x_{i}}{x_{i}} \geqslant\left(\sum_{i=1}^{n} \frac{1}{\sqrt{x_{i}}}\right)^{2}
	$$
	
	即
	
	$$
	\sum_{i=1}^{n} \frac{1}{x_{i}}+n \geqslant \sum_{i=1}^{n} \frac{1}{x_{i}}+2 \sum_{1 \leqslant i<j \leqslant n} \frac{1}{\sqrt{x_{i} x_{j}}}
	$$
	
	再由平均值不等式, 得
	
	$$
	n \geqslant 2 \sum_{1 \leqslant i<j \leqslant n} \frac{1}{\sqrt{x_{i} x_{j}}} \geqslant 2 \cdot \frac{n(n-1)}{2} \cdot \sqrt[\frac{n(n-1)}{2}]{\prod_{i=1}^{n}\left(\frac{1}{\sqrt{x_{i}}}\right)^{n-1}}
	$$
	
	由此得到
	
	$$
	\prod_{i=1}^{n} x_{i} \geqslant(n-1)^{n}
	$$
\end{proof}
\begin{note}
	当 $x, y>0$ 时, $x+y=\max \{x, y\}+\min \{x, y\}$; 当 $x>0$ 时, $\min \left\{x, \frac{1}{x}\right\} \leqslant 1$.
\end{note}

\begin{example}
	设 $x, y, z \geqslant 0$, 且 $x^{2}+y^{2}+z^{2}=1$, 求证:
	
	$$
	\frac{x}{1-y z}+\frac{y}{1-x z}+\frac{z}{1-x y} \leqslant \frac{3 \sqrt{3}}{2} .
	$$
\end{example}
\begin{proof}
	设 $S=\frac{x}{1-y z}+\frac{y}{1-x z}+\frac{z}{1-x y}$, 如果 $x=0$ (或 $y=0$ 或 $z=0$ ),则
	
	$$
	S=y+z<2<\frac{3}{2} \sqrt{3}
	$$
	
	所以设 $x y z \neq 0$, 使得 $x, y, z \in(0,1)$. 因为
	
	$$
	\frac{x}{1-y z}=x+\frac{z y x}{1-y z}
	$$
	
	所以
	
	$$
	S=x+y+z+x y z\left(\frac{1}{1-y z}+\frac{1}{1-z x}+\frac{1}{1-x y}\right)
	$$
	
	因为
	
	$$
	\begin{aligned}
	1-y z & \geqslant 1-\frac{1}{2}\left(y^{2}+z^{2}\right) \\
	& =\frac{1}{2}\left(1+x^{2}\right)=\frac{1}{2}\left(2 x^{2}+y^{2}+z^{2}\right) \\
	& \geqslant 2 \sqrt[4]{x^{2} x^{2} y^{2} z^{2}}=2 x \sqrt{y z}
	\end{aligned}
	$$
	
	由平均值不等式, 得
	
	$$
	\begin{aligned}
	& x y z\left(\frac{1}{1-y z}+\frac{1}{1-z x}+\frac{1}{1-y x}\right) \\
	\leqslant & \frac{x y z}{2}\left(\frac{1}{x \sqrt{y z}}+\frac{1}{y \sqrt{z x}}+\frac{1}{z \sqrt{x y}}\right) \\
	= & \frac{1}{2}(\sqrt{y z}+\sqrt{z x}+\sqrt{x y}) \\
	\leqslant & \frac{1}{2}\left(\frac{y+z}{2}+\frac{z+x}{2}+\frac{x+y}{2}\right)=\frac{1}{2}(x+y+z)
	\end{aligned}
	$$
	
	再由柯西不等式, 得
	
	$$
	S \leqslant \frac{3}{2}(x+y+z) \leqslant \frac{3}{2}\left(1^{2}+1^{2}+1^{2}\right)^{\frac{1}{2}}\left(x^{2}+y^{2}+z^{2}\right)^{\frac{1}{2}}=\frac{3}{2} \sqrt{3}
	$$
	
	故命题成立.
\end{proof}
\begin{note}
	当 $x, y>0$ 时, $x+y=\max \{x, y\}+\min \{x, y\}$; 当 $x>0$ 时, $\min \left\{x, \frac{1}{x}\right\} \leqslant 1$.
\end{note}

\begin{example}
	设 $a, b, c>0$, 求证:
	
	$$
	\sum_{\text {cyc }} \sqrt{\frac{5 a^{2}+8 b^{2}+5 c^{2}}{4 a c}} \geqslant 3 \sqrt[9]{\frac{8(a+b)^{2}(b+c)^{2}(c+a)^{2}}{(a b c)^{2}}}
	$$
\end{example}
\begin{proof}
	由柯西不等式及均值不等式有
	
	$$
	\begin{aligned}
	5 a^{2}+8 b^{2}+5 c^{2} & >4\left(a^{2}+b^{2}\right)+4\left(b^{2}+c^{2}\right) \\
	& \geqslant 2(a+b)^{2}+2(b+c)^{2} \\
	& \geqslant 4(a+b)(b+c)
	\end{aligned}
	$$
	
	所以
	
	$$
	\begin{aligned}
	\sum_{\mathrm{cyc}} \sqrt{\frac{5 a^{2}+8 b^{2}+5 c^{2}}{4 a c}} & \geqslant \sum_{\text {cyc }} \sqrt{\frac{(a+b)(b+c)}{a c}} \\
	& \geqslant 3 \sqrt[6]{\frac{(a+b)^{2}(b+c)^{2}(c+a)^{2}}{(a b c)^{2}}}
	\end{aligned}
	$$
	
	只需证明
	
	$$
	\sqrt[6]{\frac{(a+b)^{2}(b+c)^{2}(c+a)^{2}}{(a b c)^{2}}} \geqslant \sqrt[9]{\frac{8(a+b)^{2}(b+c)^{2}(c+a)^{2}}{(a b c)^{2}}}
	$$
	
	等价于 $(a+b)(b+c)(c+a) \geqslant 8 a b c$, 即 $\sum_{\text {cyc }} a(b-c)^{2} \geqslant 0$, 明显成立.
\end{proof}
\begin{note}
	当 $x, y>0$ 时, $x+y=\max \{x, y\}+\min \{x, y\}$; 当 $x>0$ 时, $\min \left\{x, \frac{1}{x}\right\} \leqslant 1$.
\end{note}

\begin{example}
	已知数列 $\left\{a_{n}\right\}$ 满足 $a_{1}>0, a_{2}>0, a_{n+2}=\frac{2}{a_{n}+a_{n+1}} . M_{n}=$ $\max \left\{a_{n}, \frac{1}{a_{n}}, \frac{1}{a_{n+1}}, a_{n+1}\right\}$. 求证:
	
	$$
	M_{n+3} \leqslant \frac{3}{4} M_{n}+\frac{1}{4}
	$$
\end{example}
\begin{proof}
	由于
	
	$$
	M_{n+3}=\max \left\{a_{n+3}, a_{n+4}, \frac{1}{a_{n+3}}, \frac{1}{a_{n+4}}\right\}
	$$
	
	我们需证
	
	$$
	\begin{aligned}
	& a_{n+3} \leqslant \frac{3}{4} M_{n}+\frac{1}{4} \\
	& a_{n+4} \leqslant \frac{3}{4} M_{n}+\frac{1}{4} \\
	& \frac{1}{a_{n+3}} \leqslant \frac{3}{4} M_{n}+\frac{1}{4} \\
	& \frac{1}{a_{n+4}} \leqslant \frac{3}{4} M_{n}+\frac{1}{4}
	\end{aligned}
	$$
	
	由于
	
	$$
	\begin{aligned}
	a_{n+3} & =\frac{2}{a_{n+1}+a_{n+2}} \leqslant \frac{\frac{1}{a_{n+1}}+\frac{1}{a_{n+2}}}{2} \\
	& =\frac{1}{2}\left(\frac{1}{a_{n+1}}+\frac{a_{n}+a_{n+1}}{2}\right) \\
	& =\frac{1}{4}\left(a_{n+1}+\frac{1}{a_{n+1}}\right)+\frac{1}{4} \cdot \frac{1}{a_{n+1}}+\frac{1}{4} a_{n}
	\end{aligned}
	$$
	
	$$
	\begin{aligned}
	& \leqslant \frac{1}{4}\left[\min \left\{a_{n+1}, \frac{1}{a_{n+1}}\right\}+\max \left\{a_{n+1}, \frac{1}{a_{n+1}}\right\}\right]+\frac{1}{4} M_{n}+\frac{1}{4} M_{n} \\
	& \leqslant \frac{1}{4}\left(1+M_{n}\right)+\frac{1}{2} M_{n} \\
	&= \frac{3}{4} M_{n}+\frac{1}{4} ; \\
	& \frac{1}{a_{n+3}}= \frac{a_{n+1}+a_{n+2}}{2}=\frac{1}{2} \cdot \frac{1}{a_{n+1}}+\frac{1}{a_{n}+a_{n+1}} \\
	& \leqslant \frac{1}{2}+\frac{\frac{1}{a_{n}}+\frac{1}{a_{n+1}}}{4} \\
	& \leqslant \frac{1}{4}\left(a_{n+1}+\frac{1}{a_{n+1}}\right)+\frac{1}{4}\left(\frac{1}{a_{n}}+\frac{1}{a_{n+1}}\right) \\
	&= \frac{1}{4}\left[\max \left\{a_{n+1}, \frac{1}{a_{n+1}}\right\}+\min \left\{a_{n+1}, \frac{1}{a_{n+1}}\right\}\right]+\frac{1}{4}\left(\frac{1}{a_{n}}+\frac{1}{a_{n+1}}\right) \\
	& \leqslant \frac{1}{4}\left(M_{n}+1\right)+\frac{1}{4} \cdot 2 M_{n} \\
	&= \frac{3}{4} M_{n}+\frac{1}{4} ; \\
	&+\frac{1}{8}\left[\max \left\{a_{n+1}, \frac{1}{a_{n+1}}\right\}+\min \left\{a_{n+1}, \frac{1}{a_{n+1}}\right\}\right]+\frac{1}{8} a_{n}+\frac{3}{8} a_{n+1} \\
	&= \frac{1}{8}\left(M_{n}+1\right)+\frac{1}{8}\left(M_{n}+1\right)+\frac{1}{8} M_{n}+\frac{3}{8} M_{n} \\
	&= \frac{3}{4} M_{n}+\frac{1}{4} ; \\
	&= \frac{a_{n}+a_{n+1}}{4}+\frac{a_{n+1}+a_{n+2}}{4} \\
	&= \frac{1}{4} a_{n}+\frac{1}{2} a_{n+1}+\frac{1}{2} \cdot \frac{1}{a_{n}+a_{n+1}} \\
	& \leqslant \frac{1}{4} a_{n}+\frac{1}{2} a_{n+1}+\frac{1}{8}\left(\frac{1}{a_{n}}+\frac{1}{a_{n+1}}\right) \\
	&= \frac{1}{8}\left(a_{n}+\frac{1}{a_{n}}\right)+\frac{1}{8}\left(a_{n+1}+\frac{1}{a_{n+1}}\right)+\frac{1}{8} a_{n}+\frac{3}{8} a_{n+1} \\
	&\left.\max \left\{a_{n}, \frac{1}{a_{n}}\right\}+\min \left\{a_{n}, \frac{1}{a_{n}}\right\}\right] \\
	& a_{n+4}
	\end{aligned}
	$$
	
	$$
	\begin{aligned}
	\frac{1}{a_{n+4}}= & \frac{a_{n+2}+a_{n+3}}{2}=\frac{1}{a_{n}+a_{n+1}}+\frac{1}{a_{n+1}+a_{n+2}} \\
	\leqslant & \frac{\frac{1}{a_{n}}+\frac{1}{a_{n+1}}}{4}+\frac{\frac{1}{a_{n+1}}+\frac{1}{a_{n+2}}}{4} \\
	= & \frac{1}{4} \cdot \frac{1}{a_{n}}+\frac{1}{2} \cdot \frac{1}{a_{n+1}}+\frac{1}{4} \cdot \frac{1}{a_{n+2}} \\
	= & \frac{1}{4} \cdot \frac{1}{a_{n}}+\frac{1}{2} \cdot \frac{1}{a_{n+1}}+\frac{1}{8}\left(a_{n}+a_{n+1}\right) \\
	= & \frac{1}{8}\left(a_{n}+\frac{1}{a_{n}}\right)+\frac{1}{8}\left(a_{n+1}+\frac{1}{a_{n+1}}\right)+\frac{1}{8} \cdot \frac{1}{a_{n}}+\frac{3}{8} \cdot \frac{1}{a_{n+1}} \\
	= & \frac{1}{8}\left[\max \left\{a_{n}, \frac{1}{a_{n}}\right\}+\min \left\{a_{n}, \frac{1}{a_{n}}\right\}\right] \\
	& +\frac{1}{8}\left[\max \left\{a_{n+1}, \frac{1}{a_{n+1}}\right\}+\min \left\{a_{n+1}, \frac{1}{a_{n+1}}\right\}\right]+\frac{1}{8} \cdot \frac{1}{a_{n}}+\frac{3}{8} \cdot \frac{1}{a_{n+1}} \\
	\leqslant & \frac{1}{8}\left(M_{n}+1\right)+\frac{1}{8}\left(M_{n}+1\right)+\frac{1}{8} M_{n}+\frac{3}{8} M_{n} \\
	= & \frac{3}{4} M_{n}+\frac{1}{4} .
	\end{aligned}
	$$
	
	因此, $M_{n+3} \leqslant \frac{3}{4} M_{n}+\frac{1}{4}$.
\end{proof}
\begin{note}
	当 $x, y>0$ 时, $x+y=\max \{x, y\}+\min \{x, y\}$; 当 $x>0$ 时, $\min \left\{x, \frac{1}{x}\right\} \leqslant 1$.
\end{note}

\begin{example}
	已知正实数 $x 、 y 、 z$ 满足 $\sqrt{x}+\sqrt{y}+\sqrt{z}=1$. 求证:
	
	$$
	\frac{x^{2}+y z}{\sqrt{2 x^{2}(y+z)}}+\frac{y^{2}+z x}{\sqrt{2 y^{2}(z+x)}}+\frac{z^{2}+x y}{\sqrt{2 z^{2}(x+y)}} \geqslant 1
	$$
	
	证法 1 注意到
	
	$$
	\begin{aligned}
	\frac{x^{2}+y z}{\sqrt{2 x^{2}(y+z)}} & =\frac{x^{2}-x(y+z)+y z}{\sqrt{2 x^{2}(y+z)}}+\frac{x(y+z)}{\sqrt{2 x^{2}(y+z)}} \\
	& =\frac{(x-y)(x-z)}{\sqrt{2 x^{2}(y+z)}}+\sqrt{\frac{y+z}{2}} \\
	& \geqslant \frac{(x-y)(x-z)}{\sqrt{2 x^{2}(y+z)}}+\frac{\sqrt{y}+\sqrt{z}}{2}
	\end{aligned}
	$$
	
	同理,
	
	$$
	\begin{aligned}
	& \frac{y^{2}+z x}{\sqrt{2 y^{2}(z+x)}} \geqslant \frac{(y-z)(y-x)}{\sqrt{2 y^{2}(z+x)}}+\frac{\sqrt{z}+\sqrt{x}}{2} \\
	& \frac{z^{2}+x y}{\sqrt{2 z^{2}(x+y)}} \geqslant \frac{(z-x)(z-y)}{\sqrt{2 z^{2}(x+y)}}+\frac{\sqrt{x}+\sqrt{y}}{2}
	\end{aligned}
	$$
	
	以上三式相加得
	
	$$
	\begin{aligned}
	& \frac{x^{2}+y z}{\sqrt{2 x^{2}(y+z)}}+\frac{y^{2}+z x}{\sqrt{2 y^{2}(z+x)}}+\frac{z^{2}+x y}{\sqrt{2 z^{2}(x+y)}} \\
	\geqslant & \frac{(x-y)(x-z)}{\sqrt{2 x^{2}(y+z)}}+\frac{(y-z)(y-x)}{\sqrt{2 y^{2}(z+x)}}+\frac{(z-x)(z-y)}{\sqrt{2 z^{2}(x+y)}}+\sqrt{x}+\sqrt{y}+\sqrt{z} \\
	= & \frac{(x-y)(x-z)}{\sqrt{2 x^{2}(y+z)}}+\frac{(y-z)(y-x)}{\sqrt{2 y^{2}(z+x)}}+\frac{(z-x)(z-y)}{\sqrt{2 z^{2}(x+y)}}+1
	\end{aligned}
	$$
	
	从而, 只需证明
	
	$$
	\frac{(x-y)(x-z)}{\sqrt{2 x^{2}(y+z)}}+\frac{(y-z)(y-x)}{\sqrt{2 y^{2}(z+x)}}+\frac{(z-x)(z-y)}{\sqrt{2 z^{2}(x+y)}} \geqslant 0
	$$
	
	不失一般性,设 $x \geqslant y \geqslant z$. 于是,
	
	$$
	\frac{(x-y)(x-z)}{\sqrt{2 x^{2}(y+z)}} \geqslant 0
	$$
	
	且
	
	
	\begin{align*}
	& \frac{(y-z)(y-x)}{\sqrt{2 y^{2}(z+x)}}+\frac{(z-x)(z-y)}{\sqrt{2 z^{2}(x+y)}} \\
	= & \frac{(y-z)(x-z)}{\sqrt{2 z^{2}(x+y)}}-\frac{(y-z)(x-y)}{\sqrt{2 y^{2}(z+x)}} \\
	\geqslant & \frac{(y-z)(x-y)}{\sqrt{2 z^{2}(x+y)}}-\frac{(y-z)(x-y)}{\sqrt{2 y^{2}(z+x)}} \\
	= & (y-z)(x-y) \cdot\left[\frac{1}{\sqrt{2 z^{2}(x+y)}}-\frac{1}{\sqrt{2 y^{2}(z+x)}}\right] \tag{1}
	\end{align*}
	
	
	事实上,由
	
	$$
	y^{2}(z+x)=y^{2} z+y^{2} x \geqslant y z^{2}+z^{2} x=z^{2}(x+y)
	$$
	
	可知式(1)非负.
	
	从而, 题中不等式成立.
	
	证法 2 根据柯西不等式得
	
	$$
	\begin{aligned}
	& {\left[\frac{x^{2}}{\sqrt{2 x^{2}(y+z)}}+\frac{y^{2}}{\sqrt{2 y^{2}(z+x)}}+\frac{z^{2}}{\sqrt{2 z^{2}(x+y)}}\right] } \\
	& {[\sqrt{2(y+z)}+\sqrt{2(z+x)}+\sqrt{2(x+y)}] } \\
	\geqslant & (\sqrt{x}+\sqrt{y}+\sqrt{z})^{2}=1
	\end{aligned}
	$$
	
	和
	
	$$
	\begin{aligned}
	& {\left[\frac{y z}{\sqrt{2 x^{2}(y+z)}}+\frac{z x}{\sqrt{2 y^{2}(z+x)}}+\frac{x y}{\sqrt{2 z^{2}(x+y)}}\right] } \\
	& {[\sqrt{2(y+z)}+\sqrt{2(z+x)}+\sqrt{2(x+y)}] } \\
	\geqslant & \left(\sqrt{\frac{y z}{x}}+\sqrt{\frac{z x}{y}}+\sqrt{\frac{x y}{z}}\right)^{2}
	\end{aligned}
	$$
	
	以上两式相加得
	
	$$
	\begin{aligned}
	& {\left[\frac{x^{2}+y z}{\sqrt{2 x^{2}(y+z)}}+\frac{y^{2}+z x}{\sqrt{2 y^{2}(z+x)}}+\frac{z^{2}+x y}{\sqrt{2 z^{2}(x+y)}}\right] } \\
	& {[\sqrt{2(y+z)}+\sqrt{2(z+x)}+\sqrt{2(x+y)}] } \\
	\geqslant & 1+\left(\sqrt{\frac{y z}{x}}+\sqrt{\frac{z x}{y}}+\sqrt{\frac{x y}{z}}\right)^{2} \\
	\geqslant & 2\left(\sqrt{\frac{y z}{x}}+\sqrt{\frac{z x}{y}}+\sqrt{\frac{x y}{z}}\right)
	\end{aligned}
	$$
	
	从而, 只需证明
	
	$$
	2\left(\sqrt{\frac{y z}{x}}+\sqrt{\frac{z x}{y}}+\sqrt{\frac{x y}{z}}\right) \geqslant \sqrt{2(y+z)}+\sqrt{2(z+x)}+\sqrt{2(x+y)}
	$$
	
	根据均值不等式得
	
	$$
	\begin{aligned}
	& {\left[\sqrt{\frac{y z}{x}}+\left(\frac{1}{2} \sqrt{\frac{z x}{y}}+\frac{1}{2} \sqrt{\frac{x y}{z}}\right)\right]^{2} } \\
	& \geqslant 4 \sqrt{\frac{y z}{x}}\left(\frac{1}{2} \sqrt{\frac{z x}{y}}+\frac{1}{2} \sqrt{\frac{x y}{z}}\right)=2(y+z) \\
	& \sqrt{\frac{y z}{x}}+\left(\frac{1}{2} \sqrt{\frac{z x}{y}}+\frac{1}{2} \sqrt{\frac{x y}{2}}\right) \geqslant \sqrt{2(y+z)}
	\end{aligned}
	$$
	
	即
	
	同理,
	
	$$
	\begin{aligned}
	& \sqrt{\frac{z x}{y}}+\left(\frac{1}{2} \sqrt{\frac{x y}{z}}+\frac{1}{2} \sqrt{\frac{y z}{x}}\right) \geqslant \sqrt{2(z+x)} \\
	& \sqrt{\frac{x y}{2}}+\left(\frac{1}{2} \sqrt{\frac{y z}{x}}+\frac{1}{2} \sqrt{\frac{z x}{y}}\right) \geqslant \sqrt{2(x+y)}
	\end{aligned}
	$$
	
	以上三式相加得
	
	$$
	2\left(\sqrt{\frac{y z}{x}}+\sqrt{\frac{z x}{y}}+\sqrt{\frac{x y}{z}}\right) \geqslant \sqrt{2(y+z)}+\sqrt{2(z+x)}+\sqrt{2(x+y)}
	$$
	
	从而, 题中不等式成立.
	
	例 11 设正整数 $n \geqslant 2$. 求常数 $C(n)$ 的最大值, 使得对于所有满足 $x_{i} \in$ $(0,1)(i=1,2, \cdots, n)$, 且 $\left(1-x_{i}\right)\left(1-x_{j}\right) \geqslant \frac{1}{4}(1 \leqslant i<j \leqslant n)$ 的实数 $x_{1}$, $x_{2}, \cdots, x_{n}$, 均有
	
	
	\begin{equation*}
	\sum_{i=1}^{n} x_{i} \geqslant C(n) \sum_{1 \leqslant i<j \leqslant n}\left(2 x_{i} x_{j}+\sqrt{x_{i} x_{j}}\right) \tag{1}
	\end{equation*}
\end{example}
\begin{proof}
	$$
	\frac{(x-y)(x-z)}{\sqrt{2 x^{2}(y+z)}}+\frac{(y-z)(y-x)}{\sqrt{2 y^{2}(z+x)}}+\frac{(z-x)(z-y)}{\sqrt{2 z^{2}(x+y)}} \geqslant 0
	$$
	
	不失一般性,设 $x \geqslant y \geqslant z$. 于是,
	
	$$
	\frac{(x-y)(x-z)}{\sqrt{2 x^{2}(y+z)}} \geqslant 0
	$$
	
	且
	
	
	\begin{align*}
	& \frac{(y-z)(y-x)}{\sqrt{2 y^{2}(z+x)}}+\frac{(z-x)(z-y)}{\sqrt{2 z^{2}(x+y)}} \\
	= & \frac{(y-z)(x-z)}{\sqrt{2 z^{2}(x+y)}}-\frac{(y-z)(x-y)}{\sqrt{2 y^{2}(z+x)}} \\
	\geqslant & \frac{(y-z)(x-y)}{\sqrt{2 z^{2}(x+y)}}-\frac{(y-z)(x-y)}{\sqrt{2 y^{2}(z+x)}} \\
	= & (y-z)(x-y) \cdot\left[\frac{1}{\sqrt{2 z^{2}(x+y)}}-\frac{1}{\sqrt{2 y^{2}(z+x)}}\right] \tag{1}
	\end{align*}
	
	
	事实上,由
	
	$$
	y^{2}(z+x)=y^{2} z+y^{2} x \geqslant y z^{2}+z^{2} x=z^{2}(x+y)
	$$
	
	可知式(1)非负.
	
	从而, 题中不等式成立.
	
	证法 2 根据柯西不等式得
	
	$$
	\begin{aligned}
	& {\left[\frac{x^{2}}{\sqrt{2 x^{2}(y+z)}}+\frac{y^{2}}{\sqrt{2 y^{2}(z+x)}}+\frac{z^{2}}{\sqrt{2 z^{2}(x+y)}}\right] } \\
	& {[\sqrt{2(y+z)}+\sqrt{2(z+x)}+\sqrt{2(x+y)}] } \\
	\geqslant & (\sqrt{x}+\sqrt{y}+\sqrt{z})^{2}=1
	\end{aligned}
	$$
	
	和
	
	$$
	\begin{aligned}
	& {\left[\frac{y z}{\sqrt{2 x^{2}(y+z)}}+\frac{z x}{\sqrt{2 y^{2}(z+x)}}+\frac{x y}{\sqrt{2 z^{2}(x+y)}}\right] } \\
	& {[\sqrt{2(y+z)}+\sqrt{2(z+x)}+\sqrt{2(x+y)}] } \\
	\geqslant & \left(\sqrt{\frac{y z}{x}}+\sqrt{\frac{z x}{y}}+\sqrt{\frac{x y}{z}}\right)^{2}
	\end{aligned}
	$$
	
	以上两式相加得
	
	$$
	\begin{aligned}
	& {\left[\frac{x^{2}+y z}{\sqrt{2 x^{2}(y+z)}}+\frac{y^{2}+z x}{\sqrt{2 y^{2}(z+x)}}+\frac{z^{2}+x y}{\sqrt{2 z^{2}(x+y)}}\right] } \\
	& {[\sqrt{2(y+z)}+\sqrt{2(z+x)}+\sqrt{2(x+y)}] } \\
	\geqslant & 1+\left(\sqrt{\frac{y z}{x}}+\sqrt{\frac{z x}{y}}+\sqrt{\frac{x y}{z}}\right)^{2} \\
	\geqslant & 2\left(\sqrt{\frac{y z}{x}}+\sqrt{\frac{z x}{y}}+\sqrt{\frac{x y}{z}}\right)
	\end{aligned}
	$$
	
	从而, 只需证明
	
	$$
	2\left(\sqrt{\frac{y z}{x}}+\sqrt{\frac{z x}{y}}+\sqrt{\frac{x y}{z}}\right) \geqslant \sqrt{2(y+z)}+\sqrt{2(z+x)}+\sqrt{2(x+y)}
	$$
	
	根据均值不等式得
	
	$$
	\begin{aligned}
	& {\left[\sqrt{\frac{y z}{x}}+\left(\frac{1}{2} \sqrt{\frac{z x}{y}}+\frac{1}{2} \sqrt{\frac{x y}{z}}\right)\right]^{2} } \\
	& \geqslant 4 \sqrt{\frac{y z}{x}}\left(\frac{1}{2} \sqrt{\frac{z x}{y}}+\frac{1}{2} \sqrt{\frac{x y}{z}}\right)=2(y+z) \\
	& \sqrt{\frac{y z}{x}}+\left(\frac{1}{2} \sqrt{\frac{z x}{y}}+\frac{1}{2} \sqrt{\frac{x y}{2}}\right) \geqslant \sqrt{2(y+z)}
	\end{aligned}
	$$
	
	即
	
	同理,
	
	$$
	\begin{aligned}
	& \sqrt{\frac{z x}{y}}+\left(\frac{1}{2} \sqrt{\frac{x y}{z}}+\frac{1}{2} \sqrt{\frac{y z}{x}}\right) \geqslant \sqrt{2(z+x)} \\
	& \sqrt{\frac{x y}{2}}+\left(\frac{1}{2} \sqrt{\frac{y z}{x}}+\frac{1}{2} \sqrt{\frac{z x}{y}}\right) \geqslant \sqrt{2(x+y)}
	\end{aligned}
	$$
	
	以上三式相加得
	
	$$
	2\left(\sqrt{\frac{y z}{x}}+\sqrt{\frac{z x}{y}}+\sqrt{\frac{x y}{z}}\right) \geqslant \sqrt{2(y+z)}+\sqrt{2(z+x)}+\sqrt{2(x+y)}
	$$
	
	从而, 题中不等式成立.
\end{proof}

\begin{example}
	给定整数 $n \geqslant 2$ 和正实数 $a$, 正实数 $x_{1}, x_{2}, \cdots, x_{n}$ 满足 $x_{1} x_{2} \cdots$ $x_{n}=1$. 求最小的实数 $M=M(n, a)$, 使得
	
	$$
	\sum_{i=1}^{n} \frac{1}{a+S-x_{i}} \leqslant M
	$$
	
	恒成立,其中 $S=x_{1}+x_{2}+\cdots+x_{n}$.
\end{example}
\end{comment}